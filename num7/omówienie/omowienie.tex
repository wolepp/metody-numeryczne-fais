\documentclass[a4paper,11pt]{article}
\usepackage{graphicx}
\usepackage[]{latexsym}
\usepackage[]{amsmath}
\usepackage[]{mathtools}
\usepackage[]{bm}
\usepackage[]{listings}
\usepackage[]{color}
\usepackage[]{pgf}
\usepackage[]{float}
\usepackage[MeX]{polski}
\usepackage[utf8]{inputenc}
\usepackage[figurename=Wykres]{caption}

\definecolor{dkgreen}{rgb}{0,0.6,0}
\definecolor{gray}{rgb}{0.5,0.5,0.5}
\definecolor{mauve}{rgb}{0.58,0,0.82}
\lstset{
    language=Python,
    frame=tb,
    aboveskip=3mm,
    belowskip=3mm,
    numbers=none,
    basicstyle=\ttfamily\small, 
    keywordstyle=\color{blue},
    commentstyle=\color{dkgreen},
    stringstyle=\color{mauve},
    showstringspaces=false,
    breaklines=true,
}

\author{Wojciech Lepich \and Nomen Nescio}
\title{NUM7. Naturalne splajny kubiczne}

\begin{document}
\maketitle

\section{Teoria}

Interpolacja to metoda numeryczna, która wyznacza nieznane wartości funkcji
pomiędzy węzłami, czyli punktami, dla których wartość funkcji jest znana.

Interpolacja wielomianowa jest nieskomplikowana. Przy wielu równoodległych 
węzłach interpolacji może prowadzić do oscylacji Rungego zbliżając się do
brzegów interpolowanego odcinka funkcji.

Aby ich uniknąć stosuje się interpolację za pomocą funkcji sklejanych.
Funkcja sklejana rzędu \textit{k}, ``splajn'' (ang. \textit{spline}), to funkcja
która:
\begin{itemize}
    \item lokalnie jest wielomianem rzędu \textit{k}
    \item jest (\textit{k}--1)-krotnie różniczkowalna w węzłach.
\end{itemize}

Najczęściej używane są funkcje sklejane rzędu 3, czyli ``splajny kubiczne''.

\subsection{Splajny kubiczne}

Do wyznaczenia funkcji sklejanych potrzebne są wartości funkcji w węzłach oraz
drugie pochodne wyrażenia interpolacyjnego w węzłach, oznaczane
literą \nolinebreak{\( \xi \)}. W każdym przedziale 
\(\langle x_j, x_{j+1} \rangle, j=1,2,\ldots,n-1 \) wyznaczamy wielomian
trzeciego stopnia
\[
    y_j(x) = Af_j + Bf_{j+1} + C\xi_j + D\xi_{j+1},
\] gdzie
\[
    A = \frac{x_{j+1}-x}{x_{j+1}-x_j}, B = \frac{x-x_j}{x_{j+1}-x_j},
\]
\[
    C = \frac{1}{6} (A^3-A) {(x_{j+1}-x_j)}^2,
    D = \frac{1}{6} (B^3-B) {(x_{j+1}-x_j)}^2
\]

Zakładając, że jest \textit{n} węzłów interpolacji, są \textit{n}--2 punktów
sklejania, w których żądana jest ciągłość pochodnej. Stąd, przy równoodległych
węzłach, pojawia się poniższy układ \textit{n}--2 równań z n niewiadomymi.
Dla naturalnego splajnu kubicznego przyjmuje się, że \(\xi_1 = \xi_n = 0\).

\[
\begin{bmatrix}
    4 & 1 &   &     \\
    1 & 4 & 1 &     \\
      & 1 & 4 & 1   \\
    \cdots & \cdots & \cdots & \cdots & \cdots & \cdots & \cdots \\
    & & & & 1 & 4 & 1\\
    & & & & & 1 & 4
\end{bmatrix}
\begin{bmatrix}
    \xi_2 \\ \xi_3 \\ \xi_4 \\ \vdots \\ \xi_{n-2} \\ \xi_{n-1}
\end{bmatrix}
= \frac{6}{h^2}
\begin{bmatrix}
    f_1 - 2f_2 + f_3 \\
    f_2 - 2f_3 + f_4 \\
    f_3 - 2f_4 + f_5 \\
    \vdots \\
    f_{n-3} - 2f_{n-2} + f_{n-1} \\
    f_{n-2} - 2f_{n-1} + f_{n} \\
\end{bmatrix}
\]

Łatwo znaleźć faktoryzację Choleskiego macierzy po lewej stronie (koszt O(n)).

\section{Omówienie}

\subsection{Kod}

Poniższa funkcja znajduje wielomian dla przedziału \(\langle xi, xi1 \rangle \).
Parametry \texttt{yi, yi1} to wartości funkcji w punktach \texttt{xi, xi1}.
\texttt{ksi} oraz \texttt{ksi1} to wartości drugich pochodnych w punktach.
\begin{lstlisting}[caption={funkcja s}, language=Python]
def s(xi, xi1, yi, yi1, ksi, ksi1) -> Callable
    def wielomian(x) -> Callable:
        A = (xi1 - x) / (xi1 - xi)
        B = (x - xi) / (xi1 - xi)
        C = ((A**3 - A) * (xi1 - xi)**2) / 6
        D = ((B**3 - B) * (xi1 - xi)**2) / 6
        w = A*yi + B*yi1 + C*ksi + D*ksi1
        return w
    return wielomian
\end{lstlisting}

\pagebreak

\begin{lstlisting}[caption={Obliczanie wartości \(\xi \)},language=Python]
def wartosci_ksi(yi: Sequence[float]) -> Sequence[float]:
    A = np.diagflat([4]*(N-1), k=0) +\
        np.diagflat([1]*(N-2), k=-1) +\
        np.diagflat([1]*(N-2), k=1)
    C = cholesky(A, lower=True)

    b = [yi[i] - 2*yi[i+1] + yi[i+2] for i in range(N-1)]
    h = 2/N
    b = np.multiply(6/h**2, b)

    y = solve_triangular(C, b, lower=True)
    x = solve_triangular(C.T, y, lower=False)
    Ksi = [0, *x, 0]
    return Ksi
\end{lstlisting}
W funkcji \lstinline{wartosci_ksi} znajdywane są wartości drugich pochodnych
wyrażenia interpolowanego. Na trójdiagonalnej macierzy A dokonana jest
faktoryzacja Choleskiego z pomocą pakietu SciPy. Następnie tworzony jest wektor
b oraz wyznaczana długość \textit{h} przedziału.
Równanie obliczane jest przez forward i backward substitution. 
W przedostatniej linijce tworzona jest lista wartości \(\xi \).

\subsection{Wyniki}

Na wykresach funkcja \(f(x) = \frac{1}{1+25x^2}\) zaznaczona jest przerywaną 
czerwoną linią. Każdy wielomian składający się na splajn jest zaznaczony linią
ciągłą, dla odróżnienia w różnych kolorach.

\begin{figure}[H]
    \begin{center}
        %% Creator: Matplotlib, PGF backend
%%
%% To include the figure in your LaTeX document, write
%%   \input{<filename>.pgf}
%%
%% Make sure the required packages are loaded in your preamble
%%   \usepackage{pgf}
%%
%% Figures using additional raster images can only be included by \input if
%% they are in the same directory as the main LaTeX file. For loading figures
%% from other directories you can use the `import` package
%%   \usepackage{import}
%% and then include the figures with
%%   \import{<path to file>}{<filename>.pgf}
%%
%% Matplotlib used the following preamble
%%
\begingroup%
\makeatletter%
\begin{pgfpicture}%
\pgfpathrectangle{\pgfpointorigin}{\pgfqpoint{5.500000in}{3.500000in}}%
\pgfusepath{use as bounding box, clip}%
\begin{pgfscope}%
\pgfsetbuttcap%
\pgfsetmiterjoin%
\definecolor{currentfill}{rgb}{1.000000,1.000000,1.000000}%
\pgfsetfillcolor{currentfill}%
\pgfsetlinewidth{0.000000pt}%
\definecolor{currentstroke}{rgb}{1.000000,1.000000,1.000000}%
\pgfsetstrokecolor{currentstroke}%
\pgfsetdash{}{0pt}%
\pgfpathmoveto{\pgfqpoint{0.000000in}{0.000000in}}%
\pgfpathlineto{\pgfqpoint{5.500000in}{0.000000in}}%
\pgfpathlineto{\pgfqpoint{5.500000in}{3.500000in}}%
\pgfpathlineto{\pgfqpoint{0.000000in}{3.500000in}}%
\pgfpathclose%
\pgfusepath{fill}%
\end{pgfscope}%
\begin{pgfscope}%
\pgfsetbuttcap%
\pgfsetmiterjoin%
\definecolor{currentfill}{rgb}{1.000000,1.000000,1.000000}%
\pgfsetfillcolor{currentfill}%
\pgfsetlinewidth{0.000000pt}%
\definecolor{currentstroke}{rgb}{0.000000,0.000000,0.000000}%
\pgfsetstrokecolor{currentstroke}%
\pgfsetstrokeopacity{0.000000}%
\pgfsetdash{}{0pt}%
\pgfpathmoveto{\pgfqpoint{0.687500in}{0.385000in}}%
\pgfpathlineto{\pgfqpoint{4.950000in}{0.385000in}}%
\pgfpathlineto{\pgfqpoint{4.950000in}{3.080000in}}%
\pgfpathlineto{\pgfqpoint{0.687500in}{3.080000in}}%
\pgfpathclose%
\pgfusepath{fill}%
\end{pgfscope}%
\begin{pgfscope}%
\pgfsetbuttcap%
\pgfsetroundjoin%
\definecolor{currentfill}{rgb}{0.000000,0.000000,0.000000}%
\pgfsetfillcolor{currentfill}%
\pgfsetlinewidth{0.803000pt}%
\definecolor{currentstroke}{rgb}{0.000000,0.000000,0.000000}%
\pgfsetstrokecolor{currentstroke}%
\pgfsetdash{}{0pt}%
\pgfsys@defobject{currentmarker}{\pgfqpoint{0.000000in}{-0.048611in}}{\pgfqpoint{0.000000in}{0.000000in}}{%
\pgfpathmoveto{\pgfqpoint{0.000000in}{0.000000in}}%
\pgfpathlineto{\pgfqpoint{0.000000in}{-0.048611in}}%
\pgfusepath{stroke,fill}%
}%
\begin{pgfscope}%
\pgfsys@transformshift{0.881250in}{0.385000in}%
\pgfsys@useobject{currentmarker}{}%
\end{pgfscope}%
\end{pgfscope}%
\begin{pgfscope}%
\definecolor{textcolor}{rgb}{0.000000,0.000000,0.000000}%
\pgfsetstrokecolor{textcolor}%
\pgfsetfillcolor{textcolor}%
\pgftext[x=0.881250in,y=0.287778in,,top]{\color{textcolor}\rmfamily\fontsize{10.000000}{12.000000}\selectfont \(\displaystyle -1.00\)}%
\end{pgfscope}%
\begin{pgfscope}%
\pgfsetbuttcap%
\pgfsetroundjoin%
\definecolor{currentfill}{rgb}{0.000000,0.000000,0.000000}%
\pgfsetfillcolor{currentfill}%
\pgfsetlinewidth{0.803000pt}%
\definecolor{currentstroke}{rgb}{0.000000,0.000000,0.000000}%
\pgfsetstrokecolor{currentstroke}%
\pgfsetdash{}{0pt}%
\pgfsys@defobject{currentmarker}{\pgfqpoint{0.000000in}{-0.048611in}}{\pgfqpoint{0.000000in}{0.000000in}}{%
\pgfpathmoveto{\pgfqpoint{0.000000in}{0.000000in}}%
\pgfpathlineto{\pgfqpoint{0.000000in}{-0.048611in}}%
\pgfusepath{stroke,fill}%
}%
\begin{pgfscope}%
\pgfsys@transformshift{1.365625in}{0.385000in}%
\pgfsys@useobject{currentmarker}{}%
\end{pgfscope}%
\end{pgfscope}%
\begin{pgfscope}%
\definecolor{textcolor}{rgb}{0.000000,0.000000,0.000000}%
\pgfsetstrokecolor{textcolor}%
\pgfsetfillcolor{textcolor}%
\pgftext[x=1.365625in,y=0.287778in,,top]{\color{textcolor}\rmfamily\fontsize{10.000000}{12.000000}\selectfont \(\displaystyle -0.75\)}%
\end{pgfscope}%
\begin{pgfscope}%
\pgfsetbuttcap%
\pgfsetroundjoin%
\definecolor{currentfill}{rgb}{0.000000,0.000000,0.000000}%
\pgfsetfillcolor{currentfill}%
\pgfsetlinewidth{0.803000pt}%
\definecolor{currentstroke}{rgb}{0.000000,0.000000,0.000000}%
\pgfsetstrokecolor{currentstroke}%
\pgfsetdash{}{0pt}%
\pgfsys@defobject{currentmarker}{\pgfqpoint{0.000000in}{-0.048611in}}{\pgfqpoint{0.000000in}{0.000000in}}{%
\pgfpathmoveto{\pgfqpoint{0.000000in}{0.000000in}}%
\pgfpathlineto{\pgfqpoint{0.000000in}{-0.048611in}}%
\pgfusepath{stroke,fill}%
}%
\begin{pgfscope}%
\pgfsys@transformshift{1.850000in}{0.385000in}%
\pgfsys@useobject{currentmarker}{}%
\end{pgfscope}%
\end{pgfscope}%
\begin{pgfscope}%
\definecolor{textcolor}{rgb}{0.000000,0.000000,0.000000}%
\pgfsetstrokecolor{textcolor}%
\pgfsetfillcolor{textcolor}%
\pgftext[x=1.850000in,y=0.287778in,,top]{\color{textcolor}\rmfamily\fontsize{10.000000}{12.000000}\selectfont \(\displaystyle -0.50\)}%
\end{pgfscope}%
\begin{pgfscope}%
\pgfsetbuttcap%
\pgfsetroundjoin%
\definecolor{currentfill}{rgb}{0.000000,0.000000,0.000000}%
\pgfsetfillcolor{currentfill}%
\pgfsetlinewidth{0.803000pt}%
\definecolor{currentstroke}{rgb}{0.000000,0.000000,0.000000}%
\pgfsetstrokecolor{currentstroke}%
\pgfsetdash{}{0pt}%
\pgfsys@defobject{currentmarker}{\pgfqpoint{0.000000in}{-0.048611in}}{\pgfqpoint{0.000000in}{0.000000in}}{%
\pgfpathmoveto{\pgfqpoint{0.000000in}{0.000000in}}%
\pgfpathlineto{\pgfqpoint{0.000000in}{-0.048611in}}%
\pgfusepath{stroke,fill}%
}%
\begin{pgfscope}%
\pgfsys@transformshift{2.334375in}{0.385000in}%
\pgfsys@useobject{currentmarker}{}%
\end{pgfscope}%
\end{pgfscope}%
\begin{pgfscope}%
\definecolor{textcolor}{rgb}{0.000000,0.000000,0.000000}%
\pgfsetstrokecolor{textcolor}%
\pgfsetfillcolor{textcolor}%
\pgftext[x=2.334375in,y=0.287778in,,top]{\color{textcolor}\rmfamily\fontsize{10.000000}{12.000000}\selectfont \(\displaystyle -0.25\)}%
\end{pgfscope}%
\begin{pgfscope}%
\pgfsetbuttcap%
\pgfsetroundjoin%
\definecolor{currentfill}{rgb}{0.000000,0.000000,0.000000}%
\pgfsetfillcolor{currentfill}%
\pgfsetlinewidth{0.803000pt}%
\definecolor{currentstroke}{rgb}{0.000000,0.000000,0.000000}%
\pgfsetstrokecolor{currentstroke}%
\pgfsetdash{}{0pt}%
\pgfsys@defobject{currentmarker}{\pgfqpoint{0.000000in}{-0.048611in}}{\pgfqpoint{0.000000in}{0.000000in}}{%
\pgfpathmoveto{\pgfqpoint{0.000000in}{0.000000in}}%
\pgfpathlineto{\pgfqpoint{0.000000in}{-0.048611in}}%
\pgfusepath{stroke,fill}%
}%
\begin{pgfscope}%
\pgfsys@transformshift{2.818750in}{0.385000in}%
\pgfsys@useobject{currentmarker}{}%
\end{pgfscope}%
\end{pgfscope}%
\begin{pgfscope}%
\definecolor{textcolor}{rgb}{0.000000,0.000000,0.000000}%
\pgfsetstrokecolor{textcolor}%
\pgfsetfillcolor{textcolor}%
\pgftext[x=2.818750in,y=0.287778in,,top]{\color{textcolor}\rmfamily\fontsize{10.000000}{12.000000}\selectfont \(\displaystyle 0.00\)}%
\end{pgfscope}%
\begin{pgfscope}%
\pgfsetbuttcap%
\pgfsetroundjoin%
\definecolor{currentfill}{rgb}{0.000000,0.000000,0.000000}%
\pgfsetfillcolor{currentfill}%
\pgfsetlinewidth{0.803000pt}%
\definecolor{currentstroke}{rgb}{0.000000,0.000000,0.000000}%
\pgfsetstrokecolor{currentstroke}%
\pgfsetdash{}{0pt}%
\pgfsys@defobject{currentmarker}{\pgfqpoint{0.000000in}{-0.048611in}}{\pgfqpoint{0.000000in}{0.000000in}}{%
\pgfpathmoveto{\pgfqpoint{0.000000in}{0.000000in}}%
\pgfpathlineto{\pgfqpoint{0.000000in}{-0.048611in}}%
\pgfusepath{stroke,fill}%
}%
\begin{pgfscope}%
\pgfsys@transformshift{3.303125in}{0.385000in}%
\pgfsys@useobject{currentmarker}{}%
\end{pgfscope}%
\end{pgfscope}%
\begin{pgfscope}%
\definecolor{textcolor}{rgb}{0.000000,0.000000,0.000000}%
\pgfsetstrokecolor{textcolor}%
\pgfsetfillcolor{textcolor}%
\pgftext[x=3.303125in,y=0.287778in,,top]{\color{textcolor}\rmfamily\fontsize{10.000000}{12.000000}\selectfont \(\displaystyle 0.25\)}%
\end{pgfscope}%
\begin{pgfscope}%
\pgfsetbuttcap%
\pgfsetroundjoin%
\definecolor{currentfill}{rgb}{0.000000,0.000000,0.000000}%
\pgfsetfillcolor{currentfill}%
\pgfsetlinewidth{0.803000pt}%
\definecolor{currentstroke}{rgb}{0.000000,0.000000,0.000000}%
\pgfsetstrokecolor{currentstroke}%
\pgfsetdash{}{0pt}%
\pgfsys@defobject{currentmarker}{\pgfqpoint{0.000000in}{-0.048611in}}{\pgfqpoint{0.000000in}{0.000000in}}{%
\pgfpathmoveto{\pgfqpoint{0.000000in}{0.000000in}}%
\pgfpathlineto{\pgfqpoint{0.000000in}{-0.048611in}}%
\pgfusepath{stroke,fill}%
}%
\begin{pgfscope}%
\pgfsys@transformshift{3.787500in}{0.385000in}%
\pgfsys@useobject{currentmarker}{}%
\end{pgfscope}%
\end{pgfscope}%
\begin{pgfscope}%
\definecolor{textcolor}{rgb}{0.000000,0.000000,0.000000}%
\pgfsetstrokecolor{textcolor}%
\pgfsetfillcolor{textcolor}%
\pgftext[x=3.787500in,y=0.287778in,,top]{\color{textcolor}\rmfamily\fontsize{10.000000}{12.000000}\selectfont \(\displaystyle 0.50\)}%
\end{pgfscope}%
\begin{pgfscope}%
\pgfsetbuttcap%
\pgfsetroundjoin%
\definecolor{currentfill}{rgb}{0.000000,0.000000,0.000000}%
\pgfsetfillcolor{currentfill}%
\pgfsetlinewidth{0.803000pt}%
\definecolor{currentstroke}{rgb}{0.000000,0.000000,0.000000}%
\pgfsetstrokecolor{currentstroke}%
\pgfsetdash{}{0pt}%
\pgfsys@defobject{currentmarker}{\pgfqpoint{0.000000in}{-0.048611in}}{\pgfqpoint{0.000000in}{0.000000in}}{%
\pgfpathmoveto{\pgfqpoint{0.000000in}{0.000000in}}%
\pgfpathlineto{\pgfqpoint{0.000000in}{-0.048611in}}%
\pgfusepath{stroke,fill}%
}%
\begin{pgfscope}%
\pgfsys@transformshift{4.271875in}{0.385000in}%
\pgfsys@useobject{currentmarker}{}%
\end{pgfscope}%
\end{pgfscope}%
\begin{pgfscope}%
\definecolor{textcolor}{rgb}{0.000000,0.000000,0.000000}%
\pgfsetstrokecolor{textcolor}%
\pgfsetfillcolor{textcolor}%
\pgftext[x=4.271875in,y=0.287778in,,top]{\color{textcolor}\rmfamily\fontsize{10.000000}{12.000000}\selectfont \(\displaystyle 0.75\)}%
\end{pgfscope}%
\begin{pgfscope}%
\pgfsetbuttcap%
\pgfsetroundjoin%
\definecolor{currentfill}{rgb}{0.000000,0.000000,0.000000}%
\pgfsetfillcolor{currentfill}%
\pgfsetlinewidth{0.803000pt}%
\definecolor{currentstroke}{rgb}{0.000000,0.000000,0.000000}%
\pgfsetstrokecolor{currentstroke}%
\pgfsetdash{}{0pt}%
\pgfsys@defobject{currentmarker}{\pgfqpoint{0.000000in}{-0.048611in}}{\pgfqpoint{0.000000in}{0.000000in}}{%
\pgfpathmoveto{\pgfqpoint{0.000000in}{0.000000in}}%
\pgfpathlineto{\pgfqpoint{0.000000in}{-0.048611in}}%
\pgfusepath{stroke,fill}%
}%
\begin{pgfscope}%
\pgfsys@transformshift{4.756250in}{0.385000in}%
\pgfsys@useobject{currentmarker}{}%
\end{pgfscope}%
\end{pgfscope}%
\begin{pgfscope}%
\definecolor{textcolor}{rgb}{0.000000,0.000000,0.000000}%
\pgfsetstrokecolor{textcolor}%
\pgfsetfillcolor{textcolor}%
\pgftext[x=4.756250in,y=0.287778in,,top]{\color{textcolor}\rmfamily\fontsize{10.000000}{12.000000}\selectfont \(\displaystyle 1.00\)}%
\end{pgfscope}%
\begin{pgfscope}%
\definecolor{textcolor}{rgb}{0.000000,0.000000,0.000000}%
\pgfsetstrokecolor{textcolor}%
\pgfsetfillcolor{textcolor}%
\pgftext[x=2.818750in,y=0.108766in,,top]{\color{textcolor}\rmfamily\fontsize{10.000000}{12.000000}\selectfont x}%
\end{pgfscope}%
\begin{pgfscope}%
\pgfsetbuttcap%
\pgfsetroundjoin%
\definecolor{currentfill}{rgb}{0.000000,0.000000,0.000000}%
\pgfsetfillcolor{currentfill}%
\pgfsetlinewidth{0.803000pt}%
\definecolor{currentstroke}{rgb}{0.000000,0.000000,0.000000}%
\pgfsetstrokecolor{currentstroke}%
\pgfsetdash{}{0pt}%
\pgfsys@defobject{currentmarker}{\pgfqpoint{-0.048611in}{0.000000in}}{\pgfqpoint{0.000000in}{0.000000in}}{%
\pgfpathmoveto{\pgfqpoint{0.000000in}{0.000000in}}%
\pgfpathlineto{\pgfqpoint{-0.048611in}{0.000000in}}%
\pgfusepath{stroke,fill}%
}%
\begin{pgfscope}%
\pgfsys@transformshift{0.687500in}{0.409500in}%
\pgfsys@useobject{currentmarker}{}%
\end{pgfscope}%
\end{pgfscope}%
\begin{pgfscope}%
\definecolor{textcolor}{rgb}{0.000000,0.000000,0.000000}%
\pgfsetstrokecolor{textcolor}%
\pgfsetfillcolor{textcolor}%
\pgftext[x=0.412808in,y=0.361275in,left,base]{\color{textcolor}\rmfamily\fontsize{10.000000}{12.000000}\selectfont \(\displaystyle 0.0\)}%
\end{pgfscope}%
\begin{pgfscope}%
\pgfsetbuttcap%
\pgfsetroundjoin%
\definecolor{currentfill}{rgb}{0.000000,0.000000,0.000000}%
\pgfsetfillcolor{currentfill}%
\pgfsetlinewidth{0.803000pt}%
\definecolor{currentstroke}{rgb}{0.000000,0.000000,0.000000}%
\pgfsetstrokecolor{currentstroke}%
\pgfsetdash{}{0pt}%
\pgfsys@defobject{currentmarker}{\pgfqpoint{-0.048611in}{0.000000in}}{\pgfqpoint{0.000000in}{0.000000in}}{%
\pgfpathmoveto{\pgfqpoint{0.000000in}{0.000000in}}%
\pgfpathlineto{\pgfqpoint{-0.048611in}{0.000000in}}%
\pgfusepath{stroke,fill}%
}%
\begin{pgfscope}%
\pgfsys@transformshift{0.687500in}{0.919100in}%
\pgfsys@useobject{currentmarker}{}%
\end{pgfscope}%
\end{pgfscope}%
\begin{pgfscope}%
\definecolor{textcolor}{rgb}{0.000000,0.000000,0.000000}%
\pgfsetstrokecolor{textcolor}%
\pgfsetfillcolor{textcolor}%
\pgftext[x=0.412808in,y=0.870875in,left,base]{\color{textcolor}\rmfamily\fontsize{10.000000}{12.000000}\selectfont \(\displaystyle 0.2\)}%
\end{pgfscope}%
\begin{pgfscope}%
\pgfsetbuttcap%
\pgfsetroundjoin%
\definecolor{currentfill}{rgb}{0.000000,0.000000,0.000000}%
\pgfsetfillcolor{currentfill}%
\pgfsetlinewidth{0.803000pt}%
\definecolor{currentstroke}{rgb}{0.000000,0.000000,0.000000}%
\pgfsetstrokecolor{currentstroke}%
\pgfsetdash{}{0pt}%
\pgfsys@defobject{currentmarker}{\pgfqpoint{-0.048611in}{0.000000in}}{\pgfqpoint{0.000000in}{0.000000in}}{%
\pgfpathmoveto{\pgfqpoint{0.000000in}{0.000000in}}%
\pgfpathlineto{\pgfqpoint{-0.048611in}{0.000000in}}%
\pgfusepath{stroke,fill}%
}%
\begin{pgfscope}%
\pgfsys@transformshift{0.687500in}{1.428700in}%
\pgfsys@useobject{currentmarker}{}%
\end{pgfscope}%
\end{pgfscope}%
\begin{pgfscope}%
\definecolor{textcolor}{rgb}{0.000000,0.000000,0.000000}%
\pgfsetstrokecolor{textcolor}%
\pgfsetfillcolor{textcolor}%
\pgftext[x=0.412808in,y=1.380475in,left,base]{\color{textcolor}\rmfamily\fontsize{10.000000}{12.000000}\selectfont \(\displaystyle 0.4\)}%
\end{pgfscope}%
\begin{pgfscope}%
\pgfsetbuttcap%
\pgfsetroundjoin%
\definecolor{currentfill}{rgb}{0.000000,0.000000,0.000000}%
\pgfsetfillcolor{currentfill}%
\pgfsetlinewidth{0.803000pt}%
\definecolor{currentstroke}{rgb}{0.000000,0.000000,0.000000}%
\pgfsetstrokecolor{currentstroke}%
\pgfsetdash{}{0pt}%
\pgfsys@defobject{currentmarker}{\pgfqpoint{-0.048611in}{0.000000in}}{\pgfqpoint{0.000000in}{0.000000in}}{%
\pgfpathmoveto{\pgfqpoint{0.000000in}{0.000000in}}%
\pgfpathlineto{\pgfqpoint{-0.048611in}{0.000000in}}%
\pgfusepath{stroke,fill}%
}%
\begin{pgfscope}%
\pgfsys@transformshift{0.687500in}{1.938300in}%
\pgfsys@useobject{currentmarker}{}%
\end{pgfscope}%
\end{pgfscope}%
\begin{pgfscope}%
\definecolor{textcolor}{rgb}{0.000000,0.000000,0.000000}%
\pgfsetstrokecolor{textcolor}%
\pgfsetfillcolor{textcolor}%
\pgftext[x=0.412808in,y=1.890075in,left,base]{\color{textcolor}\rmfamily\fontsize{10.000000}{12.000000}\selectfont \(\displaystyle 0.6\)}%
\end{pgfscope}%
\begin{pgfscope}%
\pgfsetbuttcap%
\pgfsetroundjoin%
\definecolor{currentfill}{rgb}{0.000000,0.000000,0.000000}%
\pgfsetfillcolor{currentfill}%
\pgfsetlinewidth{0.803000pt}%
\definecolor{currentstroke}{rgb}{0.000000,0.000000,0.000000}%
\pgfsetstrokecolor{currentstroke}%
\pgfsetdash{}{0pt}%
\pgfsys@defobject{currentmarker}{\pgfqpoint{-0.048611in}{0.000000in}}{\pgfqpoint{0.000000in}{0.000000in}}{%
\pgfpathmoveto{\pgfqpoint{0.000000in}{0.000000in}}%
\pgfpathlineto{\pgfqpoint{-0.048611in}{0.000000in}}%
\pgfusepath{stroke,fill}%
}%
\begin{pgfscope}%
\pgfsys@transformshift{0.687500in}{2.447900in}%
\pgfsys@useobject{currentmarker}{}%
\end{pgfscope}%
\end{pgfscope}%
\begin{pgfscope}%
\definecolor{textcolor}{rgb}{0.000000,0.000000,0.000000}%
\pgfsetstrokecolor{textcolor}%
\pgfsetfillcolor{textcolor}%
\pgftext[x=0.412808in,y=2.399675in,left,base]{\color{textcolor}\rmfamily\fontsize{10.000000}{12.000000}\selectfont \(\displaystyle 0.8\)}%
\end{pgfscope}%
\begin{pgfscope}%
\pgfsetbuttcap%
\pgfsetroundjoin%
\definecolor{currentfill}{rgb}{0.000000,0.000000,0.000000}%
\pgfsetfillcolor{currentfill}%
\pgfsetlinewidth{0.803000pt}%
\definecolor{currentstroke}{rgb}{0.000000,0.000000,0.000000}%
\pgfsetstrokecolor{currentstroke}%
\pgfsetdash{}{0pt}%
\pgfsys@defobject{currentmarker}{\pgfqpoint{-0.048611in}{0.000000in}}{\pgfqpoint{0.000000in}{0.000000in}}{%
\pgfpathmoveto{\pgfqpoint{0.000000in}{0.000000in}}%
\pgfpathlineto{\pgfqpoint{-0.048611in}{0.000000in}}%
\pgfusepath{stroke,fill}%
}%
\begin{pgfscope}%
\pgfsys@transformshift{0.687500in}{2.957500in}%
\pgfsys@useobject{currentmarker}{}%
\end{pgfscope}%
\end{pgfscope}%
\begin{pgfscope}%
\definecolor{textcolor}{rgb}{0.000000,0.000000,0.000000}%
\pgfsetstrokecolor{textcolor}%
\pgfsetfillcolor{textcolor}%
\pgftext[x=0.412808in,y=2.909275in,left,base]{\color{textcolor}\rmfamily\fontsize{10.000000}{12.000000}\selectfont \(\displaystyle 1.0\)}%
\end{pgfscope}%
\begin{pgfscope}%
\definecolor{textcolor}{rgb}{0.000000,0.000000,0.000000}%
\pgfsetstrokecolor{textcolor}%
\pgfsetfillcolor{textcolor}%
\pgftext[x=0.357253in,y=1.732500in,,bottom,rotate=90.000000]{\color{textcolor}\rmfamily\fontsize{10.000000}{12.000000}\selectfont y}%
\end{pgfscope}%
\begin{pgfscope}%
\pgfpathrectangle{\pgfqpoint{0.687500in}{0.385000in}}{\pgfqpoint{4.262500in}{2.695000in}}%
\pgfusepath{clip}%
\pgfsetbuttcap%
\pgfsetroundjoin%
\pgfsetlinewidth{1.505625pt}%
\definecolor{currentstroke}{rgb}{1.000000,0.000000,0.000000}%
\pgfsetstrokecolor{currentstroke}%
\pgfsetdash{{5.550000pt}{2.400000pt}}{0.000000pt}%
\pgfpathmoveto{\pgfqpoint{0.881250in}{0.507500in}}%
\pgfpathlineto{\pgfqpoint{0.994744in}{0.519532in}}%
\pgfpathlineto{\pgfqpoint{1.108237in}{0.533881in}}%
\pgfpathlineto{\pgfqpoint{1.205518in}{0.548491in}}%
\pgfpathlineto{\pgfqpoint{1.286585in}{0.562681in}}%
\pgfpathlineto{\pgfqpoint{1.367652in}{0.579103in}}%
\pgfpathlineto{\pgfqpoint{1.432505in}{0.594167in}}%
\pgfpathlineto{\pgfqpoint{1.497359in}{0.611268in}}%
\pgfpathlineto{\pgfqpoint{1.562212in}{0.630777in}}%
\pgfpathlineto{\pgfqpoint{1.627066in}{0.653152in}}%
\pgfpathlineto{\pgfqpoint{1.675706in}{0.672146in}}%
\pgfpathlineto{\pgfqpoint{1.724346in}{0.693353in}}%
\pgfpathlineto{\pgfqpoint{1.772986in}{0.717110in}}%
\pgfpathlineto{\pgfqpoint{1.821627in}{0.743819in}}%
\pgfpathlineto{\pgfqpoint{1.870267in}{0.773957in}}%
\pgfpathlineto{\pgfqpoint{1.902694in}{0.796230in}}%
\pgfpathlineto{\pgfqpoint{1.935120in}{0.820473in}}%
\pgfpathlineto{\pgfqpoint{1.967547in}{0.846904in}}%
\pgfpathlineto{\pgfqpoint{1.999974in}{0.875770in}}%
\pgfpathlineto{\pgfqpoint{2.032401in}{0.907349in}}%
\pgfpathlineto{\pgfqpoint{2.064827in}{0.941955in}}%
\pgfpathlineto{\pgfqpoint{2.097254in}{0.979935in}}%
\pgfpathlineto{\pgfqpoint{2.129681in}{1.021684in}}%
\pgfpathlineto{\pgfqpoint{2.162108in}{1.067637in}}%
\pgfpathlineto{\pgfqpoint{2.194535in}{1.118276in}}%
\pgfpathlineto{\pgfqpoint{2.226961in}{1.174130in}}%
\pgfpathlineto{\pgfqpoint{2.259388in}{1.235771in}}%
\pgfpathlineto{\pgfqpoint{2.291815in}{1.303803in}}%
\pgfpathlineto{\pgfqpoint{2.324242in}{1.378852in}}%
\pgfpathlineto{\pgfqpoint{2.356668in}{1.461531in}}%
\pgfpathlineto{\pgfqpoint{2.389095in}{1.552404in}}%
\pgfpathlineto{\pgfqpoint{2.421522in}{1.651918in}}%
\pgfpathlineto{\pgfqpoint{2.453949in}{1.760309in}}%
\pgfpathlineto{\pgfqpoint{2.486376in}{1.877480in}}%
\pgfpathlineto{\pgfqpoint{2.518802in}{2.002830in}}%
\pgfpathlineto{\pgfqpoint{2.567442in}{2.203110in}}%
\pgfpathlineto{\pgfqpoint{2.648509in}{2.545272in}}%
\pgfpathlineto{\pgfqpoint{2.680936in}{2.671401in}}%
\pgfpathlineto{\pgfqpoint{2.697150in}{2.729079in}}%
\pgfpathlineto{\pgfqpoint{2.713363in}{2.782015in}}%
\pgfpathlineto{\pgfqpoint{2.729576in}{2.829350in}}%
\pgfpathlineto{\pgfqpoint{2.745790in}{2.870263in}}%
\pgfpathlineto{\pgfqpoint{2.762003in}{2.904004in}}%
\pgfpathlineto{\pgfqpoint{2.778217in}{2.929922in}}%
\pgfpathlineto{\pgfqpoint{2.794430in}{2.947503in}}%
\pgfpathlineto{\pgfqpoint{2.810643in}{2.956385in}}%
\pgfpathlineto{\pgfqpoint{2.826857in}{2.956385in}}%
\pgfpathlineto{\pgfqpoint{2.843070in}{2.947503in}}%
\pgfpathlineto{\pgfqpoint{2.859283in}{2.929922in}}%
\pgfpathlineto{\pgfqpoint{2.875497in}{2.904004in}}%
\pgfpathlineto{\pgfqpoint{2.891710in}{2.870263in}}%
\pgfpathlineto{\pgfqpoint{2.907924in}{2.829350in}}%
\pgfpathlineto{\pgfqpoint{2.924137in}{2.782015in}}%
\pgfpathlineto{\pgfqpoint{2.940350in}{2.729079in}}%
\pgfpathlineto{\pgfqpoint{2.972777in}{2.609849in}}%
\pgfpathlineto{\pgfqpoint{3.005204in}{2.478478in}}%
\pgfpathlineto{\pgfqpoint{3.118698in}{2.002830in}}%
\pgfpathlineto{\pgfqpoint{3.151124in}{1.877480in}}%
\pgfpathlineto{\pgfqpoint{3.183551in}{1.760309in}}%
\pgfpathlineto{\pgfqpoint{3.215978in}{1.651918in}}%
\pgfpathlineto{\pgfqpoint{3.248405in}{1.552404in}}%
\pgfpathlineto{\pgfqpoint{3.280832in}{1.461531in}}%
\pgfpathlineto{\pgfqpoint{3.313258in}{1.378852in}}%
\pgfpathlineto{\pgfqpoint{3.345685in}{1.303803in}}%
\pgfpathlineto{\pgfqpoint{3.378112in}{1.235771in}}%
\pgfpathlineto{\pgfqpoint{3.410539in}{1.174130in}}%
\pgfpathlineto{\pgfqpoint{3.442965in}{1.118276in}}%
\pgfpathlineto{\pgfqpoint{3.475392in}{1.067637in}}%
\pgfpathlineto{\pgfqpoint{3.507819in}{1.021684in}}%
\pgfpathlineto{\pgfqpoint{3.540246in}{0.979935in}}%
\pgfpathlineto{\pgfqpoint{3.572673in}{0.941955in}}%
\pgfpathlineto{\pgfqpoint{3.605099in}{0.907349in}}%
\pgfpathlineto{\pgfqpoint{3.637526in}{0.875770in}}%
\pgfpathlineto{\pgfqpoint{3.669953in}{0.846904in}}%
\pgfpathlineto{\pgfqpoint{3.702380in}{0.820473in}}%
\pgfpathlineto{\pgfqpoint{3.734806in}{0.796230in}}%
\pgfpathlineto{\pgfqpoint{3.767233in}{0.773957in}}%
\pgfpathlineto{\pgfqpoint{3.815873in}{0.743819in}}%
\pgfpathlineto{\pgfqpoint{3.864514in}{0.717110in}}%
\pgfpathlineto{\pgfqpoint{3.913154in}{0.693353in}}%
\pgfpathlineto{\pgfqpoint{3.961794in}{0.672146in}}%
\pgfpathlineto{\pgfqpoint{4.010434in}{0.653152in}}%
\pgfpathlineto{\pgfqpoint{4.059074in}{0.636082in}}%
\pgfpathlineto{\pgfqpoint{4.123928in}{0.615903in}}%
\pgfpathlineto{\pgfqpoint{4.188781in}{0.598238in}}%
\pgfpathlineto{\pgfqpoint{4.253635in}{0.582696in}}%
\pgfpathlineto{\pgfqpoint{4.334702in}{0.565773in}}%
\pgfpathlineto{\pgfqpoint{4.415769in}{0.551170in}}%
\pgfpathlineto{\pgfqpoint{4.513049in}{0.536154in}}%
\pgfpathlineto{\pgfqpoint{4.610329in}{0.523371in}}%
\pgfpathlineto{\pgfqpoint{4.723823in}{0.510731in}}%
\pgfpathlineto{\pgfqpoint{4.756250in}{0.507500in}}%
\pgfpathlineto{\pgfqpoint{4.756250in}{0.507500in}}%
\pgfusepath{stroke}%
\end{pgfscope}%
\begin{pgfscope}%
\pgfpathrectangle{\pgfqpoint{0.687500in}{0.385000in}}{\pgfqpoint{4.262500in}{2.695000in}}%
\pgfusepath{clip}%
\pgfsetrectcap%
\pgfsetroundjoin%
\pgfsetlinewidth{1.505625pt}%
\definecolor{currentstroke}{rgb}{0.121569,0.466667,0.705882}%
\pgfsetstrokecolor{currentstroke}%
\pgfsetdash{}{0pt}%
\pgfpathmoveto{\pgfqpoint{0.881250in}{0.507500in}}%
\pgfpathlineto{\pgfqpoint{0.897953in}{0.508874in}}%
\pgfpathlineto{\pgfqpoint{0.914655in}{0.510256in}}%
\pgfpathlineto{\pgfqpoint{0.931358in}{0.511654in}}%
\pgfpathlineto{\pgfqpoint{0.948060in}{0.513075in}}%
\pgfpathlineto{\pgfqpoint{0.964763in}{0.514526in}}%
\pgfpathlineto{\pgfqpoint{0.981466in}{0.516016in}}%
\pgfpathlineto{\pgfqpoint{0.998168in}{0.517552in}}%
\pgfpathlineto{\pgfqpoint{1.014871in}{0.519143in}}%
\pgfpathlineto{\pgfqpoint{1.031573in}{0.520795in}}%
\pgfpathlineto{\pgfqpoint{1.048276in}{0.522516in}}%
\pgfpathlineto{\pgfqpoint{1.064978in}{0.524314in}}%
\pgfpathlineto{\pgfqpoint{1.081681in}{0.526198in}}%
\pgfpathlineto{\pgfqpoint{1.098384in}{0.528174in}}%
\pgfpathlineto{\pgfqpoint{1.115086in}{0.530250in}}%
\pgfpathlineto{\pgfqpoint{1.131789in}{0.532434in}}%
\pgfpathlineto{\pgfqpoint{1.148491in}{0.534733in}}%
\pgfpathlineto{\pgfqpoint{1.165194in}{0.537156in}}%
\pgfpathlineto{\pgfqpoint{1.181897in}{0.539710in}}%
\pgfpathlineto{\pgfqpoint{1.198599in}{0.542403in}}%
\pgfpathlineto{\pgfqpoint{1.215302in}{0.545243in}}%
\pgfpathlineto{\pgfqpoint{1.232004in}{0.548236in}}%
\pgfpathlineto{\pgfqpoint{1.248707in}{0.551392in}}%
\pgfpathlineto{\pgfqpoint{1.265409in}{0.554717in}}%
\pgfpathlineto{\pgfqpoint{1.282112in}{0.558219in}}%
\pgfpathlineto{\pgfqpoint{1.298815in}{0.561907in}}%
\pgfpathlineto{\pgfqpoint{1.315517in}{0.565787in}}%
\pgfpathlineto{\pgfqpoint{1.332220in}{0.569868in}}%
\pgfpathlineto{\pgfqpoint{1.348922in}{0.574157in}}%
\pgfpathlineto{\pgfqpoint{1.365625in}{0.578662in}}%
\pgfusepath{stroke}%
\end{pgfscope}%
\begin{pgfscope}%
\pgfpathrectangle{\pgfqpoint{0.687500in}{0.385000in}}{\pgfqpoint{4.262500in}{2.695000in}}%
\pgfusepath{clip}%
\pgfsetrectcap%
\pgfsetroundjoin%
\pgfsetlinewidth{1.505625pt}%
\definecolor{currentstroke}{rgb}{1.000000,0.498039,0.054902}%
\pgfsetstrokecolor{currentstroke}%
\pgfsetdash{}{0pt}%
\pgfpathmoveto{\pgfqpoint{1.365625in}{0.578662in}}%
\pgfpathlineto{\pgfqpoint{1.382328in}{0.583387in}}%
\pgfpathlineto{\pgfqpoint{1.399030in}{0.588325in}}%
\pgfpathlineto{\pgfqpoint{1.415733in}{0.593464in}}%
\pgfpathlineto{\pgfqpoint{1.432435in}{0.598793in}}%
\pgfpathlineto{\pgfqpoint{1.449138in}{0.604300in}}%
\pgfpathlineto{\pgfqpoint{1.465841in}{0.609975in}}%
\pgfpathlineto{\pgfqpoint{1.482543in}{0.615807in}}%
\pgfpathlineto{\pgfqpoint{1.499246in}{0.621783in}}%
\pgfpathlineto{\pgfqpoint{1.515948in}{0.627894in}}%
\pgfpathlineto{\pgfqpoint{1.532651in}{0.634127in}}%
\pgfpathlineto{\pgfqpoint{1.549353in}{0.640471in}}%
\pgfpathlineto{\pgfqpoint{1.566056in}{0.646916in}}%
\pgfpathlineto{\pgfqpoint{1.582759in}{0.653450in}}%
\pgfpathlineto{\pgfqpoint{1.599461in}{0.660062in}}%
\pgfpathlineto{\pgfqpoint{1.616164in}{0.666740in}}%
\pgfpathlineto{\pgfqpoint{1.632866in}{0.673474in}}%
\pgfpathlineto{\pgfqpoint{1.649569in}{0.680251in}}%
\pgfpathlineto{\pgfqpoint{1.666272in}{0.687062in}}%
\pgfpathlineto{\pgfqpoint{1.682974in}{0.693895in}}%
\pgfpathlineto{\pgfqpoint{1.699677in}{0.700738in}}%
\pgfpathlineto{\pgfqpoint{1.716379in}{0.707580in}}%
\pgfpathlineto{\pgfqpoint{1.733082in}{0.714410in}}%
\pgfpathlineto{\pgfqpoint{1.749784in}{0.721218in}}%
\pgfpathlineto{\pgfqpoint{1.766487in}{0.727991in}}%
\pgfpathlineto{\pgfqpoint{1.783190in}{0.734718in}}%
\pgfpathlineto{\pgfqpoint{1.799892in}{0.741389in}}%
\pgfpathlineto{\pgfqpoint{1.816595in}{0.747992in}}%
\pgfpathlineto{\pgfqpoint{1.833297in}{0.754515in}}%
\pgfpathlineto{\pgfqpoint{1.850000in}{0.760948in}}%
\pgfusepath{stroke}%
\end{pgfscope}%
\begin{pgfscope}%
\pgfpathrectangle{\pgfqpoint{0.687500in}{0.385000in}}{\pgfqpoint{4.262500in}{2.695000in}}%
\pgfusepath{clip}%
\pgfsetrectcap%
\pgfsetroundjoin%
\pgfsetlinewidth{1.505625pt}%
\definecolor{currentstroke}{rgb}{0.172549,0.627451,0.172549}%
\pgfsetstrokecolor{currentstroke}%
\pgfsetdash{}{0pt}%
\pgfpathmoveto{\pgfqpoint{1.850000in}{0.760948in}}%
\pgfpathlineto{\pgfqpoint{1.866703in}{0.767302in}}%
\pgfpathlineto{\pgfqpoint{1.883405in}{0.773678in}}%
\pgfpathlineto{\pgfqpoint{1.900108in}{0.780198in}}%
\pgfpathlineto{\pgfqpoint{1.916810in}{0.786985in}}%
\pgfpathlineto{\pgfqpoint{1.933513in}{0.794164in}}%
\pgfpathlineto{\pgfqpoint{1.950216in}{0.801857in}}%
\pgfpathlineto{\pgfqpoint{1.966918in}{0.810186in}}%
\pgfpathlineto{\pgfqpoint{1.983621in}{0.819277in}}%
\pgfpathlineto{\pgfqpoint{2.000323in}{0.829250in}}%
\pgfpathlineto{\pgfqpoint{2.017026in}{0.840230in}}%
\pgfpathlineto{\pgfqpoint{2.033728in}{0.852339in}}%
\pgfpathlineto{\pgfqpoint{2.050431in}{0.865702in}}%
\pgfpathlineto{\pgfqpoint{2.067134in}{0.880440in}}%
\pgfpathlineto{\pgfqpoint{2.083836in}{0.896677in}}%
\pgfpathlineto{\pgfqpoint{2.100539in}{0.914537in}}%
\pgfpathlineto{\pgfqpoint{2.117241in}{0.934141in}}%
\pgfpathlineto{\pgfqpoint{2.133944in}{0.955614in}}%
\pgfpathlineto{\pgfqpoint{2.150647in}{0.979079in}}%
\pgfpathlineto{\pgfqpoint{2.167349in}{1.004658in}}%
\pgfpathlineto{\pgfqpoint{2.184052in}{1.032475in}}%
\pgfpathlineto{\pgfqpoint{2.200754in}{1.062652in}}%
\pgfpathlineto{\pgfqpoint{2.217457in}{1.095314in}}%
\pgfpathlineto{\pgfqpoint{2.234159in}{1.130583in}}%
\pgfpathlineto{\pgfqpoint{2.250862in}{1.168582in}}%
\pgfpathlineto{\pgfqpoint{2.267565in}{1.209435in}}%
\pgfpathlineto{\pgfqpoint{2.284267in}{1.253264in}}%
\pgfpathlineto{\pgfqpoint{2.300970in}{1.300193in}}%
\pgfpathlineto{\pgfqpoint{2.317672in}{1.350344in}}%
\pgfpathlineto{\pgfqpoint{2.334375in}{1.403841in}}%
\pgfusepath{stroke}%
\end{pgfscope}%
\begin{pgfscope}%
\pgfpathrectangle{\pgfqpoint{0.687500in}{0.385000in}}{\pgfqpoint{4.262500in}{2.695000in}}%
\pgfusepath{clip}%
\pgfsetrectcap%
\pgfsetroundjoin%
\pgfsetlinewidth{1.505625pt}%
\definecolor{currentstroke}{rgb}{0.839216,0.152941,0.156863}%
\pgfsetstrokecolor{currentstroke}%
\pgfsetdash{}{0pt}%
\pgfpathmoveto{\pgfqpoint{2.334375in}{1.403841in}}%
\pgfpathlineto{\pgfqpoint{2.351078in}{1.460726in}}%
\pgfpathlineto{\pgfqpoint{2.367780in}{1.520708in}}%
\pgfpathlineto{\pgfqpoint{2.384483in}{1.583419in}}%
\pgfpathlineto{\pgfqpoint{2.401185in}{1.648486in}}%
\pgfpathlineto{\pgfqpoint{2.417888in}{1.715541in}}%
\pgfpathlineto{\pgfqpoint{2.434591in}{1.784212in}}%
\pgfpathlineto{\pgfqpoint{2.451293in}{1.854129in}}%
\pgfpathlineto{\pgfqpoint{2.467996in}{1.924921in}}%
\pgfpathlineto{\pgfqpoint{2.484698in}{1.996218in}}%
\pgfpathlineto{\pgfqpoint{2.501401in}{2.067648in}}%
\pgfpathlineto{\pgfqpoint{2.518103in}{2.138843in}}%
\pgfpathlineto{\pgfqpoint{2.534806in}{2.209430in}}%
\pgfpathlineto{\pgfqpoint{2.551509in}{2.279040in}}%
\pgfpathlineto{\pgfqpoint{2.568211in}{2.347302in}}%
\pgfpathlineto{\pgfqpoint{2.584914in}{2.413846in}}%
\pgfpathlineto{\pgfqpoint{2.601616in}{2.478300in}}%
\pgfpathlineto{\pgfqpoint{2.618319in}{2.540294in}}%
\pgfpathlineto{\pgfqpoint{2.635022in}{2.599459in}}%
\pgfpathlineto{\pgfqpoint{2.651724in}{2.655423in}}%
\pgfpathlineto{\pgfqpoint{2.668427in}{2.707815in}}%
\pgfpathlineto{\pgfqpoint{2.685129in}{2.756266in}}%
\pgfpathlineto{\pgfqpoint{2.701832in}{2.800404in}}%
\pgfpathlineto{\pgfqpoint{2.718534in}{2.839859in}}%
\pgfpathlineto{\pgfqpoint{2.735237in}{2.874261in}}%
\pgfpathlineto{\pgfqpoint{2.751940in}{2.903239in}}%
\pgfpathlineto{\pgfqpoint{2.768642in}{2.926422in}}%
\pgfpathlineto{\pgfqpoint{2.785345in}{2.943441in}}%
\pgfpathlineto{\pgfqpoint{2.802047in}{2.953923in}}%
\pgfpathlineto{\pgfqpoint{2.818750in}{2.957500in}}%
\pgfusepath{stroke}%
\end{pgfscope}%
\begin{pgfscope}%
\pgfpathrectangle{\pgfqpoint{0.687500in}{0.385000in}}{\pgfqpoint{4.262500in}{2.695000in}}%
\pgfusepath{clip}%
\pgfsetrectcap%
\pgfsetroundjoin%
\pgfsetlinewidth{1.505625pt}%
\definecolor{currentstroke}{rgb}{0.580392,0.403922,0.741176}%
\pgfsetstrokecolor{currentstroke}%
\pgfsetdash{}{0pt}%
\pgfpathmoveto{\pgfqpoint{2.818750in}{2.957500in}}%
\pgfpathlineto{\pgfqpoint{2.835453in}{2.953923in}}%
\pgfpathlineto{\pgfqpoint{2.852155in}{2.943441in}}%
\pgfpathlineto{\pgfqpoint{2.868858in}{2.926422in}}%
\pgfpathlineto{\pgfqpoint{2.885560in}{2.903239in}}%
\pgfpathlineto{\pgfqpoint{2.902263in}{2.874261in}}%
\pgfpathlineto{\pgfqpoint{2.918966in}{2.839859in}}%
\pgfpathlineto{\pgfqpoint{2.935668in}{2.800404in}}%
\pgfpathlineto{\pgfqpoint{2.952371in}{2.756266in}}%
\pgfpathlineto{\pgfqpoint{2.969073in}{2.707815in}}%
\pgfpathlineto{\pgfqpoint{2.985776in}{2.655423in}}%
\pgfpathlineto{\pgfqpoint{3.002478in}{2.599459in}}%
\pgfpathlineto{\pgfqpoint{3.019181in}{2.540294in}}%
\pgfpathlineto{\pgfqpoint{3.035884in}{2.478300in}}%
\pgfpathlineto{\pgfqpoint{3.052586in}{2.413846in}}%
\pgfpathlineto{\pgfqpoint{3.069289in}{2.347302in}}%
\pgfpathlineto{\pgfqpoint{3.085991in}{2.279040in}}%
\pgfpathlineto{\pgfqpoint{3.102694in}{2.209430in}}%
\pgfpathlineto{\pgfqpoint{3.119397in}{2.138843in}}%
\pgfpathlineto{\pgfqpoint{3.136099in}{2.067648in}}%
\pgfpathlineto{\pgfqpoint{3.152802in}{1.996218in}}%
\pgfpathlineto{\pgfqpoint{3.169504in}{1.924921in}}%
\pgfpathlineto{\pgfqpoint{3.186207in}{1.854129in}}%
\pgfpathlineto{\pgfqpoint{3.202909in}{1.784212in}}%
\pgfpathlineto{\pgfqpoint{3.219612in}{1.715541in}}%
\pgfpathlineto{\pgfqpoint{3.236315in}{1.648486in}}%
\pgfpathlineto{\pgfqpoint{3.253017in}{1.583419in}}%
\pgfpathlineto{\pgfqpoint{3.269720in}{1.520708in}}%
\pgfpathlineto{\pgfqpoint{3.286422in}{1.460726in}}%
\pgfpathlineto{\pgfqpoint{3.303125in}{1.403841in}}%
\pgfusepath{stroke}%
\end{pgfscope}%
\begin{pgfscope}%
\pgfpathrectangle{\pgfqpoint{0.687500in}{0.385000in}}{\pgfqpoint{4.262500in}{2.695000in}}%
\pgfusepath{clip}%
\pgfsetrectcap%
\pgfsetroundjoin%
\pgfsetlinewidth{1.505625pt}%
\definecolor{currentstroke}{rgb}{0.549020,0.337255,0.294118}%
\pgfsetstrokecolor{currentstroke}%
\pgfsetdash{}{0pt}%
\pgfpathmoveto{\pgfqpoint{3.303125in}{1.403841in}}%
\pgfpathlineto{\pgfqpoint{3.319828in}{1.350344in}}%
\pgfpathlineto{\pgfqpoint{3.336530in}{1.300193in}}%
\pgfpathlineto{\pgfqpoint{3.353233in}{1.253264in}}%
\pgfpathlineto{\pgfqpoint{3.369935in}{1.209435in}}%
\pgfpathlineto{\pgfqpoint{3.386638in}{1.168582in}}%
\pgfpathlineto{\pgfqpoint{3.403341in}{1.130583in}}%
\pgfpathlineto{\pgfqpoint{3.420043in}{1.095314in}}%
\pgfpathlineto{\pgfqpoint{3.436746in}{1.062652in}}%
\pgfpathlineto{\pgfqpoint{3.453448in}{1.032475in}}%
\pgfpathlineto{\pgfqpoint{3.470151in}{1.004658in}}%
\pgfpathlineto{\pgfqpoint{3.486853in}{0.979079in}}%
\pgfpathlineto{\pgfqpoint{3.503556in}{0.955614in}}%
\pgfpathlineto{\pgfqpoint{3.520259in}{0.934141in}}%
\pgfpathlineto{\pgfqpoint{3.536961in}{0.914537in}}%
\pgfpathlineto{\pgfqpoint{3.553664in}{0.896677in}}%
\pgfpathlineto{\pgfqpoint{3.570366in}{0.880440in}}%
\pgfpathlineto{\pgfqpoint{3.587069in}{0.865702in}}%
\pgfpathlineto{\pgfqpoint{3.603772in}{0.852339in}}%
\pgfpathlineto{\pgfqpoint{3.620474in}{0.840230in}}%
\pgfpathlineto{\pgfqpoint{3.637177in}{0.829250in}}%
\pgfpathlineto{\pgfqpoint{3.653879in}{0.819277in}}%
\pgfpathlineto{\pgfqpoint{3.670582in}{0.810186in}}%
\pgfpathlineto{\pgfqpoint{3.687284in}{0.801857in}}%
\pgfpathlineto{\pgfqpoint{3.703987in}{0.794164in}}%
\pgfpathlineto{\pgfqpoint{3.720690in}{0.786985in}}%
\pgfpathlineto{\pgfqpoint{3.737392in}{0.780198in}}%
\pgfpathlineto{\pgfqpoint{3.754095in}{0.773678in}}%
\pgfpathlineto{\pgfqpoint{3.770797in}{0.767302in}}%
\pgfpathlineto{\pgfqpoint{3.787500in}{0.760948in}}%
\pgfusepath{stroke}%
\end{pgfscope}%
\begin{pgfscope}%
\pgfpathrectangle{\pgfqpoint{0.687500in}{0.385000in}}{\pgfqpoint{4.262500in}{2.695000in}}%
\pgfusepath{clip}%
\pgfsetrectcap%
\pgfsetroundjoin%
\pgfsetlinewidth{1.505625pt}%
\definecolor{currentstroke}{rgb}{0.890196,0.466667,0.760784}%
\pgfsetstrokecolor{currentstroke}%
\pgfsetdash{}{0pt}%
\pgfpathmoveto{\pgfqpoint{3.787500in}{0.760948in}}%
\pgfpathlineto{\pgfqpoint{3.804203in}{0.754515in}}%
\pgfpathlineto{\pgfqpoint{3.820905in}{0.747992in}}%
\pgfpathlineto{\pgfqpoint{3.837608in}{0.741389in}}%
\pgfpathlineto{\pgfqpoint{3.854310in}{0.734718in}}%
\pgfpathlineto{\pgfqpoint{3.871013in}{0.727991in}}%
\pgfpathlineto{\pgfqpoint{3.887716in}{0.721218in}}%
\pgfpathlineto{\pgfqpoint{3.904418in}{0.714410in}}%
\pgfpathlineto{\pgfqpoint{3.921121in}{0.707580in}}%
\pgfpathlineto{\pgfqpoint{3.937823in}{0.700738in}}%
\pgfpathlineto{\pgfqpoint{3.954526in}{0.693895in}}%
\pgfpathlineto{\pgfqpoint{3.971228in}{0.687062in}}%
\pgfpathlineto{\pgfqpoint{3.987931in}{0.680251in}}%
\pgfpathlineto{\pgfqpoint{4.004634in}{0.673474in}}%
\pgfpathlineto{\pgfqpoint{4.021336in}{0.666740in}}%
\pgfpathlineto{\pgfqpoint{4.038039in}{0.660062in}}%
\pgfpathlineto{\pgfqpoint{4.054741in}{0.653450in}}%
\pgfpathlineto{\pgfqpoint{4.071444in}{0.646916in}}%
\pgfpathlineto{\pgfqpoint{4.088147in}{0.640471in}}%
\pgfpathlineto{\pgfqpoint{4.104849in}{0.634127in}}%
\pgfpathlineto{\pgfqpoint{4.121552in}{0.627894in}}%
\pgfpathlineto{\pgfqpoint{4.138254in}{0.621783in}}%
\pgfpathlineto{\pgfqpoint{4.154957in}{0.615807in}}%
\pgfpathlineto{\pgfqpoint{4.171659in}{0.609975in}}%
\pgfpathlineto{\pgfqpoint{4.188362in}{0.604300in}}%
\pgfpathlineto{\pgfqpoint{4.205065in}{0.598793in}}%
\pgfpathlineto{\pgfqpoint{4.221767in}{0.593464in}}%
\pgfpathlineto{\pgfqpoint{4.238470in}{0.588325in}}%
\pgfpathlineto{\pgfqpoint{4.255172in}{0.583387in}}%
\pgfpathlineto{\pgfqpoint{4.271875in}{0.578662in}}%
\pgfusepath{stroke}%
\end{pgfscope}%
\begin{pgfscope}%
\pgfpathrectangle{\pgfqpoint{0.687500in}{0.385000in}}{\pgfqpoint{4.262500in}{2.695000in}}%
\pgfusepath{clip}%
\pgfsetrectcap%
\pgfsetroundjoin%
\pgfsetlinewidth{1.505625pt}%
\definecolor{currentstroke}{rgb}{0.498039,0.498039,0.498039}%
\pgfsetstrokecolor{currentstroke}%
\pgfsetdash{}{0pt}%
\pgfpathmoveto{\pgfqpoint{4.271875in}{0.578662in}}%
\pgfpathlineto{\pgfqpoint{4.288578in}{0.574157in}}%
\pgfpathlineto{\pgfqpoint{4.305280in}{0.569868in}}%
\pgfpathlineto{\pgfqpoint{4.321983in}{0.565787in}}%
\pgfpathlineto{\pgfqpoint{4.338685in}{0.561907in}}%
\pgfpathlineto{\pgfqpoint{4.355388in}{0.558219in}}%
\pgfpathlineto{\pgfqpoint{4.372091in}{0.554717in}}%
\pgfpathlineto{\pgfqpoint{4.388793in}{0.551392in}}%
\pgfpathlineto{\pgfqpoint{4.405496in}{0.548236in}}%
\pgfpathlineto{\pgfqpoint{4.422198in}{0.545243in}}%
\pgfpathlineto{\pgfqpoint{4.438901in}{0.542403in}}%
\pgfpathlineto{\pgfqpoint{4.455603in}{0.539710in}}%
\pgfpathlineto{\pgfqpoint{4.472306in}{0.537156in}}%
\pgfpathlineto{\pgfqpoint{4.489009in}{0.534733in}}%
\pgfpathlineto{\pgfqpoint{4.505711in}{0.532434in}}%
\pgfpathlineto{\pgfqpoint{4.522414in}{0.530250in}}%
\pgfpathlineto{\pgfqpoint{4.539116in}{0.528174in}}%
\pgfpathlineto{\pgfqpoint{4.555819in}{0.526198in}}%
\pgfpathlineto{\pgfqpoint{4.572522in}{0.524314in}}%
\pgfpathlineto{\pgfqpoint{4.589224in}{0.522516in}}%
\pgfpathlineto{\pgfqpoint{4.605927in}{0.520795in}}%
\pgfpathlineto{\pgfqpoint{4.622629in}{0.519143in}}%
\pgfpathlineto{\pgfqpoint{4.639332in}{0.517552in}}%
\pgfpathlineto{\pgfqpoint{4.656034in}{0.516016in}}%
\pgfpathlineto{\pgfqpoint{4.672737in}{0.514526in}}%
\pgfpathlineto{\pgfqpoint{4.689440in}{0.513075in}}%
\pgfpathlineto{\pgfqpoint{4.706142in}{0.511654in}}%
\pgfpathlineto{\pgfqpoint{4.722845in}{0.510256in}}%
\pgfpathlineto{\pgfqpoint{4.739547in}{0.508874in}}%
\pgfpathlineto{\pgfqpoint{4.756250in}{0.507500in}}%
\pgfusepath{stroke}%
\end{pgfscope}%
\begin{pgfscope}%
\pgfsetrectcap%
\pgfsetmiterjoin%
\pgfsetlinewidth{0.803000pt}%
\definecolor{currentstroke}{rgb}{0.000000,0.000000,0.000000}%
\pgfsetstrokecolor{currentstroke}%
\pgfsetdash{}{0pt}%
\pgfpathmoveto{\pgfqpoint{0.687500in}{0.385000in}}%
\pgfpathlineto{\pgfqpoint{0.687500in}{3.080000in}}%
\pgfusepath{stroke}%
\end{pgfscope}%
\begin{pgfscope}%
\pgfsetrectcap%
\pgfsetmiterjoin%
\pgfsetlinewidth{0.803000pt}%
\definecolor{currentstroke}{rgb}{0.000000,0.000000,0.000000}%
\pgfsetstrokecolor{currentstroke}%
\pgfsetdash{}{0pt}%
\pgfpathmoveto{\pgfqpoint{4.950000in}{0.385000in}}%
\pgfpathlineto{\pgfqpoint{4.950000in}{3.080000in}}%
\pgfusepath{stroke}%
\end{pgfscope}%
\begin{pgfscope}%
\pgfsetrectcap%
\pgfsetmiterjoin%
\pgfsetlinewidth{0.803000pt}%
\definecolor{currentstroke}{rgb}{0.000000,0.000000,0.000000}%
\pgfsetstrokecolor{currentstroke}%
\pgfsetdash{}{0pt}%
\pgfpathmoveto{\pgfqpoint{0.687500in}{0.385000in}}%
\pgfpathlineto{\pgfqpoint{4.950000in}{0.385000in}}%
\pgfusepath{stroke}%
\end{pgfscope}%
\begin{pgfscope}%
\pgfsetrectcap%
\pgfsetmiterjoin%
\pgfsetlinewidth{0.803000pt}%
\definecolor{currentstroke}{rgb}{0.000000,0.000000,0.000000}%
\pgfsetstrokecolor{currentstroke}%
\pgfsetdash{}{0pt}%
\pgfpathmoveto{\pgfqpoint{0.687500in}{3.080000in}}%
\pgfpathlineto{\pgfqpoint{4.950000in}{3.080000in}}%
\pgfusepath{stroke}%
\end{pgfscope}%
\begin{pgfscope}%
\definecolor{textcolor}{rgb}{0.000000,0.000000,0.000000}%
\pgfsetstrokecolor{textcolor}%
\pgfsetfillcolor{textcolor}%
\pgftext[x=2.818750in,y=3.163333in,,base]{\color{textcolor}\rmfamily\fontsize{12.000000}{14.400000}\selectfont N = 8}%
\end{pgfscope}%
\begin{pgfscope}%
\definecolor{textcolor}{rgb}{0.000000,0.000000,0.000000}%
\pgfsetstrokecolor{textcolor}%
\pgfsetfillcolor{textcolor}%
\pgftext[x=2.750000in,y=3.430000in,,top]{\color{textcolor}\rmfamily\fontsize{12.000000}{14.400000}\selectfont Naturalny splajn kubiczny}%
\end{pgfscope}%
\end{pgfpicture}%
\makeatother%
\endgroup%
        
    \end{center}
\end{figure}

Dla \(N=8\) splajn już na pierwszy rzut oka nie jest wystarczająco dokładny,
aby korzystać z niego jako interpolacji funkcji \(f\). Spróbujmy zwiększyć N.

\begin{figure}[H]
    \begin{center}
        %% Creator: Matplotlib, PGF backend
%%
%% To include the figure in your LaTeX document, write
%%   \input{<filename>.pgf}
%%
%% Make sure the required packages are loaded in your preamble
%%   \usepackage{pgf}
%%
%% Figures using additional raster images can only be included by \input if
%% they are in the same directory as the main LaTeX file. For loading figures
%% from other directories you can use the `import` package
%%   \usepackage{import}
%% and then include the figures with
%%   \import{<path to file>}{<filename>.pgf}
%%
%% Matplotlib used the following preamble
%%
\begingroup%
\makeatletter%
\begin{pgfpicture}%
\pgfpathrectangle{\pgfpointorigin}{\pgfqpoint{5.500000in}{3.500000in}}%
\pgfusepath{use as bounding box, clip}%
\begin{pgfscope}%
\pgfsetbuttcap%
\pgfsetmiterjoin%
\definecolor{currentfill}{rgb}{1.000000,1.000000,1.000000}%
\pgfsetfillcolor{currentfill}%
\pgfsetlinewidth{0.000000pt}%
\definecolor{currentstroke}{rgb}{1.000000,1.000000,1.000000}%
\pgfsetstrokecolor{currentstroke}%
\pgfsetdash{}{0pt}%
\pgfpathmoveto{\pgfqpoint{0.000000in}{0.000000in}}%
\pgfpathlineto{\pgfqpoint{5.500000in}{0.000000in}}%
\pgfpathlineto{\pgfqpoint{5.500000in}{3.500000in}}%
\pgfpathlineto{\pgfqpoint{0.000000in}{3.500000in}}%
\pgfpathclose%
\pgfusepath{fill}%
\end{pgfscope}%
\begin{pgfscope}%
\pgfsetbuttcap%
\pgfsetmiterjoin%
\definecolor{currentfill}{rgb}{1.000000,1.000000,1.000000}%
\pgfsetfillcolor{currentfill}%
\pgfsetlinewidth{0.000000pt}%
\definecolor{currentstroke}{rgb}{0.000000,0.000000,0.000000}%
\pgfsetstrokecolor{currentstroke}%
\pgfsetstrokeopacity{0.000000}%
\pgfsetdash{}{0pt}%
\pgfpathmoveto{\pgfqpoint{0.687500in}{0.385000in}}%
\pgfpathlineto{\pgfqpoint{4.950000in}{0.385000in}}%
\pgfpathlineto{\pgfqpoint{4.950000in}{3.080000in}}%
\pgfpathlineto{\pgfqpoint{0.687500in}{3.080000in}}%
\pgfpathclose%
\pgfusepath{fill}%
\end{pgfscope}%
\begin{pgfscope}%
\pgfsetbuttcap%
\pgfsetroundjoin%
\definecolor{currentfill}{rgb}{0.000000,0.000000,0.000000}%
\pgfsetfillcolor{currentfill}%
\pgfsetlinewidth{0.803000pt}%
\definecolor{currentstroke}{rgb}{0.000000,0.000000,0.000000}%
\pgfsetstrokecolor{currentstroke}%
\pgfsetdash{}{0pt}%
\pgfsys@defobject{currentmarker}{\pgfqpoint{0.000000in}{-0.048611in}}{\pgfqpoint{0.000000in}{0.000000in}}{%
\pgfpathmoveto{\pgfqpoint{0.000000in}{0.000000in}}%
\pgfpathlineto{\pgfqpoint{0.000000in}{-0.048611in}}%
\pgfusepath{stroke,fill}%
}%
\begin{pgfscope}%
\pgfsys@transformshift{0.881250in}{0.385000in}%
\pgfsys@useobject{currentmarker}{}%
\end{pgfscope}%
\end{pgfscope}%
\begin{pgfscope}%
\definecolor{textcolor}{rgb}{0.000000,0.000000,0.000000}%
\pgfsetstrokecolor{textcolor}%
\pgfsetfillcolor{textcolor}%
\pgftext[x=0.881250in,y=0.287778in,,top]{\color{textcolor}\rmfamily\fontsize{10.000000}{12.000000}\selectfont \(\displaystyle -1.00\)}%
\end{pgfscope}%
\begin{pgfscope}%
\pgfsetbuttcap%
\pgfsetroundjoin%
\definecolor{currentfill}{rgb}{0.000000,0.000000,0.000000}%
\pgfsetfillcolor{currentfill}%
\pgfsetlinewidth{0.803000pt}%
\definecolor{currentstroke}{rgb}{0.000000,0.000000,0.000000}%
\pgfsetstrokecolor{currentstroke}%
\pgfsetdash{}{0pt}%
\pgfsys@defobject{currentmarker}{\pgfqpoint{0.000000in}{-0.048611in}}{\pgfqpoint{0.000000in}{0.000000in}}{%
\pgfpathmoveto{\pgfqpoint{0.000000in}{0.000000in}}%
\pgfpathlineto{\pgfqpoint{0.000000in}{-0.048611in}}%
\pgfusepath{stroke,fill}%
}%
\begin{pgfscope}%
\pgfsys@transformshift{1.365625in}{0.385000in}%
\pgfsys@useobject{currentmarker}{}%
\end{pgfscope}%
\end{pgfscope}%
\begin{pgfscope}%
\definecolor{textcolor}{rgb}{0.000000,0.000000,0.000000}%
\pgfsetstrokecolor{textcolor}%
\pgfsetfillcolor{textcolor}%
\pgftext[x=1.365625in,y=0.287778in,,top]{\color{textcolor}\rmfamily\fontsize{10.000000}{12.000000}\selectfont \(\displaystyle -0.75\)}%
\end{pgfscope}%
\begin{pgfscope}%
\pgfsetbuttcap%
\pgfsetroundjoin%
\definecolor{currentfill}{rgb}{0.000000,0.000000,0.000000}%
\pgfsetfillcolor{currentfill}%
\pgfsetlinewidth{0.803000pt}%
\definecolor{currentstroke}{rgb}{0.000000,0.000000,0.000000}%
\pgfsetstrokecolor{currentstroke}%
\pgfsetdash{}{0pt}%
\pgfsys@defobject{currentmarker}{\pgfqpoint{0.000000in}{-0.048611in}}{\pgfqpoint{0.000000in}{0.000000in}}{%
\pgfpathmoveto{\pgfqpoint{0.000000in}{0.000000in}}%
\pgfpathlineto{\pgfqpoint{0.000000in}{-0.048611in}}%
\pgfusepath{stroke,fill}%
}%
\begin{pgfscope}%
\pgfsys@transformshift{1.850000in}{0.385000in}%
\pgfsys@useobject{currentmarker}{}%
\end{pgfscope}%
\end{pgfscope}%
\begin{pgfscope}%
\definecolor{textcolor}{rgb}{0.000000,0.000000,0.000000}%
\pgfsetstrokecolor{textcolor}%
\pgfsetfillcolor{textcolor}%
\pgftext[x=1.850000in,y=0.287778in,,top]{\color{textcolor}\rmfamily\fontsize{10.000000}{12.000000}\selectfont \(\displaystyle -0.50\)}%
\end{pgfscope}%
\begin{pgfscope}%
\pgfsetbuttcap%
\pgfsetroundjoin%
\definecolor{currentfill}{rgb}{0.000000,0.000000,0.000000}%
\pgfsetfillcolor{currentfill}%
\pgfsetlinewidth{0.803000pt}%
\definecolor{currentstroke}{rgb}{0.000000,0.000000,0.000000}%
\pgfsetstrokecolor{currentstroke}%
\pgfsetdash{}{0pt}%
\pgfsys@defobject{currentmarker}{\pgfqpoint{0.000000in}{-0.048611in}}{\pgfqpoint{0.000000in}{0.000000in}}{%
\pgfpathmoveto{\pgfqpoint{0.000000in}{0.000000in}}%
\pgfpathlineto{\pgfqpoint{0.000000in}{-0.048611in}}%
\pgfusepath{stroke,fill}%
}%
\begin{pgfscope}%
\pgfsys@transformshift{2.334375in}{0.385000in}%
\pgfsys@useobject{currentmarker}{}%
\end{pgfscope}%
\end{pgfscope}%
\begin{pgfscope}%
\definecolor{textcolor}{rgb}{0.000000,0.000000,0.000000}%
\pgfsetstrokecolor{textcolor}%
\pgfsetfillcolor{textcolor}%
\pgftext[x=2.334375in,y=0.287778in,,top]{\color{textcolor}\rmfamily\fontsize{10.000000}{12.000000}\selectfont \(\displaystyle -0.25\)}%
\end{pgfscope}%
\begin{pgfscope}%
\pgfsetbuttcap%
\pgfsetroundjoin%
\definecolor{currentfill}{rgb}{0.000000,0.000000,0.000000}%
\pgfsetfillcolor{currentfill}%
\pgfsetlinewidth{0.803000pt}%
\definecolor{currentstroke}{rgb}{0.000000,0.000000,0.000000}%
\pgfsetstrokecolor{currentstroke}%
\pgfsetdash{}{0pt}%
\pgfsys@defobject{currentmarker}{\pgfqpoint{0.000000in}{-0.048611in}}{\pgfqpoint{0.000000in}{0.000000in}}{%
\pgfpathmoveto{\pgfqpoint{0.000000in}{0.000000in}}%
\pgfpathlineto{\pgfqpoint{0.000000in}{-0.048611in}}%
\pgfusepath{stroke,fill}%
}%
\begin{pgfscope}%
\pgfsys@transformshift{2.818750in}{0.385000in}%
\pgfsys@useobject{currentmarker}{}%
\end{pgfscope}%
\end{pgfscope}%
\begin{pgfscope}%
\definecolor{textcolor}{rgb}{0.000000,0.000000,0.000000}%
\pgfsetstrokecolor{textcolor}%
\pgfsetfillcolor{textcolor}%
\pgftext[x=2.818750in,y=0.287778in,,top]{\color{textcolor}\rmfamily\fontsize{10.000000}{12.000000}\selectfont \(\displaystyle 0.00\)}%
\end{pgfscope}%
\begin{pgfscope}%
\pgfsetbuttcap%
\pgfsetroundjoin%
\definecolor{currentfill}{rgb}{0.000000,0.000000,0.000000}%
\pgfsetfillcolor{currentfill}%
\pgfsetlinewidth{0.803000pt}%
\definecolor{currentstroke}{rgb}{0.000000,0.000000,0.000000}%
\pgfsetstrokecolor{currentstroke}%
\pgfsetdash{}{0pt}%
\pgfsys@defobject{currentmarker}{\pgfqpoint{0.000000in}{-0.048611in}}{\pgfqpoint{0.000000in}{0.000000in}}{%
\pgfpathmoveto{\pgfqpoint{0.000000in}{0.000000in}}%
\pgfpathlineto{\pgfqpoint{0.000000in}{-0.048611in}}%
\pgfusepath{stroke,fill}%
}%
\begin{pgfscope}%
\pgfsys@transformshift{3.303125in}{0.385000in}%
\pgfsys@useobject{currentmarker}{}%
\end{pgfscope}%
\end{pgfscope}%
\begin{pgfscope}%
\definecolor{textcolor}{rgb}{0.000000,0.000000,0.000000}%
\pgfsetstrokecolor{textcolor}%
\pgfsetfillcolor{textcolor}%
\pgftext[x=3.303125in,y=0.287778in,,top]{\color{textcolor}\rmfamily\fontsize{10.000000}{12.000000}\selectfont \(\displaystyle 0.25\)}%
\end{pgfscope}%
\begin{pgfscope}%
\pgfsetbuttcap%
\pgfsetroundjoin%
\definecolor{currentfill}{rgb}{0.000000,0.000000,0.000000}%
\pgfsetfillcolor{currentfill}%
\pgfsetlinewidth{0.803000pt}%
\definecolor{currentstroke}{rgb}{0.000000,0.000000,0.000000}%
\pgfsetstrokecolor{currentstroke}%
\pgfsetdash{}{0pt}%
\pgfsys@defobject{currentmarker}{\pgfqpoint{0.000000in}{-0.048611in}}{\pgfqpoint{0.000000in}{0.000000in}}{%
\pgfpathmoveto{\pgfqpoint{0.000000in}{0.000000in}}%
\pgfpathlineto{\pgfqpoint{0.000000in}{-0.048611in}}%
\pgfusepath{stroke,fill}%
}%
\begin{pgfscope}%
\pgfsys@transformshift{3.787500in}{0.385000in}%
\pgfsys@useobject{currentmarker}{}%
\end{pgfscope}%
\end{pgfscope}%
\begin{pgfscope}%
\definecolor{textcolor}{rgb}{0.000000,0.000000,0.000000}%
\pgfsetstrokecolor{textcolor}%
\pgfsetfillcolor{textcolor}%
\pgftext[x=3.787500in,y=0.287778in,,top]{\color{textcolor}\rmfamily\fontsize{10.000000}{12.000000}\selectfont \(\displaystyle 0.50\)}%
\end{pgfscope}%
\begin{pgfscope}%
\pgfsetbuttcap%
\pgfsetroundjoin%
\definecolor{currentfill}{rgb}{0.000000,0.000000,0.000000}%
\pgfsetfillcolor{currentfill}%
\pgfsetlinewidth{0.803000pt}%
\definecolor{currentstroke}{rgb}{0.000000,0.000000,0.000000}%
\pgfsetstrokecolor{currentstroke}%
\pgfsetdash{}{0pt}%
\pgfsys@defobject{currentmarker}{\pgfqpoint{0.000000in}{-0.048611in}}{\pgfqpoint{0.000000in}{0.000000in}}{%
\pgfpathmoveto{\pgfqpoint{0.000000in}{0.000000in}}%
\pgfpathlineto{\pgfqpoint{0.000000in}{-0.048611in}}%
\pgfusepath{stroke,fill}%
}%
\begin{pgfscope}%
\pgfsys@transformshift{4.271875in}{0.385000in}%
\pgfsys@useobject{currentmarker}{}%
\end{pgfscope}%
\end{pgfscope}%
\begin{pgfscope}%
\definecolor{textcolor}{rgb}{0.000000,0.000000,0.000000}%
\pgfsetstrokecolor{textcolor}%
\pgfsetfillcolor{textcolor}%
\pgftext[x=4.271875in,y=0.287778in,,top]{\color{textcolor}\rmfamily\fontsize{10.000000}{12.000000}\selectfont \(\displaystyle 0.75\)}%
\end{pgfscope}%
\begin{pgfscope}%
\pgfsetbuttcap%
\pgfsetroundjoin%
\definecolor{currentfill}{rgb}{0.000000,0.000000,0.000000}%
\pgfsetfillcolor{currentfill}%
\pgfsetlinewidth{0.803000pt}%
\definecolor{currentstroke}{rgb}{0.000000,0.000000,0.000000}%
\pgfsetstrokecolor{currentstroke}%
\pgfsetdash{}{0pt}%
\pgfsys@defobject{currentmarker}{\pgfqpoint{0.000000in}{-0.048611in}}{\pgfqpoint{0.000000in}{0.000000in}}{%
\pgfpathmoveto{\pgfqpoint{0.000000in}{0.000000in}}%
\pgfpathlineto{\pgfqpoint{0.000000in}{-0.048611in}}%
\pgfusepath{stroke,fill}%
}%
\begin{pgfscope}%
\pgfsys@transformshift{4.756250in}{0.385000in}%
\pgfsys@useobject{currentmarker}{}%
\end{pgfscope}%
\end{pgfscope}%
\begin{pgfscope}%
\definecolor{textcolor}{rgb}{0.000000,0.000000,0.000000}%
\pgfsetstrokecolor{textcolor}%
\pgfsetfillcolor{textcolor}%
\pgftext[x=4.756250in,y=0.287778in,,top]{\color{textcolor}\rmfamily\fontsize{10.000000}{12.000000}\selectfont \(\displaystyle 1.00\)}%
\end{pgfscope}%
\begin{pgfscope}%
\definecolor{textcolor}{rgb}{0.000000,0.000000,0.000000}%
\pgfsetstrokecolor{textcolor}%
\pgfsetfillcolor{textcolor}%
\pgftext[x=2.818750in,y=0.108766in,,top]{\color{textcolor}\rmfamily\fontsize{10.000000}{12.000000}\selectfont x}%
\end{pgfscope}%
\begin{pgfscope}%
\pgfsetbuttcap%
\pgfsetroundjoin%
\definecolor{currentfill}{rgb}{0.000000,0.000000,0.000000}%
\pgfsetfillcolor{currentfill}%
\pgfsetlinewidth{0.803000pt}%
\definecolor{currentstroke}{rgb}{0.000000,0.000000,0.000000}%
\pgfsetstrokecolor{currentstroke}%
\pgfsetdash{}{0pt}%
\pgfsys@defobject{currentmarker}{\pgfqpoint{-0.048611in}{0.000000in}}{\pgfqpoint{0.000000in}{0.000000in}}{%
\pgfpathmoveto{\pgfqpoint{0.000000in}{0.000000in}}%
\pgfpathlineto{\pgfqpoint{-0.048611in}{0.000000in}}%
\pgfusepath{stroke,fill}%
}%
\begin{pgfscope}%
\pgfsys@transformshift{0.687500in}{0.409500in}%
\pgfsys@useobject{currentmarker}{}%
\end{pgfscope}%
\end{pgfscope}%
\begin{pgfscope}%
\definecolor{textcolor}{rgb}{0.000000,0.000000,0.000000}%
\pgfsetstrokecolor{textcolor}%
\pgfsetfillcolor{textcolor}%
\pgftext[x=0.412808in,y=0.361275in,left,base]{\color{textcolor}\rmfamily\fontsize{10.000000}{12.000000}\selectfont \(\displaystyle 0.0\)}%
\end{pgfscope}%
\begin{pgfscope}%
\pgfsetbuttcap%
\pgfsetroundjoin%
\definecolor{currentfill}{rgb}{0.000000,0.000000,0.000000}%
\pgfsetfillcolor{currentfill}%
\pgfsetlinewidth{0.803000pt}%
\definecolor{currentstroke}{rgb}{0.000000,0.000000,0.000000}%
\pgfsetstrokecolor{currentstroke}%
\pgfsetdash{}{0pt}%
\pgfsys@defobject{currentmarker}{\pgfqpoint{-0.048611in}{0.000000in}}{\pgfqpoint{0.000000in}{0.000000in}}{%
\pgfpathmoveto{\pgfqpoint{0.000000in}{0.000000in}}%
\pgfpathlineto{\pgfqpoint{-0.048611in}{0.000000in}}%
\pgfusepath{stroke,fill}%
}%
\begin{pgfscope}%
\pgfsys@transformshift{0.687500in}{0.919100in}%
\pgfsys@useobject{currentmarker}{}%
\end{pgfscope}%
\end{pgfscope}%
\begin{pgfscope}%
\definecolor{textcolor}{rgb}{0.000000,0.000000,0.000000}%
\pgfsetstrokecolor{textcolor}%
\pgfsetfillcolor{textcolor}%
\pgftext[x=0.412808in,y=0.870875in,left,base]{\color{textcolor}\rmfamily\fontsize{10.000000}{12.000000}\selectfont \(\displaystyle 0.2\)}%
\end{pgfscope}%
\begin{pgfscope}%
\pgfsetbuttcap%
\pgfsetroundjoin%
\definecolor{currentfill}{rgb}{0.000000,0.000000,0.000000}%
\pgfsetfillcolor{currentfill}%
\pgfsetlinewidth{0.803000pt}%
\definecolor{currentstroke}{rgb}{0.000000,0.000000,0.000000}%
\pgfsetstrokecolor{currentstroke}%
\pgfsetdash{}{0pt}%
\pgfsys@defobject{currentmarker}{\pgfqpoint{-0.048611in}{0.000000in}}{\pgfqpoint{0.000000in}{0.000000in}}{%
\pgfpathmoveto{\pgfqpoint{0.000000in}{0.000000in}}%
\pgfpathlineto{\pgfqpoint{-0.048611in}{0.000000in}}%
\pgfusepath{stroke,fill}%
}%
\begin{pgfscope}%
\pgfsys@transformshift{0.687500in}{1.428700in}%
\pgfsys@useobject{currentmarker}{}%
\end{pgfscope}%
\end{pgfscope}%
\begin{pgfscope}%
\definecolor{textcolor}{rgb}{0.000000,0.000000,0.000000}%
\pgfsetstrokecolor{textcolor}%
\pgfsetfillcolor{textcolor}%
\pgftext[x=0.412808in,y=1.380475in,left,base]{\color{textcolor}\rmfamily\fontsize{10.000000}{12.000000}\selectfont \(\displaystyle 0.4\)}%
\end{pgfscope}%
\begin{pgfscope}%
\pgfsetbuttcap%
\pgfsetroundjoin%
\definecolor{currentfill}{rgb}{0.000000,0.000000,0.000000}%
\pgfsetfillcolor{currentfill}%
\pgfsetlinewidth{0.803000pt}%
\definecolor{currentstroke}{rgb}{0.000000,0.000000,0.000000}%
\pgfsetstrokecolor{currentstroke}%
\pgfsetdash{}{0pt}%
\pgfsys@defobject{currentmarker}{\pgfqpoint{-0.048611in}{0.000000in}}{\pgfqpoint{0.000000in}{0.000000in}}{%
\pgfpathmoveto{\pgfqpoint{0.000000in}{0.000000in}}%
\pgfpathlineto{\pgfqpoint{-0.048611in}{0.000000in}}%
\pgfusepath{stroke,fill}%
}%
\begin{pgfscope}%
\pgfsys@transformshift{0.687500in}{1.938300in}%
\pgfsys@useobject{currentmarker}{}%
\end{pgfscope}%
\end{pgfscope}%
\begin{pgfscope}%
\definecolor{textcolor}{rgb}{0.000000,0.000000,0.000000}%
\pgfsetstrokecolor{textcolor}%
\pgfsetfillcolor{textcolor}%
\pgftext[x=0.412808in,y=1.890075in,left,base]{\color{textcolor}\rmfamily\fontsize{10.000000}{12.000000}\selectfont \(\displaystyle 0.6\)}%
\end{pgfscope}%
\begin{pgfscope}%
\pgfsetbuttcap%
\pgfsetroundjoin%
\definecolor{currentfill}{rgb}{0.000000,0.000000,0.000000}%
\pgfsetfillcolor{currentfill}%
\pgfsetlinewidth{0.803000pt}%
\definecolor{currentstroke}{rgb}{0.000000,0.000000,0.000000}%
\pgfsetstrokecolor{currentstroke}%
\pgfsetdash{}{0pt}%
\pgfsys@defobject{currentmarker}{\pgfqpoint{-0.048611in}{0.000000in}}{\pgfqpoint{0.000000in}{0.000000in}}{%
\pgfpathmoveto{\pgfqpoint{0.000000in}{0.000000in}}%
\pgfpathlineto{\pgfqpoint{-0.048611in}{0.000000in}}%
\pgfusepath{stroke,fill}%
}%
\begin{pgfscope}%
\pgfsys@transformshift{0.687500in}{2.447900in}%
\pgfsys@useobject{currentmarker}{}%
\end{pgfscope}%
\end{pgfscope}%
\begin{pgfscope}%
\definecolor{textcolor}{rgb}{0.000000,0.000000,0.000000}%
\pgfsetstrokecolor{textcolor}%
\pgfsetfillcolor{textcolor}%
\pgftext[x=0.412808in,y=2.399675in,left,base]{\color{textcolor}\rmfamily\fontsize{10.000000}{12.000000}\selectfont \(\displaystyle 0.8\)}%
\end{pgfscope}%
\begin{pgfscope}%
\pgfsetbuttcap%
\pgfsetroundjoin%
\definecolor{currentfill}{rgb}{0.000000,0.000000,0.000000}%
\pgfsetfillcolor{currentfill}%
\pgfsetlinewidth{0.803000pt}%
\definecolor{currentstroke}{rgb}{0.000000,0.000000,0.000000}%
\pgfsetstrokecolor{currentstroke}%
\pgfsetdash{}{0pt}%
\pgfsys@defobject{currentmarker}{\pgfqpoint{-0.048611in}{0.000000in}}{\pgfqpoint{0.000000in}{0.000000in}}{%
\pgfpathmoveto{\pgfqpoint{0.000000in}{0.000000in}}%
\pgfpathlineto{\pgfqpoint{-0.048611in}{0.000000in}}%
\pgfusepath{stroke,fill}%
}%
\begin{pgfscope}%
\pgfsys@transformshift{0.687500in}{2.957500in}%
\pgfsys@useobject{currentmarker}{}%
\end{pgfscope}%
\end{pgfscope}%
\begin{pgfscope}%
\definecolor{textcolor}{rgb}{0.000000,0.000000,0.000000}%
\pgfsetstrokecolor{textcolor}%
\pgfsetfillcolor{textcolor}%
\pgftext[x=0.412808in,y=2.909275in,left,base]{\color{textcolor}\rmfamily\fontsize{10.000000}{12.000000}\selectfont \(\displaystyle 1.0\)}%
\end{pgfscope}%
\begin{pgfscope}%
\definecolor{textcolor}{rgb}{0.000000,0.000000,0.000000}%
\pgfsetstrokecolor{textcolor}%
\pgfsetfillcolor{textcolor}%
\pgftext[x=0.357253in,y=1.732500in,,bottom,rotate=90.000000]{\color{textcolor}\rmfamily\fontsize{10.000000}{12.000000}\selectfont y}%
\end{pgfscope}%
\begin{pgfscope}%
\pgfpathrectangle{\pgfqpoint{0.687500in}{0.385000in}}{\pgfqpoint{4.262500in}{2.695000in}}%
\pgfusepath{clip}%
\pgfsetbuttcap%
\pgfsetroundjoin%
\pgfsetlinewidth{1.505625pt}%
\definecolor{currentstroke}{rgb}{1.000000,0.000000,0.000000}%
\pgfsetstrokecolor{currentstroke}%
\pgfsetdash{{5.550000pt}{2.400000pt}}{0.000000pt}%
\pgfpathmoveto{\pgfqpoint{0.881250in}{0.507500in}}%
\pgfpathlineto{\pgfqpoint{1.001477in}{0.520313in}}%
\pgfpathlineto{\pgfqpoint{1.112455in}{0.534467in}}%
\pgfpathlineto{\pgfqpoint{1.204937in}{0.548397in}}%
\pgfpathlineto{\pgfqpoint{1.288171in}{0.562980in}}%
\pgfpathlineto{\pgfqpoint{1.371405in}{0.579925in}}%
\pgfpathlineto{\pgfqpoint{1.445391in}{0.597391in}}%
\pgfpathlineto{\pgfqpoint{1.510128in}{0.614906in}}%
\pgfpathlineto{\pgfqpoint{1.574866in}{0.634902in}}%
\pgfpathlineto{\pgfqpoint{1.630355in}{0.654372in}}%
\pgfpathlineto{\pgfqpoint{1.685844in}{0.676373in}}%
\pgfpathlineto{\pgfqpoint{1.732085in}{0.696952in}}%
\pgfpathlineto{\pgfqpoint{1.778326in}{0.719890in}}%
\pgfpathlineto{\pgfqpoint{1.824567in}{0.745538in}}%
\pgfpathlineto{\pgfqpoint{1.861560in}{0.768286in}}%
\pgfpathlineto{\pgfqpoint{1.898553in}{0.793281in}}%
\pgfpathlineto{\pgfqpoint{1.935546in}{0.820805in}}%
\pgfpathlineto{\pgfqpoint{1.972539in}{0.851182in}}%
\pgfpathlineto{\pgfqpoint{2.009532in}{0.884783in}}%
\pgfpathlineto{\pgfqpoint{2.046524in}{0.922030in}}%
\pgfpathlineto{\pgfqpoint{2.083517in}{0.963409in}}%
\pgfpathlineto{\pgfqpoint{2.111262in}{0.997483in}}%
\pgfpathlineto{\pgfqpoint{2.139007in}{1.034449in}}%
\pgfpathlineto{\pgfqpoint{2.166751in}{1.074589in}}%
\pgfpathlineto{\pgfqpoint{2.194496in}{1.118213in}}%
\pgfpathlineto{\pgfqpoint{2.222240in}{1.165653in}}%
\pgfpathlineto{\pgfqpoint{2.249985in}{1.217266in}}%
\pgfpathlineto{\pgfqpoint{2.277730in}{1.273427in}}%
\pgfpathlineto{\pgfqpoint{2.305474in}{1.334526in}}%
\pgfpathlineto{\pgfqpoint{2.333219in}{1.400952in}}%
\pgfpathlineto{\pgfqpoint{2.360964in}{1.473086in}}%
\pgfpathlineto{\pgfqpoint{2.388708in}{1.551270in}}%
\pgfpathlineto{\pgfqpoint{2.416453in}{1.635779in}}%
\pgfpathlineto{\pgfqpoint{2.444197in}{1.726779in}}%
\pgfpathlineto{\pgfqpoint{2.471942in}{1.824267in}}%
\pgfpathlineto{\pgfqpoint{2.508935in}{1.963882in}}%
\pgfpathlineto{\pgfqpoint{2.545928in}{2.113054in}}%
\pgfpathlineto{\pgfqpoint{2.601417in}{2.347787in}}%
\pgfpathlineto{\pgfqpoint{2.647658in}{2.541813in}}%
\pgfpathlineto{\pgfqpoint{2.675403in}{2.650787in}}%
\pgfpathlineto{\pgfqpoint{2.703147in}{2.749260in}}%
\pgfpathlineto{\pgfqpoint{2.721644in}{2.806944in}}%
\pgfpathlineto{\pgfqpoint{2.740140in}{2.856785in}}%
\pgfpathlineto{\pgfqpoint{2.758637in}{2.897622in}}%
\pgfpathlineto{\pgfqpoint{2.767885in}{2.914340in}}%
\pgfpathlineto{\pgfqpoint{2.777133in}{2.928445in}}%
\pgfpathlineto{\pgfqpoint{2.786381in}{2.939844in}}%
\pgfpathlineto{\pgfqpoint{2.795629in}{2.948461in}}%
\pgfpathlineto{\pgfqpoint{2.804878in}{2.954239in}}%
\pgfpathlineto{\pgfqpoint{2.814126in}{2.957137in}}%
\pgfpathlineto{\pgfqpoint{2.823374in}{2.957137in}}%
\pgfpathlineto{\pgfqpoint{2.832622in}{2.954239in}}%
\pgfpathlineto{\pgfqpoint{2.841871in}{2.948461in}}%
\pgfpathlineto{\pgfqpoint{2.851119in}{2.939844in}}%
\pgfpathlineto{\pgfqpoint{2.860367in}{2.928445in}}%
\pgfpathlineto{\pgfqpoint{2.869615in}{2.914340in}}%
\pgfpathlineto{\pgfqpoint{2.878863in}{2.897622in}}%
\pgfpathlineto{\pgfqpoint{2.897360in}{2.856785in}}%
\pgfpathlineto{\pgfqpoint{2.915856in}{2.806944in}}%
\pgfpathlineto{\pgfqpoint{2.934353in}{2.749260in}}%
\pgfpathlineto{\pgfqpoint{2.952849in}{2.684990in}}%
\pgfpathlineto{\pgfqpoint{2.980594in}{2.579043in}}%
\pgfpathlineto{\pgfqpoint{3.017587in}{2.426442in}}%
\pgfpathlineto{\pgfqpoint{3.110069in}{2.037419in}}%
\pgfpathlineto{\pgfqpoint{3.147061in}{1.892757in}}%
\pgfpathlineto{\pgfqpoint{3.184054in}{1.758560in}}%
\pgfpathlineto{\pgfqpoint{3.211799in}{1.665387in}}%
\pgfpathlineto{\pgfqpoint{3.239544in}{1.578726in}}%
\pgfpathlineto{\pgfqpoint{3.267288in}{1.498460in}}%
\pgfpathlineto{\pgfqpoint{3.295033in}{1.424345in}}%
\pgfpathlineto{\pgfqpoint{3.322777in}{1.356057in}}%
\pgfpathlineto{\pgfqpoint{3.350522in}{1.293225in}}%
\pgfpathlineto{\pgfqpoint{3.378267in}{1.235462in}}%
\pgfpathlineto{\pgfqpoint{3.406011in}{1.182375in}}%
\pgfpathlineto{\pgfqpoint{3.433756in}{1.133585in}}%
\pgfpathlineto{\pgfqpoint{3.461501in}{1.088727in}}%
\pgfpathlineto{\pgfqpoint{3.489245in}{1.047461in}}%
\pgfpathlineto{\pgfqpoint{3.516990in}{1.009470in}}%
\pgfpathlineto{\pgfqpoint{3.544734in}{0.974463in}}%
\pgfpathlineto{\pgfqpoint{3.581727in}{0.931968in}}%
\pgfpathlineto{\pgfqpoint{3.618720in}{0.893735in}}%
\pgfpathlineto{\pgfqpoint{3.655713in}{0.859264in}}%
\pgfpathlineto{\pgfqpoint{3.692706in}{0.828118in}}%
\pgfpathlineto{\pgfqpoint{3.729699in}{0.799913in}}%
\pgfpathlineto{\pgfqpoint{3.766692in}{0.774314in}}%
\pgfpathlineto{\pgfqpoint{3.803684in}{0.751029in}}%
\pgfpathlineto{\pgfqpoint{3.849925in}{0.724790in}}%
\pgfpathlineto{\pgfqpoint{3.896166in}{0.701341in}}%
\pgfpathlineto{\pgfqpoint{3.942408in}{0.680316in}}%
\pgfpathlineto{\pgfqpoint{3.997897in}{0.657853in}}%
\pgfpathlineto{\pgfqpoint{4.053386in}{0.637987in}}%
\pgfpathlineto{\pgfqpoint{4.108875in}{0.620347in}}%
\pgfpathlineto{\pgfqpoint{4.173613in}{0.602166in}}%
\pgfpathlineto{\pgfqpoint{4.238350in}{0.586185in}}%
\pgfpathlineto{\pgfqpoint{4.312336in}{0.570191in}}%
\pgfpathlineto{\pgfqpoint{4.395570in}{0.554615in}}%
\pgfpathlineto{\pgfqpoint{4.488052in}{0.539781in}}%
\pgfpathlineto{\pgfqpoint{4.589782in}{0.525908in}}%
\pgfpathlineto{\pgfqpoint{4.700761in}{0.513126in}}%
\pgfpathlineto{\pgfqpoint{4.756250in}{0.507500in}}%
\pgfpathlineto{\pgfqpoint{4.756250in}{0.507500in}}%
\pgfusepath{stroke}%
\end{pgfscope}%
\begin{pgfscope}%
\pgfpathrectangle{\pgfqpoint{0.687500in}{0.385000in}}{\pgfqpoint{4.262500in}{2.695000in}}%
\pgfusepath{clip}%
\pgfsetrectcap%
\pgfsetroundjoin%
\pgfsetlinewidth{1.505625pt}%
\definecolor{currentstroke}{rgb}{0.121569,0.466667,0.705882}%
\pgfsetstrokecolor{currentstroke}%
\pgfsetdash{}{0pt}%
\pgfpathmoveto{\pgfqpoint{0.881250in}{0.507500in}}%
\pgfpathlineto{\pgfqpoint{0.890794in}{0.508532in}}%
\pgfpathlineto{\pgfqpoint{0.900339in}{0.509566in}}%
\pgfpathlineto{\pgfqpoint{0.909883in}{0.510601in}}%
\pgfpathlineto{\pgfqpoint{0.919427in}{0.511638in}}%
\pgfpathlineto{\pgfqpoint{0.928972in}{0.512680in}}%
\pgfpathlineto{\pgfqpoint{0.938516in}{0.513725in}}%
\pgfpathlineto{\pgfqpoint{0.948060in}{0.514776in}}%
\pgfpathlineto{\pgfqpoint{0.957605in}{0.515834in}}%
\pgfpathlineto{\pgfqpoint{0.967149in}{0.516898in}}%
\pgfpathlineto{\pgfqpoint{0.976693in}{0.517971in}}%
\pgfpathlineto{\pgfqpoint{0.986238in}{0.519052in}}%
\pgfpathlineto{\pgfqpoint{0.995782in}{0.520143in}}%
\pgfpathlineto{\pgfqpoint{1.005326in}{0.521245in}}%
\pgfpathlineto{\pgfqpoint{1.014871in}{0.522359in}}%
\pgfpathlineto{\pgfqpoint{1.024415in}{0.523485in}}%
\pgfpathlineto{\pgfqpoint{1.033959in}{0.524624in}}%
\pgfpathlineto{\pgfqpoint{1.043504in}{0.525778in}}%
\pgfpathlineto{\pgfqpoint{1.053048in}{0.526947in}}%
\pgfpathlineto{\pgfqpoint{1.062592in}{0.528132in}}%
\pgfpathlineto{\pgfqpoint{1.072137in}{0.529334in}}%
\pgfpathlineto{\pgfqpoint{1.081681in}{0.530554in}}%
\pgfpathlineto{\pgfqpoint{1.091225in}{0.531792in}}%
\pgfpathlineto{\pgfqpoint{1.100770in}{0.533050in}}%
\pgfpathlineto{\pgfqpoint{1.110314in}{0.534329in}}%
\pgfpathlineto{\pgfqpoint{1.119858in}{0.535629in}}%
\pgfpathlineto{\pgfqpoint{1.129403in}{0.536952in}}%
\pgfpathlineto{\pgfqpoint{1.138947in}{0.538297in}}%
\pgfpathlineto{\pgfqpoint{1.148491in}{0.539667in}}%
\pgfpathlineto{\pgfqpoint{1.158036in}{0.541062in}}%
\pgfusepath{stroke}%
\end{pgfscope}%
\begin{pgfscope}%
\pgfpathrectangle{\pgfqpoint{0.687500in}{0.385000in}}{\pgfqpoint{4.262500in}{2.695000in}}%
\pgfusepath{clip}%
\pgfsetrectcap%
\pgfsetroundjoin%
\pgfsetlinewidth{1.505625pt}%
\definecolor{currentstroke}{rgb}{1.000000,0.498039,0.054902}%
\pgfsetstrokecolor{currentstroke}%
\pgfsetdash{}{0pt}%
\pgfpathmoveto{\pgfqpoint{1.158036in}{0.541062in}}%
\pgfpathlineto{\pgfqpoint{1.167580in}{0.542482in}}%
\pgfpathlineto{\pgfqpoint{1.177124in}{0.543929in}}%
\pgfpathlineto{\pgfqpoint{1.186669in}{0.545403in}}%
\pgfpathlineto{\pgfqpoint{1.196213in}{0.546904in}}%
\pgfpathlineto{\pgfqpoint{1.205757in}{0.548433in}}%
\pgfpathlineto{\pgfqpoint{1.215302in}{0.549990in}}%
\pgfpathlineto{\pgfqpoint{1.224846in}{0.551576in}}%
\pgfpathlineto{\pgfqpoint{1.234390in}{0.553192in}}%
\pgfpathlineto{\pgfqpoint{1.243935in}{0.554837in}}%
\pgfpathlineto{\pgfqpoint{1.253479in}{0.556512in}}%
\pgfpathlineto{\pgfqpoint{1.263023in}{0.558218in}}%
\pgfpathlineto{\pgfqpoint{1.272568in}{0.559956in}}%
\pgfpathlineto{\pgfqpoint{1.282112in}{0.561725in}}%
\pgfpathlineto{\pgfqpoint{1.291656in}{0.563526in}}%
\pgfpathlineto{\pgfqpoint{1.301201in}{0.565360in}}%
\pgfpathlineto{\pgfqpoint{1.310745in}{0.567227in}}%
\pgfpathlineto{\pgfqpoint{1.320289in}{0.569128in}}%
\pgfpathlineto{\pgfqpoint{1.329834in}{0.571063in}}%
\pgfpathlineto{\pgfqpoint{1.339378in}{0.573032in}}%
\pgfpathlineto{\pgfqpoint{1.348922in}{0.575037in}}%
\pgfpathlineto{\pgfqpoint{1.358467in}{0.577077in}}%
\pgfpathlineto{\pgfqpoint{1.368011in}{0.579154in}}%
\pgfpathlineto{\pgfqpoint{1.377555in}{0.581267in}}%
\pgfpathlineto{\pgfqpoint{1.387100in}{0.583417in}}%
\pgfpathlineto{\pgfqpoint{1.396644in}{0.585605in}}%
\pgfpathlineto{\pgfqpoint{1.406188in}{0.587830in}}%
\pgfpathlineto{\pgfqpoint{1.415733in}{0.590094in}}%
\pgfpathlineto{\pgfqpoint{1.425277in}{0.592398in}}%
\pgfpathlineto{\pgfqpoint{1.434821in}{0.594740in}}%
\pgfusepath{stroke}%
\end{pgfscope}%
\begin{pgfscope}%
\pgfpathrectangle{\pgfqpoint{0.687500in}{0.385000in}}{\pgfqpoint{4.262500in}{2.695000in}}%
\pgfusepath{clip}%
\pgfsetrectcap%
\pgfsetroundjoin%
\pgfsetlinewidth{1.505625pt}%
\definecolor{currentstroke}{rgb}{0.172549,0.627451,0.172549}%
\pgfsetstrokecolor{currentstroke}%
\pgfsetdash{}{0pt}%
\pgfpathmoveto{\pgfqpoint{1.434821in}{0.594740in}}%
\pgfpathlineto{\pgfqpoint{1.444366in}{0.597123in}}%
\pgfpathlineto{\pgfqpoint{1.453910in}{0.599548in}}%
\pgfpathlineto{\pgfqpoint{1.463454in}{0.602016in}}%
\pgfpathlineto{\pgfqpoint{1.472999in}{0.604530in}}%
\pgfpathlineto{\pgfqpoint{1.482543in}{0.607092in}}%
\pgfpathlineto{\pgfqpoint{1.492087in}{0.609702in}}%
\pgfpathlineto{\pgfqpoint{1.501632in}{0.612364in}}%
\pgfpathlineto{\pgfqpoint{1.511176in}{0.615078in}}%
\pgfpathlineto{\pgfqpoint{1.520720in}{0.617847in}}%
\pgfpathlineto{\pgfqpoint{1.530265in}{0.620673in}}%
\pgfpathlineto{\pgfqpoint{1.539809in}{0.623557in}}%
\pgfpathlineto{\pgfqpoint{1.549353in}{0.626502in}}%
\pgfpathlineto{\pgfqpoint{1.558898in}{0.629508in}}%
\pgfpathlineto{\pgfqpoint{1.568442in}{0.632579in}}%
\pgfpathlineto{\pgfqpoint{1.577986in}{0.635715in}}%
\pgfpathlineto{\pgfqpoint{1.587531in}{0.638919in}}%
\pgfpathlineto{\pgfqpoint{1.597075in}{0.642193in}}%
\pgfpathlineto{\pgfqpoint{1.606619in}{0.645538in}}%
\pgfpathlineto{\pgfqpoint{1.616164in}{0.648956in}}%
\pgfpathlineto{\pgfqpoint{1.625708in}{0.652449in}}%
\pgfpathlineto{\pgfqpoint{1.635252in}{0.656019in}}%
\pgfpathlineto{\pgfqpoint{1.644797in}{0.659667in}}%
\pgfpathlineto{\pgfqpoint{1.654341in}{0.663397in}}%
\pgfpathlineto{\pgfqpoint{1.663885in}{0.667208in}}%
\pgfpathlineto{\pgfqpoint{1.673430in}{0.671104in}}%
\pgfpathlineto{\pgfqpoint{1.682974in}{0.675086in}}%
\pgfpathlineto{\pgfqpoint{1.692518in}{0.679156in}}%
\pgfpathlineto{\pgfqpoint{1.702063in}{0.683316in}}%
\pgfpathlineto{\pgfqpoint{1.711607in}{0.687567in}}%
\pgfusepath{stroke}%
\end{pgfscope}%
\begin{pgfscope}%
\pgfpathrectangle{\pgfqpoint{0.687500in}{0.385000in}}{\pgfqpoint{4.262500in}{2.695000in}}%
\pgfusepath{clip}%
\pgfsetrectcap%
\pgfsetroundjoin%
\pgfsetlinewidth{1.505625pt}%
\definecolor{currentstroke}{rgb}{0.839216,0.152941,0.156863}%
\pgfsetstrokecolor{currentstroke}%
\pgfsetdash{}{0pt}%
\pgfpathmoveto{\pgfqpoint{1.711607in}{0.687567in}}%
\pgfpathlineto{\pgfqpoint{1.721151in}{0.691912in}}%
\pgfpathlineto{\pgfqpoint{1.730696in}{0.696354in}}%
\pgfpathlineto{\pgfqpoint{1.740240in}{0.700896in}}%
\pgfpathlineto{\pgfqpoint{1.749784in}{0.705541in}}%
\pgfpathlineto{\pgfqpoint{1.759329in}{0.710294in}}%
\pgfpathlineto{\pgfqpoint{1.768873in}{0.715157in}}%
\pgfpathlineto{\pgfqpoint{1.778417in}{0.720133in}}%
\pgfpathlineto{\pgfqpoint{1.787962in}{0.725226in}}%
\pgfpathlineto{\pgfqpoint{1.797506in}{0.730440in}}%
\pgfpathlineto{\pgfqpoint{1.807050in}{0.735778in}}%
\pgfpathlineto{\pgfqpoint{1.816595in}{0.741242in}}%
\pgfpathlineto{\pgfqpoint{1.826139in}{0.746837in}}%
\pgfpathlineto{\pgfqpoint{1.835683in}{0.752566in}}%
\pgfpathlineto{\pgfqpoint{1.845228in}{0.758432in}}%
\pgfpathlineto{\pgfqpoint{1.854772in}{0.764438in}}%
\pgfpathlineto{\pgfqpoint{1.864317in}{0.770588in}}%
\pgfpathlineto{\pgfqpoint{1.873861in}{0.776886in}}%
\pgfpathlineto{\pgfqpoint{1.883405in}{0.783334in}}%
\pgfpathlineto{\pgfqpoint{1.892950in}{0.789936in}}%
\pgfpathlineto{\pgfqpoint{1.902494in}{0.796696in}}%
\pgfpathlineto{\pgfqpoint{1.912038in}{0.803616in}}%
\pgfpathlineto{\pgfqpoint{1.921583in}{0.810700in}}%
\pgfpathlineto{\pgfqpoint{1.931127in}{0.817951in}}%
\pgfpathlineto{\pgfqpoint{1.940671in}{0.825374in}}%
\pgfpathlineto{\pgfqpoint{1.950216in}{0.832970in}}%
\pgfpathlineto{\pgfqpoint{1.959760in}{0.840744in}}%
\pgfpathlineto{\pgfqpoint{1.969304in}{0.848699in}}%
\pgfpathlineto{\pgfqpoint{1.978849in}{0.856838in}}%
\pgfpathlineto{\pgfqpoint{1.988393in}{0.865164in}}%
\pgfusepath{stroke}%
\end{pgfscope}%
\begin{pgfscope}%
\pgfpathrectangle{\pgfqpoint{0.687500in}{0.385000in}}{\pgfqpoint{4.262500in}{2.695000in}}%
\pgfusepath{clip}%
\pgfsetrectcap%
\pgfsetroundjoin%
\pgfsetlinewidth{1.505625pt}%
\definecolor{currentstroke}{rgb}{0.580392,0.403922,0.741176}%
\pgfsetstrokecolor{currentstroke}%
\pgfsetdash{}{0pt}%
\pgfpathmoveto{\pgfqpoint{1.988393in}{0.865164in}}%
\pgfpathlineto{\pgfqpoint{1.997937in}{0.873683in}}%
\pgfpathlineto{\pgfqpoint{2.007482in}{0.882408in}}%
\pgfpathlineto{\pgfqpoint{2.017026in}{0.891352in}}%
\pgfpathlineto{\pgfqpoint{2.026570in}{0.900529in}}%
\pgfpathlineto{\pgfqpoint{2.036115in}{0.909954in}}%
\pgfpathlineto{\pgfqpoint{2.045659in}{0.919641in}}%
\pgfpathlineto{\pgfqpoint{2.055203in}{0.929604in}}%
\pgfpathlineto{\pgfqpoint{2.064748in}{0.939858in}}%
\pgfpathlineto{\pgfqpoint{2.074292in}{0.950416in}}%
\pgfpathlineto{\pgfqpoint{2.083836in}{0.961293in}}%
\pgfpathlineto{\pgfqpoint{2.093381in}{0.972503in}}%
\pgfpathlineto{\pgfqpoint{2.102925in}{0.984060in}}%
\pgfpathlineto{\pgfqpoint{2.112469in}{0.995979in}}%
\pgfpathlineto{\pgfqpoint{2.122014in}{1.008274in}}%
\pgfpathlineto{\pgfqpoint{2.131558in}{1.020958in}}%
\pgfpathlineto{\pgfqpoint{2.141102in}{1.034046in}}%
\pgfpathlineto{\pgfqpoint{2.150647in}{1.047553in}}%
\pgfpathlineto{\pgfqpoint{2.160191in}{1.061492in}}%
\pgfpathlineto{\pgfqpoint{2.169735in}{1.075878in}}%
\pgfpathlineto{\pgfqpoint{2.179280in}{1.090725in}}%
\pgfpathlineto{\pgfqpoint{2.188824in}{1.106047in}}%
\pgfpathlineto{\pgfqpoint{2.198368in}{1.121858in}}%
\pgfpathlineto{\pgfqpoint{2.207913in}{1.138173in}}%
\pgfpathlineto{\pgfqpoint{2.217457in}{1.155006in}}%
\pgfpathlineto{\pgfqpoint{2.227001in}{1.172370in}}%
\pgfpathlineto{\pgfqpoint{2.236546in}{1.190281in}}%
\pgfpathlineto{\pgfqpoint{2.246090in}{1.208752in}}%
\pgfpathlineto{\pgfqpoint{2.255634in}{1.227798in}}%
\pgfpathlineto{\pgfqpoint{2.265179in}{1.247433in}}%
\pgfusepath{stroke}%
\end{pgfscope}%
\begin{pgfscope}%
\pgfpathrectangle{\pgfqpoint{0.687500in}{0.385000in}}{\pgfqpoint{4.262500in}{2.695000in}}%
\pgfusepath{clip}%
\pgfsetrectcap%
\pgfsetroundjoin%
\pgfsetlinewidth{1.505625pt}%
\definecolor{currentstroke}{rgb}{0.549020,0.337255,0.294118}%
\pgfsetstrokecolor{currentstroke}%
\pgfsetdash{}{0pt}%
\pgfpathmoveto{\pgfqpoint{2.265179in}{1.247433in}}%
\pgfpathlineto{\pgfqpoint{2.274723in}{1.267669in}}%
\pgfpathlineto{\pgfqpoint{2.284267in}{1.288512in}}%
\pgfpathlineto{\pgfqpoint{2.293812in}{1.309968in}}%
\pgfpathlineto{\pgfqpoint{2.303356in}{1.332039in}}%
\pgfpathlineto{\pgfqpoint{2.312900in}{1.354730in}}%
\pgfpathlineto{\pgfqpoint{2.322445in}{1.378046in}}%
\pgfpathlineto{\pgfqpoint{2.331989in}{1.401992in}}%
\pgfpathlineto{\pgfqpoint{2.341533in}{1.426570in}}%
\pgfpathlineto{\pgfqpoint{2.351078in}{1.451786in}}%
\pgfpathlineto{\pgfqpoint{2.360622in}{1.477645in}}%
\pgfpathlineto{\pgfqpoint{2.370166in}{1.504149in}}%
\pgfpathlineto{\pgfqpoint{2.379711in}{1.531304in}}%
\pgfpathlineto{\pgfqpoint{2.389255in}{1.559115in}}%
\pgfpathlineto{\pgfqpoint{2.398799in}{1.587584in}}%
\pgfpathlineto{\pgfqpoint{2.408344in}{1.616717in}}%
\pgfpathlineto{\pgfqpoint{2.417888in}{1.646518in}}%
\pgfpathlineto{\pgfqpoint{2.427432in}{1.676992in}}%
\pgfpathlineto{\pgfqpoint{2.436977in}{1.708142in}}%
\pgfpathlineto{\pgfqpoint{2.446521in}{1.739973in}}%
\pgfpathlineto{\pgfqpoint{2.456065in}{1.772489in}}%
\pgfpathlineto{\pgfqpoint{2.465610in}{1.805695in}}%
\pgfpathlineto{\pgfqpoint{2.475154in}{1.839595in}}%
\pgfpathlineto{\pgfqpoint{2.484698in}{1.874193in}}%
\pgfpathlineto{\pgfqpoint{2.494243in}{1.909494in}}%
\pgfpathlineto{\pgfqpoint{2.503787in}{1.945502in}}%
\pgfpathlineto{\pgfqpoint{2.513331in}{1.982221in}}%
\pgfpathlineto{\pgfqpoint{2.522876in}{2.019656in}}%
\pgfpathlineto{\pgfqpoint{2.532420in}{2.057810in}}%
\pgfpathlineto{\pgfqpoint{2.541964in}{2.096689in}}%
\pgfusepath{stroke}%
\end{pgfscope}%
\begin{pgfscope}%
\pgfpathrectangle{\pgfqpoint{0.687500in}{0.385000in}}{\pgfqpoint{4.262500in}{2.695000in}}%
\pgfusepath{clip}%
\pgfsetrectcap%
\pgfsetroundjoin%
\pgfsetlinewidth{1.505625pt}%
\definecolor{currentstroke}{rgb}{0.890196,0.466667,0.760784}%
\pgfsetstrokecolor{currentstroke}%
\pgfsetdash{}{0pt}%
\pgfpathmoveto{\pgfqpoint{2.541964in}{2.096689in}}%
\pgfpathlineto{\pgfqpoint{2.551509in}{2.136272in}}%
\pgfpathlineto{\pgfqpoint{2.561053in}{2.176440in}}%
\pgfpathlineto{\pgfqpoint{2.570597in}{2.217049in}}%
\pgfpathlineto{\pgfqpoint{2.580142in}{2.257956in}}%
\pgfpathlineto{\pgfqpoint{2.589686in}{2.299018in}}%
\pgfpathlineto{\pgfqpoint{2.599230in}{2.340090in}}%
\pgfpathlineto{\pgfqpoint{2.608775in}{2.381029in}}%
\pgfpathlineto{\pgfqpoint{2.618319in}{2.421692in}}%
\pgfpathlineto{\pgfqpoint{2.627863in}{2.461935in}}%
\pgfpathlineto{\pgfqpoint{2.637408in}{2.501614in}}%
\pgfpathlineto{\pgfqpoint{2.646952in}{2.540587in}}%
\pgfpathlineto{\pgfqpoint{2.656496in}{2.578708in}}%
\pgfpathlineto{\pgfqpoint{2.666041in}{2.615836in}}%
\pgfpathlineto{\pgfqpoint{2.675585in}{2.651825in}}%
\pgfpathlineto{\pgfqpoint{2.685129in}{2.686533in}}%
\pgfpathlineto{\pgfqpoint{2.694674in}{2.719816in}}%
\pgfpathlineto{\pgfqpoint{2.704218in}{2.751531in}}%
\pgfpathlineto{\pgfqpoint{2.713762in}{2.781533in}}%
\pgfpathlineto{\pgfqpoint{2.723307in}{2.809680in}}%
\pgfpathlineto{\pgfqpoint{2.732851in}{2.835828in}}%
\pgfpathlineto{\pgfqpoint{2.742395in}{2.859833in}}%
\pgfpathlineto{\pgfqpoint{2.751940in}{2.881551in}}%
\pgfpathlineto{\pgfqpoint{2.761484in}{2.900839in}}%
\pgfpathlineto{\pgfqpoint{2.771028in}{2.917554in}}%
\pgfpathlineto{\pgfqpoint{2.780573in}{2.931552in}}%
\pgfpathlineto{\pgfqpoint{2.790117in}{2.942689in}}%
\pgfpathlineto{\pgfqpoint{2.799661in}{2.950821in}}%
\pgfpathlineto{\pgfqpoint{2.809206in}{2.955806in}}%
\pgfpathlineto{\pgfqpoint{2.818750in}{2.957500in}}%
\pgfusepath{stroke}%
\end{pgfscope}%
\begin{pgfscope}%
\pgfpathrectangle{\pgfqpoint{0.687500in}{0.385000in}}{\pgfqpoint{4.262500in}{2.695000in}}%
\pgfusepath{clip}%
\pgfsetrectcap%
\pgfsetroundjoin%
\pgfsetlinewidth{1.505625pt}%
\definecolor{currentstroke}{rgb}{0.498039,0.498039,0.498039}%
\pgfsetstrokecolor{currentstroke}%
\pgfsetdash{}{0pt}%
\pgfpathmoveto{\pgfqpoint{2.818750in}{2.957500in}}%
\pgfpathlineto{\pgfqpoint{2.828294in}{2.955806in}}%
\pgfpathlineto{\pgfqpoint{2.837839in}{2.950821in}}%
\pgfpathlineto{\pgfqpoint{2.847383in}{2.942689in}}%
\pgfpathlineto{\pgfqpoint{2.856927in}{2.931552in}}%
\pgfpathlineto{\pgfqpoint{2.866472in}{2.917554in}}%
\pgfpathlineto{\pgfqpoint{2.876016in}{2.900839in}}%
\pgfpathlineto{\pgfqpoint{2.885560in}{2.881551in}}%
\pgfpathlineto{\pgfqpoint{2.895105in}{2.859833in}}%
\pgfpathlineto{\pgfqpoint{2.904649in}{2.835828in}}%
\pgfpathlineto{\pgfqpoint{2.914193in}{2.809680in}}%
\pgfpathlineto{\pgfqpoint{2.923738in}{2.781533in}}%
\pgfpathlineto{\pgfqpoint{2.933282in}{2.751531in}}%
\pgfpathlineto{\pgfqpoint{2.942826in}{2.719816in}}%
\pgfpathlineto{\pgfqpoint{2.952371in}{2.686533in}}%
\pgfpathlineto{\pgfqpoint{2.961915in}{2.651825in}}%
\pgfpathlineto{\pgfqpoint{2.971459in}{2.615836in}}%
\pgfpathlineto{\pgfqpoint{2.981004in}{2.578708in}}%
\pgfpathlineto{\pgfqpoint{2.990548in}{2.540587in}}%
\pgfpathlineto{\pgfqpoint{3.000092in}{2.501614in}}%
\pgfpathlineto{\pgfqpoint{3.009637in}{2.461935in}}%
\pgfpathlineto{\pgfqpoint{3.019181in}{2.421692in}}%
\pgfpathlineto{\pgfqpoint{3.028725in}{2.381029in}}%
\pgfpathlineto{\pgfqpoint{3.038270in}{2.340090in}}%
\pgfpathlineto{\pgfqpoint{3.047814in}{2.299018in}}%
\pgfpathlineto{\pgfqpoint{3.057358in}{2.257956in}}%
\pgfpathlineto{\pgfqpoint{3.066903in}{2.217049in}}%
\pgfpathlineto{\pgfqpoint{3.076447in}{2.176440in}}%
\pgfpathlineto{\pgfqpoint{3.085991in}{2.136272in}}%
\pgfpathlineto{\pgfqpoint{3.095536in}{2.096689in}}%
\pgfusepath{stroke}%
\end{pgfscope}%
\begin{pgfscope}%
\pgfpathrectangle{\pgfqpoint{0.687500in}{0.385000in}}{\pgfqpoint{4.262500in}{2.695000in}}%
\pgfusepath{clip}%
\pgfsetrectcap%
\pgfsetroundjoin%
\pgfsetlinewidth{1.505625pt}%
\definecolor{currentstroke}{rgb}{0.737255,0.741176,0.133333}%
\pgfsetstrokecolor{currentstroke}%
\pgfsetdash{}{0pt}%
\pgfpathmoveto{\pgfqpoint{3.095536in}{2.096689in}}%
\pgfpathlineto{\pgfqpoint{3.105080in}{2.057810in}}%
\pgfpathlineto{\pgfqpoint{3.114624in}{2.019656in}}%
\pgfpathlineto{\pgfqpoint{3.124169in}{1.982221in}}%
\pgfpathlineto{\pgfqpoint{3.133713in}{1.945502in}}%
\pgfpathlineto{\pgfqpoint{3.143257in}{1.909494in}}%
\pgfpathlineto{\pgfqpoint{3.152802in}{1.874193in}}%
\pgfpathlineto{\pgfqpoint{3.162346in}{1.839595in}}%
\pgfpathlineto{\pgfqpoint{3.171890in}{1.805695in}}%
\pgfpathlineto{\pgfqpoint{3.181435in}{1.772489in}}%
\pgfpathlineto{\pgfqpoint{3.190979in}{1.739973in}}%
\pgfpathlineto{\pgfqpoint{3.200523in}{1.708142in}}%
\pgfpathlineto{\pgfqpoint{3.210068in}{1.676992in}}%
\pgfpathlineto{\pgfqpoint{3.219612in}{1.646518in}}%
\pgfpathlineto{\pgfqpoint{3.229156in}{1.616717in}}%
\pgfpathlineto{\pgfqpoint{3.238701in}{1.587584in}}%
\pgfpathlineto{\pgfqpoint{3.248245in}{1.559115in}}%
\pgfpathlineto{\pgfqpoint{3.257789in}{1.531304in}}%
\pgfpathlineto{\pgfqpoint{3.267334in}{1.504149in}}%
\pgfpathlineto{\pgfqpoint{3.276878in}{1.477645in}}%
\pgfpathlineto{\pgfqpoint{3.286422in}{1.451786in}}%
\pgfpathlineto{\pgfqpoint{3.295967in}{1.426570in}}%
\pgfpathlineto{\pgfqpoint{3.305511in}{1.401992in}}%
\pgfpathlineto{\pgfqpoint{3.315055in}{1.378046in}}%
\pgfpathlineto{\pgfqpoint{3.324600in}{1.354730in}}%
\pgfpathlineto{\pgfqpoint{3.334144in}{1.332039in}}%
\pgfpathlineto{\pgfqpoint{3.343688in}{1.309968in}}%
\pgfpathlineto{\pgfqpoint{3.353233in}{1.288512in}}%
\pgfpathlineto{\pgfqpoint{3.362777in}{1.267669in}}%
\pgfpathlineto{\pgfqpoint{3.372321in}{1.247433in}}%
\pgfusepath{stroke}%
\end{pgfscope}%
\begin{pgfscope}%
\pgfpathrectangle{\pgfqpoint{0.687500in}{0.385000in}}{\pgfqpoint{4.262500in}{2.695000in}}%
\pgfusepath{clip}%
\pgfsetrectcap%
\pgfsetroundjoin%
\pgfsetlinewidth{1.505625pt}%
\definecolor{currentstroke}{rgb}{0.090196,0.745098,0.811765}%
\pgfsetstrokecolor{currentstroke}%
\pgfsetdash{}{0pt}%
\pgfpathmoveto{\pgfqpoint{3.372321in}{1.247433in}}%
\pgfpathlineto{\pgfqpoint{3.381866in}{1.227798in}}%
\pgfpathlineto{\pgfqpoint{3.391410in}{1.208752in}}%
\pgfpathlineto{\pgfqpoint{3.400954in}{1.190281in}}%
\pgfpathlineto{\pgfqpoint{3.410499in}{1.172370in}}%
\pgfpathlineto{\pgfqpoint{3.420043in}{1.155006in}}%
\pgfpathlineto{\pgfqpoint{3.429587in}{1.138173in}}%
\pgfpathlineto{\pgfqpoint{3.439132in}{1.121858in}}%
\pgfpathlineto{\pgfqpoint{3.448676in}{1.106047in}}%
\pgfpathlineto{\pgfqpoint{3.458220in}{1.090725in}}%
\pgfpathlineto{\pgfqpoint{3.467765in}{1.075878in}}%
\pgfpathlineto{\pgfqpoint{3.477309in}{1.061492in}}%
\pgfpathlineto{\pgfqpoint{3.486853in}{1.047553in}}%
\pgfpathlineto{\pgfqpoint{3.496398in}{1.034046in}}%
\pgfpathlineto{\pgfqpoint{3.505942in}{1.020958in}}%
\pgfpathlineto{\pgfqpoint{3.515486in}{1.008274in}}%
\pgfpathlineto{\pgfqpoint{3.525031in}{0.995979in}}%
\pgfpathlineto{\pgfqpoint{3.534575in}{0.984060in}}%
\pgfpathlineto{\pgfqpoint{3.544119in}{0.972503in}}%
\pgfpathlineto{\pgfqpoint{3.553664in}{0.961293in}}%
\pgfpathlineto{\pgfqpoint{3.563208in}{0.950416in}}%
\pgfpathlineto{\pgfqpoint{3.572752in}{0.939858in}}%
\pgfpathlineto{\pgfqpoint{3.582297in}{0.929604in}}%
\pgfpathlineto{\pgfqpoint{3.591841in}{0.919641in}}%
\pgfpathlineto{\pgfqpoint{3.601385in}{0.909954in}}%
\pgfpathlineto{\pgfqpoint{3.610930in}{0.900529in}}%
\pgfpathlineto{\pgfqpoint{3.620474in}{0.891352in}}%
\pgfpathlineto{\pgfqpoint{3.630018in}{0.882408in}}%
\pgfpathlineto{\pgfqpoint{3.639563in}{0.873683in}}%
\pgfpathlineto{\pgfqpoint{3.649107in}{0.865164in}}%
\pgfusepath{stroke}%
\end{pgfscope}%
\begin{pgfscope}%
\pgfpathrectangle{\pgfqpoint{0.687500in}{0.385000in}}{\pgfqpoint{4.262500in}{2.695000in}}%
\pgfusepath{clip}%
\pgfsetrectcap%
\pgfsetroundjoin%
\pgfsetlinewidth{1.505625pt}%
\definecolor{currentstroke}{rgb}{0.121569,0.466667,0.705882}%
\pgfsetstrokecolor{currentstroke}%
\pgfsetdash{}{0pt}%
\pgfpathmoveto{\pgfqpoint{3.649107in}{0.865164in}}%
\pgfpathlineto{\pgfqpoint{3.658651in}{0.856838in}}%
\pgfpathlineto{\pgfqpoint{3.668196in}{0.848699in}}%
\pgfpathlineto{\pgfqpoint{3.677740in}{0.840744in}}%
\pgfpathlineto{\pgfqpoint{3.687284in}{0.832970in}}%
\pgfpathlineto{\pgfqpoint{3.696829in}{0.825374in}}%
\pgfpathlineto{\pgfqpoint{3.706373in}{0.817951in}}%
\pgfpathlineto{\pgfqpoint{3.715917in}{0.810700in}}%
\pgfpathlineto{\pgfqpoint{3.725462in}{0.803616in}}%
\pgfpathlineto{\pgfqpoint{3.735006in}{0.796696in}}%
\pgfpathlineto{\pgfqpoint{3.744550in}{0.789936in}}%
\pgfpathlineto{\pgfqpoint{3.754095in}{0.783334in}}%
\pgfpathlineto{\pgfqpoint{3.763639in}{0.776886in}}%
\pgfpathlineto{\pgfqpoint{3.773183in}{0.770588in}}%
\pgfpathlineto{\pgfqpoint{3.782728in}{0.764438in}}%
\pgfpathlineto{\pgfqpoint{3.792272in}{0.758432in}}%
\pgfpathlineto{\pgfqpoint{3.801817in}{0.752566in}}%
\pgfpathlineto{\pgfqpoint{3.811361in}{0.746837in}}%
\pgfpathlineto{\pgfqpoint{3.820905in}{0.741242in}}%
\pgfpathlineto{\pgfqpoint{3.830450in}{0.735778in}}%
\pgfpathlineto{\pgfqpoint{3.839994in}{0.730440in}}%
\pgfpathlineto{\pgfqpoint{3.849538in}{0.725226in}}%
\pgfpathlineto{\pgfqpoint{3.859083in}{0.720133in}}%
\pgfpathlineto{\pgfqpoint{3.868627in}{0.715157in}}%
\pgfpathlineto{\pgfqpoint{3.878171in}{0.710294in}}%
\pgfpathlineto{\pgfqpoint{3.887716in}{0.705541in}}%
\pgfpathlineto{\pgfqpoint{3.897260in}{0.700896in}}%
\pgfpathlineto{\pgfqpoint{3.906804in}{0.696354in}}%
\pgfpathlineto{\pgfqpoint{3.916349in}{0.691912in}}%
\pgfpathlineto{\pgfqpoint{3.925893in}{0.687567in}}%
\pgfusepath{stroke}%
\end{pgfscope}%
\begin{pgfscope}%
\pgfpathrectangle{\pgfqpoint{0.687500in}{0.385000in}}{\pgfqpoint{4.262500in}{2.695000in}}%
\pgfusepath{clip}%
\pgfsetrectcap%
\pgfsetroundjoin%
\pgfsetlinewidth{1.505625pt}%
\definecolor{currentstroke}{rgb}{1.000000,0.498039,0.054902}%
\pgfsetstrokecolor{currentstroke}%
\pgfsetdash{}{0pt}%
\pgfpathmoveto{\pgfqpoint{3.925893in}{0.687567in}}%
\pgfpathlineto{\pgfqpoint{3.935437in}{0.683316in}}%
\pgfpathlineto{\pgfqpoint{3.944982in}{0.679156in}}%
\pgfpathlineto{\pgfqpoint{3.954526in}{0.675086in}}%
\pgfpathlineto{\pgfqpoint{3.964070in}{0.671104in}}%
\pgfpathlineto{\pgfqpoint{3.973615in}{0.667208in}}%
\pgfpathlineto{\pgfqpoint{3.983159in}{0.663397in}}%
\pgfpathlineto{\pgfqpoint{3.992703in}{0.659667in}}%
\pgfpathlineto{\pgfqpoint{4.002248in}{0.656019in}}%
\pgfpathlineto{\pgfqpoint{4.011792in}{0.652449in}}%
\pgfpathlineto{\pgfqpoint{4.021336in}{0.648956in}}%
\pgfpathlineto{\pgfqpoint{4.030881in}{0.645538in}}%
\pgfpathlineto{\pgfqpoint{4.040425in}{0.642193in}}%
\pgfpathlineto{\pgfqpoint{4.049969in}{0.638919in}}%
\pgfpathlineto{\pgfqpoint{4.059514in}{0.635715in}}%
\pgfpathlineto{\pgfqpoint{4.069058in}{0.632579in}}%
\pgfpathlineto{\pgfqpoint{4.078602in}{0.629508in}}%
\pgfpathlineto{\pgfqpoint{4.088147in}{0.626502in}}%
\pgfpathlineto{\pgfqpoint{4.097691in}{0.623557in}}%
\pgfpathlineto{\pgfqpoint{4.107235in}{0.620673in}}%
\pgfpathlineto{\pgfqpoint{4.116780in}{0.617847in}}%
\pgfpathlineto{\pgfqpoint{4.126324in}{0.615078in}}%
\pgfpathlineto{\pgfqpoint{4.135868in}{0.612364in}}%
\pgfpathlineto{\pgfqpoint{4.145413in}{0.609702in}}%
\pgfpathlineto{\pgfqpoint{4.154957in}{0.607092in}}%
\pgfpathlineto{\pgfqpoint{4.164501in}{0.604530in}}%
\pgfpathlineto{\pgfqpoint{4.174046in}{0.602016in}}%
\pgfpathlineto{\pgfqpoint{4.183590in}{0.599548in}}%
\pgfpathlineto{\pgfqpoint{4.193134in}{0.597123in}}%
\pgfpathlineto{\pgfqpoint{4.202679in}{0.594740in}}%
\pgfusepath{stroke}%
\end{pgfscope}%
\begin{pgfscope}%
\pgfpathrectangle{\pgfqpoint{0.687500in}{0.385000in}}{\pgfqpoint{4.262500in}{2.695000in}}%
\pgfusepath{clip}%
\pgfsetrectcap%
\pgfsetroundjoin%
\pgfsetlinewidth{1.505625pt}%
\definecolor{currentstroke}{rgb}{0.172549,0.627451,0.172549}%
\pgfsetstrokecolor{currentstroke}%
\pgfsetdash{}{0pt}%
\pgfpathmoveto{\pgfqpoint{4.202679in}{0.594740in}}%
\pgfpathlineto{\pgfqpoint{4.212223in}{0.592398in}}%
\pgfpathlineto{\pgfqpoint{4.221767in}{0.590094in}}%
\pgfpathlineto{\pgfqpoint{4.231312in}{0.587830in}}%
\pgfpathlineto{\pgfqpoint{4.240856in}{0.585605in}}%
\pgfpathlineto{\pgfqpoint{4.250400in}{0.583417in}}%
\pgfpathlineto{\pgfqpoint{4.259945in}{0.581267in}}%
\pgfpathlineto{\pgfqpoint{4.269489in}{0.579154in}}%
\pgfpathlineto{\pgfqpoint{4.279033in}{0.577077in}}%
\pgfpathlineto{\pgfqpoint{4.288578in}{0.575037in}}%
\pgfpathlineto{\pgfqpoint{4.298122in}{0.573032in}}%
\pgfpathlineto{\pgfqpoint{4.307666in}{0.571063in}}%
\pgfpathlineto{\pgfqpoint{4.317211in}{0.569128in}}%
\pgfpathlineto{\pgfqpoint{4.326755in}{0.567227in}}%
\pgfpathlineto{\pgfqpoint{4.336299in}{0.565360in}}%
\pgfpathlineto{\pgfqpoint{4.345844in}{0.563526in}}%
\pgfpathlineto{\pgfqpoint{4.355388in}{0.561725in}}%
\pgfpathlineto{\pgfqpoint{4.364932in}{0.559956in}}%
\pgfpathlineto{\pgfqpoint{4.374477in}{0.558218in}}%
\pgfpathlineto{\pgfqpoint{4.384021in}{0.556512in}}%
\pgfpathlineto{\pgfqpoint{4.393565in}{0.554837in}}%
\pgfpathlineto{\pgfqpoint{4.403110in}{0.553192in}}%
\pgfpathlineto{\pgfqpoint{4.412654in}{0.551576in}}%
\pgfpathlineto{\pgfqpoint{4.422198in}{0.549990in}}%
\pgfpathlineto{\pgfqpoint{4.431743in}{0.548433in}}%
\pgfpathlineto{\pgfqpoint{4.441287in}{0.546904in}}%
\pgfpathlineto{\pgfqpoint{4.450831in}{0.545403in}}%
\pgfpathlineto{\pgfqpoint{4.460376in}{0.543929in}}%
\pgfpathlineto{\pgfqpoint{4.469920in}{0.542482in}}%
\pgfpathlineto{\pgfqpoint{4.479464in}{0.541062in}}%
\pgfusepath{stroke}%
\end{pgfscope}%
\begin{pgfscope}%
\pgfpathrectangle{\pgfqpoint{0.687500in}{0.385000in}}{\pgfqpoint{4.262500in}{2.695000in}}%
\pgfusepath{clip}%
\pgfsetrectcap%
\pgfsetroundjoin%
\pgfsetlinewidth{1.505625pt}%
\definecolor{currentstroke}{rgb}{0.839216,0.152941,0.156863}%
\pgfsetstrokecolor{currentstroke}%
\pgfsetdash{}{0pt}%
\pgfpathmoveto{\pgfqpoint{4.479464in}{0.541062in}}%
\pgfpathlineto{\pgfqpoint{4.489009in}{0.539667in}}%
\pgfpathlineto{\pgfqpoint{4.498553in}{0.538297in}}%
\pgfpathlineto{\pgfqpoint{4.508097in}{0.536952in}}%
\pgfpathlineto{\pgfqpoint{4.517642in}{0.535629in}}%
\pgfpathlineto{\pgfqpoint{4.527186in}{0.534329in}}%
\pgfpathlineto{\pgfqpoint{4.536730in}{0.533050in}}%
\pgfpathlineto{\pgfqpoint{4.546275in}{0.531792in}}%
\pgfpathlineto{\pgfqpoint{4.555819in}{0.530554in}}%
\pgfpathlineto{\pgfqpoint{4.565363in}{0.529334in}}%
\pgfpathlineto{\pgfqpoint{4.574908in}{0.528132in}}%
\pgfpathlineto{\pgfqpoint{4.584452in}{0.526947in}}%
\pgfpathlineto{\pgfqpoint{4.593996in}{0.525778in}}%
\pgfpathlineto{\pgfqpoint{4.603541in}{0.524624in}}%
\pgfpathlineto{\pgfqpoint{4.613085in}{0.523485in}}%
\pgfpathlineto{\pgfqpoint{4.622629in}{0.522359in}}%
\pgfpathlineto{\pgfqpoint{4.632174in}{0.521245in}}%
\pgfpathlineto{\pgfqpoint{4.641718in}{0.520143in}}%
\pgfpathlineto{\pgfqpoint{4.651262in}{0.519052in}}%
\pgfpathlineto{\pgfqpoint{4.660807in}{0.517971in}}%
\pgfpathlineto{\pgfqpoint{4.670351in}{0.516898in}}%
\pgfpathlineto{\pgfqpoint{4.679895in}{0.515834in}}%
\pgfpathlineto{\pgfqpoint{4.689440in}{0.514776in}}%
\pgfpathlineto{\pgfqpoint{4.698984in}{0.513725in}}%
\pgfpathlineto{\pgfqpoint{4.708528in}{0.512680in}}%
\pgfpathlineto{\pgfqpoint{4.718073in}{0.511638in}}%
\pgfpathlineto{\pgfqpoint{4.727617in}{0.510601in}}%
\pgfpathlineto{\pgfqpoint{4.737161in}{0.509566in}}%
\pgfpathlineto{\pgfqpoint{4.746706in}{0.508532in}}%
\pgfpathlineto{\pgfqpoint{4.756250in}{0.507500in}}%
\pgfusepath{stroke}%
\end{pgfscope}%
\begin{pgfscope}%
\pgfsetrectcap%
\pgfsetmiterjoin%
\pgfsetlinewidth{0.803000pt}%
\definecolor{currentstroke}{rgb}{0.000000,0.000000,0.000000}%
\pgfsetstrokecolor{currentstroke}%
\pgfsetdash{}{0pt}%
\pgfpathmoveto{\pgfqpoint{0.687500in}{0.385000in}}%
\pgfpathlineto{\pgfqpoint{0.687500in}{3.080000in}}%
\pgfusepath{stroke}%
\end{pgfscope}%
\begin{pgfscope}%
\pgfsetrectcap%
\pgfsetmiterjoin%
\pgfsetlinewidth{0.803000pt}%
\definecolor{currentstroke}{rgb}{0.000000,0.000000,0.000000}%
\pgfsetstrokecolor{currentstroke}%
\pgfsetdash{}{0pt}%
\pgfpathmoveto{\pgfqpoint{4.950000in}{0.385000in}}%
\pgfpathlineto{\pgfqpoint{4.950000in}{3.080000in}}%
\pgfusepath{stroke}%
\end{pgfscope}%
\begin{pgfscope}%
\pgfsetrectcap%
\pgfsetmiterjoin%
\pgfsetlinewidth{0.803000pt}%
\definecolor{currentstroke}{rgb}{0.000000,0.000000,0.000000}%
\pgfsetstrokecolor{currentstroke}%
\pgfsetdash{}{0pt}%
\pgfpathmoveto{\pgfqpoint{0.687500in}{0.385000in}}%
\pgfpathlineto{\pgfqpoint{4.950000in}{0.385000in}}%
\pgfusepath{stroke}%
\end{pgfscope}%
\begin{pgfscope}%
\pgfsetrectcap%
\pgfsetmiterjoin%
\pgfsetlinewidth{0.803000pt}%
\definecolor{currentstroke}{rgb}{0.000000,0.000000,0.000000}%
\pgfsetstrokecolor{currentstroke}%
\pgfsetdash{}{0pt}%
\pgfpathmoveto{\pgfqpoint{0.687500in}{3.080000in}}%
\pgfpathlineto{\pgfqpoint{4.950000in}{3.080000in}}%
\pgfusepath{stroke}%
\end{pgfscope}%
\begin{pgfscope}%
\definecolor{textcolor}{rgb}{0.000000,0.000000,0.000000}%
\pgfsetstrokecolor{textcolor}%
\pgfsetfillcolor{textcolor}%
\pgftext[x=2.818750in,y=3.163333in,,base]{\color{textcolor}\rmfamily\fontsize{12.000000}{14.400000}\selectfont N = 14}%
\end{pgfscope}%
\begin{pgfscope}%
\definecolor{textcolor}{rgb}{0.000000,0.000000,0.000000}%
\pgfsetstrokecolor{textcolor}%
\pgfsetfillcolor{textcolor}%
\pgftext[x=2.750000in,y=3.430000in,,top]{\color{textcolor}\rmfamily\fontsize{12.000000}{14.400000}\selectfont Naturalny splajn kubiczny}%
\end{pgfscope}%
\end{pgfpicture}%
\makeatother%
\endgroup%
        
    \end{center}
\end{figure}

Dla \(N=14\) splajn pokrywa funkcję \(f\) na tyle dobrze, że na wykresie
całkowicie przesłania czerwoną przerywaną linię. Sprawdźmy \(N\) nieparzyste.

\begin{figure}[H]
    \begin{center}
        %% Creator: Matplotlib, PGF backend
%%
%% To include the figure in your LaTeX document, write
%%   \input{<filename>.pgf}
%%
%% Make sure the required packages are loaded in your preamble
%%   \usepackage{pgf}
%%
%% Figures using additional raster images can only be included by \input if
%% they are in the same directory as the main LaTeX file. For loading figures
%% from other directories you can use the `import` package
%%   \usepackage{import}
%% and then include the figures with
%%   \import{<path to file>}{<filename>.pgf}
%%
%% Matplotlib used the following preamble
%%
\begingroup%
\makeatletter%
\begin{pgfpicture}%
\pgfpathrectangle{\pgfpointorigin}{\pgfqpoint{5.500000in}{3.500000in}}%
\pgfusepath{use as bounding box, clip}%
\begin{pgfscope}%
\pgfsetbuttcap%
\pgfsetmiterjoin%
\definecolor{currentfill}{rgb}{1.000000,1.000000,1.000000}%
\pgfsetfillcolor{currentfill}%
\pgfsetlinewidth{0.000000pt}%
\definecolor{currentstroke}{rgb}{1.000000,1.000000,1.000000}%
\pgfsetstrokecolor{currentstroke}%
\pgfsetdash{}{0pt}%
\pgfpathmoveto{\pgfqpoint{0.000000in}{0.000000in}}%
\pgfpathlineto{\pgfqpoint{5.500000in}{0.000000in}}%
\pgfpathlineto{\pgfqpoint{5.500000in}{3.500000in}}%
\pgfpathlineto{\pgfqpoint{0.000000in}{3.500000in}}%
\pgfpathclose%
\pgfusepath{fill}%
\end{pgfscope}%
\begin{pgfscope}%
\pgfsetbuttcap%
\pgfsetmiterjoin%
\definecolor{currentfill}{rgb}{1.000000,1.000000,1.000000}%
\pgfsetfillcolor{currentfill}%
\pgfsetlinewidth{0.000000pt}%
\definecolor{currentstroke}{rgb}{0.000000,0.000000,0.000000}%
\pgfsetstrokecolor{currentstroke}%
\pgfsetstrokeopacity{0.000000}%
\pgfsetdash{}{0pt}%
\pgfpathmoveto{\pgfqpoint{0.687500in}{0.385000in}}%
\pgfpathlineto{\pgfqpoint{4.950000in}{0.385000in}}%
\pgfpathlineto{\pgfqpoint{4.950000in}{3.080000in}}%
\pgfpathlineto{\pgfqpoint{0.687500in}{3.080000in}}%
\pgfpathclose%
\pgfusepath{fill}%
\end{pgfscope}%
\begin{pgfscope}%
\pgfsetbuttcap%
\pgfsetroundjoin%
\definecolor{currentfill}{rgb}{0.000000,0.000000,0.000000}%
\pgfsetfillcolor{currentfill}%
\pgfsetlinewidth{0.803000pt}%
\definecolor{currentstroke}{rgb}{0.000000,0.000000,0.000000}%
\pgfsetstrokecolor{currentstroke}%
\pgfsetdash{}{0pt}%
\pgfsys@defobject{currentmarker}{\pgfqpoint{0.000000in}{-0.048611in}}{\pgfqpoint{0.000000in}{0.000000in}}{%
\pgfpathmoveto{\pgfqpoint{0.000000in}{0.000000in}}%
\pgfpathlineto{\pgfqpoint{0.000000in}{-0.048611in}}%
\pgfusepath{stroke,fill}%
}%
\begin{pgfscope}%
\pgfsys@transformshift{0.881250in}{0.385000in}%
\pgfsys@useobject{currentmarker}{}%
\end{pgfscope}%
\end{pgfscope}%
\begin{pgfscope}%
\definecolor{textcolor}{rgb}{0.000000,0.000000,0.000000}%
\pgfsetstrokecolor{textcolor}%
\pgfsetfillcolor{textcolor}%
\pgftext[x=0.881250in,y=0.287778in,,top]{\color{textcolor}\rmfamily\fontsize{10.000000}{12.000000}\selectfont \(\displaystyle -1.00\)}%
\end{pgfscope}%
\begin{pgfscope}%
\pgfsetbuttcap%
\pgfsetroundjoin%
\definecolor{currentfill}{rgb}{0.000000,0.000000,0.000000}%
\pgfsetfillcolor{currentfill}%
\pgfsetlinewidth{0.803000pt}%
\definecolor{currentstroke}{rgb}{0.000000,0.000000,0.000000}%
\pgfsetstrokecolor{currentstroke}%
\pgfsetdash{}{0pt}%
\pgfsys@defobject{currentmarker}{\pgfqpoint{0.000000in}{-0.048611in}}{\pgfqpoint{0.000000in}{0.000000in}}{%
\pgfpathmoveto{\pgfqpoint{0.000000in}{0.000000in}}%
\pgfpathlineto{\pgfqpoint{0.000000in}{-0.048611in}}%
\pgfusepath{stroke,fill}%
}%
\begin{pgfscope}%
\pgfsys@transformshift{1.365625in}{0.385000in}%
\pgfsys@useobject{currentmarker}{}%
\end{pgfscope}%
\end{pgfscope}%
\begin{pgfscope}%
\definecolor{textcolor}{rgb}{0.000000,0.000000,0.000000}%
\pgfsetstrokecolor{textcolor}%
\pgfsetfillcolor{textcolor}%
\pgftext[x=1.365625in,y=0.287778in,,top]{\color{textcolor}\rmfamily\fontsize{10.000000}{12.000000}\selectfont \(\displaystyle -0.75\)}%
\end{pgfscope}%
\begin{pgfscope}%
\pgfsetbuttcap%
\pgfsetroundjoin%
\definecolor{currentfill}{rgb}{0.000000,0.000000,0.000000}%
\pgfsetfillcolor{currentfill}%
\pgfsetlinewidth{0.803000pt}%
\definecolor{currentstroke}{rgb}{0.000000,0.000000,0.000000}%
\pgfsetstrokecolor{currentstroke}%
\pgfsetdash{}{0pt}%
\pgfsys@defobject{currentmarker}{\pgfqpoint{0.000000in}{-0.048611in}}{\pgfqpoint{0.000000in}{0.000000in}}{%
\pgfpathmoveto{\pgfqpoint{0.000000in}{0.000000in}}%
\pgfpathlineto{\pgfqpoint{0.000000in}{-0.048611in}}%
\pgfusepath{stroke,fill}%
}%
\begin{pgfscope}%
\pgfsys@transformshift{1.850000in}{0.385000in}%
\pgfsys@useobject{currentmarker}{}%
\end{pgfscope}%
\end{pgfscope}%
\begin{pgfscope}%
\definecolor{textcolor}{rgb}{0.000000,0.000000,0.000000}%
\pgfsetstrokecolor{textcolor}%
\pgfsetfillcolor{textcolor}%
\pgftext[x=1.850000in,y=0.287778in,,top]{\color{textcolor}\rmfamily\fontsize{10.000000}{12.000000}\selectfont \(\displaystyle -0.50\)}%
\end{pgfscope}%
\begin{pgfscope}%
\pgfsetbuttcap%
\pgfsetroundjoin%
\definecolor{currentfill}{rgb}{0.000000,0.000000,0.000000}%
\pgfsetfillcolor{currentfill}%
\pgfsetlinewidth{0.803000pt}%
\definecolor{currentstroke}{rgb}{0.000000,0.000000,0.000000}%
\pgfsetstrokecolor{currentstroke}%
\pgfsetdash{}{0pt}%
\pgfsys@defobject{currentmarker}{\pgfqpoint{0.000000in}{-0.048611in}}{\pgfqpoint{0.000000in}{0.000000in}}{%
\pgfpathmoveto{\pgfqpoint{0.000000in}{0.000000in}}%
\pgfpathlineto{\pgfqpoint{0.000000in}{-0.048611in}}%
\pgfusepath{stroke,fill}%
}%
\begin{pgfscope}%
\pgfsys@transformshift{2.334375in}{0.385000in}%
\pgfsys@useobject{currentmarker}{}%
\end{pgfscope}%
\end{pgfscope}%
\begin{pgfscope}%
\definecolor{textcolor}{rgb}{0.000000,0.000000,0.000000}%
\pgfsetstrokecolor{textcolor}%
\pgfsetfillcolor{textcolor}%
\pgftext[x=2.334375in,y=0.287778in,,top]{\color{textcolor}\rmfamily\fontsize{10.000000}{12.000000}\selectfont \(\displaystyle -0.25\)}%
\end{pgfscope}%
\begin{pgfscope}%
\pgfsetbuttcap%
\pgfsetroundjoin%
\definecolor{currentfill}{rgb}{0.000000,0.000000,0.000000}%
\pgfsetfillcolor{currentfill}%
\pgfsetlinewidth{0.803000pt}%
\definecolor{currentstroke}{rgb}{0.000000,0.000000,0.000000}%
\pgfsetstrokecolor{currentstroke}%
\pgfsetdash{}{0pt}%
\pgfsys@defobject{currentmarker}{\pgfqpoint{0.000000in}{-0.048611in}}{\pgfqpoint{0.000000in}{0.000000in}}{%
\pgfpathmoveto{\pgfqpoint{0.000000in}{0.000000in}}%
\pgfpathlineto{\pgfqpoint{0.000000in}{-0.048611in}}%
\pgfusepath{stroke,fill}%
}%
\begin{pgfscope}%
\pgfsys@transformshift{2.818750in}{0.385000in}%
\pgfsys@useobject{currentmarker}{}%
\end{pgfscope}%
\end{pgfscope}%
\begin{pgfscope}%
\definecolor{textcolor}{rgb}{0.000000,0.000000,0.000000}%
\pgfsetstrokecolor{textcolor}%
\pgfsetfillcolor{textcolor}%
\pgftext[x=2.818750in,y=0.287778in,,top]{\color{textcolor}\rmfamily\fontsize{10.000000}{12.000000}\selectfont \(\displaystyle 0.00\)}%
\end{pgfscope}%
\begin{pgfscope}%
\pgfsetbuttcap%
\pgfsetroundjoin%
\definecolor{currentfill}{rgb}{0.000000,0.000000,0.000000}%
\pgfsetfillcolor{currentfill}%
\pgfsetlinewidth{0.803000pt}%
\definecolor{currentstroke}{rgb}{0.000000,0.000000,0.000000}%
\pgfsetstrokecolor{currentstroke}%
\pgfsetdash{}{0pt}%
\pgfsys@defobject{currentmarker}{\pgfqpoint{0.000000in}{-0.048611in}}{\pgfqpoint{0.000000in}{0.000000in}}{%
\pgfpathmoveto{\pgfqpoint{0.000000in}{0.000000in}}%
\pgfpathlineto{\pgfqpoint{0.000000in}{-0.048611in}}%
\pgfusepath{stroke,fill}%
}%
\begin{pgfscope}%
\pgfsys@transformshift{3.303125in}{0.385000in}%
\pgfsys@useobject{currentmarker}{}%
\end{pgfscope}%
\end{pgfscope}%
\begin{pgfscope}%
\definecolor{textcolor}{rgb}{0.000000,0.000000,0.000000}%
\pgfsetstrokecolor{textcolor}%
\pgfsetfillcolor{textcolor}%
\pgftext[x=3.303125in,y=0.287778in,,top]{\color{textcolor}\rmfamily\fontsize{10.000000}{12.000000}\selectfont \(\displaystyle 0.25\)}%
\end{pgfscope}%
\begin{pgfscope}%
\pgfsetbuttcap%
\pgfsetroundjoin%
\definecolor{currentfill}{rgb}{0.000000,0.000000,0.000000}%
\pgfsetfillcolor{currentfill}%
\pgfsetlinewidth{0.803000pt}%
\definecolor{currentstroke}{rgb}{0.000000,0.000000,0.000000}%
\pgfsetstrokecolor{currentstroke}%
\pgfsetdash{}{0pt}%
\pgfsys@defobject{currentmarker}{\pgfqpoint{0.000000in}{-0.048611in}}{\pgfqpoint{0.000000in}{0.000000in}}{%
\pgfpathmoveto{\pgfqpoint{0.000000in}{0.000000in}}%
\pgfpathlineto{\pgfqpoint{0.000000in}{-0.048611in}}%
\pgfusepath{stroke,fill}%
}%
\begin{pgfscope}%
\pgfsys@transformshift{3.787500in}{0.385000in}%
\pgfsys@useobject{currentmarker}{}%
\end{pgfscope}%
\end{pgfscope}%
\begin{pgfscope}%
\definecolor{textcolor}{rgb}{0.000000,0.000000,0.000000}%
\pgfsetstrokecolor{textcolor}%
\pgfsetfillcolor{textcolor}%
\pgftext[x=3.787500in,y=0.287778in,,top]{\color{textcolor}\rmfamily\fontsize{10.000000}{12.000000}\selectfont \(\displaystyle 0.50\)}%
\end{pgfscope}%
\begin{pgfscope}%
\pgfsetbuttcap%
\pgfsetroundjoin%
\definecolor{currentfill}{rgb}{0.000000,0.000000,0.000000}%
\pgfsetfillcolor{currentfill}%
\pgfsetlinewidth{0.803000pt}%
\definecolor{currentstroke}{rgb}{0.000000,0.000000,0.000000}%
\pgfsetstrokecolor{currentstroke}%
\pgfsetdash{}{0pt}%
\pgfsys@defobject{currentmarker}{\pgfqpoint{0.000000in}{-0.048611in}}{\pgfqpoint{0.000000in}{0.000000in}}{%
\pgfpathmoveto{\pgfqpoint{0.000000in}{0.000000in}}%
\pgfpathlineto{\pgfqpoint{0.000000in}{-0.048611in}}%
\pgfusepath{stroke,fill}%
}%
\begin{pgfscope}%
\pgfsys@transformshift{4.271875in}{0.385000in}%
\pgfsys@useobject{currentmarker}{}%
\end{pgfscope}%
\end{pgfscope}%
\begin{pgfscope}%
\definecolor{textcolor}{rgb}{0.000000,0.000000,0.000000}%
\pgfsetstrokecolor{textcolor}%
\pgfsetfillcolor{textcolor}%
\pgftext[x=4.271875in,y=0.287778in,,top]{\color{textcolor}\rmfamily\fontsize{10.000000}{12.000000}\selectfont \(\displaystyle 0.75\)}%
\end{pgfscope}%
\begin{pgfscope}%
\pgfsetbuttcap%
\pgfsetroundjoin%
\definecolor{currentfill}{rgb}{0.000000,0.000000,0.000000}%
\pgfsetfillcolor{currentfill}%
\pgfsetlinewidth{0.803000pt}%
\definecolor{currentstroke}{rgb}{0.000000,0.000000,0.000000}%
\pgfsetstrokecolor{currentstroke}%
\pgfsetdash{}{0pt}%
\pgfsys@defobject{currentmarker}{\pgfqpoint{0.000000in}{-0.048611in}}{\pgfqpoint{0.000000in}{0.000000in}}{%
\pgfpathmoveto{\pgfqpoint{0.000000in}{0.000000in}}%
\pgfpathlineto{\pgfqpoint{0.000000in}{-0.048611in}}%
\pgfusepath{stroke,fill}%
}%
\begin{pgfscope}%
\pgfsys@transformshift{4.756250in}{0.385000in}%
\pgfsys@useobject{currentmarker}{}%
\end{pgfscope}%
\end{pgfscope}%
\begin{pgfscope}%
\definecolor{textcolor}{rgb}{0.000000,0.000000,0.000000}%
\pgfsetstrokecolor{textcolor}%
\pgfsetfillcolor{textcolor}%
\pgftext[x=4.756250in,y=0.287778in,,top]{\color{textcolor}\rmfamily\fontsize{10.000000}{12.000000}\selectfont \(\displaystyle 1.00\)}%
\end{pgfscope}%
\begin{pgfscope}%
\definecolor{textcolor}{rgb}{0.000000,0.000000,0.000000}%
\pgfsetstrokecolor{textcolor}%
\pgfsetfillcolor{textcolor}%
\pgftext[x=2.818750in,y=0.108766in,,top]{\color{textcolor}\rmfamily\fontsize{10.000000}{12.000000}\selectfont x}%
\end{pgfscope}%
\begin{pgfscope}%
\pgfsetbuttcap%
\pgfsetroundjoin%
\definecolor{currentfill}{rgb}{0.000000,0.000000,0.000000}%
\pgfsetfillcolor{currentfill}%
\pgfsetlinewidth{0.803000pt}%
\definecolor{currentstroke}{rgb}{0.000000,0.000000,0.000000}%
\pgfsetstrokecolor{currentstroke}%
\pgfsetdash{}{0pt}%
\pgfsys@defobject{currentmarker}{\pgfqpoint{-0.048611in}{0.000000in}}{\pgfqpoint{0.000000in}{0.000000in}}{%
\pgfpathmoveto{\pgfqpoint{0.000000in}{0.000000in}}%
\pgfpathlineto{\pgfqpoint{-0.048611in}{0.000000in}}%
\pgfusepath{stroke,fill}%
}%
\begin{pgfscope}%
\pgfsys@transformshift{0.687500in}{0.409476in}%
\pgfsys@useobject{currentmarker}{}%
\end{pgfscope}%
\end{pgfscope}%
\begin{pgfscope}%
\definecolor{textcolor}{rgb}{0.000000,0.000000,0.000000}%
\pgfsetstrokecolor{textcolor}%
\pgfsetfillcolor{textcolor}%
\pgftext[x=0.412808in,y=0.361251in,left,base]{\color{textcolor}\rmfamily\fontsize{10.000000}{12.000000}\selectfont \(\displaystyle 0.0\)}%
\end{pgfscope}%
\begin{pgfscope}%
\pgfsetbuttcap%
\pgfsetroundjoin%
\definecolor{currentfill}{rgb}{0.000000,0.000000,0.000000}%
\pgfsetfillcolor{currentfill}%
\pgfsetlinewidth{0.803000pt}%
\definecolor{currentstroke}{rgb}{0.000000,0.000000,0.000000}%
\pgfsetstrokecolor{currentstroke}%
\pgfsetdash{}{0pt}%
\pgfsys@defobject{currentmarker}{\pgfqpoint{-0.048611in}{0.000000in}}{\pgfqpoint{0.000000in}{0.000000in}}{%
\pgfpathmoveto{\pgfqpoint{0.000000in}{0.000000in}}%
\pgfpathlineto{\pgfqpoint{-0.048611in}{0.000000in}}%
\pgfusepath{stroke,fill}%
}%
\begin{pgfscope}%
\pgfsys@transformshift{0.687500in}{0.919199in}%
\pgfsys@useobject{currentmarker}{}%
\end{pgfscope}%
\end{pgfscope}%
\begin{pgfscope}%
\definecolor{textcolor}{rgb}{0.000000,0.000000,0.000000}%
\pgfsetstrokecolor{textcolor}%
\pgfsetfillcolor{textcolor}%
\pgftext[x=0.412808in,y=0.870974in,left,base]{\color{textcolor}\rmfamily\fontsize{10.000000}{12.000000}\selectfont \(\displaystyle 0.2\)}%
\end{pgfscope}%
\begin{pgfscope}%
\pgfsetbuttcap%
\pgfsetroundjoin%
\definecolor{currentfill}{rgb}{0.000000,0.000000,0.000000}%
\pgfsetfillcolor{currentfill}%
\pgfsetlinewidth{0.803000pt}%
\definecolor{currentstroke}{rgb}{0.000000,0.000000,0.000000}%
\pgfsetstrokecolor{currentstroke}%
\pgfsetdash{}{0pt}%
\pgfsys@defobject{currentmarker}{\pgfqpoint{-0.048611in}{0.000000in}}{\pgfqpoint{0.000000in}{0.000000in}}{%
\pgfpathmoveto{\pgfqpoint{0.000000in}{0.000000in}}%
\pgfpathlineto{\pgfqpoint{-0.048611in}{0.000000in}}%
\pgfusepath{stroke,fill}%
}%
\begin{pgfscope}%
\pgfsys@transformshift{0.687500in}{1.428921in}%
\pgfsys@useobject{currentmarker}{}%
\end{pgfscope}%
\end{pgfscope}%
\begin{pgfscope}%
\definecolor{textcolor}{rgb}{0.000000,0.000000,0.000000}%
\pgfsetstrokecolor{textcolor}%
\pgfsetfillcolor{textcolor}%
\pgftext[x=0.412808in,y=1.380696in,left,base]{\color{textcolor}\rmfamily\fontsize{10.000000}{12.000000}\selectfont \(\displaystyle 0.4\)}%
\end{pgfscope}%
\begin{pgfscope}%
\pgfsetbuttcap%
\pgfsetroundjoin%
\definecolor{currentfill}{rgb}{0.000000,0.000000,0.000000}%
\pgfsetfillcolor{currentfill}%
\pgfsetlinewidth{0.803000pt}%
\definecolor{currentstroke}{rgb}{0.000000,0.000000,0.000000}%
\pgfsetstrokecolor{currentstroke}%
\pgfsetdash{}{0pt}%
\pgfsys@defobject{currentmarker}{\pgfqpoint{-0.048611in}{0.000000in}}{\pgfqpoint{0.000000in}{0.000000in}}{%
\pgfpathmoveto{\pgfqpoint{0.000000in}{0.000000in}}%
\pgfpathlineto{\pgfqpoint{-0.048611in}{0.000000in}}%
\pgfusepath{stroke,fill}%
}%
\begin{pgfscope}%
\pgfsys@transformshift{0.687500in}{1.938644in}%
\pgfsys@useobject{currentmarker}{}%
\end{pgfscope}%
\end{pgfscope}%
\begin{pgfscope}%
\definecolor{textcolor}{rgb}{0.000000,0.000000,0.000000}%
\pgfsetstrokecolor{textcolor}%
\pgfsetfillcolor{textcolor}%
\pgftext[x=0.412808in,y=1.890418in,left,base]{\color{textcolor}\rmfamily\fontsize{10.000000}{12.000000}\selectfont \(\displaystyle 0.6\)}%
\end{pgfscope}%
\begin{pgfscope}%
\pgfsetbuttcap%
\pgfsetroundjoin%
\definecolor{currentfill}{rgb}{0.000000,0.000000,0.000000}%
\pgfsetfillcolor{currentfill}%
\pgfsetlinewidth{0.803000pt}%
\definecolor{currentstroke}{rgb}{0.000000,0.000000,0.000000}%
\pgfsetstrokecolor{currentstroke}%
\pgfsetdash{}{0pt}%
\pgfsys@defobject{currentmarker}{\pgfqpoint{-0.048611in}{0.000000in}}{\pgfqpoint{0.000000in}{0.000000in}}{%
\pgfpathmoveto{\pgfqpoint{0.000000in}{0.000000in}}%
\pgfpathlineto{\pgfqpoint{-0.048611in}{0.000000in}}%
\pgfusepath{stroke,fill}%
}%
\begin{pgfscope}%
\pgfsys@transformshift{0.687500in}{2.448366in}%
\pgfsys@useobject{currentmarker}{}%
\end{pgfscope}%
\end{pgfscope}%
\begin{pgfscope}%
\definecolor{textcolor}{rgb}{0.000000,0.000000,0.000000}%
\pgfsetstrokecolor{textcolor}%
\pgfsetfillcolor{textcolor}%
\pgftext[x=0.412808in,y=2.400141in,left,base]{\color{textcolor}\rmfamily\fontsize{10.000000}{12.000000}\selectfont \(\displaystyle 0.8\)}%
\end{pgfscope}%
\begin{pgfscope}%
\pgfsetbuttcap%
\pgfsetroundjoin%
\definecolor{currentfill}{rgb}{0.000000,0.000000,0.000000}%
\pgfsetfillcolor{currentfill}%
\pgfsetlinewidth{0.803000pt}%
\definecolor{currentstroke}{rgb}{0.000000,0.000000,0.000000}%
\pgfsetstrokecolor{currentstroke}%
\pgfsetdash{}{0pt}%
\pgfsys@defobject{currentmarker}{\pgfqpoint{-0.048611in}{0.000000in}}{\pgfqpoint{0.000000in}{0.000000in}}{%
\pgfpathmoveto{\pgfqpoint{0.000000in}{0.000000in}}%
\pgfpathlineto{\pgfqpoint{-0.048611in}{0.000000in}}%
\pgfusepath{stroke,fill}%
}%
\begin{pgfscope}%
\pgfsys@transformshift{0.687500in}{2.958089in}%
\pgfsys@useobject{currentmarker}{}%
\end{pgfscope}%
\end{pgfscope}%
\begin{pgfscope}%
\definecolor{textcolor}{rgb}{0.000000,0.000000,0.000000}%
\pgfsetstrokecolor{textcolor}%
\pgfsetfillcolor{textcolor}%
\pgftext[x=0.412808in,y=2.909863in,left,base]{\color{textcolor}\rmfamily\fontsize{10.000000}{12.000000}\selectfont \(\displaystyle 1.0\)}%
\end{pgfscope}%
\begin{pgfscope}%
\definecolor{textcolor}{rgb}{0.000000,0.000000,0.000000}%
\pgfsetstrokecolor{textcolor}%
\pgfsetfillcolor{textcolor}%
\pgftext[x=0.357253in,y=1.732500in,,bottom,rotate=90.000000]{\color{textcolor}\rmfamily\fontsize{10.000000}{12.000000}\selectfont y}%
\end{pgfscope}%
\begin{pgfscope}%
\pgfpathrectangle{\pgfqpoint{0.687500in}{0.385000in}}{\pgfqpoint{4.262500in}{2.695000in}}%
\pgfusepath{clip}%
\pgfsetbuttcap%
\pgfsetroundjoin%
\pgfsetlinewidth{1.505625pt}%
\definecolor{currentstroke}{rgb}{1.000000,0.000000,0.000000}%
\pgfsetstrokecolor{currentstroke}%
\pgfsetdash{{5.550000pt}{2.400000pt}}{0.000000pt}%
\pgfpathmoveto{\pgfqpoint{0.881250in}{0.507500in}}%
\pgfpathlineto{\pgfqpoint{0.999031in}{0.520032in}}%
\pgfpathlineto{\pgfqpoint{1.105034in}{0.533446in}}%
\pgfpathlineto{\pgfqpoint{1.199259in}{0.547487in}}%
\pgfpathlineto{\pgfqpoint{1.281706in}{0.561781in}}%
\pgfpathlineto{\pgfqpoint{1.364153in}{0.578359in}}%
\pgfpathlineto{\pgfqpoint{1.434821in}{0.594761in}}%
\pgfpathlineto{\pgfqpoint{1.505490in}{0.613599in}}%
\pgfpathlineto{\pgfqpoint{1.564381in}{0.631506in}}%
\pgfpathlineto{\pgfqpoint{1.623271in}{0.651789in}}%
\pgfpathlineto{\pgfqpoint{1.670384in}{0.670005in}}%
\pgfpathlineto{\pgfqpoint{1.717496in}{0.690264in}}%
\pgfpathlineto{\pgfqpoint{1.764609in}{0.712869in}}%
\pgfpathlineto{\pgfqpoint{1.811721in}{0.738174in}}%
\pgfpathlineto{\pgfqpoint{1.858834in}{0.766598in}}%
\pgfpathlineto{\pgfqpoint{1.894168in}{0.790259in}}%
\pgfpathlineto{\pgfqpoint{1.929502in}{0.816196in}}%
\pgfpathlineto{\pgfqpoint{1.964837in}{0.844686in}}%
\pgfpathlineto{\pgfqpoint{2.000171in}{0.876042in}}%
\pgfpathlineto{\pgfqpoint{2.035505in}{0.910623in}}%
\pgfpathlineto{\pgfqpoint{2.070840in}{0.948836in}}%
\pgfpathlineto{\pgfqpoint{2.106174in}{0.991142in}}%
\pgfpathlineto{\pgfqpoint{2.141508in}{1.038061in}}%
\pgfpathlineto{\pgfqpoint{2.176843in}{1.090176in}}%
\pgfpathlineto{\pgfqpoint{2.212177in}{1.148136in}}%
\pgfpathlineto{\pgfqpoint{2.235733in}{1.190373in}}%
\pgfpathlineto{\pgfqpoint{2.259290in}{1.235749in}}%
\pgfpathlineto{\pgfqpoint{2.282846in}{1.284497in}}%
\pgfpathlineto{\pgfqpoint{2.306402in}{1.336857in}}%
\pgfpathlineto{\pgfqpoint{2.329958in}{1.393069in}}%
\pgfpathlineto{\pgfqpoint{2.353514in}{1.453365in}}%
\pgfpathlineto{\pgfqpoint{2.377071in}{1.517962in}}%
\pgfpathlineto{\pgfqpoint{2.400627in}{1.587045in}}%
\pgfpathlineto{\pgfqpoint{2.435961in}{1.699370in}}%
\pgfpathlineto{\pgfqpoint{2.471296in}{1.822238in}}%
\pgfpathlineto{\pgfqpoint{2.506630in}{1.955229in}}%
\pgfpathlineto{\pgfqpoint{2.541964in}{2.097071in}}%
\pgfpathlineto{\pgfqpoint{2.600855in}{2.345828in}}%
\pgfpathlineto{\pgfqpoint{2.647967in}{2.543559in}}%
\pgfpathlineto{\pgfqpoint{2.671524in}{2.636595in}}%
\pgfpathlineto{\pgfqpoint{2.695080in}{2.722494in}}%
\pgfpathlineto{\pgfqpoint{2.718636in}{2.798615in}}%
\pgfpathlineto{\pgfqpoint{2.730414in}{2.832187in}}%
\pgfpathlineto{\pgfqpoint{2.742192in}{2.862345in}}%
\pgfpathlineto{\pgfqpoint{2.753970in}{2.888799in}}%
\pgfpathlineto{\pgfqpoint{2.765748in}{2.911284in}}%
\pgfpathlineto{\pgfqpoint{2.777527in}{2.929568in}}%
\pgfpathlineto{\pgfqpoint{2.789305in}{2.943457in}}%
\pgfpathlineto{\pgfqpoint{2.801083in}{2.952802in}}%
\pgfpathlineto{\pgfqpoint{2.812861in}{2.957500in}}%
\pgfpathlineto{\pgfqpoint{2.824639in}{2.957500in}}%
\pgfpathlineto{\pgfqpoint{2.836417in}{2.952802in}}%
\pgfpathlineto{\pgfqpoint{2.848195in}{2.943457in}}%
\pgfpathlineto{\pgfqpoint{2.859973in}{2.929568in}}%
\pgfpathlineto{\pgfqpoint{2.871752in}{2.911284in}}%
\pgfpathlineto{\pgfqpoint{2.883530in}{2.888799in}}%
\pgfpathlineto{\pgfqpoint{2.895308in}{2.862345in}}%
\pgfpathlineto{\pgfqpoint{2.907086in}{2.832187in}}%
\pgfpathlineto{\pgfqpoint{2.930642in}{2.761943in}}%
\pgfpathlineto{\pgfqpoint{2.954198in}{2.680600in}}%
\pgfpathlineto{\pgfqpoint{2.977755in}{2.590808in}}%
\pgfpathlineto{\pgfqpoint{3.013089in}{2.445886in}}%
\pgfpathlineto{\pgfqpoint{3.119092in}{2.001620in}}%
\pgfpathlineto{\pgfqpoint{3.154426in}{1.865486in}}%
\pgfpathlineto{\pgfqpoint{3.189761in}{1.739161in}}%
\pgfpathlineto{\pgfqpoint{3.225095in}{1.623315in}}%
\pgfpathlineto{\pgfqpoint{3.260429in}{1.517962in}}%
\pgfpathlineto{\pgfqpoint{3.295764in}{1.422693in}}%
\pgfpathlineto{\pgfqpoint{3.319320in}{1.364467in}}%
\pgfpathlineto{\pgfqpoint{3.342876in}{1.310211in}}%
\pgfpathlineto{\pgfqpoint{3.366432in}{1.259687in}}%
\pgfpathlineto{\pgfqpoint{3.401767in}{1.190373in}}%
\pgfpathlineto{\pgfqpoint{3.437101in}{1.128124in}}%
\pgfpathlineto{\pgfqpoint{3.472435in}{1.072188in}}%
\pgfpathlineto{\pgfqpoint{3.507770in}{1.021874in}}%
\pgfpathlineto{\pgfqpoint{3.543104in}{0.976555in}}%
\pgfpathlineto{\pgfqpoint{3.578438in}{0.935668in}}%
\pgfpathlineto{\pgfqpoint{3.613773in}{0.898714in}}%
\pgfpathlineto{\pgfqpoint{3.649107in}{0.865250in}}%
\pgfpathlineto{\pgfqpoint{3.684441in}{0.834887in}}%
\pgfpathlineto{\pgfqpoint{3.719776in}{0.807282in}}%
\pgfpathlineto{\pgfqpoint{3.755110in}{0.782132in}}%
\pgfpathlineto{\pgfqpoint{3.790445in}{0.759174in}}%
\pgfpathlineto{\pgfqpoint{3.837557in}{0.731573in}}%
\pgfpathlineto{\pgfqpoint{3.884669in}{0.706979in}}%
\pgfpathlineto{\pgfqpoint{3.931782in}{0.684992in}}%
\pgfpathlineto{\pgfqpoint{3.978894in}{0.665270in}}%
\pgfpathlineto{\pgfqpoint{4.037785in}{0.643366in}}%
\pgfpathlineto{\pgfqpoint{4.096676in}{0.624079in}}%
\pgfpathlineto{\pgfqpoint{4.155566in}{0.607021in}}%
\pgfpathlineto{\pgfqpoint{4.226235in}{0.589045in}}%
\pgfpathlineto{\pgfqpoint{4.296903in}{0.573363in}}%
\pgfpathlineto{\pgfqpoint{4.379350in}{0.557483in}}%
\pgfpathlineto{\pgfqpoint{4.461797in}{0.543765in}}%
\pgfpathlineto{\pgfqpoint{4.556022in}{0.530265in}}%
\pgfpathlineto{\pgfqpoint{4.662025in}{0.517343in}}%
\pgfpathlineto{\pgfqpoint{4.756250in}{0.507500in}}%
\pgfpathlineto{\pgfqpoint{4.756250in}{0.507500in}}%
\pgfusepath{stroke}%
\end{pgfscope}%
\begin{pgfscope}%
\pgfpathrectangle{\pgfqpoint{0.687500in}{0.385000in}}{\pgfqpoint{4.262500in}{2.695000in}}%
\pgfusepath{clip}%
\pgfsetrectcap%
\pgfsetroundjoin%
\pgfsetlinewidth{1.505625pt}%
\definecolor{currentstroke}{rgb}{0.121569,0.466667,0.705882}%
\pgfsetstrokecolor{currentstroke}%
\pgfsetdash{}{0pt}%
\pgfpathmoveto{\pgfqpoint{0.881250in}{0.507500in}}%
\pgfpathlineto{\pgfqpoint{0.893397in}{0.508930in}}%
\pgfpathlineto{\pgfqpoint{0.905545in}{0.510362in}}%
\pgfpathlineto{\pgfqpoint{0.917692in}{0.511796in}}%
\pgfpathlineto{\pgfqpoint{0.929839in}{0.513232in}}%
\pgfpathlineto{\pgfqpoint{0.941987in}{0.514673in}}%
\pgfpathlineto{\pgfqpoint{0.954134in}{0.516119in}}%
\pgfpathlineto{\pgfqpoint{0.966281in}{0.517571in}}%
\pgfpathlineto{\pgfqpoint{0.978429in}{0.519031in}}%
\pgfpathlineto{\pgfqpoint{0.990576in}{0.520498in}}%
\pgfpathlineto{\pgfqpoint{1.002723in}{0.521975in}}%
\pgfpathlineto{\pgfqpoint{1.014871in}{0.523462in}}%
\pgfpathlineto{\pgfqpoint{1.027018in}{0.524961in}}%
\pgfpathlineto{\pgfqpoint{1.039165in}{0.526472in}}%
\pgfpathlineto{\pgfqpoint{1.051313in}{0.527997in}}%
\pgfpathlineto{\pgfqpoint{1.063460in}{0.529535in}}%
\pgfpathlineto{\pgfqpoint{1.075607in}{0.531090in}}%
\pgfpathlineto{\pgfqpoint{1.087755in}{0.532661in}}%
\pgfpathlineto{\pgfqpoint{1.099902in}{0.534249in}}%
\pgfpathlineto{\pgfqpoint{1.112049in}{0.535856in}}%
\pgfpathlineto{\pgfqpoint{1.124197in}{0.537483in}}%
\pgfpathlineto{\pgfqpoint{1.136344in}{0.539131in}}%
\pgfpathlineto{\pgfqpoint{1.148491in}{0.540800in}}%
\pgfpathlineto{\pgfqpoint{1.160639in}{0.542491in}}%
\pgfpathlineto{\pgfqpoint{1.172786in}{0.544207in}}%
\pgfpathlineto{\pgfqpoint{1.184933in}{0.545947in}}%
\pgfpathlineto{\pgfqpoint{1.197081in}{0.547714in}}%
\pgfpathlineto{\pgfqpoint{1.209228in}{0.549507in}}%
\pgfpathlineto{\pgfqpoint{1.221375in}{0.551327in}}%
\pgfpathlineto{\pgfqpoint{1.233523in}{0.553177in}}%
\pgfusepath{stroke}%
\end{pgfscope}%
\begin{pgfscope}%
\pgfpathrectangle{\pgfqpoint{0.687500in}{0.385000in}}{\pgfqpoint{4.262500in}{2.695000in}}%
\pgfusepath{clip}%
\pgfsetrectcap%
\pgfsetroundjoin%
\pgfsetlinewidth{1.505625pt}%
\definecolor{currentstroke}{rgb}{1.000000,0.498039,0.054902}%
\pgfsetstrokecolor{currentstroke}%
\pgfsetdash{}{0pt}%
\pgfpathmoveto{\pgfqpoint{1.233523in}{0.553177in}}%
\pgfpathlineto{\pgfqpoint{1.245670in}{0.555058in}}%
\pgfpathlineto{\pgfqpoint{1.257817in}{0.556973in}}%
\pgfpathlineto{\pgfqpoint{1.269965in}{0.558927in}}%
\pgfpathlineto{\pgfqpoint{1.282112in}{0.560925in}}%
\pgfpathlineto{\pgfqpoint{1.294259in}{0.562971in}}%
\pgfpathlineto{\pgfqpoint{1.306407in}{0.565070in}}%
\pgfpathlineto{\pgfqpoint{1.318554in}{0.567227in}}%
\pgfpathlineto{\pgfqpoint{1.330701in}{0.569446in}}%
\pgfpathlineto{\pgfqpoint{1.342849in}{0.571732in}}%
\pgfpathlineto{\pgfqpoint{1.354996in}{0.574090in}}%
\pgfpathlineto{\pgfqpoint{1.367143in}{0.576523in}}%
\pgfpathlineto{\pgfqpoint{1.379291in}{0.579038in}}%
\pgfpathlineto{\pgfqpoint{1.391438in}{0.581637in}}%
\pgfpathlineto{\pgfqpoint{1.403585in}{0.584326in}}%
\pgfpathlineto{\pgfqpoint{1.415733in}{0.587110in}}%
\pgfpathlineto{\pgfqpoint{1.427880in}{0.589993in}}%
\pgfpathlineto{\pgfqpoint{1.440027in}{0.592980in}}%
\pgfpathlineto{\pgfqpoint{1.452175in}{0.596075in}}%
\pgfpathlineto{\pgfqpoint{1.464322in}{0.599282in}}%
\pgfpathlineto{\pgfqpoint{1.476469in}{0.602608in}}%
\pgfpathlineto{\pgfqpoint{1.488617in}{0.606055in}}%
\pgfpathlineto{\pgfqpoint{1.500764in}{0.609630in}}%
\pgfpathlineto{\pgfqpoint{1.512911in}{0.613335in}}%
\pgfpathlineto{\pgfqpoint{1.525059in}{0.617177in}}%
\pgfpathlineto{\pgfqpoint{1.537206in}{0.621159in}}%
\pgfpathlineto{\pgfqpoint{1.549353in}{0.625287in}}%
\pgfpathlineto{\pgfqpoint{1.561501in}{0.629564in}}%
\pgfpathlineto{\pgfqpoint{1.573648in}{0.633996in}}%
\pgfpathlineto{\pgfqpoint{1.585795in}{0.638586in}}%
\pgfusepath{stroke}%
\end{pgfscope}%
\begin{pgfscope}%
\pgfpathrectangle{\pgfqpoint{0.687500in}{0.385000in}}{\pgfqpoint{4.262500in}{2.695000in}}%
\pgfusepath{clip}%
\pgfsetrectcap%
\pgfsetroundjoin%
\pgfsetlinewidth{1.505625pt}%
\definecolor{currentstroke}{rgb}{0.172549,0.627451,0.172549}%
\pgfsetstrokecolor{currentstroke}%
\pgfsetdash{}{0pt}%
\pgfpathmoveto{\pgfqpoint{1.585795in}{0.638586in}}%
\pgfpathlineto{\pgfqpoint{1.597943in}{0.643339in}}%
\pgfpathlineto{\pgfqpoint{1.610090in}{0.648251in}}%
\pgfpathlineto{\pgfqpoint{1.622237in}{0.653316in}}%
\pgfpathlineto{\pgfqpoint{1.634385in}{0.658530in}}%
\pgfpathlineto{\pgfqpoint{1.646532in}{0.663888in}}%
\pgfpathlineto{\pgfqpoint{1.658679in}{0.669385in}}%
\pgfpathlineto{\pgfqpoint{1.670827in}{0.675016in}}%
\pgfpathlineto{\pgfqpoint{1.682974in}{0.680777in}}%
\pgfpathlineto{\pgfqpoint{1.695121in}{0.686661in}}%
\pgfpathlineto{\pgfqpoint{1.707269in}{0.692665in}}%
\pgfpathlineto{\pgfqpoint{1.719416in}{0.698784in}}%
\pgfpathlineto{\pgfqpoint{1.731563in}{0.705011in}}%
\pgfpathlineto{\pgfqpoint{1.743711in}{0.711344in}}%
\pgfpathlineto{\pgfqpoint{1.755858in}{0.717775in}}%
\pgfpathlineto{\pgfqpoint{1.768005in}{0.724302in}}%
\pgfpathlineto{\pgfqpoint{1.780153in}{0.730918in}}%
\pgfpathlineto{\pgfqpoint{1.792300in}{0.737619in}}%
\pgfpathlineto{\pgfqpoint{1.804447in}{0.744400in}}%
\pgfpathlineto{\pgfqpoint{1.816595in}{0.751257in}}%
\pgfpathlineto{\pgfqpoint{1.828742in}{0.758183in}}%
\pgfpathlineto{\pgfqpoint{1.840889in}{0.765174in}}%
\pgfpathlineto{\pgfqpoint{1.853037in}{0.772225in}}%
\pgfpathlineto{\pgfqpoint{1.865184in}{0.779332in}}%
\pgfpathlineto{\pgfqpoint{1.877332in}{0.786489in}}%
\pgfpathlineto{\pgfqpoint{1.889479in}{0.793692in}}%
\pgfpathlineto{\pgfqpoint{1.901626in}{0.800935in}}%
\pgfpathlineto{\pgfqpoint{1.913774in}{0.808213in}}%
\pgfpathlineto{\pgfqpoint{1.925921in}{0.815522in}}%
\pgfpathlineto{\pgfqpoint{1.938068in}{0.822857in}}%
\pgfusepath{stroke}%
\end{pgfscope}%
\begin{pgfscope}%
\pgfpathrectangle{\pgfqpoint{0.687500in}{0.385000in}}{\pgfqpoint{4.262500in}{2.695000in}}%
\pgfusepath{clip}%
\pgfsetrectcap%
\pgfsetroundjoin%
\pgfsetlinewidth{1.505625pt}%
\definecolor{currentstroke}{rgb}{0.839216,0.152941,0.156863}%
\pgfsetstrokecolor{currentstroke}%
\pgfsetdash{}{0pt}%
\pgfpathmoveto{\pgfqpoint{1.938068in}{0.822857in}}%
\pgfpathlineto{\pgfqpoint{1.950216in}{0.830224in}}%
\pgfpathlineto{\pgfqpoint{1.962363in}{0.837675in}}%
\pgfpathlineto{\pgfqpoint{1.974510in}{0.845273in}}%
\pgfpathlineto{\pgfqpoint{1.986658in}{0.853080in}}%
\pgfpathlineto{\pgfqpoint{1.998805in}{0.861161in}}%
\pgfpathlineto{\pgfqpoint{2.010952in}{0.869577in}}%
\pgfpathlineto{\pgfqpoint{2.023100in}{0.878392in}}%
\pgfpathlineto{\pgfqpoint{2.035247in}{0.887669in}}%
\pgfpathlineto{\pgfqpoint{2.047394in}{0.897470in}}%
\pgfpathlineto{\pgfqpoint{2.059542in}{0.907860in}}%
\pgfpathlineto{\pgfqpoint{2.071689in}{0.918901in}}%
\pgfpathlineto{\pgfqpoint{2.083836in}{0.930655in}}%
\pgfpathlineto{\pgfqpoint{2.095984in}{0.943186in}}%
\pgfpathlineto{\pgfqpoint{2.108131in}{0.956557in}}%
\pgfpathlineto{\pgfqpoint{2.120278in}{0.970831in}}%
\pgfpathlineto{\pgfqpoint{2.132426in}{0.986071in}}%
\pgfpathlineto{\pgfqpoint{2.144573in}{1.002340in}}%
\pgfpathlineto{\pgfqpoint{2.156720in}{1.019700in}}%
\pgfpathlineto{\pgfqpoint{2.168868in}{1.038216in}}%
\pgfpathlineto{\pgfqpoint{2.181015in}{1.057950in}}%
\pgfpathlineto{\pgfqpoint{2.193162in}{1.078964in}}%
\pgfpathlineto{\pgfqpoint{2.205310in}{1.101322in}}%
\pgfpathlineto{\pgfqpoint{2.217457in}{1.125088in}}%
\pgfpathlineto{\pgfqpoint{2.229604in}{1.150323in}}%
\pgfpathlineto{\pgfqpoint{2.241752in}{1.177091in}}%
\pgfpathlineto{\pgfqpoint{2.253899in}{1.205456in}}%
\pgfpathlineto{\pgfqpoint{2.266046in}{1.235479in}}%
\pgfpathlineto{\pgfqpoint{2.278194in}{1.267224in}}%
\pgfpathlineto{\pgfqpoint{2.290341in}{1.300754in}}%
\pgfusepath{stroke}%
\end{pgfscope}%
\begin{pgfscope}%
\pgfpathrectangle{\pgfqpoint{0.687500in}{0.385000in}}{\pgfqpoint{4.262500in}{2.695000in}}%
\pgfusepath{clip}%
\pgfsetrectcap%
\pgfsetroundjoin%
\pgfsetlinewidth{1.505625pt}%
\definecolor{currentstroke}{rgb}{0.580392,0.403922,0.741176}%
\pgfsetstrokecolor{currentstroke}%
\pgfsetdash{}{0pt}%
\pgfpathmoveto{\pgfqpoint{2.290341in}{1.300754in}}%
\pgfpathlineto{\pgfqpoint{2.302488in}{1.336099in}}%
\pgfpathlineto{\pgfqpoint{2.314636in}{1.373155in}}%
\pgfpathlineto{\pgfqpoint{2.326783in}{1.411787in}}%
\pgfpathlineto{\pgfqpoint{2.338930in}{1.451856in}}%
\pgfpathlineto{\pgfqpoint{2.351078in}{1.493228in}}%
\pgfpathlineto{\pgfqpoint{2.363225in}{1.535764in}}%
\pgfpathlineto{\pgfqpoint{2.375372in}{1.579330in}}%
\pgfpathlineto{\pgfqpoint{2.387520in}{1.623787in}}%
\pgfpathlineto{\pgfqpoint{2.399667in}{1.669001in}}%
\pgfpathlineto{\pgfqpoint{2.411814in}{1.714833in}}%
\pgfpathlineto{\pgfqpoint{2.423962in}{1.761148in}}%
\pgfpathlineto{\pgfqpoint{2.436109in}{1.807809in}}%
\pgfpathlineto{\pgfqpoint{2.448256in}{1.854680in}}%
\pgfpathlineto{\pgfqpoint{2.460404in}{1.901623in}}%
\pgfpathlineto{\pgfqpoint{2.472551in}{1.948503in}}%
\pgfpathlineto{\pgfqpoint{2.484698in}{1.995183in}}%
\pgfpathlineto{\pgfqpoint{2.496846in}{2.041526in}}%
\pgfpathlineto{\pgfqpoint{2.508993in}{2.087396in}}%
\pgfpathlineto{\pgfqpoint{2.521140in}{2.132656in}}%
\pgfpathlineto{\pgfqpoint{2.533288in}{2.177170in}}%
\pgfpathlineto{\pgfqpoint{2.545435in}{2.220801in}}%
\pgfpathlineto{\pgfqpoint{2.557582in}{2.263413in}}%
\pgfpathlineto{\pgfqpoint{2.569730in}{2.304869in}}%
\pgfpathlineto{\pgfqpoint{2.581877in}{2.345032in}}%
\pgfpathlineto{\pgfqpoint{2.594024in}{2.383766in}}%
\pgfpathlineto{\pgfqpoint{2.606172in}{2.420935in}}%
\pgfpathlineto{\pgfqpoint{2.618319in}{2.456401in}}%
\pgfpathlineto{\pgfqpoint{2.630466in}{2.490029in}}%
\pgfpathlineto{\pgfqpoint{2.642614in}{2.521682in}}%
\pgfusepath{stroke}%
\end{pgfscope}%
\begin{pgfscope}%
\pgfpathrectangle{\pgfqpoint{0.687500in}{0.385000in}}{\pgfqpoint{4.262500in}{2.695000in}}%
\pgfusepath{clip}%
\pgfsetrectcap%
\pgfsetroundjoin%
\pgfsetlinewidth{1.505625pt}%
\definecolor{currentstroke}{rgb}{0.549020,0.337255,0.294118}%
\pgfsetstrokecolor{currentstroke}%
\pgfsetdash{}{0pt}%
\pgfpathmoveto{\pgfqpoint{2.642614in}{2.521682in}}%
\pgfpathlineto{\pgfqpoint{2.654761in}{2.551246in}}%
\pgfpathlineto{\pgfqpoint{2.666908in}{2.578698in}}%
\pgfpathlineto{\pgfqpoint{2.679056in}{2.604039in}}%
\pgfpathlineto{\pgfqpoint{2.691203in}{2.627268in}}%
\pgfpathlineto{\pgfqpoint{2.703350in}{2.648385in}}%
\pgfpathlineto{\pgfqpoint{2.715498in}{2.667390in}}%
\pgfpathlineto{\pgfqpoint{2.727645in}{2.684284in}}%
\pgfpathlineto{\pgfqpoint{2.739792in}{2.699066in}}%
\pgfpathlineto{\pgfqpoint{2.751940in}{2.711736in}}%
\pgfpathlineto{\pgfqpoint{2.764087in}{2.722295in}}%
\pgfpathlineto{\pgfqpoint{2.776234in}{2.730741in}}%
\pgfpathlineto{\pgfqpoint{2.788382in}{2.737076in}}%
\pgfpathlineto{\pgfqpoint{2.800529in}{2.741300in}}%
\pgfpathlineto{\pgfqpoint{2.812676in}{2.743412in}}%
\pgfpathlineto{\pgfqpoint{2.824824in}{2.743412in}}%
\pgfpathlineto{\pgfqpoint{2.836971in}{2.741300in}}%
\pgfpathlineto{\pgfqpoint{2.849118in}{2.737076in}}%
\pgfpathlineto{\pgfqpoint{2.861266in}{2.730741in}}%
\pgfpathlineto{\pgfqpoint{2.873413in}{2.722295in}}%
\pgfpathlineto{\pgfqpoint{2.885560in}{2.711736in}}%
\pgfpathlineto{\pgfqpoint{2.897708in}{2.699066in}}%
\pgfpathlineto{\pgfqpoint{2.909855in}{2.684284in}}%
\pgfpathlineto{\pgfqpoint{2.922002in}{2.667390in}}%
\pgfpathlineto{\pgfqpoint{2.934150in}{2.648385in}}%
\pgfpathlineto{\pgfqpoint{2.946297in}{2.627268in}}%
\pgfpathlineto{\pgfqpoint{2.958444in}{2.604039in}}%
\pgfpathlineto{\pgfqpoint{2.970592in}{2.578698in}}%
\pgfpathlineto{\pgfqpoint{2.982739in}{2.551246in}}%
\pgfpathlineto{\pgfqpoint{2.994886in}{2.521682in}}%
\pgfusepath{stroke}%
\end{pgfscope}%
\begin{pgfscope}%
\pgfpathrectangle{\pgfqpoint{0.687500in}{0.385000in}}{\pgfqpoint{4.262500in}{2.695000in}}%
\pgfusepath{clip}%
\pgfsetrectcap%
\pgfsetroundjoin%
\pgfsetlinewidth{1.505625pt}%
\definecolor{currentstroke}{rgb}{0.890196,0.466667,0.760784}%
\pgfsetstrokecolor{currentstroke}%
\pgfsetdash{}{0pt}%
\pgfpathmoveto{\pgfqpoint{2.994886in}{2.521682in}}%
\pgfpathlineto{\pgfqpoint{3.007034in}{2.490029in}}%
\pgfpathlineto{\pgfqpoint{3.019181in}{2.456401in}}%
\pgfpathlineto{\pgfqpoint{3.031328in}{2.420935in}}%
\pgfpathlineto{\pgfqpoint{3.043476in}{2.383766in}}%
\pgfpathlineto{\pgfqpoint{3.055623in}{2.345032in}}%
\pgfpathlineto{\pgfqpoint{3.067770in}{2.304869in}}%
\pgfpathlineto{\pgfqpoint{3.079918in}{2.263413in}}%
\pgfpathlineto{\pgfqpoint{3.092065in}{2.220801in}}%
\pgfpathlineto{\pgfqpoint{3.104212in}{2.177170in}}%
\pgfpathlineto{\pgfqpoint{3.116360in}{2.132656in}}%
\pgfpathlineto{\pgfqpoint{3.128507in}{2.087396in}}%
\pgfpathlineto{\pgfqpoint{3.140654in}{2.041526in}}%
\pgfpathlineto{\pgfqpoint{3.152802in}{1.995183in}}%
\pgfpathlineto{\pgfqpoint{3.164949in}{1.948503in}}%
\pgfpathlineto{\pgfqpoint{3.177096in}{1.901623in}}%
\pgfpathlineto{\pgfqpoint{3.189244in}{1.854680in}}%
\pgfpathlineto{\pgfqpoint{3.201391in}{1.807809in}}%
\pgfpathlineto{\pgfqpoint{3.213538in}{1.761148in}}%
\pgfpathlineto{\pgfqpoint{3.225686in}{1.714833in}}%
\pgfpathlineto{\pgfqpoint{3.237833in}{1.669001in}}%
\pgfpathlineto{\pgfqpoint{3.249980in}{1.623787in}}%
\pgfpathlineto{\pgfqpoint{3.262128in}{1.579330in}}%
\pgfpathlineto{\pgfqpoint{3.274275in}{1.535764in}}%
\pgfpathlineto{\pgfqpoint{3.286422in}{1.493228in}}%
\pgfpathlineto{\pgfqpoint{3.298570in}{1.451856in}}%
\pgfpathlineto{\pgfqpoint{3.310717in}{1.411787in}}%
\pgfpathlineto{\pgfqpoint{3.322864in}{1.373155in}}%
\pgfpathlineto{\pgfqpoint{3.335012in}{1.336099in}}%
\pgfpathlineto{\pgfqpoint{3.347159in}{1.300754in}}%
\pgfusepath{stroke}%
\end{pgfscope}%
\begin{pgfscope}%
\pgfpathrectangle{\pgfqpoint{0.687500in}{0.385000in}}{\pgfqpoint{4.262500in}{2.695000in}}%
\pgfusepath{clip}%
\pgfsetrectcap%
\pgfsetroundjoin%
\pgfsetlinewidth{1.505625pt}%
\definecolor{currentstroke}{rgb}{0.498039,0.498039,0.498039}%
\pgfsetstrokecolor{currentstroke}%
\pgfsetdash{}{0pt}%
\pgfpathmoveto{\pgfqpoint{3.347159in}{1.300754in}}%
\pgfpathlineto{\pgfqpoint{3.359306in}{1.267224in}}%
\pgfpathlineto{\pgfqpoint{3.371454in}{1.235479in}}%
\pgfpathlineto{\pgfqpoint{3.383601in}{1.205456in}}%
\pgfpathlineto{\pgfqpoint{3.395748in}{1.177091in}}%
\pgfpathlineto{\pgfqpoint{3.407896in}{1.150323in}}%
\pgfpathlineto{\pgfqpoint{3.420043in}{1.125088in}}%
\pgfpathlineto{\pgfqpoint{3.432190in}{1.101322in}}%
\pgfpathlineto{\pgfqpoint{3.444338in}{1.078964in}}%
\pgfpathlineto{\pgfqpoint{3.456485in}{1.057950in}}%
\pgfpathlineto{\pgfqpoint{3.468632in}{1.038216in}}%
\pgfpathlineto{\pgfqpoint{3.480780in}{1.019700in}}%
\pgfpathlineto{\pgfqpoint{3.492927in}{1.002340in}}%
\pgfpathlineto{\pgfqpoint{3.505074in}{0.986071in}}%
\pgfpathlineto{\pgfqpoint{3.517222in}{0.970831in}}%
\pgfpathlineto{\pgfqpoint{3.529369in}{0.956557in}}%
\pgfpathlineto{\pgfqpoint{3.541516in}{0.943186in}}%
\pgfpathlineto{\pgfqpoint{3.553664in}{0.930655in}}%
\pgfpathlineto{\pgfqpoint{3.565811in}{0.918901in}}%
\pgfpathlineto{\pgfqpoint{3.577958in}{0.907860in}}%
\pgfpathlineto{\pgfqpoint{3.590106in}{0.897470in}}%
\pgfpathlineto{\pgfqpoint{3.602253in}{0.887669in}}%
\pgfpathlineto{\pgfqpoint{3.614400in}{0.878392in}}%
\pgfpathlineto{\pgfqpoint{3.626548in}{0.869577in}}%
\pgfpathlineto{\pgfqpoint{3.638695in}{0.861161in}}%
\pgfpathlineto{\pgfqpoint{3.650842in}{0.853080in}}%
\pgfpathlineto{\pgfqpoint{3.662990in}{0.845273in}}%
\pgfpathlineto{\pgfqpoint{3.675137in}{0.837675in}}%
\pgfpathlineto{\pgfqpoint{3.687284in}{0.830224in}}%
\pgfpathlineto{\pgfqpoint{3.699432in}{0.822857in}}%
\pgfusepath{stroke}%
\end{pgfscope}%
\begin{pgfscope}%
\pgfpathrectangle{\pgfqpoint{0.687500in}{0.385000in}}{\pgfqpoint{4.262500in}{2.695000in}}%
\pgfusepath{clip}%
\pgfsetrectcap%
\pgfsetroundjoin%
\pgfsetlinewidth{1.505625pt}%
\definecolor{currentstroke}{rgb}{0.737255,0.741176,0.133333}%
\pgfsetstrokecolor{currentstroke}%
\pgfsetdash{}{0pt}%
\pgfpathmoveto{\pgfqpoint{3.699432in}{0.822857in}}%
\pgfpathlineto{\pgfqpoint{3.711579in}{0.815522in}}%
\pgfpathlineto{\pgfqpoint{3.723726in}{0.808213in}}%
\pgfpathlineto{\pgfqpoint{3.735874in}{0.800935in}}%
\pgfpathlineto{\pgfqpoint{3.748021in}{0.793692in}}%
\pgfpathlineto{\pgfqpoint{3.760168in}{0.786489in}}%
\pgfpathlineto{\pgfqpoint{3.772316in}{0.779332in}}%
\pgfpathlineto{\pgfqpoint{3.784463in}{0.772225in}}%
\pgfpathlineto{\pgfqpoint{3.796611in}{0.765174in}}%
\pgfpathlineto{\pgfqpoint{3.808758in}{0.758183in}}%
\pgfpathlineto{\pgfqpoint{3.820905in}{0.751257in}}%
\pgfpathlineto{\pgfqpoint{3.833053in}{0.744400in}}%
\pgfpathlineto{\pgfqpoint{3.845200in}{0.737619in}}%
\pgfpathlineto{\pgfqpoint{3.857347in}{0.730918in}}%
\pgfpathlineto{\pgfqpoint{3.869495in}{0.724302in}}%
\pgfpathlineto{\pgfqpoint{3.881642in}{0.717775in}}%
\pgfpathlineto{\pgfqpoint{3.893789in}{0.711344in}}%
\pgfpathlineto{\pgfqpoint{3.905937in}{0.705011in}}%
\pgfpathlineto{\pgfqpoint{3.918084in}{0.698784in}}%
\pgfpathlineto{\pgfqpoint{3.930231in}{0.692665in}}%
\pgfpathlineto{\pgfqpoint{3.942379in}{0.686661in}}%
\pgfpathlineto{\pgfqpoint{3.954526in}{0.680777in}}%
\pgfpathlineto{\pgfqpoint{3.966673in}{0.675016in}}%
\pgfpathlineto{\pgfqpoint{3.978821in}{0.669385in}}%
\pgfpathlineto{\pgfqpoint{3.990968in}{0.663888in}}%
\pgfpathlineto{\pgfqpoint{4.003115in}{0.658530in}}%
\pgfpathlineto{\pgfqpoint{4.015263in}{0.653316in}}%
\pgfpathlineto{\pgfqpoint{4.027410in}{0.648251in}}%
\pgfpathlineto{\pgfqpoint{4.039557in}{0.643339in}}%
\pgfpathlineto{\pgfqpoint{4.051705in}{0.638586in}}%
\pgfusepath{stroke}%
\end{pgfscope}%
\begin{pgfscope}%
\pgfpathrectangle{\pgfqpoint{0.687500in}{0.385000in}}{\pgfqpoint{4.262500in}{2.695000in}}%
\pgfusepath{clip}%
\pgfsetrectcap%
\pgfsetroundjoin%
\pgfsetlinewidth{1.505625pt}%
\definecolor{currentstroke}{rgb}{0.090196,0.745098,0.811765}%
\pgfsetstrokecolor{currentstroke}%
\pgfsetdash{}{0pt}%
\pgfpathmoveto{\pgfqpoint{4.051705in}{0.638586in}}%
\pgfpathlineto{\pgfqpoint{4.063852in}{0.633996in}}%
\pgfpathlineto{\pgfqpoint{4.075999in}{0.629564in}}%
\pgfpathlineto{\pgfqpoint{4.088147in}{0.625287in}}%
\pgfpathlineto{\pgfqpoint{4.100294in}{0.621159in}}%
\pgfpathlineto{\pgfqpoint{4.112441in}{0.617177in}}%
\pgfpathlineto{\pgfqpoint{4.124589in}{0.613335in}}%
\pgfpathlineto{\pgfqpoint{4.136736in}{0.609630in}}%
\pgfpathlineto{\pgfqpoint{4.148883in}{0.606055in}}%
\pgfpathlineto{\pgfqpoint{4.161031in}{0.602608in}}%
\pgfpathlineto{\pgfqpoint{4.173178in}{0.599282in}}%
\pgfpathlineto{\pgfqpoint{4.185325in}{0.596075in}}%
\pgfpathlineto{\pgfqpoint{4.197473in}{0.592980in}}%
\pgfpathlineto{\pgfqpoint{4.209620in}{0.589993in}}%
\pgfpathlineto{\pgfqpoint{4.221767in}{0.587110in}}%
\pgfpathlineto{\pgfqpoint{4.233915in}{0.584326in}}%
\pgfpathlineto{\pgfqpoint{4.246062in}{0.581637in}}%
\pgfpathlineto{\pgfqpoint{4.258209in}{0.579038in}}%
\pgfpathlineto{\pgfqpoint{4.270357in}{0.576523in}}%
\pgfpathlineto{\pgfqpoint{4.282504in}{0.574090in}}%
\pgfpathlineto{\pgfqpoint{4.294651in}{0.571732in}}%
\pgfpathlineto{\pgfqpoint{4.306799in}{0.569446in}}%
\pgfpathlineto{\pgfqpoint{4.318946in}{0.567227in}}%
\pgfpathlineto{\pgfqpoint{4.331093in}{0.565070in}}%
\pgfpathlineto{\pgfqpoint{4.343241in}{0.562971in}}%
\pgfpathlineto{\pgfqpoint{4.355388in}{0.560925in}}%
\pgfpathlineto{\pgfqpoint{4.367535in}{0.558927in}}%
\pgfpathlineto{\pgfqpoint{4.379683in}{0.556973in}}%
\pgfpathlineto{\pgfqpoint{4.391830in}{0.555058in}}%
\pgfpathlineto{\pgfqpoint{4.403977in}{0.553177in}}%
\pgfusepath{stroke}%
\end{pgfscope}%
\begin{pgfscope}%
\pgfpathrectangle{\pgfqpoint{0.687500in}{0.385000in}}{\pgfqpoint{4.262500in}{2.695000in}}%
\pgfusepath{clip}%
\pgfsetrectcap%
\pgfsetroundjoin%
\pgfsetlinewidth{1.505625pt}%
\definecolor{currentstroke}{rgb}{0.121569,0.466667,0.705882}%
\pgfsetstrokecolor{currentstroke}%
\pgfsetdash{}{0pt}%
\pgfpathmoveto{\pgfqpoint{4.403977in}{0.553177in}}%
\pgfpathlineto{\pgfqpoint{4.416125in}{0.551327in}}%
\pgfpathlineto{\pgfqpoint{4.428272in}{0.549507in}}%
\pgfpathlineto{\pgfqpoint{4.440419in}{0.547714in}}%
\pgfpathlineto{\pgfqpoint{4.452567in}{0.545947in}}%
\pgfpathlineto{\pgfqpoint{4.464714in}{0.544207in}}%
\pgfpathlineto{\pgfqpoint{4.476861in}{0.542491in}}%
\pgfpathlineto{\pgfqpoint{4.489009in}{0.540800in}}%
\pgfpathlineto{\pgfqpoint{4.501156in}{0.539131in}}%
\pgfpathlineto{\pgfqpoint{4.513303in}{0.537483in}}%
\pgfpathlineto{\pgfqpoint{4.525451in}{0.535856in}}%
\pgfpathlineto{\pgfqpoint{4.537598in}{0.534249in}}%
\pgfpathlineto{\pgfqpoint{4.549745in}{0.532661in}}%
\pgfpathlineto{\pgfqpoint{4.561893in}{0.531090in}}%
\pgfpathlineto{\pgfqpoint{4.574040in}{0.529535in}}%
\pgfpathlineto{\pgfqpoint{4.586187in}{0.527997in}}%
\pgfpathlineto{\pgfqpoint{4.598335in}{0.526472in}}%
\pgfpathlineto{\pgfqpoint{4.610482in}{0.524961in}}%
\pgfpathlineto{\pgfqpoint{4.622629in}{0.523462in}}%
\pgfpathlineto{\pgfqpoint{4.634777in}{0.521975in}}%
\pgfpathlineto{\pgfqpoint{4.646924in}{0.520498in}}%
\pgfpathlineto{\pgfqpoint{4.659071in}{0.519031in}}%
\pgfpathlineto{\pgfqpoint{4.671219in}{0.517571in}}%
\pgfpathlineto{\pgfqpoint{4.683366in}{0.516119in}}%
\pgfpathlineto{\pgfqpoint{4.695513in}{0.514673in}}%
\pgfpathlineto{\pgfqpoint{4.707661in}{0.513232in}}%
\pgfpathlineto{\pgfqpoint{4.719808in}{0.511796in}}%
\pgfpathlineto{\pgfqpoint{4.731955in}{0.510362in}}%
\pgfpathlineto{\pgfqpoint{4.744103in}{0.508930in}}%
\pgfpathlineto{\pgfqpoint{4.756250in}{0.507500in}}%
\pgfusepath{stroke}%
\end{pgfscope}%
\begin{pgfscope}%
\pgfsetrectcap%
\pgfsetmiterjoin%
\pgfsetlinewidth{0.803000pt}%
\definecolor{currentstroke}{rgb}{0.000000,0.000000,0.000000}%
\pgfsetstrokecolor{currentstroke}%
\pgfsetdash{}{0pt}%
\pgfpathmoveto{\pgfqpoint{0.687500in}{0.385000in}}%
\pgfpathlineto{\pgfqpoint{0.687500in}{3.080000in}}%
\pgfusepath{stroke}%
\end{pgfscope}%
\begin{pgfscope}%
\pgfsetrectcap%
\pgfsetmiterjoin%
\pgfsetlinewidth{0.803000pt}%
\definecolor{currentstroke}{rgb}{0.000000,0.000000,0.000000}%
\pgfsetstrokecolor{currentstroke}%
\pgfsetdash{}{0pt}%
\pgfpathmoveto{\pgfqpoint{4.950000in}{0.385000in}}%
\pgfpathlineto{\pgfqpoint{4.950000in}{3.080000in}}%
\pgfusepath{stroke}%
\end{pgfscope}%
\begin{pgfscope}%
\pgfsetrectcap%
\pgfsetmiterjoin%
\pgfsetlinewidth{0.803000pt}%
\definecolor{currentstroke}{rgb}{0.000000,0.000000,0.000000}%
\pgfsetstrokecolor{currentstroke}%
\pgfsetdash{}{0pt}%
\pgfpathmoveto{\pgfqpoint{0.687500in}{0.385000in}}%
\pgfpathlineto{\pgfqpoint{4.950000in}{0.385000in}}%
\pgfusepath{stroke}%
\end{pgfscope}%
\begin{pgfscope}%
\pgfsetrectcap%
\pgfsetmiterjoin%
\pgfsetlinewidth{0.803000pt}%
\definecolor{currentstroke}{rgb}{0.000000,0.000000,0.000000}%
\pgfsetstrokecolor{currentstroke}%
\pgfsetdash{}{0pt}%
\pgfpathmoveto{\pgfqpoint{0.687500in}{3.080000in}}%
\pgfpathlineto{\pgfqpoint{4.950000in}{3.080000in}}%
\pgfusepath{stroke}%
\end{pgfscope}%
\begin{pgfscope}%
\definecolor{textcolor}{rgb}{0.000000,0.000000,0.000000}%
\pgfsetstrokecolor{textcolor}%
\pgfsetfillcolor{textcolor}%
\pgftext[x=2.818750in,y=3.163333in,,base]{\color{textcolor}\rmfamily\fontsize{12.000000}{14.400000}\selectfont N = 11}%
\end{pgfscope}%
\begin{pgfscope}%
\definecolor{textcolor}{rgb}{0.000000,0.000000,0.000000}%
\pgfsetstrokecolor{textcolor}%
\pgfsetfillcolor{textcolor}%
\pgftext[x=2.750000in,y=3.430000in,,top]{\color{textcolor}\rmfamily\fontsize{12.000000}{14.400000}\selectfont Naturalny splajn kubiczny}%
\end{pgfscope}%
\end{pgfpicture}%
\makeatother%
\endgroup%
        
    \end{center}
\end{figure}

Ze względu na symetryczność funkcji w przedziale, parzyste \(N\) wyznacza węzeł
w punkcie \(x=0\), To powoduje, że splajn przechodzi przez maksimum funkcji 
\(f\). Nieparzyste \(N\), w tym przypadku \(N=11\) omija ten charakterystyczny
dla funkcji punkt. W związku z tym, dla funkcji takich jak \(f\) lepiej
skorzystać z \(N\) parzystego.




\end{document}