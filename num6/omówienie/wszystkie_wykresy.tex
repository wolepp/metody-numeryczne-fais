\documentclass[a4paper,11pt]{article}
\usepackage{graphicx}
\usepackage[]{latexsym}
\usepackage[]{amsmath}
\usepackage[]{pgf}
\usepackage[]{float}
\usepackage[MeX]{polski}
\usepackage[utf8]{inputenc}
\usepackage[figurename=Wykres]{caption}

\author{Wojciech Lepich \and Nomen Nescio}
\title{NUM6. Wykresy}
\begin{document}

\renewcommand*\listfigurename{Spis wykresów}
\maketitle
\listoffigures

% ------------------------------------------------------------------------------

\begin{figure}[h]
    \begin{center}
        %% Creator: Matplotlib, PGF backend
%%
%% To include the figure in your LaTeX document, write
%%   \input{<filename>.pgf}
%%
%% Make sure the required packages are loaded in your preamble
%%   \usepackage{pgf}
%%
%% Figures using additional raster images can only be included by \input if
%% they are in the same directory as the main LaTeX file. For loading figures
%% from other directories you can use the `import` package
%%   \usepackage{import}
%% and then include the figures with
%%   \import{<path to file>}{<filename>.pgf}
%%
%% Matplotlib used the following preamble
%%
\begingroup%
\makeatletter%
\begin{pgfpicture}%
\pgfpathrectangle{\pgfpointorigin}{\pgfqpoint{5.500000in}{3.500000in}}%
\pgfusepath{use as bounding box, clip}%
\begin{pgfscope}%
\pgfsetbuttcap%
\pgfsetmiterjoin%
\definecolor{currentfill}{rgb}{1.000000,1.000000,1.000000}%
\pgfsetfillcolor{currentfill}%
\pgfsetlinewidth{0.000000pt}%
\definecolor{currentstroke}{rgb}{1.000000,1.000000,1.000000}%
\pgfsetstrokecolor{currentstroke}%
\pgfsetdash{}{0pt}%
\pgfpathmoveto{\pgfqpoint{0.000000in}{0.000000in}}%
\pgfpathlineto{\pgfqpoint{5.500000in}{0.000000in}}%
\pgfpathlineto{\pgfqpoint{5.500000in}{3.500000in}}%
\pgfpathlineto{\pgfqpoint{0.000000in}{3.500000in}}%
\pgfpathclose%
\pgfusepath{fill}%
\end{pgfscope}%
\begin{pgfscope}%
\pgfsetbuttcap%
\pgfsetmiterjoin%
\definecolor{currentfill}{rgb}{1.000000,1.000000,1.000000}%
\pgfsetfillcolor{currentfill}%
\pgfsetlinewidth{0.000000pt}%
\definecolor{currentstroke}{rgb}{0.000000,0.000000,0.000000}%
\pgfsetstrokecolor{currentstroke}%
\pgfsetstrokeopacity{0.000000}%
\pgfsetdash{}{0pt}%
\pgfpathmoveto{\pgfqpoint{0.687500in}{0.385000in}}%
\pgfpathlineto{\pgfqpoint{4.950000in}{0.385000in}}%
\pgfpathlineto{\pgfqpoint{4.950000in}{3.080000in}}%
\pgfpathlineto{\pgfqpoint{0.687500in}{3.080000in}}%
\pgfpathclose%
\pgfusepath{fill}%
\end{pgfscope}%
\begin{pgfscope}%
\pgfsetbuttcap%
\pgfsetroundjoin%
\definecolor{currentfill}{rgb}{0.000000,0.000000,0.000000}%
\pgfsetfillcolor{currentfill}%
\pgfsetlinewidth{0.803000pt}%
\definecolor{currentstroke}{rgb}{0.000000,0.000000,0.000000}%
\pgfsetstrokecolor{currentstroke}%
\pgfsetdash{}{0pt}%
\pgfsys@defobject{currentmarker}{\pgfqpoint{0.000000in}{-0.048611in}}{\pgfqpoint{0.000000in}{0.000000in}}{%
\pgfpathmoveto{\pgfqpoint{0.000000in}{0.000000in}}%
\pgfpathlineto{\pgfqpoint{0.000000in}{-0.048611in}}%
\pgfusepath{stroke,fill}%
}%
\begin{pgfscope}%
\pgfsys@transformshift{0.881250in}{0.385000in}%
\pgfsys@useobject{currentmarker}{}%
\end{pgfscope}%
\end{pgfscope}%
\begin{pgfscope}%
\definecolor{textcolor}{rgb}{0.000000,0.000000,0.000000}%
\pgfsetstrokecolor{textcolor}%
\pgfsetfillcolor{textcolor}%
\pgftext[x=0.881250in,y=0.287778in,,top]{\color{textcolor}\rmfamily\fontsize{10.000000}{12.000000}\selectfont \(\displaystyle -1.00\)}%
\end{pgfscope}%
\begin{pgfscope}%
\pgfsetbuttcap%
\pgfsetroundjoin%
\definecolor{currentfill}{rgb}{0.000000,0.000000,0.000000}%
\pgfsetfillcolor{currentfill}%
\pgfsetlinewidth{0.803000pt}%
\definecolor{currentstroke}{rgb}{0.000000,0.000000,0.000000}%
\pgfsetstrokecolor{currentstroke}%
\pgfsetdash{}{0pt}%
\pgfsys@defobject{currentmarker}{\pgfqpoint{0.000000in}{-0.048611in}}{\pgfqpoint{0.000000in}{0.000000in}}{%
\pgfpathmoveto{\pgfqpoint{0.000000in}{0.000000in}}%
\pgfpathlineto{\pgfqpoint{0.000000in}{-0.048611in}}%
\pgfusepath{stroke,fill}%
}%
\begin{pgfscope}%
\pgfsys@transformshift{1.365625in}{0.385000in}%
\pgfsys@useobject{currentmarker}{}%
\end{pgfscope}%
\end{pgfscope}%
\begin{pgfscope}%
\definecolor{textcolor}{rgb}{0.000000,0.000000,0.000000}%
\pgfsetstrokecolor{textcolor}%
\pgfsetfillcolor{textcolor}%
\pgftext[x=1.365625in,y=0.287778in,,top]{\color{textcolor}\rmfamily\fontsize{10.000000}{12.000000}\selectfont \(\displaystyle -0.75\)}%
\end{pgfscope}%
\begin{pgfscope}%
\pgfsetbuttcap%
\pgfsetroundjoin%
\definecolor{currentfill}{rgb}{0.000000,0.000000,0.000000}%
\pgfsetfillcolor{currentfill}%
\pgfsetlinewidth{0.803000pt}%
\definecolor{currentstroke}{rgb}{0.000000,0.000000,0.000000}%
\pgfsetstrokecolor{currentstroke}%
\pgfsetdash{}{0pt}%
\pgfsys@defobject{currentmarker}{\pgfqpoint{0.000000in}{-0.048611in}}{\pgfqpoint{0.000000in}{0.000000in}}{%
\pgfpathmoveto{\pgfqpoint{0.000000in}{0.000000in}}%
\pgfpathlineto{\pgfqpoint{0.000000in}{-0.048611in}}%
\pgfusepath{stroke,fill}%
}%
\begin{pgfscope}%
\pgfsys@transformshift{1.850000in}{0.385000in}%
\pgfsys@useobject{currentmarker}{}%
\end{pgfscope}%
\end{pgfscope}%
\begin{pgfscope}%
\definecolor{textcolor}{rgb}{0.000000,0.000000,0.000000}%
\pgfsetstrokecolor{textcolor}%
\pgfsetfillcolor{textcolor}%
\pgftext[x=1.850000in,y=0.287778in,,top]{\color{textcolor}\rmfamily\fontsize{10.000000}{12.000000}\selectfont \(\displaystyle -0.50\)}%
\end{pgfscope}%
\begin{pgfscope}%
\pgfsetbuttcap%
\pgfsetroundjoin%
\definecolor{currentfill}{rgb}{0.000000,0.000000,0.000000}%
\pgfsetfillcolor{currentfill}%
\pgfsetlinewidth{0.803000pt}%
\definecolor{currentstroke}{rgb}{0.000000,0.000000,0.000000}%
\pgfsetstrokecolor{currentstroke}%
\pgfsetdash{}{0pt}%
\pgfsys@defobject{currentmarker}{\pgfqpoint{0.000000in}{-0.048611in}}{\pgfqpoint{0.000000in}{0.000000in}}{%
\pgfpathmoveto{\pgfqpoint{0.000000in}{0.000000in}}%
\pgfpathlineto{\pgfqpoint{0.000000in}{-0.048611in}}%
\pgfusepath{stroke,fill}%
}%
\begin{pgfscope}%
\pgfsys@transformshift{2.334375in}{0.385000in}%
\pgfsys@useobject{currentmarker}{}%
\end{pgfscope}%
\end{pgfscope}%
\begin{pgfscope}%
\definecolor{textcolor}{rgb}{0.000000,0.000000,0.000000}%
\pgfsetstrokecolor{textcolor}%
\pgfsetfillcolor{textcolor}%
\pgftext[x=2.334375in,y=0.287778in,,top]{\color{textcolor}\rmfamily\fontsize{10.000000}{12.000000}\selectfont \(\displaystyle -0.25\)}%
\end{pgfscope}%
\begin{pgfscope}%
\pgfsetbuttcap%
\pgfsetroundjoin%
\definecolor{currentfill}{rgb}{0.000000,0.000000,0.000000}%
\pgfsetfillcolor{currentfill}%
\pgfsetlinewidth{0.803000pt}%
\definecolor{currentstroke}{rgb}{0.000000,0.000000,0.000000}%
\pgfsetstrokecolor{currentstroke}%
\pgfsetdash{}{0pt}%
\pgfsys@defobject{currentmarker}{\pgfqpoint{0.000000in}{-0.048611in}}{\pgfqpoint{0.000000in}{0.000000in}}{%
\pgfpathmoveto{\pgfqpoint{0.000000in}{0.000000in}}%
\pgfpathlineto{\pgfqpoint{0.000000in}{-0.048611in}}%
\pgfusepath{stroke,fill}%
}%
\begin{pgfscope}%
\pgfsys@transformshift{2.818750in}{0.385000in}%
\pgfsys@useobject{currentmarker}{}%
\end{pgfscope}%
\end{pgfscope}%
\begin{pgfscope}%
\definecolor{textcolor}{rgb}{0.000000,0.000000,0.000000}%
\pgfsetstrokecolor{textcolor}%
\pgfsetfillcolor{textcolor}%
\pgftext[x=2.818750in,y=0.287778in,,top]{\color{textcolor}\rmfamily\fontsize{10.000000}{12.000000}\selectfont \(\displaystyle 0.00\)}%
\end{pgfscope}%
\begin{pgfscope}%
\pgfsetbuttcap%
\pgfsetroundjoin%
\definecolor{currentfill}{rgb}{0.000000,0.000000,0.000000}%
\pgfsetfillcolor{currentfill}%
\pgfsetlinewidth{0.803000pt}%
\definecolor{currentstroke}{rgb}{0.000000,0.000000,0.000000}%
\pgfsetstrokecolor{currentstroke}%
\pgfsetdash{}{0pt}%
\pgfsys@defobject{currentmarker}{\pgfqpoint{0.000000in}{-0.048611in}}{\pgfqpoint{0.000000in}{0.000000in}}{%
\pgfpathmoveto{\pgfqpoint{0.000000in}{0.000000in}}%
\pgfpathlineto{\pgfqpoint{0.000000in}{-0.048611in}}%
\pgfusepath{stroke,fill}%
}%
\begin{pgfscope}%
\pgfsys@transformshift{3.303125in}{0.385000in}%
\pgfsys@useobject{currentmarker}{}%
\end{pgfscope}%
\end{pgfscope}%
\begin{pgfscope}%
\definecolor{textcolor}{rgb}{0.000000,0.000000,0.000000}%
\pgfsetstrokecolor{textcolor}%
\pgfsetfillcolor{textcolor}%
\pgftext[x=3.303125in,y=0.287778in,,top]{\color{textcolor}\rmfamily\fontsize{10.000000}{12.000000}\selectfont \(\displaystyle 0.25\)}%
\end{pgfscope}%
\begin{pgfscope}%
\pgfsetbuttcap%
\pgfsetroundjoin%
\definecolor{currentfill}{rgb}{0.000000,0.000000,0.000000}%
\pgfsetfillcolor{currentfill}%
\pgfsetlinewidth{0.803000pt}%
\definecolor{currentstroke}{rgb}{0.000000,0.000000,0.000000}%
\pgfsetstrokecolor{currentstroke}%
\pgfsetdash{}{0pt}%
\pgfsys@defobject{currentmarker}{\pgfqpoint{0.000000in}{-0.048611in}}{\pgfqpoint{0.000000in}{0.000000in}}{%
\pgfpathmoveto{\pgfqpoint{0.000000in}{0.000000in}}%
\pgfpathlineto{\pgfqpoint{0.000000in}{-0.048611in}}%
\pgfusepath{stroke,fill}%
}%
\begin{pgfscope}%
\pgfsys@transformshift{3.787500in}{0.385000in}%
\pgfsys@useobject{currentmarker}{}%
\end{pgfscope}%
\end{pgfscope}%
\begin{pgfscope}%
\definecolor{textcolor}{rgb}{0.000000,0.000000,0.000000}%
\pgfsetstrokecolor{textcolor}%
\pgfsetfillcolor{textcolor}%
\pgftext[x=3.787500in,y=0.287778in,,top]{\color{textcolor}\rmfamily\fontsize{10.000000}{12.000000}\selectfont \(\displaystyle 0.50\)}%
\end{pgfscope}%
\begin{pgfscope}%
\pgfsetbuttcap%
\pgfsetroundjoin%
\definecolor{currentfill}{rgb}{0.000000,0.000000,0.000000}%
\pgfsetfillcolor{currentfill}%
\pgfsetlinewidth{0.803000pt}%
\definecolor{currentstroke}{rgb}{0.000000,0.000000,0.000000}%
\pgfsetstrokecolor{currentstroke}%
\pgfsetdash{}{0pt}%
\pgfsys@defobject{currentmarker}{\pgfqpoint{0.000000in}{-0.048611in}}{\pgfqpoint{0.000000in}{0.000000in}}{%
\pgfpathmoveto{\pgfqpoint{0.000000in}{0.000000in}}%
\pgfpathlineto{\pgfqpoint{0.000000in}{-0.048611in}}%
\pgfusepath{stroke,fill}%
}%
\begin{pgfscope}%
\pgfsys@transformshift{4.271875in}{0.385000in}%
\pgfsys@useobject{currentmarker}{}%
\end{pgfscope}%
\end{pgfscope}%
\begin{pgfscope}%
\definecolor{textcolor}{rgb}{0.000000,0.000000,0.000000}%
\pgfsetstrokecolor{textcolor}%
\pgfsetfillcolor{textcolor}%
\pgftext[x=4.271875in,y=0.287778in,,top]{\color{textcolor}\rmfamily\fontsize{10.000000}{12.000000}\selectfont \(\displaystyle 0.75\)}%
\end{pgfscope}%
\begin{pgfscope}%
\pgfsetbuttcap%
\pgfsetroundjoin%
\definecolor{currentfill}{rgb}{0.000000,0.000000,0.000000}%
\pgfsetfillcolor{currentfill}%
\pgfsetlinewidth{0.803000pt}%
\definecolor{currentstroke}{rgb}{0.000000,0.000000,0.000000}%
\pgfsetstrokecolor{currentstroke}%
\pgfsetdash{}{0pt}%
\pgfsys@defobject{currentmarker}{\pgfqpoint{0.000000in}{-0.048611in}}{\pgfqpoint{0.000000in}{0.000000in}}{%
\pgfpathmoveto{\pgfqpoint{0.000000in}{0.000000in}}%
\pgfpathlineto{\pgfqpoint{0.000000in}{-0.048611in}}%
\pgfusepath{stroke,fill}%
}%
\begin{pgfscope}%
\pgfsys@transformshift{4.756250in}{0.385000in}%
\pgfsys@useobject{currentmarker}{}%
\end{pgfscope}%
\end{pgfscope}%
\begin{pgfscope}%
\definecolor{textcolor}{rgb}{0.000000,0.000000,0.000000}%
\pgfsetstrokecolor{textcolor}%
\pgfsetfillcolor{textcolor}%
\pgftext[x=4.756250in,y=0.287778in,,top]{\color{textcolor}\rmfamily\fontsize{10.000000}{12.000000}\selectfont \(\displaystyle 1.00\)}%
\end{pgfscope}%
\begin{pgfscope}%
\definecolor{textcolor}{rgb}{0.000000,0.000000,0.000000}%
\pgfsetstrokecolor{textcolor}%
\pgfsetfillcolor{textcolor}%
\pgftext[x=2.818750in,y=0.108766in,,top]{\color{textcolor}\rmfamily\fontsize{10.000000}{12.000000}\selectfont x}%
\end{pgfscope}%
\begin{pgfscope}%
\pgfsetbuttcap%
\pgfsetroundjoin%
\definecolor{currentfill}{rgb}{0.000000,0.000000,0.000000}%
\pgfsetfillcolor{currentfill}%
\pgfsetlinewidth{0.803000pt}%
\definecolor{currentstroke}{rgb}{0.000000,0.000000,0.000000}%
\pgfsetstrokecolor{currentstroke}%
\pgfsetdash{}{0pt}%
\pgfsys@defobject{currentmarker}{\pgfqpoint{-0.048611in}{0.000000in}}{\pgfqpoint{0.000000in}{0.000000in}}{%
\pgfpathmoveto{\pgfqpoint{0.000000in}{0.000000in}}%
\pgfpathlineto{\pgfqpoint{-0.048611in}{0.000000in}}%
\pgfusepath{stroke,fill}%
}%
\begin{pgfscope}%
\pgfsys@transformshift{0.687500in}{0.474833in}%
\pgfsys@useobject{currentmarker}{}%
\end{pgfscope}%
\end{pgfscope}%
\begin{pgfscope}%
\definecolor{textcolor}{rgb}{0.000000,0.000000,0.000000}%
\pgfsetstrokecolor{textcolor}%
\pgfsetfillcolor{textcolor}%
\pgftext[x=0.304783in,y=0.426608in,left,base]{\color{textcolor}\rmfamily\fontsize{10.000000}{12.000000}\selectfont \(\displaystyle -0.2\)}%
\end{pgfscope}%
\begin{pgfscope}%
\pgfsetbuttcap%
\pgfsetroundjoin%
\definecolor{currentfill}{rgb}{0.000000,0.000000,0.000000}%
\pgfsetfillcolor{currentfill}%
\pgfsetlinewidth{0.803000pt}%
\definecolor{currentstroke}{rgb}{0.000000,0.000000,0.000000}%
\pgfsetstrokecolor{currentstroke}%
\pgfsetdash{}{0pt}%
\pgfsys@defobject{currentmarker}{\pgfqpoint{-0.048611in}{0.000000in}}{\pgfqpoint{0.000000in}{0.000000in}}{%
\pgfpathmoveto{\pgfqpoint{0.000000in}{0.000000in}}%
\pgfpathlineto{\pgfqpoint{-0.048611in}{0.000000in}}%
\pgfusepath{stroke,fill}%
}%
\begin{pgfscope}%
\pgfsys@transformshift{0.687500in}{0.834167in}%
\pgfsys@useobject{currentmarker}{}%
\end{pgfscope}%
\end{pgfscope}%
\begin{pgfscope}%
\definecolor{textcolor}{rgb}{0.000000,0.000000,0.000000}%
\pgfsetstrokecolor{textcolor}%
\pgfsetfillcolor{textcolor}%
\pgftext[x=0.412808in,y=0.785941in,left,base]{\color{textcolor}\rmfamily\fontsize{10.000000}{12.000000}\selectfont \(\displaystyle 0.0\)}%
\end{pgfscope}%
\begin{pgfscope}%
\pgfsetbuttcap%
\pgfsetroundjoin%
\definecolor{currentfill}{rgb}{0.000000,0.000000,0.000000}%
\pgfsetfillcolor{currentfill}%
\pgfsetlinewidth{0.803000pt}%
\definecolor{currentstroke}{rgb}{0.000000,0.000000,0.000000}%
\pgfsetstrokecolor{currentstroke}%
\pgfsetdash{}{0pt}%
\pgfsys@defobject{currentmarker}{\pgfqpoint{-0.048611in}{0.000000in}}{\pgfqpoint{0.000000in}{0.000000in}}{%
\pgfpathmoveto{\pgfqpoint{0.000000in}{0.000000in}}%
\pgfpathlineto{\pgfqpoint{-0.048611in}{0.000000in}}%
\pgfusepath{stroke,fill}%
}%
\begin{pgfscope}%
\pgfsys@transformshift{0.687500in}{1.193500in}%
\pgfsys@useobject{currentmarker}{}%
\end{pgfscope}%
\end{pgfscope}%
\begin{pgfscope}%
\definecolor{textcolor}{rgb}{0.000000,0.000000,0.000000}%
\pgfsetstrokecolor{textcolor}%
\pgfsetfillcolor{textcolor}%
\pgftext[x=0.412808in,y=1.145275in,left,base]{\color{textcolor}\rmfamily\fontsize{10.000000}{12.000000}\selectfont \(\displaystyle 0.2\)}%
\end{pgfscope}%
\begin{pgfscope}%
\pgfsetbuttcap%
\pgfsetroundjoin%
\definecolor{currentfill}{rgb}{0.000000,0.000000,0.000000}%
\pgfsetfillcolor{currentfill}%
\pgfsetlinewidth{0.803000pt}%
\definecolor{currentstroke}{rgb}{0.000000,0.000000,0.000000}%
\pgfsetstrokecolor{currentstroke}%
\pgfsetdash{}{0pt}%
\pgfsys@defobject{currentmarker}{\pgfqpoint{-0.048611in}{0.000000in}}{\pgfqpoint{0.000000in}{0.000000in}}{%
\pgfpathmoveto{\pgfqpoint{0.000000in}{0.000000in}}%
\pgfpathlineto{\pgfqpoint{-0.048611in}{0.000000in}}%
\pgfusepath{stroke,fill}%
}%
\begin{pgfscope}%
\pgfsys@transformshift{0.687500in}{1.552833in}%
\pgfsys@useobject{currentmarker}{}%
\end{pgfscope}%
\end{pgfscope}%
\begin{pgfscope}%
\definecolor{textcolor}{rgb}{0.000000,0.000000,0.000000}%
\pgfsetstrokecolor{textcolor}%
\pgfsetfillcolor{textcolor}%
\pgftext[x=0.412808in,y=1.504608in,left,base]{\color{textcolor}\rmfamily\fontsize{10.000000}{12.000000}\selectfont \(\displaystyle 0.4\)}%
\end{pgfscope}%
\begin{pgfscope}%
\pgfsetbuttcap%
\pgfsetroundjoin%
\definecolor{currentfill}{rgb}{0.000000,0.000000,0.000000}%
\pgfsetfillcolor{currentfill}%
\pgfsetlinewidth{0.803000pt}%
\definecolor{currentstroke}{rgb}{0.000000,0.000000,0.000000}%
\pgfsetstrokecolor{currentstroke}%
\pgfsetdash{}{0pt}%
\pgfsys@defobject{currentmarker}{\pgfqpoint{-0.048611in}{0.000000in}}{\pgfqpoint{0.000000in}{0.000000in}}{%
\pgfpathmoveto{\pgfqpoint{0.000000in}{0.000000in}}%
\pgfpathlineto{\pgfqpoint{-0.048611in}{0.000000in}}%
\pgfusepath{stroke,fill}%
}%
\begin{pgfscope}%
\pgfsys@transformshift{0.687500in}{1.912167in}%
\pgfsys@useobject{currentmarker}{}%
\end{pgfscope}%
\end{pgfscope}%
\begin{pgfscope}%
\definecolor{textcolor}{rgb}{0.000000,0.000000,0.000000}%
\pgfsetstrokecolor{textcolor}%
\pgfsetfillcolor{textcolor}%
\pgftext[x=0.412808in,y=1.863941in,left,base]{\color{textcolor}\rmfamily\fontsize{10.000000}{12.000000}\selectfont \(\displaystyle 0.6\)}%
\end{pgfscope}%
\begin{pgfscope}%
\pgfsetbuttcap%
\pgfsetroundjoin%
\definecolor{currentfill}{rgb}{0.000000,0.000000,0.000000}%
\pgfsetfillcolor{currentfill}%
\pgfsetlinewidth{0.803000pt}%
\definecolor{currentstroke}{rgb}{0.000000,0.000000,0.000000}%
\pgfsetstrokecolor{currentstroke}%
\pgfsetdash{}{0pt}%
\pgfsys@defobject{currentmarker}{\pgfqpoint{-0.048611in}{0.000000in}}{\pgfqpoint{0.000000in}{0.000000in}}{%
\pgfpathmoveto{\pgfqpoint{0.000000in}{0.000000in}}%
\pgfpathlineto{\pgfqpoint{-0.048611in}{0.000000in}}%
\pgfusepath{stroke,fill}%
}%
\begin{pgfscope}%
\pgfsys@transformshift{0.687500in}{2.271500in}%
\pgfsys@useobject{currentmarker}{}%
\end{pgfscope}%
\end{pgfscope}%
\begin{pgfscope}%
\definecolor{textcolor}{rgb}{0.000000,0.000000,0.000000}%
\pgfsetstrokecolor{textcolor}%
\pgfsetfillcolor{textcolor}%
\pgftext[x=0.412808in,y=2.223275in,left,base]{\color{textcolor}\rmfamily\fontsize{10.000000}{12.000000}\selectfont \(\displaystyle 0.8\)}%
\end{pgfscope}%
\begin{pgfscope}%
\pgfsetbuttcap%
\pgfsetroundjoin%
\definecolor{currentfill}{rgb}{0.000000,0.000000,0.000000}%
\pgfsetfillcolor{currentfill}%
\pgfsetlinewidth{0.803000pt}%
\definecolor{currentstroke}{rgb}{0.000000,0.000000,0.000000}%
\pgfsetstrokecolor{currentstroke}%
\pgfsetdash{}{0pt}%
\pgfsys@defobject{currentmarker}{\pgfqpoint{-0.048611in}{0.000000in}}{\pgfqpoint{0.000000in}{0.000000in}}{%
\pgfpathmoveto{\pgfqpoint{0.000000in}{0.000000in}}%
\pgfpathlineto{\pgfqpoint{-0.048611in}{0.000000in}}%
\pgfusepath{stroke,fill}%
}%
\begin{pgfscope}%
\pgfsys@transformshift{0.687500in}{2.630833in}%
\pgfsys@useobject{currentmarker}{}%
\end{pgfscope}%
\end{pgfscope}%
\begin{pgfscope}%
\definecolor{textcolor}{rgb}{0.000000,0.000000,0.000000}%
\pgfsetstrokecolor{textcolor}%
\pgfsetfillcolor{textcolor}%
\pgftext[x=0.412808in,y=2.582608in,left,base]{\color{textcolor}\rmfamily\fontsize{10.000000}{12.000000}\selectfont \(\displaystyle 1.0\)}%
\end{pgfscope}%
\begin{pgfscope}%
\pgfsetbuttcap%
\pgfsetroundjoin%
\definecolor{currentfill}{rgb}{0.000000,0.000000,0.000000}%
\pgfsetfillcolor{currentfill}%
\pgfsetlinewidth{0.803000pt}%
\definecolor{currentstroke}{rgb}{0.000000,0.000000,0.000000}%
\pgfsetstrokecolor{currentstroke}%
\pgfsetdash{}{0pt}%
\pgfsys@defobject{currentmarker}{\pgfqpoint{-0.048611in}{0.000000in}}{\pgfqpoint{0.000000in}{0.000000in}}{%
\pgfpathmoveto{\pgfqpoint{0.000000in}{0.000000in}}%
\pgfpathlineto{\pgfqpoint{-0.048611in}{0.000000in}}%
\pgfusepath{stroke,fill}%
}%
\begin{pgfscope}%
\pgfsys@transformshift{0.687500in}{2.990167in}%
\pgfsys@useobject{currentmarker}{}%
\end{pgfscope}%
\end{pgfscope}%
\begin{pgfscope}%
\definecolor{textcolor}{rgb}{0.000000,0.000000,0.000000}%
\pgfsetstrokecolor{textcolor}%
\pgfsetfillcolor{textcolor}%
\pgftext[x=0.412808in,y=2.941941in,left,base]{\color{textcolor}\rmfamily\fontsize{10.000000}{12.000000}\selectfont \(\displaystyle 1.2\)}%
\end{pgfscope}%
\begin{pgfscope}%
\definecolor{textcolor}{rgb}{0.000000,0.000000,0.000000}%
\pgfsetstrokecolor{textcolor}%
\pgfsetfillcolor{textcolor}%
\pgftext[x=0.249228in,y=1.732500in,,bottom,rotate=90.000000]{\color{textcolor}\rmfamily\fontsize{10.000000}{12.000000}\selectfont y}%
\end{pgfscope}%
\begin{pgfscope}%
\pgfpathrectangle{\pgfqpoint{0.687500in}{0.385000in}}{\pgfqpoint{4.262500in}{2.695000in}}%
\pgfusepath{clip}%
\pgfsetrectcap%
\pgfsetroundjoin%
\pgfsetlinewidth{1.505625pt}%
\definecolor{currentstroke}{rgb}{0.121569,0.466667,0.705882}%
\pgfsetstrokecolor{currentstroke}%
\pgfsetdash{}{0pt}%
\pgfpathmoveto{\pgfqpoint{0.881250in}{0.903269in}}%
\pgfpathlineto{\pgfqpoint{1.017557in}{0.913644in}}%
\pgfpathlineto{\pgfqpoint{1.134391in}{0.924478in}}%
\pgfpathlineto{\pgfqpoint{1.251225in}{0.937638in}}%
\pgfpathlineto{\pgfqpoint{1.348587in}{0.950877in}}%
\pgfpathlineto{\pgfqpoint{1.426476in}{0.963336in}}%
\pgfpathlineto{\pgfqpoint{1.504366in}{0.977838in}}%
\pgfpathlineto{\pgfqpoint{1.582255in}{0.994839in}}%
\pgfpathlineto{\pgfqpoint{1.640672in}{1.009574in}}%
\pgfpathlineto{\pgfqpoint{1.699089in}{1.026346in}}%
\pgfpathlineto{\pgfqpoint{1.757506in}{1.045528in}}%
\pgfpathlineto{\pgfqpoint{1.815923in}{1.067578in}}%
\pgfpathlineto{\pgfqpoint{1.854868in}{1.084143in}}%
\pgfpathlineto{\pgfqpoint{1.893813in}{1.102428in}}%
\pgfpathlineto{\pgfqpoint{1.932758in}{1.122659in}}%
\pgfpathlineto{\pgfqpoint{1.971702in}{1.145101in}}%
\pgfpathlineto{\pgfqpoint{2.010647in}{1.170055in}}%
\pgfpathlineto{\pgfqpoint{2.049592in}{1.197870in}}%
\pgfpathlineto{\pgfqpoint{2.088536in}{1.228948in}}%
\pgfpathlineto{\pgfqpoint{2.127481in}{1.263748in}}%
\pgfpathlineto{\pgfqpoint{2.166426in}{1.302794in}}%
\pgfpathlineto{\pgfqpoint{2.205371in}{1.346677in}}%
\pgfpathlineto{\pgfqpoint{2.244315in}{1.396056in}}%
\pgfpathlineto{\pgfqpoint{2.283260in}{1.451647in}}%
\pgfpathlineto{\pgfqpoint{2.322205in}{1.514206in}}%
\pgfpathlineto{\pgfqpoint{2.361149in}{1.584486in}}%
\pgfpathlineto{\pgfqpoint{2.400094in}{1.663167in}}%
\pgfpathlineto{\pgfqpoint{2.439039in}{1.750738in}}%
\pgfpathlineto{\pgfqpoint{2.477984in}{1.847321in}}%
\pgfpathlineto{\pgfqpoint{2.516928in}{1.952417in}}%
\pgfpathlineto{\pgfqpoint{2.555873in}{2.064579in}}%
\pgfpathlineto{\pgfqpoint{2.653235in}{2.353617in}}%
\pgfpathlineto{\pgfqpoint{2.672707in}{2.407372in}}%
\pgfpathlineto{\pgfqpoint{2.692180in}{2.457628in}}%
\pgfpathlineto{\pgfqpoint{2.711652in}{2.503331in}}%
\pgfpathlineto{\pgfqpoint{2.731124in}{2.543430in}}%
\pgfpathlineto{\pgfqpoint{2.750597in}{2.576924in}}%
\pgfpathlineto{\pgfqpoint{2.770069in}{2.602918in}}%
\pgfpathlineto{\pgfqpoint{2.789541in}{2.620683in}}%
\pgfpathlineto{\pgfqpoint{2.809014in}{2.629700in}}%
\pgfpathlineto{\pgfqpoint{2.828486in}{2.629700in}}%
\pgfpathlineto{\pgfqpoint{2.847959in}{2.620683in}}%
\pgfpathlineto{\pgfqpoint{2.867431in}{2.602918in}}%
\pgfpathlineto{\pgfqpoint{2.886903in}{2.576924in}}%
\pgfpathlineto{\pgfqpoint{2.906376in}{2.543430in}}%
\pgfpathlineto{\pgfqpoint{2.925848in}{2.503331in}}%
\pgfpathlineto{\pgfqpoint{2.945320in}{2.457628in}}%
\pgfpathlineto{\pgfqpoint{2.964793in}{2.407372in}}%
\pgfpathlineto{\pgfqpoint{3.003737in}{2.297371in}}%
\pgfpathlineto{\pgfqpoint{3.120572in}{1.952417in}}%
\pgfpathlineto{\pgfqpoint{3.159516in}{1.847321in}}%
\pgfpathlineto{\pgfqpoint{3.198461in}{1.750738in}}%
\pgfpathlineto{\pgfqpoint{3.237406in}{1.663167in}}%
\pgfpathlineto{\pgfqpoint{3.276351in}{1.584486in}}%
\pgfpathlineto{\pgfqpoint{3.315295in}{1.514206in}}%
\pgfpathlineto{\pgfqpoint{3.354240in}{1.451647in}}%
\pgfpathlineto{\pgfqpoint{3.393185in}{1.396056in}}%
\pgfpathlineto{\pgfqpoint{3.432129in}{1.346677in}}%
\pgfpathlineto{\pgfqpoint{3.471074in}{1.302794in}}%
\pgfpathlineto{\pgfqpoint{3.510019in}{1.263748in}}%
\pgfpathlineto{\pgfqpoint{3.548964in}{1.228948in}}%
\pgfpathlineto{\pgfqpoint{3.587908in}{1.197870in}}%
\pgfpathlineto{\pgfqpoint{3.626853in}{1.170055in}}%
\pgfpathlineto{\pgfqpoint{3.665798in}{1.145101in}}%
\pgfpathlineto{\pgfqpoint{3.704742in}{1.122659in}}%
\pgfpathlineto{\pgfqpoint{3.743687in}{1.102428in}}%
\pgfpathlineto{\pgfqpoint{3.782632in}{1.084143in}}%
\pgfpathlineto{\pgfqpoint{3.841049in}{1.059877in}}%
\pgfpathlineto{\pgfqpoint{3.899466in}{1.038840in}}%
\pgfpathlineto{\pgfqpoint{3.957883in}{1.020508in}}%
\pgfpathlineto{\pgfqpoint{4.016300in}{1.004452in}}%
\pgfpathlineto{\pgfqpoint{4.074717in}{0.990325in}}%
\pgfpathlineto{\pgfqpoint{4.152607in}{0.973998in}}%
\pgfpathlineto{\pgfqpoint{4.230496in}{0.960045in}}%
\pgfpathlineto{\pgfqpoint{4.327858in}{0.945299in}}%
\pgfpathlineto{\pgfqpoint{4.425220in}{0.932955in}}%
\pgfpathlineto{\pgfqpoint{4.542054in}{0.920637in}}%
\pgfpathlineto{\pgfqpoint{4.678361in}{0.908933in}}%
\pgfpathlineto{\pgfqpoint{4.756250in}{0.903269in}}%
\pgfpathlineto{\pgfqpoint{4.756250in}{0.903269in}}%
\pgfusepath{stroke}%
\end{pgfscope}%
\begin{pgfscope}%
\pgfpathrectangle{\pgfqpoint{0.687500in}{0.385000in}}{\pgfqpoint{4.262500in}{2.695000in}}%
\pgfusepath{clip}%
\pgfsetbuttcap%
\pgfsetroundjoin%
\definecolor{currentfill}{rgb}{1.000000,0.498039,0.054902}%
\pgfsetfillcolor{currentfill}%
\pgfsetlinewidth{1.003750pt}%
\definecolor{currentstroke}{rgb}{1.000000,0.498039,0.054902}%
\pgfsetstrokecolor{currentstroke}%
\pgfsetdash{}{0pt}%
\pgfsys@defobject{currentmarker}{\pgfqpoint{-0.020833in}{-0.020833in}}{\pgfqpoint{0.020833in}{0.020833in}}{%
\pgfpathmoveto{\pgfqpoint{0.000000in}{-0.020833in}}%
\pgfpathcurveto{\pgfqpoint{0.005525in}{-0.020833in}}{\pgfqpoint{0.010825in}{-0.018638in}}{\pgfqpoint{0.014731in}{-0.014731in}}%
\pgfpathcurveto{\pgfqpoint{0.018638in}{-0.010825in}}{\pgfqpoint{0.020833in}{-0.005525in}}{\pgfqpoint{0.020833in}{0.000000in}}%
\pgfpathcurveto{\pgfqpoint{0.020833in}{0.005525in}}{\pgfqpoint{0.018638in}{0.010825in}}{\pgfqpoint{0.014731in}{0.014731in}}%
\pgfpathcurveto{\pgfqpoint{0.010825in}{0.018638in}}{\pgfqpoint{0.005525in}{0.020833in}}{\pgfqpoint{0.000000in}{0.020833in}}%
\pgfpathcurveto{\pgfqpoint{-0.005525in}{0.020833in}}{\pgfqpoint{-0.010825in}{0.018638in}}{\pgfqpoint{-0.014731in}{0.014731in}}%
\pgfpathcurveto{\pgfqpoint{-0.018638in}{0.010825in}}{\pgfqpoint{-0.020833in}{0.005525in}}{\pgfqpoint{-0.020833in}{0.000000in}}%
\pgfpathcurveto{\pgfqpoint{-0.020833in}{-0.005525in}}{\pgfqpoint{-0.018638in}{-0.010825in}}{\pgfqpoint{-0.014731in}{-0.014731in}}%
\pgfpathcurveto{\pgfqpoint{-0.010825in}{-0.018638in}}{\pgfqpoint{-0.005525in}{-0.020833in}}{\pgfqpoint{0.000000in}{-0.020833in}}%
\pgfpathclose%
\pgfusepath{stroke,fill}%
}%
\begin{pgfscope}%
\pgfsys@transformshift{0.881250in}{0.903269in}%
\pgfsys@useobject{currentmarker}{}%
\end{pgfscope}%
\begin{pgfscope}%
\pgfsys@transformshift{2.172917in}{1.309755in}%
\pgfsys@useobject{currentmarker}{}%
\end{pgfscope}%
\begin{pgfscope}%
\pgfsys@transformshift{3.464583in}{1.309755in}%
\pgfsys@useobject{currentmarker}{}%
\end{pgfscope}%
\end{pgfscope}%
\begin{pgfscope}%
\pgfpathrectangle{\pgfqpoint{0.687500in}{0.385000in}}{\pgfqpoint{4.262500in}{2.695000in}}%
\pgfusepath{clip}%
\pgfsetrectcap%
\pgfsetroundjoin%
\pgfsetlinewidth{1.505625pt}%
\definecolor{currentstroke}{rgb}{0.172549,0.627451,0.172549}%
\pgfsetstrokecolor{currentstroke}%
\pgfsetdash{}{0pt}%
\pgfpathmoveto{\pgfqpoint{0.881250in}{0.903269in}}%
\pgfpathlineto{\pgfqpoint{0.978612in}{0.948074in}}%
\pgfpathlineto{\pgfqpoint{1.075974in}{0.990569in}}%
\pgfpathlineto{\pgfqpoint{1.173335in}{1.030755in}}%
\pgfpathlineto{\pgfqpoint{1.270697in}{1.068631in}}%
\pgfpathlineto{\pgfqpoint{1.368059in}{1.104197in}}%
\pgfpathlineto{\pgfqpoint{1.465421in}{1.137454in}}%
\pgfpathlineto{\pgfqpoint{1.562783in}{1.168402in}}%
\pgfpathlineto{\pgfqpoint{1.660144in}{1.197040in}}%
\pgfpathlineto{\pgfqpoint{1.757506in}{1.223369in}}%
\pgfpathlineto{\pgfqpoint{1.854868in}{1.247388in}}%
\pgfpathlineto{\pgfqpoint{1.952230in}{1.269097in}}%
\pgfpathlineto{\pgfqpoint{2.049592in}{1.288497in}}%
\pgfpathlineto{\pgfqpoint{2.146954in}{1.305588in}}%
\pgfpathlineto{\pgfqpoint{2.244315in}{1.320368in}}%
\pgfpathlineto{\pgfqpoint{2.341677in}{1.332840in}}%
\pgfpathlineto{\pgfqpoint{2.439039in}{1.343002in}}%
\pgfpathlineto{\pgfqpoint{2.536401in}{1.350854in}}%
\pgfpathlineto{\pgfqpoint{2.633763in}{1.356397in}}%
\pgfpathlineto{\pgfqpoint{2.731124in}{1.359630in}}%
\pgfpathlineto{\pgfqpoint{2.828486in}{1.360554in}}%
\pgfpathlineto{\pgfqpoint{2.925848in}{1.359168in}}%
\pgfpathlineto{\pgfqpoint{3.023210in}{1.355473in}}%
\pgfpathlineto{\pgfqpoint{3.120572in}{1.349468in}}%
\pgfpathlineto{\pgfqpoint{3.217933in}{1.341154in}}%
\pgfpathlineto{\pgfqpoint{3.315295in}{1.330530in}}%
\pgfpathlineto{\pgfqpoint{3.412657in}{1.317597in}}%
\pgfpathlineto{\pgfqpoint{3.510019in}{1.302354in}}%
\pgfpathlineto{\pgfqpoint{3.607381in}{1.284802in}}%
\pgfpathlineto{\pgfqpoint{3.704742in}{1.264940in}}%
\pgfpathlineto{\pgfqpoint{3.802104in}{1.242769in}}%
\pgfpathlineto{\pgfqpoint{3.899466in}{1.218288in}}%
\pgfpathlineto{\pgfqpoint{3.996828in}{1.191497in}}%
\pgfpathlineto{\pgfqpoint{4.094190in}{1.162397in}}%
\pgfpathlineto{\pgfqpoint{4.191552in}{1.130988in}}%
\pgfpathlineto{\pgfqpoint{4.288913in}{1.097269in}}%
\pgfpathlineto{\pgfqpoint{4.386275in}{1.061240in}}%
\pgfpathlineto{\pgfqpoint{4.483637in}{1.022902in}}%
\pgfpathlineto{\pgfqpoint{4.580999in}{0.982255in}}%
\pgfpathlineto{\pgfqpoint{4.678361in}{0.939298in}}%
\pgfpathlineto{\pgfqpoint{4.756250in}{0.903269in}}%
\pgfpathlineto{\pgfqpoint{4.756250in}{0.903269in}}%
\pgfusepath{stroke}%
\end{pgfscope}%
\begin{pgfscope}%
\pgfpathrectangle{\pgfqpoint{0.687500in}{0.385000in}}{\pgfqpoint{4.262500in}{2.695000in}}%
\pgfusepath{clip}%
\pgfsetbuttcap%
\pgfsetroundjoin%
\definecolor{currentfill}{rgb}{0.839216,0.152941,0.156863}%
\pgfsetfillcolor{currentfill}%
\pgfsetlinewidth{1.003750pt}%
\definecolor{currentstroke}{rgb}{0.839216,0.152941,0.156863}%
\pgfsetstrokecolor{currentstroke}%
\pgfsetdash{}{0pt}%
\pgfsys@defobject{currentmarker}{\pgfqpoint{-0.020833in}{-0.020833in}}{\pgfqpoint{0.020833in}{0.020833in}}{%
\pgfpathmoveto{\pgfqpoint{0.000000in}{-0.020833in}}%
\pgfpathcurveto{\pgfqpoint{0.005525in}{-0.020833in}}{\pgfqpoint{0.010825in}{-0.018638in}}{\pgfqpoint{0.014731in}{-0.014731in}}%
\pgfpathcurveto{\pgfqpoint{0.018638in}{-0.010825in}}{\pgfqpoint{0.020833in}{-0.005525in}}{\pgfqpoint{0.020833in}{0.000000in}}%
\pgfpathcurveto{\pgfqpoint{0.020833in}{0.005525in}}{\pgfqpoint{0.018638in}{0.010825in}}{\pgfqpoint{0.014731in}{0.014731in}}%
\pgfpathcurveto{\pgfqpoint{0.010825in}{0.018638in}}{\pgfqpoint{0.005525in}{0.020833in}}{\pgfqpoint{0.000000in}{0.020833in}}%
\pgfpathcurveto{\pgfqpoint{-0.005525in}{0.020833in}}{\pgfqpoint{-0.010825in}{0.018638in}}{\pgfqpoint{-0.014731in}{0.014731in}}%
\pgfpathcurveto{\pgfqpoint{-0.018638in}{0.010825in}}{\pgfqpoint{-0.020833in}{0.005525in}}{\pgfqpoint{-0.020833in}{0.000000in}}%
\pgfpathcurveto{\pgfqpoint{-0.020833in}{-0.005525in}}{\pgfqpoint{-0.018638in}{-0.010825in}}{\pgfqpoint{-0.014731in}{-0.014731in}}%
\pgfpathcurveto{\pgfqpoint{-0.010825in}{-0.018638in}}{\pgfqpoint{-0.005525in}{-0.020833in}}{\pgfqpoint{0.000000in}{-0.020833in}}%
\pgfpathclose%
\pgfusepath{stroke,fill}%
}%
\begin{pgfscope}%
\pgfsys@transformshift{0.881250in}{0.903269in}%
\pgfsys@useobject{currentmarker}{}%
\end{pgfscope}%
\begin{pgfscope}%
\pgfsys@transformshift{1.656250in}{1.013833in}%
\pgfsys@useobject{currentmarker}{}%
\end{pgfscope}%
\begin{pgfscope}%
\pgfsys@transformshift{2.431250in}{1.732500in}%
\pgfsys@useobject{currentmarker}{}%
\end{pgfscope}%
\begin{pgfscope}%
\pgfsys@transformshift{3.206250in}{1.732500in}%
\pgfsys@useobject{currentmarker}{}%
\end{pgfscope}%
\begin{pgfscope}%
\pgfsys@transformshift{3.981250in}{1.013833in}%
\pgfsys@useobject{currentmarker}{}%
\end{pgfscope}%
\end{pgfscope}%
\begin{pgfscope}%
\pgfpathrectangle{\pgfqpoint{0.687500in}{0.385000in}}{\pgfqpoint{4.262500in}{2.695000in}}%
\pgfusepath{clip}%
\pgfsetrectcap%
\pgfsetroundjoin%
\pgfsetlinewidth{1.505625pt}%
\definecolor{currentstroke}{rgb}{0.580392,0.403922,0.741176}%
\pgfsetstrokecolor{currentstroke}%
\pgfsetdash{}{0pt}%
\pgfpathmoveto{\pgfqpoint{0.881250in}{0.903269in}}%
\pgfpathlineto{\pgfqpoint{0.900722in}{0.879948in}}%
\pgfpathlineto{\pgfqpoint{0.920195in}{0.858563in}}%
\pgfpathlineto{\pgfqpoint{0.939667in}{0.839064in}}%
\pgfpathlineto{\pgfqpoint{0.959139in}{0.821398in}}%
\pgfpathlineto{\pgfqpoint{0.978612in}{0.805516in}}%
\pgfpathlineto{\pgfqpoint{0.998084in}{0.791366in}}%
\pgfpathlineto{\pgfqpoint{1.017557in}{0.778899in}}%
\pgfpathlineto{\pgfqpoint{1.037029in}{0.768067in}}%
\pgfpathlineto{\pgfqpoint{1.056501in}{0.758820in}}%
\pgfpathlineto{\pgfqpoint{1.075974in}{0.751110in}}%
\pgfpathlineto{\pgfqpoint{1.095446in}{0.744889in}}%
\pgfpathlineto{\pgfqpoint{1.114918in}{0.740112in}}%
\pgfpathlineto{\pgfqpoint{1.134391in}{0.736730in}}%
\pgfpathlineto{\pgfqpoint{1.153863in}{0.734698in}}%
\pgfpathlineto{\pgfqpoint{1.173335in}{0.733971in}}%
\pgfpathlineto{\pgfqpoint{1.192808in}{0.734504in}}%
\pgfpathlineto{\pgfqpoint{1.212280in}{0.736251in}}%
\pgfpathlineto{\pgfqpoint{1.231753in}{0.739170in}}%
\pgfpathlineto{\pgfqpoint{1.270697in}{0.748349in}}%
\pgfpathlineto{\pgfqpoint{1.309642in}{0.761700in}}%
\pgfpathlineto{\pgfqpoint{1.348587in}{0.778891in}}%
\pgfpathlineto{\pgfqpoint{1.387531in}{0.799597in}}%
\pgfpathlineto{\pgfqpoint{1.426476in}{0.823505in}}%
\pgfpathlineto{\pgfqpoint{1.465421in}{0.850307in}}%
\pgfpathlineto{\pgfqpoint{1.504366in}{0.879705in}}%
\pgfpathlineto{\pgfqpoint{1.543310in}{0.911409in}}%
\pgfpathlineto{\pgfqpoint{1.601727in}{0.962677in}}%
\pgfpathlineto{\pgfqpoint{1.660144in}{1.017590in}}%
\pgfpathlineto{\pgfqpoint{1.738034in}{1.094977in}}%
\pgfpathlineto{\pgfqpoint{1.835396in}{1.195700in}}%
\pgfpathlineto{\pgfqpoint{2.010647in}{1.377830in}}%
\pgfpathlineto{\pgfqpoint{2.088536in}{1.455303in}}%
\pgfpathlineto{\pgfqpoint{2.146954in}{1.510791in}}%
\pgfpathlineto{\pgfqpoint{2.205371in}{1.563461in}}%
\pgfpathlineto{\pgfqpoint{2.263788in}{1.612842in}}%
\pgfpathlineto{\pgfqpoint{2.322205in}{1.658505in}}%
\pgfpathlineto{\pgfqpoint{2.361149in}{1.686690in}}%
\pgfpathlineto{\pgfqpoint{2.400094in}{1.712947in}}%
\pgfpathlineto{\pgfqpoint{2.439039in}{1.737181in}}%
\pgfpathlineto{\pgfqpoint{2.477984in}{1.759304in}}%
\pgfpathlineto{\pgfqpoint{2.516928in}{1.779240in}}%
\pgfpathlineto{\pgfqpoint{2.555873in}{1.796918in}}%
\pgfpathlineto{\pgfqpoint{2.594818in}{1.812276in}}%
\pgfpathlineto{\pgfqpoint{2.633763in}{1.825262in}}%
\pgfpathlineto{\pgfqpoint{2.672707in}{1.835831in}}%
\pgfpathlineto{\pgfqpoint{2.711652in}{1.843948in}}%
\pgfpathlineto{\pgfqpoint{2.750597in}{1.849585in}}%
\pgfpathlineto{\pgfqpoint{2.789541in}{1.852723in}}%
\pgfpathlineto{\pgfqpoint{2.828486in}{1.853351in}}%
\pgfpathlineto{\pgfqpoint{2.867431in}{1.851467in}}%
\pgfpathlineto{\pgfqpoint{2.906376in}{1.847078in}}%
\pgfpathlineto{\pgfqpoint{2.945320in}{1.840198in}}%
\pgfpathlineto{\pgfqpoint{2.984265in}{1.830851in}}%
\pgfpathlineto{\pgfqpoint{3.023210in}{1.819068in}}%
\pgfpathlineto{\pgfqpoint{3.062155in}{1.804890in}}%
\pgfpathlineto{\pgfqpoint{3.101099in}{1.788365in}}%
\pgfpathlineto{\pgfqpoint{3.140044in}{1.769550in}}%
\pgfpathlineto{\pgfqpoint{3.178989in}{1.748511in}}%
\pgfpathlineto{\pgfqpoint{3.217933in}{1.725322in}}%
\pgfpathlineto{\pgfqpoint{3.256878in}{1.700066in}}%
\pgfpathlineto{\pgfqpoint{3.295823in}{1.672832in}}%
\pgfpathlineto{\pgfqpoint{3.354240in}{1.628496in}}%
\pgfpathlineto{\pgfqpoint{3.412657in}{1.580309in}}%
\pgfpathlineto{\pgfqpoint{3.471074in}{1.528685in}}%
\pgfpathlineto{\pgfqpoint{3.529491in}{1.474081in}}%
\pgfpathlineto{\pgfqpoint{3.607381in}{1.397511in}}%
\pgfpathlineto{\pgfqpoint{3.704742in}{1.297602in}}%
\pgfpathlineto{\pgfqpoint{3.918938in}{1.075276in}}%
\pgfpathlineto{\pgfqpoint{3.977356in}{1.017590in}}%
\pgfpathlineto{\pgfqpoint{4.035773in}{0.962677in}}%
\pgfpathlineto{\pgfqpoint{4.094190in}{0.911409in}}%
\pgfpathlineto{\pgfqpoint{4.133134in}{0.879705in}}%
\pgfpathlineto{\pgfqpoint{4.172079in}{0.850307in}}%
\pgfpathlineto{\pgfqpoint{4.211024in}{0.823505in}}%
\pgfpathlineto{\pgfqpoint{4.249969in}{0.799597in}}%
\pgfpathlineto{\pgfqpoint{4.288913in}{0.778891in}}%
\pgfpathlineto{\pgfqpoint{4.327858in}{0.761700in}}%
\pgfpathlineto{\pgfqpoint{4.366803in}{0.748349in}}%
\pgfpathlineto{\pgfqpoint{4.405747in}{0.739170in}}%
\pgfpathlineto{\pgfqpoint{4.425220in}{0.736251in}}%
\pgfpathlineto{\pgfqpoint{4.444692in}{0.734504in}}%
\pgfpathlineto{\pgfqpoint{4.464165in}{0.733971in}}%
\pgfpathlineto{\pgfqpoint{4.483637in}{0.734698in}}%
\pgfpathlineto{\pgfqpoint{4.503109in}{0.736730in}}%
\pgfpathlineto{\pgfqpoint{4.522582in}{0.740112in}}%
\pgfpathlineto{\pgfqpoint{4.542054in}{0.744889in}}%
\pgfpathlineto{\pgfqpoint{4.561526in}{0.751110in}}%
\pgfpathlineto{\pgfqpoint{4.580999in}{0.758820in}}%
\pgfpathlineto{\pgfqpoint{4.600471in}{0.768067in}}%
\pgfpathlineto{\pgfqpoint{4.619943in}{0.778899in}}%
\pgfpathlineto{\pgfqpoint{4.639416in}{0.791366in}}%
\pgfpathlineto{\pgfqpoint{4.658888in}{0.805516in}}%
\pgfpathlineto{\pgfqpoint{4.678361in}{0.821398in}}%
\pgfpathlineto{\pgfqpoint{4.697833in}{0.839064in}}%
\pgfpathlineto{\pgfqpoint{4.717305in}{0.858563in}}%
\pgfpathlineto{\pgfqpoint{4.736778in}{0.879948in}}%
\pgfpathlineto{\pgfqpoint{4.756250in}{0.903269in}}%
\pgfpathlineto{\pgfqpoint{4.756250in}{0.903269in}}%
\pgfusepath{stroke}%
\end{pgfscope}%
\begin{pgfscope}%
\pgfsetrectcap%
\pgfsetmiterjoin%
\pgfsetlinewidth{0.803000pt}%
\definecolor{currentstroke}{rgb}{0.000000,0.000000,0.000000}%
\pgfsetstrokecolor{currentstroke}%
\pgfsetdash{}{0pt}%
\pgfpathmoveto{\pgfqpoint{0.687500in}{0.385000in}}%
\pgfpathlineto{\pgfqpoint{0.687500in}{3.080000in}}%
\pgfusepath{stroke}%
\end{pgfscope}%
\begin{pgfscope}%
\pgfsetrectcap%
\pgfsetmiterjoin%
\pgfsetlinewidth{0.803000pt}%
\definecolor{currentstroke}{rgb}{0.000000,0.000000,0.000000}%
\pgfsetstrokecolor{currentstroke}%
\pgfsetdash{}{0pt}%
\pgfpathmoveto{\pgfqpoint{4.950000in}{0.385000in}}%
\pgfpathlineto{\pgfqpoint{4.950000in}{3.080000in}}%
\pgfusepath{stroke}%
\end{pgfscope}%
\begin{pgfscope}%
\pgfsetrectcap%
\pgfsetmiterjoin%
\pgfsetlinewidth{0.803000pt}%
\definecolor{currentstroke}{rgb}{0.000000,0.000000,0.000000}%
\pgfsetstrokecolor{currentstroke}%
\pgfsetdash{}{0pt}%
\pgfpathmoveto{\pgfqpoint{0.687500in}{0.385000in}}%
\pgfpathlineto{\pgfqpoint{4.950000in}{0.385000in}}%
\pgfusepath{stroke}%
\end{pgfscope}%
\begin{pgfscope}%
\pgfsetrectcap%
\pgfsetmiterjoin%
\pgfsetlinewidth{0.803000pt}%
\definecolor{currentstroke}{rgb}{0.000000,0.000000,0.000000}%
\pgfsetstrokecolor{currentstroke}%
\pgfsetdash{}{0pt}%
\pgfpathmoveto{\pgfqpoint{0.687500in}{3.080000in}}%
\pgfpathlineto{\pgfqpoint{4.950000in}{3.080000in}}%
\pgfusepath{stroke}%
\end{pgfscope}%
\begin{pgfscope}%
\definecolor{textcolor}{rgb}{0.000000,0.000000,0.000000}%
\pgfsetstrokecolor{textcolor}%
\pgfsetfillcolor{textcolor}%
\pgftext[x=2.818750in,y=3.163333in,,base]{\color{textcolor}\rmfamily\fontsize{12.000000}{14.400000}\selectfont N=2, 4}%
\end{pgfscope}%
\begin{pgfscope}%
\pgfsetbuttcap%
\pgfsetmiterjoin%
\definecolor{currentfill}{rgb}{1.000000,1.000000,1.000000}%
\pgfsetfillcolor{currentfill}%
\pgfsetfillopacity{0.800000}%
\pgfsetlinewidth{1.003750pt}%
\definecolor{currentstroke}{rgb}{0.800000,0.800000,0.800000}%
\pgfsetstrokecolor{currentstroke}%
\pgfsetstrokeopacity{0.800000}%
\pgfsetdash{}{0pt}%
\pgfpathmoveto{\pgfqpoint{3.448142in}{2.192467in}}%
\pgfpathlineto{\pgfqpoint{4.852778in}{2.192467in}}%
\pgfpathquadraticcurveto{\pgfqpoint{4.880556in}{2.192467in}}{\pgfqpoint{4.880556in}{2.220245in}}%
\pgfpathlineto{\pgfqpoint{4.880556in}{2.982778in}}%
\pgfpathquadraticcurveto{\pgfqpoint{4.880556in}{3.010556in}}{\pgfqpoint{4.852778in}{3.010556in}}%
\pgfpathlineto{\pgfqpoint{3.448142in}{3.010556in}}%
\pgfpathquadraticcurveto{\pgfqpoint{3.420364in}{3.010556in}}{\pgfqpoint{3.420364in}{2.982778in}}%
\pgfpathlineto{\pgfqpoint{3.420364in}{2.220245in}}%
\pgfpathquadraticcurveto{\pgfqpoint{3.420364in}{2.192467in}}{\pgfqpoint{3.448142in}{2.192467in}}%
\pgfpathclose%
\pgfusepath{stroke,fill}%
\end{pgfscope}%
\begin{pgfscope}%
\pgfsetrectcap%
\pgfsetroundjoin%
\pgfsetlinewidth{1.505625pt}%
\definecolor{currentstroke}{rgb}{0.121569,0.466667,0.705882}%
\pgfsetstrokecolor{currentstroke}%
\pgfsetdash{}{0pt}%
\pgfpathmoveto{\pgfqpoint{3.475920in}{2.820146in}}%
\pgfpathlineto{\pgfqpoint{3.753698in}{2.820146in}}%
\pgfusepath{stroke}%
\end{pgfscope}%
\begin{pgfscope}%
\definecolor{textcolor}{rgb}{0.000000,0.000000,0.000000}%
\pgfsetstrokecolor{textcolor}%
\pgfsetfillcolor{textcolor}%
\pgftext[x=3.864809in,y=2.771534in,left,base]{\color{textcolor}\rmfamily\fontsize{10.000000}{12.000000}\selectfont \(\displaystyle y(x)=\)\(\displaystyle \frac{1}{1+25x^{2}}\)}%
\end{pgfscope}%
\begin{pgfscope}%
\pgfsetrectcap%
\pgfsetroundjoin%
\pgfsetlinewidth{1.505625pt}%
\definecolor{currentstroke}{rgb}{0.172549,0.627451,0.172549}%
\pgfsetstrokecolor{currentstroke}%
\pgfsetdash{}{0pt}%
\pgfpathmoveto{\pgfqpoint{3.475920in}{2.539689in}}%
\pgfpathlineto{\pgfqpoint{3.753698in}{2.539689in}}%
\pgfusepath{stroke}%
\end{pgfscope}%
\begin{pgfscope}%
\definecolor{textcolor}{rgb}{0.000000,0.000000,0.000000}%
\pgfsetstrokecolor{textcolor}%
\pgfsetfillcolor{textcolor}%
\pgftext[x=3.864809in,y=2.491078in,left,base]{\color{textcolor}\rmfamily\fontsize{10.000000}{12.000000}\selectfont W2(x)}%
\end{pgfscope}%
\begin{pgfscope}%
\pgfsetrectcap%
\pgfsetroundjoin%
\pgfsetlinewidth{1.505625pt}%
\definecolor{currentstroke}{rgb}{0.580392,0.403922,0.741176}%
\pgfsetstrokecolor{currentstroke}%
\pgfsetdash{}{0pt}%
\pgfpathmoveto{\pgfqpoint{3.475920in}{2.331356in}}%
\pgfpathlineto{\pgfqpoint{3.753698in}{2.331356in}}%
\pgfusepath{stroke}%
\end{pgfscope}%
\begin{pgfscope}%
\definecolor{textcolor}{rgb}{0.000000,0.000000,0.000000}%
\pgfsetstrokecolor{textcolor}%
\pgfsetfillcolor{textcolor}%
\pgftext[x=3.864809in,y=2.282745in,left,base]{\color{textcolor}\rmfamily\fontsize{10.000000}{12.000000}\selectfont W4(x)}%
\end{pgfscope}%
\end{pgfpicture}%
\makeatother%
\endgroup%
        
    \end{center}
    \caption{Węzły jednorodne, funkcja \(y\), \(N=2,4\)}
\end{figure}

\begin{figure}[h]
    \begin{center}
        %% Creator: Matplotlib, PGF backend
%%
%% To include the figure in your LaTeX document, write
%%   \input{<filename>.pgf}
%%
%% Make sure the required packages are loaded in your preamble
%%   \usepackage{pgf}
%%
%% Figures using additional raster images can only be included by \input if
%% they are in the same directory as the main LaTeX file. For loading figures
%% from other directories you can use the `import` package
%%   \usepackage{import}
%% and then include the figures with
%%   \import{<path to file>}{<filename>.pgf}
%%
%% Matplotlib used the following preamble
%%
\begingroup%
\makeatletter%
\begin{pgfpicture}%
\pgfpathrectangle{\pgfpointorigin}{\pgfqpoint{5.500000in}{3.500000in}}%
\pgfusepath{use as bounding box, clip}%
\begin{pgfscope}%
\pgfsetbuttcap%
\pgfsetmiterjoin%
\definecolor{currentfill}{rgb}{1.000000,1.000000,1.000000}%
\pgfsetfillcolor{currentfill}%
\pgfsetlinewidth{0.000000pt}%
\definecolor{currentstroke}{rgb}{1.000000,1.000000,1.000000}%
\pgfsetstrokecolor{currentstroke}%
\pgfsetdash{}{0pt}%
\pgfpathmoveto{\pgfqpoint{0.000000in}{0.000000in}}%
\pgfpathlineto{\pgfqpoint{5.500000in}{0.000000in}}%
\pgfpathlineto{\pgfqpoint{5.500000in}{3.500000in}}%
\pgfpathlineto{\pgfqpoint{0.000000in}{3.500000in}}%
\pgfpathclose%
\pgfusepath{fill}%
\end{pgfscope}%
\begin{pgfscope}%
\pgfsetbuttcap%
\pgfsetmiterjoin%
\definecolor{currentfill}{rgb}{1.000000,1.000000,1.000000}%
\pgfsetfillcolor{currentfill}%
\pgfsetlinewidth{0.000000pt}%
\definecolor{currentstroke}{rgb}{0.000000,0.000000,0.000000}%
\pgfsetstrokecolor{currentstroke}%
\pgfsetstrokeopacity{0.000000}%
\pgfsetdash{}{0pt}%
\pgfpathmoveto{\pgfqpoint{0.687500in}{0.385000in}}%
\pgfpathlineto{\pgfqpoint{4.950000in}{0.385000in}}%
\pgfpathlineto{\pgfqpoint{4.950000in}{3.080000in}}%
\pgfpathlineto{\pgfqpoint{0.687500in}{3.080000in}}%
\pgfpathclose%
\pgfusepath{fill}%
\end{pgfscope}%
\begin{pgfscope}%
\pgfsetbuttcap%
\pgfsetroundjoin%
\definecolor{currentfill}{rgb}{0.000000,0.000000,0.000000}%
\pgfsetfillcolor{currentfill}%
\pgfsetlinewidth{0.803000pt}%
\definecolor{currentstroke}{rgb}{0.000000,0.000000,0.000000}%
\pgfsetstrokecolor{currentstroke}%
\pgfsetdash{}{0pt}%
\pgfsys@defobject{currentmarker}{\pgfqpoint{0.000000in}{-0.048611in}}{\pgfqpoint{0.000000in}{0.000000in}}{%
\pgfpathmoveto{\pgfqpoint{0.000000in}{0.000000in}}%
\pgfpathlineto{\pgfqpoint{0.000000in}{-0.048611in}}%
\pgfusepath{stroke,fill}%
}%
\begin{pgfscope}%
\pgfsys@transformshift{0.881250in}{0.385000in}%
\pgfsys@useobject{currentmarker}{}%
\end{pgfscope}%
\end{pgfscope}%
\begin{pgfscope}%
\definecolor{textcolor}{rgb}{0.000000,0.000000,0.000000}%
\pgfsetstrokecolor{textcolor}%
\pgfsetfillcolor{textcolor}%
\pgftext[x=0.881250in,y=0.287778in,,top]{\color{textcolor}\rmfamily\fontsize{10.000000}{12.000000}\selectfont \(\displaystyle -1.00\)}%
\end{pgfscope}%
\begin{pgfscope}%
\pgfsetbuttcap%
\pgfsetroundjoin%
\definecolor{currentfill}{rgb}{0.000000,0.000000,0.000000}%
\pgfsetfillcolor{currentfill}%
\pgfsetlinewidth{0.803000pt}%
\definecolor{currentstroke}{rgb}{0.000000,0.000000,0.000000}%
\pgfsetstrokecolor{currentstroke}%
\pgfsetdash{}{0pt}%
\pgfsys@defobject{currentmarker}{\pgfqpoint{0.000000in}{-0.048611in}}{\pgfqpoint{0.000000in}{0.000000in}}{%
\pgfpathmoveto{\pgfqpoint{0.000000in}{0.000000in}}%
\pgfpathlineto{\pgfqpoint{0.000000in}{-0.048611in}}%
\pgfusepath{stroke,fill}%
}%
\begin{pgfscope}%
\pgfsys@transformshift{1.365625in}{0.385000in}%
\pgfsys@useobject{currentmarker}{}%
\end{pgfscope}%
\end{pgfscope}%
\begin{pgfscope}%
\definecolor{textcolor}{rgb}{0.000000,0.000000,0.000000}%
\pgfsetstrokecolor{textcolor}%
\pgfsetfillcolor{textcolor}%
\pgftext[x=1.365625in,y=0.287778in,,top]{\color{textcolor}\rmfamily\fontsize{10.000000}{12.000000}\selectfont \(\displaystyle -0.75\)}%
\end{pgfscope}%
\begin{pgfscope}%
\pgfsetbuttcap%
\pgfsetroundjoin%
\definecolor{currentfill}{rgb}{0.000000,0.000000,0.000000}%
\pgfsetfillcolor{currentfill}%
\pgfsetlinewidth{0.803000pt}%
\definecolor{currentstroke}{rgb}{0.000000,0.000000,0.000000}%
\pgfsetstrokecolor{currentstroke}%
\pgfsetdash{}{0pt}%
\pgfsys@defobject{currentmarker}{\pgfqpoint{0.000000in}{-0.048611in}}{\pgfqpoint{0.000000in}{0.000000in}}{%
\pgfpathmoveto{\pgfqpoint{0.000000in}{0.000000in}}%
\pgfpathlineto{\pgfqpoint{0.000000in}{-0.048611in}}%
\pgfusepath{stroke,fill}%
}%
\begin{pgfscope}%
\pgfsys@transformshift{1.850000in}{0.385000in}%
\pgfsys@useobject{currentmarker}{}%
\end{pgfscope}%
\end{pgfscope}%
\begin{pgfscope}%
\definecolor{textcolor}{rgb}{0.000000,0.000000,0.000000}%
\pgfsetstrokecolor{textcolor}%
\pgfsetfillcolor{textcolor}%
\pgftext[x=1.850000in,y=0.287778in,,top]{\color{textcolor}\rmfamily\fontsize{10.000000}{12.000000}\selectfont \(\displaystyle -0.50\)}%
\end{pgfscope}%
\begin{pgfscope}%
\pgfsetbuttcap%
\pgfsetroundjoin%
\definecolor{currentfill}{rgb}{0.000000,0.000000,0.000000}%
\pgfsetfillcolor{currentfill}%
\pgfsetlinewidth{0.803000pt}%
\definecolor{currentstroke}{rgb}{0.000000,0.000000,0.000000}%
\pgfsetstrokecolor{currentstroke}%
\pgfsetdash{}{0pt}%
\pgfsys@defobject{currentmarker}{\pgfqpoint{0.000000in}{-0.048611in}}{\pgfqpoint{0.000000in}{0.000000in}}{%
\pgfpathmoveto{\pgfqpoint{0.000000in}{0.000000in}}%
\pgfpathlineto{\pgfqpoint{0.000000in}{-0.048611in}}%
\pgfusepath{stroke,fill}%
}%
\begin{pgfscope}%
\pgfsys@transformshift{2.334375in}{0.385000in}%
\pgfsys@useobject{currentmarker}{}%
\end{pgfscope}%
\end{pgfscope}%
\begin{pgfscope}%
\definecolor{textcolor}{rgb}{0.000000,0.000000,0.000000}%
\pgfsetstrokecolor{textcolor}%
\pgfsetfillcolor{textcolor}%
\pgftext[x=2.334375in,y=0.287778in,,top]{\color{textcolor}\rmfamily\fontsize{10.000000}{12.000000}\selectfont \(\displaystyle -0.25\)}%
\end{pgfscope}%
\begin{pgfscope}%
\pgfsetbuttcap%
\pgfsetroundjoin%
\definecolor{currentfill}{rgb}{0.000000,0.000000,0.000000}%
\pgfsetfillcolor{currentfill}%
\pgfsetlinewidth{0.803000pt}%
\definecolor{currentstroke}{rgb}{0.000000,0.000000,0.000000}%
\pgfsetstrokecolor{currentstroke}%
\pgfsetdash{}{0pt}%
\pgfsys@defobject{currentmarker}{\pgfqpoint{0.000000in}{-0.048611in}}{\pgfqpoint{0.000000in}{0.000000in}}{%
\pgfpathmoveto{\pgfqpoint{0.000000in}{0.000000in}}%
\pgfpathlineto{\pgfqpoint{0.000000in}{-0.048611in}}%
\pgfusepath{stroke,fill}%
}%
\begin{pgfscope}%
\pgfsys@transformshift{2.818750in}{0.385000in}%
\pgfsys@useobject{currentmarker}{}%
\end{pgfscope}%
\end{pgfscope}%
\begin{pgfscope}%
\definecolor{textcolor}{rgb}{0.000000,0.000000,0.000000}%
\pgfsetstrokecolor{textcolor}%
\pgfsetfillcolor{textcolor}%
\pgftext[x=2.818750in,y=0.287778in,,top]{\color{textcolor}\rmfamily\fontsize{10.000000}{12.000000}\selectfont \(\displaystyle 0.00\)}%
\end{pgfscope}%
\begin{pgfscope}%
\pgfsetbuttcap%
\pgfsetroundjoin%
\definecolor{currentfill}{rgb}{0.000000,0.000000,0.000000}%
\pgfsetfillcolor{currentfill}%
\pgfsetlinewidth{0.803000pt}%
\definecolor{currentstroke}{rgb}{0.000000,0.000000,0.000000}%
\pgfsetstrokecolor{currentstroke}%
\pgfsetdash{}{0pt}%
\pgfsys@defobject{currentmarker}{\pgfqpoint{0.000000in}{-0.048611in}}{\pgfqpoint{0.000000in}{0.000000in}}{%
\pgfpathmoveto{\pgfqpoint{0.000000in}{0.000000in}}%
\pgfpathlineto{\pgfqpoint{0.000000in}{-0.048611in}}%
\pgfusepath{stroke,fill}%
}%
\begin{pgfscope}%
\pgfsys@transformshift{3.303125in}{0.385000in}%
\pgfsys@useobject{currentmarker}{}%
\end{pgfscope}%
\end{pgfscope}%
\begin{pgfscope}%
\definecolor{textcolor}{rgb}{0.000000,0.000000,0.000000}%
\pgfsetstrokecolor{textcolor}%
\pgfsetfillcolor{textcolor}%
\pgftext[x=3.303125in,y=0.287778in,,top]{\color{textcolor}\rmfamily\fontsize{10.000000}{12.000000}\selectfont \(\displaystyle 0.25\)}%
\end{pgfscope}%
\begin{pgfscope}%
\pgfsetbuttcap%
\pgfsetroundjoin%
\definecolor{currentfill}{rgb}{0.000000,0.000000,0.000000}%
\pgfsetfillcolor{currentfill}%
\pgfsetlinewidth{0.803000pt}%
\definecolor{currentstroke}{rgb}{0.000000,0.000000,0.000000}%
\pgfsetstrokecolor{currentstroke}%
\pgfsetdash{}{0pt}%
\pgfsys@defobject{currentmarker}{\pgfqpoint{0.000000in}{-0.048611in}}{\pgfqpoint{0.000000in}{0.000000in}}{%
\pgfpathmoveto{\pgfqpoint{0.000000in}{0.000000in}}%
\pgfpathlineto{\pgfqpoint{0.000000in}{-0.048611in}}%
\pgfusepath{stroke,fill}%
}%
\begin{pgfscope}%
\pgfsys@transformshift{3.787500in}{0.385000in}%
\pgfsys@useobject{currentmarker}{}%
\end{pgfscope}%
\end{pgfscope}%
\begin{pgfscope}%
\definecolor{textcolor}{rgb}{0.000000,0.000000,0.000000}%
\pgfsetstrokecolor{textcolor}%
\pgfsetfillcolor{textcolor}%
\pgftext[x=3.787500in,y=0.287778in,,top]{\color{textcolor}\rmfamily\fontsize{10.000000}{12.000000}\selectfont \(\displaystyle 0.50\)}%
\end{pgfscope}%
\begin{pgfscope}%
\pgfsetbuttcap%
\pgfsetroundjoin%
\definecolor{currentfill}{rgb}{0.000000,0.000000,0.000000}%
\pgfsetfillcolor{currentfill}%
\pgfsetlinewidth{0.803000pt}%
\definecolor{currentstroke}{rgb}{0.000000,0.000000,0.000000}%
\pgfsetstrokecolor{currentstroke}%
\pgfsetdash{}{0pt}%
\pgfsys@defobject{currentmarker}{\pgfqpoint{0.000000in}{-0.048611in}}{\pgfqpoint{0.000000in}{0.000000in}}{%
\pgfpathmoveto{\pgfqpoint{0.000000in}{0.000000in}}%
\pgfpathlineto{\pgfqpoint{0.000000in}{-0.048611in}}%
\pgfusepath{stroke,fill}%
}%
\begin{pgfscope}%
\pgfsys@transformshift{4.271875in}{0.385000in}%
\pgfsys@useobject{currentmarker}{}%
\end{pgfscope}%
\end{pgfscope}%
\begin{pgfscope}%
\definecolor{textcolor}{rgb}{0.000000,0.000000,0.000000}%
\pgfsetstrokecolor{textcolor}%
\pgfsetfillcolor{textcolor}%
\pgftext[x=4.271875in,y=0.287778in,,top]{\color{textcolor}\rmfamily\fontsize{10.000000}{12.000000}\selectfont \(\displaystyle 0.75\)}%
\end{pgfscope}%
\begin{pgfscope}%
\pgfsetbuttcap%
\pgfsetroundjoin%
\definecolor{currentfill}{rgb}{0.000000,0.000000,0.000000}%
\pgfsetfillcolor{currentfill}%
\pgfsetlinewidth{0.803000pt}%
\definecolor{currentstroke}{rgb}{0.000000,0.000000,0.000000}%
\pgfsetstrokecolor{currentstroke}%
\pgfsetdash{}{0pt}%
\pgfsys@defobject{currentmarker}{\pgfqpoint{0.000000in}{-0.048611in}}{\pgfqpoint{0.000000in}{0.000000in}}{%
\pgfpathmoveto{\pgfqpoint{0.000000in}{0.000000in}}%
\pgfpathlineto{\pgfqpoint{0.000000in}{-0.048611in}}%
\pgfusepath{stroke,fill}%
}%
\begin{pgfscope}%
\pgfsys@transformshift{4.756250in}{0.385000in}%
\pgfsys@useobject{currentmarker}{}%
\end{pgfscope}%
\end{pgfscope}%
\begin{pgfscope}%
\definecolor{textcolor}{rgb}{0.000000,0.000000,0.000000}%
\pgfsetstrokecolor{textcolor}%
\pgfsetfillcolor{textcolor}%
\pgftext[x=4.756250in,y=0.287778in,,top]{\color{textcolor}\rmfamily\fontsize{10.000000}{12.000000}\selectfont \(\displaystyle 1.00\)}%
\end{pgfscope}%
\begin{pgfscope}%
\definecolor{textcolor}{rgb}{0.000000,0.000000,0.000000}%
\pgfsetstrokecolor{textcolor}%
\pgfsetfillcolor{textcolor}%
\pgftext[x=2.818750in,y=0.108766in,,top]{\color{textcolor}\rmfamily\fontsize{10.000000}{12.000000}\selectfont x}%
\end{pgfscope}%
\begin{pgfscope}%
\pgfsetbuttcap%
\pgfsetroundjoin%
\definecolor{currentfill}{rgb}{0.000000,0.000000,0.000000}%
\pgfsetfillcolor{currentfill}%
\pgfsetlinewidth{0.803000pt}%
\definecolor{currentstroke}{rgb}{0.000000,0.000000,0.000000}%
\pgfsetstrokecolor{currentstroke}%
\pgfsetdash{}{0pt}%
\pgfsys@defobject{currentmarker}{\pgfqpoint{-0.048611in}{0.000000in}}{\pgfqpoint{0.000000in}{0.000000in}}{%
\pgfpathmoveto{\pgfqpoint{0.000000in}{0.000000in}}%
\pgfpathlineto{\pgfqpoint{-0.048611in}{0.000000in}}%
\pgfusepath{stroke,fill}%
}%
\begin{pgfscope}%
\pgfsys@transformshift{0.687500in}{0.474833in}%
\pgfsys@useobject{currentmarker}{}%
\end{pgfscope}%
\end{pgfscope}%
\begin{pgfscope}%
\definecolor{textcolor}{rgb}{0.000000,0.000000,0.000000}%
\pgfsetstrokecolor{textcolor}%
\pgfsetfillcolor{textcolor}%
\pgftext[x=0.304783in,y=0.426608in,left,base]{\color{textcolor}\rmfamily\fontsize{10.000000}{12.000000}\selectfont \(\displaystyle -0.2\)}%
\end{pgfscope}%
\begin{pgfscope}%
\pgfsetbuttcap%
\pgfsetroundjoin%
\definecolor{currentfill}{rgb}{0.000000,0.000000,0.000000}%
\pgfsetfillcolor{currentfill}%
\pgfsetlinewidth{0.803000pt}%
\definecolor{currentstroke}{rgb}{0.000000,0.000000,0.000000}%
\pgfsetstrokecolor{currentstroke}%
\pgfsetdash{}{0pt}%
\pgfsys@defobject{currentmarker}{\pgfqpoint{-0.048611in}{0.000000in}}{\pgfqpoint{0.000000in}{0.000000in}}{%
\pgfpathmoveto{\pgfqpoint{0.000000in}{0.000000in}}%
\pgfpathlineto{\pgfqpoint{-0.048611in}{0.000000in}}%
\pgfusepath{stroke,fill}%
}%
\begin{pgfscope}%
\pgfsys@transformshift{0.687500in}{0.834167in}%
\pgfsys@useobject{currentmarker}{}%
\end{pgfscope}%
\end{pgfscope}%
\begin{pgfscope}%
\definecolor{textcolor}{rgb}{0.000000,0.000000,0.000000}%
\pgfsetstrokecolor{textcolor}%
\pgfsetfillcolor{textcolor}%
\pgftext[x=0.412808in,y=0.785941in,left,base]{\color{textcolor}\rmfamily\fontsize{10.000000}{12.000000}\selectfont \(\displaystyle 0.0\)}%
\end{pgfscope}%
\begin{pgfscope}%
\pgfsetbuttcap%
\pgfsetroundjoin%
\definecolor{currentfill}{rgb}{0.000000,0.000000,0.000000}%
\pgfsetfillcolor{currentfill}%
\pgfsetlinewidth{0.803000pt}%
\definecolor{currentstroke}{rgb}{0.000000,0.000000,0.000000}%
\pgfsetstrokecolor{currentstroke}%
\pgfsetdash{}{0pt}%
\pgfsys@defobject{currentmarker}{\pgfqpoint{-0.048611in}{0.000000in}}{\pgfqpoint{0.000000in}{0.000000in}}{%
\pgfpathmoveto{\pgfqpoint{0.000000in}{0.000000in}}%
\pgfpathlineto{\pgfqpoint{-0.048611in}{0.000000in}}%
\pgfusepath{stroke,fill}%
}%
\begin{pgfscope}%
\pgfsys@transformshift{0.687500in}{1.193500in}%
\pgfsys@useobject{currentmarker}{}%
\end{pgfscope}%
\end{pgfscope}%
\begin{pgfscope}%
\definecolor{textcolor}{rgb}{0.000000,0.000000,0.000000}%
\pgfsetstrokecolor{textcolor}%
\pgfsetfillcolor{textcolor}%
\pgftext[x=0.412808in,y=1.145275in,left,base]{\color{textcolor}\rmfamily\fontsize{10.000000}{12.000000}\selectfont \(\displaystyle 0.2\)}%
\end{pgfscope}%
\begin{pgfscope}%
\pgfsetbuttcap%
\pgfsetroundjoin%
\definecolor{currentfill}{rgb}{0.000000,0.000000,0.000000}%
\pgfsetfillcolor{currentfill}%
\pgfsetlinewidth{0.803000pt}%
\definecolor{currentstroke}{rgb}{0.000000,0.000000,0.000000}%
\pgfsetstrokecolor{currentstroke}%
\pgfsetdash{}{0pt}%
\pgfsys@defobject{currentmarker}{\pgfqpoint{-0.048611in}{0.000000in}}{\pgfqpoint{0.000000in}{0.000000in}}{%
\pgfpathmoveto{\pgfqpoint{0.000000in}{0.000000in}}%
\pgfpathlineto{\pgfqpoint{-0.048611in}{0.000000in}}%
\pgfusepath{stroke,fill}%
}%
\begin{pgfscope}%
\pgfsys@transformshift{0.687500in}{1.552833in}%
\pgfsys@useobject{currentmarker}{}%
\end{pgfscope}%
\end{pgfscope}%
\begin{pgfscope}%
\definecolor{textcolor}{rgb}{0.000000,0.000000,0.000000}%
\pgfsetstrokecolor{textcolor}%
\pgfsetfillcolor{textcolor}%
\pgftext[x=0.412808in,y=1.504608in,left,base]{\color{textcolor}\rmfamily\fontsize{10.000000}{12.000000}\selectfont \(\displaystyle 0.4\)}%
\end{pgfscope}%
\begin{pgfscope}%
\pgfsetbuttcap%
\pgfsetroundjoin%
\definecolor{currentfill}{rgb}{0.000000,0.000000,0.000000}%
\pgfsetfillcolor{currentfill}%
\pgfsetlinewidth{0.803000pt}%
\definecolor{currentstroke}{rgb}{0.000000,0.000000,0.000000}%
\pgfsetstrokecolor{currentstroke}%
\pgfsetdash{}{0pt}%
\pgfsys@defobject{currentmarker}{\pgfqpoint{-0.048611in}{0.000000in}}{\pgfqpoint{0.000000in}{0.000000in}}{%
\pgfpathmoveto{\pgfqpoint{0.000000in}{0.000000in}}%
\pgfpathlineto{\pgfqpoint{-0.048611in}{0.000000in}}%
\pgfusepath{stroke,fill}%
}%
\begin{pgfscope}%
\pgfsys@transformshift{0.687500in}{1.912167in}%
\pgfsys@useobject{currentmarker}{}%
\end{pgfscope}%
\end{pgfscope}%
\begin{pgfscope}%
\definecolor{textcolor}{rgb}{0.000000,0.000000,0.000000}%
\pgfsetstrokecolor{textcolor}%
\pgfsetfillcolor{textcolor}%
\pgftext[x=0.412808in,y=1.863941in,left,base]{\color{textcolor}\rmfamily\fontsize{10.000000}{12.000000}\selectfont \(\displaystyle 0.6\)}%
\end{pgfscope}%
\begin{pgfscope}%
\pgfsetbuttcap%
\pgfsetroundjoin%
\definecolor{currentfill}{rgb}{0.000000,0.000000,0.000000}%
\pgfsetfillcolor{currentfill}%
\pgfsetlinewidth{0.803000pt}%
\definecolor{currentstroke}{rgb}{0.000000,0.000000,0.000000}%
\pgfsetstrokecolor{currentstroke}%
\pgfsetdash{}{0pt}%
\pgfsys@defobject{currentmarker}{\pgfqpoint{-0.048611in}{0.000000in}}{\pgfqpoint{0.000000in}{0.000000in}}{%
\pgfpathmoveto{\pgfqpoint{0.000000in}{0.000000in}}%
\pgfpathlineto{\pgfqpoint{-0.048611in}{0.000000in}}%
\pgfusepath{stroke,fill}%
}%
\begin{pgfscope}%
\pgfsys@transformshift{0.687500in}{2.271500in}%
\pgfsys@useobject{currentmarker}{}%
\end{pgfscope}%
\end{pgfscope}%
\begin{pgfscope}%
\definecolor{textcolor}{rgb}{0.000000,0.000000,0.000000}%
\pgfsetstrokecolor{textcolor}%
\pgfsetfillcolor{textcolor}%
\pgftext[x=0.412808in,y=2.223275in,left,base]{\color{textcolor}\rmfamily\fontsize{10.000000}{12.000000}\selectfont \(\displaystyle 0.8\)}%
\end{pgfscope}%
\begin{pgfscope}%
\pgfsetbuttcap%
\pgfsetroundjoin%
\definecolor{currentfill}{rgb}{0.000000,0.000000,0.000000}%
\pgfsetfillcolor{currentfill}%
\pgfsetlinewidth{0.803000pt}%
\definecolor{currentstroke}{rgb}{0.000000,0.000000,0.000000}%
\pgfsetstrokecolor{currentstroke}%
\pgfsetdash{}{0pt}%
\pgfsys@defobject{currentmarker}{\pgfqpoint{-0.048611in}{0.000000in}}{\pgfqpoint{0.000000in}{0.000000in}}{%
\pgfpathmoveto{\pgfqpoint{0.000000in}{0.000000in}}%
\pgfpathlineto{\pgfqpoint{-0.048611in}{0.000000in}}%
\pgfusepath{stroke,fill}%
}%
\begin{pgfscope}%
\pgfsys@transformshift{0.687500in}{2.630833in}%
\pgfsys@useobject{currentmarker}{}%
\end{pgfscope}%
\end{pgfscope}%
\begin{pgfscope}%
\definecolor{textcolor}{rgb}{0.000000,0.000000,0.000000}%
\pgfsetstrokecolor{textcolor}%
\pgfsetfillcolor{textcolor}%
\pgftext[x=0.412808in,y=2.582608in,left,base]{\color{textcolor}\rmfamily\fontsize{10.000000}{12.000000}\selectfont \(\displaystyle 1.0\)}%
\end{pgfscope}%
\begin{pgfscope}%
\pgfsetbuttcap%
\pgfsetroundjoin%
\definecolor{currentfill}{rgb}{0.000000,0.000000,0.000000}%
\pgfsetfillcolor{currentfill}%
\pgfsetlinewidth{0.803000pt}%
\definecolor{currentstroke}{rgb}{0.000000,0.000000,0.000000}%
\pgfsetstrokecolor{currentstroke}%
\pgfsetdash{}{0pt}%
\pgfsys@defobject{currentmarker}{\pgfqpoint{-0.048611in}{0.000000in}}{\pgfqpoint{0.000000in}{0.000000in}}{%
\pgfpathmoveto{\pgfqpoint{0.000000in}{0.000000in}}%
\pgfpathlineto{\pgfqpoint{-0.048611in}{0.000000in}}%
\pgfusepath{stroke,fill}%
}%
\begin{pgfscope}%
\pgfsys@transformshift{0.687500in}{2.990167in}%
\pgfsys@useobject{currentmarker}{}%
\end{pgfscope}%
\end{pgfscope}%
\begin{pgfscope}%
\definecolor{textcolor}{rgb}{0.000000,0.000000,0.000000}%
\pgfsetstrokecolor{textcolor}%
\pgfsetfillcolor{textcolor}%
\pgftext[x=0.412808in,y=2.941941in,left,base]{\color{textcolor}\rmfamily\fontsize{10.000000}{12.000000}\selectfont \(\displaystyle 1.2\)}%
\end{pgfscope}%
\begin{pgfscope}%
\definecolor{textcolor}{rgb}{0.000000,0.000000,0.000000}%
\pgfsetstrokecolor{textcolor}%
\pgfsetfillcolor{textcolor}%
\pgftext[x=0.249228in,y=1.732500in,,bottom,rotate=90.000000]{\color{textcolor}\rmfamily\fontsize{10.000000}{12.000000}\selectfont y}%
\end{pgfscope}%
\begin{pgfscope}%
\pgfpathrectangle{\pgfqpoint{0.687500in}{0.385000in}}{\pgfqpoint{4.262500in}{2.695000in}}%
\pgfusepath{clip}%
\pgfsetrectcap%
\pgfsetroundjoin%
\pgfsetlinewidth{1.505625pt}%
\definecolor{currentstroke}{rgb}{0.121569,0.466667,0.705882}%
\pgfsetstrokecolor{currentstroke}%
\pgfsetdash{}{0pt}%
\pgfpathmoveto{\pgfqpoint{0.881250in}{0.903269in}}%
\pgfpathlineto{\pgfqpoint{1.017557in}{0.913644in}}%
\pgfpathlineto{\pgfqpoint{1.134391in}{0.924478in}}%
\pgfpathlineto{\pgfqpoint{1.251225in}{0.937638in}}%
\pgfpathlineto{\pgfqpoint{1.348587in}{0.950877in}}%
\pgfpathlineto{\pgfqpoint{1.426476in}{0.963336in}}%
\pgfpathlineto{\pgfqpoint{1.504366in}{0.977838in}}%
\pgfpathlineto{\pgfqpoint{1.582255in}{0.994839in}}%
\pgfpathlineto{\pgfqpoint{1.640672in}{1.009574in}}%
\pgfpathlineto{\pgfqpoint{1.699089in}{1.026346in}}%
\pgfpathlineto{\pgfqpoint{1.757506in}{1.045528in}}%
\pgfpathlineto{\pgfqpoint{1.815923in}{1.067578in}}%
\pgfpathlineto{\pgfqpoint{1.854868in}{1.084143in}}%
\pgfpathlineto{\pgfqpoint{1.893813in}{1.102428in}}%
\pgfpathlineto{\pgfqpoint{1.932758in}{1.122659in}}%
\pgfpathlineto{\pgfqpoint{1.971702in}{1.145101in}}%
\pgfpathlineto{\pgfqpoint{2.010647in}{1.170055in}}%
\pgfpathlineto{\pgfqpoint{2.049592in}{1.197870in}}%
\pgfpathlineto{\pgfqpoint{2.088536in}{1.228948in}}%
\pgfpathlineto{\pgfqpoint{2.127481in}{1.263748in}}%
\pgfpathlineto{\pgfqpoint{2.166426in}{1.302794in}}%
\pgfpathlineto{\pgfqpoint{2.205371in}{1.346677in}}%
\pgfpathlineto{\pgfqpoint{2.244315in}{1.396056in}}%
\pgfpathlineto{\pgfqpoint{2.283260in}{1.451647in}}%
\pgfpathlineto{\pgfqpoint{2.322205in}{1.514206in}}%
\pgfpathlineto{\pgfqpoint{2.361149in}{1.584486in}}%
\pgfpathlineto{\pgfqpoint{2.400094in}{1.663167in}}%
\pgfpathlineto{\pgfqpoint{2.439039in}{1.750738in}}%
\pgfpathlineto{\pgfqpoint{2.477984in}{1.847321in}}%
\pgfpathlineto{\pgfqpoint{2.516928in}{1.952417in}}%
\pgfpathlineto{\pgfqpoint{2.555873in}{2.064579in}}%
\pgfpathlineto{\pgfqpoint{2.653235in}{2.353617in}}%
\pgfpathlineto{\pgfqpoint{2.672707in}{2.407372in}}%
\pgfpathlineto{\pgfqpoint{2.692180in}{2.457628in}}%
\pgfpathlineto{\pgfqpoint{2.711652in}{2.503331in}}%
\pgfpathlineto{\pgfqpoint{2.731124in}{2.543430in}}%
\pgfpathlineto{\pgfqpoint{2.750597in}{2.576924in}}%
\pgfpathlineto{\pgfqpoint{2.770069in}{2.602918in}}%
\pgfpathlineto{\pgfqpoint{2.789541in}{2.620683in}}%
\pgfpathlineto{\pgfqpoint{2.809014in}{2.629700in}}%
\pgfpathlineto{\pgfqpoint{2.828486in}{2.629700in}}%
\pgfpathlineto{\pgfqpoint{2.847959in}{2.620683in}}%
\pgfpathlineto{\pgfqpoint{2.867431in}{2.602918in}}%
\pgfpathlineto{\pgfqpoint{2.886903in}{2.576924in}}%
\pgfpathlineto{\pgfqpoint{2.906376in}{2.543430in}}%
\pgfpathlineto{\pgfqpoint{2.925848in}{2.503331in}}%
\pgfpathlineto{\pgfqpoint{2.945320in}{2.457628in}}%
\pgfpathlineto{\pgfqpoint{2.964793in}{2.407372in}}%
\pgfpathlineto{\pgfqpoint{3.003737in}{2.297371in}}%
\pgfpathlineto{\pgfqpoint{3.120572in}{1.952417in}}%
\pgfpathlineto{\pgfqpoint{3.159516in}{1.847321in}}%
\pgfpathlineto{\pgfqpoint{3.198461in}{1.750738in}}%
\pgfpathlineto{\pgfqpoint{3.237406in}{1.663167in}}%
\pgfpathlineto{\pgfqpoint{3.276351in}{1.584486in}}%
\pgfpathlineto{\pgfqpoint{3.315295in}{1.514206in}}%
\pgfpathlineto{\pgfqpoint{3.354240in}{1.451647in}}%
\pgfpathlineto{\pgfqpoint{3.393185in}{1.396056in}}%
\pgfpathlineto{\pgfqpoint{3.432129in}{1.346677in}}%
\pgfpathlineto{\pgfqpoint{3.471074in}{1.302794in}}%
\pgfpathlineto{\pgfqpoint{3.510019in}{1.263748in}}%
\pgfpathlineto{\pgfqpoint{3.548964in}{1.228948in}}%
\pgfpathlineto{\pgfqpoint{3.587908in}{1.197870in}}%
\pgfpathlineto{\pgfqpoint{3.626853in}{1.170055in}}%
\pgfpathlineto{\pgfqpoint{3.665798in}{1.145101in}}%
\pgfpathlineto{\pgfqpoint{3.704742in}{1.122659in}}%
\pgfpathlineto{\pgfqpoint{3.743687in}{1.102428in}}%
\pgfpathlineto{\pgfqpoint{3.782632in}{1.084143in}}%
\pgfpathlineto{\pgfqpoint{3.841049in}{1.059877in}}%
\pgfpathlineto{\pgfqpoint{3.899466in}{1.038840in}}%
\pgfpathlineto{\pgfqpoint{3.957883in}{1.020508in}}%
\pgfpathlineto{\pgfqpoint{4.016300in}{1.004452in}}%
\pgfpathlineto{\pgfqpoint{4.074717in}{0.990325in}}%
\pgfpathlineto{\pgfqpoint{4.152607in}{0.973998in}}%
\pgfpathlineto{\pgfqpoint{4.230496in}{0.960045in}}%
\pgfpathlineto{\pgfqpoint{4.327858in}{0.945299in}}%
\pgfpathlineto{\pgfqpoint{4.425220in}{0.932955in}}%
\pgfpathlineto{\pgfqpoint{4.542054in}{0.920637in}}%
\pgfpathlineto{\pgfqpoint{4.678361in}{0.908933in}}%
\pgfpathlineto{\pgfqpoint{4.756250in}{0.903269in}}%
\pgfpathlineto{\pgfqpoint{4.756250in}{0.903269in}}%
\pgfusepath{stroke}%
\end{pgfscope}%
\begin{pgfscope}%
\pgfpathrectangle{\pgfqpoint{0.687500in}{0.385000in}}{\pgfqpoint{4.262500in}{2.695000in}}%
\pgfusepath{clip}%
\pgfsetbuttcap%
\pgfsetroundjoin%
\definecolor{currentfill}{rgb}{1.000000,0.498039,0.054902}%
\pgfsetfillcolor{currentfill}%
\pgfsetlinewidth{1.003750pt}%
\definecolor{currentstroke}{rgb}{1.000000,0.498039,0.054902}%
\pgfsetstrokecolor{currentstroke}%
\pgfsetdash{}{0pt}%
\pgfsys@defobject{currentmarker}{\pgfqpoint{-0.020833in}{-0.020833in}}{\pgfqpoint{0.020833in}{0.020833in}}{%
\pgfpathmoveto{\pgfqpoint{0.000000in}{-0.020833in}}%
\pgfpathcurveto{\pgfqpoint{0.005525in}{-0.020833in}}{\pgfqpoint{0.010825in}{-0.018638in}}{\pgfqpoint{0.014731in}{-0.014731in}}%
\pgfpathcurveto{\pgfqpoint{0.018638in}{-0.010825in}}{\pgfqpoint{0.020833in}{-0.005525in}}{\pgfqpoint{0.020833in}{0.000000in}}%
\pgfpathcurveto{\pgfqpoint{0.020833in}{0.005525in}}{\pgfqpoint{0.018638in}{0.010825in}}{\pgfqpoint{0.014731in}{0.014731in}}%
\pgfpathcurveto{\pgfqpoint{0.010825in}{0.018638in}}{\pgfqpoint{0.005525in}{0.020833in}}{\pgfqpoint{0.000000in}{0.020833in}}%
\pgfpathcurveto{\pgfqpoint{-0.005525in}{0.020833in}}{\pgfqpoint{-0.010825in}{0.018638in}}{\pgfqpoint{-0.014731in}{0.014731in}}%
\pgfpathcurveto{\pgfqpoint{-0.018638in}{0.010825in}}{\pgfqpoint{-0.020833in}{0.005525in}}{\pgfqpoint{-0.020833in}{0.000000in}}%
\pgfpathcurveto{\pgfqpoint{-0.020833in}{-0.005525in}}{\pgfqpoint{-0.018638in}{-0.010825in}}{\pgfqpoint{-0.014731in}{-0.014731in}}%
\pgfpathcurveto{\pgfqpoint{-0.010825in}{-0.018638in}}{\pgfqpoint{-0.005525in}{-0.020833in}}{\pgfqpoint{0.000000in}{-0.020833in}}%
\pgfpathclose%
\pgfusepath{stroke,fill}%
}%
\begin{pgfscope}%
\pgfsys@transformshift{0.881250in}{0.903269in}%
\pgfsys@useobject{currentmarker}{}%
\end{pgfscope}%
\begin{pgfscope}%
\pgfsys@transformshift{1.434821in}{0.964785in}%
\pgfsys@useobject{currentmarker}{}%
\end{pgfscope}%
\begin{pgfscope}%
\pgfsys@transformshift{1.988393in}{1.155468in}%
\pgfsys@useobject{currentmarker}{}%
\end{pgfscope}%
\begin{pgfscope}%
\pgfsys@transformshift{2.541964in}{2.023851in}%
\pgfsys@useobject{currentmarker}{}%
\end{pgfscope}%
\begin{pgfscope}%
\pgfsys@transformshift{3.095536in}{2.023851in}%
\pgfsys@useobject{currentmarker}{}%
\end{pgfscope}%
\begin{pgfscope}%
\pgfsys@transformshift{3.649107in}{1.155468in}%
\pgfsys@useobject{currentmarker}{}%
\end{pgfscope}%
\begin{pgfscope}%
\pgfsys@transformshift{4.202679in}{0.964785in}%
\pgfsys@useobject{currentmarker}{}%
\end{pgfscope}%
\end{pgfscope}%
\begin{pgfscope}%
\pgfpathrectangle{\pgfqpoint{0.687500in}{0.385000in}}{\pgfqpoint{4.262500in}{2.695000in}}%
\pgfusepath{clip}%
\pgfsetrectcap%
\pgfsetroundjoin%
\pgfsetlinewidth{1.505625pt}%
\definecolor{currentstroke}{rgb}{0.172549,0.627451,0.172549}%
\pgfsetstrokecolor{currentstroke}%
\pgfsetdash{}{0pt}%
\pgfpathmoveto{\pgfqpoint{0.881250in}{0.903269in}}%
\pgfpathlineto{\pgfqpoint{0.900722in}{0.970053in}}%
\pgfpathlineto{\pgfqpoint{0.920195in}{1.027476in}}%
\pgfpathlineto{\pgfqpoint{0.939667in}{1.076231in}}%
\pgfpathlineto{\pgfqpoint{0.959139in}{1.116979in}}%
\pgfpathlineto{\pgfqpoint{0.978612in}{1.150353in}}%
\pgfpathlineto{\pgfqpoint{0.998084in}{1.176960in}}%
\pgfpathlineto{\pgfqpoint{1.017557in}{1.197380in}}%
\pgfpathlineto{\pgfqpoint{1.037029in}{1.212165in}}%
\pgfpathlineto{\pgfqpoint{1.056501in}{1.221844in}}%
\pgfpathlineto{\pgfqpoint{1.075974in}{1.226920in}}%
\pgfpathlineto{\pgfqpoint{1.095446in}{1.227871in}}%
\pgfpathlineto{\pgfqpoint{1.114918in}{1.225152in}}%
\pgfpathlineto{\pgfqpoint{1.134391in}{1.219195in}}%
\pgfpathlineto{\pgfqpoint{1.153863in}{1.210411in}}%
\pgfpathlineto{\pgfqpoint{1.173335in}{1.199186in}}%
\pgfpathlineto{\pgfqpoint{1.192808in}{1.185888in}}%
\pgfpathlineto{\pgfqpoint{1.231753in}{1.154435in}}%
\pgfpathlineto{\pgfqpoint{1.270697in}{1.118582in}}%
\pgfpathlineto{\pgfqpoint{1.387531in}{1.005499in}}%
\pgfpathlineto{\pgfqpoint{1.426476in}{0.971564in}}%
\pgfpathlineto{\pgfqpoint{1.465421in}{0.941692in}}%
\pgfpathlineto{\pgfqpoint{1.484893in}{0.928591in}}%
\pgfpathlineto{\pgfqpoint{1.504366in}{0.916854in}}%
\pgfpathlineto{\pgfqpoint{1.523838in}{0.906571in}}%
\pgfpathlineto{\pgfqpoint{1.543310in}{0.897820in}}%
\pgfpathlineto{\pgfqpoint{1.562783in}{0.890668in}}%
\pgfpathlineto{\pgfqpoint{1.582255in}{0.885172in}}%
\pgfpathlineto{\pgfqpoint{1.601727in}{0.881376in}}%
\pgfpathlineto{\pgfqpoint{1.621200in}{0.879318in}}%
\pgfpathlineto{\pgfqpoint{1.640672in}{0.879024in}}%
\pgfpathlineto{\pgfqpoint{1.660144in}{0.880510in}}%
\pgfpathlineto{\pgfqpoint{1.679617in}{0.883787in}}%
\pgfpathlineto{\pgfqpoint{1.699089in}{0.888853in}}%
\pgfpathlineto{\pgfqpoint{1.718562in}{0.895703in}}%
\pgfpathlineto{\pgfqpoint{1.738034in}{0.904321in}}%
\pgfpathlineto{\pgfqpoint{1.757506in}{0.914686in}}%
\pgfpathlineto{\pgfqpoint{1.776979in}{0.926770in}}%
\pgfpathlineto{\pgfqpoint{1.796451in}{0.940537in}}%
\pgfpathlineto{\pgfqpoint{1.815923in}{0.955947in}}%
\pgfpathlineto{\pgfqpoint{1.835396in}{0.972955in}}%
\pgfpathlineto{\pgfqpoint{1.874340in}{1.011555in}}%
\pgfpathlineto{\pgfqpoint{1.913285in}{1.055870in}}%
\pgfpathlineto{\pgfqpoint{1.952230in}{1.105371in}}%
\pgfpathlineto{\pgfqpoint{1.991175in}{1.159472in}}%
\pgfpathlineto{\pgfqpoint{2.030119in}{1.217541in}}%
\pgfpathlineto{\pgfqpoint{2.069064in}{1.278912in}}%
\pgfpathlineto{\pgfqpoint{2.127481in}{1.375635in}}%
\pgfpathlineto{\pgfqpoint{2.224843in}{1.543309in}}%
\pgfpathlineto{\pgfqpoint{2.302732in}{1.676797in}}%
\pgfpathlineto{\pgfqpoint{2.361149in}{1.772864in}}%
\pgfpathlineto{\pgfqpoint{2.400094in}{1.833743in}}%
\pgfpathlineto{\pgfqpoint{2.439039in}{1.891368in}}%
\pgfpathlineto{\pgfqpoint{2.477984in}{1.945165in}}%
\pgfpathlineto{\pgfqpoint{2.516928in}{1.994601in}}%
\pgfpathlineto{\pgfqpoint{2.555873in}{2.039192in}}%
\pgfpathlineto{\pgfqpoint{2.594818in}{2.078501in}}%
\pgfpathlineto{\pgfqpoint{2.633763in}{2.112149in}}%
\pgfpathlineto{\pgfqpoint{2.653235in}{2.126746in}}%
\pgfpathlineto{\pgfqpoint{2.672707in}{2.139811in}}%
\pgfpathlineto{\pgfqpoint{2.692180in}{2.151312in}}%
\pgfpathlineto{\pgfqpoint{2.711652in}{2.161221in}}%
\pgfpathlineto{\pgfqpoint{2.731124in}{2.169515in}}%
\pgfpathlineto{\pgfqpoint{2.750597in}{2.176174in}}%
\pgfpathlineto{\pgfqpoint{2.770069in}{2.181183in}}%
\pgfpathlineto{\pgfqpoint{2.789541in}{2.184528in}}%
\pgfpathlineto{\pgfqpoint{2.809014in}{2.186203in}}%
\pgfpathlineto{\pgfqpoint{2.828486in}{2.186203in}}%
\pgfpathlineto{\pgfqpoint{2.847959in}{2.184528in}}%
\pgfpathlineto{\pgfqpoint{2.867431in}{2.181183in}}%
\pgfpathlineto{\pgfqpoint{2.886903in}{2.176174in}}%
\pgfpathlineto{\pgfqpoint{2.906376in}{2.169515in}}%
\pgfpathlineto{\pgfqpoint{2.925848in}{2.161221in}}%
\pgfpathlineto{\pgfqpoint{2.945320in}{2.151312in}}%
\pgfpathlineto{\pgfqpoint{2.964793in}{2.139811in}}%
\pgfpathlineto{\pgfqpoint{2.984265in}{2.126746in}}%
\pgfpathlineto{\pgfqpoint{3.003737in}{2.112149in}}%
\pgfpathlineto{\pgfqpoint{3.023210in}{2.096055in}}%
\pgfpathlineto{\pgfqpoint{3.062155in}{2.059532in}}%
\pgfpathlineto{\pgfqpoint{3.101099in}{2.017531in}}%
\pgfpathlineto{\pgfqpoint{3.140044in}{1.970460in}}%
\pgfpathlineto{\pgfqpoint{3.178989in}{1.918780in}}%
\pgfpathlineto{\pgfqpoint{3.217933in}{1.862999in}}%
\pgfpathlineto{\pgfqpoint{3.256878in}{1.803673in}}%
\pgfpathlineto{\pgfqpoint{3.315295in}{1.709345in}}%
\pgfpathlineto{\pgfqpoint{3.393185in}{1.577002in}}%
\pgfpathlineto{\pgfqpoint{3.529491in}{1.342890in}}%
\pgfpathlineto{\pgfqpoint{3.587908in}{1.247857in}}%
\pgfpathlineto{\pgfqpoint{3.626853in}{1.188052in}}%
\pgfpathlineto{\pgfqpoint{3.665798in}{1.131885in}}%
\pgfpathlineto{\pgfqpoint{3.704742in}{1.080008in}}%
\pgfpathlineto{\pgfqpoint{3.743687in}{1.033030in}}%
\pgfpathlineto{\pgfqpoint{3.782632in}{0.991510in}}%
\pgfpathlineto{\pgfqpoint{3.802104in}{0.972955in}}%
\pgfpathlineto{\pgfqpoint{3.821577in}{0.955947in}}%
\pgfpathlineto{\pgfqpoint{3.841049in}{0.940537in}}%
\pgfpathlineto{\pgfqpoint{3.860521in}{0.926770in}}%
\pgfpathlineto{\pgfqpoint{3.879994in}{0.914686in}}%
\pgfpathlineto{\pgfqpoint{3.899466in}{0.904321in}}%
\pgfpathlineto{\pgfqpoint{3.918938in}{0.895703in}}%
\pgfpathlineto{\pgfqpoint{3.938411in}{0.888853in}}%
\pgfpathlineto{\pgfqpoint{3.957883in}{0.883787in}}%
\pgfpathlineto{\pgfqpoint{3.977356in}{0.880510in}}%
\pgfpathlineto{\pgfqpoint{3.996828in}{0.879024in}}%
\pgfpathlineto{\pgfqpoint{4.016300in}{0.879318in}}%
\pgfpathlineto{\pgfqpoint{4.035773in}{0.881376in}}%
\pgfpathlineto{\pgfqpoint{4.055245in}{0.885172in}}%
\pgfpathlineto{\pgfqpoint{4.074717in}{0.890668in}}%
\pgfpathlineto{\pgfqpoint{4.094190in}{0.897820in}}%
\pgfpathlineto{\pgfqpoint{4.113662in}{0.906571in}}%
\pgfpathlineto{\pgfqpoint{4.133134in}{0.916854in}}%
\pgfpathlineto{\pgfqpoint{4.152607in}{0.928591in}}%
\pgfpathlineto{\pgfqpoint{4.191552in}{0.956055in}}%
\pgfpathlineto{\pgfqpoint{4.230496in}{0.988093in}}%
\pgfpathlineto{\pgfqpoint{4.269441in}{1.023626in}}%
\pgfpathlineto{\pgfqpoint{4.347330in}{1.099714in}}%
\pgfpathlineto{\pgfqpoint{4.386275in}{1.136913in}}%
\pgfpathlineto{\pgfqpoint{4.425220in}{1.170862in}}%
\pgfpathlineto{\pgfqpoint{4.444692in}{1.185888in}}%
\pgfpathlineto{\pgfqpoint{4.464165in}{1.199186in}}%
\pgfpathlineto{\pgfqpoint{4.483637in}{1.210411in}}%
\pgfpathlineto{\pgfqpoint{4.503109in}{1.219195in}}%
\pgfpathlineto{\pgfqpoint{4.522582in}{1.225152in}}%
\pgfpathlineto{\pgfqpoint{4.542054in}{1.227871in}}%
\pgfpathlineto{\pgfqpoint{4.561526in}{1.226920in}}%
\pgfpathlineto{\pgfqpoint{4.580999in}{1.221844in}}%
\pgfpathlineto{\pgfqpoint{4.600471in}{1.212165in}}%
\pgfpathlineto{\pgfqpoint{4.619943in}{1.197380in}}%
\pgfpathlineto{\pgfqpoint{4.639416in}{1.176960in}}%
\pgfpathlineto{\pgfqpoint{4.658888in}{1.150353in}}%
\pgfpathlineto{\pgfqpoint{4.678361in}{1.116979in}}%
\pgfpathlineto{\pgfqpoint{4.697833in}{1.076231in}}%
\pgfpathlineto{\pgfqpoint{4.717305in}{1.027476in}}%
\pgfpathlineto{\pgfqpoint{4.736778in}{0.970053in}}%
\pgfpathlineto{\pgfqpoint{4.756250in}{0.903269in}}%
\pgfpathlineto{\pgfqpoint{4.756250in}{0.903269in}}%
\pgfusepath{stroke}%
\end{pgfscope}%
\begin{pgfscope}%
\pgfpathrectangle{\pgfqpoint{0.687500in}{0.385000in}}{\pgfqpoint{4.262500in}{2.695000in}}%
\pgfusepath{clip}%
\pgfsetbuttcap%
\pgfsetroundjoin%
\definecolor{currentfill}{rgb}{0.839216,0.152941,0.156863}%
\pgfsetfillcolor{currentfill}%
\pgfsetlinewidth{1.003750pt}%
\definecolor{currentstroke}{rgb}{0.839216,0.152941,0.156863}%
\pgfsetstrokecolor{currentstroke}%
\pgfsetdash{}{0pt}%
\pgfsys@defobject{currentmarker}{\pgfqpoint{-0.020833in}{-0.020833in}}{\pgfqpoint{0.020833in}{0.020833in}}{%
\pgfpathmoveto{\pgfqpoint{0.000000in}{-0.020833in}}%
\pgfpathcurveto{\pgfqpoint{0.005525in}{-0.020833in}}{\pgfqpoint{0.010825in}{-0.018638in}}{\pgfqpoint{0.014731in}{-0.014731in}}%
\pgfpathcurveto{\pgfqpoint{0.018638in}{-0.010825in}}{\pgfqpoint{0.020833in}{-0.005525in}}{\pgfqpoint{0.020833in}{0.000000in}}%
\pgfpathcurveto{\pgfqpoint{0.020833in}{0.005525in}}{\pgfqpoint{0.018638in}{0.010825in}}{\pgfqpoint{0.014731in}{0.014731in}}%
\pgfpathcurveto{\pgfqpoint{0.010825in}{0.018638in}}{\pgfqpoint{0.005525in}{0.020833in}}{\pgfqpoint{0.000000in}{0.020833in}}%
\pgfpathcurveto{\pgfqpoint{-0.005525in}{0.020833in}}{\pgfqpoint{-0.010825in}{0.018638in}}{\pgfqpoint{-0.014731in}{0.014731in}}%
\pgfpathcurveto{\pgfqpoint{-0.018638in}{0.010825in}}{\pgfqpoint{-0.020833in}{0.005525in}}{\pgfqpoint{-0.020833in}{0.000000in}}%
\pgfpathcurveto{\pgfqpoint{-0.020833in}{-0.005525in}}{\pgfqpoint{-0.018638in}{-0.010825in}}{\pgfqpoint{-0.014731in}{-0.014731in}}%
\pgfpathcurveto{\pgfqpoint{-0.010825in}{-0.018638in}}{\pgfqpoint{-0.005525in}{-0.020833in}}{\pgfqpoint{0.000000in}{-0.020833in}}%
\pgfpathclose%
\pgfusepath{stroke,fill}%
}%
\begin{pgfscope}%
\pgfsys@transformshift{0.881250in}{0.903269in}%
\pgfsys@useobject{currentmarker}{}%
\end{pgfscope}%
\begin{pgfscope}%
\pgfsys@transformshift{1.311806in}{0.945599in}%
\pgfsys@useobject{currentmarker}{}%
\end{pgfscope}%
\begin{pgfscope}%
\pgfsys@transformshift{1.742361in}{1.040300in}%
\pgfsys@useobject{currentmarker}{}%
\end{pgfscope}%
\begin{pgfscope}%
\pgfsys@transformshift{2.172917in}{1.309755in}%
\pgfsys@useobject{currentmarker}{}%
\end{pgfscope}%
\begin{pgfscope}%
\pgfsys@transformshift{2.603472in}{2.207091in}%
\pgfsys@useobject{currentmarker}{}%
\end{pgfscope}%
\begin{pgfscope}%
\pgfsys@transformshift{3.034028in}{2.207091in}%
\pgfsys@useobject{currentmarker}{}%
\end{pgfscope}%
\begin{pgfscope}%
\pgfsys@transformshift{3.464583in}{1.309755in}%
\pgfsys@useobject{currentmarker}{}%
\end{pgfscope}%
\begin{pgfscope}%
\pgfsys@transformshift{3.895139in}{1.040300in}%
\pgfsys@useobject{currentmarker}{}%
\end{pgfscope}%
\begin{pgfscope}%
\pgfsys@transformshift{4.325694in}{0.945599in}%
\pgfsys@useobject{currentmarker}{}%
\end{pgfscope}%
\end{pgfscope}%
\begin{pgfscope}%
\pgfpathrectangle{\pgfqpoint{0.687500in}{0.385000in}}{\pgfqpoint{4.262500in}{2.695000in}}%
\pgfusepath{clip}%
\pgfsetrectcap%
\pgfsetroundjoin%
\pgfsetlinewidth{1.505625pt}%
\definecolor{currentstroke}{rgb}{0.580392,0.403922,0.741176}%
\pgfsetstrokecolor{currentstroke}%
\pgfsetdash{}{0pt}%
\pgfpathmoveto{\pgfqpoint{0.881250in}{0.903269in}}%
\pgfpathlineto{\pgfqpoint{0.900722in}{0.743344in}}%
\pgfpathlineto{\pgfqpoint{0.920195in}{0.618023in}}%
\pgfpathlineto{\pgfqpoint{0.939667in}{0.523088in}}%
\pgfpathlineto{\pgfqpoint{0.959139in}{0.454665in}}%
\pgfpathlineto{\pgfqpoint{0.978612in}{0.409204in}}%
\pgfpathlineto{\pgfqpoint{0.998084in}{0.383458in}}%
\pgfpathlineto{\pgfqpoint{1.016411in}{0.375000in}}%
\pgfpathmoveto{\pgfqpoint{1.019580in}{0.375000in}}%
\pgfpathlineto{\pgfqpoint{1.037029in}{0.379559in}}%
\pgfpathlineto{\pgfqpoint{1.056501in}{0.396289in}}%
\pgfpathlineto{\pgfqpoint{1.075974in}{0.422470in}}%
\pgfpathlineto{\pgfqpoint{1.095446in}{0.456136in}}%
\pgfpathlineto{\pgfqpoint{1.114918in}{0.495528in}}%
\pgfpathlineto{\pgfqpoint{1.134391in}{0.539085in}}%
\pgfpathlineto{\pgfqpoint{1.173335in}{0.633346in}}%
\pgfpathlineto{\pgfqpoint{1.231753in}{0.776771in}}%
\pgfpathlineto{\pgfqpoint{1.270697in}{0.864687in}}%
\pgfpathlineto{\pgfqpoint{1.290170in}{0.904731in}}%
\pgfpathlineto{\pgfqpoint{1.309642in}{0.941693in}}%
\pgfpathlineto{\pgfqpoint{1.329114in}{0.975331in}}%
\pgfpathlineto{\pgfqpoint{1.348587in}{1.005481in}}%
\pgfpathlineto{\pgfqpoint{1.368059in}{1.032050in}}%
\pgfpathlineto{\pgfqpoint{1.387531in}{1.055007in}}%
\pgfpathlineto{\pgfqpoint{1.407004in}{1.074382in}}%
\pgfpathlineto{\pgfqpoint{1.426476in}{1.090250in}}%
\pgfpathlineto{\pgfqpoint{1.445948in}{1.102735in}}%
\pgfpathlineto{\pgfqpoint{1.465421in}{1.111994in}}%
\pgfpathlineto{\pgfqpoint{1.484893in}{1.118221in}}%
\pgfpathlineto{\pgfqpoint{1.504366in}{1.121635in}}%
\pgfpathlineto{\pgfqpoint{1.523838in}{1.122478in}}%
\pgfpathlineto{\pgfqpoint{1.543310in}{1.121012in}}%
\pgfpathlineto{\pgfqpoint{1.562783in}{1.117510in}}%
\pgfpathlineto{\pgfqpoint{1.582255in}{1.112257in}}%
\pgfpathlineto{\pgfqpoint{1.621200in}{1.097669in}}%
\pgfpathlineto{\pgfqpoint{1.660144in}{1.079600in}}%
\pgfpathlineto{\pgfqpoint{1.738034in}{1.042168in}}%
\pgfpathlineto{\pgfqpoint{1.776979in}{1.027029in}}%
\pgfpathlineto{\pgfqpoint{1.796451in}{1.021178in}}%
\pgfpathlineto{\pgfqpoint{1.815923in}{1.016747in}}%
\pgfpathlineto{\pgfqpoint{1.835396in}{1.013919in}}%
\pgfpathlineto{\pgfqpoint{1.854868in}{1.012859in}}%
\pgfpathlineto{\pgfqpoint{1.874340in}{1.013714in}}%
\pgfpathlineto{\pgfqpoint{1.893813in}{1.016612in}}%
\pgfpathlineto{\pgfqpoint{1.913285in}{1.021661in}}%
\pgfpathlineto{\pgfqpoint{1.932758in}{1.028951in}}%
\pgfpathlineto{\pgfqpoint{1.952230in}{1.038550in}}%
\pgfpathlineto{\pgfqpoint{1.971702in}{1.050511in}}%
\pgfpathlineto{\pgfqpoint{1.991175in}{1.064862in}}%
\pgfpathlineto{\pgfqpoint{2.010647in}{1.081618in}}%
\pgfpathlineto{\pgfqpoint{2.030119in}{1.100772in}}%
\pgfpathlineto{\pgfqpoint{2.049592in}{1.122301in}}%
\pgfpathlineto{\pgfqpoint{2.069064in}{1.146162in}}%
\pgfpathlineto{\pgfqpoint{2.088536in}{1.172299in}}%
\pgfpathlineto{\pgfqpoint{2.108009in}{1.200637in}}%
\pgfpathlineto{\pgfqpoint{2.146954in}{1.263550in}}%
\pgfpathlineto{\pgfqpoint{2.185898in}{1.334023in}}%
\pgfpathlineto{\pgfqpoint{2.224843in}{1.410981in}}%
\pgfpathlineto{\pgfqpoint{2.263788in}{1.493178in}}%
\pgfpathlineto{\pgfqpoint{2.322205in}{1.623254in}}%
\pgfpathlineto{\pgfqpoint{2.458511in}{1.931674in}}%
\pgfpathlineto{\pgfqpoint{2.497456in}{2.014009in}}%
\pgfpathlineto{\pgfqpoint{2.536401in}{2.090998in}}%
\pgfpathlineto{\pgfqpoint{2.575345in}{2.161256in}}%
\pgfpathlineto{\pgfqpoint{2.614290in}{2.223516in}}%
\pgfpathlineto{\pgfqpoint{2.633763in}{2.251291in}}%
\pgfpathlineto{\pgfqpoint{2.653235in}{2.276658in}}%
\pgfpathlineto{\pgfqpoint{2.672707in}{2.299503in}}%
\pgfpathlineto{\pgfqpoint{2.692180in}{2.319723in}}%
\pgfpathlineto{\pgfqpoint{2.711652in}{2.337227in}}%
\pgfpathlineto{\pgfqpoint{2.731124in}{2.351936in}}%
\pgfpathlineto{\pgfqpoint{2.750597in}{2.363783in}}%
\pgfpathlineto{\pgfqpoint{2.770069in}{2.372716in}}%
\pgfpathlineto{\pgfqpoint{2.789541in}{2.378693in}}%
\pgfpathlineto{\pgfqpoint{2.809014in}{2.381689in}}%
\pgfpathlineto{\pgfqpoint{2.828486in}{2.381689in}}%
\pgfpathlineto{\pgfqpoint{2.847959in}{2.378693in}}%
\pgfpathlineto{\pgfqpoint{2.867431in}{2.372716in}}%
\pgfpathlineto{\pgfqpoint{2.886903in}{2.363783in}}%
\pgfpathlineto{\pgfqpoint{2.906376in}{2.351936in}}%
\pgfpathlineto{\pgfqpoint{2.925848in}{2.337227in}}%
\pgfpathlineto{\pgfqpoint{2.945320in}{2.319723in}}%
\pgfpathlineto{\pgfqpoint{2.964793in}{2.299503in}}%
\pgfpathlineto{\pgfqpoint{2.984265in}{2.276658in}}%
\pgfpathlineto{\pgfqpoint{3.003737in}{2.251291in}}%
\pgfpathlineto{\pgfqpoint{3.023210in}{2.223516in}}%
\pgfpathlineto{\pgfqpoint{3.042682in}{2.193459in}}%
\pgfpathlineto{\pgfqpoint{3.081627in}{2.127051in}}%
\pgfpathlineto{\pgfqpoint{3.120572in}{2.053261in}}%
\pgfpathlineto{\pgfqpoint{3.159516in}{1.973419in}}%
\pgfpathlineto{\pgfqpoint{3.198461in}{1.888961in}}%
\pgfpathlineto{\pgfqpoint{3.276351in}{1.712305in}}%
\pgfpathlineto{\pgfqpoint{3.354240in}{1.535815in}}%
\pgfpathlineto{\pgfqpoint{3.393185in}{1.451508in}}%
\pgfpathlineto{\pgfqpoint{3.432129in}{1.371766in}}%
\pgfpathlineto{\pgfqpoint{3.471074in}{1.297905in}}%
\pgfpathlineto{\pgfqpoint{3.510019in}{1.231089in}}%
\pgfpathlineto{\pgfqpoint{3.529491in}{1.200637in}}%
\pgfpathlineto{\pgfqpoint{3.548964in}{1.172299in}}%
\pgfpathlineto{\pgfqpoint{3.568436in}{1.146162in}}%
\pgfpathlineto{\pgfqpoint{3.587908in}{1.122301in}}%
\pgfpathlineto{\pgfqpoint{3.607381in}{1.100772in}}%
\pgfpathlineto{\pgfqpoint{3.626853in}{1.081618in}}%
\pgfpathlineto{\pgfqpoint{3.646325in}{1.064862in}}%
\pgfpathlineto{\pgfqpoint{3.665798in}{1.050511in}}%
\pgfpathlineto{\pgfqpoint{3.685270in}{1.038550in}}%
\pgfpathlineto{\pgfqpoint{3.704742in}{1.028951in}}%
\pgfpathlineto{\pgfqpoint{3.724215in}{1.021661in}}%
\pgfpathlineto{\pgfqpoint{3.743687in}{1.016612in}}%
\pgfpathlineto{\pgfqpoint{3.763160in}{1.013714in}}%
\pgfpathlineto{\pgfqpoint{3.782632in}{1.012859in}}%
\pgfpathlineto{\pgfqpoint{3.802104in}{1.013919in}}%
\pgfpathlineto{\pgfqpoint{3.821577in}{1.016747in}}%
\pgfpathlineto{\pgfqpoint{3.841049in}{1.021178in}}%
\pgfpathlineto{\pgfqpoint{3.860521in}{1.027029in}}%
\pgfpathlineto{\pgfqpoint{3.899466in}{1.042168in}}%
\pgfpathlineto{\pgfqpoint{3.938411in}{1.060366in}}%
\pgfpathlineto{\pgfqpoint{4.016300in}{1.097669in}}%
\pgfpathlineto{\pgfqpoint{4.055245in}{1.112257in}}%
\pgfpathlineto{\pgfqpoint{4.074717in}{1.117510in}}%
\pgfpathlineto{\pgfqpoint{4.094190in}{1.121012in}}%
\pgfpathlineto{\pgfqpoint{4.113662in}{1.122478in}}%
\pgfpathlineto{\pgfqpoint{4.133134in}{1.121635in}}%
\pgfpathlineto{\pgfqpoint{4.152607in}{1.118221in}}%
\pgfpathlineto{\pgfqpoint{4.172079in}{1.111994in}}%
\pgfpathlineto{\pgfqpoint{4.191552in}{1.102735in}}%
\pgfpathlineto{\pgfqpoint{4.211024in}{1.090250in}}%
\pgfpathlineto{\pgfqpoint{4.230496in}{1.074382in}}%
\pgfpathlineto{\pgfqpoint{4.249969in}{1.055007in}}%
\pgfpathlineto{\pgfqpoint{4.269441in}{1.032050in}}%
\pgfpathlineto{\pgfqpoint{4.288913in}{1.005481in}}%
\pgfpathlineto{\pgfqpoint{4.308386in}{0.975331in}}%
\pgfpathlineto{\pgfqpoint{4.327858in}{0.941693in}}%
\pgfpathlineto{\pgfqpoint{4.347330in}{0.904731in}}%
\pgfpathlineto{\pgfqpoint{4.366803in}{0.864687in}}%
\pgfpathlineto{\pgfqpoint{4.405747in}{0.776771in}}%
\pgfpathlineto{\pgfqpoint{4.464165in}{0.633346in}}%
\pgfpathlineto{\pgfqpoint{4.503109in}{0.539085in}}%
\pgfpathlineto{\pgfqpoint{4.522582in}{0.495528in}}%
\pgfpathlineto{\pgfqpoint{4.542054in}{0.456136in}}%
\pgfpathlineto{\pgfqpoint{4.561526in}{0.422470in}}%
\pgfpathlineto{\pgfqpoint{4.580999in}{0.396289in}}%
\pgfpathlineto{\pgfqpoint{4.600471in}{0.379559in}}%
\pgfpathlineto{\pgfqpoint{4.617920in}{0.375000in}}%
\pgfpathmoveto{\pgfqpoint{4.621089in}{0.375000in}}%
\pgfpathlineto{\pgfqpoint{4.639416in}{0.383458in}}%
\pgfpathlineto{\pgfqpoint{4.658888in}{0.409204in}}%
\pgfpathlineto{\pgfqpoint{4.678361in}{0.454665in}}%
\pgfpathlineto{\pgfqpoint{4.697833in}{0.523088in}}%
\pgfpathlineto{\pgfqpoint{4.717305in}{0.618023in}}%
\pgfpathlineto{\pgfqpoint{4.736778in}{0.743344in}}%
\pgfpathlineto{\pgfqpoint{4.756250in}{0.903269in}}%
\pgfpathlineto{\pgfqpoint{4.756250in}{0.903269in}}%
\pgfusepath{stroke}%
\end{pgfscope}%
\begin{pgfscope}%
\pgfpathrectangle{\pgfqpoint{0.687500in}{0.385000in}}{\pgfqpoint{4.262500in}{2.695000in}}%
\pgfusepath{clip}%
\pgfsetbuttcap%
\pgfsetroundjoin%
\definecolor{currentfill}{rgb}{0.549020,0.337255,0.294118}%
\pgfsetfillcolor{currentfill}%
\pgfsetlinewidth{1.003750pt}%
\definecolor{currentstroke}{rgb}{0.549020,0.337255,0.294118}%
\pgfsetstrokecolor{currentstroke}%
\pgfsetdash{}{0pt}%
\pgfsys@defobject{currentmarker}{\pgfqpoint{-0.020833in}{-0.020833in}}{\pgfqpoint{0.020833in}{0.020833in}}{%
\pgfpathmoveto{\pgfqpoint{0.000000in}{-0.020833in}}%
\pgfpathcurveto{\pgfqpoint{0.005525in}{-0.020833in}}{\pgfqpoint{0.010825in}{-0.018638in}}{\pgfqpoint{0.014731in}{-0.014731in}}%
\pgfpathcurveto{\pgfqpoint{0.018638in}{-0.010825in}}{\pgfqpoint{0.020833in}{-0.005525in}}{\pgfqpoint{0.020833in}{0.000000in}}%
\pgfpathcurveto{\pgfqpoint{0.020833in}{0.005525in}}{\pgfqpoint{0.018638in}{0.010825in}}{\pgfqpoint{0.014731in}{0.014731in}}%
\pgfpathcurveto{\pgfqpoint{0.010825in}{0.018638in}}{\pgfqpoint{0.005525in}{0.020833in}}{\pgfqpoint{0.000000in}{0.020833in}}%
\pgfpathcurveto{\pgfqpoint{-0.005525in}{0.020833in}}{\pgfqpoint{-0.010825in}{0.018638in}}{\pgfqpoint{-0.014731in}{0.014731in}}%
\pgfpathcurveto{\pgfqpoint{-0.018638in}{0.010825in}}{\pgfqpoint{-0.020833in}{0.005525in}}{\pgfqpoint{-0.020833in}{0.000000in}}%
\pgfpathcurveto{\pgfqpoint{-0.020833in}{-0.005525in}}{\pgfqpoint{-0.018638in}{-0.010825in}}{\pgfqpoint{-0.014731in}{-0.014731in}}%
\pgfpathcurveto{\pgfqpoint{-0.010825in}{-0.018638in}}{\pgfqpoint{-0.005525in}{-0.020833in}}{\pgfqpoint{0.000000in}{-0.020833in}}%
\pgfpathclose%
\pgfusepath{stroke,fill}%
}%
\begin{pgfscope}%
\pgfsys@transformshift{0.881250in}{0.903269in}%
\pgfsys@useobject{currentmarker}{}%
\end{pgfscope}%
\begin{pgfscope}%
\pgfsys@transformshift{1.109191in}{0.921965in}%
\pgfsys@useobject{currentmarker}{}%
\end{pgfscope}%
\begin{pgfscope}%
\pgfsys@transformshift{1.337132in}{0.949195in}%
\pgfsys@useobject{currentmarker}{}%
\end{pgfscope}%
\begin{pgfscope}%
\pgfsys@transformshift{1.565074in}{0.990846in}%
\pgfsys@useobject{currentmarker}{}%
\end{pgfscope}%
\begin{pgfscope}%
\pgfsys@transformshift{1.793015in}{1.058556in}%
\pgfsys@useobject{currentmarker}{}%
\end{pgfscope}%
\begin{pgfscope}%
\pgfsys@transformshift{2.020956in}{1.177124in}%
\pgfsys@useobject{currentmarker}{}%
\end{pgfscope}%
\begin{pgfscope}%
\pgfsys@transformshift{2.248897in}{1.402259in}%
\pgfsys@useobject{currentmarker}{}%
\end{pgfscope}%
\begin{pgfscope}%
\pgfsys@transformshift{2.476838in}{1.844355in}%
\pgfsys@useobject{currentmarker}{}%
\end{pgfscope}%
\begin{pgfscope}%
\pgfsys@transformshift{2.704779in}{2.487787in}%
\pgfsys@useobject{currentmarker}{}%
\end{pgfscope}%
\begin{pgfscope}%
\pgfsys@transformshift{2.932721in}{2.487787in}%
\pgfsys@useobject{currentmarker}{}%
\end{pgfscope}%
\begin{pgfscope}%
\pgfsys@transformshift{3.160662in}{1.844355in}%
\pgfsys@useobject{currentmarker}{}%
\end{pgfscope}%
\begin{pgfscope}%
\pgfsys@transformshift{3.388603in}{1.402259in}%
\pgfsys@useobject{currentmarker}{}%
\end{pgfscope}%
\begin{pgfscope}%
\pgfsys@transformshift{3.616544in}{1.177124in}%
\pgfsys@useobject{currentmarker}{}%
\end{pgfscope}%
\begin{pgfscope}%
\pgfsys@transformshift{3.844485in}{1.058556in}%
\pgfsys@useobject{currentmarker}{}%
\end{pgfscope}%
\begin{pgfscope}%
\pgfsys@transformshift{4.072426in}{0.990846in}%
\pgfsys@useobject{currentmarker}{}%
\end{pgfscope}%
\begin{pgfscope}%
\pgfsys@transformshift{4.300368in}{0.949195in}%
\pgfsys@useobject{currentmarker}{}%
\end{pgfscope}%
\begin{pgfscope}%
\pgfsys@transformshift{4.528309in}{0.921965in}%
\pgfsys@useobject{currentmarker}{}%
\end{pgfscope}%
\end{pgfscope}%
\begin{pgfscope}%
\pgfpathrectangle{\pgfqpoint{0.687500in}{0.385000in}}{\pgfqpoint{4.262500in}{2.695000in}}%
\pgfusepath{clip}%
\pgfsetrectcap%
\pgfsetroundjoin%
\pgfsetlinewidth{1.505625pt}%
\definecolor{currentstroke}{rgb}{0.890196,0.466667,0.760784}%
\pgfsetstrokecolor{currentstroke}%
\pgfsetdash{}{0pt}%
\pgfpathmoveto{\pgfqpoint{0.881250in}{0.903269in}}%
\pgfpathlineto{\pgfqpoint{0.883479in}{0.375000in}}%
\pgfpathmoveto{\pgfqpoint{1.090199in}{0.375000in}}%
\pgfpathlineto{\pgfqpoint{1.095446in}{0.555774in}}%
\pgfpathlineto{\pgfqpoint{1.114918in}{1.049779in}}%
\pgfpathlineto{\pgfqpoint{1.134391in}{1.384418in}}%
\pgfpathlineto{\pgfqpoint{1.153863in}{1.583274in}}%
\pgfpathlineto{\pgfqpoint{1.173335in}{1.672338in}}%
\pgfpathlineto{\pgfqpoint{1.192808in}{1.677454in}}%
\pgfpathlineto{\pgfqpoint{1.212280in}{1.622630in}}%
\pgfpathlineto{\pgfqpoint{1.231753in}{1.529016in}}%
\pgfpathlineto{\pgfqpoint{1.251225in}{1.414380in}}%
\pgfpathlineto{\pgfqpoint{1.290170in}{1.175527in}}%
\pgfpathlineto{\pgfqpoint{1.309642in}{1.069750in}}%
\pgfpathlineto{\pgfqpoint{1.329114in}{0.980488in}}%
\pgfpathlineto{\pgfqpoint{1.348587in}{0.910274in}}%
\pgfpathlineto{\pgfqpoint{1.368059in}{0.859759in}}%
\pgfpathlineto{\pgfqpoint{1.387531in}{0.828147in}}%
\pgfpathlineto{\pgfqpoint{1.407004in}{0.813606in}}%
\pgfpathlineto{\pgfqpoint{1.426476in}{0.813629in}}%
\pgfpathlineto{\pgfqpoint{1.445948in}{0.825344in}}%
\pgfpathlineto{\pgfqpoint{1.465421in}{0.845771in}}%
\pgfpathlineto{\pgfqpoint{1.484893in}{0.872021in}}%
\pgfpathlineto{\pgfqpoint{1.562783in}{0.987902in}}%
\pgfpathlineto{\pgfqpoint{1.582255in}{1.011271in}}%
\pgfpathlineto{\pgfqpoint{1.601727in}{1.030563in}}%
\pgfpathlineto{\pgfqpoint{1.621200in}{1.045546in}}%
\pgfpathlineto{\pgfqpoint{1.640672in}{1.056307in}}%
\pgfpathlineto{\pgfqpoint{1.660144in}{1.063190in}}%
\pgfpathlineto{\pgfqpoint{1.679617in}{1.066735in}}%
\pgfpathlineto{\pgfqpoint{1.699089in}{1.067619in}}%
\pgfpathlineto{\pgfqpoint{1.718562in}{1.066595in}}%
\pgfpathlineto{\pgfqpoint{1.796451in}{1.058422in}}%
\pgfpathlineto{\pgfqpoint{1.815923in}{1.058576in}}%
\pgfpathlineto{\pgfqpoint{1.835396in}{1.060544in}}%
\pgfpathlineto{\pgfqpoint{1.854868in}{1.064582in}}%
\pgfpathlineto{\pgfqpoint{1.874340in}{1.070823in}}%
\pgfpathlineto{\pgfqpoint{1.893813in}{1.079288in}}%
\pgfpathlineto{\pgfqpoint{1.913285in}{1.089896in}}%
\pgfpathlineto{\pgfqpoint{1.932758in}{1.102480in}}%
\pgfpathlineto{\pgfqpoint{1.952230in}{1.116811in}}%
\pgfpathlineto{\pgfqpoint{1.991175in}{1.149584in}}%
\pgfpathlineto{\pgfqpoint{2.049592in}{1.204602in}}%
\pgfpathlineto{\pgfqpoint{2.224843in}{1.375581in}}%
\pgfpathlineto{\pgfqpoint{2.263788in}{1.419948in}}%
\pgfpathlineto{\pgfqpoint{2.283260in}{1.444746in}}%
\pgfpathlineto{\pgfqpoint{2.302732in}{1.471750in}}%
\pgfpathlineto{\pgfqpoint{2.322205in}{1.501278in}}%
\pgfpathlineto{\pgfqpoint{2.341677in}{1.533606in}}%
\pgfpathlineto{\pgfqpoint{2.361149in}{1.568958in}}%
\pgfpathlineto{\pgfqpoint{2.380622in}{1.607490in}}%
\pgfpathlineto{\pgfqpoint{2.400094in}{1.649283in}}%
\pgfpathlineto{\pgfqpoint{2.419567in}{1.694331in}}%
\pgfpathlineto{\pgfqpoint{2.439039in}{1.742539in}}%
\pgfpathlineto{\pgfqpoint{2.477984in}{1.847592in}}%
\pgfpathlineto{\pgfqpoint{2.516928in}{1.961825in}}%
\pgfpathlineto{\pgfqpoint{2.633763in}{2.314597in}}%
\pgfpathlineto{\pgfqpoint{2.653235in}{2.367471in}}%
\pgfpathlineto{\pgfqpoint{2.672707in}{2.416660in}}%
\pgfpathlineto{\pgfqpoint{2.692180in}{2.461455in}}%
\pgfpathlineto{\pgfqpoint{2.711652in}{2.501198in}}%
\pgfpathlineto{\pgfqpoint{2.731124in}{2.535293in}}%
\pgfpathlineto{\pgfqpoint{2.750597in}{2.563223in}}%
\pgfpathlineto{\pgfqpoint{2.770069in}{2.584560in}}%
\pgfpathlineto{\pgfqpoint{2.789541in}{2.598971in}}%
\pgfpathlineto{\pgfqpoint{2.809014in}{2.606234in}}%
\pgfpathlineto{\pgfqpoint{2.828486in}{2.606234in}}%
\pgfpathlineto{\pgfqpoint{2.847959in}{2.598971in}}%
\pgfpathlineto{\pgfqpoint{2.867431in}{2.584560in}}%
\pgfpathlineto{\pgfqpoint{2.886903in}{2.563223in}}%
\pgfpathlineto{\pgfqpoint{2.906376in}{2.535293in}}%
\pgfpathlineto{\pgfqpoint{2.925848in}{2.501198in}}%
\pgfpathlineto{\pgfqpoint{2.945320in}{2.461455in}}%
\pgfpathlineto{\pgfqpoint{2.964793in}{2.416660in}}%
\pgfpathlineto{\pgfqpoint{2.984265in}{2.367471in}}%
\pgfpathlineto{\pgfqpoint{3.023210in}{2.258779in}}%
\pgfpathlineto{\pgfqpoint{3.062155in}{2.141354in}}%
\pgfpathlineto{\pgfqpoint{3.140044in}{1.903780in}}%
\pgfpathlineto{\pgfqpoint{3.178989in}{1.793720in}}%
\pgfpathlineto{\pgfqpoint{3.217933in}{1.694331in}}%
\pgfpathlineto{\pgfqpoint{3.237406in}{1.649283in}}%
\pgfpathlineto{\pgfqpoint{3.256878in}{1.607490in}}%
\pgfpathlineto{\pgfqpoint{3.276351in}{1.568958in}}%
\pgfpathlineto{\pgfqpoint{3.295823in}{1.533606in}}%
\pgfpathlineto{\pgfqpoint{3.315295in}{1.501278in}}%
\pgfpathlineto{\pgfqpoint{3.334768in}{1.471750in}}%
\pgfpathlineto{\pgfqpoint{3.354240in}{1.444746in}}%
\pgfpathlineto{\pgfqpoint{3.373712in}{1.419948in}}%
\pgfpathlineto{\pgfqpoint{3.412657in}{1.375581in}}%
\pgfpathlineto{\pgfqpoint{3.451602in}{1.335855in}}%
\pgfpathlineto{\pgfqpoint{3.607381in}{1.185849in}}%
\pgfpathlineto{\pgfqpoint{3.646325in}{1.149584in}}%
\pgfpathlineto{\pgfqpoint{3.685270in}{1.116811in}}%
\pgfpathlineto{\pgfqpoint{3.704742in}{1.102480in}}%
\pgfpathlineto{\pgfqpoint{3.724215in}{1.089896in}}%
\pgfpathlineto{\pgfqpoint{3.743687in}{1.079288in}}%
\pgfpathlineto{\pgfqpoint{3.763160in}{1.070823in}}%
\pgfpathlineto{\pgfqpoint{3.782632in}{1.064582in}}%
\pgfpathlineto{\pgfqpoint{3.802104in}{1.060544in}}%
\pgfpathlineto{\pgfqpoint{3.821577in}{1.058576in}}%
\pgfpathlineto{\pgfqpoint{3.841049in}{1.058422in}}%
\pgfpathlineto{\pgfqpoint{3.879994in}{1.061909in}}%
\pgfpathlineto{\pgfqpoint{3.918938in}{1.066595in}}%
\pgfpathlineto{\pgfqpoint{3.938411in}{1.067619in}}%
\pgfpathlineto{\pgfqpoint{3.957883in}{1.066735in}}%
\pgfpathlineto{\pgfqpoint{3.977356in}{1.063190in}}%
\pgfpathlineto{\pgfqpoint{3.996828in}{1.056307in}}%
\pgfpathlineto{\pgfqpoint{4.016300in}{1.045546in}}%
\pgfpathlineto{\pgfqpoint{4.035773in}{1.030563in}}%
\pgfpathlineto{\pgfqpoint{4.055245in}{1.011271in}}%
\pgfpathlineto{\pgfqpoint{4.074717in}{0.987902in}}%
\pgfpathlineto{\pgfqpoint{4.094190in}{0.961059in}}%
\pgfpathlineto{\pgfqpoint{4.172079in}{0.845771in}}%
\pgfpathlineto{\pgfqpoint{4.191552in}{0.825344in}}%
\pgfpathlineto{\pgfqpoint{4.211024in}{0.813629in}}%
\pgfpathlineto{\pgfqpoint{4.230496in}{0.813606in}}%
\pgfpathlineto{\pgfqpoint{4.249969in}{0.828147in}}%
\pgfpathlineto{\pgfqpoint{4.269441in}{0.859759in}}%
\pgfpathlineto{\pgfqpoint{4.288913in}{0.910274in}}%
\pgfpathlineto{\pgfqpoint{4.308386in}{0.980488in}}%
\pgfpathlineto{\pgfqpoint{4.327858in}{1.069750in}}%
\pgfpathlineto{\pgfqpoint{4.347330in}{1.175527in}}%
\pgfpathlineto{\pgfqpoint{4.405747in}{1.529016in}}%
\pgfpathlineto{\pgfqpoint{4.425220in}{1.622630in}}%
\pgfpathlineto{\pgfqpoint{4.444692in}{1.677454in}}%
\pgfpathlineto{\pgfqpoint{4.464165in}{1.672338in}}%
\pgfpathlineto{\pgfqpoint{4.483637in}{1.583274in}}%
\pgfpathlineto{\pgfqpoint{4.503109in}{1.384418in}}%
\pgfpathlineto{\pgfqpoint{4.522582in}{1.049779in}}%
\pgfpathlineto{\pgfqpoint{4.542054in}{0.555774in}}%
\pgfpathlineto{\pgfqpoint{4.547301in}{0.375000in}}%
\pgfpathmoveto{\pgfqpoint{4.754021in}{0.375000in}}%
\pgfpathlineto{\pgfqpoint{4.756250in}{0.903269in}}%
\pgfpathlineto{\pgfqpoint{4.756250in}{0.903269in}}%
\pgfusepath{stroke}%
\end{pgfscope}%
\begin{pgfscope}%
\pgfpathrectangle{\pgfqpoint{0.687500in}{0.385000in}}{\pgfqpoint{4.262500in}{2.695000in}}%
\pgfusepath{clip}%
\pgfsetbuttcap%
\pgfsetroundjoin%
\definecolor{currentfill}{rgb}{0.498039,0.498039,0.498039}%
\pgfsetfillcolor{currentfill}%
\pgfsetlinewidth{1.003750pt}%
\definecolor{currentstroke}{rgb}{0.498039,0.498039,0.498039}%
\pgfsetstrokecolor{currentstroke}%
\pgfsetdash{}{0pt}%
\pgfsys@defobject{currentmarker}{\pgfqpoint{-0.020833in}{-0.020833in}}{\pgfqpoint{0.020833in}{0.020833in}}{%
\pgfpathmoveto{\pgfqpoint{0.000000in}{-0.020833in}}%
\pgfpathcurveto{\pgfqpoint{0.005525in}{-0.020833in}}{\pgfqpoint{0.010825in}{-0.018638in}}{\pgfqpoint{0.014731in}{-0.014731in}}%
\pgfpathcurveto{\pgfqpoint{0.018638in}{-0.010825in}}{\pgfqpoint{0.020833in}{-0.005525in}}{\pgfqpoint{0.020833in}{0.000000in}}%
\pgfpathcurveto{\pgfqpoint{0.020833in}{0.005525in}}{\pgfqpoint{0.018638in}{0.010825in}}{\pgfqpoint{0.014731in}{0.014731in}}%
\pgfpathcurveto{\pgfqpoint{0.010825in}{0.018638in}}{\pgfqpoint{0.005525in}{0.020833in}}{\pgfqpoint{0.000000in}{0.020833in}}%
\pgfpathcurveto{\pgfqpoint{-0.005525in}{0.020833in}}{\pgfqpoint{-0.010825in}{0.018638in}}{\pgfqpoint{-0.014731in}{0.014731in}}%
\pgfpathcurveto{\pgfqpoint{-0.018638in}{0.010825in}}{\pgfqpoint{-0.020833in}{0.005525in}}{\pgfqpoint{-0.020833in}{0.000000in}}%
\pgfpathcurveto{\pgfqpoint{-0.020833in}{-0.005525in}}{\pgfqpoint{-0.018638in}{-0.010825in}}{\pgfqpoint{-0.014731in}{-0.014731in}}%
\pgfpathcurveto{\pgfqpoint{-0.010825in}{-0.018638in}}{\pgfqpoint{-0.005525in}{-0.020833in}}{\pgfqpoint{0.000000in}{-0.020833in}}%
\pgfpathclose%
\pgfusepath{stroke,fill}%
}%
\begin{pgfscope}%
\pgfsys@transformshift{0.881250in}{0.903269in}%
\pgfsys@useobject{currentmarker}{}%
\end{pgfscope}%
\begin{pgfscope}%
\pgfsys@transformshift{0.998674in}{0.912074in}%
\pgfsys@useobject{currentmarker}{}%
\end{pgfscope}%
\begin{pgfscope}%
\pgfsys@transformshift{1.116098in}{0.922643in}%
\pgfsys@useobject{currentmarker}{}%
\end{pgfscope}%
\begin{pgfscope}%
\pgfsys@transformshift{1.233523in}{0.935470in}%
\pgfsys@useobject{currentmarker}{}%
\end{pgfscope}%
\begin{pgfscope}%
\pgfsys@transformshift{1.350947in}{0.951228in}%
\pgfsys@useobject{currentmarker}{}%
\end{pgfscope}%
\begin{pgfscope}%
\pgfsys@transformshift{1.468371in}{0.970856in}%
\pgfsys@useobject{currentmarker}{}%
\end{pgfscope}%
\begin{pgfscope}%
\pgfsys@transformshift{1.585795in}{0.995680in}%
\pgfsys@useobject{currentmarker}{}%
\end{pgfscope}%
\begin{pgfscope}%
\pgfsys@transformshift{1.703220in}{1.027618in}%
\pgfsys@useobject{currentmarker}{}%
\end{pgfscope}%
\begin{pgfscope}%
\pgfsys@transformshift{1.820644in}{1.069501in}%
\pgfsys@useobject{currentmarker}{}%
\end{pgfscope}%
\begin{pgfscope}%
\pgfsys@transformshift{1.938068in}{1.125583in}%
\pgfsys@useobject{currentmarker}{}%
\end{pgfscope}%
\begin{pgfscope}%
\pgfsys@transformshift{2.055492in}{1.202358in}%
\pgfsys@useobject{currentmarker}{}%
\end{pgfscope}%
\begin{pgfscope}%
\pgfsys@transformshift{2.172917in}{1.309755in}%
\pgfsys@useobject{currentmarker}{}%
\end{pgfscope}%
\begin{pgfscope}%
\pgfsys@transformshift{2.290341in}{1.462481in}%
\pgfsys@useobject{currentmarker}{}%
\end{pgfscope}%
\begin{pgfscope}%
\pgfsys@transformshift{2.407765in}{1.679703in}%
\pgfsys@useobject{currentmarker}{}%
\end{pgfscope}%
\begin{pgfscope}%
\pgfsys@transformshift{2.525189in}{1.975689in}%
\pgfsys@useobject{currentmarker}{}%
\end{pgfscope}%
\begin{pgfscope}%
\pgfsys@transformshift{2.642614in}{2.323185in}%
\pgfsys@useobject{currentmarker}{}%
\end{pgfscope}%
\begin{pgfscope}%
\pgfsys@transformshift{2.760038in}{2.590513in}%
\pgfsys@useobject{currentmarker}{}%
\end{pgfscope}%
\begin{pgfscope}%
\pgfsys@transformshift{2.877462in}{2.590513in}%
\pgfsys@useobject{currentmarker}{}%
\end{pgfscope}%
\begin{pgfscope}%
\pgfsys@transformshift{2.994886in}{2.323185in}%
\pgfsys@useobject{currentmarker}{}%
\end{pgfscope}%
\begin{pgfscope}%
\pgfsys@transformshift{3.112311in}{1.975689in}%
\pgfsys@useobject{currentmarker}{}%
\end{pgfscope}%
\begin{pgfscope}%
\pgfsys@transformshift{3.229735in}{1.679703in}%
\pgfsys@useobject{currentmarker}{}%
\end{pgfscope}%
\begin{pgfscope}%
\pgfsys@transformshift{3.347159in}{1.462481in}%
\pgfsys@useobject{currentmarker}{}%
\end{pgfscope}%
\begin{pgfscope}%
\pgfsys@transformshift{3.464583in}{1.309755in}%
\pgfsys@useobject{currentmarker}{}%
\end{pgfscope}%
\begin{pgfscope}%
\pgfsys@transformshift{3.582008in}{1.202358in}%
\pgfsys@useobject{currentmarker}{}%
\end{pgfscope}%
\begin{pgfscope}%
\pgfsys@transformshift{3.699432in}{1.125583in}%
\pgfsys@useobject{currentmarker}{}%
\end{pgfscope}%
\begin{pgfscope}%
\pgfsys@transformshift{3.816856in}{1.069501in}%
\pgfsys@useobject{currentmarker}{}%
\end{pgfscope}%
\begin{pgfscope}%
\pgfsys@transformshift{3.934280in}{1.027618in}%
\pgfsys@useobject{currentmarker}{}%
\end{pgfscope}%
\begin{pgfscope}%
\pgfsys@transformshift{4.051705in}{0.995680in}%
\pgfsys@useobject{currentmarker}{}%
\end{pgfscope}%
\begin{pgfscope}%
\pgfsys@transformshift{4.169129in}{0.970856in}%
\pgfsys@useobject{currentmarker}{}%
\end{pgfscope}%
\begin{pgfscope}%
\pgfsys@transformshift{4.286553in}{0.951228in}%
\pgfsys@useobject{currentmarker}{}%
\end{pgfscope}%
\begin{pgfscope}%
\pgfsys@transformshift{4.403977in}{0.935470in}%
\pgfsys@useobject{currentmarker}{}%
\end{pgfscope}%
\begin{pgfscope}%
\pgfsys@transformshift{4.521402in}{0.922643in}%
\pgfsys@useobject{currentmarker}{}%
\end{pgfscope}%
\begin{pgfscope}%
\pgfsys@transformshift{4.638826in}{0.912074in}%
\pgfsys@useobject{currentmarker}{}%
\end{pgfscope}%
\end{pgfscope}%
\begin{pgfscope}%
\pgfpathrectangle{\pgfqpoint{0.687500in}{0.385000in}}{\pgfqpoint{4.262500in}{2.695000in}}%
\pgfusepath{clip}%
\pgfsetrectcap%
\pgfsetroundjoin%
\pgfsetlinewidth{1.505625pt}%
\definecolor{currentstroke}{rgb}{0.737255,0.741176,0.133333}%
\pgfsetstrokecolor{currentstroke}%
\pgfsetdash{}{0pt}%
\pgfpathmoveto{\pgfqpoint{0.881250in}{0.903269in}}%
\pgfpathlineto{\pgfqpoint{0.881254in}{0.375000in}}%
\pgfpathmoveto{\pgfqpoint{0.998959in}{0.375000in}}%
\pgfpathlineto{\pgfqpoint{0.999456in}{3.090000in}}%
\pgfpathmoveto{\pgfqpoint{1.113418in}{3.090000in}}%
\pgfpathlineto{\pgfqpoint{1.114918in}{1.685121in}}%
\pgfpathlineto{\pgfqpoint{1.117944in}{0.375000in}}%
\pgfpathmoveto{\pgfqpoint{1.227348in}{0.375000in}}%
\pgfpathlineto{\pgfqpoint{1.231753in}{0.807720in}}%
\pgfpathlineto{\pgfqpoint{1.251225in}{1.819498in}}%
\pgfpathlineto{\pgfqpoint{1.270697in}{2.113855in}}%
\pgfpathlineto{\pgfqpoint{1.290170in}{1.962977in}}%
\pgfpathlineto{\pgfqpoint{1.329114in}{1.255634in}}%
\pgfpathlineto{\pgfqpoint{1.348587in}{0.977182in}}%
\pgfpathlineto{\pgfqpoint{1.368059in}{0.815121in}}%
\pgfpathlineto{\pgfqpoint{1.387531in}{0.759438in}}%
\pgfpathlineto{\pgfqpoint{1.407004in}{0.778851in}}%
\pgfpathlineto{\pgfqpoint{1.426476in}{0.837604in}}%
\pgfpathlineto{\pgfqpoint{1.445948in}{0.905761in}}%
\pgfpathlineto{\pgfqpoint{1.465421in}{0.963672in}}%
\pgfpathlineto{\pgfqpoint{1.484893in}{1.002202in}}%
\pgfpathlineto{\pgfqpoint{1.504366in}{1.020515in}}%
\pgfpathlineto{\pgfqpoint{1.523838in}{1.022919in}}%
\pgfpathlineto{\pgfqpoint{1.543310in}{1.015850in}}%
\pgfpathlineto{\pgfqpoint{1.562783in}{1.005580in}}%
\pgfpathlineto{\pgfqpoint{1.582255in}{0.996856in}}%
\pgfpathlineto{\pgfqpoint{1.601727in}{0.992387in}}%
\pgfpathlineto{\pgfqpoint{1.621200in}{0.992982in}}%
\pgfpathlineto{\pgfqpoint{1.640672in}{0.998038in}}%
\pgfpathlineto{\pgfqpoint{1.660144in}{1.006178in}}%
\pgfpathlineto{\pgfqpoint{1.718562in}{1.034665in}}%
\pgfpathlineto{\pgfqpoint{1.757506in}{1.049457in}}%
\pgfpathlineto{\pgfqpoint{1.835396in}{1.074838in}}%
\pgfpathlineto{\pgfqpoint{1.874340in}{1.091457in}}%
\pgfpathlineto{\pgfqpoint{1.913285in}{1.111519in}}%
\pgfpathlineto{\pgfqpoint{1.952230in}{1.133947in}}%
\pgfpathlineto{\pgfqpoint{2.010647in}{1.170708in}}%
\pgfpathlineto{\pgfqpoint{2.049592in}{1.197960in}}%
\pgfpathlineto{\pgfqpoint{2.088536in}{1.228576in}}%
\pgfpathlineto{\pgfqpoint{2.127481in}{1.263364in}}%
\pgfpathlineto{\pgfqpoint{2.166426in}{1.302733in}}%
\pgfpathlineto{\pgfqpoint{2.205371in}{1.346915in}}%
\pgfpathlineto{\pgfqpoint{2.244315in}{1.396318in}}%
\pgfpathlineto{\pgfqpoint{2.283260in}{1.451695in}}%
\pgfpathlineto{\pgfqpoint{2.322205in}{1.514030in}}%
\pgfpathlineto{\pgfqpoint{2.361149in}{1.584278in}}%
\pgfpathlineto{\pgfqpoint{2.400094in}{1.663125in}}%
\pgfpathlineto{\pgfqpoint{2.439039in}{1.750889in}}%
\pgfpathlineto{\pgfqpoint{2.477984in}{1.847510in}}%
\pgfpathlineto{\pgfqpoint{2.516928in}{1.952460in}}%
\pgfpathlineto{\pgfqpoint{2.555873in}{2.064435in}}%
\pgfpathlineto{\pgfqpoint{2.653235in}{2.353674in}}%
\pgfpathlineto{\pgfqpoint{2.672707in}{2.407520in}}%
\pgfpathlineto{\pgfqpoint{2.692180in}{2.457828in}}%
\pgfpathlineto{\pgfqpoint{2.711652in}{2.503531in}}%
\pgfpathlineto{\pgfqpoint{2.731124in}{2.543577in}}%
\pgfpathlineto{\pgfqpoint{2.750597in}{2.576976in}}%
\pgfpathlineto{\pgfqpoint{2.770069in}{2.602862in}}%
\pgfpathlineto{\pgfqpoint{2.789541in}{2.620532in}}%
\pgfpathlineto{\pgfqpoint{2.809014in}{2.629494in}}%
\pgfpathlineto{\pgfqpoint{2.828486in}{2.629494in}}%
\pgfpathlineto{\pgfqpoint{2.847959in}{2.620532in}}%
\pgfpathlineto{\pgfqpoint{2.867431in}{2.602862in}}%
\pgfpathlineto{\pgfqpoint{2.886903in}{2.576976in}}%
\pgfpathlineto{\pgfqpoint{2.906376in}{2.543577in}}%
\pgfpathlineto{\pgfqpoint{2.925848in}{2.503531in}}%
\pgfpathlineto{\pgfqpoint{2.945320in}{2.457828in}}%
\pgfpathlineto{\pgfqpoint{2.964793in}{2.407520in}}%
\pgfpathlineto{\pgfqpoint{3.003737in}{2.297324in}}%
\pgfpathlineto{\pgfqpoint{3.120572in}{1.952460in}}%
\pgfpathlineto{\pgfqpoint{3.159516in}{1.847510in}}%
\pgfpathlineto{\pgfqpoint{3.198461in}{1.750889in}}%
\pgfpathlineto{\pgfqpoint{3.237406in}{1.663125in}}%
\pgfpathlineto{\pgfqpoint{3.276351in}{1.584278in}}%
\pgfpathlineto{\pgfqpoint{3.315295in}{1.514030in}}%
\pgfpathlineto{\pgfqpoint{3.354240in}{1.451695in}}%
\pgfpathlineto{\pgfqpoint{3.393185in}{1.396318in}}%
\pgfpathlineto{\pgfqpoint{3.432129in}{1.346915in}}%
\pgfpathlineto{\pgfqpoint{3.471074in}{1.302733in}}%
\pgfpathlineto{\pgfqpoint{3.510019in}{1.263364in}}%
\pgfpathlineto{\pgfqpoint{3.548964in}{1.228576in}}%
\pgfpathlineto{\pgfqpoint{3.587908in}{1.197960in}}%
\pgfpathlineto{\pgfqpoint{3.626853in}{1.170708in}}%
\pgfpathlineto{\pgfqpoint{3.685270in}{1.133947in}}%
\pgfpathlineto{\pgfqpoint{3.724215in}{1.111519in}}%
\pgfpathlineto{\pgfqpoint{3.763160in}{1.091457in}}%
\pgfpathlineto{\pgfqpoint{3.802104in}{1.074838in}}%
\pgfpathlineto{\pgfqpoint{3.841049in}{1.061613in}}%
\pgfpathlineto{\pgfqpoint{3.899466in}{1.042590in}}%
\pgfpathlineto{\pgfqpoint{3.938411in}{1.025616in}}%
\pgfpathlineto{\pgfqpoint{3.977356in}{1.006178in}}%
\pgfpathlineto{\pgfqpoint{3.996828in}{0.998038in}}%
\pgfpathlineto{\pgfqpoint{4.016300in}{0.992982in}}%
\pgfpathlineto{\pgfqpoint{4.035773in}{0.992387in}}%
\pgfpathlineto{\pgfqpoint{4.055245in}{0.996856in}}%
\pgfpathlineto{\pgfqpoint{4.074717in}{1.005580in}}%
\pgfpathlineto{\pgfqpoint{4.094190in}{1.015850in}}%
\pgfpathlineto{\pgfqpoint{4.113662in}{1.022919in}}%
\pgfpathlineto{\pgfqpoint{4.133134in}{1.020515in}}%
\pgfpathlineto{\pgfqpoint{4.152607in}{1.002202in}}%
\pgfpathlineto{\pgfqpoint{4.172079in}{0.963672in}}%
\pgfpathlineto{\pgfqpoint{4.191552in}{0.905761in}}%
\pgfpathlineto{\pgfqpoint{4.211024in}{0.837604in}}%
\pgfpathlineto{\pgfqpoint{4.230496in}{0.778851in}}%
\pgfpathlineto{\pgfqpoint{4.249969in}{0.759438in}}%
\pgfpathlineto{\pgfqpoint{4.269441in}{0.815121in}}%
\pgfpathlineto{\pgfqpoint{4.288913in}{0.977182in}}%
\pgfpathlineto{\pgfqpoint{4.308386in}{1.255634in}}%
\pgfpathlineto{\pgfqpoint{4.347330in}{1.962977in}}%
\pgfpathlineto{\pgfqpoint{4.366803in}{2.113855in}}%
\pgfpathlineto{\pgfqpoint{4.386275in}{1.819498in}}%
\pgfpathlineto{\pgfqpoint{4.405747in}{0.807720in}}%
\pgfpathlineto{\pgfqpoint{4.410152in}{0.375000in}}%
\pgfpathmoveto{\pgfqpoint{4.519556in}{0.375000in}}%
\pgfpathlineto{\pgfqpoint{4.522582in}{1.685121in}}%
\pgfpathlineto{\pgfqpoint{4.524082in}{3.090000in}}%
\pgfpathmoveto{\pgfqpoint{4.638044in}{3.090000in}}%
\pgfpathlineto{\pgfqpoint{4.638541in}{0.375000in}}%
\pgfpathmoveto{\pgfqpoint{4.756246in}{0.375000in}}%
\pgfpathlineto{\pgfqpoint{4.756250in}{0.903271in}}%
\pgfpathlineto{\pgfqpoint{4.756250in}{0.903271in}}%
\pgfusepath{stroke}%
\end{pgfscope}%
\begin{pgfscope}%
\pgfsetrectcap%
\pgfsetmiterjoin%
\pgfsetlinewidth{0.803000pt}%
\definecolor{currentstroke}{rgb}{0.000000,0.000000,0.000000}%
\pgfsetstrokecolor{currentstroke}%
\pgfsetdash{}{0pt}%
\pgfpathmoveto{\pgfqpoint{0.687500in}{0.385000in}}%
\pgfpathlineto{\pgfqpoint{0.687500in}{3.080000in}}%
\pgfusepath{stroke}%
\end{pgfscope}%
\begin{pgfscope}%
\pgfsetrectcap%
\pgfsetmiterjoin%
\pgfsetlinewidth{0.803000pt}%
\definecolor{currentstroke}{rgb}{0.000000,0.000000,0.000000}%
\pgfsetstrokecolor{currentstroke}%
\pgfsetdash{}{0pt}%
\pgfpathmoveto{\pgfqpoint{4.950000in}{0.385000in}}%
\pgfpathlineto{\pgfqpoint{4.950000in}{3.080000in}}%
\pgfusepath{stroke}%
\end{pgfscope}%
\begin{pgfscope}%
\pgfsetrectcap%
\pgfsetmiterjoin%
\pgfsetlinewidth{0.803000pt}%
\definecolor{currentstroke}{rgb}{0.000000,0.000000,0.000000}%
\pgfsetstrokecolor{currentstroke}%
\pgfsetdash{}{0pt}%
\pgfpathmoveto{\pgfqpoint{0.687500in}{0.385000in}}%
\pgfpathlineto{\pgfqpoint{4.950000in}{0.385000in}}%
\pgfusepath{stroke}%
\end{pgfscope}%
\begin{pgfscope}%
\pgfsetrectcap%
\pgfsetmiterjoin%
\pgfsetlinewidth{0.803000pt}%
\definecolor{currentstroke}{rgb}{0.000000,0.000000,0.000000}%
\pgfsetstrokecolor{currentstroke}%
\pgfsetdash{}{0pt}%
\pgfpathmoveto{\pgfqpoint{0.687500in}{3.080000in}}%
\pgfpathlineto{\pgfqpoint{4.950000in}{3.080000in}}%
\pgfusepath{stroke}%
\end{pgfscope}%
\begin{pgfscope}%
\definecolor{textcolor}{rgb}{0.000000,0.000000,0.000000}%
\pgfsetstrokecolor{textcolor}%
\pgfsetfillcolor{textcolor}%
\pgftext[x=2.818750in,y=3.163333in,,base]{\color{textcolor}\rmfamily\fontsize{12.000000}{14.400000}\selectfont N=6, 8, 16, 32}%
\end{pgfscope}%
\begin{pgfscope}%
\pgfsetbuttcap%
\pgfsetmiterjoin%
\definecolor{currentfill}{rgb}{1.000000,1.000000,1.000000}%
\pgfsetfillcolor{currentfill}%
\pgfsetfillopacity{0.800000}%
\pgfsetlinewidth{1.003750pt}%
\definecolor{currentstroke}{rgb}{0.800000,0.800000,0.800000}%
\pgfsetstrokecolor{currentstroke}%
\pgfsetstrokeopacity{0.800000}%
\pgfsetdash{}{0pt}%
\pgfpathmoveto{\pgfqpoint{3.448142in}{1.775801in}}%
\pgfpathlineto{\pgfqpoint{4.852778in}{1.775801in}}%
\pgfpathquadraticcurveto{\pgfqpoint{4.880556in}{1.775801in}}{\pgfqpoint{4.880556in}{1.803578in}}%
\pgfpathlineto{\pgfqpoint{4.880556in}{2.982778in}}%
\pgfpathquadraticcurveto{\pgfqpoint{4.880556in}{3.010556in}}{\pgfqpoint{4.852778in}{3.010556in}}%
\pgfpathlineto{\pgfqpoint{3.448142in}{3.010556in}}%
\pgfpathquadraticcurveto{\pgfqpoint{3.420364in}{3.010556in}}{\pgfqpoint{3.420364in}{2.982778in}}%
\pgfpathlineto{\pgfqpoint{3.420364in}{1.803578in}}%
\pgfpathquadraticcurveto{\pgfqpoint{3.420364in}{1.775801in}}{\pgfqpoint{3.448142in}{1.775801in}}%
\pgfpathclose%
\pgfusepath{stroke,fill}%
\end{pgfscope}%
\begin{pgfscope}%
\pgfsetrectcap%
\pgfsetroundjoin%
\pgfsetlinewidth{1.505625pt}%
\definecolor{currentstroke}{rgb}{0.121569,0.466667,0.705882}%
\pgfsetstrokecolor{currentstroke}%
\pgfsetdash{}{0pt}%
\pgfpathmoveto{\pgfqpoint{3.475920in}{2.820146in}}%
\pgfpathlineto{\pgfqpoint{3.753698in}{2.820146in}}%
\pgfusepath{stroke}%
\end{pgfscope}%
\begin{pgfscope}%
\definecolor{textcolor}{rgb}{0.000000,0.000000,0.000000}%
\pgfsetstrokecolor{textcolor}%
\pgfsetfillcolor{textcolor}%
\pgftext[x=3.864809in,y=2.771534in,left,base]{\color{textcolor}\rmfamily\fontsize{10.000000}{12.000000}\selectfont \(\displaystyle y(x)=\)\(\displaystyle \frac{1}{1+25x^{2}}\)}%
\end{pgfscope}%
\begin{pgfscope}%
\pgfsetrectcap%
\pgfsetroundjoin%
\pgfsetlinewidth{1.505625pt}%
\definecolor{currentstroke}{rgb}{0.172549,0.627451,0.172549}%
\pgfsetstrokecolor{currentstroke}%
\pgfsetdash{}{0pt}%
\pgfpathmoveto{\pgfqpoint{3.475920in}{2.539689in}}%
\pgfpathlineto{\pgfqpoint{3.753698in}{2.539689in}}%
\pgfusepath{stroke}%
\end{pgfscope}%
\begin{pgfscope}%
\definecolor{textcolor}{rgb}{0.000000,0.000000,0.000000}%
\pgfsetstrokecolor{textcolor}%
\pgfsetfillcolor{textcolor}%
\pgftext[x=3.864809in,y=2.491078in,left,base]{\color{textcolor}\rmfamily\fontsize{10.000000}{12.000000}\selectfont W6(x)}%
\end{pgfscope}%
\begin{pgfscope}%
\pgfsetrectcap%
\pgfsetroundjoin%
\pgfsetlinewidth{1.505625pt}%
\definecolor{currentstroke}{rgb}{0.580392,0.403922,0.741176}%
\pgfsetstrokecolor{currentstroke}%
\pgfsetdash{}{0pt}%
\pgfpathmoveto{\pgfqpoint{3.475920in}{2.331356in}}%
\pgfpathlineto{\pgfqpoint{3.753698in}{2.331356in}}%
\pgfusepath{stroke}%
\end{pgfscope}%
\begin{pgfscope}%
\definecolor{textcolor}{rgb}{0.000000,0.000000,0.000000}%
\pgfsetstrokecolor{textcolor}%
\pgfsetfillcolor{textcolor}%
\pgftext[x=3.864809in,y=2.282745in,left,base]{\color{textcolor}\rmfamily\fontsize{10.000000}{12.000000}\selectfont W8(x)}%
\end{pgfscope}%
\begin{pgfscope}%
\pgfsetrectcap%
\pgfsetroundjoin%
\pgfsetlinewidth{1.505625pt}%
\definecolor{currentstroke}{rgb}{0.890196,0.466667,0.760784}%
\pgfsetstrokecolor{currentstroke}%
\pgfsetdash{}{0pt}%
\pgfpathmoveto{\pgfqpoint{3.475920in}{2.123023in}}%
\pgfpathlineto{\pgfqpoint{3.753698in}{2.123023in}}%
\pgfusepath{stroke}%
\end{pgfscope}%
\begin{pgfscope}%
\definecolor{textcolor}{rgb}{0.000000,0.000000,0.000000}%
\pgfsetstrokecolor{textcolor}%
\pgfsetfillcolor{textcolor}%
\pgftext[x=3.864809in,y=2.074412in,left,base]{\color{textcolor}\rmfamily\fontsize{10.000000}{12.000000}\selectfont W16(x)}%
\end{pgfscope}%
\begin{pgfscope}%
\pgfsetrectcap%
\pgfsetroundjoin%
\pgfsetlinewidth{1.505625pt}%
\definecolor{currentstroke}{rgb}{0.737255,0.741176,0.133333}%
\pgfsetstrokecolor{currentstroke}%
\pgfsetdash{}{0pt}%
\pgfpathmoveto{\pgfqpoint{3.475920in}{1.914689in}}%
\pgfpathlineto{\pgfqpoint{3.753698in}{1.914689in}}%
\pgfusepath{stroke}%
\end{pgfscope}%
\begin{pgfscope}%
\definecolor{textcolor}{rgb}{0.000000,0.000000,0.000000}%
\pgfsetstrokecolor{textcolor}%
\pgfsetfillcolor{textcolor}%
\pgftext[x=3.864809in,y=1.866078in,left,base]{\color{textcolor}\rmfamily\fontsize{10.000000}{12.000000}\selectfont W32(x)}%
\end{pgfscope}%
\end{pgfpicture}%
\makeatother%
\endgroup%
        
    \end{center}
    \caption{Węzły jednorodne, funkcja \(y\), \(N=6,8,16,32\)}
\end{figure}

\begin{figure}[h]
    \begin{center}
        %% Creator: Matplotlib, PGF backend
%%
%% To include the figure in your LaTeX document, write
%%   \input{<filename>.pgf}
%%
%% Make sure the required packages are loaded in your preamble
%%   \usepackage{pgf}
%%
%% Figures using additional raster images can only be included by \input if
%% they are in the same directory as the main LaTeX file. For loading figures
%% from other directories you can use the `import` package
%%   \usepackage{import}
%% and then include the figures with
%%   \import{<path to file>}{<filename>.pgf}
%%
%% Matplotlib used the following preamble
%%
\begingroup%
\makeatletter%
\begin{pgfpicture}%
\pgfpathrectangle{\pgfpointorigin}{\pgfqpoint{5.500000in}{3.500000in}}%
\pgfusepath{use as bounding box, clip}%
\begin{pgfscope}%
\pgfsetbuttcap%
\pgfsetmiterjoin%
\definecolor{currentfill}{rgb}{1.000000,1.000000,1.000000}%
\pgfsetfillcolor{currentfill}%
\pgfsetlinewidth{0.000000pt}%
\definecolor{currentstroke}{rgb}{1.000000,1.000000,1.000000}%
\pgfsetstrokecolor{currentstroke}%
\pgfsetdash{}{0pt}%
\pgfpathmoveto{\pgfqpoint{0.000000in}{0.000000in}}%
\pgfpathlineto{\pgfqpoint{5.500000in}{0.000000in}}%
\pgfpathlineto{\pgfqpoint{5.500000in}{3.500000in}}%
\pgfpathlineto{\pgfqpoint{0.000000in}{3.500000in}}%
\pgfpathclose%
\pgfusepath{fill}%
\end{pgfscope}%
\begin{pgfscope}%
\pgfsetbuttcap%
\pgfsetmiterjoin%
\definecolor{currentfill}{rgb}{1.000000,1.000000,1.000000}%
\pgfsetfillcolor{currentfill}%
\pgfsetlinewidth{0.000000pt}%
\definecolor{currentstroke}{rgb}{0.000000,0.000000,0.000000}%
\pgfsetstrokecolor{currentstroke}%
\pgfsetstrokeopacity{0.000000}%
\pgfsetdash{}{0pt}%
\pgfpathmoveto{\pgfqpoint{0.687500in}{0.385000in}}%
\pgfpathlineto{\pgfqpoint{4.950000in}{0.385000in}}%
\pgfpathlineto{\pgfqpoint{4.950000in}{3.080000in}}%
\pgfpathlineto{\pgfqpoint{0.687500in}{3.080000in}}%
\pgfpathclose%
\pgfusepath{fill}%
\end{pgfscope}%
\begin{pgfscope}%
\pgfsetbuttcap%
\pgfsetroundjoin%
\definecolor{currentfill}{rgb}{0.000000,0.000000,0.000000}%
\pgfsetfillcolor{currentfill}%
\pgfsetlinewidth{0.803000pt}%
\definecolor{currentstroke}{rgb}{0.000000,0.000000,0.000000}%
\pgfsetstrokecolor{currentstroke}%
\pgfsetdash{}{0pt}%
\pgfsys@defobject{currentmarker}{\pgfqpoint{0.000000in}{-0.048611in}}{\pgfqpoint{0.000000in}{0.000000in}}{%
\pgfpathmoveto{\pgfqpoint{0.000000in}{0.000000in}}%
\pgfpathlineto{\pgfqpoint{0.000000in}{-0.048611in}}%
\pgfusepath{stroke,fill}%
}%
\begin{pgfscope}%
\pgfsys@transformshift{0.881250in}{0.385000in}%
\pgfsys@useobject{currentmarker}{}%
\end{pgfscope}%
\end{pgfscope}%
\begin{pgfscope}%
\definecolor{textcolor}{rgb}{0.000000,0.000000,0.000000}%
\pgfsetstrokecolor{textcolor}%
\pgfsetfillcolor{textcolor}%
\pgftext[x=0.881250in,y=0.287778in,,top]{\color{textcolor}\rmfamily\fontsize{10.000000}{12.000000}\selectfont \(\displaystyle -1.00\)}%
\end{pgfscope}%
\begin{pgfscope}%
\pgfsetbuttcap%
\pgfsetroundjoin%
\definecolor{currentfill}{rgb}{0.000000,0.000000,0.000000}%
\pgfsetfillcolor{currentfill}%
\pgfsetlinewidth{0.803000pt}%
\definecolor{currentstroke}{rgb}{0.000000,0.000000,0.000000}%
\pgfsetstrokecolor{currentstroke}%
\pgfsetdash{}{0pt}%
\pgfsys@defobject{currentmarker}{\pgfqpoint{0.000000in}{-0.048611in}}{\pgfqpoint{0.000000in}{0.000000in}}{%
\pgfpathmoveto{\pgfqpoint{0.000000in}{0.000000in}}%
\pgfpathlineto{\pgfqpoint{0.000000in}{-0.048611in}}%
\pgfusepath{stroke,fill}%
}%
\begin{pgfscope}%
\pgfsys@transformshift{1.365625in}{0.385000in}%
\pgfsys@useobject{currentmarker}{}%
\end{pgfscope}%
\end{pgfscope}%
\begin{pgfscope}%
\definecolor{textcolor}{rgb}{0.000000,0.000000,0.000000}%
\pgfsetstrokecolor{textcolor}%
\pgfsetfillcolor{textcolor}%
\pgftext[x=1.365625in,y=0.287778in,,top]{\color{textcolor}\rmfamily\fontsize{10.000000}{12.000000}\selectfont \(\displaystyle -0.75\)}%
\end{pgfscope}%
\begin{pgfscope}%
\pgfsetbuttcap%
\pgfsetroundjoin%
\definecolor{currentfill}{rgb}{0.000000,0.000000,0.000000}%
\pgfsetfillcolor{currentfill}%
\pgfsetlinewidth{0.803000pt}%
\definecolor{currentstroke}{rgb}{0.000000,0.000000,0.000000}%
\pgfsetstrokecolor{currentstroke}%
\pgfsetdash{}{0pt}%
\pgfsys@defobject{currentmarker}{\pgfqpoint{0.000000in}{-0.048611in}}{\pgfqpoint{0.000000in}{0.000000in}}{%
\pgfpathmoveto{\pgfqpoint{0.000000in}{0.000000in}}%
\pgfpathlineto{\pgfqpoint{0.000000in}{-0.048611in}}%
\pgfusepath{stroke,fill}%
}%
\begin{pgfscope}%
\pgfsys@transformshift{1.850000in}{0.385000in}%
\pgfsys@useobject{currentmarker}{}%
\end{pgfscope}%
\end{pgfscope}%
\begin{pgfscope}%
\definecolor{textcolor}{rgb}{0.000000,0.000000,0.000000}%
\pgfsetstrokecolor{textcolor}%
\pgfsetfillcolor{textcolor}%
\pgftext[x=1.850000in,y=0.287778in,,top]{\color{textcolor}\rmfamily\fontsize{10.000000}{12.000000}\selectfont \(\displaystyle -0.50\)}%
\end{pgfscope}%
\begin{pgfscope}%
\pgfsetbuttcap%
\pgfsetroundjoin%
\definecolor{currentfill}{rgb}{0.000000,0.000000,0.000000}%
\pgfsetfillcolor{currentfill}%
\pgfsetlinewidth{0.803000pt}%
\definecolor{currentstroke}{rgb}{0.000000,0.000000,0.000000}%
\pgfsetstrokecolor{currentstroke}%
\pgfsetdash{}{0pt}%
\pgfsys@defobject{currentmarker}{\pgfqpoint{0.000000in}{-0.048611in}}{\pgfqpoint{0.000000in}{0.000000in}}{%
\pgfpathmoveto{\pgfqpoint{0.000000in}{0.000000in}}%
\pgfpathlineto{\pgfqpoint{0.000000in}{-0.048611in}}%
\pgfusepath{stroke,fill}%
}%
\begin{pgfscope}%
\pgfsys@transformshift{2.334375in}{0.385000in}%
\pgfsys@useobject{currentmarker}{}%
\end{pgfscope}%
\end{pgfscope}%
\begin{pgfscope}%
\definecolor{textcolor}{rgb}{0.000000,0.000000,0.000000}%
\pgfsetstrokecolor{textcolor}%
\pgfsetfillcolor{textcolor}%
\pgftext[x=2.334375in,y=0.287778in,,top]{\color{textcolor}\rmfamily\fontsize{10.000000}{12.000000}\selectfont \(\displaystyle -0.25\)}%
\end{pgfscope}%
\begin{pgfscope}%
\pgfsetbuttcap%
\pgfsetroundjoin%
\definecolor{currentfill}{rgb}{0.000000,0.000000,0.000000}%
\pgfsetfillcolor{currentfill}%
\pgfsetlinewidth{0.803000pt}%
\definecolor{currentstroke}{rgb}{0.000000,0.000000,0.000000}%
\pgfsetstrokecolor{currentstroke}%
\pgfsetdash{}{0pt}%
\pgfsys@defobject{currentmarker}{\pgfqpoint{0.000000in}{-0.048611in}}{\pgfqpoint{0.000000in}{0.000000in}}{%
\pgfpathmoveto{\pgfqpoint{0.000000in}{0.000000in}}%
\pgfpathlineto{\pgfqpoint{0.000000in}{-0.048611in}}%
\pgfusepath{stroke,fill}%
}%
\begin{pgfscope}%
\pgfsys@transformshift{2.818750in}{0.385000in}%
\pgfsys@useobject{currentmarker}{}%
\end{pgfscope}%
\end{pgfscope}%
\begin{pgfscope}%
\definecolor{textcolor}{rgb}{0.000000,0.000000,0.000000}%
\pgfsetstrokecolor{textcolor}%
\pgfsetfillcolor{textcolor}%
\pgftext[x=2.818750in,y=0.287778in,,top]{\color{textcolor}\rmfamily\fontsize{10.000000}{12.000000}\selectfont \(\displaystyle 0.00\)}%
\end{pgfscope}%
\begin{pgfscope}%
\pgfsetbuttcap%
\pgfsetroundjoin%
\definecolor{currentfill}{rgb}{0.000000,0.000000,0.000000}%
\pgfsetfillcolor{currentfill}%
\pgfsetlinewidth{0.803000pt}%
\definecolor{currentstroke}{rgb}{0.000000,0.000000,0.000000}%
\pgfsetstrokecolor{currentstroke}%
\pgfsetdash{}{0pt}%
\pgfsys@defobject{currentmarker}{\pgfqpoint{0.000000in}{-0.048611in}}{\pgfqpoint{0.000000in}{0.000000in}}{%
\pgfpathmoveto{\pgfqpoint{0.000000in}{0.000000in}}%
\pgfpathlineto{\pgfqpoint{0.000000in}{-0.048611in}}%
\pgfusepath{stroke,fill}%
}%
\begin{pgfscope}%
\pgfsys@transformshift{3.303125in}{0.385000in}%
\pgfsys@useobject{currentmarker}{}%
\end{pgfscope}%
\end{pgfscope}%
\begin{pgfscope}%
\definecolor{textcolor}{rgb}{0.000000,0.000000,0.000000}%
\pgfsetstrokecolor{textcolor}%
\pgfsetfillcolor{textcolor}%
\pgftext[x=3.303125in,y=0.287778in,,top]{\color{textcolor}\rmfamily\fontsize{10.000000}{12.000000}\selectfont \(\displaystyle 0.25\)}%
\end{pgfscope}%
\begin{pgfscope}%
\pgfsetbuttcap%
\pgfsetroundjoin%
\definecolor{currentfill}{rgb}{0.000000,0.000000,0.000000}%
\pgfsetfillcolor{currentfill}%
\pgfsetlinewidth{0.803000pt}%
\definecolor{currentstroke}{rgb}{0.000000,0.000000,0.000000}%
\pgfsetstrokecolor{currentstroke}%
\pgfsetdash{}{0pt}%
\pgfsys@defobject{currentmarker}{\pgfqpoint{0.000000in}{-0.048611in}}{\pgfqpoint{0.000000in}{0.000000in}}{%
\pgfpathmoveto{\pgfqpoint{0.000000in}{0.000000in}}%
\pgfpathlineto{\pgfqpoint{0.000000in}{-0.048611in}}%
\pgfusepath{stroke,fill}%
}%
\begin{pgfscope}%
\pgfsys@transformshift{3.787500in}{0.385000in}%
\pgfsys@useobject{currentmarker}{}%
\end{pgfscope}%
\end{pgfscope}%
\begin{pgfscope}%
\definecolor{textcolor}{rgb}{0.000000,0.000000,0.000000}%
\pgfsetstrokecolor{textcolor}%
\pgfsetfillcolor{textcolor}%
\pgftext[x=3.787500in,y=0.287778in,,top]{\color{textcolor}\rmfamily\fontsize{10.000000}{12.000000}\selectfont \(\displaystyle 0.50\)}%
\end{pgfscope}%
\begin{pgfscope}%
\pgfsetbuttcap%
\pgfsetroundjoin%
\definecolor{currentfill}{rgb}{0.000000,0.000000,0.000000}%
\pgfsetfillcolor{currentfill}%
\pgfsetlinewidth{0.803000pt}%
\definecolor{currentstroke}{rgb}{0.000000,0.000000,0.000000}%
\pgfsetstrokecolor{currentstroke}%
\pgfsetdash{}{0pt}%
\pgfsys@defobject{currentmarker}{\pgfqpoint{0.000000in}{-0.048611in}}{\pgfqpoint{0.000000in}{0.000000in}}{%
\pgfpathmoveto{\pgfqpoint{0.000000in}{0.000000in}}%
\pgfpathlineto{\pgfqpoint{0.000000in}{-0.048611in}}%
\pgfusepath{stroke,fill}%
}%
\begin{pgfscope}%
\pgfsys@transformshift{4.271875in}{0.385000in}%
\pgfsys@useobject{currentmarker}{}%
\end{pgfscope}%
\end{pgfscope}%
\begin{pgfscope}%
\definecolor{textcolor}{rgb}{0.000000,0.000000,0.000000}%
\pgfsetstrokecolor{textcolor}%
\pgfsetfillcolor{textcolor}%
\pgftext[x=4.271875in,y=0.287778in,,top]{\color{textcolor}\rmfamily\fontsize{10.000000}{12.000000}\selectfont \(\displaystyle 0.75\)}%
\end{pgfscope}%
\begin{pgfscope}%
\pgfsetbuttcap%
\pgfsetroundjoin%
\definecolor{currentfill}{rgb}{0.000000,0.000000,0.000000}%
\pgfsetfillcolor{currentfill}%
\pgfsetlinewidth{0.803000pt}%
\definecolor{currentstroke}{rgb}{0.000000,0.000000,0.000000}%
\pgfsetstrokecolor{currentstroke}%
\pgfsetdash{}{0pt}%
\pgfsys@defobject{currentmarker}{\pgfqpoint{0.000000in}{-0.048611in}}{\pgfqpoint{0.000000in}{0.000000in}}{%
\pgfpathmoveto{\pgfqpoint{0.000000in}{0.000000in}}%
\pgfpathlineto{\pgfqpoint{0.000000in}{-0.048611in}}%
\pgfusepath{stroke,fill}%
}%
\begin{pgfscope}%
\pgfsys@transformshift{4.756250in}{0.385000in}%
\pgfsys@useobject{currentmarker}{}%
\end{pgfscope}%
\end{pgfscope}%
\begin{pgfscope}%
\definecolor{textcolor}{rgb}{0.000000,0.000000,0.000000}%
\pgfsetstrokecolor{textcolor}%
\pgfsetfillcolor{textcolor}%
\pgftext[x=4.756250in,y=0.287778in,,top]{\color{textcolor}\rmfamily\fontsize{10.000000}{12.000000}\selectfont \(\displaystyle 1.00\)}%
\end{pgfscope}%
\begin{pgfscope}%
\definecolor{textcolor}{rgb}{0.000000,0.000000,0.000000}%
\pgfsetstrokecolor{textcolor}%
\pgfsetfillcolor{textcolor}%
\pgftext[x=2.818750in,y=0.108766in,,top]{\color{textcolor}\rmfamily\fontsize{10.000000}{12.000000}\selectfont x}%
\end{pgfscope}%
\begin{pgfscope}%
\pgfsetbuttcap%
\pgfsetroundjoin%
\definecolor{currentfill}{rgb}{0.000000,0.000000,0.000000}%
\pgfsetfillcolor{currentfill}%
\pgfsetlinewidth{0.803000pt}%
\definecolor{currentstroke}{rgb}{0.000000,0.000000,0.000000}%
\pgfsetstrokecolor{currentstroke}%
\pgfsetdash{}{0pt}%
\pgfsys@defobject{currentmarker}{\pgfqpoint{-0.048611in}{0.000000in}}{\pgfqpoint{0.000000in}{0.000000in}}{%
\pgfpathmoveto{\pgfqpoint{0.000000in}{0.000000in}}%
\pgfpathlineto{\pgfqpoint{-0.048611in}{0.000000in}}%
\pgfusepath{stroke,fill}%
}%
\begin{pgfscope}%
\pgfsys@transformshift{0.687500in}{0.474833in}%
\pgfsys@useobject{currentmarker}{}%
\end{pgfscope}%
\end{pgfscope}%
\begin{pgfscope}%
\definecolor{textcolor}{rgb}{0.000000,0.000000,0.000000}%
\pgfsetstrokecolor{textcolor}%
\pgfsetfillcolor{textcolor}%
\pgftext[x=0.304783in,y=0.426608in,left,base]{\color{textcolor}\rmfamily\fontsize{10.000000}{12.000000}\selectfont \(\displaystyle -0.2\)}%
\end{pgfscope}%
\begin{pgfscope}%
\pgfsetbuttcap%
\pgfsetroundjoin%
\definecolor{currentfill}{rgb}{0.000000,0.000000,0.000000}%
\pgfsetfillcolor{currentfill}%
\pgfsetlinewidth{0.803000pt}%
\definecolor{currentstroke}{rgb}{0.000000,0.000000,0.000000}%
\pgfsetstrokecolor{currentstroke}%
\pgfsetdash{}{0pt}%
\pgfsys@defobject{currentmarker}{\pgfqpoint{-0.048611in}{0.000000in}}{\pgfqpoint{0.000000in}{0.000000in}}{%
\pgfpathmoveto{\pgfqpoint{0.000000in}{0.000000in}}%
\pgfpathlineto{\pgfqpoint{-0.048611in}{0.000000in}}%
\pgfusepath{stroke,fill}%
}%
\begin{pgfscope}%
\pgfsys@transformshift{0.687500in}{0.834167in}%
\pgfsys@useobject{currentmarker}{}%
\end{pgfscope}%
\end{pgfscope}%
\begin{pgfscope}%
\definecolor{textcolor}{rgb}{0.000000,0.000000,0.000000}%
\pgfsetstrokecolor{textcolor}%
\pgfsetfillcolor{textcolor}%
\pgftext[x=0.412808in,y=0.785941in,left,base]{\color{textcolor}\rmfamily\fontsize{10.000000}{12.000000}\selectfont \(\displaystyle 0.0\)}%
\end{pgfscope}%
\begin{pgfscope}%
\pgfsetbuttcap%
\pgfsetroundjoin%
\definecolor{currentfill}{rgb}{0.000000,0.000000,0.000000}%
\pgfsetfillcolor{currentfill}%
\pgfsetlinewidth{0.803000pt}%
\definecolor{currentstroke}{rgb}{0.000000,0.000000,0.000000}%
\pgfsetstrokecolor{currentstroke}%
\pgfsetdash{}{0pt}%
\pgfsys@defobject{currentmarker}{\pgfqpoint{-0.048611in}{0.000000in}}{\pgfqpoint{0.000000in}{0.000000in}}{%
\pgfpathmoveto{\pgfqpoint{0.000000in}{0.000000in}}%
\pgfpathlineto{\pgfqpoint{-0.048611in}{0.000000in}}%
\pgfusepath{stroke,fill}%
}%
\begin{pgfscope}%
\pgfsys@transformshift{0.687500in}{1.193500in}%
\pgfsys@useobject{currentmarker}{}%
\end{pgfscope}%
\end{pgfscope}%
\begin{pgfscope}%
\definecolor{textcolor}{rgb}{0.000000,0.000000,0.000000}%
\pgfsetstrokecolor{textcolor}%
\pgfsetfillcolor{textcolor}%
\pgftext[x=0.412808in,y=1.145275in,left,base]{\color{textcolor}\rmfamily\fontsize{10.000000}{12.000000}\selectfont \(\displaystyle 0.2\)}%
\end{pgfscope}%
\begin{pgfscope}%
\pgfsetbuttcap%
\pgfsetroundjoin%
\definecolor{currentfill}{rgb}{0.000000,0.000000,0.000000}%
\pgfsetfillcolor{currentfill}%
\pgfsetlinewidth{0.803000pt}%
\definecolor{currentstroke}{rgb}{0.000000,0.000000,0.000000}%
\pgfsetstrokecolor{currentstroke}%
\pgfsetdash{}{0pt}%
\pgfsys@defobject{currentmarker}{\pgfqpoint{-0.048611in}{0.000000in}}{\pgfqpoint{0.000000in}{0.000000in}}{%
\pgfpathmoveto{\pgfqpoint{0.000000in}{0.000000in}}%
\pgfpathlineto{\pgfqpoint{-0.048611in}{0.000000in}}%
\pgfusepath{stroke,fill}%
}%
\begin{pgfscope}%
\pgfsys@transformshift{0.687500in}{1.552833in}%
\pgfsys@useobject{currentmarker}{}%
\end{pgfscope}%
\end{pgfscope}%
\begin{pgfscope}%
\definecolor{textcolor}{rgb}{0.000000,0.000000,0.000000}%
\pgfsetstrokecolor{textcolor}%
\pgfsetfillcolor{textcolor}%
\pgftext[x=0.412808in,y=1.504608in,left,base]{\color{textcolor}\rmfamily\fontsize{10.000000}{12.000000}\selectfont \(\displaystyle 0.4\)}%
\end{pgfscope}%
\begin{pgfscope}%
\pgfsetbuttcap%
\pgfsetroundjoin%
\definecolor{currentfill}{rgb}{0.000000,0.000000,0.000000}%
\pgfsetfillcolor{currentfill}%
\pgfsetlinewidth{0.803000pt}%
\definecolor{currentstroke}{rgb}{0.000000,0.000000,0.000000}%
\pgfsetstrokecolor{currentstroke}%
\pgfsetdash{}{0pt}%
\pgfsys@defobject{currentmarker}{\pgfqpoint{-0.048611in}{0.000000in}}{\pgfqpoint{0.000000in}{0.000000in}}{%
\pgfpathmoveto{\pgfqpoint{0.000000in}{0.000000in}}%
\pgfpathlineto{\pgfqpoint{-0.048611in}{0.000000in}}%
\pgfusepath{stroke,fill}%
}%
\begin{pgfscope}%
\pgfsys@transformshift{0.687500in}{1.912167in}%
\pgfsys@useobject{currentmarker}{}%
\end{pgfscope}%
\end{pgfscope}%
\begin{pgfscope}%
\definecolor{textcolor}{rgb}{0.000000,0.000000,0.000000}%
\pgfsetstrokecolor{textcolor}%
\pgfsetfillcolor{textcolor}%
\pgftext[x=0.412808in,y=1.863941in,left,base]{\color{textcolor}\rmfamily\fontsize{10.000000}{12.000000}\selectfont \(\displaystyle 0.6\)}%
\end{pgfscope}%
\begin{pgfscope}%
\pgfsetbuttcap%
\pgfsetroundjoin%
\definecolor{currentfill}{rgb}{0.000000,0.000000,0.000000}%
\pgfsetfillcolor{currentfill}%
\pgfsetlinewidth{0.803000pt}%
\definecolor{currentstroke}{rgb}{0.000000,0.000000,0.000000}%
\pgfsetstrokecolor{currentstroke}%
\pgfsetdash{}{0pt}%
\pgfsys@defobject{currentmarker}{\pgfqpoint{-0.048611in}{0.000000in}}{\pgfqpoint{0.000000in}{0.000000in}}{%
\pgfpathmoveto{\pgfqpoint{0.000000in}{0.000000in}}%
\pgfpathlineto{\pgfqpoint{-0.048611in}{0.000000in}}%
\pgfusepath{stroke,fill}%
}%
\begin{pgfscope}%
\pgfsys@transformshift{0.687500in}{2.271500in}%
\pgfsys@useobject{currentmarker}{}%
\end{pgfscope}%
\end{pgfscope}%
\begin{pgfscope}%
\definecolor{textcolor}{rgb}{0.000000,0.000000,0.000000}%
\pgfsetstrokecolor{textcolor}%
\pgfsetfillcolor{textcolor}%
\pgftext[x=0.412808in,y=2.223275in,left,base]{\color{textcolor}\rmfamily\fontsize{10.000000}{12.000000}\selectfont \(\displaystyle 0.8\)}%
\end{pgfscope}%
\begin{pgfscope}%
\pgfsetbuttcap%
\pgfsetroundjoin%
\definecolor{currentfill}{rgb}{0.000000,0.000000,0.000000}%
\pgfsetfillcolor{currentfill}%
\pgfsetlinewidth{0.803000pt}%
\definecolor{currentstroke}{rgb}{0.000000,0.000000,0.000000}%
\pgfsetstrokecolor{currentstroke}%
\pgfsetdash{}{0pt}%
\pgfsys@defobject{currentmarker}{\pgfqpoint{-0.048611in}{0.000000in}}{\pgfqpoint{0.000000in}{0.000000in}}{%
\pgfpathmoveto{\pgfqpoint{0.000000in}{0.000000in}}%
\pgfpathlineto{\pgfqpoint{-0.048611in}{0.000000in}}%
\pgfusepath{stroke,fill}%
}%
\begin{pgfscope}%
\pgfsys@transformshift{0.687500in}{2.630833in}%
\pgfsys@useobject{currentmarker}{}%
\end{pgfscope}%
\end{pgfscope}%
\begin{pgfscope}%
\definecolor{textcolor}{rgb}{0.000000,0.000000,0.000000}%
\pgfsetstrokecolor{textcolor}%
\pgfsetfillcolor{textcolor}%
\pgftext[x=0.412808in,y=2.582608in,left,base]{\color{textcolor}\rmfamily\fontsize{10.000000}{12.000000}\selectfont \(\displaystyle 1.0\)}%
\end{pgfscope}%
\begin{pgfscope}%
\pgfsetbuttcap%
\pgfsetroundjoin%
\definecolor{currentfill}{rgb}{0.000000,0.000000,0.000000}%
\pgfsetfillcolor{currentfill}%
\pgfsetlinewidth{0.803000pt}%
\definecolor{currentstroke}{rgb}{0.000000,0.000000,0.000000}%
\pgfsetstrokecolor{currentstroke}%
\pgfsetdash{}{0pt}%
\pgfsys@defobject{currentmarker}{\pgfqpoint{-0.048611in}{0.000000in}}{\pgfqpoint{0.000000in}{0.000000in}}{%
\pgfpathmoveto{\pgfqpoint{0.000000in}{0.000000in}}%
\pgfpathlineto{\pgfqpoint{-0.048611in}{0.000000in}}%
\pgfusepath{stroke,fill}%
}%
\begin{pgfscope}%
\pgfsys@transformshift{0.687500in}{2.990167in}%
\pgfsys@useobject{currentmarker}{}%
\end{pgfscope}%
\end{pgfscope}%
\begin{pgfscope}%
\definecolor{textcolor}{rgb}{0.000000,0.000000,0.000000}%
\pgfsetstrokecolor{textcolor}%
\pgfsetfillcolor{textcolor}%
\pgftext[x=0.412808in,y=2.941941in,left,base]{\color{textcolor}\rmfamily\fontsize{10.000000}{12.000000}\selectfont \(\displaystyle 1.2\)}%
\end{pgfscope}%
\begin{pgfscope}%
\definecolor{textcolor}{rgb}{0.000000,0.000000,0.000000}%
\pgfsetstrokecolor{textcolor}%
\pgfsetfillcolor{textcolor}%
\pgftext[x=0.249228in,y=1.732500in,,bottom,rotate=90.000000]{\color{textcolor}\rmfamily\fontsize{10.000000}{12.000000}\selectfont y}%
\end{pgfscope}%
\begin{pgfscope}%
\pgfpathrectangle{\pgfqpoint{0.687500in}{0.385000in}}{\pgfqpoint{4.262500in}{2.695000in}}%
\pgfusepath{clip}%
\pgfsetrectcap%
\pgfsetroundjoin%
\pgfsetlinewidth{1.505625pt}%
\definecolor{currentstroke}{rgb}{0.121569,0.466667,0.705882}%
\pgfsetstrokecolor{currentstroke}%
\pgfsetdash{}{0pt}%
\pgfpathmoveto{\pgfqpoint{0.881250in}{0.903269in}}%
\pgfpathlineto{\pgfqpoint{1.017557in}{0.913644in}}%
\pgfpathlineto{\pgfqpoint{1.134391in}{0.924478in}}%
\pgfpathlineto{\pgfqpoint{1.251225in}{0.937638in}}%
\pgfpathlineto{\pgfqpoint{1.348587in}{0.950877in}}%
\pgfpathlineto{\pgfqpoint{1.426476in}{0.963336in}}%
\pgfpathlineto{\pgfqpoint{1.504366in}{0.977838in}}%
\pgfpathlineto{\pgfqpoint{1.582255in}{0.994839in}}%
\pgfpathlineto{\pgfqpoint{1.640672in}{1.009574in}}%
\pgfpathlineto{\pgfqpoint{1.699089in}{1.026346in}}%
\pgfpathlineto{\pgfqpoint{1.757506in}{1.045528in}}%
\pgfpathlineto{\pgfqpoint{1.815923in}{1.067578in}}%
\pgfpathlineto{\pgfqpoint{1.854868in}{1.084143in}}%
\pgfpathlineto{\pgfqpoint{1.893813in}{1.102428in}}%
\pgfpathlineto{\pgfqpoint{1.932758in}{1.122659in}}%
\pgfpathlineto{\pgfqpoint{1.971702in}{1.145101in}}%
\pgfpathlineto{\pgfqpoint{2.010647in}{1.170055in}}%
\pgfpathlineto{\pgfqpoint{2.049592in}{1.197870in}}%
\pgfpathlineto{\pgfqpoint{2.088536in}{1.228948in}}%
\pgfpathlineto{\pgfqpoint{2.127481in}{1.263748in}}%
\pgfpathlineto{\pgfqpoint{2.166426in}{1.302794in}}%
\pgfpathlineto{\pgfqpoint{2.205371in}{1.346677in}}%
\pgfpathlineto{\pgfqpoint{2.244315in}{1.396056in}}%
\pgfpathlineto{\pgfqpoint{2.283260in}{1.451647in}}%
\pgfpathlineto{\pgfqpoint{2.322205in}{1.514206in}}%
\pgfpathlineto{\pgfqpoint{2.361149in}{1.584486in}}%
\pgfpathlineto{\pgfqpoint{2.400094in}{1.663167in}}%
\pgfpathlineto{\pgfqpoint{2.439039in}{1.750738in}}%
\pgfpathlineto{\pgfqpoint{2.477984in}{1.847321in}}%
\pgfpathlineto{\pgfqpoint{2.516928in}{1.952417in}}%
\pgfpathlineto{\pgfqpoint{2.555873in}{2.064579in}}%
\pgfpathlineto{\pgfqpoint{2.653235in}{2.353617in}}%
\pgfpathlineto{\pgfqpoint{2.672707in}{2.407372in}}%
\pgfpathlineto{\pgfqpoint{2.692180in}{2.457628in}}%
\pgfpathlineto{\pgfqpoint{2.711652in}{2.503331in}}%
\pgfpathlineto{\pgfqpoint{2.731124in}{2.543430in}}%
\pgfpathlineto{\pgfqpoint{2.750597in}{2.576924in}}%
\pgfpathlineto{\pgfqpoint{2.770069in}{2.602918in}}%
\pgfpathlineto{\pgfqpoint{2.789541in}{2.620683in}}%
\pgfpathlineto{\pgfqpoint{2.809014in}{2.629700in}}%
\pgfpathlineto{\pgfqpoint{2.828486in}{2.629700in}}%
\pgfpathlineto{\pgfqpoint{2.847959in}{2.620683in}}%
\pgfpathlineto{\pgfqpoint{2.867431in}{2.602918in}}%
\pgfpathlineto{\pgfqpoint{2.886903in}{2.576924in}}%
\pgfpathlineto{\pgfqpoint{2.906376in}{2.543430in}}%
\pgfpathlineto{\pgfqpoint{2.925848in}{2.503331in}}%
\pgfpathlineto{\pgfqpoint{2.945320in}{2.457628in}}%
\pgfpathlineto{\pgfqpoint{2.964793in}{2.407372in}}%
\pgfpathlineto{\pgfqpoint{3.003737in}{2.297371in}}%
\pgfpathlineto{\pgfqpoint{3.120572in}{1.952417in}}%
\pgfpathlineto{\pgfqpoint{3.159516in}{1.847321in}}%
\pgfpathlineto{\pgfqpoint{3.198461in}{1.750738in}}%
\pgfpathlineto{\pgfqpoint{3.237406in}{1.663167in}}%
\pgfpathlineto{\pgfqpoint{3.276351in}{1.584486in}}%
\pgfpathlineto{\pgfqpoint{3.315295in}{1.514206in}}%
\pgfpathlineto{\pgfqpoint{3.354240in}{1.451647in}}%
\pgfpathlineto{\pgfqpoint{3.393185in}{1.396056in}}%
\pgfpathlineto{\pgfqpoint{3.432129in}{1.346677in}}%
\pgfpathlineto{\pgfqpoint{3.471074in}{1.302794in}}%
\pgfpathlineto{\pgfqpoint{3.510019in}{1.263748in}}%
\pgfpathlineto{\pgfqpoint{3.548964in}{1.228948in}}%
\pgfpathlineto{\pgfqpoint{3.587908in}{1.197870in}}%
\pgfpathlineto{\pgfqpoint{3.626853in}{1.170055in}}%
\pgfpathlineto{\pgfqpoint{3.665798in}{1.145101in}}%
\pgfpathlineto{\pgfqpoint{3.704742in}{1.122659in}}%
\pgfpathlineto{\pgfqpoint{3.743687in}{1.102428in}}%
\pgfpathlineto{\pgfqpoint{3.782632in}{1.084143in}}%
\pgfpathlineto{\pgfqpoint{3.841049in}{1.059877in}}%
\pgfpathlineto{\pgfqpoint{3.899466in}{1.038840in}}%
\pgfpathlineto{\pgfqpoint{3.957883in}{1.020508in}}%
\pgfpathlineto{\pgfqpoint{4.016300in}{1.004452in}}%
\pgfpathlineto{\pgfqpoint{4.074717in}{0.990325in}}%
\pgfpathlineto{\pgfqpoint{4.152607in}{0.973998in}}%
\pgfpathlineto{\pgfqpoint{4.230496in}{0.960045in}}%
\pgfpathlineto{\pgfqpoint{4.327858in}{0.945299in}}%
\pgfpathlineto{\pgfqpoint{4.425220in}{0.932955in}}%
\pgfpathlineto{\pgfqpoint{4.542054in}{0.920637in}}%
\pgfpathlineto{\pgfqpoint{4.678361in}{0.908933in}}%
\pgfpathlineto{\pgfqpoint{4.756250in}{0.903269in}}%
\pgfpathlineto{\pgfqpoint{4.756250in}{0.903269in}}%
\pgfusepath{stroke}%
\end{pgfscope}%
\begin{pgfscope}%
\pgfpathrectangle{\pgfqpoint{0.687500in}{0.385000in}}{\pgfqpoint{4.262500in}{2.695000in}}%
\pgfusepath{clip}%
\pgfsetbuttcap%
\pgfsetroundjoin%
\definecolor{currentfill}{rgb}{1.000000,0.498039,0.054902}%
\pgfsetfillcolor{currentfill}%
\pgfsetlinewidth{1.003750pt}%
\definecolor{currentstroke}{rgb}{1.000000,0.498039,0.054902}%
\pgfsetstrokecolor{currentstroke}%
\pgfsetdash{}{0pt}%
\pgfsys@defobject{currentmarker}{\pgfqpoint{-0.020833in}{-0.020833in}}{\pgfqpoint{0.020833in}{0.020833in}}{%
\pgfpathmoveto{\pgfqpoint{0.000000in}{-0.020833in}}%
\pgfpathcurveto{\pgfqpoint{0.005525in}{-0.020833in}}{\pgfqpoint{0.010825in}{-0.018638in}}{\pgfqpoint{0.014731in}{-0.014731in}}%
\pgfpathcurveto{\pgfqpoint{0.018638in}{-0.010825in}}{\pgfqpoint{0.020833in}{-0.005525in}}{\pgfqpoint{0.020833in}{0.000000in}}%
\pgfpathcurveto{\pgfqpoint{0.020833in}{0.005525in}}{\pgfqpoint{0.018638in}{0.010825in}}{\pgfqpoint{0.014731in}{0.014731in}}%
\pgfpathcurveto{\pgfqpoint{0.010825in}{0.018638in}}{\pgfqpoint{0.005525in}{0.020833in}}{\pgfqpoint{0.000000in}{0.020833in}}%
\pgfpathcurveto{\pgfqpoint{-0.005525in}{0.020833in}}{\pgfqpoint{-0.010825in}{0.018638in}}{\pgfqpoint{-0.014731in}{0.014731in}}%
\pgfpathcurveto{\pgfqpoint{-0.018638in}{0.010825in}}{\pgfqpoint{-0.020833in}{0.005525in}}{\pgfqpoint{-0.020833in}{0.000000in}}%
\pgfpathcurveto{\pgfqpoint{-0.020833in}{-0.005525in}}{\pgfqpoint{-0.018638in}{-0.010825in}}{\pgfqpoint{-0.014731in}{-0.014731in}}%
\pgfpathcurveto{\pgfqpoint{-0.010825in}{-0.018638in}}{\pgfqpoint{-0.005525in}{-0.020833in}}{\pgfqpoint{0.000000in}{-0.020833in}}%
\pgfpathclose%
\pgfusepath{stroke,fill}%
}%
\begin{pgfscope}%
\pgfsys@transformshift{0.881250in}{0.903269in}%
\pgfsys@useobject{currentmarker}{}%
\end{pgfscope}%
\begin{pgfscope}%
\pgfsys@transformshift{1.434821in}{0.964785in}%
\pgfsys@useobject{currentmarker}{}%
\end{pgfscope}%
\begin{pgfscope}%
\pgfsys@transformshift{1.988393in}{1.155468in}%
\pgfsys@useobject{currentmarker}{}%
\end{pgfscope}%
\begin{pgfscope}%
\pgfsys@transformshift{2.541964in}{2.023851in}%
\pgfsys@useobject{currentmarker}{}%
\end{pgfscope}%
\begin{pgfscope}%
\pgfsys@transformshift{3.095536in}{2.023851in}%
\pgfsys@useobject{currentmarker}{}%
\end{pgfscope}%
\begin{pgfscope}%
\pgfsys@transformshift{3.649107in}{1.155468in}%
\pgfsys@useobject{currentmarker}{}%
\end{pgfscope}%
\begin{pgfscope}%
\pgfsys@transformshift{4.202679in}{0.964785in}%
\pgfsys@useobject{currentmarker}{}%
\end{pgfscope}%
\end{pgfscope}%
\begin{pgfscope}%
\pgfpathrectangle{\pgfqpoint{0.687500in}{0.385000in}}{\pgfqpoint{4.262500in}{2.695000in}}%
\pgfusepath{clip}%
\pgfsetrectcap%
\pgfsetroundjoin%
\pgfsetlinewidth{1.505625pt}%
\definecolor{currentstroke}{rgb}{0.172549,0.627451,0.172549}%
\pgfsetstrokecolor{currentstroke}%
\pgfsetdash{}{0pt}%
\pgfpathmoveto{\pgfqpoint{0.881250in}{0.903269in}}%
\pgfpathlineto{\pgfqpoint{0.900722in}{0.970053in}}%
\pgfpathlineto{\pgfqpoint{0.920195in}{1.027476in}}%
\pgfpathlineto{\pgfqpoint{0.939667in}{1.076231in}}%
\pgfpathlineto{\pgfqpoint{0.959139in}{1.116979in}}%
\pgfpathlineto{\pgfqpoint{0.978612in}{1.150353in}}%
\pgfpathlineto{\pgfqpoint{0.998084in}{1.176960in}}%
\pgfpathlineto{\pgfqpoint{1.017557in}{1.197380in}}%
\pgfpathlineto{\pgfqpoint{1.037029in}{1.212165in}}%
\pgfpathlineto{\pgfqpoint{1.056501in}{1.221844in}}%
\pgfpathlineto{\pgfqpoint{1.075974in}{1.226920in}}%
\pgfpathlineto{\pgfqpoint{1.095446in}{1.227871in}}%
\pgfpathlineto{\pgfqpoint{1.114918in}{1.225152in}}%
\pgfpathlineto{\pgfqpoint{1.134391in}{1.219195in}}%
\pgfpathlineto{\pgfqpoint{1.153863in}{1.210411in}}%
\pgfpathlineto{\pgfqpoint{1.173335in}{1.199186in}}%
\pgfpathlineto{\pgfqpoint{1.192808in}{1.185888in}}%
\pgfpathlineto{\pgfqpoint{1.231753in}{1.154435in}}%
\pgfpathlineto{\pgfqpoint{1.270697in}{1.118582in}}%
\pgfpathlineto{\pgfqpoint{1.387531in}{1.005499in}}%
\pgfpathlineto{\pgfqpoint{1.426476in}{0.971564in}}%
\pgfpathlineto{\pgfqpoint{1.465421in}{0.941692in}}%
\pgfpathlineto{\pgfqpoint{1.484893in}{0.928591in}}%
\pgfpathlineto{\pgfqpoint{1.504366in}{0.916854in}}%
\pgfpathlineto{\pgfqpoint{1.523838in}{0.906571in}}%
\pgfpathlineto{\pgfqpoint{1.543310in}{0.897820in}}%
\pgfpathlineto{\pgfqpoint{1.562783in}{0.890668in}}%
\pgfpathlineto{\pgfqpoint{1.582255in}{0.885172in}}%
\pgfpathlineto{\pgfqpoint{1.601727in}{0.881376in}}%
\pgfpathlineto{\pgfqpoint{1.621200in}{0.879318in}}%
\pgfpathlineto{\pgfqpoint{1.640672in}{0.879024in}}%
\pgfpathlineto{\pgfqpoint{1.660144in}{0.880510in}}%
\pgfpathlineto{\pgfqpoint{1.679617in}{0.883787in}}%
\pgfpathlineto{\pgfqpoint{1.699089in}{0.888853in}}%
\pgfpathlineto{\pgfqpoint{1.718562in}{0.895703in}}%
\pgfpathlineto{\pgfqpoint{1.738034in}{0.904321in}}%
\pgfpathlineto{\pgfqpoint{1.757506in}{0.914686in}}%
\pgfpathlineto{\pgfqpoint{1.776979in}{0.926770in}}%
\pgfpathlineto{\pgfqpoint{1.796451in}{0.940537in}}%
\pgfpathlineto{\pgfqpoint{1.815923in}{0.955947in}}%
\pgfpathlineto{\pgfqpoint{1.835396in}{0.972955in}}%
\pgfpathlineto{\pgfqpoint{1.874340in}{1.011555in}}%
\pgfpathlineto{\pgfqpoint{1.913285in}{1.055870in}}%
\pgfpathlineto{\pgfqpoint{1.952230in}{1.105371in}}%
\pgfpathlineto{\pgfqpoint{1.991175in}{1.159472in}}%
\pgfpathlineto{\pgfqpoint{2.030119in}{1.217541in}}%
\pgfpathlineto{\pgfqpoint{2.069064in}{1.278912in}}%
\pgfpathlineto{\pgfqpoint{2.127481in}{1.375635in}}%
\pgfpathlineto{\pgfqpoint{2.224843in}{1.543309in}}%
\pgfpathlineto{\pgfqpoint{2.302732in}{1.676797in}}%
\pgfpathlineto{\pgfqpoint{2.361149in}{1.772864in}}%
\pgfpathlineto{\pgfqpoint{2.400094in}{1.833743in}}%
\pgfpathlineto{\pgfqpoint{2.439039in}{1.891368in}}%
\pgfpathlineto{\pgfqpoint{2.477984in}{1.945165in}}%
\pgfpathlineto{\pgfqpoint{2.516928in}{1.994601in}}%
\pgfpathlineto{\pgfqpoint{2.555873in}{2.039192in}}%
\pgfpathlineto{\pgfqpoint{2.594818in}{2.078501in}}%
\pgfpathlineto{\pgfqpoint{2.633763in}{2.112149in}}%
\pgfpathlineto{\pgfqpoint{2.653235in}{2.126746in}}%
\pgfpathlineto{\pgfqpoint{2.672707in}{2.139811in}}%
\pgfpathlineto{\pgfqpoint{2.692180in}{2.151312in}}%
\pgfpathlineto{\pgfqpoint{2.711652in}{2.161221in}}%
\pgfpathlineto{\pgfqpoint{2.731124in}{2.169515in}}%
\pgfpathlineto{\pgfqpoint{2.750597in}{2.176174in}}%
\pgfpathlineto{\pgfqpoint{2.770069in}{2.181183in}}%
\pgfpathlineto{\pgfqpoint{2.789541in}{2.184528in}}%
\pgfpathlineto{\pgfqpoint{2.809014in}{2.186203in}}%
\pgfpathlineto{\pgfqpoint{2.828486in}{2.186203in}}%
\pgfpathlineto{\pgfqpoint{2.847959in}{2.184528in}}%
\pgfpathlineto{\pgfqpoint{2.867431in}{2.181183in}}%
\pgfpathlineto{\pgfqpoint{2.886903in}{2.176174in}}%
\pgfpathlineto{\pgfqpoint{2.906376in}{2.169515in}}%
\pgfpathlineto{\pgfqpoint{2.925848in}{2.161221in}}%
\pgfpathlineto{\pgfqpoint{2.945320in}{2.151312in}}%
\pgfpathlineto{\pgfqpoint{2.964793in}{2.139811in}}%
\pgfpathlineto{\pgfqpoint{2.984265in}{2.126746in}}%
\pgfpathlineto{\pgfqpoint{3.003737in}{2.112149in}}%
\pgfpathlineto{\pgfqpoint{3.023210in}{2.096055in}}%
\pgfpathlineto{\pgfqpoint{3.062155in}{2.059532in}}%
\pgfpathlineto{\pgfqpoint{3.101099in}{2.017531in}}%
\pgfpathlineto{\pgfqpoint{3.140044in}{1.970460in}}%
\pgfpathlineto{\pgfqpoint{3.178989in}{1.918780in}}%
\pgfpathlineto{\pgfqpoint{3.217933in}{1.862999in}}%
\pgfpathlineto{\pgfqpoint{3.256878in}{1.803673in}}%
\pgfpathlineto{\pgfqpoint{3.315295in}{1.709345in}}%
\pgfpathlineto{\pgfqpoint{3.393185in}{1.577002in}}%
\pgfpathlineto{\pgfqpoint{3.529491in}{1.342890in}}%
\pgfpathlineto{\pgfqpoint{3.587908in}{1.247857in}}%
\pgfpathlineto{\pgfqpoint{3.626853in}{1.188052in}}%
\pgfpathlineto{\pgfqpoint{3.665798in}{1.131885in}}%
\pgfpathlineto{\pgfqpoint{3.704742in}{1.080008in}}%
\pgfpathlineto{\pgfqpoint{3.743687in}{1.033030in}}%
\pgfpathlineto{\pgfqpoint{3.782632in}{0.991510in}}%
\pgfpathlineto{\pgfqpoint{3.802104in}{0.972955in}}%
\pgfpathlineto{\pgfqpoint{3.821577in}{0.955947in}}%
\pgfpathlineto{\pgfqpoint{3.841049in}{0.940537in}}%
\pgfpathlineto{\pgfqpoint{3.860521in}{0.926770in}}%
\pgfpathlineto{\pgfqpoint{3.879994in}{0.914686in}}%
\pgfpathlineto{\pgfqpoint{3.899466in}{0.904321in}}%
\pgfpathlineto{\pgfqpoint{3.918938in}{0.895703in}}%
\pgfpathlineto{\pgfqpoint{3.938411in}{0.888853in}}%
\pgfpathlineto{\pgfqpoint{3.957883in}{0.883787in}}%
\pgfpathlineto{\pgfqpoint{3.977356in}{0.880510in}}%
\pgfpathlineto{\pgfqpoint{3.996828in}{0.879024in}}%
\pgfpathlineto{\pgfqpoint{4.016300in}{0.879318in}}%
\pgfpathlineto{\pgfqpoint{4.035773in}{0.881376in}}%
\pgfpathlineto{\pgfqpoint{4.055245in}{0.885172in}}%
\pgfpathlineto{\pgfqpoint{4.074717in}{0.890668in}}%
\pgfpathlineto{\pgfqpoint{4.094190in}{0.897820in}}%
\pgfpathlineto{\pgfqpoint{4.113662in}{0.906571in}}%
\pgfpathlineto{\pgfqpoint{4.133134in}{0.916854in}}%
\pgfpathlineto{\pgfqpoint{4.152607in}{0.928591in}}%
\pgfpathlineto{\pgfqpoint{4.191552in}{0.956055in}}%
\pgfpathlineto{\pgfqpoint{4.230496in}{0.988093in}}%
\pgfpathlineto{\pgfqpoint{4.269441in}{1.023626in}}%
\pgfpathlineto{\pgfqpoint{4.347330in}{1.099714in}}%
\pgfpathlineto{\pgfqpoint{4.386275in}{1.136913in}}%
\pgfpathlineto{\pgfqpoint{4.425220in}{1.170862in}}%
\pgfpathlineto{\pgfqpoint{4.444692in}{1.185888in}}%
\pgfpathlineto{\pgfqpoint{4.464165in}{1.199186in}}%
\pgfpathlineto{\pgfqpoint{4.483637in}{1.210411in}}%
\pgfpathlineto{\pgfqpoint{4.503109in}{1.219195in}}%
\pgfpathlineto{\pgfqpoint{4.522582in}{1.225152in}}%
\pgfpathlineto{\pgfqpoint{4.542054in}{1.227871in}}%
\pgfpathlineto{\pgfqpoint{4.561526in}{1.226920in}}%
\pgfpathlineto{\pgfqpoint{4.580999in}{1.221844in}}%
\pgfpathlineto{\pgfqpoint{4.600471in}{1.212165in}}%
\pgfpathlineto{\pgfqpoint{4.619943in}{1.197380in}}%
\pgfpathlineto{\pgfqpoint{4.639416in}{1.176960in}}%
\pgfpathlineto{\pgfqpoint{4.658888in}{1.150353in}}%
\pgfpathlineto{\pgfqpoint{4.678361in}{1.116979in}}%
\pgfpathlineto{\pgfqpoint{4.697833in}{1.076231in}}%
\pgfpathlineto{\pgfqpoint{4.717305in}{1.027476in}}%
\pgfpathlineto{\pgfqpoint{4.736778in}{0.970053in}}%
\pgfpathlineto{\pgfqpoint{4.756250in}{0.903269in}}%
\pgfpathlineto{\pgfqpoint{4.756250in}{0.903269in}}%
\pgfusepath{stroke}%
\end{pgfscope}%
\begin{pgfscope}%
\pgfpathrectangle{\pgfqpoint{0.687500in}{0.385000in}}{\pgfqpoint{4.262500in}{2.695000in}}%
\pgfusepath{clip}%
\pgfsetbuttcap%
\pgfsetroundjoin%
\definecolor{currentfill}{rgb}{0.839216,0.152941,0.156863}%
\pgfsetfillcolor{currentfill}%
\pgfsetlinewidth{1.003750pt}%
\definecolor{currentstroke}{rgb}{0.839216,0.152941,0.156863}%
\pgfsetstrokecolor{currentstroke}%
\pgfsetdash{}{0pt}%
\pgfsys@defobject{currentmarker}{\pgfqpoint{-0.020833in}{-0.020833in}}{\pgfqpoint{0.020833in}{0.020833in}}{%
\pgfpathmoveto{\pgfqpoint{0.000000in}{-0.020833in}}%
\pgfpathcurveto{\pgfqpoint{0.005525in}{-0.020833in}}{\pgfqpoint{0.010825in}{-0.018638in}}{\pgfqpoint{0.014731in}{-0.014731in}}%
\pgfpathcurveto{\pgfqpoint{0.018638in}{-0.010825in}}{\pgfqpoint{0.020833in}{-0.005525in}}{\pgfqpoint{0.020833in}{0.000000in}}%
\pgfpathcurveto{\pgfqpoint{0.020833in}{0.005525in}}{\pgfqpoint{0.018638in}{0.010825in}}{\pgfqpoint{0.014731in}{0.014731in}}%
\pgfpathcurveto{\pgfqpoint{0.010825in}{0.018638in}}{\pgfqpoint{0.005525in}{0.020833in}}{\pgfqpoint{0.000000in}{0.020833in}}%
\pgfpathcurveto{\pgfqpoint{-0.005525in}{0.020833in}}{\pgfqpoint{-0.010825in}{0.018638in}}{\pgfqpoint{-0.014731in}{0.014731in}}%
\pgfpathcurveto{\pgfqpoint{-0.018638in}{0.010825in}}{\pgfqpoint{-0.020833in}{0.005525in}}{\pgfqpoint{-0.020833in}{0.000000in}}%
\pgfpathcurveto{\pgfqpoint{-0.020833in}{-0.005525in}}{\pgfqpoint{-0.018638in}{-0.010825in}}{\pgfqpoint{-0.014731in}{-0.014731in}}%
\pgfpathcurveto{\pgfqpoint{-0.010825in}{-0.018638in}}{\pgfqpoint{-0.005525in}{-0.020833in}}{\pgfqpoint{0.000000in}{-0.020833in}}%
\pgfpathclose%
\pgfusepath{stroke,fill}%
}%
\begin{pgfscope}%
\pgfsys@transformshift{0.881250in}{0.903269in}%
\pgfsys@useobject{currentmarker}{}%
\end{pgfscope}%
\begin{pgfscope}%
\pgfsys@transformshift{0.940865in}{0.907544in}%
\pgfsys@useobject{currentmarker}{}%
\end{pgfscope}%
\begin{pgfscope}%
\pgfsys@transformshift{1.000481in}{0.912222in}%
\pgfsys@useobject{currentmarker}{}%
\end{pgfscope}%
\begin{pgfscope}%
\pgfsys@transformshift{1.060096in}{0.917355in}%
\pgfsys@useobject{currentmarker}{}%
\end{pgfscope}%
\begin{pgfscope}%
\pgfsys@transformshift{1.119712in}{0.923001in}%
\pgfsys@useobject{currentmarker}{}%
\end{pgfscope}%
\begin{pgfscope}%
\pgfsys@transformshift{1.179327in}{0.929231in}%
\pgfsys@useobject{currentmarker}{}%
\end{pgfscope}%
\begin{pgfscope}%
\pgfsys@transformshift{1.238942in}{0.936127in}%
\pgfsys@useobject{currentmarker}{}%
\end{pgfscope}%
\begin{pgfscope}%
\pgfsys@transformshift{1.298558in}{0.943783in}%
\pgfsys@useobject{currentmarker}{}%
\end{pgfscope}%
\begin{pgfscope}%
\pgfsys@transformshift{1.358173in}{0.952313in}%
\pgfsys@useobject{currentmarker}{}%
\end{pgfscope}%
\begin{pgfscope}%
\pgfsys@transformshift{1.417788in}{0.961852in}%
\pgfsys@useobject{currentmarker}{}%
\end{pgfscope}%
\begin{pgfscope}%
\pgfsys@transformshift{1.477404in}{0.972561in}%
\pgfsys@useobject{currentmarker}{}%
\end{pgfscope}%
\begin{pgfscope}%
\pgfsys@transformshift{1.537019in}{0.984631in}%
\pgfsys@useobject{currentmarker}{}%
\end{pgfscope}%
\begin{pgfscope}%
\pgfsys@transformshift{1.596635in}{0.998295in}%
\pgfsys@useobject{currentmarker}{}%
\end{pgfscope}%
\begin{pgfscope}%
\pgfsys@transformshift{1.656250in}{1.013833in}%
\pgfsys@useobject{currentmarker}{}%
\end{pgfscope}%
\begin{pgfscope}%
\pgfsys@transformshift{1.715865in}{1.031590in}%
\pgfsys@useobject{currentmarker}{}%
\end{pgfscope}%
\begin{pgfscope}%
\pgfsys@transformshift{1.775481in}{1.051984in}%
\pgfsys@useobject{currentmarker}{}%
\end{pgfscope}%
\begin{pgfscope}%
\pgfsys@transformshift{1.835096in}{1.075531in}%
\pgfsys@useobject{currentmarker}{}%
\end{pgfscope}%
\begin{pgfscope}%
\pgfsys@transformshift{1.894712in}{1.102872in}%
\pgfsys@useobject{currentmarker}{}%
\end{pgfscope}%
\begin{pgfscope}%
\pgfsys@transformshift{1.954327in}{1.134797in}%
\pgfsys@useobject{currentmarker}{}%
\end{pgfscope}%
\begin{pgfscope}%
\pgfsys@transformshift{2.013942in}{1.172292in}%
\pgfsys@useobject{currentmarker}{}%
\end{pgfscope}%
\begin{pgfscope}%
\pgfsys@transformshift{2.073558in}{1.216581in}%
\pgfsys@useobject{currentmarker}{}%
\end{pgfscope}%
\begin{pgfscope}%
\pgfsys@transformshift{2.133173in}{1.269176in}%
\pgfsys@useobject{currentmarker}{}%
\end{pgfscope}%
\begin{pgfscope}%
\pgfsys@transformshift{2.192788in}{1.331932in}%
\pgfsys@useobject{currentmarker}{}%
\end{pgfscope}%
\begin{pgfscope}%
\pgfsys@transformshift{2.252404in}{1.407066in}%
\pgfsys@useobject{currentmarker}{}%
\end{pgfscope}%
\begin{pgfscope}%
\pgfsys@transformshift{2.312019in}{1.497129in}%
\pgfsys@useobject{currentmarker}{}%
\end{pgfscope}%
\begin{pgfscope}%
\pgfsys@transformshift{2.371635in}{1.604818in}%
\pgfsys@useobject{currentmarker}{}%
\end{pgfscope}%
\begin{pgfscope}%
\pgfsys@transformshift{2.431250in}{1.732500in}%
\pgfsys@useobject{currentmarker}{}%
\end{pgfscope}%
\begin{pgfscope}%
\pgfsys@transformshift{2.490865in}{1.881190in}%
\pgfsys@useobject{currentmarker}{}%
\end{pgfscope}%
\begin{pgfscope}%
\pgfsys@transformshift{2.550481in}{2.048713in}%
\pgfsys@useobject{currentmarker}{}%
\end{pgfscope}%
\begin{pgfscope}%
\pgfsys@transformshift{2.610096in}{2.226995in}%
\pgfsys@useobject{currentmarker}{}%
\end{pgfscope}%
\begin{pgfscope}%
\pgfsys@transformshift{2.669712in}{2.399304in}%
\pgfsys@useobject{currentmarker}{}%
\end{pgfscope}%
\begin{pgfscope}%
\pgfsys@transformshift{2.729327in}{2.539991in}%
\pgfsys@useobject{currentmarker}{}%
\end{pgfscope}%
\begin{pgfscope}%
\pgfsys@transformshift{2.788942in}{2.620265in}%
\pgfsys@useobject{currentmarker}{}%
\end{pgfscope}%
\begin{pgfscope}%
\pgfsys@transformshift{2.848558in}{2.620265in}%
\pgfsys@useobject{currentmarker}{}%
\end{pgfscope}%
\begin{pgfscope}%
\pgfsys@transformshift{2.908173in}{2.539991in}%
\pgfsys@useobject{currentmarker}{}%
\end{pgfscope}%
\begin{pgfscope}%
\pgfsys@transformshift{2.967788in}{2.399304in}%
\pgfsys@useobject{currentmarker}{}%
\end{pgfscope}%
\begin{pgfscope}%
\pgfsys@transformshift{3.027404in}{2.226995in}%
\pgfsys@useobject{currentmarker}{}%
\end{pgfscope}%
\begin{pgfscope}%
\pgfsys@transformshift{3.087019in}{2.048713in}%
\pgfsys@useobject{currentmarker}{}%
\end{pgfscope}%
\begin{pgfscope}%
\pgfsys@transformshift{3.146635in}{1.881190in}%
\pgfsys@useobject{currentmarker}{}%
\end{pgfscope}%
\begin{pgfscope}%
\pgfsys@transformshift{3.206250in}{1.732500in}%
\pgfsys@useobject{currentmarker}{}%
\end{pgfscope}%
\begin{pgfscope}%
\pgfsys@transformshift{3.265865in}{1.604818in}%
\pgfsys@useobject{currentmarker}{}%
\end{pgfscope}%
\begin{pgfscope}%
\pgfsys@transformshift{3.325481in}{1.497129in}%
\pgfsys@useobject{currentmarker}{}%
\end{pgfscope}%
\begin{pgfscope}%
\pgfsys@transformshift{3.385096in}{1.407066in}%
\pgfsys@useobject{currentmarker}{}%
\end{pgfscope}%
\begin{pgfscope}%
\pgfsys@transformshift{3.444712in}{1.331932in}%
\pgfsys@useobject{currentmarker}{}%
\end{pgfscope}%
\begin{pgfscope}%
\pgfsys@transformshift{3.504327in}{1.269176in}%
\pgfsys@useobject{currentmarker}{}%
\end{pgfscope}%
\begin{pgfscope}%
\pgfsys@transformshift{3.563942in}{1.216581in}%
\pgfsys@useobject{currentmarker}{}%
\end{pgfscope}%
\begin{pgfscope}%
\pgfsys@transformshift{3.623558in}{1.172292in}%
\pgfsys@useobject{currentmarker}{}%
\end{pgfscope}%
\begin{pgfscope}%
\pgfsys@transformshift{3.683173in}{1.134797in}%
\pgfsys@useobject{currentmarker}{}%
\end{pgfscope}%
\begin{pgfscope}%
\pgfsys@transformshift{3.742788in}{1.102872in}%
\pgfsys@useobject{currentmarker}{}%
\end{pgfscope}%
\begin{pgfscope}%
\pgfsys@transformshift{3.802404in}{1.075531in}%
\pgfsys@useobject{currentmarker}{}%
\end{pgfscope}%
\begin{pgfscope}%
\pgfsys@transformshift{3.862019in}{1.051984in}%
\pgfsys@useobject{currentmarker}{}%
\end{pgfscope}%
\begin{pgfscope}%
\pgfsys@transformshift{3.921635in}{1.031590in}%
\pgfsys@useobject{currentmarker}{}%
\end{pgfscope}%
\begin{pgfscope}%
\pgfsys@transformshift{3.981250in}{1.013833in}%
\pgfsys@useobject{currentmarker}{}%
\end{pgfscope}%
\begin{pgfscope}%
\pgfsys@transformshift{4.040865in}{0.998295in}%
\pgfsys@useobject{currentmarker}{}%
\end{pgfscope}%
\begin{pgfscope}%
\pgfsys@transformshift{4.100481in}{0.984631in}%
\pgfsys@useobject{currentmarker}{}%
\end{pgfscope}%
\begin{pgfscope}%
\pgfsys@transformshift{4.160096in}{0.972561in}%
\pgfsys@useobject{currentmarker}{}%
\end{pgfscope}%
\begin{pgfscope}%
\pgfsys@transformshift{4.219712in}{0.961852in}%
\pgfsys@useobject{currentmarker}{}%
\end{pgfscope}%
\begin{pgfscope}%
\pgfsys@transformshift{4.279327in}{0.952313in}%
\pgfsys@useobject{currentmarker}{}%
\end{pgfscope}%
\begin{pgfscope}%
\pgfsys@transformshift{4.338942in}{0.943783in}%
\pgfsys@useobject{currentmarker}{}%
\end{pgfscope}%
\begin{pgfscope}%
\pgfsys@transformshift{4.398558in}{0.936127in}%
\pgfsys@useobject{currentmarker}{}%
\end{pgfscope}%
\begin{pgfscope}%
\pgfsys@transformshift{4.458173in}{0.929231in}%
\pgfsys@useobject{currentmarker}{}%
\end{pgfscope}%
\begin{pgfscope}%
\pgfsys@transformshift{4.517788in}{0.923001in}%
\pgfsys@useobject{currentmarker}{}%
\end{pgfscope}%
\begin{pgfscope}%
\pgfsys@transformshift{4.577404in}{0.917355in}%
\pgfsys@useobject{currentmarker}{}%
\end{pgfscope}%
\begin{pgfscope}%
\pgfsys@transformshift{4.637019in}{0.912222in}%
\pgfsys@useobject{currentmarker}{}%
\end{pgfscope}%
\begin{pgfscope}%
\pgfsys@transformshift{4.696635in}{0.907544in}%
\pgfsys@useobject{currentmarker}{}%
\end{pgfscope}%
\end{pgfscope}%
\begin{pgfscope}%
\pgfpathrectangle{\pgfqpoint{0.687500in}{0.385000in}}{\pgfqpoint{4.262500in}{2.695000in}}%
\pgfusepath{clip}%
\pgfsetrectcap%
\pgfsetroundjoin%
\pgfsetlinewidth{1.505625pt}%
\definecolor{currentstroke}{rgb}{0.580392,0.403922,0.741176}%
\pgfsetstrokecolor{currentstroke}%
\pgfsetdash{}{0pt}%
\pgfpathmoveto{\pgfqpoint{0.881250in}{0.903269in}}%
\pgfpathlineto{\pgfqpoint{0.881250in}{0.375000in}}%
\pgfpathmoveto{\pgfqpoint{0.943544in}{0.375000in}}%
\pgfpathlineto{\pgfqpoint{0.943544in}{3.090000in}}%
\pgfpathmoveto{\pgfqpoint{1.004081in}{3.090000in}}%
\pgfpathlineto{\pgfqpoint{1.004082in}{0.375000in}}%
\pgfpathmoveto{\pgfqpoint{1.064025in}{0.375000in}}%
\pgfpathlineto{\pgfqpoint{1.064026in}{3.090000in}}%
\pgfpathmoveto{\pgfqpoint{1.123657in}{3.090000in}}%
\pgfpathlineto{\pgfqpoint{1.123674in}{0.375000in}}%
\pgfpathmoveto{\pgfqpoint{1.183107in}{0.375000in}}%
\pgfpathlineto{\pgfqpoint{1.183274in}{3.090000in}}%
\pgfpathmoveto{\pgfqpoint{1.241387in}{3.090000in}}%
\pgfpathlineto{\pgfqpoint{1.242790in}{0.375000in}}%
\pgfpathmoveto{\pgfqpoint{1.299677in}{0.375000in}}%
\pgfpathlineto{\pgfqpoint{1.309642in}{3.062355in}}%
\pgfpathlineto{\pgfqpoint{1.329114in}{2.991917in}}%
\pgfpathlineto{\pgfqpoint{1.348587in}{1.486528in}}%
\pgfpathlineto{\pgfqpoint{1.368059in}{0.651267in}}%
\pgfpathlineto{\pgfqpoint{1.387531in}{0.618245in}}%
\pgfpathlineto{\pgfqpoint{1.407004in}{0.857023in}}%
\pgfpathlineto{\pgfqpoint{1.426476in}{1.011800in}}%
\pgfpathlineto{\pgfqpoint{1.445948in}{1.030164in}}%
\pgfpathlineto{\pgfqpoint{1.465421in}{0.992401in}}%
\pgfpathlineto{\pgfqpoint{1.484893in}{0.965428in}}%
\pgfpathlineto{\pgfqpoint{1.504366in}{0.964552in}}%
\pgfpathlineto{\pgfqpoint{1.543310in}{0.987640in}}%
\pgfpathlineto{\pgfqpoint{1.562783in}{0.993414in}}%
\pgfpathlineto{\pgfqpoint{1.601727in}{0.999206in}}%
\pgfpathlineto{\pgfqpoint{1.640672in}{1.009181in}}%
\pgfpathlineto{\pgfqpoint{1.738034in}{1.038772in}}%
\pgfpathlineto{\pgfqpoint{1.796451in}{1.059900in}}%
\pgfpathlineto{\pgfqpoint{1.854868in}{1.084135in}}%
\pgfpathlineto{\pgfqpoint{1.893813in}{1.102427in}}%
\pgfpathlineto{\pgfqpoint{1.932758in}{1.122662in}}%
\pgfpathlineto{\pgfqpoint{1.971702in}{1.145099in}}%
\pgfpathlineto{\pgfqpoint{2.010647in}{1.170054in}}%
\pgfpathlineto{\pgfqpoint{2.049592in}{1.197870in}}%
\pgfpathlineto{\pgfqpoint{2.088536in}{1.228947in}}%
\pgfpathlineto{\pgfqpoint{2.127481in}{1.263748in}}%
\pgfpathlineto{\pgfqpoint{2.166426in}{1.302794in}}%
\pgfpathlineto{\pgfqpoint{2.205371in}{1.346677in}}%
\pgfpathlineto{\pgfqpoint{2.244315in}{1.396056in}}%
\pgfpathlineto{\pgfqpoint{2.283260in}{1.451647in}}%
\pgfpathlineto{\pgfqpoint{2.322205in}{1.514206in}}%
\pgfpathlineto{\pgfqpoint{2.361149in}{1.584486in}}%
\pgfpathlineto{\pgfqpoint{2.400094in}{1.663167in}}%
\pgfpathlineto{\pgfqpoint{2.439039in}{1.750738in}}%
\pgfpathlineto{\pgfqpoint{2.477984in}{1.847321in}}%
\pgfpathlineto{\pgfqpoint{2.516928in}{1.952417in}}%
\pgfpathlineto{\pgfqpoint{2.555873in}{2.064579in}}%
\pgfpathlineto{\pgfqpoint{2.653235in}{2.353617in}}%
\pgfpathlineto{\pgfqpoint{2.672707in}{2.407372in}}%
\pgfpathlineto{\pgfqpoint{2.692180in}{2.457628in}}%
\pgfpathlineto{\pgfqpoint{2.711652in}{2.503331in}}%
\pgfpathlineto{\pgfqpoint{2.731124in}{2.543430in}}%
\pgfpathlineto{\pgfqpoint{2.750597in}{2.576924in}}%
\pgfpathlineto{\pgfqpoint{2.770069in}{2.602918in}}%
\pgfpathlineto{\pgfqpoint{2.789541in}{2.620683in}}%
\pgfpathlineto{\pgfqpoint{2.809014in}{2.629700in}}%
\pgfpathlineto{\pgfqpoint{2.828486in}{2.629700in}}%
\pgfpathlineto{\pgfqpoint{2.847959in}{2.620683in}}%
\pgfpathlineto{\pgfqpoint{2.867431in}{2.602918in}}%
\pgfpathlineto{\pgfqpoint{2.886903in}{2.576924in}}%
\pgfpathlineto{\pgfqpoint{2.906376in}{2.543430in}}%
\pgfpathlineto{\pgfqpoint{2.925848in}{2.503331in}}%
\pgfpathlineto{\pgfqpoint{2.945320in}{2.457628in}}%
\pgfpathlineto{\pgfqpoint{2.964793in}{2.407372in}}%
\pgfpathlineto{\pgfqpoint{3.003737in}{2.297371in}}%
\pgfpathlineto{\pgfqpoint{3.120572in}{1.952417in}}%
\pgfpathlineto{\pgfqpoint{3.159516in}{1.847321in}}%
\pgfpathlineto{\pgfqpoint{3.198461in}{1.750738in}}%
\pgfpathlineto{\pgfqpoint{3.237406in}{1.663167in}}%
\pgfpathlineto{\pgfqpoint{3.276351in}{1.584486in}}%
\pgfpathlineto{\pgfqpoint{3.315295in}{1.514206in}}%
\pgfpathlineto{\pgfqpoint{3.354240in}{1.451647in}}%
\pgfpathlineto{\pgfqpoint{3.393185in}{1.396056in}}%
\pgfpathlineto{\pgfqpoint{3.432129in}{1.346677in}}%
\pgfpathlineto{\pgfqpoint{3.471074in}{1.302794in}}%
\pgfpathlineto{\pgfqpoint{3.510019in}{1.263748in}}%
\pgfpathlineto{\pgfqpoint{3.548964in}{1.228947in}}%
\pgfpathlineto{\pgfqpoint{3.587908in}{1.197870in}}%
\pgfpathlineto{\pgfqpoint{3.626853in}{1.170054in}}%
\pgfpathlineto{\pgfqpoint{3.665798in}{1.145099in}}%
\pgfpathlineto{\pgfqpoint{3.704742in}{1.122662in}}%
\pgfpathlineto{\pgfqpoint{3.743687in}{1.102427in}}%
\pgfpathlineto{\pgfqpoint{3.782632in}{1.084135in}}%
\pgfpathlineto{\pgfqpoint{3.841049in}{1.059900in}}%
\pgfpathlineto{\pgfqpoint{3.899466in}{1.038772in}}%
\pgfpathlineto{\pgfqpoint{3.957883in}{1.020730in}}%
\pgfpathlineto{\pgfqpoint{4.016300in}{1.003661in}}%
\pgfpathlineto{\pgfqpoint{4.035773in}{0.999206in}}%
\pgfpathlineto{\pgfqpoint{4.074717in}{0.993414in}}%
\pgfpathlineto{\pgfqpoint{4.094190in}{0.987640in}}%
\pgfpathlineto{\pgfqpoint{4.133134in}{0.964552in}}%
\pgfpathlineto{\pgfqpoint{4.152607in}{0.965428in}}%
\pgfpathlineto{\pgfqpoint{4.172079in}{0.992401in}}%
\pgfpathlineto{\pgfqpoint{4.191552in}{1.030164in}}%
\pgfpathlineto{\pgfqpoint{4.211024in}{1.011800in}}%
\pgfpathlineto{\pgfqpoint{4.230496in}{0.857023in}}%
\pgfpathlineto{\pgfqpoint{4.249969in}{0.618245in}}%
\pgfpathlineto{\pgfqpoint{4.269441in}{0.651267in}}%
\pgfpathlineto{\pgfqpoint{4.288913in}{1.486528in}}%
\pgfpathlineto{\pgfqpoint{4.308386in}{2.991917in}}%
\pgfpathlineto{\pgfqpoint{4.327858in}{3.062355in}}%
\pgfpathlineto{\pgfqpoint{4.337823in}{0.375000in}}%
\pgfpathmoveto{\pgfqpoint{4.394710in}{0.375000in}}%
\pgfpathlineto{\pgfqpoint{4.396113in}{3.090000in}}%
\pgfpathmoveto{\pgfqpoint{4.454226in}{3.090000in}}%
\pgfpathlineto{\pgfqpoint{4.454393in}{0.375000in}}%
\pgfpathmoveto{\pgfqpoint{4.513826in}{0.375000in}}%
\pgfpathlineto{\pgfqpoint{4.513843in}{3.090000in}}%
\pgfpathmoveto{\pgfqpoint{4.573474in}{3.090000in}}%
\pgfpathlineto{\pgfqpoint{4.573475in}{0.375000in}}%
\pgfpathmoveto{\pgfqpoint{4.633418in}{0.375000in}}%
\pgfpathlineto{\pgfqpoint{4.633419in}{3.090000in}}%
\pgfpathmoveto{\pgfqpoint{4.693956in}{3.090000in}}%
\pgfpathlineto{\pgfqpoint{4.693956in}{0.375000in}}%
\pgfpathlineto{\pgfqpoint{4.693956in}{0.375000in}}%
\pgfusepath{stroke}%
\end{pgfscope}%
\begin{pgfscope}%
\pgfsetrectcap%
\pgfsetmiterjoin%
\pgfsetlinewidth{0.803000pt}%
\definecolor{currentstroke}{rgb}{0.000000,0.000000,0.000000}%
\pgfsetstrokecolor{currentstroke}%
\pgfsetdash{}{0pt}%
\pgfpathmoveto{\pgfqpoint{0.687500in}{0.385000in}}%
\pgfpathlineto{\pgfqpoint{0.687500in}{3.080000in}}%
\pgfusepath{stroke}%
\end{pgfscope}%
\begin{pgfscope}%
\pgfsetrectcap%
\pgfsetmiterjoin%
\pgfsetlinewidth{0.803000pt}%
\definecolor{currentstroke}{rgb}{0.000000,0.000000,0.000000}%
\pgfsetstrokecolor{currentstroke}%
\pgfsetdash{}{0pt}%
\pgfpathmoveto{\pgfqpoint{4.950000in}{0.385000in}}%
\pgfpathlineto{\pgfqpoint{4.950000in}{3.080000in}}%
\pgfusepath{stroke}%
\end{pgfscope}%
\begin{pgfscope}%
\pgfsetrectcap%
\pgfsetmiterjoin%
\pgfsetlinewidth{0.803000pt}%
\definecolor{currentstroke}{rgb}{0.000000,0.000000,0.000000}%
\pgfsetstrokecolor{currentstroke}%
\pgfsetdash{}{0pt}%
\pgfpathmoveto{\pgfqpoint{0.687500in}{0.385000in}}%
\pgfpathlineto{\pgfqpoint{4.950000in}{0.385000in}}%
\pgfusepath{stroke}%
\end{pgfscope}%
\begin{pgfscope}%
\pgfsetrectcap%
\pgfsetmiterjoin%
\pgfsetlinewidth{0.803000pt}%
\definecolor{currentstroke}{rgb}{0.000000,0.000000,0.000000}%
\pgfsetstrokecolor{currentstroke}%
\pgfsetdash{}{0pt}%
\pgfpathmoveto{\pgfqpoint{0.687500in}{3.080000in}}%
\pgfpathlineto{\pgfqpoint{4.950000in}{3.080000in}}%
\pgfusepath{stroke}%
\end{pgfscope}%
\begin{pgfscope}%
\definecolor{textcolor}{rgb}{0.000000,0.000000,0.000000}%
\pgfsetstrokecolor{textcolor}%
\pgfsetfillcolor{textcolor}%
\pgftext[x=2.818750in,y=3.163333in,,base]{\color{textcolor}\rmfamily\fontsize{12.000000}{14.400000}\selectfont N=6, 64}%
\end{pgfscope}%
\begin{pgfscope}%
\pgfsetbuttcap%
\pgfsetmiterjoin%
\definecolor{currentfill}{rgb}{1.000000,1.000000,1.000000}%
\pgfsetfillcolor{currentfill}%
\pgfsetfillopacity{0.800000}%
\pgfsetlinewidth{1.003750pt}%
\definecolor{currentstroke}{rgb}{0.800000,0.800000,0.800000}%
\pgfsetstrokecolor{currentstroke}%
\pgfsetstrokeopacity{0.800000}%
\pgfsetdash{}{0pt}%
\pgfpathmoveto{\pgfqpoint{3.448142in}{2.192467in}}%
\pgfpathlineto{\pgfqpoint{4.852778in}{2.192467in}}%
\pgfpathquadraticcurveto{\pgfqpoint{4.880556in}{2.192467in}}{\pgfqpoint{4.880556in}{2.220245in}}%
\pgfpathlineto{\pgfqpoint{4.880556in}{2.982778in}}%
\pgfpathquadraticcurveto{\pgfqpoint{4.880556in}{3.010556in}}{\pgfqpoint{4.852778in}{3.010556in}}%
\pgfpathlineto{\pgfqpoint{3.448142in}{3.010556in}}%
\pgfpathquadraticcurveto{\pgfqpoint{3.420364in}{3.010556in}}{\pgfqpoint{3.420364in}{2.982778in}}%
\pgfpathlineto{\pgfqpoint{3.420364in}{2.220245in}}%
\pgfpathquadraticcurveto{\pgfqpoint{3.420364in}{2.192467in}}{\pgfqpoint{3.448142in}{2.192467in}}%
\pgfpathclose%
\pgfusepath{stroke,fill}%
\end{pgfscope}%
\begin{pgfscope}%
\pgfsetrectcap%
\pgfsetroundjoin%
\pgfsetlinewidth{1.505625pt}%
\definecolor{currentstroke}{rgb}{0.121569,0.466667,0.705882}%
\pgfsetstrokecolor{currentstroke}%
\pgfsetdash{}{0pt}%
\pgfpathmoveto{\pgfqpoint{3.475920in}{2.820146in}}%
\pgfpathlineto{\pgfqpoint{3.753698in}{2.820146in}}%
\pgfusepath{stroke}%
\end{pgfscope}%
\begin{pgfscope}%
\definecolor{textcolor}{rgb}{0.000000,0.000000,0.000000}%
\pgfsetstrokecolor{textcolor}%
\pgfsetfillcolor{textcolor}%
\pgftext[x=3.864809in,y=2.771534in,left,base]{\color{textcolor}\rmfamily\fontsize{10.000000}{12.000000}\selectfont \(\displaystyle y(x)=\)\(\displaystyle \frac{1}{1+25x^{2}}\)}%
\end{pgfscope}%
\begin{pgfscope}%
\pgfsetrectcap%
\pgfsetroundjoin%
\pgfsetlinewidth{1.505625pt}%
\definecolor{currentstroke}{rgb}{0.172549,0.627451,0.172549}%
\pgfsetstrokecolor{currentstroke}%
\pgfsetdash{}{0pt}%
\pgfpathmoveto{\pgfqpoint{3.475920in}{2.539689in}}%
\pgfpathlineto{\pgfqpoint{3.753698in}{2.539689in}}%
\pgfusepath{stroke}%
\end{pgfscope}%
\begin{pgfscope}%
\definecolor{textcolor}{rgb}{0.000000,0.000000,0.000000}%
\pgfsetstrokecolor{textcolor}%
\pgfsetfillcolor{textcolor}%
\pgftext[x=3.864809in,y=2.491078in,left,base]{\color{textcolor}\rmfamily\fontsize{10.000000}{12.000000}\selectfont W6(x)}%
\end{pgfscope}%
\begin{pgfscope}%
\pgfsetrectcap%
\pgfsetroundjoin%
\pgfsetlinewidth{1.505625pt}%
\definecolor{currentstroke}{rgb}{0.580392,0.403922,0.741176}%
\pgfsetstrokecolor{currentstroke}%
\pgfsetdash{}{0pt}%
\pgfpathmoveto{\pgfqpoint{3.475920in}{2.331356in}}%
\pgfpathlineto{\pgfqpoint{3.753698in}{2.331356in}}%
\pgfusepath{stroke}%
\end{pgfscope}%
\begin{pgfscope}%
\definecolor{textcolor}{rgb}{0.000000,0.000000,0.000000}%
\pgfsetstrokecolor{textcolor}%
\pgfsetfillcolor{textcolor}%
\pgftext[x=3.864809in,y=2.282745in,left,base]{\color{textcolor}\rmfamily\fontsize{10.000000}{12.000000}\selectfont W64(x)}%
\end{pgfscope}%
\end{pgfpicture}%
\makeatother%
\endgroup%
        
    \end{center}
    \caption{Węzły jednorodne, funkcja \(y\), \(N=6,64\)}
\end{figure}

\begin{figure}[h]
    \begin{center}
        %% Creator: Matplotlib, PGF backend
%%
%% To include the figure in your LaTeX document, write
%%   \input{<filename>.pgf}
%%
%% Make sure the required packages are loaded in your preamble
%%   \usepackage{pgf}
%%
%% Figures using additional raster images can only be included by \input if
%% they are in the same directory as the main LaTeX file. For loading figures
%% from other directories you can use the `import` package
%%   \usepackage{import}
%% and then include the figures with
%%   \import{<path to file>}{<filename>.pgf}
%%
%% Matplotlib used the following preamble
%%
\begingroup%
\makeatletter%
\begin{pgfpicture}%
\pgfpathrectangle{\pgfpointorigin}{\pgfqpoint{5.500000in}{3.500000in}}%
\pgfusepath{use as bounding box, clip}%
\begin{pgfscope}%
\pgfsetbuttcap%
\pgfsetmiterjoin%
\definecolor{currentfill}{rgb}{1.000000,1.000000,1.000000}%
\pgfsetfillcolor{currentfill}%
\pgfsetlinewidth{0.000000pt}%
\definecolor{currentstroke}{rgb}{1.000000,1.000000,1.000000}%
\pgfsetstrokecolor{currentstroke}%
\pgfsetdash{}{0pt}%
\pgfpathmoveto{\pgfqpoint{0.000000in}{0.000000in}}%
\pgfpathlineto{\pgfqpoint{5.500000in}{0.000000in}}%
\pgfpathlineto{\pgfqpoint{5.500000in}{3.500000in}}%
\pgfpathlineto{\pgfqpoint{0.000000in}{3.500000in}}%
\pgfpathclose%
\pgfusepath{fill}%
\end{pgfscope}%
\begin{pgfscope}%
\pgfsetbuttcap%
\pgfsetmiterjoin%
\definecolor{currentfill}{rgb}{1.000000,1.000000,1.000000}%
\pgfsetfillcolor{currentfill}%
\pgfsetlinewidth{0.000000pt}%
\definecolor{currentstroke}{rgb}{0.000000,0.000000,0.000000}%
\pgfsetstrokecolor{currentstroke}%
\pgfsetstrokeopacity{0.000000}%
\pgfsetdash{}{0pt}%
\pgfpathmoveto{\pgfqpoint{0.687500in}{0.385000in}}%
\pgfpathlineto{\pgfqpoint{4.950000in}{0.385000in}}%
\pgfpathlineto{\pgfqpoint{4.950000in}{3.080000in}}%
\pgfpathlineto{\pgfqpoint{0.687500in}{3.080000in}}%
\pgfpathclose%
\pgfusepath{fill}%
\end{pgfscope}%
\begin{pgfscope}%
\pgfsetbuttcap%
\pgfsetroundjoin%
\definecolor{currentfill}{rgb}{0.000000,0.000000,0.000000}%
\pgfsetfillcolor{currentfill}%
\pgfsetlinewidth{0.803000pt}%
\definecolor{currentstroke}{rgb}{0.000000,0.000000,0.000000}%
\pgfsetstrokecolor{currentstroke}%
\pgfsetdash{}{0pt}%
\pgfsys@defobject{currentmarker}{\pgfqpoint{0.000000in}{-0.048611in}}{\pgfqpoint{0.000000in}{0.000000in}}{%
\pgfpathmoveto{\pgfqpoint{0.000000in}{0.000000in}}%
\pgfpathlineto{\pgfqpoint{0.000000in}{-0.048611in}}%
\pgfusepath{stroke,fill}%
}%
\begin{pgfscope}%
\pgfsys@transformshift{0.881250in}{0.385000in}%
\pgfsys@useobject{currentmarker}{}%
\end{pgfscope}%
\end{pgfscope}%
\begin{pgfscope}%
\definecolor{textcolor}{rgb}{0.000000,0.000000,0.000000}%
\pgfsetstrokecolor{textcolor}%
\pgfsetfillcolor{textcolor}%
\pgftext[x=0.881250in,y=0.287778in,,top]{\color{textcolor}\rmfamily\fontsize{10.000000}{12.000000}\selectfont \(\displaystyle -1.00\)}%
\end{pgfscope}%
\begin{pgfscope}%
\pgfsetbuttcap%
\pgfsetroundjoin%
\definecolor{currentfill}{rgb}{0.000000,0.000000,0.000000}%
\pgfsetfillcolor{currentfill}%
\pgfsetlinewidth{0.803000pt}%
\definecolor{currentstroke}{rgb}{0.000000,0.000000,0.000000}%
\pgfsetstrokecolor{currentstroke}%
\pgfsetdash{}{0pt}%
\pgfsys@defobject{currentmarker}{\pgfqpoint{0.000000in}{-0.048611in}}{\pgfqpoint{0.000000in}{0.000000in}}{%
\pgfpathmoveto{\pgfqpoint{0.000000in}{0.000000in}}%
\pgfpathlineto{\pgfqpoint{0.000000in}{-0.048611in}}%
\pgfusepath{stroke,fill}%
}%
\begin{pgfscope}%
\pgfsys@transformshift{1.365625in}{0.385000in}%
\pgfsys@useobject{currentmarker}{}%
\end{pgfscope}%
\end{pgfscope}%
\begin{pgfscope}%
\definecolor{textcolor}{rgb}{0.000000,0.000000,0.000000}%
\pgfsetstrokecolor{textcolor}%
\pgfsetfillcolor{textcolor}%
\pgftext[x=1.365625in,y=0.287778in,,top]{\color{textcolor}\rmfamily\fontsize{10.000000}{12.000000}\selectfont \(\displaystyle -0.75\)}%
\end{pgfscope}%
\begin{pgfscope}%
\pgfsetbuttcap%
\pgfsetroundjoin%
\definecolor{currentfill}{rgb}{0.000000,0.000000,0.000000}%
\pgfsetfillcolor{currentfill}%
\pgfsetlinewidth{0.803000pt}%
\definecolor{currentstroke}{rgb}{0.000000,0.000000,0.000000}%
\pgfsetstrokecolor{currentstroke}%
\pgfsetdash{}{0pt}%
\pgfsys@defobject{currentmarker}{\pgfqpoint{0.000000in}{-0.048611in}}{\pgfqpoint{0.000000in}{0.000000in}}{%
\pgfpathmoveto{\pgfqpoint{0.000000in}{0.000000in}}%
\pgfpathlineto{\pgfqpoint{0.000000in}{-0.048611in}}%
\pgfusepath{stroke,fill}%
}%
\begin{pgfscope}%
\pgfsys@transformshift{1.850000in}{0.385000in}%
\pgfsys@useobject{currentmarker}{}%
\end{pgfscope}%
\end{pgfscope}%
\begin{pgfscope}%
\definecolor{textcolor}{rgb}{0.000000,0.000000,0.000000}%
\pgfsetstrokecolor{textcolor}%
\pgfsetfillcolor{textcolor}%
\pgftext[x=1.850000in,y=0.287778in,,top]{\color{textcolor}\rmfamily\fontsize{10.000000}{12.000000}\selectfont \(\displaystyle -0.50\)}%
\end{pgfscope}%
\begin{pgfscope}%
\pgfsetbuttcap%
\pgfsetroundjoin%
\definecolor{currentfill}{rgb}{0.000000,0.000000,0.000000}%
\pgfsetfillcolor{currentfill}%
\pgfsetlinewidth{0.803000pt}%
\definecolor{currentstroke}{rgb}{0.000000,0.000000,0.000000}%
\pgfsetstrokecolor{currentstroke}%
\pgfsetdash{}{0pt}%
\pgfsys@defobject{currentmarker}{\pgfqpoint{0.000000in}{-0.048611in}}{\pgfqpoint{0.000000in}{0.000000in}}{%
\pgfpathmoveto{\pgfqpoint{0.000000in}{0.000000in}}%
\pgfpathlineto{\pgfqpoint{0.000000in}{-0.048611in}}%
\pgfusepath{stroke,fill}%
}%
\begin{pgfscope}%
\pgfsys@transformshift{2.334375in}{0.385000in}%
\pgfsys@useobject{currentmarker}{}%
\end{pgfscope}%
\end{pgfscope}%
\begin{pgfscope}%
\definecolor{textcolor}{rgb}{0.000000,0.000000,0.000000}%
\pgfsetstrokecolor{textcolor}%
\pgfsetfillcolor{textcolor}%
\pgftext[x=2.334375in,y=0.287778in,,top]{\color{textcolor}\rmfamily\fontsize{10.000000}{12.000000}\selectfont \(\displaystyle -0.25\)}%
\end{pgfscope}%
\begin{pgfscope}%
\pgfsetbuttcap%
\pgfsetroundjoin%
\definecolor{currentfill}{rgb}{0.000000,0.000000,0.000000}%
\pgfsetfillcolor{currentfill}%
\pgfsetlinewidth{0.803000pt}%
\definecolor{currentstroke}{rgb}{0.000000,0.000000,0.000000}%
\pgfsetstrokecolor{currentstroke}%
\pgfsetdash{}{0pt}%
\pgfsys@defobject{currentmarker}{\pgfqpoint{0.000000in}{-0.048611in}}{\pgfqpoint{0.000000in}{0.000000in}}{%
\pgfpathmoveto{\pgfqpoint{0.000000in}{0.000000in}}%
\pgfpathlineto{\pgfqpoint{0.000000in}{-0.048611in}}%
\pgfusepath{stroke,fill}%
}%
\begin{pgfscope}%
\pgfsys@transformshift{2.818750in}{0.385000in}%
\pgfsys@useobject{currentmarker}{}%
\end{pgfscope}%
\end{pgfscope}%
\begin{pgfscope}%
\definecolor{textcolor}{rgb}{0.000000,0.000000,0.000000}%
\pgfsetstrokecolor{textcolor}%
\pgfsetfillcolor{textcolor}%
\pgftext[x=2.818750in,y=0.287778in,,top]{\color{textcolor}\rmfamily\fontsize{10.000000}{12.000000}\selectfont \(\displaystyle 0.00\)}%
\end{pgfscope}%
\begin{pgfscope}%
\pgfsetbuttcap%
\pgfsetroundjoin%
\definecolor{currentfill}{rgb}{0.000000,0.000000,0.000000}%
\pgfsetfillcolor{currentfill}%
\pgfsetlinewidth{0.803000pt}%
\definecolor{currentstroke}{rgb}{0.000000,0.000000,0.000000}%
\pgfsetstrokecolor{currentstroke}%
\pgfsetdash{}{0pt}%
\pgfsys@defobject{currentmarker}{\pgfqpoint{0.000000in}{-0.048611in}}{\pgfqpoint{0.000000in}{0.000000in}}{%
\pgfpathmoveto{\pgfqpoint{0.000000in}{0.000000in}}%
\pgfpathlineto{\pgfqpoint{0.000000in}{-0.048611in}}%
\pgfusepath{stroke,fill}%
}%
\begin{pgfscope}%
\pgfsys@transformshift{3.303125in}{0.385000in}%
\pgfsys@useobject{currentmarker}{}%
\end{pgfscope}%
\end{pgfscope}%
\begin{pgfscope}%
\definecolor{textcolor}{rgb}{0.000000,0.000000,0.000000}%
\pgfsetstrokecolor{textcolor}%
\pgfsetfillcolor{textcolor}%
\pgftext[x=3.303125in,y=0.287778in,,top]{\color{textcolor}\rmfamily\fontsize{10.000000}{12.000000}\selectfont \(\displaystyle 0.25\)}%
\end{pgfscope}%
\begin{pgfscope}%
\pgfsetbuttcap%
\pgfsetroundjoin%
\definecolor{currentfill}{rgb}{0.000000,0.000000,0.000000}%
\pgfsetfillcolor{currentfill}%
\pgfsetlinewidth{0.803000pt}%
\definecolor{currentstroke}{rgb}{0.000000,0.000000,0.000000}%
\pgfsetstrokecolor{currentstroke}%
\pgfsetdash{}{0pt}%
\pgfsys@defobject{currentmarker}{\pgfqpoint{0.000000in}{-0.048611in}}{\pgfqpoint{0.000000in}{0.000000in}}{%
\pgfpathmoveto{\pgfqpoint{0.000000in}{0.000000in}}%
\pgfpathlineto{\pgfqpoint{0.000000in}{-0.048611in}}%
\pgfusepath{stroke,fill}%
}%
\begin{pgfscope}%
\pgfsys@transformshift{3.787500in}{0.385000in}%
\pgfsys@useobject{currentmarker}{}%
\end{pgfscope}%
\end{pgfscope}%
\begin{pgfscope}%
\definecolor{textcolor}{rgb}{0.000000,0.000000,0.000000}%
\pgfsetstrokecolor{textcolor}%
\pgfsetfillcolor{textcolor}%
\pgftext[x=3.787500in,y=0.287778in,,top]{\color{textcolor}\rmfamily\fontsize{10.000000}{12.000000}\selectfont \(\displaystyle 0.50\)}%
\end{pgfscope}%
\begin{pgfscope}%
\pgfsetbuttcap%
\pgfsetroundjoin%
\definecolor{currentfill}{rgb}{0.000000,0.000000,0.000000}%
\pgfsetfillcolor{currentfill}%
\pgfsetlinewidth{0.803000pt}%
\definecolor{currentstroke}{rgb}{0.000000,0.000000,0.000000}%
\pgfsetstrokecolor{currentstroke}%
\pgfsetdash{}{0pt}%
\pgfsys@defobject{currentmarker}{\pgfqpoint{0.000000in}{-0.048611in}}{\pgfqpoint{0.000000in}{0.000000in}}{%
\pgfpathmoveto{\pgfqpoint{0.000000in}{0.000000in}}%
\pgfpathlineto{\pgfqpoint{0.000000in}{-0.048611in}}%
\pgfusepath{stroke,fill}%
}%
\begin{pgfscope}%
\pgfsys@transformshift{4.271875in}{0.385000in}%
\pgfsys@useobject{currentmarker}{}%
\end{pgfscope}%
\end{pgfscope}%
\begin{pgfscope}%
\definecolor{textcolor}{rgb}{0.000000,0.000000,0.000000}%
\pgfsetstrokecolor{textcolor}%
\pgfsetfillcolor{textcolor}%
\pgftext[x=4.271875in,y=0.287778in,,top]{\color{textcolor}\rmfamily\fontsize{10.000000}{12.000000}\selectfont \(\displaystyle 0.75\)}%
\end{pgfscope}%
\begin{pgfscope}%
\pgfsetbuttcap%
\pgfsetroundjoin%
\definecolor{currentfill}{rgb}{0.000000,0.000000,0.000000}%
\pgfsetfillcolor{currentfill}%
\pgfsetlinewidth{0.803000pt}%
\definecolor{currentstroke}{rgb}{0.000000,0.000000,0.000000}%
\pgfsetstrokecolor{currentstroke}%
\pgfsetdash{}{0pt}%
\pgfsys@defobject{currentmarker}{\pgfqpoint{0.000000in}{-0.048611in}}{\pgfqpoint{0.000000in}{0.000000in}}{%
\pgfpathmoveto{\pgfqpoint{0.000000in}{0.000000in}}%
\pgfpathlineto{\pgfqpoint{0.000000in}{-0.048611in}}%
\pgfusepath{stroke,fill}%
}%
\begin{pgfscope}%
\pgfsys@transformshift{4.756250in}{0.385000in}%
\pgfsys@useobject{currentmarker}{}%
\end{pgfscope}%
\end{pgfscope}%
\begin{pgfscope}%
\definecolor{textcolor}{rgb}{0.000000,0.000000,0.000000}%
\pgfsetstrokecolor{textcolor}%
\pgfsetfillcolor{textcolor}%
\pgftext[x=4.756250in,y=0.287778in,,top]{\color{textcolor}\rmfamily\fontsize{10.000000}{12.000000}\selectfont \(\displaystyle 1.00\)}%
\end{pgfscope}%
\begin{pgfscope}%
\definecolor{textcolor}{rgb}{0.000000,0.000000,0.000000}%
\pgfsetstrokecolor{textcolor}%
\pgfsetfillcolor{textcolor}%
\pgftext[x=2.818750in,y=0.108766in,,top]{\color{textcolor}\rmfamily\fontsize{10.000000}{12.000000}\selectfont x}%
\end{pgfscope}%
\begin{pgfscope}%
\pgfsetbuttcap%
\pgfsetroundjoin%
\definecolor{currentfill}{rgb}{0.000000,0.000000,0.000000}%
\pgfsetfillcolor{currentfill}%
\pgfsetlinewidth{0.803000pt}%
\definecolor{currentstroke}{rgb}{0.000000,0.000000,0.000000}%
\pgfsetstrokecolor{currentstroke}%
\pgfsetdash{}{0pt}%
\pgfsys@defobject{currentmarker}{\pgfqpoint{-0.048611in}{0.000000in}}{\pgfqpoint{0.000000in}{0.000000in}}{%
\pgfpathmoveto{\pgfqpoint{0.000000in}{0.000000in}}%
\pgfpathlineto{\pgfqpoint{-0.048611in}{0.000000in}}%
\pgfusepath{stroke,fill}%
}%
\begin{pgfscope}%
\pgfsys@transformshift{0.687500in}{0.474833in}%
\pgfsys@useobject{currentmarker}{}%
\end{pgfscope}%
\end{pgfscope}%
\begin{pgfscope}%
\definecolor{textcolor}{rgb}{0.000000,0.000000,0.000000}%
\pgfsetstrokecolor{textcolor}%
\pgfsetfillcolor{textcolor}%
\pgftext[x=0.304783in,y=0.426608in,left,base]{\color{textcolor}\rmfamily\fontsize{10.000000}{12.000000}\selectfont \(\displaystyle -0.2\)}%
\end{pgfscope}%
\begin{pgfscope}%
\pgfsetbuttcap%
\pgfsetroundjoin%
\definecolor{currentfill}{rgb}{0.000000,0.000000,0.000000}%
\pgfsetfillcolor{currentfill}%
\pgfsetlinewidth{0.803000pt}%
\definecolor{currentstroke}{rgb}{0.000000,0.000000,0.000000}%
\pgfsetstrokecolor{currentstroke}%
\pgfsetdash{}{0pt}%
\pgfsys@defobject{currentmarker}{\pgfqpoint{-0.048611in}{0.000000in}}{\pgfqpoint{0.000000in}{0.000000in}}{%
\pgfpathmoveto{\pgfqpoint{0.000000in}{0.000000in}}%
\pgfpathlineto{\pgfqpoint{-0.048611in}{0.000000in}}%
\pgfusepath{stroke,fill}%
}%
\begin{pgfscope}%
\pgfsys@transformshift{0.687500in}{0.834167in}%
\pgfsys@useobject{currentmarker}{}%
\end{pgfscope}%
\end{pgfscope}%
\begin{pgfscope}%
\definecolor{textcolor}{rgb}{0.000000,0.000000,0.000000}%
\pgfsetstrokecolor{textcolor}%
\pgfsetfillcolor{textcolor}%
\pgftext[x=0.412808in,y=0.785941in,left,base]{\color{textcolor}\rmfamily\fontsize{10.000000}{12.000000}\selectfont \(\displaystyle 0.0\)}%
\end{pgfscope}%
\begin{pgfscope}%
\pgfsetbuttcap%
\pgfsetroundjoin%
\definecolor{currentfill}{rgb}{0.000000,0.000000,0.000000}%
\pgfsetfillcolor{currentfill}%
\pgfsetlinewidth{0.803000pt}%
\definecolor{currentstroke}{rgb}{0.000000,0.000000,0.000000}%
\pgfsetstrokecolor{currentstroke}%
\pgfsetdash{}{0pt}%
\pgfsys@defobject{currentmarker}{\pgfqpoint{-0.048611in}{0.000000in}}{\pgfqpoint{0.000000in}{0.000000in}}{%
\pgfpathmoveto{\pgfqpoint{0.000000in}{0.000000in}}%
\pgfpathlineto{\pgfqpoint{-0.048611in}{0.000000in}}%
\pgfusepath{stroke,fill}%
}%
\begin{pgfscope}%
\pgfsys@transformshift{0.687500in}{1.193500in}%
\pgfsys@useobject{currentmarker}{}%
\end{pgfscope}%
\end{pgfscope}%
\begin{pgfscope}%
\definecolor{textcolor}{rgb}{0.000000,0.000000,0.000000}%
\pgfsetstrokecolor{textcolor}%
\pgfsetfillcolor{textcolor}%
\pgftext[x=0.412808in,y=1.145275in,left,base]{\color{textcolor}\rmfamily\fontsize{10.000000}{12.000000}\selectfont \(\displaystyle 0.2\)}%
\end{pgfscope}%
\begin{pgfscope}%
\pgfsetbuttcap%
\pgfsetroundjoin%
\definecolor{currentfill}{rgb}{0.000000,0.000000,0.000000}%
\pgfsetfillcolor{currentfill}%
\pgfsetlinewidth{0.803000pt}%
\definecolor{currentstroke}{rgb}{0.000000,0.000000,0.000000}%
\pgfsetstrokecolor{currentstroke}%
\pgfsetdash{}{0pt}%
\pgfsys@defobject{currentmarker}{\pgfqpoint{-0.048611in}{0.000000in}}{\pgfqpoint{0.000000in}{0.000000in}}{%
\pgfpathmoveto{\pgfqpoint{0.000000in}{0.000000in}}%
\pgfpathlineto{\pgfqpoint{-0.048611in}{0.000000in}}%
\pgfusepath{stroke,fill}%
}%
\begin{pgfscope}%
\pgfsys@transformshift{0.687500in}{1.552833in}%
\pgfsys@useobject{currentmarker}{}%
\end{pgfscope}%
\end{pgfscope}%
\begin{pgfscope}%
\definecolor{textcolor}{rgb}{0.000000,0.000000,0.000000}%
\pgfsetstrokecolor{textcolor}%
\pgfsetfillcolor{textcolor}%
\pgftext[x=0.412808in,y=1.504608in,left,base]{\color{textcolor}\rmfamily\fontsize{10.000000}{12.000000}\selectfont \(\displaystyle 0.4\)}%
\end{pgfscope}%
\begin{pgfscope}%
\pgfsetbuttcap%
\pgfsetroundjoin%
\definecolor{currentfill}{rgb}{0.000000,0.000000,0.000000}%
\pgfsetfillcolor{currentfill}%
\pgfsetlinewidth{0.803000pt}%
\definecolor{currentstroke}{rgb}{0.000000,0.000000,0.000000}%
\pgfsetstrokecolor{currentstroke}%
\pgfsetdash{}{0pt}%
\pgfsys@defobject{currentmarker}{\pgfqpoint{-0.048611in}{0.000000in}}{\pgfqpoint{0.000000in}{0.000000in}}{%
\pgfpathmoveto{\pgfqpoint{0.000000in}{0.000000in}}%
\pgfpathlineto{\pgfqpoint{-0.048611in}{0.000000in}}%
\pgfusepath{stroke,fill}%
}%
\begin{pgfscope}%
\pgfsys@transformshift{0.687500in}{1.912167in}%
\pgfsys@useobject{currentmarker}{}%
\end{pgfscope}%
\end{pgfscope}%
\begin{pgfscope}%
\definecolor{textcolor}{rgb}{0.000000,0.000000,0.000000}%
\pgfsetstrokecolor{textcolor}%
\pgfsetfillcolor{textcolor}%
\pgftext[x=0.412808in,y=1.863941in,left,base]{\color{textcolor}\rmfamily\fontsize{10.000000}{12.000000}\selectfont \(\displaystyle 0.6\)}%
\end{pgfscope}%
\begin{pgfscope}%
\pgfsetbuttcap%
\pgfsetroundjoin%
\definecolor{currentfill}{rgb}{0.000000,0.000000,0.000000}%
\pgfsetfillcolor{currentfill}%
\pgfsetlinewidth{0.803000pt}%
\definecolor{currentstroke}{rgb}{0.000000,0.000000,0.000000}%
\pgfsetstrokecolor{currentstroke}%
\pgfsetdash{}{0pt}%
\pgfsys@defobject{currentmarker}{\pgfqpoint{-0.048611in}{0.000000in}}{\pgfqpoint{0.000000in}{0.000000in}}{%
\pgfpathmoveto{\pgfqpoint{0.000000in}{0.000000in}}%
\pgfpathlineto{\pgfqpoint{-0.048611in}{0.000000in}}%
\pgfusepath{stroke,fill}%
}%
\begin{pgfscope}%
\pgfsys@transformshift{0.687500in}{2.271500in}%
\pgfsys@useobject{currentmarker}{}%
\end{pgfscope}%
\end{pgfscope}%
\begin{pgfscope}%
\definecolor{textcolor}{rgb}{0.000000,0.000000,0.000000}%
\pgfsetstrokecolor{textcolor}%
\pgfsetfillcolor{textcolor}%
\pgftext[x=0.412808in,y=2.223275in,left,base]{\color{textcolor}\rmfamily\fontsize{10.000000}{12.000000}\selectfont \(\displaystyle 0.8\)}%
\end{pgfscope}%
\begin{pgfscope}%
\pgfsetbuttcap%
\pgfsetroundjoin%
\definecolor{currentfill}{rgb}{0.000000,0.000000,0.000000}%
\pgfsetfillcolor{currentfill}%
\pgfsetlinewidth{0.803000pt}%
\definecolor{currentstroke}{rgb}{0.000000,0.000000,0.000000}%
\pgfsetstrokecolor{currentstroke}%
\pgfsetdash{}{0pt}%
\pgfsys@defobject{currentmarker}{\pgfqpoint{-0.048611in}{0.000000in}}{\pgfqpoint{0.000000in}{0.000000in}}{%
\pgfpathmoveto{\pgfqpoint{0.000000in}{0.000000in}}%
\pgfpathlineto{\pgfqpoint{-0.048611in}{0.000000in}}%
\pgfusepath{stroke,fill}%
}%
\begin{pgfscope}%
\pgfsys@transformshift{0.687500in}{2.630833in}%
\pgfsys@useobject{currentmarker}{}%
\end{pgfscope}%
\end{pgfscope}%
\begin{pgfscope}%
\definecolor{textcolor}{rgb}{0.000000,0.000000,0.000000}%
\pgfsetstrokecolor{textcolor}%
\pgfsetfillcolor{textcolor}%
\pgftext[x=0.412808in,y=2.582608in,left,base]{\color{textcolor}\rmfamily\fontsize{10.000000}{12.000000}\selectfont \(\displaystyle 1.0\)}%
\end{pgfscope}%
\begin{pgfscope}%
\pgfsetbuttcap%
\pgfsetroundjoin%
\definecolor{currentfill}{rgb}{0.000000,0.000000,0.000000}%
\pgfsetfillcolor{currentfill}%
\pgfsetlinewidth{0.803000pt}%
\definecolor{currentstroke}{rgb}{0.000000,0.000000,0.000000}%
\pgfsetstrokecolor{currentstroke}%
\pgfsetdash{}{0pt}%
\pgfsys@defobject{currentmarker}{\pgfqpoint{-0.048611in}{0.000000in}}{\pgfqpoint{0.000000in}{0.000000in}}{%
\pgfpathmoveto{\pgfqpoint{0.000000in}{0.000000in}}%
\pgfpathlineto{\pgfqpoint{-0.048611in}{0.000000in}}%
\pgfusepath{stroke,fill}%
}%
\begin{pgfscope}%
\pgfsys@transformshift{0.687500in}{2.990167in}%
\pgfsys@useobject{currentmarker}{}%
\end{pgfscope}%
\end{pgfscope}%
\begin{pgfscope}%
\definecolor{textcolor}{rgb}{0.000000,0.000000,0.000000}%
\pgfsetstrokecolor{textcolor}%
\pgfsetfillcolor{textcolor}%
\pgftext[x=0.412808in,y=2.941941in,left,base]{\color{textcolor}\rmfamily\fontsize{10.000000}{12.000000}\selectfont \(\displaystyle 1.2\)}%
\end{pgfscope}%
\begin{pgfscope}%
\definecolor{textcolor}{rgb}{0.000000,0.000000,0.000000}%
\pgfsetstrokecolor{textcolor}%
\pgfsetfillcolor{textcolor}%
\pgftext[x=0.249228in,y=1.732500in,,bottom,rotate=90.000000]{\color{textcolor}\rmfamily\fontsize{10.000000}{12.000000}\selectfont y}%
\end{pgfscope}%
\begin{pgfscope}%
\pgfpathrectangle{\pgfqpoint{0.687500in}{0.385000in}}{\pgfqpoint{4.262500in}{2.695000in}}%
\pgfusepath{clip}%
\pgfsetrectcap%
\pgfsetroundjoin%
\pgfsetlinewidth{1.505625pt}%
\definecolor{currentstroke}{rgb}{0.121569,0.466667,0.705882}%
\pgfsetstrokecolor{currentstroke}%
\pgfsetdash{}{0pt}%
\pgfpathmoveto{\pgfqpoint{0.881250in}{0.903269in}}%
\pgfpathlineto{\pgfqpoint{1.017557in}{0.913644in}}%
\pgfpathlineto{\pgfqpoint{1.134391in}{0.924478in}}%
\pgfpathlineto{\pgfqpoint{1.251225in}{0.937638in}}%
\pgfpathlineto{\pgfqpoint{1.348587in}{0.950877in}}%
\pgfpathlineto{\pgfqpoint{1.426476in}{0.963336in}}%
\pgfpathlineto{\pgfqpoint{1.504366in}{0.977838in}}%
\pgfpathlineto{\pgfqpoint{1.582255in}{0.994839in}}%
\pgfpathlineto{\pgfqpoint{1.640672in}{1.009574in}}%
\pgfpathlineto{\pgfqpoint{1.699089in}{1.026346in}}%
\pgfpathlineto{\pgfqpoint{1.757506in}{1.045528in}}%
\pgfpathlineto{\pgfqpoint{1.815923in}{1.067578in}}%
\pgfpathlineto{\pgfqpoint{1.854868in}{1.084143in}}%
\pgfpathlineto{\pgfqpoint{1.893813in}{1.102428in}}%
\pgfpathlineto{\pgfqpoint{1.932758in}{1.122659in}}%
\pgfpathlineto{\pgfqpoint{1.971702in}{1.145101in}}%
\pgfpathlineto{\pgfqpoint{2.010647in}{1.170055in}}%
\pgfpathlineto{\pgfqpoint{2.049592in}{1.197870in}}%
\pgfpathlineto{\pgfqpoint{2.088536in}{1.228948in}}%
\pgfpathlineto{\pgfqpoint{2.127481in}{1.263748in}}%
\pgfpathlineto{\pgfqpoint{2.166426in}{1.302794in}}%
\pgfpathlineto{\pgfqpoint{2.205371in}{1.346677in}}%
\pgfpathlineto{\pgfqpoint{2.244315in}{1.396056in}}%
\pgfpathlineto{\pgfqpoint{2.283260in}{1.451647in}}%
\pgfpathlineto{\pgfqpoint{2.322205in}{1.514206in}}%
\pgfpathlineto{\pgfqpoint{2.361149in}{1.584486in}}%
\pgfpathlineto{\pgfqpoint{2.400094in}{1.663167in}}%
\pgfpathlineto{\pgfqpoint{2.439039in}{1.750738in}}%
\pgfpathlineto{\pgfqpoint{2.477984in}{1.847321in}}%
\pgfpathlineto{\pgfqpoint{2.516928in}{1.952417in}}%
\pgfpathlineto{\pgfqpoint{2.555873in}{2.064579in}}%
\pgfpathlineto{\pgfqpoint{2.653235in}{2.353617in}}%
\pgfpathlineto{\pgfqpoint{2.672707in}{2.407372in}}%
\pgfpathlineto{\pgfqpoint{2.692180in}{2.457628in}}%
\pgfpathlineto{\pgfqpoint{2.711652in}{2.503331in}}%
\pgfpathlineto{\pgfqpoint{2.731124in}{2.543430in}}%
\pgfpathlineto{\pgfqpoint{2.750597in}{2.576924in}}%
\pgfpathlineto{\pgfqpoint{2.770069in}{2.602918in}}%
\pgfpathlineto{\pgfqpoint{2.789541in}{2.620683in}}%
\pgfpathlineto{\pgfqpoint{2.809014in}{2.629700in}}%
\pgfpathlineto{\pgfqpoint{2.828486in}{2.629700in}}%
\pgfpathlineto{\pgfqpoint{2.847959in}{2.620683in}}%
\pgfpathlineto{\pgfqpoint{2.867431in}{2.602918in}}%
\pgfpathlineto{\pgfqpoint{2.886903in}{2.576924in}}%
\pgfpathlineto{\pgfqpoint{2.906376in}{2.543430in}}%
\pgfpathlineto{\pgfqpoint{2.925848in}{2.503331in}}%
\pgfpathlineto{\pgfqpoint{2.945320in}{2.457628in}}%
\pgfpathlineto{\pgfqpoint{2.964793in}{2.407372in}}%
\pgfpathlineto{\pgfqpoint{3.003737in}{2.297371in}}%
\pgfpathlineto{\pgfqpoint{3.120572in}{1.952417in}}%
\pgfpathlineto{\pgfqpoint{3.159516in}{1.847321in}}%
\pgfpathlineto{\pgfqpoint{3.198461in}{1.750738in}}%
\pgfpathlineto{\pgfqpoint{3.237406in}{1.663167in}}%
\pgfpathlineto{\pgfqpoint{3.276351in}{1.584486in}}%
\pgfpathlineto{\pgfqpoint{3.315295in}{1.514206in}}%
\pgfpathlineto{\pgfqpoint{3.354240in}{1.451647in}}%
\pgfpathlineto{\pgfqpoint{3.393185in}{1.396056in}}%
\pgfpathlineto{\pgfqpoint{3.432129in}{1.346677in}}%
\pgfpathlineto{\pgfqpoint{3.471074in}{1.302794in}}%
\pgfpathlineto{\pgfqpoint{3.510019in}{1.263748in}}%
\pgfpathlineto{\pgfqpoint{3.548964in}{1.228948in}}%
\pgfpathlineto{\pgfqpoint{3.587908in}{1.197870in}}%
\pgfpathlineto{\pgfqpoint{3.626853in}{1.170055in}}%
\pgfpathlineto{\pgfqpoint{3.665798in}{1.145101in}}%
\pgfpathlineto{\pgfqpoint{3.704742in}{1.122659in}}%
\pgfpathlineto{\pgfqpoint{3.743687in}{1.102428in}}%
\pgfpathlineto{\pgfqpoint{3.782632in}{1.084143in}}%
\pgfpathlineto{\pgfqpoint{3.841049in}{1.059877in}}%
\pgfpathlineto{\pgfqpoint{3.899466in}{1.038840in}}%
\pgfpathlineto{\pgfqpoint{3.957883in}{1.020508in}}%
\pgfpathlineto{\pgfqpoint{4.016300in}{1.004452in}}%
\pgfpathlineto{\pgfqpoint{4.074717in}{0.990325in}}%
\pgfpathlineto{\pgfqpoint{4.152607in}{0.973998in}}%
\pgfpathlineto{\pgfqpoint{4.230496in}{0.960045in}}%
\pgfpathlineto{\pgfqpoint{4.327858in}{0.945299in}}%
\pgfpathlineto{\pgfqpoint{4.425220in}{0.932955in}}%
\pgfpathlineto{\pgfqpoint{4.542054in}{0.920637in}}%
\pgfpathlineto{\pgfqpoint{4.678361in}{0.908933in}}%
\pgfpathlineto{\pgfqpoint{4.756250in}{0.903269in}}%
\pgfpathlineto{\pgfqpoint{4.756250in}{0.903269in}}%
\pgfusepath{stroke}%
\end{pgfscope}%
\begin{pgfscope}%
\pgfpathrectangle{\pgfqpoint{0.687500in}{0.385000in}}{\pgfqpoint{4.262500in}{2.695000in}}%
\pgfusepath{clip}%
\pgfsetbuttcap%
\pgfsetroundjoin%
\definecolor{currentfill}{rgb}{1.000000,0.498039,0.054902}%
\pgfsetfillcolor{currentfill}%
\pgfsetlinewidth{1.003750pt}%
\definecolor{currentstroke}{rgb}{1.000000,0.498039,0.054902}%
\pgfsetstrokecolor{currentstroke}%
\pgfsetdash{}{0pt}%
\pgfsys@defobject{currentmarker}{\pgfqpoint{-0.020833in}{-0.020833in}}{\pgfqpoint{0.020833in}{0.020833in}}{%
\pgfpathmoveto{\pgfqpoint{0.000000in}{-0.020833in}}%
\pgfpathcurveto{\pgfqpoint{0.005525in}{-0.020833in}}{\pgfqpoint{0.010825in}{-0.018638in}}{\pgfqpoint{0.014731in}{-0.014731in}}%
\pgfpathcurveto{\pgfqpoint{0.018638in}{-0.010825in}}{\pgfqpoint{0.020833in}{-0.005525in}}{\pgfqpoint{0.020833in}{0.000000in}}%
\pgfpathcurveto{\pgfqpoint{0.020833in}{0.005525in}}{\pgfqpoint{0.018638in}{0.010825in}}{\pgfqpoint{0.014731in}{0.014731in}}%
\pgfpathcurveto{\pgfqpoint{0.010825in}{0.018638in}}{\pgfqpoint{0.005525in}{0.020833in}}{\pgfqpoint{0.000000in}{0.020833in}}%
\pgfpathcurveto{\pgfqpoint{-0.005525in}{0.020833in}}{\pgfqpoint{-0.010825in}{0.018638in}}{\pgfqpoint{-0.014731in}{0.014731in}}%
\pgfpathcurveto{\pgfqpoint{-0.018638in}{0.010825in}}{\pgfqpoint{-0.020833in}{0.005525in}}{\pgfqpoint{-0.020833in}{0.000000in}}%
\pgfpathcurveto{\pgfqpoint{-0.020833in}{-0.005525in}}{\pgfqpoint{-0.018638in}{-0.010825in}}{\pgfqpoint{-0.014731in}{-0.014731in}}%
\pgfpathcurveto{\pgfqpoint{-0.010825in}{-0.018638in}}{\pgfqpoint{-0.005525in}{-0.020833in}}{\pgfqpoint{0.000000in}{-0.020833in}}%
\pgfpathclose%
\pgfusepath{stroke,fill}%
}%
\begin{pgfscope}%
\pgfsys@transformshift{0.881250in}{0.903269in}%
\pgfsys@useobject{currentmarker}{}%
\end{pgfscope}%
\begin{pgfscope}%
\pgfsys@transformshift{1.527083in}{0.982515in}%
\pgfsys@useobject{currentmarker}{}%
\end{pgfscope}%
\begin{pgfscope}%
\pgfsys@transformshift{2.172917in}{1.309755in}%
\pgfsys@useobject{currentmarker}{}%
\end{pgfscope}%
\begin{pgfscope}%
\pgfsys@transformshift{2.818750in}{2.630833in}%
\pgfsys@useobject{currentmarker}{}%
\end{pgfscope}%
\begin{pgfscope}%
\pgfsys@transformshift{3.464583in}{1.309755in}%
\pgfsys@useobject{currentmarker}{}%
\end{pgfscope}%
\begin{pgfscope}%
\pgfsys@transformshift{4.110417in}{0.982515in}%
\pgfsys@useobject{currentmarker}{}%
\end{pgfscope}%
\end{pgfscope}%
\begin{pgfscope}%
\pgfpathrectangle{\pgfqpoint{0.687500in}{0.385000in}}{\pgfqpoint{4.262500in}{2.695000in}}%
\pgfusepath{clip}%
\pgfsetrectcap%
\pgfsetroundjoin%
\pgfsetlinewidth{1.505625pt}%
\definecolor{currentstroke}{rgb}{0.172549,0.627451,0.172549}%
\pgfsetstrokecolor{currentstroke}%
\pgfsetdash{}{0pt}%
\pgfpathmoveto{\pgfqpoint{0.881250in}{0.903269in}}%
\pgfpathlineto{\pgfqpoint{0.900722in}{1.006379in}}%
\pgfpathlineto{\pgfqpoint{0.920195in}{1.097302in}}%
\pgfpathlineto{\pgfqpoint{0.939667in}{1.176765in}}%
\pgfpathlineto{\pgfqpoint{0.959139in}{1.245470in}}%
\pgfpathlineto{\pgfqpoint{0.978612in}{1.304100in}}%
\pgfpathlineto{\pgfqpoint{0.998084in}{1.353313in}}%
\pgfpathlineto{\pgfqpoint{1.017557in}{1.393748in}}%
\pgfpathlineto{\pgfqpoint{1.037029in}{1.426022in}}%
\pgfpathlineto{\pgfqpoint{1.056501in}{1.450731in}}%
\pgfpathlineto{\pgfqpoint{1.075974in}{1.468451in}}%
\pgfpathlineto{\pgfqpoint{1.095446in}{1.479736in}}%
\pgfpathlineto{\pgfqpoint{1.114918in}{1.485123in}}%
\pgfpathlineto{\pgfqpoint{1.134391in}{1.485127in}}%
\pgfpathlineto{\pgfqpoint{1.153863in}{1.480244in}}%
\pgfpathlineto{\pgfqpoint{1.173335in}{1.470952in}}%
\pgfpathlineto{\pgfqpoint{1.192808in}{1.457708in}}%
\pgfpathlineto{\pgfqpoint{1.212280in}{1.440953in}}%
\pgfpathlineto{\pgfqpoint{1.231753in}{1.421108in}}%
\pgfpathlineto{\pgfqpoint{1.251225in}{1.398577in}}%
\pgfpathlineto{\pgfqpoint{1.270697in}{1.373746in}}%
\pgfpathlineto{\pgfqpoint{1.309642in}{1.318640in}}%
\pgfpathlineto{\pgfqpoint{1.348587in}{1.258536in}}%
\pgfpathlineto{\pgfqpoint{1.484893in}{1.042375in}}%
\pgfpathlineto{\pgfqpoint{1.523838in}{0.986902in}}%
\pgfpathlineto{\pgfqpoint{1.562783in}{0.937064in}}%
\pgfpathlineto{\pgfqpoint{1.582255in}{0.914640in}}%
\pgfpathlineto{\pgfqpoint{1.601727in}{0.894058in}}%
\pgfpathlineto{\pgfqpoint{1.621200in}{0.875436in}}%
\pgfpathlineto{\pgfqpoint{1.640672in}{0.858878in}}%
\pgfpathlineto{\pgfqpoint{1.660144in}{0.844479in}}%
\pgfpathlineto{\pgfqpoint{1.679617in}{0.832321in}}%
\pgfpathlineto{\pgfqpoint{1.699089in}{0.822475in}}%
\pgfpathlineto{\pgfqpoint{1.718562in}{0.815001in}}%
\pgfpathlineto{\pgfqpoint{1.738034in}{0.809948in}}%
\pgfpathlineto{\pgfqpoint{1.757506in}{0.807357in}}%
\pgfpathlineto{\pgfqpoint{1.776979in}{0.807254in}}%
\pgfpathlineto{\pgfqpoint{1.796451in}{0.809659in}}%
\pgfpathlineto{\pgfqpoint{1.815923in}{0.814583in}}%
\pgfpathlineto{\pgfqpoint{1.835396in}{0.822024in}}%
\pgfpathlineto{\pgfqpoint{1.854868in}{0.831975in}}%
\pgfpathlineto{\pgfqpoint{1.874340in}{0.844418in}}%
\pgfpathlineto{\pgfqpoint{1.893813in}{0.859327in}}%
\pgfpathlineto{\pgfqpoint{1.913285in}{0.876669in}}%
\pgfpathlineto{\pgfqpoint{1.932758in}{0.896402in}}%
\pgfpathlineto{\pgfqpoint{1.952230in}{0.918477in}}%
\pgfpathlineto{\pgfqpoint{1.971702in}{0.942837in}}%
\pgfpathlineto{\pgfqpoint{1.991175in}{0.969420in}}%
\pgfpathlineto{\pgfqpoint{2.010647in}{0.998156in}}%
\pgfpathlineto{\pgfqpoint{2.049592in}{1.061776in}}%
\pgfpathlineto{\pgfqpoint{2.088536in}{1.133017in}}%
\pgfpathlineto{\pgfqpoint{2.127481in}{1.211112in}}%
\pgfpathlineto{\pgfqpoint{2.166426in}{1.295217in}}%
\pgfpathlineto{\pgfqpoint{2.205371in}{1.384416in}}%
\pgfpathlineto{\pgfqpoint{2.263788in}{1.525625in}}%
\pgfpathlineto{\pgfqpoint{2.341677in}{1.722258in}}%
\pgfpathlineto{\pgfqpoint{2.439039in}{1.969110in}}%
\pgfpathlineto{\pgfqpoint{2.497456in}{2.111017in}}%
\pgfpathlineto{\pgfqpoint{2.536401in}{2.200684in}}%
\pgfpathlineto{\pgfqpoint{2.575345in}{2.285111in}}%
\pgfpathlineto{\pgfqpoint{2.614290in}{2.363222in}}%
\pgfpathlineto{\pgfqpoint{2.653235in}{2.433982in}}%
\pgfpathlineto{\pgfqpoint{2.672707in}{2.466293in}}%
\pgfpathlineto{\pgfqpoint{2.692180in}{2.496399in}}%
\pgfpathlineto{\pgfqpoint{2.711652in}{2.524185in}}%
\pgfpathlineto{\pgfqpoint{2.731124in}{2.549542in}}%
\pgfpathlineto{\pgfqpoint{2.750597in}{2.572361in}}%
\pgfpathlineto{\pgfqpoint{2.770069in}{2.592542in}}%
\pgfpathlineto{\pgfqpoint{2.789541in}{2.609989in}}%
\pgfpathlineto{\pgfqpoint{2.809014in}{2.624610in}}%
\pgfpathlineto{\pgfqpoint{2.828486in}{2.636319in}}%
\pgfpathlineto{\pgfqpoint{2.847959in}{2.645036in}}%
\pgfpathlineto{\pgfqpoint{2.867431in}{2.650689in}}%
\pgfpathlineto{\pgfqpoint{2.886903in}{2.653209in}}%
\pgfpathlineto{\pgfqpoint{2.906376in}{2.652536in}}%
\pgfpathlineto{\pgfqpoint{2.925848in}{2.648617in}}%
\pgfpathlineto{\pgfqpoint{2.945320in}{2.641406in}}%
\pgfpathlineto{\pgfqpoint{2.964793in}{2.630864in}}%
\pgfpathlineto{\pgfqpoint{2.984265in}{2.616960in}}%
\pgfpathlineto{\pgfqpoint{3.003737in}{2.599672in}}%
\pgfpathlineto{\pgfqpoint{3.023210in}{2.578986in}}%
\pgfpathlineto{\pgfqpoint{3.042682in}{2.554897in}}%
\pgfpathlineto{\pgfqpoint{3.062155in}{2.527408in}}%
\pgfpathlineto{\pgfqpoint{3.081627in}{2.496534in}}%
\pgfpathlineto{\pgfqpoint{3.101099in}{2.462296in}}%
\pgfpathlineto{\pgfqpoint{3.120572in}{2.424727in}}%
\pgfpathlineto{\pgfqpoint{3.140044in}{2.383871in}}%
\pgfpathlineto{\pgfqpoint{3.159516in}{2.339781in}}%
\pgfpathlineto{\pgfqpoint{3.178989in}{2.292521in}}%
\pgfpathlineto{\pgfqpoint{3.217933in}{2.188805in}}%
\pgfpathlineto{\pgfqpoint{3.256878in}{2.073465in}}%
\pgfpathlineto{\pgfqpoint{3.295823in}{1.947432in}}%
\pgfpathlineto{\pgfqpoint{3.334768in}{1.811842in}}%
\pgfpathlineto{\pgfqpoint{3.373712in}{1.668037in}}%
\pgfpathlineto{\pgfqpoint{3.432129in}{1.440408in}}%
\pgfpathlineto{\pgfqpoint{3.529491in}{1.045413in}}%
\pgfpathlineto{\pgfqpoint{3.587908in}{0.811931in}}%
\pgfpathlineto{\pgfqpoint{3.626853in}{0.663507in}}%
\pgfpathlineto{\pgfqpoint{3.665798in}{0.524575in}}%
\pgfpathlineto{\pgfqpoint{3.704742in}{0.398641in}}%
\pgfpathlineto{\pgfqpoint{3.712830in}{0.375000in}}%
\pgfpathmoveto{\pgfqpoint{4.014505in}{0.375000in}}%
\pgfpathlineto{\pgfqpoint{4.016300in}{0.382183in}}%
\pgfpathlineto{\pgfqpoint{4.035773in}{0.474864in}}%
\pgfpathlineto{\pgfqpoint{4.055245in}{0.583156in}}%
\pgfpathlineto{\pgfqpoint{4.074717in}{0.707935in}}%
\pgfpathlineto{\pgfqpoint{4.094190in}{0.850099in}}%
\pgfpathlineto{\pgfqpoint{4.113662in}{1.010572in}}%
\pgfpathlineto{\pgfqpoint{4.133134in}{1.190306in}}%
\pgfpathlineto{\pgfqpoint{4.152607in}{1.390272in}}%
\pgfpathlineto{\pgfqpoint{4.172079in}{1.611473in}}%
\pgfpathlineto{\pgfqpoint{4.191552in}{1.854934in}}%
\pgfpathlineto{\pgfqpoint{4.211024in}{2.121707in}}%
\pgfpathlineto{\pgfqpoint{4.230496in}{2.412871in}}%
\pgfpathlineto{\pgfqpoint{4.249969in}{2.729532in}}%
\pgfpathlineto{\pgfqpoint{4.270342in}{3.090000in}}%
\pgfpathlineto{\pgfqpoint{4.270342in}{3.090000in}}%
\pgfusepath{stroke}%
\end{pgfscope}%
\begin{pgfscope}%
\pgfpathrectangle{\pgfqpoint{0.687500in}{0.385000in}}{\pgfqpoint{4.262500in}{2.695000in}}%
\pgfusepath{clip}%
\pgfsetbuttcap%
\pgfsetroundjoin%
\definecolor{currentfill}{rgb}{0.839216,0.152941,0.156863}%
\pgfsetfillcolor{currentfill}%
\pgfsetlinewidth{1.003750pt}%
\definecolor{currentstroke}{rgb}{0.839216,0.152941,0.156863}%
\pgfsetstrokecolor{currentstroke}%
\pgfsetdash{}{0pt}%
\pgfsys@defobject{currentmarker}{\pgfqpoint{-0.020833in}{-0.020833in}}{\pgfqpoint{0.020833in}{0.020833in}}{%
\pgfpathmoveto{\pgfqpoint{0.000000in}{-0.020833in}}%
\pgfpathcurveto{\pgfqpoint{0.005525in}{-0.020833in}}{\pgfqpoint{0.010825in}{-0.018638in}}{\pgfqpoint{0.014731in}{-0.014731in}}%
\pgfpathcurveto{\pgfqpoint{0.018638in}{-0.010825in}}{\pgfqpoint{0.020833in}{-0.005525in}}{\pgfqpoint{0.020833in}{0.000000in}}%
\pgfpathcurveto{\pgfqpoint{0.020833in}{0.005525in}}{\pgfqpoint{0.018638in}{0.010825in}}{\pgfqpoint{0.014731in}{0.014731in}}%
\pgfpathcurveto{\pgfqpoint{0.010825in}{0.018638in}}{\pgfqpoint{0.005525in}{0.020833in}}{\pgfqpoint{0.000000in}{0.020833in}}%
\pgfpathcurveto{\pgfqpoint{-0.005525in}{0.020833in}}{\pgfqpoint{-0.010825in}{0.018638in}}{\pgfqpoint{-0.014731in}{0.014731in}}%
\pgfpathcurveto{\pgfqpoint{-0.018638in}{0.010825in}}{\pgfqpoint{-0.020833in}{0.005525in}}{\pgfqpoint{-0.020833in}{0.000000in}}%
\pgfpathcurveto{\pgfqpoint{-0.020833in}{-0.005525in}}{\pgfqpoint{-0.018638in}{-0.010825in}}{\pgfqpoint{-0.014731in}{-0.014731in}}%
\pgfpathcurveto{\pgfqpoint{-0.010825in}{-0.018638in}}{\pgfqpoint{-0.005525in}{-0.020833in}}{\pgfqpoint{0.000000in}{-0.020833in}}%
\pgfpathclose%
\pgfusepath{stroke,fill}%
}%
\begin{pgfscope}%
\pgfsys@transformshift{0.881250in}{0.903269in}%
\pgfsys@useobject{currentmarker}{}%
\end{pgfscope}%
\begin{pgfscope}%
\pgfsys@transformshift{1.365625in}{0.953447in}%
\pgfsys@useobject{currentmarker}{}%
\end{pgfscope}%
\begin{pgfscope}%
\pgfsys@transformshift{1.850000in}{1.081983in}%
\pgfsys@useobject{currentmarker}{}%
\end{pgfscope}%
\begin{pgfscope}%
\pgfsys@transformshift{2.334375in}{1.535305in}%
\pgfsys@useobject{currentmarker}{}%
\end{pgfscope}%
\begin{pgfscope}%
\pgfsys@transformshift{2.818750in}{2.630833in}%
\pgfsys@useobject{currentmarker}{}%
\end{pgfscope}%
\begin{pgfscope}%
\pgfsys@transformshift{3.303125in}{1.535305in}%
\pgfsys@useobject{currentmarker}{}%
\end{pgfscope}%
\begin{pgfscope}%
\pgfsys@transformshift{3.787500in}{1.081983in}%
\pgfsys@useobject{currentmarker}{}%
\end{pgfscope}%
\begin{pgfscope}%
\pgfsys@transformshift{4.271875in}{0.953447in}%
\pgfsys@useobject{currentmarker}{}%
\end{pgfscope}%
\end{pgfscope}%
\begin{pgfscope}%
\pgfpathrectangle{\pgfqpoint{0.687500in}{0.385000in}}{\pgfqpoint{4.262500in}{2.695000in}}%
\pgfusepath{clip}%
\pgfsetrectcap%
\pgfsetroundjoin%
\pgfsetlinewidth{1.505625pt}%
\definecolor{currentstroke}{rgb}{0.580392,0.403922,0.741176}%
\pgfsetstrokecolor{currentstroke}%
\pgfsetdash{}{0pt}%
\pgfpathmoveto{\pgfqpoint{0.881250in}{0.903269in}}%
\pgfpathlineto{\pgfqpoint{0.900722in}{0.654686in}}%
\pgfpathlineto{\pgfqpoint{0.920195in}{0.453318in}}%
\pgfpathlineto{\pgfqpoint{0.929777in}{0.375000in}}%
\pgfpathmoveto{\pgfqpoint{1.198622in}{0.375000in}}%
\pgfpathlineto{\pgfqpoint{1.290170in}{0.715688in}}%
\pgfpathlineto{\pgfqpoint{1.329114in}{0.845858in}}%
\pgfpathlineto{\pgfqpoint{1.348587in}{0.905156in}}%
\pgfpathlineto{\pgfqpoint{1.368059in}{0.960060in}}%
\pgfpathlineto{\pgfqpoint{1.387531in}{1.010288in}}%
\pgfpathlineto{\pgfqpoint{1.407004in}{1.055643in}}%
\pgfpathlineto{\pgfqpoint{1.426476in}{1.096007in}}%
\pgfpathlineto{\pgfqpoint{1.445948in}{1.131331in}}%
\pgfpathlineto{\pgfqpoint{1.465421in}{1.161633in}}%
\pgfpathlineto{\pgfqpoint{1.484893in}{1.186989in}}%
\pgfpathlineto{\pgfqpoint{1.504366in}{1.207527in}}%
\pgfpathlineto{\pgfqpoint{1.523838in}{1.223421in}}%
\pgfpathlineto{\pgfqpoint{1.543310in}{1.234888in}}%
\pgfpathlineto{\pgfqpoint{1.562783in}{1.242177in}}%
\pgfpathlineto{\pgfqpoint{1.582255in}{1.245571in}}%
\pgfpathlineto{\pgfqpoint{1.601727in}{1.245378in}}%
\pgfpathlineto{\pgfqpoint{1.621200in}{1.241927in}}%
\pgfpathlineto{\pgfqpoint{1.640672in}{1.235563in}}%
\pgfpathlineto{\pgfqpoint{1.660144in}{1.226646in}}%
\pgfpathlineto{\pgfqpoint{1.679617in}{1.215545in}}%
\pgfpathlineto{\pgfqpoint{1.699089in}{1.202633in}}%
\pgfpathlineto{\pgfqpoint{1.738034in}{1.172886in}}%
\pgfpathlineto{\pgfqpoint{1.835396in}{1.092770in}}%
\pgfpathlineto{\pgfqpoint{1.874340in}{1.065643in}}%
\pgfpathlineto{\pgfqpoint{1.893813in}{1.054367in}}%
\pgfpathlineto{\pgfqpoint{1.913285in}{1.044974in}}%
\pgfpathlineto{\pgfqpoint{1.932758in}{1.037706in}}%
\pgfpathlineto{\pgfqpoint{1.952230in}{1.032783in}}%
\pgfpathlineto{\pgfqpoint{1.971702in}{1.030403in}}%
\pgfpathlineto{\pgfqpoint{1.991175in}{1.030740in}}%
\pgfpathlineto{\pgfqpoint{2.010647in}{1.033945in}}%
\pgfpathlineto{\pgfqpoint{2.030119in}{1.040144in}}%
\pgfpathlineto{\pgfqpoint{2.049592in}{1.049437in}}%
\pgfpathlineto{\pgfqpoint{2.069064in}{1.061900in}}%
\pgfpathlineto{\pgfqpoint{2.088536in}{1.077586in}}%
\pgfpathlineto{\pgfqpoint{2.108009in}{1.096520in}}%
\pgfpathlineto{\pgfqpoint{2.127481in}{1.118705in}}%
\pgfpathlineto{\pgfqpoint{2.146954in}{1.144119in}}%
\pgfpathlineto{\pgfqpoint{2.166426in}{1.172715in}}%
\pgfpathlineto{\pgfqpoint{2.185898in}{1.204424in}}%
\pgfpathlineto{\pgfqpoint{2.205371in}{1.239156in}}%
\pgfpathlineto{\pgfqpoint{2.224843in}{1.276796in}}%
\pgfpathlineto{\pgfqpoint{2.244315in}{1.317210in}}%
\pgfpathlineto{\pgfqpoint{2.283260in}{1.405726in}}%
\pgfpathlineto{\pgfqpoint{2.322205in}{1.503248in}}%
\pgfpathlineto{\pgfqpoint{2.361149in}{1.608053in}}%
\pgfpathlineto{\pgfqpoint{2.419567in}{1.774595in}}%
\pgfpathlineto{\pgfqpoint{2.536401in}{2.113686in}}%
\pgfpathlineto{\pgfqpoint{2.575345in}{2.219807in}}%
\pgfpathlineto{\pgfqpoint{2.614290in}{2.318495in}}%
\pgfpathlineto{\pgfqpoint{2.653235in}{2.407460in}}%
\pgfpathlineto{\pgfqpoint{2.672707in}{2.447616in}}%
\pgfpathlineto{\pgfqpoint{2.692180in}{2.484545in}}%
\pgfpathlineto{\pgfqpoint{2.711652in}{2.518009in}}%
\pgfpathlineto{\pgfqpoint{2.731124in}{2.547780in}}%
\pgfpathlineto{\pgfqpoint{2.750597in}{2.573648in}}%
\pgfpathlineto{\pgfqpoint{2.770069in}{2.595420in}}%
\pgfpathlineto{\pgfqpoint{2.789541in}{2.612922in}}%
\pgfpathlineto{\pgfqpoint{2.809014in}{2.625998in}}%
\pgfpathlineto{\pgfqpoint{2.828486in}{2.634514in}}%
\pgfpathlineto{\pgfqpoint{2.847959in}{2.638357in}}%
\pgfpathlineto{\pgfqpoint{2.867431in}{2.637440in}}%
\pgfpathlineto{\pgfqpoint{2.886903in}{2.631694in}}%
\pgfpathlineto{\pgfqpoint{2.906376in}{2.621082in}}%
\pgfpathlineto{\pgfqpoint{2.925848in}{2.605586in}}%
\pgfpathlineto{\pgfqpoint{2.945320in}{2.585220in}}%
\pgfpathlineto{\pgfqpoint{2.964793in}{2.560019in}}%
\pgfpathlineto{\pgfqpoint{2.984265in}{2.530052in}}%
\pgfpathlineto{\pgfqpoint{3.003737in}{2.495410in}}%
\pgfpathlineto{\pgfqpoint{3.023210in}{2.456216in}}%
\pgfpathlineto{\pgfqpoint{3.042682in}{2.412618in}}%
\pgfpathlineto{\pgfqpoint{3.062155in}{2.364796in}}%
\pgfpathlineto{\pgfqpoint{3.081627in}{2.312955in}}%
\pgfpathlineto{\pgfqpoint{3.101099in}{2.257330in}}%
\pgfpathlineto{\pgfqpoint{3.140044in}{2.135799in}}%
\pgfpathlineto{\pgfqpoint{3.178989in}{2.002622in}}%
\pgfpathlineto{\pgfqpoint{3.217933in}{1.860623in}}%
\pgfpathlineto{\pgfqpoint{3.295823in}{1.563301in}}%
\pgfpathlineto{\pgfqpoint{3.354240in}{1.343235in}}%
\pgfpathlineto{\pgfqpoint{3.393185in}{1.205433in}}%
\pgfpathlineto{\pgfqpoint{3.432129in}{1.079627in}}%
\pgfpathlineto{\pgfqpoint{3.451602in}{1.022527in}}%
\pgfpathlineto{\pgfqpoint{3.471074in}{0.969982in}}%
\pgfpathlineto{\pgfqpoint{3.490546in}{0.922490in}}%
\pgfpathlineto{\pgfqpoint{3.510019in}{0.880535in}}%
\pgfpathlineto{\pgfqpoint{3.529491in}{0.844580in}}%
\pgfpathlineto{\pgfqpoint{3.548964in}{0.815063in}}%
\pgfpathlineto{\pgfqpoint{3.568436in}{0.792392in}}%
\pgfpathlineto{\pgfqpoint{3.587908in}{0.776939in}}%
\pgfpathlineto{\pgfqpoint{3.607381in}{0.769036in}}%
\pgfpathlineto{\pgfqpoint{3.626853in}{0.768969in}}%
\pgfpathlineto{\pgfqpoint{3.646325in}{0.776972in}}%
\pgfpathlineto{\pgfqpoint{3.665798in}{0.793218in}}%
\pgfpathlineto{\pgfqpoint{3.685270in}{0.817818in}}%
\pgfpathlineto{\pgfqpoint{3.704742in}{0.850809in}}%
\pgfpathlineto{\pgfqpoint{3.724215in}{0.892151in}}%
\pgfpathlineto{\pgfqpoint{3.743687in}{0.941715in}}%
\pgfpathlineto{\pgfqpoint{3.763160in}{0.999280in}}%
\pgfpathlineto{\pgfqpoint{3.782632in}{1.064521in}}%
\pgfpathlineto{\pgfqpoint{3.802104in}{1.137005in}}%
\pgfpathlineto{\pgfqpoint{3.821577in}{1.216178in}}%
\pgfpathlineto{\pgfqpoint{3.841049in}{1.301358in}}%
\pgfpathlineto{\pgfqpoint{3.879994in}{1.486315in}}%
\pgfpathlineto{\pgfqpoint{3.996828in}{2.067280in}}%
\pgfpathlineto{\pgfqpoint{4.016300in}{2.149830in}}%
\pgfpathlineto{\pgfqpoint{4.035773in}{2.222306in}}%
\pgfpathlineto{\pgfqpoint{4.055245in}{2.281926in}}%
\pgfpathlineto{\pgfqpoint{4.074717in}{2.325648in}}%
\pgfpathlineto{\pgfqpoint{4.094190in}{2.350149in}}%
\pgfpathlineto{\pgfqpoint{4.113662in}{2.351819in}}%
\pgfpathlineto{\pgfqpoint{4.133134in}{2.326740in}}%
\pgfpathlineto{\pgfqpoint{4.152607in}{2.270677in}}%
\pgfpathlineto{\pgfqpoint{4.172079in}{2.179058in}}%
\pgfpathlineto{\pgfqpoint{4.191552in}{2.046960in}}%
\pgfpathlineto{\pgfqpoint{4.211024in}{1.869094in}}%
\pgfpathlineto{\pgfqpoint{4.230496in}{1.639787in}}%
\pgfpathlineto{\pgfqpoint{4.249969in}{1.352966in}}%
\pgfpathlineto{\pgfqpoint{4.269441in}{1.002141in}}%
\pgfpathlineto{\pgfqpoint{4.288913in}{0.580384in}}%
\pgfpathlineto{\pgfqpoint{4.296911in}{0.375000in}}%
\pgfpathlineto{\pgfqpoint{4.296911in}{0.375000in}}%
\pgfusepath{stroke}%
\end{pgfscope}%
\begin{pgfscope}%
\pgfpathrectangle{\pgfqpoint{0.687500in}{0.385000in}}{\pgfqpoint{4.262500in}{2.695000in}}%
\pgfusepath{clip}%
\pgfsetbuttcap%
\pgfsetroundjoin%
\definecolor{currentfill}{rgb}{0.549020,0.337255,0.294118}%
\pgfsetfillcolor{currentfill}%
\pgfsetlinewidth{1.003750pt}%
\definecolor{currentstroke}{rgb}{0.549020,0.337255,0.294118}%
\pgfsetstrokecolor{currentstroke}%
\pgfsetdash{}{0pt}%
\pgfsys@defobject{currentmarker}{\pgfqpoint{-0.020833in}{-0.020833in}}{\pgfqpoint{0.020833in}{0.020833in}}{%
\pgfpathmoveto{\pgfqpoint{0.000000in}{-0.020833in}}%
\pgfpathcurveto{\pgfqpoint{0.005525in}{-0.020833in}}{\pgfqpoint{0.010825in}{-0.018638in}}{\pgfqpoint{0.014731in}{-0.014731in}}%
\pgfpathcurveto{\pgfqpoint{0.018638in}{-0.010825in}}{\pgfqpoint{0.020833in}{-0.005525in}}{\pgfqpoint{0.020833in}{0.000000in}}%
\pgfpathcurveto{\pgfqpoint{0.020833in}{0.005525in}}{\pgfqpoint{0.018638in}{0.010825in}}{\pgfqpoint{0.014731in}{0.014731in}}%
\pgfpathcurveto{\pgfqpoint{0.010825in}{0.018638in}}{\pgfqpoint{0.005525in}{0.020833in}}{\pgfqpoint{0.000000in}{0.020833in}}%
\pgfpathcurveto{\pgfqpoint{-0.005525in}{0.020833in}}{\pgfqpoint{-0.010825in}{0.018638in}}{\pgfqpoint{-0.014731in}{0.014731in}}%
\pgfpathcurveto{\pgfqpoint{-0.018638in}{0.010825in}}{\pgfqpoint{-0.020833in}{0.005525in}}{\pgfqpoint{-0.020833in}{0.000000in}}%
\pgfpathcurveto{\pgfqpoint{-0.020833in}{-0.005525in}}{\pgfqpoint{-0.018638in}{-0.010825in}}{\pgfqpoint{-0.014731in}{-0.014731in}}%
\pgfpathcurveto{\pgfqpoint{-0.010825in}{-0.018638in}}{\pgfqpoint{-0.005525in}{-0.020833in}}{\pgfqpoint{0.000000in}{-0.020833in}}%
\pgfpathclose%
\pgfusepath{stroke,fill}%
}%
\begin{pgfscope}%
\pgfsys@transformshift{0.881250in}{0.903269in}%
\pgfsys@useobject{currentmarker}{}%
\end{pgfscope}%
\begin{pgfscope}%
\pgfsys@transformshift{1.123438in}{0.923373in}%
\pgfsys@useobject{currentmarker}{}%
\end{pgfscope}%
\begin{pgfscope}%
\pgfsys@transformshift{1.365625in}{0.953447in}%
\pgfsys@useobject{currentmarker}{}%
\end{pgfscope}%
\begin{pgfscope}%
\pgfsys@transformshift{1.607813in}{1.001056in}%
\pgfsys@useobject{currentmarker}{}%
\end{pgfscope}%
\begin{pgfscope}%
\pgfsys@transformshift{1.850000in}{1.081983in}%
\pgfsys@useobject{currentmarker}{}%
\end{pgfscope}%
\begin{pgfscope}%
\pgfsys@transformshift{2.092188in}{1.232044in}%
\pgfsys@useobject{currentmarker}{}%
\end{pgfscope}%
\begin{pgfscope}%
\pgfsys@transformshift{2.334375in}{1.535305in}%
\pgfsys@useobject{currentmarker}{}%
\end{pgfscope}%
\begin{pgfscope}%
\pgfsys@transformshift{2.576563in}{2.126152in}%
\pgfsys@useobject{currentmarker}{}%
\end{pgfscope}%
\begin{pgfscope}%
\pgfsys@transformshift{2.818750in}{2.630833in}%
\pgfsys@useobject{currentmarker}{}%
\end{pgfscope}%
\begin{pgfscope}%
\pgfsys@transformshift{3.060938in}{2.126152in}%
\pgfsys@useobject{currentmarker}{}%
\end{pgfscope}%
\begin{pgfscope}%
\pgfsys@transformshift{3.303125in}{1.535305in}%
\pgfsys@useobject{currentmarker}{}%
\end{pgfscope}%
\begin{pgfscope}%
\pgfsys@transformshift{3.545313in}{1.232044in}%
\pgfsys@useobject{currentmarker}{}%
\end{pgfscope}%
\begin{pgfscope}%
\pgfsys@transformshift{3.787500in}{1.081983in}%
\pgfsys@useobject{currentmarker}{}%
\end{pgfscope}%
\begin{pgfscope}%
\pgfsys@transformshift{4.029687in}{1.001056in}%
\pgfsys@useobject{currentmarker}{}%
\end{pgfscope}%
\begin{pgfscope}%
\pgfsys@transformshift{4.271875in}{0.953447in}%
\pgfsys@useobject{currentmarker}{}%
\end{pgfscope}%
\begin{pgfscope}%
\pgfsys@transformshift{4.514063in}{0.923373in}%
\pgfsys@useobject{currentmarker}{}%
\end{pgfscope}%
\end{pgfscope}%
\begin{pgfscope}%
\pgfpathrectangle{\pgfqpoint{0.687500in}{0.385000in}}{\pgfqpoint{4.262500in}{2.695000in}}%
\pgfusepath{clip}%
\pgfsetrectcap%
\pgfsetroundjoin%
\pgfsetlinewidth{1.505625pt}%
\definecolor{currentstroke}{rgb}{0.890196,0.466667,0.760784}%
\pgfsetstrokecolor{currentstroke}%
\pgfsetdash{}{0pt}%
\pgfpathmoveto{\pgfqpoint{0.881250in}{0.903269in}}%
\pgfpathlineto{\pgfqpoint{0.882653in}{0.375000in}}%
\pgfpathmoveto{\pgfqpoint{1.110618in}{0.375000in}}%
\pgfpathlineto{\pgfqpoint{1.114918in}{0.586255in}}%
\pgfpathlineto{\pgfqpoint{1.134391in}{1.288405in}}%
\pgfpathlineto{\pgfqpoint{1.153863in}{1.763228in}}%
\pgfpathlineto{\pgfqpoint{1.173335in}{2.044575in}}%
\pgfpathlineto{\pgfqpoint{1.192808in}{2.168753in}}%
\pgfpathlineto{\pgfqpoint{1.212280in}{2.171555in}}%
\pgfpathlineto{\pgfqpoint{1.231753in}{2.086256in}}%
\pgfpathlineto{\pgfqpoint{1.251225in}{1.942368in}}%
\pgfpathlineto{\pgfqpoint{1.270697in}{1.764988in}}%
\pgfpathlineto{\pgfqpoint{1.309642in}{1.387000in}}%
\pgfpathlineto{\pgfqpoint{1.329114in}{1.213945in}}%
\pgfpathlineto{\pgfqpoint{1.348587in}{1.063262in}}%
\pgfpathlineto{\pgfqpoint{1.368059in}{0.939545in}}%
\pgfpathlineto{\pgfqpoint{1.387531in}{0.844683in}}%
\pgfpathlineto{\pgfqpoint{1.407004in}{0.778416in}}%
\pgfpathlineto{\pgfqpoint{1.426476in}{0.738865in}}%
\pgfpathlineto{\pgfqpoint{1.445948in}{0.723010in}}%
\pgfpathlineto{\pgfqpoint{1.465421in}{0.727119in}}%
\pgfpathlineto{\pgfqpoint{1.484893in}{0.747109in}}%
\pgfpathlineto{\pgfqpoint{1.504366in}{0.778841in}}%
\pgfpathlineto{\pgfqpoint{1.523838in}{0.818354in}}%
\pgfpathlineto{\pgfqpoint{1.582255in}{0.949981in}}%
\pgfpathlineto{\pgfqpoint{1.601727in}{0.989633in}}%
\pgfpathlineto{\pgfqpoint{1.621200in}{1.024308in}}%
\pgfpathlineto{\pgfqpoint{1.640672in}{1.053117in}}%
\pgfpathlineto{\pgfqpoint{1.660144in}{1.075667in}}%
\pgfpathlineto{\pgfqpoint{1.679617in}{1.091993in}}%
\pgfpathlineto{\pgfqpoint{1.699089in}{1.102487in}}%
\pgfpathlineto{\pgfqpoint{1.718562in}{1.107816in}}%
\pgfpathlineto{\pgfqpoint{1.738034in}{1.108848in}}%
\pgfpathlineto{\pgfqpoint{1.757506in}{1.106575in}}%
\pgfpathlineto{\pgfqpoint{1.776979in}{1.102047in}}%
\pgfpathlineto{\pgfqpoint{1.835396in}{1.085049in}}%
\pgfpathlineto{\pgfqpoint{1.854868in}{1.081188in}}%
\pgfpathlineto{\pgfqpoint{1.874340in}{1.079378in}}%
\pgfpathlineto{\pgfqpoint{1.893813in}{1.080076in}}%
\pgfpathlineto{\pgfqpoint{1.913285in}{1.083581in}}%
\pgfpathlineto{\pgfqpoint{1.932758in}{1.090035in}}%
\pgfpathlineto{\pgfqpoint{1.952230in}{1.099440in}}%
\pgfpathlineto{\pgfqpoint{1.971702in}{1.111671in}}%
\pgfpathlineto{\pgfqpoint{1.991175in}{1.126504in}}%
\pgfpathlineto{\pgfqpoint{2.010647in}{1.143633in}}%
\pgfpathlineto{\pgfqpoint{2.030119in}{1.162704in}}%
\pgfpathlineto{\pgfqpoint{2.069064in}{1.205132in}}%
\pgfpathlineto{\pgfqpoint{2.127481in}{1.274160in}}%
\pgfpathlineto{\pgfqpoint{2.263788in}{1.437812in}}%
\pgfpathlineto{\pgfqpoint{2.302732in}{1.489018in}}%
\pgfpathlineto{\pgfqpoint{2.322205in}{1.516888in}}%
\pgfpathlineto{\pgfqpoint{2.341677in}{1.546769in}}%
\pgfpathlineto{\pgfqpoint{2.361149in}{1.579011in}}%
\pgfpathlineto{\pgfqpoint{2.380622in}{1.613928in}}%
\pgfpathlineto{\pgfqpoint{2.400094in}{1.651777in}}%
\pgfpathlineto{\pgfqpoint{2.419567in}{1.692740in}}%
\pgfpathlineto{\pgfqpoint{2.439039in}{1.736914in}}%
\pgfpathlineto{\pgfqpoint{2.458511in}{1.784299in}}%
\pgfpathlineto{\pgfqpoint{2.497456in}{1.888164in}}%
\pgfpathlineto{\pgfqpoint{2.536401in}{2.002153in}}%
\pgfpathlineto{\pgfqpoint{2.653235in}{2.358603in}}%
\pgfpathlineto{\pgfqpoint{2.672707in}{2.411904in}}%
\pgfpathlineto{\pgfqpoint{2.692180in}{2.461146in}}%
\pgfpathlineto{\pgfqpoint{2.711652in}{2.505483in}}%
\pgfpathlineto{\pgfqpoint{2.731124in}{2.544127in}}%
\pgfpathlineto{\pgfqpoint{2.750597in}{2.576363in}}%
\pgfpathlineto{\pgfqpoint{2.770069in}{2.601571in}}%
\pgfpathlineto{\pgfqpoint{2.789541in}{2.619245in}}%
\pgfpathlineto{\pgfqpoint{2.809014in}{2.629003in}}%
\pgfpathlineto{\pgfqpoint{2.828486in}{2.630606in}}%
\pgfpathlineto{\pgfqpoint{2.847959in}{2.623958in}}%
\pgfpathlineto{\pgfqpoint{2.867431in}{2.609120in}}%
\pgfpathlineto{\pgfqpoint{2.886903in}{2.586303in}}%
\pgfpathlineto{\pgfqpoint{2.906376in}{2.555869in}}%
\pgfpathlineto{\pgfqpoint{2.925848in}{2.518324in}}%
\pgfpathlineto{\pgfqpoint{2.945320in}{2.474303in}}%
\pgfpathlineto{\pgfqpoint{2.964793in}{2.424562in}}%
\pgfpathlineto{\pgfqpoint{2.984265in}{2.369953in}}%
\pgfpathlineto{\pgfqpoint{3.023210in}{2.249911in}}%
\pgfpathlineto{\pgfqpoint{3.120572in}{1.933160in}}%
\pgfpathlineto{\pgfqpoint{3.159516in}{1.818909in}}%
\pgfpathlineto{\pgfqpoint{3.178989in}{1.767202in}}%
\pgfpathlineto{\pgfqpoint{3.198461in}{1.719628in}}%
\pgfpathlineto{\pgfqpoint{3.217933in}{1.676375in}}%
\pgfpathlineto{\pgfqpoint{3.237406in}{1.637470in}}%
\pgfpathlineto{\pgfqpoint{3.256878in}{1.602779in}}%
\pgfpathlineto{\pgfqpoint{3.276351in}{1.572012in}}%
\pgfpathlineto{\pgfqpoint{3.295823in}{1.544740in}}%
\pgfpathlineto{\pgfqpoint{3.315295in}{1.520413in}}%
\pgfpathlineto{\pgfqpoint{3.354240in}{1.477941in}}%
\pgfpathlineto{\pgfqpoint{3.412657in}{1.418670in}}%
\pgfpathlineto{\pgfqpoint{3.432129in}{1.397277in}}%
\pgfpathlineto{\pgfqpoint{3.451602in}{1.374114in}}%
\pgfpathlineto{\pgfqpoint{3.471074in}{1.348815in}}%
\pgfpathlineto{\pgfqpoint{3.490546in}{1.321190in}}%
\pgfpathlineto{\pgfqpoint{3.510019in}{1.291257in}}%
\pgfpathlineto{\pgfqpoint{3.548964in}{1.225651in}}%
\pgfpathlineto{\pgfqpoint{3.607381in}{1.123244in}}%
\pgfpathlineto{\pgfqpoint{3.626853in}{1.092197in}}%
\pgfpathlineto{\pgfqpoint{3.646325in}{1.064840in}}%
\pgfpathlineto{\pgfqpoint{3.665798in}{1.042542in}}%
\pgfpathlineto{\pgfqpoint{3.685270in}{1.026622in}}%
\pgfpathlineto{\pgfqpoint{3.704742in}{1.018263in}}%
\pgfpathlineto{\pgfqpoint{3.724215in}{1.018412in}}%
\pgfpathlineto{\pgfqpoint{3.743687in}{1.027685in}}%
\pgfpathlineto{\pgfqpoint{3.763160in}{1.046264in}}%
\pgfpathlineto{\pgfqpoint{3.782632in}{1.073798in}}%
\pgfpathlineto{\pgfqpoint{3.802104in}{1.109324in}}%
\pgfpathlineto{\pgfqpoint{3.821577in}{1.151191in}}%
\pgfpathlineto{\pgfqpoint{3.879994in}{1.287363in}}%
\pgfpathlineto{\pgfqpoint{3.899466in}{1.323562in}}%
\pgfpathlineto{\pgfqpoint{3.918938in}{1.347305in}}%
\pgfpathlineto{\pgfqpoint{3.938411in}{1.353298in}}%
\pgfpathlineto{\pgfqpoint{3.957883in}{1.336209in}}%
\pgfpathlineto{\pgfqpoint{3.977356in}{1.291003in}}%
\pgfpathlineto{\pgfqpoint{3.996828in}{1.213370in}}%
\pgfpathlineto{\pgfqpoint{4.016300in}{1.100219in}}%
\pgfpathlineto{\pgfqpoint{4.035773in}{0.950247in}}%
\pgfpathlineto{\pgfqpoint{4.055245in}{0.764568in}}%
\pgfpathlineto{\pgfqpoint{4.074717in}{0.547380in}}%
\pgfpathlineto{\pgfqpoint{4.088661in}{0.375000in}}%
\pgfpathmoveto{\pgfqpoint{4.256748in}{0.375000in}}%
\pgfpathlineto{\pgfqpoint{4.269441in}{0.843242in}}%
\pgfpathlineto{\pgfqpoint{4.288913in}{1.861556in}}%
\pgfpathlineto{\pgfqpoint{4.306704in}{3.090000in}}%
\pgfpathmoveto{\pgfqpoint{4.509135in}{3.090000in}}%
\pgfpathlineto{\pgfqpoint{4.513752in}{0.375000in}}%
\pgfpathlineto{\pgfqpoint{4.513752in}{0.375000in}}%
\pgfusepath{stroke}%
\end{pgfscope}%
\begin{pgfscope}%
\pgfpathrectangle{\pgfqpoint{0.687500in}{0.385000in}}{\pgfqpoint{4.262500in}{2.695000in}}%
\pgfusepath{clip}%
\pgfsetbuttcap%
\pgfsetroundjoin%
\definecolor{currentfill}{rgb}{0.498039,0.498039,0.498039}%
\pgfsetfillcolor{currentfill}%
\pgfsetlinewidth{1.003750pt}%
\definecolor{currentstroke}{rgb}{0.498039,0.498039,0.498039}%
\pgfsetstrokecolor{currentstroke}%
\pgfsetdash{}{0pt}%
\pgfsys@defobject{currentmarker}{\pgfqpoint{-0.020833in}{-0.020833in}}{\pgfqpoint{0.020833in}{0.020833in}}{%
\pgfpathmoveto{\pgfqpoint{0.000000in}{-0.020833in}}%
\pgfpathcurveto{\pgfqpoint{0.005525in}{-0.020833in}}{\pgfqpoint{0.010825in}{-0.018638in}}{\pgfqpoint{0.014731in}{-0.014731in}}%
\pgfpathcurveto{\pgfqpoint{0.018638in}{-0.010825in}}{\pgfqpoint{0.020833in}{-0.005525in}}{\pgfqpoint{0.020833in}{0.000000in}}%
\pgfpathcurveto{\pgfqpoint{0.020833in}{0.005525in}}{\pgfqpoint{0.018638in}{0.010825in}}{\pgfqpoint{0.014731in}{0.014731in}}%
\pgfpathcurveto{\pgfqpoint{0.010825in}{0.018638in}}{\pgfqpoint{0.005525in}{0.020833in}}{\pgfqpoint{0.000000in}{0.020833in}}%
\pgfpathcurveto{\pgfqpoint{-0.005525in}{0.020833in}}{\pgfqpoint{-0.010825in}{0.018638in}}{\pgfqpoint{-0.014731in}{0.014731in}}%
\pgfpathcurveto{\pgfqpoint{-0.018638in}{0.010825in}}{\pgfqpoint{-0.020833in}{0.005525in}}{\pgfqpoint{-0.020833in}{0.000000in}}%
\pgfpathcurveto{\pgfqpoint{-0.020833in}{-0.005525in}}{\pgfqpoint{-0.018638in}{-0.010825in}}{\pgfqpoint{-0.014731in}{-0.014731in}}%
\pgfpathcurveto{\pgfqpoint{-0.010825in}{-0.018638in}}{\pgfqpoint{-0.005525in}{-0.020833in}}{\pgfqpoint{0.000000in}{-0.020833in}}%
\pgfpathclose%
\pgfusepath{stroke,fill}%
}%
\begin{pgfscope}%
\pgfsys@transformshift{0.881250in}{0.903269in}%
\pgfsys@useobject{currentmarker}{}%
\end{pgfscope}%
\begin{pgfscope}%
\pgfsys@transformshift{1.002344in}{0.912376in}%
\pgfsys@useobject{currentmarker}{}%
\end{pgfscope}%
\begin{pgfscope}%
\pgfsys@transformshift{1.123438in}{0.923373in}%
\pgfsys@useobject{currentmarker}{}%
\end{pgfscope}%
\begin{pgfscope}%
\pgfsys@transformshift{1.244531in}{0.936810in}%
\pgfsys@useobject{currentmarker}{}%
\end{pgfscope}%
\begin{pgfscope}%
\pgfsys@transformshift{1.365625in}{0.953447in}%
\pgfsys@useobject{currentmarker}{}%
\end{pgfscope}%
\begin{pgfscope}%
\pgfsys@transformshift{1.486719in}{0.974352in}%
\pgfsys@useobject{currentmarker}{}%
\end{pgfscope}%
\begin{pgfscope}%
\pgfsys@transformshift{1.607813in}{1.001056in}%
\pgfsys@useobject{currentmarker}{}%
\end{pgfscope}%
\begin{pgfscope}%
\pgfsys@transformshift{1.728906in}{1.035809in}%
\pgfsys@useobject{currentmarker}{}%
\end{pgfscope}%
\begin{pgfscope}%
\pgfsys@transformshift{1.850000in}{1.081983in}%
\pgfsys@useobject{currentmarker}{}%
\end{pgfscope}%
\begin{pgfscope}%
\pgfsys@transformshift{1.971094in}{1.144732in}%
\pgfsys@useobject{currentmarker}{}%
\end{pgfscope}%
\begin{pgfscope}%
\pgfsys@transformshift{2.092188in}{1.232044in}%
\pgfsys@useobject{currentmarker}{}%
\end{pgfscope}%
\begin{pgfscope}%
\pgfsys@transformshift{2.213281in}{1.356240in}%
\pgfsys@useobject{currentmarker}{}%
\end{pgfscope}%
\begin{pgfscope}%
\pgfsys@transformshift{2.334375in}{1.535305in}%
\pgfsys@useobject{currentmarker}{}%
\end{pgfscope}%
\begin{pgfscope}%
\pgfsys@transformshift{2.455469in}{1.790397in}%
\pgfsys@useobject{currentmarker}{}%
\end{pgfscope}%
\begin{pgfscope}%
\pgfsys@transformshift{2.576563in}{2.126152in}%
\pgfsys@useobject{currentmarker}{}%
\end{pgfscope}%
\begin{pgfscope}%
\pgfsys@transformshift{2.697656in}{2.470988in}%
\pgfsys@useobject{currentmarker}{}%
\end{pgfscope}%
\begin{pgfscope}%
\pgfsys@transformshift{2.818750in}{2.630833in}%
\pgfsys@useobject{currentmarker}{}%
\end{pgfscope}%
\begin{pgfscope}%
\pgfsys@transformshift{2.939844in}{2.470988in}%
\pgfsys@useobject{currentmarker}{}%
\end{pgfscope}%
\begin{pgfscope}%
\pgfsys@transformshift{3.060938in}{2.126152in}%
\pgfsys@useobject{currentmarker}{}%
\end{pgfscope}%
\begin{pgfscope}%
\pgfsys@transformshift{3.182031in}{1.790397in}%
\pgfsys@useobject{currentmarker}{}%
\end{pgfscope}%
\begin{pgfscope}%
\pgfsys@transformshift{3.303125in}{1.535305in}%
\pgfsys@useobject{currentmarker}{}%
\end{pgfscope}%
\begin{pgfscope}%
\pgfsys@transformshift{3.424219in}{1.356240in}%
\pgfsys@useobject{currentmarker}{}%
\end{pgfscope}%
\begin{pgfscope}%
\pgfsys@transformshift{3.545313in}{1.232044in}%
\pgfsys@useobject{currentmarker}{}%
\end{pgfscope}%
\begin{pgfscope}%
\pgfsys@transformshift{3.666406in}{1.144732in}%
\pgfsys@useobject{currentmarker}{}%
\end{pgfscope}%
\begin{pgfscope}%
\pgfsys@transformshift{3.787500in}{1.081983in}%
\pgfsys@useobject{currentmarker}{}%
\end{pgfscope}%
\begin{pgfscope}%
\pgfsys@transformshift{3.908594in}{1.035809in}%
\pgfsys@useobject{currentmarker}{}%
\end{pgfscope}%
\begin{pgfscope}%
\pgfsys@transformshift{4.029687in}{1.001056in}%
\pgfsys@useobject{currentmarker}{}%
\end{pgfscope}%
\begin{pgfscope}%
\pgfsys@transformshift{4.150781in}{0.974352in}%
\pgfsys@useobject{currentmarker}{}%
\end{pgfscope}%
\begin{pgfscope}%
\pgfsys@transformshift{4.271875in}{0.953447in}%
\pgfsys@useobject{currentmarker}{}%
\end{pgfscope}%
\begin{pgfscope}%
\pgfsys@transformshift{4.392969in}{0.936810in}%
\pgfsys@useobject{currentmarker}{}%
\end{pgfscope}%
\begin{pgfscope}%
\pgfsys@transformshift{4.514063in}{0.923373in}%
\pgfsys@useobject{currentmarker}{}%
\end{pgfscope}%
\begin{pgfscope}%
\pgfsys@transformshift{4.635156in}{0.912376in}%
\pgfsys@useobject{currentmarker}{}%
\end{pgfscope}%
\end{pgfscope}%
\begin{pgfscope}%
\pgfpathrectangle{\pgfqpoint{0.687500in}{0.385000in}}{\pgfqpoint{4.262500in}{2.695000in}}%
\pgfusepath{clip}%
\pgfsetrectcap%
\pgfsetroundjoin%
\pgfsetlinewidth{1.505625pt}%
\definecolor{currentstroke}{rgb}{0.737255,0.741176,0.133333}%
\pgfsetstrokecolor{currentstroke}%
\pgfsetdash{}{0pt}%
\pgfpathmoveto{\pgfqpoint{0.881250in}{0.903269in}}%
\pgfpathlineto{\pgfqpoint{0.881252in}{0.375000in}}%
\pgfpathmoveto{\pgfqpoint{1.004149in}{0.375000in}}%
\pgfpathlineto{\pgfqpoint{1.004399in}{3.090000in}}%
\pgfpathmoveto{\pgfqpoint{1.123071in}{3.090000in}}%
\pgfpathlineto{\pgfqpoint{1.125905in}{0.375000in}}%
\pgfpathmoveto{\pgfqpoint{1.241208in}{0.375000in}}%
\pgfpathlineto{\pgfqpoint{1.251225in}{1.591030in}}%
\pgfpathlineto{\pgfqpoint{1.270697in}{2.665162in}}%
\pgfpathlineto{\pgfqpoint{1.290170in}{2.805969in}}%
\pgfpathlineto{\pgfqpoint{1.309642in}{2.419563in}}%
\pgfpathlineto{\pgfqpoint{1.329114in}{1.844004in}}%
\pgfpathlineto{\pgfqpoint{1.348587in}{1.304379in}}%
\pgfpathlineto{\pgfqpoint{1.368059in}{0.914505in}}%
\pgfpathlineto{\pgfqpoint{1.387531in}{0.702349in}}%
\pgfpathlineto{\pgfqpoint{1.407004in}{0.642504in}}%
\pgfpathlineto{\pgfqpoint{1.426476in}{0.685542in}}%
\pgfpathlineto{\pgfqpoint{1.445948in}{0.779499in}}%
\pgfpathlineto{\pgfqpoint{1.465421in}{0.882501in}}%
\pgfpathlineto{\pgfqpoint{1.484893in}{0.967797in}}%
\pgfpathlineto{\pgfqpoint{1.504366in}{1.023432in}}%
\pgfpathlineto{\pgfqpoint{1.523838in}{1.048897in}}%
\pgfpathlineto{\pgfqpoint{1.543310in}{1.050708in}}%
\pgfpathlineto{\pgfqpoint{1.562783in}{1.038250in}}%
\pgfpathlineto{\pgfqpoint{1.582255in}{1.020646in}}%
\pgfpathlineto{\pgfqpoint{1.601727in}{1.004878in}}%
\pgfpathlineto{\pgfqpoint{1.621200in}{0.995062in}}%
\pgfpathlineto{\pgfqpoint{1.640672in}{0.992596in}}%
\pgfpathlineto{\pgfqpoint{1.660144in}{0.996819in}}%
\pgfpathlineto{\pgfqpoint{1.679617in}{1.005868in}}%
\pgfpathlineto{\pgfqpoint{1.738034in}{1.040914in}}%
\pgfpathlineto{\pgfqpoint{1.757506in}{1.050551in}}%
\pgfpathlineto{\pgfqpoint{1.776979in}{1.058487in}}%
\pgfpathlineto{\pgfqpoint{1.815923in}{1.071087in}}%
\pgfpathlineto{\pgfqpoint{1.854868in}{1.083723in}}%
\pgfpathlineto{\pgfqpoint{1.893813in}{1.100165in}}%
\pgfpathlineto{\pgfqpoint{1.932758in}{1.121071in}}%
\pgfpathlineto{\pgfqpoint{1.971702in}{1.145124in}}%
\pgfpathlineto{\pgfqpoint{2.030119in}{1.184600in}}%
\pgfpathlineto{\pgfqpoint{2.069064in}{1.213435in}}%
\pgfpathlineto{\pgfqpoint{2.108009in}{1.245587in}}%
\pgfpathlineto{\pgfqpoint{2.146954in}{1.282165in}}%
\pgfpathlineto{\pgfqpoint{2.185898in}{1.323795in}}%
\pgfpathlineto{\pgfqpoint{2.224843in}{1.370749in}}%
\pgfpathlineto{\pgfqpoint{2.263788in}{1.423344in}}%
\pgfpathlineto{\pgfqpoint{2.302732in}{1.482213in}}%
\pgfpathlineto{\pgfqpoint{2.341677in}{1.548290in}}%
\pgfpathlineto{\pgfqpoint{2.380622in}{1.622542in}}%
\pgfpathlineto{\pgfqpoint{2.419567in}{1.705674in}}%
\pgfpathlineto{\pgfqpoint{2.458511in}{1.797929in}}%
\pgfpathlineto{\pgfqpoint{2.497456in}{1.898994in}}%
\pgfpathlineto{\pgfqpoint{2.536401in}{2.007856in}}%
\pgfpathlineto{\pgfqpoint{2.594818in}{2.180997in}}%
\pgfpathlineto{\pgfqpoint{2.653235in}{2.353565in}}%
\pgfpathlineto{\pgfqpoint{2.672707in}{2.407345in}}%
\pgfpathlineto{\pgfqpoint{2.692180in}{2.457623in}}%
\pgfpathlineto{\pgfqpoint{2.711652in}{2.503338in}}%
\pgfpathlineto{\pgfqpoint{2.731124in}{2.543435in}}%
\pgfpathlineto{\pgfqpoint{2.750597in}{2.576917in}}%
\pgfpathlineto{\pgfqpoint{2.770069in}{2.602898in}}%
\pgfpathlineto{\pgfqpoint{2.789541in}{2.620659in}}%
\pgfpathlineto{\pgfqpoint{2.809014in}{2.629687in}}%
\pgfpathlineto{\pgfqpoint{2.828486in}{2.629716in}}%
\pgfpathlineto{\pgfqpoint{2.847959in}{2.620738in}}%
\pgfpathlineto{\pgfqpoint{2.867431in}{2.603009in}}%
\pgfpathlineto{\pgfqpoint{2.886903in}{2.577032in}}%
\pgfpathlineto{\pgfqpoint{2.906376in}{2.543526in}}%
\pgfpathlineto{\pgfqpoint{2.925848in}{2.503381in}}%
\pgfpathlineto{\pgfqpoint{2.945320in}{2.457605in}}%
\pgfpathlineto{\pgfqpoint{2.964793in}{2.407269in}}%
\pgfpathlineto{\pgfqpoint{3.003737in}{2.297171in}}%
\pgfpathlineto{\pgfqpoint{3.120572in}{1.952719in}}%
\pgfpathlineto{\pgfqpoint{3.159516in}{1.847510in}}%
\pgfpathlineto{\pgfqpoint{3.198461in}{1.750577in}}%
\pgfpathlineto{\pgfqpoint{3.237406in}{1.662723in}}%
\pgfpathlineto{\pgfqpoint{3.276351in}{1.584155in}}%
\pgfpathlineto{\pgfqpoint{3.315295in}{1.514394in}}%
\pgfpathlineto{\pgfqpoint{3.354240in}{1.452340in}}%
\pgfpathlineto{\pgfqpoint{3.393185in}{1.396672in}}%
\pgfpathlineto{\pgfqpoint{3.432129in}{1.346466in}}%
\pgfpathlineto{\pgfqpoint{3.471074in}{1.301601in}}%
\pgfpathlineto{\pgfqpoint{3.510019in}{1.262490in}}%
\pgfpathlineto{\pgfqpoint{3.548964in}{1.229138in}}%
\pgfpathlineto{\pgfqpoint{3.587908in}{1.200178in}}%
\pgfpathlineto{\pgfqpoint{3.724215in}{1.105208in}}%
\pgfpathlineto{\pgfqpoint{3.743687in}{1.094862in}}%
\pgfpathlineto{\pgfqpoint{3.763160in}{1.087244in}}%
\pgfpathlineto{\pgfqpoint{3.782632in}{1.082672in}}%
\pgfpathlineto{\pgfqpoint{3.802104in}{1.080807in}}%
\pgfpathlineto{\pgfqpoint{3.841049in}{1.079598in}}%
\pgfpathlineto{\pgfqpoint{3.860521in}{1.075515in}}%
\pgfpathlineto{\pgfqpoint{3.879994in}{1.065428in}}%
\pgfpathlineto{\pgfqpoint{3.899466in}{1.047275in}}%
\pgfpathlineto{\pgfqpoint{3.918938in}{1.020751in}}%
\pgfpathlineto{\pgfqpoint{3.957883in}{0.955856in}}%
\pgfpathlineto{\pgfqpoint{3.977356in}{0.932648in}}%
\pgfpathlineto{\pgfqpoint{3.996828in}{0.930111in}}%
\pgfpathlineto{\pgfqpoint{4.016300in}{0.959160in}}%
\pgfpathlineto{\pgfqpoint{4.035773in}{1.026072in}}%
\pgfpathlineto{\pgfqpoint{4.055245in}{1.127316in}}%
\pgfpathlineto{\pgfqpoint{4.074717in}{1.244424in}}%
\pgfpathlineto{\pgfqpoint{4.094190in}{1.340687in}}%
\pgfpathlineto{\pgfqpoint{4.113662in}{1.362135in}}%
\pgfpathlineto{\pgfqpoint{4.133134in}{1.245597in}}%
\pgfpathlineto{\pgfqpoint{4.152607in}{0.936244in}}%
\pgfpathlineto{\pgfqpoint{4.173245in}{0.375000in}}%
\pgfpathmoveto{\pgfqpoint{4.266060in}{0.375000in}}%
\pgfpathlineto{\pgfqpoint{4.269441in}{0.649289in}}%
\pgfpathlineto{\pgfqpoint{4.284452in}{3.090000in}}%
\pgfpathmoveto{\pgfqpoint{4.389812in}{3.090000in}}%
\pgfpathlineto{\pgfqpoint{4.391863in}{0.375000in}}%
\pgfpathmoveto{\pgfqpoint{4.511726in}{0.375000in}}%
\pgfpathlineto{\pgfqpoint{4.511898in}{3.090000in}}%
\pgfpathmoveto{\pgfqpoint{4.632612in}{3.090000in}}%
\pgfpathlineto{\pgfqpoint{4.632620in}{0.375000in}}%
\pgfpathlineto{\pgfqpoint{4.632620in}{0.375000in}}%
\pgfusepath{stroke}%
\end{pgfscope}%
\begin{pgfscope}%
\pgfsetrectcap%
\pgfsetmiterjoin%
\pgfsetlinewidth{0.803000pt}%
\definecolor{currentstroke}{rgb}{0.000000,0.000000,0.000000}%
\pgfsetstrokecolor{currentstroke}%
\pgfsetdash{}{0pt}%
\pgfpathmoveto{\pgfqpoint{0.687500in}{0.385000in}}%
\pgfpathlineto{\pgfqpoint{0.687500in}{3.080000in}}%
\pgfusepath{stroke}%
\end{pgfscope}%
\begin{pgfscope}%
\pgfsetrectcap%
\pgfsetmiterjoin%
\pgfsetlinewidth{0.803000pt}%
\definecolor{currentstroke}{rgb}{0.000000,0.000000,0.000000}%
\pgfsetstrokecolor{currentstroke}%
\pgfsetdash{}{0pt}%
\pgfpathmoveto{\pgfqpoint{4.950000in}{0.385000in}}%
\pgfpathlineto{\pgfqpoint{4.950000in}{3.080000in}}%
\pgfusepath{stroke}%
\end{pgfscope}%
\begin{pgfscope}%
\pgfsetrectcap%
\pgfsetmiterjoin%
\pgfsetlinewidth{0.803000pt}%
\definecolor{currentstroke}{rgb}{0.000000,0.000000,0.000000}%
\pgfsetstrokecolor{currentstroke}%
\pgfsetdash{}{0pt}%
\pgfpathmoveto{\pgfqpoint{0.687500in}{0.385000in}}%
\pgfpathlineto{\pgfqpoint{4.950000in}{0.385000in}}%
\pgfusepath{stroke}%
\end{pgfscope}%
\begin{pgfscope}%
\pgfsetrectcap%
\pgfsetmiterjoin%
\pgfsetlinewidth{0.803000pt}%
\definecolor{currentstroke}{rgb}{0.000000,0.000000,0.000000}%
\pgfsetstrokecolor{currentstroke}%
\pgfsetdash{}{0pt}%
\pgfpathmoveto{\pgfqpoint{0.687500in}{3.080000in}}%
\pgfpathlineto{\pgfqpoint{4.950000in}{3.080000in}}%
\pgfusepath{stroke}%
\end{pgfscope}%
\begin{pgfscope}%
\definecolor{textcolor}{rgb}{0.000000,0.000000,0.000000}%
\pgfsetstrokecolor{textcolor}%
\pgfsetfillcolor{textcolor}%
\pgftext[x=2.818750in,y=3.163333in,,base]{\color{textcolor}\rmfamily\fontsize{12.000000}{14.400000}\selectfont N=5, 7, 15, 31}%
\end{pgfscope}%
\begin{pgfscope}%
\pgfsetbuttcap%
\pgfsetmiterjoin%
\definecolor{currentfill}{rgb}{1.000000,1.000000,1.000000}%
\pgfsetfillcolor{currentfill}%
\pgfsetfillopacity{0.800000}%
\pgfsetlinewidth{1.003750pt}%
\definecolor{currentstroke}{rgb}{0.800000,0.800000,0.800000}%
\pgfsetstrokecolor{currentstroke}%
\pgfsetstrokeopacity{0.800000}%
\pgfsetdash{}{0pt}%
\pgfpathmoveto{\pgfqpoint{0.784722in}{1.775801in}}%
\pgfpathlineto{\pgfqpoint{2.189358in}{1.775801in}}%
\pgfpathquadraticcurveto{\pgfqpoint{2.217136in}{1.775801in}}{\pgfqpoint{2.217136in}{1.803578in}}%
\pgfpathlineto{\pgfqpoint{2.217136in}{2.982778in}}%
\pgfpathquadraticcurveto{\pgfqpoint{2.217136in}{3.010556in}}{\pgfqpoint{2.189358in}{3.010556in}}%
\pgfpathlineto{\pgfqpoint{0.784722in}{3.010556in}}%
\pgfpathquadraticcurveto{\pgfqpoint{0.756944in}{3.010556in}}{\pgfqpoint{0.756944in}{2.982778in}}%
\pgfpathlineto{\pgfqpoint{0.756944in}{1.803578in}}%
\pgfpathquadraticcurveto{\pgfqpoint{0.756944in}{1.775801in}}{\pgfqpoint{0.784722in}{1.775801in}}%
\pgfpathclose%
\pgfusepath{stroke,fill}%
\end{pgfscope}%
\begin{pgfscope}%
\pgfsetrectcap%
\pgfsetroundjoin%
\pgfsetlinewidth{1.505625pt}%
\definecolor{currentstroke}{rgb}{0.121569,0.466667,0.705882}%
\pgfsetstrokecolor{currentstroke}%
\pgfsetdash{}{0pt}%
\pgfpathmoveto{\pgfqpoint{0.812500in}{2.820146in}}%
\pgfpathlineto{\pgfqpoint{1.090278in}{2.820146in}}%
\pgfusepath{stroke}%
\end{pgfscope}%
\begin{pgfscope}%
\definecolor{textcolor}{rgb}{0.000000,0.000000,0.000000}%
\pgfsetstrokecolor{textcolor}%
\pgfsetfillcolor{textcolor}%
\pgftext[x=1.201389in,y=2.771534in,left,base]{\color{textcolor}\rmfamily\fontsize{10.000000}{12.000000}\selectfont \(\displaystyle y(x)=\)\(\displaystyle \frac{1}{1+25x^{2}}\)}%
\end{pgfscope}%
\begin{pgfscope}%
\pgfsetrectcap%
\pgfsetroundjoin%
\pgfsetlinewidth{1.505625pt}%
\definecolor{currentstroke}{rgb}{0.172549,0.627451,0.172549}%
\pgfsetstrokecolor{currentstroke}%
\pgfsetdash{}{0pt}%
\pgfpathmoveto{\pgfqpoint{0.812500in}{2.539689in}}%
\pgfpathlineto{\pgfqpoint{1.090278in}{2.539689in}}%
\pgfusepath{stroke}%
\end{pgfscope}%
\begin{pgfscope}%
\definecolor{textcolor}{rgb}{0.000000,0.000000,0.000000}%
\pgfsetstrokecolor{textcolor}%
\pgfsetfillcolor{textcolor}%
\pgftext[x=1.201389in,y=2.491078in,left,base]{\color{textcolor}\rmfamily\fontsize{10.000000}{12.000000}\selectfont W5(x)}%
\end{pgfscope}%
\begin{pgfscope}%
\pgfsetrectcap%
\pgfsetroundjoin%
\pgfsetlinewidth{1.505625pt}%
\definecolor{currentstroke}{rgb}{0.580392,0.403922,0.741176}%
\pgfsetstrokecolor{currentstroke}%
\pgfsetdash{}{0pt}%
\pgfpathmoveto{\pgfqpoint{0.812500in}{2.331356in}}%
\pgfpathlineto{\pgfqpoint{1.090278in}{2.331356in}}%
\pgfusepath{stroke}%
\end{pgfscope}%
\begin{pgfscope}%
\definecolor{textcolor}{rgb}{0.000000,0.000000,0.000000}%
\pgfsetstrokecolor{textcolor}%
\pgfsetfillcolor{textcolor}%
\pgftext[x=1.201389in,y=2.282745in,left,base]{\color{textcolor}\rmfamily\fontsize{10.000000}{12.000000}\selectfont W7(x)}%
\end{pgfscope}%
\begin{pgfscope}%
\pgfsetrectcap%
\pgfsetroundjoin%
\pgfsetlinewidth{1.505625pt}%
\definecolor{currentstroke}{rgb}{0.890196,0.466667,0.760784}%
\pgfsetstrokecolor{currentstroke}%
\pgfsetdash{}{0pt}%
\pgfpathmoveto{\pgfqpoint{0.812500in}{2.123023in}}%
\pgfpathlineto{\pgfqpoint{1.090278in}{2.123023in}}%
\pgfusepath{stroke}%
\end{pgfscope}%
\begin{pgfscope}%
\definecolor{textcolor}{rgb}{0.000000,0.000000,0.000000}%
\pgfsetstrokecolor{textcolor}%
\pgfsetfillcolor{textcolor}%
\pgftext[x=1.201389in,y=2.074412in,left,base]{\color{textcolor}\rmfamily\fontsize{10.000000}{12.000000}\selectfont W15(x)}%
\end{pgfscope}%
\begin{pgfscope}%
\pgfsetrectcap%
\pgfsetroundjoin%
\pgfsetlinewidth{1.505625pt}%
\definecolor{currentstroke}{rgb}{0.737255,0.741176,0.133333}%
\pgfsetstrokecolor{currentstroke}%
\pgfsetdash{}{0pt}%
\pgfpathmoveto{\pgfqpoint{0.812500in}{1.914689in}}%
\pgfpathlineto{\pgfqpoint{1.090278in}{1.914689in}}%
\pgfusepath{stroke}%
\end{pgfscope}%
\begin{pgfscope}%
\definecolor{textcolor}{rgb}{0.000000,0.000000,0.000000}%
\pgfsetstrokecolor{textcolor}%
\pgfsetfillcolor{textcolor}%
\pgftext[x=1.201389in,y=1.866078in,left,base]{\color{textcolor}\rmfamily\fontsize{10.000000}{12.000000}\selectfont W31(x)}%
\end{pgfscope}%
\end{pgfpicture}%
\makeatother%
\endgroup%
        
    \end{center}
    \caption{Węzły jednorodne, funkcja \(y\), \(N=5,7,15,31\)}
\end{figure}

% ------------------------------------------------------------------------------

\begin{figure}[h]
    \begin{center}
        %% Creator: Matplotlib, PGF backend
%%
%% To include the figure in your LaTeX document, write
%%   \input{<filename>.pgf}
%%
%% Make sure the required packages are loaded in your preamble
%%   \usepackage{pgf}
%%
%% Figures using additional raster images can only be included by \input if
%% they are in the same directory as the main LaTeX file. For loading figures
%% from other directories you can use the `import` package
%%   \usepackage{import}
%% and then include the figures with
%%   \import{<path to file>}{<filename>.pgf}
%%
%% Matplotlib used the following preamble
%%
\begingroup%
\makeatletter%
\begin{pgfpicture}%
\pgfpathrectangle{\pgfpointorigin}{\pgfqpoint{5.500000in}{3.500000in}}%
\pgfusepath{use as bounding box, clip}%
\begin{pgfscope}%
\pgfsetbuttcap%
\pgfsetmiterjoin%
\definecolor{currentfill}{rgb}{1.000000,1.000000,1.000000}%
\pgfsetfillcolor{currentfill}%
\pgfsetlinewidth{0.000000pt}%
\definecolor{currentstroke}{rgb}{1.000000,1.000000,1.000000}%
\pgfsetstrokecolor{currentstroke}%
\pgfsetdash{}{0pt}%
\pgfpathmoveto{\pgfqpoint{0.000000in}{0.000000in}}%
\pgfpathlineto{\pgfqpoint{5.500000in}{0.000000in}}%
\pgfpathlineto{\pgfqpoint{5.500000in}{3.500000in}}%
\pgfpathlineto{\pgfqpoint{0.000000in}{3.500000in}}%
\pgfpathclose%
\pgfusepath{fill}%
\end{pgfscope}%
\begin{pgfscope}%
\pgfsetbuttcap%
\pgfsetmiterjoin%
\definecolor{currentfill}{rgb}{1.000000,1.000000,1.000000}%
\pgfsetfillcolor{currentfill}%
\pgfsetlinewidth{0.000000pt}%
\definecolor{currentstroke}{rgb}{0.000000,0.000000,0.000000}%
\pgfsetstrokecolor{currentstroke}%
\pgfsetstrokeopacity{0.000000}%
\pgfsetdash{}{0pt}%
\pgfpathmoveto{\pgfqpoint{0.687500in}{0.385000in}}%
\pgfpathlineto{\pgfqpoint{4.950000in}{0.385000in}}%
\pgfpathlineto{\pgfqpoint{4.950000in}{3.080000in}}%
\pgfpathlineto{\pgfqpoint{0.687500in}{3.080000in}}%
\pgfpathclose%
\pgfusepath{fill}%
\end{pgfscope}%
\begin{pgfscope}%
\pgfsetbuttcap%
\pgfsetroundjoin%
\definecolor{currentfill}{rgb}{0.000000,0.000000,0.000000}%
\pgfsetfillcolor{currentfill}%
\pgfsetlinewidth{0.803000pt}%
\definecolor{currentstroke}{rgb}{0.000000,0.000000,0.000000}%
\pgfsetstrokecolor{currentstroke}%
\pgfsetdash{}{0pt}%
\pgfsys@defobject{currentmarker}{\pgfqpoint{0.000000in}{-0.048611in}}{\pgfqpoint{0.000000in}{0.000000in}}{%
\pgfpathmoveto{\pgfqpoint{0.000000in}{0.000000in}}%
\pgfpathlineto{\pgfqpoint{0.000000in}{-0.048611in}}%
\pgfusepath{stroke,fill}%
}%
\begin{pgfscope}%
\pgfsys@transformshift{0.881250in}{0.385000in}%
\pgfsys@useobject{currentmarker}{}%
\end{pgfscope}%
\end{pgfscope}%
\begin{pgfscope}%
\definecolor{textcolor}{rgb}{0.000000,0.000000,0.000000}%
\pgfsetstrokecolor{textcolor}%
\pgfsetfillcolor{textcolor}%
\pgftext[x=0.881250in,y=0.287778in,,top]{\color{textcolor}\rmfamily\fontsize{10.000000}{12.000000}\selectfont \(\displaystyle -1.00\)}%
\end{pgfscope}%
\begin{pgfscope}%
\pgfsetbuttcap%
\pgfsetroundjoin%
\definecolor{currentfill}{rgb}{0.000000,0.000000,0.000000}%
\pgfsetfillcolor{currentfill}%
\pgfsetlinewidth{0.803000pt}%
\definecolor{currentstroke}{rgb}{0.000000,0.000000,0.000000}%
\pgfsetstrokecolor{currentstroke}%
\pgfsetdash{}{0pt}%
\pgfsys@defobject{currentmarker}{\pgfqpoint{0.000000in}{-0.048611in}}{\pgfqpoint{0.000000in}{0.000000in}}{%
\pgfpathmoveto{\pgfqpoint{0.000000in}{0.000000in}}%
\pgfpathlineto{\pgfqpoint{0.000000in}{-0.048611in}}%
\pgfusepath{stroke,fill}%
}%
\begin{pgfscope}%
\pgfsys@transformshift{1.365625in}{0.385000in}%
\pgfsys@useobject{currentmarker}{}%
\end{pgfscope}%
\end{pgfscope}%
\begin{pgfscope}%
\definecolor{textcolor}{rgb}{0.000000,0.000000,0.000000}%
\pgfsetstrokecolor{textcolor}%
\pgfsetfillcolor{textcolor}%
\pgftext[x=1.365625in,y=0.287778in,,top]{\color{textcolor}\rmfamily\fontsize{10.000000}{12.000000}\selectfont \(\displaystyle -0.75\)}%
\end{pgfscope}%
\begin{pgfscope}%
\pgfsetbuttcap%
\pgfsetroundjoin%
\definecolor{currentfill}{rgb}{0.000000,0.000000,0.000000}%
\pgfsetfillcolor{currentfill}%
\pgfsetlinewidth{0.803000pt}%
\definecolor{currentstroke}{rgb}{0.000000,0.000000,0.000000}%
\pgfsetstrokecolor{currentstroke}%
\pgfsetdash{}{0pt}%
\pgfsys@defobject{currentmarker}{\pgfqpoint{0.000000in}{-0.048611in}}{\pgfqpoint{0.000000in}{0.000000in}}{%
\pgfpathmoveto{\pgfqpoint{0.000000in}{0.000000in}}%
\pgfpathlineto{\pgfqpoint{0.000000in}{-0.048611in}}%
\pgfusepath{stroke,fill}%
}%
\begin{pgfscope}%
\pgfsys@transformshift{1.850000in}{0.385000in}%
\pgfsys@useobject{currentmarker}{}%
\end{pgfscope}%
\end{pgfscope}%
\begin{pgfscope}%
\definecolor{textcolor}{rgb}{0.000000,0.000000,0.000000}%
\pgfsetstrokecolor{textcolor}%
\pgfsetfillcolor{textcolor}%
\pgftext[x=1.850000in,y=0.287778in,,top]{\color{textcolor}\rmfamily\fontsize{10.000000}{12.000000}\selectfont \(\displaystyle -0.50\)}%
\end{pgfscope}%
\begin{pgfscope}%
\pgfsetbuttcap%
\pgfsetroundjoin%
\definecolor{currentfill}{rgb}{0.000000,0.000000,0.000000}%
\pgfsetfillcolor{currentfill}%
\pgfsetlinewidth{0.803000pt}%
\definecolor{currentstroke}{rgb}{0.000000,0.000000,0.000000}%
\pgfsetstrokecolor{currentstroke}%
\pgfsetdash{}{0pt}%
\pgfsys@defobject{currentmarker}{\pgfqpoint{0.000000in}{-0.048611in}}{\pgfqpoint{0.000000in}{0.000000in}}{%
\pgfpathmoveto{\pgfqpoint{0.000000in}{0.000000in}}%
\pgfpathlineto{\pgfqpoint{0.000000in}{-0.048611in}}%
\pgfusepath{stroke,fill}%
}%
\begin{pgfscope}%
\pgfsys@transformshift{2.334375in}{0.385000in}%
\pgfsys@useobject{currentmarker}{}%
\end{pgfscope}%
\end{pgfscope}%
\begin{pgfscope}%
\definecolor{textcolor}{rgb}{0.000000,0.000000,0.000000}%
\pgfsetstrokecolor{textcolor}%
\pgfsetfillcolor{textcolor}%
\pgftext[x=2.334375in,y=0.287778in,,top]{\color{textcolor}\rmfamily\fontsize{10.000000}{12.000000}\selectfont \(\displaystyle -0.25\)}%
\end{pgfscope}%
\begin{pgfscope}%
\pgfsetbuttcap%
\pgfsetroundjoin%
\definecolor{currentfill}{rgb}{0.000000,0.000000,0.000000}%
\pgfsetfillcolor{currentfill}%
\pgfsetlinewidth{0.803000pt}%
\definecolor{currentstroke}{rgb}{0.000000,0.000000,0.000000}%
\pgfsetstrokecolor{currentstroke}%
\pgfsetdash{}{0pt}%
\pgfsys@defobject{currentmarker}{\pgfqpoint{0.000000in}{-0.048611in}}{\pgfqpoint{0.000000in}{0.000000in}}{%
\pgfpathmoveto{\pgfqpoint{0.000000in}{0.000000in}}%
\pgfpathlineto{\pgfqpoint{0.000000in}{-0.048611in}}%
\pgfusepath{stroke,fill}%
}%
\begin{pgfscope}%
\pgfsys@transformshift{2.818750in}{0.385000in}%
\pgfsys@useobject{currentmarker}{}%
\end{pgfscope}%
\end{pgfscope}%
\begin{pgfscope}%
\definecolor{textcolor}{rgb}{0.000000,0.000000,0.000000}%
\pgfsetstrokecolor{textcolor}%
\pgfsetfillcolor{textcolor}%
\pgftext[x=2.818750in,y=0.287778in,,top]{\color{textcolor}\rmfamily\fontsize{10.000000}{12.000000}\selectfont \(\displaystyle 0.00\)}%
\end{pgfscope}%
\begin{pgfscope}%
\pgfsetbuttcap%
\pgfsetroundjoin%
\definecolor{currentfill}{rgb}{0.000000,0.000000,0.000000}%
\pgfsetfillcolor{currentfill}%
\pgfsetlinewidth{0.803000pt}%
\definecolor{currentstroke}{rgb}{0.000000,0.000000,0.000000}%
\pgfsetstrokecolor{currentstroke}%
\pgfsetdash{}{0pt}%
\pgfsys@defobject{currentmarker}{\pgfqpoint{0.000000in}{-0.048611in}}{\pgfqpoint{0.000000in}{0.000000in}}{%
\pgfpathmoveto{\pgfqpoint{0.000000in}{0.000000in}}%
\pgfpathlineto{\pgfqpoint{0.000000in}{-0.048611in}}%
\pgfusepath{stroke,fill}%
}%
\begin{pgfscope}%
\pgfsys@transformshift{3.303125in}{0.385000in}%
\pgfsys@useobject{currentmarker}{}%
\end{pgfscope}%
\end{pgfscope}%
\begin{pgfscope}%
\definecolor{textcolor}{rgb}{0.000000,0.000000,0.000000}%
\pgfsetstrokecolor{textcolor}%
\pgfsetfillcolor{textcolor}%
\pgftext[x=3.303125in,y=0.287778in,,top]{\color{textcolor}\rmfamily\fontsize{10.000000}{12.000000}\selectfont \(\displaystyle 0.25\)}%
\end{pgfscope}%
\begin{pgfscope}%
\pgfsetbuttcap%
\pgfsetroundjoin%
\definecolor{currentfill}{rgb}{0.000000,0.000000,0.000000}%
\pgfsetfillcolor{currentfill}%
\pgfsetlinewidth{0.803000pt}%
\definecolor{currentstroke}{rgb}{0.000000,0.000000,0.000000}%
\pgfsetstrokecolor{currentstroke}%
\pgfsetdash{}{0pt}%
\pgfsys@defobject{currentmarker}{\pgfqpoint{0.000000in}{-0.048611in}}{\pgfqpoint{0.000000in}{0.000000in}}{%
\pgfpathmoveto{\pgfqpoint{0.000000in}{0.000000in}}%
\pgfpathlineto{\pgfqpoint{0.000000in}{-0.048611in}}%
\pgfusepath{stroke,fill}%
}%
\begin{pgfscope}%
\pgfsys@transformshift{3.787500in}{0.385000in}%
\pgfsys@useobject{currentmarker}{}%
\end{pgfscope}%
\end{pgfscope}%
\begin{pgfscope}%
\definecolor{textcolor}{rgb}{0.000000,0.000000,0.000000}%
\pgfsetstrokecolor{textcolor}%
\pgfsetfillcolor{textcolor}%
\pgftext[x=3.787500in,y=0.287778in,,top]{\color{textcolor}\rmfamily\fontsize{10.000000}{12.000000}\selectfont \(\displaystyle 0.50\)}%
\end{pgfscope}%
\begin{pgfscope}%
\pgfsetbuttcap%
\pgfsetroundjoin%
\definecolor{currentfill}{rgb}{0.000000,0.000000,0.000000}%
\pgfsetfillcolor{currentfill}%
\pgfsetlinewidth{0.803000pt}%
\definecolor{currentstroke}{rgb}{0.000000,0.000000,0.000000}%
\pgfsetstrokecolor{currentstroke}%
\pgfsetdash{}{0pt}%
\pgfsys@defobject{currentmarker}{\pgfqpoint{0.000000in}{-0.048611in}}{\pgfqpoint{0.000000in}{0.000000in}}{%
\pgfpathmoveto{\pgfqpoint{0.000000in}{0.000000in}}%
\pgfpathlineto{\pgfqpoint{0.000000in}{-0.048611in}}%
\pgfusepath{stroke,fill}%
}%
\begin{pgfscope}%
\pgfsys@transformshift{4.271875in}{0.385000in}%
\pgfsys@useobject{currentmarker}{}%
\end{pgfscope}%
\end{pgfscope}%
\begin{pgfscope}%
\definecolor{textcolor}{rgb}{0.000000,0.000000,0.000000}%
\pgfsetstrokecolor{textcolor}%
\pgfsetfillcolor{textcolor}%
\pgftext[x=4.271875in,y=0.287778in,,top]{\color{textcolor}\rmfamily\fontsize{10.000000}{12.000000}\selectfont \(\displaystyle 0.75\)}%
\end{pgfscope}%
\begin{pgfscope}%
\pgfsetbuttcap%
\pgfsetroundjoin%
\definecolor{currentfill}{rgb}{0.000000,0.000000,0.000000}%
\pgfsetfillcolor{currentfill}%
\pgfsetlinewidth{0.803000pt}%
\definecolor{currentstroke}{rgb}{0.000000,0.000000,0.000000}%
\pgfsetstrokecolor{currentstroke}%
\pgfsetdash{}{0pt}%
\pgfsys@defobject{currentmarker}{\pgfqpoint{0.000000in}{-0.048611in}}{\pgfqpoint{0.000000in}{0.000000in}}{%
\pgfpathmoveto{\pgfqpoint{0.000000in}{0.000000in}}%
\pgfpathlineto{\pgfqpoint{0.000000in}{-0.048611in}}%
\pgfusepath{stroke,fill}%
}%
\begin{pgfscope}%
\pgfsys@transformshift{4.756250in}{0.385000in}%
\pgfsys@useobject{currentmarker}{}%
\end{pgfscope}%
\end{pgfscope}%
\begin{pgfscope}%
\definecolor{textcolor}{rgb}{0.000000,0.000000,0.000000}%
\pgfsetstrokecolor{textcolor}%
\pgfsetfillcolor{textcolor}%
\pgftext[x=4.756250in,y=0.287778in,,top]{\color{textcolor}\rmfamily\fontsize{10.000000}{12.000000}\selectfont \(\displaystyle 1.00\)}%
\end{pgfscope}%
\begin{pgfscope}%
\definecolor{textcolor}{rgb}{0.000000,0.000000,0.000000}%
\pgfsetstrokecolor{textcolor}%
\pgfsetfillcolor{textcolor}%
\pgftext[x=2.818750in,y=0.108766in,,top]{\color{textcolor}\rmfamily\fontsize{10.000000}{12.000000}\selectfont x}%
\end{pgfscope}%
\begin{pgfscope}%
\pgfsetbuttcap%
\pgfsetroundjoin%
\definecolor{currentfill}{rgb}{0.000000,0.000000,0.000000}%
\pgfsetfillcolor{currentfill}%
\pgfsetlinewidth{0.803000pt}%
\definecolor{currentstroke}{rgb}{0.000000,0.000000,0.000000}%
\pgfsetstrokecolor{currentstroke}%
\pgfsetdash{}{0pt}%
\pgfsys@defobject{currentmarker}{\pgfqpoint{-0.048611in}{0.000000in}}{\pgfqpoint{0.000000in}{0.000000in}}{%
\pgfpathmoveto{\pgfqpoint{0.000000in}{0.000000in}}%
\pgfpathlineto{\pgfqpoint{-0.048611in}{0.000000in}}%
\pgfusepath{stroke,fill}%
}%
\begin{pgfscope}%
\pgfsys@transformshift{0.687500in}{0.474833in}%
\pgfsys@useobject{currentmarker}{}%
\end{pgfscope}%
\end{pgfscope}%
\begin{pgfscope}%
\definecolor{textcolor}{rgb}{0.000000,0.000000,0.000000}%
\pgfsetstrokecolor{textcolor}%
\pgfsetfillcolor{textcolor}%
\pgftext[x=0.304783in,y=0.426608in,left,base]{\color{textcolor}\rmfamily\fontsize{10.000000}{12.000000}\selectfont \(\displaystyle -0.2\)}%
\end{pgfscope}%
\begin{pgfscope}%
\pgfsetbuttcap%
\pgfsetroundjoin%
\definecolor{currentfill}{rgb}{0.000000,0.000000,0.000000}%
\pgfsetfillcolor{currentfill}%
\pgfsetlinewidth{0.803000pt}%
\definecolor{currentstroke}{rgb}{0.000000,0.000000,0.000000}%
\pgfsetstrokecolor{currentstroke}%
\pgfsetdash{}{0pt}%
\pgfsys@defobject{currentmarker}{\pgfqpoint{-0.048611in}{0.000000in}}{\pgfqpoint{0.000000in}{0.000000in}}{%
\pgfpathmoveto{\pgfqpoint{0.000000in}{0.000000in}}%
\pgfpathlineto{\pgfqpoint{-0.048611in}{0.000000in}}%
\pgfusepath{stroke,fill}%
}%
\begin{pgfscope}%
\pgfsys@transformshift{0.687500in}{0.834167in}%
\pgfsys@useobject{currentmarker}{}%
\end{pgfscope}%
\end{pgfscope}%
\begin{pgfscope}%
\definecolor{textcolor}{rgb}{0.000000,0.000000,0.000000}%
\pgfsetstrokecolor{textcolor}%
\pgfsetfillcolor{textcolor}%
\pgftext[x=0.412808in,y=0.785941in,left,base]{\color{textcolor}\rmfamily\fontsize{10.000000}{12.000000}\selectfont \(\displaystyle 0.0\)}%
\end{pgfscope}%
\begin{pgfscope}%
\pgfsetbuttcap%
\pgfsetroundjoin%
\definecolor{currentfill}{rgb}{0.000000,0.000000,0.000000}%
\pgfsetfillcolor{currentfill}%
\pgfsetlinewidth{0.803000pt}%
\definecolor{currentstroke}{rgb}{0.000000,0.000000,0.000000}%
\pgfsetstrokecolor{currentstroke}%
\pgfsetdash{}{0pt}%
\pgfsys@defobject{currentmarker}{\pgfqpoint{-0.048611in}{0.000000in}}{\pgfqpoint{0.000000in}{0.000000in}}{%
\pgfpathmoveto{\pgfqpoint{0.000000in}{0.000000in}}%
\pgfpathlineto{\pgfqpoint{-0.048611in}{0.000000in}}%
\pgfusepath{stroke,fill}%
}%
\begin{pgfscope}%
\pgfsys@transformshift{0.687500in}{1.193500in}%
\pgfsys@useobject{currentmarker}{}%
\end{pgfscope}%
\end{pgfscope}%
\begin{pgfscope}%
\definecolor{textcolor}{rgb}{0.000000,0.000000,0.000000}%
\pgfsetstrokecolor{textcolor}%
\pgfsetfillcolor{textcolor}%
\pgftext[x=0.412808in,y=1.145275in,left,base]{\color{textcolor}\rmfamily\fontsize{10.000000}{12.000000}\selectfont \(\displaystyle 0.2\)}%
\end{pgfscope}%
\begin{pgfscope}%
\pgfsetbuttcap%
\pgfsetroundjoin%
\definecolor{currentfill}{rgb}{0.000000,0.000000,0.000000}%
\pgfsetfillcolor{currentfill}%
\pgfsetlinewidth{0.803000pt}%
\definecolor{currentstroke}{rgb}{0.000000,0.000000,0.000000}%
\pgfsetstrokecolor{currentstroke}%
\pgfsetdash{}{0pt}%
\pgfsys@defobject{currentmarker}{\pgfqpoint{-0.048611in}{0.000000in}}{\pgfqpoint{0.000000in}{0.000000in}}{%
\pgfpathmoveto{\pgfqpoint{0.000000in}{0.000000in}}%
\pgfpathlineto{\pgfqpoint{-0.048611in}{0.000000in}}%
\pgfusepath{stroke,fill}%
}%
\begin{pgfscope}%
\pgfsys@transformshift{0.687500in}{1.552833in}%
\pgfsys@useobject{currentmarker}{}%
\end{pgfscope}%
\end{pgfscope}%
\begin{pgfscope}%
\definecolor{textcolor}{rgb}{0.000000,0.000000,0.000000}%
\pgfsetstrokecolor{textcolor}%
\pgfsetfillcolor{textcolor}%
\pgftext[x=0.412808in,y=1.504608in,left,base]{\color{textcolor}\rmfamily\fontsize{10.000000}{12.000000}\selectfont \(\displaystyle 0.4\)}%
\end{pgfscope}%
\begin{pgfscope}%
\pgfsetbuttcap%
\pgfsetroundjoin%
\definecolor{currentfill}{rgb}{0.000000,0.000000,0.000000}%
\pgfsetfillcolor{currentfill}%
\pgfsetlinewidth{0.803000pt}%
\definecolor{currentstroke}{rgb}{0.000000,0.000000,0.000000}%
\pgfsetstrokecolor{currentstroke}%
\pgfsetdash{}{0pt}%
\pgfsys@defobject{currentmarker}{\pgfqpoint{-0.048611in}{0.000000in}}{\pgfqpoint{0.000000in}{0.000000in}}{%
\pgfpathmoveto{\pgfqpoint{0.000000in}{0.000000in}}%
\pgfpathlineto{\pgfqpoint{-0.048611in}{0.000000in}}%
\pgfusepath{stroke,fill}%
}%
\begin{pgfscope}%
\pgfsys@transformshift{0.687500in}{1.912167in}%
\pgfsys@useobject{currentmarker}{}%
\end{pgfscope}%
\end{pgfscope}%
\begin{pgfscope}%
\definecolor{textcolor}{rgb}{0.000000,0.000000,0.000000}%
\pgfsetstrokecolor{textcolor}%
\pgfsetfillcolor{textcolor}%
\pgftext[x=0.412808in,y=1.863941in,left,base]{\color{textcolor}\rmfamily\fontsize{10.000000}{12.000000}\selectfont \(\displaystyle 0.6\)}%
\end{pgfscope}%
\begin{pgfscope}%
\pgfsetbuttcap%
\pgfsetroundjoin%
\definecolor{currentfill}{rgb}{0.000000,0.000000,0.000000}%
\pgfsetfillcolor{currentfill}%
\pgfsetlinewidth{0.803000pt}%
\definecolor{currentstroke}{rgb}{0.000000,0.000000,0.000000}%
\pgfsetstrokecolor{currentstroke}%
\pgfsetdash{}{0pt}%
\pgfsys@defobject{currentmarker}{\pgfqpoint{-0.048611in}{0.000000in}}{\pgfqpoint{0.000000in}{0.000000in}}{%
\pgfpathmoveto{\pgfqpoint{0.000000in}{0.000000in}}%
\pgfpathlineto{\pgfqpoint{-0.048611in}{0.000000in}}%
\pgfusepath{stroke,fill}%
}%
\begin{pgfscope}%
\pgfsys@transformshift{0.687500in}{2.271500in}%
\pgfsys@useobject{currentmarker}{}%
\end{pgfscope}%
\end{pgfscope}%
\begin{pgfscope}%
\definecolor{textcolor}{rgb}{0.000000,0.000000,0.000000}%
\pgfsetstrokecolor{textcolor}%
\pgfsetfillcolor{textcolor}%
\pgftext[x=0.412808in,y=2.223275in,left,base]{\color{textcolor}\rmfamily\fontsize{10.000000}{12.000000}\selectfont \(\displaystyle 0.8\)}%
\end{pgfscope}%
\begin{pgfscope}%
\pgfsetbuttcap%
\pgfsetroundjoin%
\definecolor{currentfill}{rgb}{0.000000,0.000000,0.000000}%
\pgfsetfillcolor{currentfill}%
\pgfsetlinewidth{0.803000pt}%
\definecolor{currentstroke}{rgb}{0.000000,0.000000,0.000000}%
\pgfsetstrokecolor{currentstroke}%
\pgfsetdash{}{0pt}%
\pgfsys@defobject{currentmarker}{\pgfqpoint{-0.048611in}{0.000000in}}{\pgfqpoint{0.000000in}{0.000000in}}{%
\pgfpathmoveto{\pgfqpoint{0.000000in}{0.000000in}}%
\pgfpathlineto{\pgfqpoint{-0.048611in}{0.000000in}}%
\pgfusepath{stroke,fill}%
}%
\begin{pgfscope}%
\pgfsys@transformshift{0.687500in}{2.630833in}%
\pgfsys@useobject{currentmarker}{}%
\end{pgfscope}%
\end{pgfscope}%
\begin{pgfscope}%
\definecolor{textcolor}{rgb}{0.000000,0.000000,0.000000}%
\pgfsetstrokecolor{textcolor}%
\pgfsetfillcolor{textcolor}%
\pgftext[x=0.412808in,y=2.582608in,left,base]{\color{textcolor}\rmfamily\fontsize{10.000000}{12.000000}\selectfont \(\displaystyle 1.0\)}%
\end{pgfscope}%
\begin{pgfscope}%
\pgfsetbuttcap%
\pgfsetroundjoin%
\definecolor{currentfill}{rgb}{0.000000,0.000000,0.000000}%
\pgfsetfillcolor{currentfill}%
\pgfsetlinewidth{0.803000pt}%
\definecolor{currentstroke}{rgb}{0.000000,0.000000,0.000000}%
\pgfsetstrokecolor{currentstroke}%
\pgfsetdash{}{0pt}%
\pgfsys@defobject{currentmarker}{\pgfqpoint{-0.048611in}{0.000000in}}{\pgfqpoint{0.000000in}{0.000000in}}{%
\pgfpathmoveto{\pgfqpoint{0.000000in}{0.000000in}}%
\pgfpathlineto{\pgfqpoint{-0.048611in}{0.000000in}}%
\pgfusepath{stroke,fill}%
}%
\begin{pgfscope}%
\pgfsys@transformshift{0.687500in}{2.990167in}%
\pgfsys@useobject{currentmarker}{}%
\end{pgfscope}%
\end{pgfscope}%
\begin{pgfscope}%
\definecolor{textcolor}{rgb}{0.000000,0.000000,0.000000}%
\pgfsetstrokecolor{textcolor}%
\pgfsetfillcolor{textcolor}%
\pgftext[x=0.412808in,y=2.941941in,left,base]{\color{textcolor}\rmfamily\fontsize{10.000000}{12.000000}\selectfont \(\displaystyle 1.2\)}%
\end{pgfscope}%
\begin{pgfscope}%
\definecolor{textcolor}{rgb}{0.000000,0.000000,0.000000}%
\pgfsetstrokecolor{textcolor}%
\pgfsetfillcolor{textcolor}%
\pgftext[x=0.249228in,y=1.732500in,,bottom,rotate=90.000000]{\color{textcolor}\rmfamily\fontsize{10.000000}{12.000000}\selectfont y}%
\end{pgfscope}%
\begin{pgfscope}%
\pgfpathrectangle{\pgfqpoint{0.687500in}{0.385000in}}{\pgfqpoint{4.262500in}{2.695000in}}%
\pgfusepath{clip}%
\pgfsetrectcap%
\pgfsetroundjoin%
\pgfsetlinewidth{1.505625pt}%
\definecolor{currentstroke}{rgb}{0.121569,0.466667,0.705882}%
\pgfsetstrokecolor{currentstroke}%
\pgfsetdash{}{0pt}%
\pgfpathmoveto{\pgfqpoint{0.881250in}{1.732500in}}%
\pgfpathlineto{\pgfqpoint{0.978612in}{1.778775in}}%
\pgfpathlineto{\pgfqpoint{1.075974in}{1.827296in}}%
\pgfpathlineto{\pgfqpoint{1.173335in}{1.878000in}}%
\pgfpathlineto{\pgfqpoint{1.290170in}{1.941556in}}%
\pgfpathlineto{\pgfqpoint{1.407004in}{2.007752in}}%
\pgfpathlineto{\pgfqpoint{1.543310in}{2.087643in}}%
\pgfpathlineto{\pgfqpoint{1.796451in}{2.239567in}}%
\pgfpathlineto{\pgfqpoint{1.932758in}{2.320107in}}%
\pgfpathlineto{\pgfqpoint{2.030119in}{2.375473in}}%
\pgfpathlineto{\pgfqpoint{2.108009in}{2.417737in}}%
\pgfpathlineto{\pgfqpoint{2.185898in}{2.457628in}}%
\pgfpathlineto{\pgfqpoint{2.263788in}{2.494605in}}%
\pgfpathlineto{\pgfqpoint{2.322205in}{2.520101in}}%
\pgfpathlineto{\pgfqpoint{2.380622in}{2.543430in}}%
\pgfpathlineto{\pgfqpoint{2.439039in}{2.564379in}}%
\pgfpathlineto{\pgfqpoint{2.497456in}{2.582749in}}%
\pgfpathlineto{\pgfqpoint{2.555873in}{2.598357in}}%
\pgfpathlineto{\pgfqpoint{2.614290in}{2.611046in}}%
\pgfpathlineto{\pgfqpoint{2.672707in}{2.620683in}}%
\pgfpathlineto{\pgfqpoint{2.731124in}{2.627166in}}%
\pgfpathlineto{\pgfqpoint{2.789541in}{2.630425in}}%
\pgfpathlineto{\pgfqpoint{2.847959in}{2.630425in}}%
\pgfpathlineto{\pgfqpoint{2.906376in}{2.627166in}}%
\pgfpathlineto{\pgfqpoint{2.964793in}{2.620683in}}%
\pgfpathlineto{\pgfqpoint{3.023210in}{2.611046in}}%
\pgfpathlineto{\pgfqpoint{3.081627in}{2.598357in}}%
\pgfpathlineto{\pgfqpoint{3.140044in}{2.582749in}}%
\pgfpathlineto{\pgfqpoint{3.198461in}{2.564379in}}%
\pgfpathlineto{\pgfqpoint{3.256878in}{2.543430in}}%
\pgfpathlineto{\pgfqpoint{3.315295in}{2.520101in}}%
\pgfpathlineto{\pgfqpoint{3.373712in}{2.494605in}}%
\pgfpathlineto{\pgfqpoint{3.451602in}{2.457628in}}%
\pgfpathlineto{\pgfqpoint{3.529491in}{2.417737in}}%
\pgfpathlineto{\pgfqpoint{3.607381in}{2.375473in}}%
\pgfpathlineto{\pgfqpoint{3.704742in}{2.320107in}}%
\pgfpathlineto{\pgfqpoint{3.841049in}{2.239567in}}%
\pgfpathlineto{\pgfqpoint{4.211024in}{2.019009in}}%
\pgfpathlineto{\pgfqpoint{4.327858in}{1.952417in}}%
\pgfpathlineto{\pgfqpoint{4.444692in}{1.888394in}}%
\pgfpathlineto{\pgfqpoint{4.542054in}{1.837265in}}%
\pgfpathlineto{\pgfqpoint{4.639416in}{1.788301in}}%
\pgfpathlineto{\pgfqpoint{4.736778in}{1.741574in}}%
\pgfpathlineto{\pgfqpoint{4.756250in}{1.732500in}}%
\pgfpathlineto{\pgfqpoint{4.756250in}{1.732500in}}%
\pgfusepath{stroke}%
\end{pgfscope}%
\begin{pgfscope}%
\pgfpathrectangle{\pgfqpoint{0.687500in}{0.385000in}}{\pgfqpoint{4.262500in}{2.695000in}}%
\pgfusepath{clip}%
\pgfsetbuttcap%
\pgfsetroundjoin%
\definecolor{currentfill}{rgb}{1.000000,0.498039,0.054902}%
\pgfsetfillcolor{currentfill}%
\pgfsetlinewidth{1.003750pt}%
\definecolor{currentstroke}{rgb}{1.000000,0.498039,0.054902}%
\pgfsetstrokecolor{currentstroke}%
\pgfsetdash{}{0pt}%
\pgfsys@defobject{currentmarker}{\pgfqpoint{-0.020833in}{-0.020833in}}{\pgfqpoint{0.020833in}{0.020833in}}{%
\pgfpathmoveto{\pgfqpoint{0.000000in}{-0.020833in}}%
\pgfpathcurveto{\pgfqpoint{0.005525in}{-0.020833in}}{\pgfqpoint{0.010825in}{-0.018638in}}{\pgfqpoint{0.014731in}{-0.014731in}}%
\pgfpathcurveto{\pgfqpoint{0.018638in}{-0.010825in}}{\pgfqpoint{0.020833in}{-0.005525in}}{\pgfqpoint{0.020833in}{0.000000in}}%
\pgfpathcurveto{\pgfqpoint{0.020833in}{0.005525in}}{\pgfqpoint{0.018638in}{0.010825in}}{\pgfqpoint{0.014731in}{0.014731in}}%
\pgfpathcurveto{\pgfqpoint{0.010825in}{0.018638in}}{\pgfqpoint{0.005525in}{0.020833in}}{\pgfqpoint{0.000000in}{0.020833in}}%
\pgfpathcurveto{\pgfqpoint{-0.005525in}{0.020833in}}{\pgfqpoint{-0.010825in}{0.018638in}}{\pgfqpoint{-0.014731in}{0.014731in}}%
\pgfpathcurveto{\pgfqpoint{-0.018638in}{0.010825in}}{\pgfqpoint{-0.020833in}{0.005525in}}{\pgfqpoint{-0.020833in}{0.000000in}}%
\pgfpathcurveto{\pgfqpoint{-0.020833in}{-0.005525in}}{\pgfqpoint{-0.018638in}{-0.010825in}}{\pgfqpoint{-0.014731in}{-0.014731in}}%
\pgfpathcurveto{\pgfqpoint{-0.010825in}{-0.018638in}}{\pgfqpoint{-0.005525in}{-0.020833in}}{\pgfqpoint{0.000000in}{-0.020833in}}%
\pgfpathclose%
\pgfusepath{stroke,fill}%
}%
\begin{pgfscope}%
\pgfsys@transformshift{0.881250in}{1.732500in}%
\pgfsys@useobject{currentmarker}{}%
\end{pgfscope}%
\begin{pgfscope}%
\pgfsys@transformshift{2.172917in}{2.451167in}%
\pgfsys@useobject{currentmarker}{}%
\end{pgfscope}%
\begin{pgfscope}%
\pgfsys@transformshift{3.464583in}{2.451167in}%
\pgfsys@useobject{currentmarker}{}%
\end{pgfscope}%
\end{pgfscope}%
\begin{pgfscope}%
\pgfpathrectangle{\pgfqpoint{0.687500in}{0.385000in}}{\pgfqpoint{4.262500in}{2.695000in}}%
\pgfusepath{clip}%
\pgfsetrectcap%
\pgfsetroundjoin%
\pgfsetlinewidth{1.505625pt}%
\definecolor{currentstroke}{rgb}{0.172549,0.627451,0.172549}%
\pgfsetstrokecolor{currentstroke}%
\pgfsetdash{}{0pt}%
\pgfpathmoveto{\pgfqpoint{0.881250in}{1.732500in}}%
\pgfpathlineto{\pgfqpoint{0.959139in}{1.796198in}}%
\pgfpathlineto{\pgfqpoint{1.037029in}{1.857284in}}%
\pgfpathlineto{\pgfqpoint{1.114918in}{1.915755in}}%
\pgfpathlineto{\pgfqpoint{1.192808in}{1.971614in}}%
\pgfpathlineto{\pgfqpoint{1.270697in}{2.024859in}}%
\pgfpathlineto{\pgfqpoint{1.348587in}{2.075491in}}%
\pgfpathlineto{\pgfqpoint{1.426476in}{2.123510in}}%
\pgfpathlineto{\pgfqpoint{1.504366in}{2.168916in}}%
\pgfpathlineto{\pgfqpoint{1.582255in}{2.211708in}}%
\pgfpathlineto{\pgfqpoint{1.660144in}{2.251887in}}%
\pgfpathlineto{\pgfqpoint{1.738034in}{2.289453in}}%
\pgfpathlineto{\pgfqpoint{1.815923in}{2.324405in}}%
\pgfpathlineto{\pgfqpoint{1.893813in}{2.356744in}}%
\pgfpathlineto{\pgfqpoint{1.971702in}{2.386470in}}%
\pgfpathlineto{\pgfqpoint{2.049592in}{2.413583in}}%
\pgfpathlineto{\pgfqpoint{2.127481in}{2.438082in}}%
\pgfpathlineto{\pgfqpoint{2.205371in}{2.459968in}}%
\pgfpathlineto{\pgfqpoint{2.283260in}{2.479241in}}%
\pgfpathlineto{\pgfqpoint{2.361149in}{2.495901in}}%
\pgfpathlineto{\pgfqpoint{2.439039in}{2.509947in}}%
\pgfpathlineto{\pgfqpoint{2.516928in}{2.521380in}}%
\pgfpathlineto{\pgfqpoint{2.594818in}{2.530200in}}%
\pgfpathlineto{\pgfqpoint{2.672707in}{2.536406in}}%
\pgfpathlineto{\pgfqpoint{2.750597in}{2.540000in}}%
\pgfpathlineto{\pgfqpoint{2.828486in}{2.540980in}}%
\pgfpathlineto{\pgfqpoint{2.906376in}{2.539346in}}%
\pgfpathlineto{\pgfqpoint{2.984265in}{2.535100in}}%
\pgfpathlineto{\pgfqpoint{3.062155in}{2.528240in}}%
\pgfpathlineto{\pgfqpoint{3.140044in}{2.518767in}}%
\pgfpathlineto{\pgfqpoint{3.217933in}{2.506680in}}%
\pgfpathlineto{\pgfqpoint{3.295823in}{2.491981in}}%
\pgfpathlineto{\pgfqpoint{3.373712in}{2.474668in}}%
\pgfpathlineto{\pgfqpoint{3.451602in}{2.454742in}}%
\pgfpathlineto{\pgfqpoint{3.529491in}{2.432202in}}%
\pgfpathlineto{\pgfqpoint{3.607381in}{2.407050in}}%
\pgfpathlineto{\pgfqpoint{3.685270in}{2.379284in}}%
\pgfpathlineto{\pgfqpoint{3.763160in}{2.348904in}}%
\pgfpathlineto{\pgfqpoint{3.841049in}{2.315912in}}%
\pgfpathlineto{\pgfqpoint{3.918938in}{2.280306in}}%
\pgfpathlineto{\pgfqpoint{3.996828in}{2.242087in}}%
\pgfpathlineto{\pgfqpoint{4.074717in}{2.201255in}}%
\pgfpathlineto{\pgfqpoint{4.152607in}{2.157809in}}%
\pgfpathlineto{\pgfqpoint{4.230496in}{2.111750in}}%
\pgfpathlineto{\pgfqpoint{4.308386in}{2.063078in}}%
\pgfpathlineto{\pgfqpoint{4.386275in}{2.011793in}}%
\pgfpathlineto{\pgfqpoint{4.464165in}{1.957894in}}%
\pgfpathlineto{\pgfqpoint{4.542054in}{1.901382in}}%
\pgfpathlineto{\pgfqpoint{4.619943in}{1.842257in}}%
\pgfpathlineto{\pgfqpoint{4.697833in}{1.780519in}}%
\pgfpathlineto{\pgfqpoint{4.756250in}{1.732500in}}%
\pgfpathlineto{\pgfqpoint{4.756250in}{1.732500in}}%
\pgfusepath{stroke}%
\end{pgfscope}%
\begin{pgfscope}%
\pgfpathrectangle{\pgfqpoint{0.687500in}{0.385000in}}{\pgfqpoint{4.262500in}{2.695000in}}%
\pgfusepath{clip}%
\pgfsetbuttcap%
\pgfsetroundjoin%
\definecolor{currentfill}{rgb}{0.839216,0.152941,0.156863}%
\pgfsetfillcolor{currentfill}%
\pgfsetlinewidth{1.003750pt}%
\definecolor{currentstroke}{rgb}{0.839216,0.152941,0.156863}%
\pgfsetstrokecolor{currentstroke}%
\pgfsetdash{}{0pt}%
\pgfsys@defobject{currentmarker}{\pgfqpoint{-0.020833in}{-0.020833in}}{\pgfqpoint{0.020833in}{0.020833in}}{%
\pgfpathmoveto{\pgfqpoint{0.000000in}{-0.020833in}}%
\pgfpathcurveto{\pgfqpoint{0.005525in}{-0.020833in}}{\pgfqpoint{0.010825in}{-0.018638in}}{\pgfqpoint{0.014731in}{-0.014731in}}%
\pgfpathcurveto{\pgfqpoint{0.018638in}{-0.010825in}}{\pgfqpoint{0.020833in}{-0.005525in}}{\pgfqpoint{0.020833in}{0.000000in}}%
\pgfpathcurveto{\pgfqpoint{0.020833in}{0.005525in}}{\pgfqpoint{0.018638in}{0.010825in}}{\pgfqpoint{0.014731in}{0.014731in}}%
\pgfpathcurveto{\pgfqpoint{0.010825in}{0.018638in}}{\pgfqpoint{0.005525in}{0.020833in}}{\pgfqpoint{0.000000in}{0.020833in}}%
\pgfpathcurveto{\pgfqpoint{-0.005525in}{0.020833in}}{\pgfqpoint{-0.010825in}{0.018638in}}{\pgfqpoint{-0.014731in}{0.014731in}}%
\pgfpathcurveto{\pgfqpoint{-0.018638in}{0.010825in}}{\pgfqpoint{-0.020833in}{0.005525in}}{\pgfqpoint{-0.020833in}{0.000000in}}%
\pgfpathcurveto{\pgfqpoint{-0.020833in}{-0.005525in}}{\pgfqpoint{-0.018638in}{-0.010825in}}{\pgfqpoint{-0.014731in}{-0.014731in}}%
\pgfpathcurveto{\pgfqpoint{-0.010825in}{-0.018638in}}{\pgfqpoint{-0.005525in}{-0.020833in}}{\pgfqpoint{0.000000in}{-0.020833in}}%
\pgfpathclose%
\pgfusepath{stroke,fill}%
}%
\begin{pgfscope}%
\pgfsys@transformshift{0.881250in}{1.732500in}%
\pgfsys@useobject{currentmarker}{}%
\end{pgfscope}%
\begin{pgfscope}%
\pgfsys@transformshift{1.656250in}{2.155245in}%
\pgfsys@useobject{currentmarker}{}%
\end{pgfscope}%
\begin{pgfscope}%
\pgfsys@transformshift{2.431250in}{2.561731in}%
\pgfsys@useobject{currentmarker}{}%
\end{pgfscope}%
\begin{pgfscope}%
\pgfsys@transformshift{3.206250in}{2.561731in}%
\pgfsys@useobject{currentmarker}{}%
\end{pgfscope}%
\begin{pgfscope}%
\pgfsys@transformshift{3.981250in}{2.155245in}%
\pgfsys@useobject{currentmarker}{}%
\end{pgfscope}%
\end{pgfscope}%
\begin{pgfscope}%
\pgfpathrectangle{\pgfqpoint{0.687500in}{0.385000in}}{\pgfqpoint{4.262500in}{2.695000in}}%
\pgfusepath{clip}%
\pgfsetrectcap%
\pgfsetroundjoin%
\pgfsetlinewidth{1.505625pt}%
\definecolor{currentstroke}{rgb}{0.580392,0.403922,0.741176}%
\pgfsetstrokecolor{currentstroke}%
\pgfsetdash{}{0pt}%
\pgfpathmoveto{\pgfqpoint{0.881250in}{1.732500in}}%
\pgfpathlineto{\pgfqpoint{0.920195in}{1.743616in}}%
\pgfpathlineto{\pgfqpoint{0.959139in}{1.756458in}}%
\pgfpathlineto{\pgfqpoint{1.017557in}{1.778694in}}%
\pgfpathlineto{\pgfqpoint{1.075974in}{1.804147in}}%
\pgfpathlineto{\pgfqpoint{1.134391in}{1.832436in}}%
\pgfpathlineto{\pgfqpoint{1.192808in}{1.863190in}}%
\pgfpathlineto{\pgfqpoint{1.251225in}{1.896054in}}%
\pgfpathlineto{\pgfqpoint{1.329114in}{1.942560in}}%
\pgfpathlineto{\pgfqpoint{1.426476in}{2.003919in}}%
\pgfpathlineto{\pgfqpoint{1.562783in}{2.093295in}}%
\pgfpathlineto{\pgfqpoint{1.776979in}{2.234080in}}%
\pgfpathlineto{\pgfqpoint{1.874340in}{2.295370in}}%
\pgfpathlineto{\pgfqpoint{1.952230in}{2.342203in}}%
\pgfpathlineto{\pgfqpoint{2.030119in}{2.386575in}}%
\pgfpathlineto{\pgfqpoint{2.108009in}{2.428065in}}%
\pgfpathlineto{\pgfqpoint{2.166426in}{2.457058in}}%
\pgfpathlineto{\pgfqpoint{2.224843in}{2.484066in}}%
\pgfpathlineto{\pgfqpoint{2.283260in}{2.508955in}}%
\pgfpathlineto{\pgfqpoint{2.341677in}{2.531603in}}%
\pgfpathlineto{\pgfqpoint{2.400094in}{2.551900in}}%
\pgfpathlineto{\pgfqpoint{2.458511in}{2.569751in}}%
\pgfpathlineto{\pgfqpoint{2.516928in}{2.585071in}}%
\pgfpathlineto{\pgfqpoint{2.575345in}{2.597788in}}%
\pgfpathlineto{\pgfqpoint{2.633763in}{2.607845in}}%
\pgfpathlineto{\pgfqpoint{2.692180in}{2.615194in}}%
\pgfpathlineto{\pgfqpoint{2.750597in}{2.619802in}}%
\pgfpathlineto{\pgfqpoint{2.809014in}{2.621649in}}%
\pgfpathlineto{\pgfqpoint{2.867431in}{2.620725in}}%
\pgfpathlineto{\pgfqpoint{2.925848in}{2.617036in}}%
\pgfpathlineto{\pgfqpoint{2.984265in}{2.610597in}}%
\pgfpathlineto{\pgfqpoint{3.042682in}{2.601438in}}%
\pgfpathlineto{\pgfqpoint{3.101099in}{2.589602in}}%
\pgfpathlineto{\pgfqpoint{3.159516in}{2.575142in}}%
\pgfpathlineto{\pgfqpoint{3.217933in}{2.558127in}}%
\pgfpathlineto{\pgfqpoint{3.276351in}{2.538635in}}%
\pgfpathlineto{\pgfqpoint{3.334768in}{2.516759in}}%
\pgfpathlineto{\pgfqpoint{3.393185in}{2.492604in}}%
\pgfpathlineto{\pgfqpoint{3.451602in}{2.466288in}}%
\pgfpathlineto{\pgfqpoint{3.510019in}{2.437941in}}%
\pgfpathlineto{\pgfqpoint{3.587908in}{2.397234in}}%
\pgfpathlineto{\pgfqpoint{3.665798in}{2.353544in}}%
\pgfpathlineto{\pgfqpoint{3.743687in}{2.307285in}}%
\pgfpathlineto{\pgfqpoint{3.841049in}{2.246541in}}%
\pgfpathlineto{\pgfqpoint{3.957883in}{2.170663in}}%
\pgfpathlineto{\pgfqpoint{4.249969in}{1.979027in}}%
\pgfpathlineto{\pgfqpoint{4.327858in}{1.930683in}}%
\pgfpathlineto{\pgfqpoint{4.405747in}{1.884887in}}%
\pgfpathlineto{\pgfqpoint{4.464165in}{1.852687in}}%
\pgfpathlineto{\pgfqpoint{4.522582in}{1.822714in}}%
\pgfpathlineto{\pgfqpoint{4.580999in}{1.795329in}}%
\pgfpathlineto{\pgfqpoint{4.639416in}{1.770905in}}%
\pgfpathlineto{\pgfqpoint{4.697833in}{1.749829in}}%
\pgfpathlineto{\pgfqpoint{4.736778in}{1.737835in}}%
\pgfpathlineto{\pgfqpoint{4.756250in}{1.732500in}}%
\pgfpathlineto{\pgfqpoint{4.756250in}{1.732500in}}%
\pgfusepath{stroke}%
\end{pgfscope}%
\begin{pgfscope}%
\pgfsetrectcap%
\pgfsetmiterjoin%
\pgfsetlinewidth{0.803000pt}%
\definecolor{currentstroke}{rgb}{0.000000,0.000000,0.000000}%
\pgfsetstrokecolor{currentstroke}%
\pgfsetdash{}{0pt}%
\pgfpathmoveto{\pgfqpoint{0.687500in}{0.385000in}}%
\pgfpathlineto{\pgfqpoint{0.687500in}{3.080000in}}%
\pgfusepath{stroke}%
\end{pgfscope}%
\begin{pgfscope}%
\pgfsetrectcap%
\pgfsetmiterjoin%
\pgfsetlinewidth{0.803000pt}%
\definecolor{currentstroke}{rgb}{0.000000,0.000000,0.000000}%
\pgfsetstrokecolor{currentstroke}%
\pgfsetdash{}{0pt}%
\pgfpathmoveto{\pgfqpoint{4.950000in}{0.385000in}}%
\pgfpathlineto{\pgfqpoint{4.950000in}{3.080000in}}%
\pgfusepath{stroke}%
\end{pgfscope}%
\begin{pgfscope}%
\pgfsetrectcap%
\pgfsetmiterjoin%
\pgfsetlinewidth{0.803000pt}%
\definecolor{currentstroke}{rgb}{0.000000,0.000000,0.000000}%
\pgfsetstrokecolor{currentstroke}%
\pgfsetdash{}{0pt}%
\pgfpathmoveto{\pgfqpoint{0.687500in}{0.385000in}}%
\pgfpathlineto{\pgfqpoint{4.950000in}{0.385000in}}%
\pgfusepath{stroke}%
\end{pgfscope}%
\begin{pgfscope}%
\pgfsetrectcap%
\pgfsetmiterjoin%
\pgfsetlinewidth{0.803000pt}%
\definecolor{currentstroke}{rgb}{0.000000,0.000000,0.000000}%
\pgfsetstrokecolor{currentstroke}%
\pgfsetdash{}{0pt}%
\pgfpathmoveto{\pgfqpoint{0.687500in}{3.080000in}}%
\pgfpathlineto{\pgfqpoint{4.950000in}{3.080000in}}%
\pgfusepath{stroke}%
\end{pgfscope}%
\begin{pgfscope}%
\definecolor{textcolor}{rgb}{0.000000,0.000000,0.000000}%
\pgfsetstrokecolor{textcolor}%
\pgfsetfillcolor{textcolor}%
\pgftext[x=2.818750in,y=3.163333in,,base]{\color{textcolor}\rmfamily\fontsize{12.000000}{14.400000}\selectfont N=2, 4}%
\end{pgfscope}%
\begin{pgfscope}%
\pgfsetbuttcap%
\pgfsetmiterjoin%
\definecolor{currentfill}{rgb}{1.000000,1.000000,1.000000}%
\pgfsetfillcolor{currentfill}%
\pgfsetfillopacity{0.800000}%
\pgfsetlinewidth{1.003750pt}%
\definecolor{currentstroke}{rgb}{0.800000,0.800000,0.800000}%
\pgfsetstrokecolor{currentstroke}%
\pgfsetstrokeopacity{0.800000}%
\pgfsetdash{}{0pt}%
\pgfpathmoveto{\pgfqpoint{0.784722in}{0.454444in}}%
\pgfpathlineto{\pgfqpoint{2.050468in}{0.454444in}}%
\pgfpathquadraticcurveto{\pgfqpoint{2.078246in}{0.454444in}}{\pgfqpoint{2.078246in}{0.482222in}}%
\pgfpathlineto{\pgfqpoint{2.078246in}{1.244755in}}%
\pgfpathquadraticcurveto{\pgfqpoint{2.078246in}{1.272533in}}{\pgfqpoint{2.050468in}{1.272533in}}%
\pgfpathlineto{\pgfqpoint{0.784722in}{1.272533in}}%
\pgfpathquadraticcurveto{\pgfqpoint{0.756944in}{1.272533in}}{\pgfqpoint{0.756944in}{1.244755in}}%
\pgfpathlineto{\pgfqpoint{0.756944in}{0.482222in}}%
\pgfpathquadraticcurveto{\pgfqpoint{0.756944in}{0.454444in}}{\pgfqpoint{0.784722in}{0.454444in}}%
\pgfpathclose%
\pgfusepath{stroke,fill}%
\end{pgfscope}%
\begin{pgfscope}%
\pgfsetrectcap%
\pgfsetroundjoin%
\pgfsetlinewidth{1.505625pt}%
\definecolor{currentstroke}{rgb}{0.121569,0.466667,0.705882}%
\pgfsetstrokecolor{currentstroke}%
\pgfsetdash{}{0pt}%
\pgfpathmoveto{\pgfqpoint{0.812500in}{1.082123in}}%
\pgfpathlineto{\pgfqpoint{1.090278in}{1.082123in}}%
\pgfusepath{stroke}%
\end{pgfscope}%
\begin{pgfscope}%
\definecolor{textcolor}{rgb}{0.000000,0.000000,0.000000}%
\pgfsetstrokecolor{textcolor}%
\pgfsetfillcolor{textcolor}%
\pgftext[x=1.201389in,y=1.033512in,left,base]{\color{textcolor}\rmfamily\fontsize{10.000000}{12.000000}\selectfont \(\displaystyle y(x)=\)\(\displaystyle \frac{1}{1+x^{2}}\)}%
\end{pgfscope}%
\begin{pgfscope}%
\pgfsetrectcap%
\pgfsetroundjoin%
\pgfsetlinewidth{1.505625pt}%
\definecolor{currentstroke}{rgb}{0.172549,0.627451,0.172549}%
\pgfsetstrokecolor{currentstroke}%
\pgfsetdash{}{0pt}%
\pgfpathmoveto{\pgfqpoint{0.812500in}{0.801667in}}%
\pgfpathlineto{\pgfqpoint{1.090278in}{0.801667in}}%
\pgfusepath{stroke}%
\end{pgfscope}%
\begin{pgfscope}%
\definecolor{textcolor}{rgb}{0.000000,0.000000,0.000000}%
\pgfsetstrokecolor{textcolor}%
\pgfsetfillcolor{textcolor}%
\pgftext[x=1.201389in,y=0.753056in,left,base]{\color{textcolor}\rmfamily\fontsize{10.000000}{12.000000}\selectfont W2(x)}%
\end{pgfscope}%
\begin{pgfscope}%
\pgfsetrectcap%
\pgfsetroundjoin%
\pgfsetlinewidth{1.505625pt}%
\definecolor{currentstroke}{rgb}{0.580392,0.403922,0.741176}%
\pgfsetstrokecolor{currentstroke}%
\pgfsetdash{}{0pt}%
\pgfpathmoveto{\pgfqpoint{0.812500in}{0.593333in}}%
\pgfpathlineto{\pgfqpoint{1.090278in}{0.593333in}}%
\pgfusepath{stroke}%
\end{pgfscope}%
\begin{pgfscope}%
\definecolor{textcolor}{rgb}{0.000000,0.000000,0.000000}%
\pgfsetstrokecolor{textcolor}%
\pgfsetfillcolor{textcolor}%
\pgftext[x=1.201389in,y=0.544722in,left,base]{\color{textcolor}\rmfamily\fontsize{10.000000}{12.000000}\selectfont W4(x)}%
\end{pgfscope}%
\end{pgfpicture}%
\makeatother%
\endgroup%
        
    \end{center}
    \caption{Węzły jednorodne, funkcja \(\tilde{y}\), \(N=2,4\)}
\end{figure}

\begin{figure}[h]
    \begin{center}
        %% Creator: Matplotlib, PGF backend
%%
%% To include the figure in your LaTeX document, write
%%   \input{<filename>.pgf}
%%
%% Make sure the required packages are loaded in your preamble
%%   \usepackage{pgf}
%%
%% Figures using additional raster images can only be included by \input if
%% they are in the same directory as the main LaTeX file. For loading figures
%% from other directories you can use the `import` package
%%   \usepackage{import}
%% and then include the figures with
%%   \import{<path to file>}{<filename>.pgf}
%%
%% Matplotlib used the following preamble
%%
\begingroup%
\makeatletter%
\begin{pgfpicture}%
\pgfpathrectangle{\pgfpointorigin}{\pgfqpoint{5.500000in}{3.500000in}}%
\pgfusepath{use as bounding box, clip}%
\begin{pgfscope}%
\pgfsetbuttcap%
\pgfsetmiterjoin%
\definecolor{currentfill}{rgb}{1.000000,1.000000,1.000000}%
\pgfsetfillcolor{currentfill}%
\pgfsetlinewidth{0.000000pt}%
\definecolor{currentstroke}{rgb}{1.000000,1.000000,1.000000}%
\pgfsetstrokecolor{currentstroke}%
\pgfsetdash{}{0pt}%
\pgfpathmoveto{\pgfqpoint{0.000000in}{0.000000in}}%
\pgfpathlineto{\pgfqpoint{5.500000in}{0.000000in}}%
\pgfpathlineto{\pgfqpoint{5.500000in}{3.500000in}}%
\pgfpathlineto{\pgfqpoint{0.000000in}{3.500000in}}%
\pgfpathclose%
\pgfusepath{fill}%
\end{pgfscope}%
\begin{pgfscope}%
\pgfsetbuttcap%
\pgfsetmiterjoin%
\definecolor{currentfill}{rgb}{1.000000,1.000000,1.000000}%
\pgfsetfillcolor{currentfill}%
\pgfsetlinewidth{0.000000pt}%
\definecolor{currentstroke}{rgb}{0.000000,0.000000,0.000000}%
\pgfsetstrokecolor{currentstroke}%
\pgfsetstrokeopacity{0.000000}%
\pgfsetdash{}{0pt}%
\pgfpathmoveto{\pgfqpoint{0.687500in}{0.385000in}}%
\pgfpathlineto{\pgfqpoint{4.950000in}{0.385000in}}%
\pgfpathlineto{\pgfqpoint{4.950000in}{3.080000in}}%
\pgfpathlineto{\pgfqpoint{0.687500in}{3.080000in}}%
\pgfpathclose%
\pgfusepath{fill}%
\end{pgfscope}%
\begin{pgfscope}%
\pgfsetbuttcap%
\pgfsetroundjoin%
\definecolor{currentfill}{rgb}{0.000000,0.000000,0.000000}%
\pgfsetfillcolor{currentfill}%
\pgfsetlinewidth{0.803000pt}%
\definecolor{currentstroke}{rgb}{0.000000,0.000000,0.000000}%
\pgfsetstrokecolor{currentstroke}%
\pgfsetdash{}{0pt}%
\pgfsys@defobject{currentmarker}{\pgfqpoint{0.000000in}{-0.048611in}}{\pgfqpoint{0.000000in}{0.000000in}}{%
\pgfpathmoveto{\pgfqpoint{0.000000in}{0.000000in}}%
\pgfpathlineto{\pgfqpoint{0.000000in}{-0.048611in}}%
\pgfusepath{stroke,fill}%
}%
\begin{pgfscope}%
\pgfsys@transformshift{0.881250in}{0.385000in}%
\pgfsys@useobject{currentmarker}{}%
\end{pgfscope}%
\end{pgfscope}%
\begin{pgfscope}%
\definecolor{textcolor}{rgb}{0.000000,0.000000,0.000000}%
\pgfsetstrokecolor{textcolor}%
\pgfsetfillcolor{textcolor}%
\pgftext[x=0.881250in,y=0.287778in,,top]{\color{textcolor}\rmfamily\fontsize{10.000000}{12.000000}\selectfont \(\displaystyle -1.00\)}%
\end{pgfscope}%
\begin{pgfscope}%
\pgfsetbuttcap%
\pgfsetroundjoin%
\definecolor{currentfill}{rgb}{0.000000,0.000000,0.000000}%
\pgfsetfillcolor{currentfill}%
\pgfsetlinewidth{0.803000pt}%
\definecolor{currentstroke}{rgb}{0.000000,0.000000,0.000000}%
\pgfsetstrokecolor{currentstroke}%
\pgfsetdash{}{0pt}%
\pgfsys@defobject{currentmarker}{\pgfqpoint{0.000000in}{-0.048611in}}{\pgfqpoint{0.000000in}{0.000000in}}{%
\pgfpathmoveto{\pgfqpoint{0.000000in}{0.000000in}}%
\pgfpathlineto{\pgfqpoint{0.000000in}{-0.048611in}}%
\pgfusepath{stroke,fill}%
}%
\begin{pgfscope}%
\pgfsys@transformshift{1.365625in}{0.385000in}%
\pgfsys@useobject{currentmarker}{}%
\end{pgfscope}%
\end{pgfscope}%
\begin{pgfscope}%
\definecolor{textcolor}{rgb}{0.000000,0.000000,0.000000}%
\pgfsetstrokecolor{textcolor}%
\pgfsetfillcolor{textcolor}%
\pgftext[x=1.365625in,y=0.287778in,,top]{\color{textcolor}\rmfamily\fontsize{10.000000}{12.000000}\selectfont \(\displaystyle -0.75\)}%
\end{pgfscope}%
\begin{pgfscope}%
\pgfsetbuttcap%
\pgfsetroundjoin%
\definecolor{currentfill}{rgb}{0.000000,0.000000,0.000000}%
\pgfsetfillcolor{currentfill}%
\pgfsetlinewidth{0.803000pt}%
\definecolor{currentstroke}{rgb}{0.000000,0.000000,0.000000}%
\pgfsetstrokecolor{currentstroke}%
\pgfsetdash{}{0pt}%
\pgfsys@defobject{currentmarker}{\pgfqpoint{0.000000in}{-0.048611in}}{\pgfqpoint{0.000000in}{0.000000in}}{%
\pgfpathmoveto{\pgfqpoint{0.000000in}{0.000000in}}%
\pgfpathlineto{\pgfqpoint{0.000000in}{-0.048611in}}%
\pgfusepath{stroke,fill}%
}%
\begin{pgfscope}%
\pgfsys@transformshift{1.850000in}{0.385000in}%
\pgfsys@useobject{currentmarker}{}%
\end{pgfscope}%
\end{pgfscope}%
\begin{pgfscope}%
\definecolor{textcolor}{rgb}{0.000000,0.000000,0.000000}%
\pgfsetstrokecolor{textcolor}%
\pgfsetfillcolor{textcolor}%
\pgftext[x=1.850000in,y=0.287778in,,top]{\color{textcolor}\rmfamily\fontsize{10.000000}{12.000000}\selectfont \(\displaystyle -0.50\)}%
\end{pgfscope}%
\begin{pgfscope}%
\pgfsetbuttcap%
\pgfsetroundjoin%
\definecolor{currentfill}{rgb}{0.000000,0.000000,0.000000}%
\pgfsetfillcolor{currentfill}%
\pgfsetlinewidth{0.803000pt}%
\definecolor{currentstroke}{rgb}{0.000000,0.000000,0.000000}%
\pgfsetstrokecolor{currentstroke}%
\pgfsetdash{}{0pt}%
\pgfsys@defobject{currentmarker}{\pgfqpoint{0.000000in}{-0.048611in}}{\pgfqpoint{0.000000in}{0.000000in}}{%
\pgfpathmoveto{\pgfqpoint{0.000000in}{0.000000in}}%
\pgfpathlineto{\pgfqpoint{0.000000in}{-0.048611in}}%
\pgfusepath{stroke,fill}%
}%
\begin{pgfscope}%
\pgfsys@transformshift{2.334375in}{0.385000in}%
\pgfsys@useobject{currentmarker}{}%
\end{pgfscope}%
\end{pgfscope}%
\begin{pgfscope}%
\definecolor{textcolor}{rgb}{0.000000,0.000000,0.000000}%
\pgfsetstrokecolor{textcolor}%
\pgfsetfillcolor{textcolor}%
\pgftext[x=2.334375in,y=0.287778in,,top]{\color{textcolor}\rmfamily\fontsize{10.000000}{12.000000}\selectfont \(\displaystyle -0.25\)}%
\end{pgfscope}%
\begin{pgfscope}%
\pgfsetbuttcap%
\pgfsetroundjoin%
\definecolor{currentfill}{rgb}{0.000000,0.000000,0.000000}%
\pgfsetfillcolor{currentfill}%
\pgfsetlinewidth{0.803000pt}%
\definecolor{currentstroke}{rgb}{0.000000,0.000000,0.000000}%
\pgfsetstrokecolor{currentstroke}%
\pgfsetdash{}{0pt}%
\pgfsys@defobject{currentmarker}{\pgfqpoint{0.000000in}{-0.048611in}}{\pgfqpoint{0.000000in}{0.000000in}}{%
\pgfpathmoveto{\pgfqpoint{0.000000in}{0.000000in}}%
\pgfpathlineto{\pgfqpoint{0.000000in}{-0.048611in}}%
\pgfusepath{stroke,fill}%
}%
\begin{pgfscope}%
\pgfsys@transformshift{2.818750in}{0.385000in}%
\pgfsys@useobject{currentmarker}{}%
\end{pgfscope}%
\end{pgfscope}%
\begin{pgfscope}%
\definecolor{textcolor}{rgb}{0.000000,0.000000,0.000000}%
\pgfsetstrokecolor{textcolor}%
\pgfsetfillcolor{textcolor}%
\pgftext[x=2.818750in,y=0.287778in,,top]{\color{textcolor}\rmfamily\fontsize{10.000000}{12.000000}\selectfont \(\displaystyle 0.00\)}%
\end{pgfscope}%
\begin{pgfscope}%
\pgfsetbuttcap%
\pgfsetroundjoin%
\definecolor{currentfill}{rgb}{0.000000,0.000000,0.000000}%
\pgfsetfillcolor{currentfill}%
\pgfsetlinewidth{0.803000pt}%
\definecolor{currentstroke}{rgb}{0.000000,0.000000,0.000000}%
\pgfsetstrokecolor{currentstroke}%
\pgfsetdash{}{0pt}%
\pgfsys@defobject{currentmarker}{\pgfqpoint{0.000000in}{-0.048611in}}{\pgfqpoint{0.000000in}{0.000000in}}{%
\pgfpathmoveto{\pgfqpoint{0.000000in}{0.000000in}}%
\pgfpathlineto{\pgfqpoint{0.000000in}{-0.048611in}}%
\pgfusepath{stroke,fill}%
}%
\begin{pgfscope}%
\pgfsys@transformshift{3.303125in}{0.385000in}%
\pgfsys@useobject{currentmarker}{}%
\end{pgfscope}%
\end{pgfscope}%
\begin{pgfscope}%
\definecolor{textcolor}{rgb}{0.000000,0.000000,0.000000}%
\pgfsetstrokecolor{textcolor}%
\pgfsetfillcolor{textcolor}%
\pgftext[x=3.303125in,y=0.287778in,,top]{\color{textcolor}\rmfamily\fontsize{10.000000}{12.000000}\selectfont \(\displaystyle 0.25\)}%
\end{pgfscope}%
\begin{pgfscope}%
\pgfsetbuttcap%
\pgfsetroundjoin%
\definecolor{currentfill}{rgb}{0.000000,0.000000,0.000000}%
\pgfsetfillcolor{currentfill}%
\pgfsetlinewidth{0.803000pt}%
\definecolor{currentstroke}{rgb}{0.000000,0.000000,0.000000}%
\pgfsetstrokecolor{currentstroke}%
\pgfsetdash{}{0pt}%
\pgfsys@defobject{currentmarker}{\pgfqpoint{0.000000in}{-0.048611in}}{\pgfqpoint{0.000000in}{0.000000in}}{%
\pgfpathmoveto{\pgfqpoint{0.000000in}{0.000000in}}%
\pgfpathlineto{\pgfqpoint{0.000000in}{-0.048611in}}%
\pgfusepath{stroke,fill}%
}%
\begin{pgfscope}%
\pgfsys@transformshift{3.787500in}{0.385000in}%
\pgfsys@useobject{currentmarker}{}%
\end{pgfscope}%
\end{pgfscope}%
\begin{pgfscope}%
\definecolor{textcolor}{rgb}{0.000000,0.000000,0.000000}%
\pgfsetstrokecolor{textcolor}%
\pgfsetfillcolor{textcolor}%
\pgftext[x=3.787500in,y=0.287778in,,top]{\color{textcolor}\rmfamily\fontsize{10.000000}{12.000000}\selectfont \(\displaystyle 0.50\)}%
\end{pgfscope}%
\begin{pgfscope}%
\pgfsetbuttcap%
\pgfsetroundjoin%
\definecolor{currentfill}{rgb}{0.000000,0.000000,0.000000}%
\pgfsetfillcolor{currentfill}%
\pgfsetlinewidth{0.803000pt}%
\definecolor{currentstroke}{rgb}{0.000000,0.000000,0.000000}%
\pgfsetstrokecolor{currentstroke}%
\pgfsetdash{}{0pt}%
\pgfsys@defobject{currentmarker}{\pgfqpoint{0.000000in}{-0.048611in}}{\pgfqpoint{0.000000in}{0.000000in}}{%
\pgfpathmoveto{\pgfqpoint{0.000000in}{0.000000in}}%
\pgfpathlineto{\pgfqpoint{0.000000in}{-0.048611in}}%
\pgfusepath{stroke,fill}%
}%
\begin{pgfscope}%
\pgfsys@transformshift{4.271875in}{0.385000in}%
\pgfsys@useobject{currentmarker}{}%
\end{pgfscope}%
\end{pgfscope}%
\begin{pgfscope}%
\definecolor{textcolor}{rgb}{0.000000,0.000000,0.000000}%
\pgfsetstrokecolor{textcolor}%
\pgfsetfillcolor{textcolor}%
\pgftext[x=4.271875in,y=0.287778in,,top]{\color{textcolor}\rmfamily\fontsize{10.000000}{12.000000}\selectfont \(\displaystyle 0.75\)}%
\end{pgfscope}%
\begin{pgfscope}%
\pgfsetbuttcap%
\pgfsetroundjoin%
\definecolor{currentfill}{rgb}{0.000000,0.000000,0.000000}%
\pgfsetfillcolor{currentfill}%
\pgfsetlinewidth{0.803000pt}%
\definecolor{currentstroke}{rgb}{0.000000,0.000000,0.000000}%
\pgfsetstrokecolor{currentstroke}%
\pgfsetdash{}{0pt}%
\pgfsys@defobject{currentmarker}{\pgfqpoint{0.000000in}{-0.048611in}}{\pgfqpoint{0.000000in}{0.000000in}}{%
\pgfpathmoveto{\pgfqpoint{0.000000in}{0.000000in}}%
\pgfpathlineto{\pgfqpoint{0.000000in}{-0.048611in}}%
\pgfusepath{stroke,fill}%
}%
\begin{pgfscope}%
\pgfsys@transformshift{4.756250in}{0.385000in}%
\pgfsys@useobject{currentmarker}{}%
\end{pgfscope}%
\end{pgfscope}%
\begin{pgfscope}%
\definecolor{textcolor}{rgb}{0.000000,0.000000,0.000000}%
\pgfsetstrokecolor{textcolor}%
\pgfsetfillcolor{textcolor}%
\pgftext[x=4.756250in,y=0.287778in,,top]{\color{textcolor}\rmfamily\fontsize{10.000000}{12.000000}\selectfont \(\displaystyle 1.00\)}%
\end{pgfscope}%
\begin{pgfscope}%
\definecolor{textcolor}{rgb}{0.000000,0.000000,0.000000}%
\pgfsetstrokecolor{textcolor}%
\pgfsetfillcolor{textcolor}%
\pgftext[x=2.818750in,y=0.108766in,,top]{\color{textcolor}\rmfamily\fontsize{10.000000}{12.000000}\selectfont x}%
\end{pgfscope}%
\begin{pgfscope}%
\pgfsetbuttcap%
\pgfsetroundjoin%
\definecolor{currentfill}{rgb}{0.000000,0.000000,0.000000}%
\pgfsetfillcolor{currentfill}%
\pgfsetlinewidth{0.803000pt}%
\definecolor{currentstroke}{rgb}{0.000000,0.000000,0.000000}%
\pgfsetstrokecolor{currentstroke}%
\pgfsetdash{}{0pt}%
\pgfsys@defobject{currentmarker}{\pgfqpoint{-0.048611in}{0.000000in}}{\pgfqpoint{0.000000in}{0.000000in}}{%
\pgfpathmoveto{\pgfqpoint{0.000000in}{0.000000in}}%
\pgfpathlineto{\pgfqpoint{-0.048611in}{0.000000in}}%
\pgfusepath{stroke,fill}%
}%
\begin{pgfscope}%
\pgfsys@transformshift{0.687500in}{0.474833in}%
\pgfsys@useobject{currentmarker}{}%
\end{pgfscope}%
\end{pgfscope}%
\begin{pgfscope}%
\definecolor{textcolor}{rgb}{0.000000,0.000000,0.000000}%
\pgfsetstrokecolor{textcolor}%
\pgfsetfillcolor{textcolor}%
\pgftext[x=0.304783in,y=0.426608in,left,base]{\color{textcolor}\rmfamily\fontsize{10.000000}{12.000000}\selectfont \(\displaystyle -0.2\)}%
\end{pgfscope}%
\begin{pgfscope}%
\pgfsetbuttcap%
\pgfsetroundjoin%
\definecolor{currentfill}{rgb}{0.000000,0.000000,0.000000}%
\pgfsetfillcolor{currentfill}%
\pgfsetlinewidth{0.803000pt}%
\definecolor{currentstroke}{rgb}{0.000000,0.000000,0.000000}%
\pgfsetstrokecolor{currentstroke}%
\pgfsetdash{}{0pt}%
\pgfsys@defobject{currentmarker}{\pgfqpoint{-0.048611in}{0.000000in}}{\pgfqpoint{0.000000in}{0.000000in}}{%
\pgfpathmoveto{\pgfqpoint{0.000000in}{0.000000in}}%
\pgfpathlineto{\pgfqpoint{-0.048611in}{0.000000in}}%
\pgfusepath{stroke,fill}%
}%
\begin{pgfscope}%
\pgfsys@transformshift{0.687500in}{0.834167in}%
\pgfsys@useobject{currentmarker}{}%
\end{pgfscope}%
\end{pgfscope}%
\begin{pgfscope}%
\definecolor{textcolor}{rgb}{0.000000,0.000000,0.000000}%
\pgfsetstrokecolor{textcolor}%
\pgfsetfillcolor{textcolor}%
\pgftext[x=0.412808in,y=0.785941in,left,base]{\color{textcolor}\rmfamily\fontsize{10.000000}{12.000000}\selectfont \(\displaystyle 0.0\)}%
\end{pgfscope}%
\begin{pgfscope}%
\pgfsetbuttcap%
\pgfsetroundjoin%
\definecolor{currentfill}{rgb}{0.000000,0.000000,0.000000}%
\pgfsetfillcolor{currentfill}%
\pgfsetlinewidth{0.803000pt}%
\definecolor{currentstroke}{rgb}{0.000000,0.000000,0.000000}%
\pgfsetstrokecolor{currentstroke}%
\pgfsetdash{}{0pt}%
\pgfsys@defobject{currentmarker}{\pgfqpoint{-0.048611in}{0.000000in}}{\pgfqpoint{0.000000in}{0.000000in}}{%
\pgfpathmoveto{\pgfqpoint{0.000000in}{0.000000in}}%
\pgfpathlineto{\pgfqpoint{-0.048611in}{0.000000in}}%
\pgfusepath{stroke,fill}%
}%
\begin{pgfscope}%
\pgfsys@transformshift{0.687500in}{1.193500in}%
\pgfsys@useobject{currentmarker}{}%
\end{pgfscope}%
\end{pgfscope}%
\begin{pgfscope}%
\definecolor{textcolor}{rgb}{0.000000,0.000000,0.000000}%
\pgfsetstrokecolor{textcolor}%
\pgfsetfillcolor{textcolor}%
\pgftext[x=0.412808in,y=1.145275in,left,base]{\color{textcolor}\rmfamily\fontsize{10.000000}{12.000000}\selectfont \(\displaystyle 0.2\)}%
\end{pgfscope}%
\begin{pgfscope}%
\pgfsetbuttcap%
\pgfsetroundjoin%
\definecolor{currentfill}{rgb}{0.000000,0.000000,0.000000}%
\pgfsetfillcolor{currentfill}%
\pgfsetlinewidth{0.803000pt}%
\definecolor{currentstroke}{rgb}{0.000000,0.000000,0.000000}%
\pgfsetstrokecolor{currentstroke}%
\pgfsetdash{}{0pt}%
\pgfsys@defobject{currentmarker}{\pgfqpoint{-0.048611in}{0.000000in}}{\pgfqpoint{0.000000in}{0.000000in}}{%
\pgfpathmoveto{\pgfqpoint{0.000000in}{0.000000in}}%
\pgfpathlineto{\pgfqpoint{-0.048611in}{0.000000in}}%
\pgfusepath{stroke,fill}%
}%
\begin{pgfscope}%
\pgfsys@transformshift{0.687500in}{1.552833in}%
\pgfsys@useobject{currentmarker}{}%
\end{pgfscope}%
\end{pgfscope}%
\begin{pgfscope}%
\definecolor{textcolor}{rgb}{0.000000,0.000000,0.000000}%
\pgfsetstrokecolor{textcolor}%
\pgfsetfillcolor{textcolor}%
\pgftext[x=0.412808in,y=1.504608in,left,base]{\color{textcolor}\rmfamily\fontsize{10.000000}{12.000000}\selectfont \(\displaystyle 0.4\)}%
\end{pgfscope}%
\begin{pgfscope}%
\pgfsetbuttcap%
\pgfsetroundjoin%
\definecolor{currentfill}{rgb}{0.000000,0.000000,0.000000}%
\pgfsetfillcolor{currentfill}%
\pgfsetlinewidth{0.803000pt}%
\definecolor{currentstroke}{rgb}{0.000000,0.000000,0.000000}%
\pgfsetstrokecolor{currentstroke}%
\pgfsetdash{}{0pt}%
\pgfsys@defobject{currentmarker}{\pgfqpoint{-0.048611in}{0.000000in}}{\pgfqpoint{0.000000in}{0.000000in}}{%
\pgfpathmoveto{\pgfqpoint{0.000000in}{0.000000in}}%
\pgfpathlineto{\pgfqpoint{-0.048611in}{0.000000in}}%
\pgfusepath{stroke,fill}%
}%
\begin{pgfscope}%
\pgfsys@transformshift{0.687500in}{1.912167in}%
\pgfsys@useobject{currentmarker}{}%
\end{pgfscope}%
\end{pgfscope}%
\begin{pgfscope}%
\definecolor{textcolor}{rgb}{0.000000,0.000000,0.000000}%
\pgfsetstrokecolor{textcolor}%
\pgfsetfillcolor{textcolor}%
\pgftext[x=0.412808in,y=1.863941in,left,base]{\color{textcolor}\rmfamily\fontsize{10.000000}{12.000000}\selectfont \(\displaystyle 0.6\)}%
\end{pgfscope}%
\begin{pgfscope}%
\pgfsetbuttcap%
\pgfsetroundjoin%
\definecolor{currentfill}{rgb}{0.000000,0.000000,0.000000}%
\pgfsetfillcolor{currentfill}%
\pgfsetlinewidth{0.803000pt}%
\definecolor{currentstroke}{rgb}{0.000000,0.000000,0.000000}%
\pgfsetstrokecolor{currentstroke}%
\pgfsetdash{}{0pt}%
\pgfsys@defobject{currentmarker}{\pgfqpoint{-0.048611in}{0.000000in}}{\pgfqpoint{0.000000in}{0.000000in}}{%
\pgfpathmoveto{\pgfqpoint{0.000000in}{0.000000in}}%
\pgfpathlineto{\pgfqpoint{-0.048611in}{0.000000in}}%
\pgfusepath{stroke,fill}%
}%
\begin{pgfscope}%
\pgfsys@transformshift{0.687500in}{2.271500in}%
\pgfsys@useobject{currentmarker}{}%
\end{pgfscope}%
\end{pgfscope}%
\begin{pgfscope}%
\definecolor{textcolor}{rgb}{0.000000,0.000000,0.000000}%
\pgfsetstrokecolor{textcolor}%
\pgfsetfillcolor{textcolor}%
\pgftext[x=0.412808in,y=2.223275in,left,base]{\color{textcolor}\rmfamily\fontsize{10.000000}{12.000000}\selectfont \(\displaystyle 0.8\)}%
\end{pgfscope}%
\begin{pgfscope}%
\pgfsetbuttcap%
\pgfsetroundjoin%
\definecolor{currentfill}{rgb}{0.000000,0.000000,0.000000}%
\pgfsetfillcolor{currentfill}%
\pgfsetlinewidth{0.803000pt}%
\definecolor{currentstroke}{rgb}{0.000000,0.000000,0.000000}%
\pgfsetstrokecolor{currentstroke}%
\pgfsetdash{}{0pt}%
\pgfsys@defobject{currentmarker}{\pgfqpoint{-0.048611in}{0.000000in}}{\pgfqpoint{0.000000in}{0.000000in}}{%
\pgfpathmoveto{\pgfqpoint{0.000000in}{0.000000in}}%
\pgfpathlineto{\pgfqpoint{-0.048611in}{0.000000in}}%
\pgfusepath{stroke,fill}%
}%
\begin{pgfscope}%
\pgfsys@transformshift{0.687500in}{2.630833in}%
\pgfsys@useobject{currentmarker}{}%
\end{pgfscope}%
\end{pgfscope}%
\begin{pgfscope}%
\definecolor{textcolor}{rgb}{0.000000,0.000000,0.000000}%
\pgfsetstrokecolor{textcolor}%
\pgfsetfillcolor{textcolor}%
\pgftext[x=0.412808in,y=2.582608in,left,base]{\color{textcolor}\rmfamily\fontsize{10.000000}{12.000000}\selectfont \(\displaystyle 1.0\)}%
\end{pgfscope}%
\begin{pgfscope}%
\pgfsetbuttcap%
\pgfsetroundjoin%
\definecolor{currentfill}{rgb}{0.000000,0.000000,0.000000}%
\pgfsetfillcolor{currentfill}%
\pgfsetlinewidth{0.803000pt}%
\definecolor{currentstroke}{rgb}{0.000000,0.000000,0.000000}%
\pgfsetstrokecolor{currentstroke}%
\pgfsetdash{}{0pt}%
\pgfsys@defobject{currentmarker}{\pgfqpoint{-0.048611in}{0.000000in}}{\pgfqpoint{0.000000in}{0.000000in}}{%
\pgfpathmoveto{\pgfqpoint{0.000000in}{0.000000in}}%
\pgfpathlineto{\pgfqpoint{-0.048611in}{0.000000in}}%
\pgfusepath{stroke,fill}%
}%
\begin{pgfscope}%
\pgfsys@transformshift{0.687500in}{2.990167in}%
\pgfsys@useobject{currentmarker}{}%
\end{pgfscope}%
\end{pgfscope}%
\begin{pgfscope}%
\definecolor{textcolor}{rgb}{0.000000,0.000000,0.000000}%
\pgfsetstrokecolor{textcolor}%
\pgfsetfillcolor{textcolor}%
\pgftext[x=0.412808in,y=2.941941in,left,base]{\color{textcolor}\rmfamily\fontsize{10.000000}{12.000000}\selectfont \(\displaystyle 1.2\)}%
\end{pgfscope}%
\begin{pgfscope}%
\definecolor{textcolor}{rgb}{0.000000,0.000000,0.000000}%
\pgfsetstrokecolor{textcolor}%
\pgfsetfillcolor{textcolor}%
\pgftext[x=0.249228in,y=1.732500in,,bottom,rotate=90.000000]{\color{textcolor}\rmfamily\fontsize{10.000000}{12.000000}\selectfont y}%
\end{pgfscope}%
\begin{pgfscope}%
\pgfpathrectangle{\pgfqpoint{0.687500in}{0.385000in}}{\pgfqpoint{4.262500in}{2.695000in}}%
\pgfusepath{clip}%
\pgfsetrectcap%
\pgfsetroundjoin%
\pgfsetlinewidth{1.505625pt}%
\definecolor{currentstroke}{rgb}{0.121569,0.466667,0.705882}%
\pgfsetstrokecolor{currentstroke}%
\pgfsetdash{}{0pt}%
\pgfpathmoveto{\pgfqpoint{0.881250in}{1.732500in}}%
\pgfpathlineto{\pgfqpoint{0.978612in}{1.778775in}}%
\pgfpathlineto{\pgfqpoint{1.075974in}{1.827296in}}%
\pgfpathlineto{\pgfqpoint{1.173335in}{1.878000in}}%
\pgfpathlineto{\pgfqpoint{1.290170in}{1.941556in}}%
\pgfpathlineto{\pgfqpoint{1.407004in}{2.007752in}}%
\pgfpathlineto{\pgfqpoint{1.543310in}{2.087643in}}%
\pgfpathlineto{\pgfqpoint{1.796451in}{2.239567in}}%
\pgfpathlineto{\pgfqpoint{1.932758in}{2.320107in}}%
\pgfpathlineto{\pgfqpoint{2.030119in}{2.375473in}}%
\pgfpathlineto{\pgfqpoint{2.108009in}{2.417737in}}%
\pgfpathlineto{\pgfqpoint{2.185898in}{2.457628in}}%
\pgfpathlineto{\pgfqpoint{2.263788in}{2.494605in}}%
\pgfpathlineto{\pgfqpoint{2.322205in}{2.520101in}}%
\pgfpathlineto{\pgfqpoint{2.380622in}{2.543430in}}%
\pgfpathlineto{\pgfqpoint{2.439039in}{2.564379in}}%
\pgfpathlineto{\pgfqpoint{2.497456in}{2.582749in}}%
\pgfpathlineto{\pgfqpoint{2.555873in}{2.598357in}}%
\pgfpathlineto{\pgfqpoint{2.614290in}{2.611046in}}%
\pgfpathlineto{\pgfqpoint{2.672707in}{2.620683in}}%
\pgfpathlineto{\pgfqpoint{2.731124in}{2.627166in}}%
\pgfpathlineto{\pgfqpoint{2.789541in}{2.630425in}}%
\pgfpathlineto{\pgfqpoint{2.847959in}{2.630425in}}%
\pgfpathlineto{\pgfqpoint{2.906376in}{2.627166in}}%
\pgfpathlineto{\pgfqpoint{2.964793in}{2.620683in}}%
\pgfpathlineto{\pgfqpoint{3.023210in}{2.611046in}}%
\pgfpathlineto{\pgfqpoint{3.081627in}{2.598357in}}%
\pgfpathlineto{\pgfqpoint{3.140044in}{2.582749in}}%
\pgfpathlineto{\pgfqpoint{3.198461in}{2.564379in}}%
\pgfpathlineto{\pgfqpoint{3.256878in}{2.543430in}}%
\pgfpathlineto{\pgfqpoint{3.315295in}{2.520101in}}%
\pgfpathlineto{\pgfqpoint{3.373712in}{2.494605in}}%
\pgfpathlineto{\pgfqpoint{3.451602in}{2.457628in}}%
\pgfpathlineto{\pgfqpoint{3.529491in}{2.417737in}}%
\pgfpathlineto{\pgfqpoint{3.607381in}{2.375473in}}%
\pgfpathlineto{\pgfqpoint{3.704742in}{2.320107in}}%
\pgfpathlineto{\pgfqpoint{3.841049in}{2.239567in}}%
\pgfpathlineto{\pgfqpoint{4.211024in}{2.019009in}}%
\pgfpathlineto{\pgfqpoint{4.327858in}{1.952417in}}%
\pgfpathlineto{\pgfqpoint{4.444692in}{1.888394in}}%
\pgfpathlineto{\pgfqpoint{4.542054in}{1.837265in}}%
\pgfpathlineto{\pgfqpoint{4.639416in}{1.788301in}}%
\pgfpathlineto{\pgfqpoint{4.736778in}{1.741574in}}%
\pgfpathlineto{\pgfqpoint{4.756250in}{1.732500in}}%
\pgfpathlineto{\pgfqpoint{4.756250in}{1.732500in}}%
\pgfusepath{stroke}%
\end{pgfscope}%
\begin{pgfscope}%
\pgfpathrectangle{\pgfqpoint{0.687500in}{0.385000in}}{\pgfqpoint{4.262500in}{2.695000in}}%
\pgfusepath{clip}%
\pgfsetbuttcap%
\pgfsetroundjoin%
\definecolor{currentfill}{rgb}{1.000000,0.498039,0.054902}%
\pgfsetfillcolor{currentfill}%
\pgfsetlinewidth{1.003750pt}%
\definecolor{currentstroke}{rgb}{1.000000,0.498039,0.054902}%
\pgfsetstrokecolor{currentstroke}%
\pgfsetdash{}{0pt}%
\pgfsys@defobject{currentmarker}{\pgfqpoint{-0.020833in}{-0.020833in}}{\pgfqpoint{0.020833in}{0.020833in}}{%
\pgfpathmoveto{\pgfqpoint{0.000000in}{-0.020833in}}%
\pgfpathcurveto{\pgfqpoint{0.005525in}{-0.020833in}}{\pgfqpoint{0.010825in}{-0.018638in}}{\pgfqpoint{0.014731in}{-0.014731in}}%
\pgfpathcurveto{\pgfqpoint{0.018638in}{-0.010825in}}{\pgfqpoint{0.020833in}{-0.005525in}}{\pgfqpoint{0.020833in}{0.000000in}}%
\pgfpathcurveto{\pgfqpoint{0.020833in}{0.005525in}}{\pgfqpoint{0.018638in}{0.010825in}}{\pgfqpoint{0.014731in}{0.014731in}}%
\pgfpathcurveto{\pgfqpoint{0.010825in}{0.018638in}}{\pgfqpoint{0.005525in}{0.020833in}}{\pgfqpoint{0.000000in}{0.020833in}}%
\pgfpathcurveto{\pgfqpoint{-0.005525in}{0.020833in}}{\pgfqpoint{-0.010825in}{0.018638in}}{\pgfqpoint{-0.014731in}{0.014731in}}%
\pgfpathcurveto{\pgfqpoint{-0.018638in}{0.010825in}}{\pgfqpoint{-0.020833in}{0.005525in}}{\pgfqpoint{-0.020833in}{0.000000in}}%
\pgfpathcurveto{\pgfqpoint{-0.020833in}{-0.005525in}}{\pgfqpoint{-0.018638in}{-0.010825in}}{\pgfqpoint{-0.014731in}{-0.014731in}}%
\pgfpathcurveto{\pgfqpoint{-0.010825in}{-0.018638in}}{\pgfqpoint{-0.005525in}{-0.020833in}}{\pgfqpoint{0.000000in}{-0.020833in}}%
\pgfpathclose%
\pgfusepath{stroke,fill}%
}%
\begin{pgfscope}%
\pgfsys@transformshift{0.881250in}{1.732500in}%
\pgfsys@useobject{currentmarker}{}%
\end{pgfscope}%
\begin{pgfscope}%
\pgfsys@transformshift{1.434821in}{2.023851in}%
\pgfsys@useobject{currentmarker}{}%
\end{pgfscope}%
\begin{pgfscope}%
\pgfsys@transformshift{1.988393in}{2.352040in}%
\pgfsys@useobject{currentmarker}{}%
\end{pgfscope}%
\begin{pgfscope}%
\pgfsys@transformshift{2.541964in}{2.594900in}%
\pgfsys@useobject{currentmarker}{}%
\end{pgfscope}%
\begin{pgfscope}%
\pgfsys@transformshift{3.095536in}{2.594900in}%
\pgfsys@useobject{currentmarker}{}%
\end{pgfscope}%
\begin{pgfscope}%
\pgfsys@transformshift{3.649107in}{2.352040in}%
\pgfsys@useobject{currentmarker}{}%
\end{pgfscope}%
\begin{pgfscope}%
\pgfsys@transformshift{4.202679in}{2.023851in}%
\pgfsys@useobject{currentmarker}{}%
\end{pgfscope}%
\end{pgfscope}%
\begin{pgfscope}%
\pgfpathrectangle{\pgfqpoint{0.687500in}{0.385000in}}{\pgfqpoint{4.262500in}{2.695000in}}%
\pgfusepath{clip}%
\pgfsetrectcap%
\pgfsetroundjoin%
\pgfsetlinewidth{1.505625pt}%
\definecolor{currentstroke}{rgb}{0.172549,0.627451,0.172549}%
\pgfsetstrokecolor{currentstroke}%
\pgfsetdash{}{0pt}%
\pgfpathmoveto{\pgfqpoint{0.881250in}{1.732500in}}%
\pgfpathlineto{\pgfqpoint{0.959139in}{1.774854in}}%
\pgfpathlineto{\pgfqpoint{1.075974in}{1.834958in}}%
\pgfpathlineto{\pgfqpoint{1.270697in}{1.934685in}}%
\pgfpathlineto{\pgfqpoint{1.387531in}{1.997531in}}%
\pgfpathlineto{\pgfqpoint{1.484893in}{2.052272in}}%
\pgfpathlineto{\pgfqpoint{1.601727in}{2.120554in}}%
\pgfpathlineto{\pgfqpoint{1.776979in}{2.226193in}}%
\pgfpathlineto{\pgfqpoint{1.932758in}{2.319641in}}%
\pgfpathlineto{\pgfqpoint{2.030119in}{2.375791in}}%
\pgfpathlineto{\pgfqpoint{2.108009in}{2.418528in}}%
\pgfpathlineto{\pgfqpoint{2.185898in}{2.458693in}}%
\pgfpathlineto{\pgfqpoint{2.244315in}{2.486791in}}%
\pgfpathlineto{\pgfqpoint{2.302732in}{2.512901in}}%
\pgfpathlineto{\pgfqpoint{2.361149in}{2.536813in}}%
\pgfpathlineto{\pgfqpoint{2.419567in}{2.558332in}}%
\pgfpathlineto{\pgfqpoint{2.477984in}{2.577280in}}%
\pgfpathlineto{\pgfqpoint{2.536401in}{2.593502in}}%
\pgfpathlineto{\pgfqpoint{2.594818in}{2.606863in}}%
\pgfpathlineto{\pgfqpoint{2.653235in}{2.617252in}}%
\pgfpathlineto{\pgfqpoint{2.711652in}{2.624582in}}%
\pgfpathlineto{\pgfqpoint{2.770069in}{2.628792in}}%
\pgfpathlineto{\pgfqpoint{2.828486in}{2.629847in}}%
\pgfpathlineto{\pgfqpoint{2.886903in}{2.627738in}}%
\pgfpathlineto{\pgfqpoint{2.945320in}{2.622483in}}%
\pgfpathlineto{\pgfqpoint{3.003737in}{2.614125in}}%
\pgfpathlineto{\pgfqpoint{3.062155in}{2.602735in}}%
\pgfpathlineto{\pgfqpoint{3.120572in}{2.588407in}}%
\pgfpathlineto{\pgfqpoint{3.178989in}{2.571260in}}%
\pgfpathlineto{\pgfqpoint{3.237406in}{2.551436in}}%
\pgfpathlineto{\pgfqpoint{3.295823in}{2.529099in}}%
\pgfpathlineto{\pgfqpoint{3.354240in}{2.504432in}}%
\pgfpathlineto{\pgfqpoint{3.412657in}{2.477635in}}%
\pgfpathlineto{\pgfqpoint{3.490546in}{2.438968in}}%
\pgfpathlineto{\pgfqpoint{3.568436in}{2.397445in}}%
\pgfpathlineto{\pgfqpoint{3.665798in}{2.342400in}}%
\pgfpathlineto{\pgfqpoint{3.782632in}{2.273257in}}%
\pgfpathlineto{\pgfqpoint{4.133134in}{2.063472in}}%
\pgfpathlineto{\pgfqpoint{4.249969in}{1.997531in}}%
\pgfpathlineto{\pgfqpoint{4.347330in}{1.944954in}}%
\pgfpathlineto{\pgfqpoint{4.464165in}{1.884341in}}%
\pgfpathlineto{\pgfqpoint{4.697833in}{1.764512in}}%
\pgfpathlineto{\pgfqpoint{4.756250in}{1.732500in}}%
\pgfpathlineto{\pgfqpoint{4.756250in}{1.732500in}}%
\pgfusepath{stroke}%
\end{pgfscope}%
\begin{pgfscope}%
\pgfpathrectangle{\pgfqpoint{0.687500in}{0.385000in}}{\pgfqpoint{4.262500in}{2.695000in}}%
\pgfusepath{clip}%
\pgfsetbuttcap%
\pgfsetroundjoin%
\definecolor{currentfill}{rgb}{0.839216,0.152941,0.156863}%
\pgfsetfillcolor{currentfill}%
\pgfsetlinewidth{1.003750pt}%
\definecolor{currentstroke}{rgb}{0.839216,0.152941,0.156863}%
\pgfsetstrokecolor{currentstroke}%
\pgfsetdash{}{0pt}%
\pgfsys@defobject{currentmarker}{\pgfqpoint{-0.020833in}{-0.020833in}}{\pgfqpoint{0.020833in}{0.020833in}}{%
\pgfpathmoveto{\pgfqpoint{0.000000in}{-0.020833in}}%
\pgfpathcurveto{\pgfqpoint{0.005525in}{-0.020833in}}{\pgfqpoint{0.010825in}{-0.018638in}}{\pgfqpoint{0.014731in}{-0.014731in}}%
\pgfpathcurveto{\pgfqpoint{0.018638in}{-0.010825in}}{\pgfqpoint{0.020833in}{-0.005525in}}{\pgfqpoint{0.020833in}{0.000000in}}%
\pgfpathcurveto{\pgfqpoint{0.020833in}{0.005525in}}{\pgfqpoint{0.018638in}{0.010825in}}{\pgfqpoint{0.014731in}{0.014731in}}%
\pgfpathcurveto{\pgfqpoint{0.010825in}{0.018638in}}{\pgfqpoint{0.005525in}{0.020833in}}{\pgfqpoint{0.000000in}{0.020833in}}%
\pgfpathcurveto{\pgfqpoint{-0.005525in}{0.020833in}}{\pgfqpoint{-0.010825in}{0.018638in}}{\pgfqpoint{-0.014731in}{0.014731in}}%
\pgfpathcurveto{\pgfqpoint{-0.018638in}{0.010825in}}{\pgfqpoint{-0.020833in}{0.005525in}}{\pgfqpoint{-0.020833in}{0.000000in}}%
\pgfpathcurveto{\pgfqpoint{-0.020833in}{-0.005525in}}{\pgfqpoint{-0.018638in}{-0.010825in}}{\pgfqpoint{-0.014731in}{-0.014731in}}%
\pgfpathcurveto{\pgfqpoint{-0.010825in}{-0.018638in}}{\pgfqpoint{-0.005525in}{-0.020833in}}{\pgfqpoint{0.000000in}{-0.020833in}}%
\pgfpathclose%
\pgfusepath{stroke,fill}%
}%
\begin{pgfscope}%
\pgfsys@transformshift{0.881250in}{1.732500in}%
\pgfsys@useobject{currentmarker}{}%
\end{pgfscope}%
\begin{pgfscope}%
\pgfsys@transformshift{1.311806in}{1.953628in}%
\pgfsys@useobject{currentmarker}{}%
\end{pgfscope}%
\begin{pgfscope}%
\pgfsys@transformshift{1.742361in}{2.207091in}%
\pgfsys@useobject{currentmarker}{}%
\end{pgfscope}%
\begin{pgfscope}%
\pgfsys@transformshift{2.172917in}{2.451167in}%
\pgfsys@useobject{currentmarker}{}%
\end{pgfscope}%
\begin{pgfscope}%
\pgfsys@transformshift{2.603472in}{2.608923in}%
\pgfsys@useobject{currentmarker}{}%
\end{pgfscope}%
\begin{pgfscope}%
\pgfsys@transformshift{3.034028in}{2.608923in}%
\pgfsys@useobject{currentmarker}{}%
\end{pgfscope}%
\begin{pgfscope}%
\pgfsys@transformshift{3.464583in}{2.451167in}%
\pgfsys@useobject{currentmarker}{}%
\end{pgfscope}%
\begin{pgfscope}%
\pgfsys@transformshift{3.895139in}{2.207091in}%
\pgfsys@useobject{currentmarker}{}%
\end{pgfscope}%
\begin{pgfscope}%
\pgfsys@transformshift{4.325694in}{1.953628in}%
\pgfsys@useobject{currentmarker}{}%
\end{pgfscope}%
\end{pgfscope}%
\begin{pgfscope}%
\pgfpathrectangle{\pgfqpoint{0.687500in}{0.385000in}}{\pgfqpoint{4.262500in}{2.695000in}}%
\pgfusepath{clip}%
\pgfsetrectcap%
\pgfsetroundjoin%
\pgfsetlinewidth{1.505625pt}%
\definecolor{currentstroke}{rgb}{0.580392,0.403922,0.741176}%
\pgfsetstrokecolor{currentstroke}%
\pgfsetdash{}{0pt}%
\pgfpathmoveto{\pgfqpoint{0.881250in}{1.732500in}}%
\pgfpathlineto{\pgfqpoint{0.939667in}{1.758093in}}%
\pgfpathlineto{\pgfqpoint{1.017557in}{1.795357in}}%
\pgfpathlineto{\pgfqpoint{1.095446in}{1.835156in}}%
\pgfpathlineto{\pgfqpoint{1.192808in}{1.887330in}}%
\pgfpathlineto{\pgfqpoint{1.309642in}{1.952403in}}%
\pgfpathlineto{\pgfqpoint{1.445948in}{2.030804in}}%
\pgfpathlineto{\pgfqpoint{1.621200in}{2.134460in}}%
\pgfpathlineto{\pgfqpoint{1.932758in}{2.319918in}}%
\pgfpathlineto{\pgfqpoint{2.030119in}{2.375330in}}%
\pgfpathlineto{\pgfqpoint{2.108009in}{2.417669in}}%
\pgfpathlineto{\pgfqpoint{2.185898in}{2.457640in}}%
\pgfpathlineto{\pgfqpoint{2.263788in}{2.494683in}}%
\pgfpathlineto{\pgfqpoint{2.322205in}{2.520207in}}%
\pgfpathlineto{\pgfqpoint{2.380622in}{2.543544in}}%
\pgfpathlineto{\pgfqpoint{2.439039in}{2.564481in}}%
\pgfpathlineto{\pgfqpoint{2.497456in}{2.582823in}}%
\pgfpathlineto{\pgfqpoint{2.555873in}{2.598392in}}%
\pgfpathlineto{\pgfqpoint{2.614290in}{2.611038in}}%
\pgfpathlineto{\pgfqpoint{2.672707in}{2.620635in}}%
\pgfpathlineto{\pgfqpoint{2.731124in}{2.627087in}}%
\pgfpathlineto{\pgfqpoint{2.789541in}{2.630330in}}%
\pgfpathlineto{\pgfqpoint{2.847959in}{2.630330in}}%
\pgfpathlineto{\pgfqpoint{2.906376in}{2.627087in}}%
\pgfpathlineto{\pgfqpoint{2.964793in}{2.620635in}}%
\pgfpathlineto{\pgfqpoint{3.023210in}{2.611038in}}%
\pgfpathlineto{\pgfqpoint{3.081627in}{2.598392in}}%
\pgfpathlineto{\pgfqpoint{3.140044in}{2.582823in}}%
\pgfpathlineto{\pgfqpoint{3.198461in}{2.564481in}}%
\pgfpathlineto{\pgfqpoint{3.256878in}{2.543544in}}%
\pgfpathlineto{\pgfqpoint{3.315295in}{2.520207in}}%
\pgfpathlineto{\pgfqpoint{3.373712in}{2.494683in}}%
\pgfpathlineto{\pgfqpoint{3.451602in}{2.457640in}}%
\pgfpathlineto{\pgfqpoint{3.529491in}{2.417669in}}%
\pgfpathlineto{\pgfqpoint{3.607381in}{2.375330in}}%
\pgfpathlineto{\pgfqpoint{3.704742in}{2.319918in}}%
\pgfpathlineto{\pgfqpoint{3.841049in}{2.239473in}}%
\pgfpathlineto{\pgfqpoint{4.211024in}{2.019465in}}%
\pgfpathlineto{\pgfqpoint{4.347330in}{1.941405in}}%
\pgfpathlineto{\pgfqpoint{4.464165in}{1.876723in}}%
\pgfpathlineto{\pgfqpoint{4.561526in}{1.825016in}}%
\pgfpathlineto{\pgfqpoint{4.639416in}{1.785765in}}%
\pgfpathlineto{\pgfqpoint{4.717305in}{1.749300in}}%
\pgfpathlineto{\pgfqpoint{4.756250in}{1.732500in}}%
\pgfpathlineto{\pgfqpoint{4.756250in}{1.732500in}}%
\pgfusepath{stroke}%
\end{pgfscope}%
\begin{pgfscope}%
\pgfpathrectangle{\pgfqpoint{0.687500in}{0.385000in}}{\pgfqpoint{4.262500in}{2.695000in}}%
\pgfusepath{clip}%
\pgfsetbuttcap%
\pgfsetroundjoin%
\definecolor{currentfill}{rgb}{0.549020,0.337255,0.294118}%
\pgfsetfillcolor{currentfill}%
\pgfsetlinewidth{1.003750pt}%
\definecolor{currentstroke}{rgb}{0.549020,0.337255,0.294118}%
\pgfsetstrokecolor{currentstroke}%
\pgfsetdash{}{0pt}%
\pgfsys@defobject{currentmarker}{\pgfqpoint{-0.020833in}{-0.020833in}}{\pgfqpoint{0.020833in}{0.020833in}}{%
\pgfpathmoveto{\pgfqpoint{0.000000in}{-0.020833in}}%
\pgfpathcurveto{\pgfqpoint{0.005525in}{-0.020833in}}{\pgfqpoint{0.010825in}{-0.018638in}}{\pgfqpoint{0.014731in}{-0.014731in}}%
\pgfpathcurveto{\pgfqpoint{0.018638in}{-0.010825in}}{\pgfqpoint{0.020833in}{-0.005525in}}{\pgfqpoint{0.020833in}{0.000000in}}%
\pgfpathcurveto{\pgfqpoint{0.020833in}{0.005525in}}{\pgfqpoint{0.018638in}{0.010825in}}{\pgfqpoint{0.014731in}{0.014731in}}%
\pgfpathcurveto{\pgfqpoint{0.010825in}{0.018638in}}{\pgfqpoint{0.005525in}{0.020833in}}{\pgfqpoint{0.000000in}{0.020833in}}%
\pgfpathcurveto{\pgfqpoint{-0.005525in}{0.020833in}}{\pgfqpoint{-0.010825in}{0.018638in}}{\pgfqpoint{-0.014731in}{0.014731in}}%
\pgfpathcurveto{\pgfqpoint{-0.018638in}{0.010825in}}{\pgfqpoint{-0.020833in}{0.005525in}}{\pgfqpoint{-0.020833in}{0.000000in}}%
\pgfpathcurveto{\pgfqpoint{-0.020833in}{-0.005525in}}{\pgfqpoint{-0.018638in}{-0.010825in}}{\pgfqpoint{-0.014731in}{-0.014731in}}%
\pgfpathcurveto{\pgfqpoint{-0.010825in}{-0.018638in}}{\pgfqpoint{-0.005525in}{-0.020833in}}{\pgfqpoint{0.000000in}{-0.020833in}}%
\pgfpathclose%
\pgfusepath{stroke,fill}%
}%
\begin{pgfscope}%
\pgfsys@transformshift{0.881250in}{1.732500in}%
\pgfsys@useobject{currentmarker}{}%
\end{pgfscope}%
\begin{pgfscope}%
\pgfsys@transformshift{1.109191in}{1.844355in}%
\pgfsys@useobject{currentmarker}{}%
\end{pgfscope}%
\begin{pgfscope}%
\pgfsys@transformshift{1.337132in}{1.967871in}%
\pgfsys@useobject{currentmarker}{}%
\end{pgfscope}%
\begin{pgfscope}%
\pgfsys@transformshift{1.565074in}{2.100598in}%
\pgfsys@useobject{currentmarker}{}%
\end{pgfscope}%
\begin{pgfscope}%
\pgfsys@transformshift{1.793015in}{2.237509in}%
\pgfsys@useobject{currentmarker}{}%
\end{pgfscope}%
\begin{pgfscope}%
\pgfsys@transformshift{2.020956in}{2.370370in}%
\pgfsys@useobject{currentmarker}{}%
\end{pgfscope}%
\begin{pgfscope}%
\pgfsys@transformshift{2.248897in}{2.487787in}%
\pgfsys@useobject{currentmarker}{}%
\end{pgfscope}%
\begin{pgfscope}%
\pgfsys@transformshift{2.476838in}{2.576572in}%
\pgfsys@useobject{currentmarker}{}%
\end{pgfscope}%
\begin{pgfscope}%
\pgfsys@transformshift{2.704779in}{2.624638in}%
\pgfsys@useobject{currentmarker}{}%
\end{pgfscope}%
\begin{pgfscope}%
\pgfsys@transformshift{2.932721in}{2.624638in}%
\pgfsys@useobject{currentmarker}{}%
\end{pgfscope}%
\begin{pgfscope}%
\pgfsys@transformshift{3.160662in}{2.576572in}%
\pgfsys@useobject{currentmarker}{}%
\end{pgfscope}%
\begin{pgfscope}%
\pgfsys@transformshift{3.388603in}{2.487787in}%
\pgfsys@useobject{currentmarker}{}%
\end{pgfscope}%
\begin{pgfscope}%
\pgfsys@transformshift{3.616544in}{2.370370in}%
\pgfsys@useobject{currentmarker}{}%
\end{pgfscope}%
\begin{pgfscope}%
\pgfsys@transformshift{3.844485in}{2.237509in}%
\pgfsys@useobject{currentmarker}{}%
\end{pgfscope}%
\begin{pgfscope}%
\pgfsys@transformshift{4.072426in}{2.100598in}%
\pgfsys@useobject{currentmarker}{}%
\end{pgfscope}%
\begin{pgfscope}%
\pgfsys@transformshift{4.300368in}{1.967871in}%
\pgfsys@useobject{currentmarker}{}%
\end{pgfscope}%
\begin{pgfscope}%
\pgfsys@transformshift{4.528309in}{1.844355in}%
\pgfsys@useobject{currentmarker}{}%
\end{pgfscope}%
\end{pgfscope}%
\begin{pgfscope}%
\pgfpathrectangle{\pgfqpoint{0.687500in}{0.385000in}}{\pgfqpoint{4.262500in}{2.695000in}}%
\pgfusepath{clip}%
\pgfsetrectcap%
\pgfsetroundjoin%
\pgfsetlinewidth{1.505625pt}%
\definecolor{currentstroke}{rgb}{0.890196,0.466667,0.760784}%
\pgfsetstrokecolor{currentstroke}%
\pgfsetdash{}{0pt}%
\pgfpathmoveto{\pgfqpoint{0.881250in}{1.732500in}}%
\pgfpathlineto{\pgfqpoint{0.978612in}{1.778737in}}%
\pgfpathlineto{\pgfqpoint{1.075974in}{1.827291in}}%
\pgfpathlineto{\pgfqpoint{1.173335in}{1.878004in}}%
\pgfpathlineto{\pgfqpoint{1.290170in}{1.941558in}}%
\pgfpathlineto{\pgfqpoint{1.407004in}{2.007752in}}%
\pgfpathlineto{\pgfqpoint{1.543310in}{2.087642in}}%
\pgfpathlineto{\pgfqpoint{1.796451in}{2.239567in}}%
\pgfpathlineto{\pgfqpoint{1.932758in}{2.320107in}}%
\pgfpathlineto{\pgfqpoint{2.030119in}{2.375473in}}%
\pgfpathlineto{\pgfqpoint{2.108009in}{2.417737in}}%
\pgfpathlineto{\pgfqpoint{2.185898in}{2.457628in}}%
\pgfpathlineto{\pgfqpoint{2.263788in}{2.494605in}}%
\pgfpathlineto{\pgfqpoint{2.322205in}{2.520101in}}%
\pgfpathlineto{\pgfqpoint{2.380622in}{2.543430in}}%
\pgfpathlineto{\pgfqpoint{2.439039in}{2.564379in}}%
\pgfpathlineto{\pgfqpoint{2.497456in}{2.582749in}}%
\pgfpathlineto{\pgfqpoint{2.555873in}{2.598357in}}%
\pgfpathlineto{\pgfqpoint{2.614290in}{2.611046in}}%
\pgfpathlineto{\pgfqpoint{2.672707in}{2.620683in}}%
\pgfpathlineto{\pgfqpoint{2.731124in}{2.627166in}}%
\pgfpathlineto{\pgfqpoint{2.789541in}{2.630425in}}%
\pgfpathlineto{\pgfqpoint{2.847959in}{2.630425in}}%
\pgfpathlineto{\pgfqpoint{2.906376in}{2.627166in}}%
\pgfpathlineto{\pgfqpoint{2.964793in}{2.620683in}}%
\pgfpathlineto{\pgfqpoint{3.023210in}{2.611046in}}%
\pgfpathlineto{\pgfqpoint{3.081627in}{2.598357in}}%
\pgfpathlineto{\pgfqpoint{3.140044in}{2.582749in}}%
\pgfpathlineto{\pgfqpoint{3.198461in}{2.564379in}}%
\pgfpathlineto{\pgfqpoint{3.256878in}{2.543430in}}%
\pgfpathlineto{\pgfqpoint{3.315295in}{2.520101in}}%
\pgfpathlineto{\pgfqpoint{3.373712in}{2.494605in}}%
\pgfpathlineto{\pgfqpoint{3.451602in}{2.457628in}}%
\pgfpathlineto{\pgfqpoint{3.529491in}{2.417737in}}%
\pgfpathlineto{\pgfqpoint{3.607381in}{2.375473in}}%
\pgfpathlineto{\pgfqpoint{3.704742in}{2.320107in}}%
\pgfpathlineto{\pgfqpoint{3.841049in}{2.239567in}}%
\pgfpathlineto{\pgfqpoint{4.211024in}{2.019009in}}%
\pgfpathlineto{\pgfqpoint{4.327858in}{1.952418in}}%
\pgfpathlineto{\pgfqpoint{4.444692in}{1.888398in}}%
\pgfpathlineto{\pgfqpoint{4.561526in}{1.827291in}}%
\pgfpathlineto{\pgfqpoint{4.658888in}{1.778737in}}%
\pgfpathlineto{\pgfqpoint{4.756250in}{1.732500in}}%
\pgfpathlineto{\pgfqpoint{4.756250in}{1.732500in}}%
\pgfusepath{stroke}%
\end{pgfscope}%
\begin{pgfscope}%
\pgfpathrectangle{\pgfqpoint{0.687500in}{0.385000in}}{\pgfqpoint{4.262500in}{2.695000in}}%
\pgfusepath{clip}%
\pgfsetbuttcap%
\pgfsetroundjoin%
\definecolor{currentfill}{rgb}{0.498039,0.498039,0.498039}%
\pgfsetfillcolor{currentfill}%
\pgfsetlinewidth{1.003750pt}%
\definecolor{currentstroke}{rgb}{0.498039,0.498039,0.498039}%
\pgfsetstrokecolor{currentstroke}%
\pgfsetdash{}{0pt}%
\pgfsys@defobject{currentmarker}{\pgfqpoint{-0.020833in}{-0.020833in}}{\pgfqpoint{0.020833in}{0.020833in}}{%
\pgfpathmoveto{\pgfqpoint{0.000000in}{-0.020833in}}%
\pgfpathcurveto{\pgfqpoint{0.005525in}{-0.020833in}}{\pgfqpoint{0.010825in}{-0.018638in}}{\pgfqpoint{0.014731in}{-0.014731in}}%
\pgfpathcurveto{\pgfqpoint{0.018638in}{-0.010825in}}{\pgfqpoint{0.020833in}{-0.005525in}}{\pgfqpoint{0.020833in}{0.000000in}}%
\pgfpathcurveto{\pgfqpoint{0.020833in}{0.005525in}}{\pgfqpoint{0.018638in}{0.010825in}}{\pgfqpoint{0.014731in}{0.014731in}}%
\pgfpathcurveto{\pgfqpoint{0.010825in}{0.018638in}}{\pgfqpoint{0.005525in}{0.020833in}}{\pgfqpoint{0.000000in}{0.020833in}}%
\pgfpathcurveto{\pgfqpoint{-0.005525in}{0.020833in}}{\pgfqpoint{-0.010825in}{0.018638in}}{\pgfqpoint{-0.014731in}{0.014731in}}%
\pgfpathcurveto{\pgfqpoint{-0.018638in}{0.010825in}}{\pgfqpoint{-0.020833in}{0.005525in}}{\pgfqpoint{-0.020833in}{0.000000in}}%
\pgfpathcurveto{\pgfqpoint{-0.020833in}{-0.005525in}}{\pgfqpoint{-0.018638in}{-0.010825in}}{\pgfqpoint{-0.014731in}{-0.014731in}}%
\pgfpathcurveto{\pgfqpoint{-0.010825in}{-0.018638in}}{\pgfqpoint{-0.005525in}{-0.020833in}}{\pgfqpoint{0.000000in}{-0.020833in}}%
\pgfpathclose%
\pgfusepath{stroke,fill}%
}%
\begin{pgfscope}%
\pgfsys@transformshift{0.881250in}{1.732500in}%
\pgfsys@useobject{currentmarker}{}%
\end{pgfscope}%
\begin{pgfscope}%
\pgfsys@transformshift{0.998674in}{1.788591in}%
\pgfsys@useobject{currentmarker}{}%
\end{pgfscope}%
\begin{pgfscope}%
\pgfsys@transformshift{1.116098in}{1.847934in}%
\pgfsys@useobject{currentmarker}{}%
\end{pgfscope}%
\begin{pgfscope}%
\pgfsys@transformshift{1.233523in}{1.910388in}%
\pgfsys@useobject{currentmarker}{}%
\end{pgfscope}%
\begin{pgfscope}%
\pgfsys@transformshift{1.350947in}{1.975689in}%
\pgfsys@useobject{currentmarker}{}%
\end{pgfscope}%
\begin{pgfscope}%
\pgfsys@transformshift{1.468371in}{2.043419in}%
\pgfsys@useobject{currentmarker}{}%
\end{pgfscope}%
\begin{pgfscope}%
\pgfsys@transformshift{1.585795in}{2.112971in}%
\pgfsys@useobject{currentmarker}{}%
\end{pgfscope}%
\begin{pgfscope}%
\pgfsys@transformshift{1.703220in}{2.183525in}%
\pgfsys@useobject{currentmarker}{}%
\end{pgfscope}%
\begin{pgfscope}%
\pgfsys@transformshift{1.820644in}{2.254029in}%
\pgfsys@useobject{currentmarker}{}%
\end{pgfscope}%
\begin{pgfscope}%
\pgfsys@transformshift{1.938068in}{2.323185in}%
\pgfsys@useobject{currentmarker}{}%
\end{pgfscope}%
\begin{pgfscope}%
\pgfsys@transformshift{2.055492in}{2.389469in}%
\pgfsys@useobject{currentmarker}{}%
\end{pgfscope}%
\begin{pgfscope}%
\pgfsys@transformshift{2.172917in}{2.451167in}%
\pgfsys@useobject{currentmarker}{}%
\end{pgfscope}%
\begin{pgfscope}%
\pgfsys@transformshift{2.290341in}{2.506449in}%
\pgfsys@useobject{currentmarker}{}%
\end{pgfscope}%
\begin{pgfscope}%
\pgfsys@transformshift{2.407765in}{2.553472in}%
\pgfsys@useobject{currentmarker}{}%
\end{pgfscope}%
\begin{pgfscope}%
\pgfsys@transformshift{2.525189in}{2.590513in}%
\pgfsys@useobject{currentmarker}{}%
\end{pgfscope}%
\begin{pgfscope}%
\pgfsys@transformshift{2.642614in}{2.616107in}%
\pgfsys@useobject{currentmarker}{}%
\end{pgfscope}%
\begin{pgfscope}%
\pgfsys@transformshift{2.760038in}{2.629185in}%
\pgfsys@useobject{currentmarker}{}%
\end{pgfscope}%
\begin{pgfscope}%
\pgfsys@transformshift{2.877462in}{2.629185in}%
\pgfsys@useobject{currentmarker}{}%
\end{pgfscope}%
\begin{pgfscope}%
\pgfsys@transformshift{2.994886in}{2.616107in}%
\pgfsys@useobject{currentmarker}{}%
\end{pgfscope}%
\begin{pgfscope}%
\pgfsys@transformshift{3.112311in}{2.590513in}%
\pgfsys@useobject{currentmarker}{}%
\end{pgfscope}%
\begin{pgfscope}%
\pgfsys@transformshift{3.229735in}{2.553472in}%
\pgfsys@useobject{currentmarker}{}%
\end{pgfscope}%
\begin{pgfscope}%
\pgfsys@transformshift{3.347159in}{2.506449in}%
\pgfsys@useobject{currentmarker}{}%
\end{pgfscope}%
\begin{pgfscope}%
\pgfsys@transformshift{3.464583in}{2.451167in}%
\pgfsys@useobject{currentmarker}{}%
\end{pgfscope}%
\begin{pgfscope}%
\pgfsys@transformshift{3.582008in}{2.389469in}%
\pgfsys@useobject{currentmarker}{}%
\end{pgfscope}%
\begin{pgfscope}%
\pgfsys@transformshift{3.699432in}{2.323185in}%
\pgfsys@useobject{currentmarker}{}%
\end{pgfscope}%
\begin{pgfscope}%
\pgfsys@transformshift{3.816856in}{2.254029in}%
\pgfsys@useobject{currentmarker}{}%
\end{pgfscope}%
\begin{pgfscope}%
\pgfsys@transformshift{3.934280in}{2.183525in}%
\pgfsys@useobject{currentmarker}{}%
\end{pgfscope}%
\begin{pgfscope}%
\pgfsys@transformshift{4.051705in}{2.112971in}%
\pgfsys@useobject{currentmarker}{}%
\end{pgfscope}%
\begin{pgfscope}%
\pgfsys@transformshift{4.169129in}{2.043419in}%
\pgfsys@useobject{currentmarker}{}%
\end{pgfscope}%
\begin{pgfscope}%
\pgfsys@transformshift{4.286553in}{1.975689in}%
\pgfsys@useobject{currentmarker}{}%
\end{pgfscope}%
\begin{pgfscope}%
\pgfsys@transformshift{4.403977in}{1.910388in}%
\pgfsys@useobject{currentmarker}{}%
\end{pgfscope}%
\begin{pgfscope}%
\pgfsys@transformshift{4.521402in}{1.847934in}%
\pgfsys@useobject{currentmarker}{}%
\end{pgfscope}%
\begin{pgfscope}%
\pgfsys@transformshift{4.638826in}{1.788591in}%
\pgfsys@useobject{currentmarker}{}%
\end{pgfscope}%
\end{pgfscope}%
\begin{pgfscope}%
\pgfpathrectangle{\pgfqpoint{0.687500in}{0.385000in}}{\pgfqpoint{4.262500in}{2.695000in}}%
\pgfusepath{clip}%
\pgfsetrectcap%
\pgfsetroundjoin%
\pgfsetlinewidth{1.505625pt}%
\definecolor{currentstroke}{rgb}{0.737255,0.741176,0.133333}%
\pgfsetstrokecolor{currentstroke}%
\pgfsetdash{}{0pt}%
\pgfpathmoveto{\pgfqpoint{0.881250in}{1.732500in}}%
\pgfpathlineto{\pgfqpoint{0.978612in}{1.778775in}}%
\pgfpathlineto{\pgfqpoint{1.075974in}{1.827296in}}%
\pgfpathlineto{\pgfqpoint{1.173335in}{1.878000in}}%
\pgfpathlineto{\pgfqpoint{1.290170in}{1.941556in}}%
\pgfpathlineto{\pgfqpoint{1.407004in}{2.007752in}}%
\pgfpathlineto{\pgfqpoint{1.543310in}{2.087643in}}%
\pgfpathlineto{\pgfqpoint{1.796451in}{2.239567in}}%
\pgfpathlineto{\pgfqpoint{1.932758in}{2.320107in}}%
\pgfpathlineto{\pgfqpoint{2.030119in}{2.375473in}}%
\pgfpathlineto{\pgfqpoint{2.108009in}{2.417737in}}%
\pgfpathlineto{\pgfqpoint{2.185898in}{2.457628in}}%
\pgfpathlineto{\pgfqpoint{2.263788in}{2.494605in}}%
\pgfpathlineto{\pgfqpoint{2.322205in}{2.520101in}}%
\pgfpathlineto{\pgfqpoint{2.380622in}{2.543430in}}%
\pgfpathlineto{\pgfqpoint{2.439039in}{2.564379in}}%
\pgfpathlineto{\pgfqpoint{2.497456in}{2.582749in}}%
\pgfpathlineto{\pgfqpoint{2.555873in}{2.598357in}}%
\pgfpathlineto{\pgfqpoint{2.614290in}{2.611046in}}%
\pgfpathlineto{\pgfqpoint{2.672707in}{2.620683in}}%
\pgfpathlineto{\pgfqpoint{2.731124in}{2.627166in}}%
\pgfpathlineto{\pgfqpoint{2.789541in}{2.630425in}}%
\pgfpathlineto{\pgfqpoint{2.847959in}{2.630425in}}%
\pgfpathlineto{\pgfqpoint{2.906376in}{2.627166in}}%
\pgfpathlineto{\pgfqpoint{2.964793in}{2.620683in}}%
\pgfpathlineto{\pgfqpoint{3.023210in}{2.611046in}}%
\pgfpathlineto{\pgfqpoint{3.081627in}{2.598357in}}%
\pgfpathlineto{\pgfqpoint{3.140044in}{2.582749in}}%
\pgfpathlineto{\pgfqpoint{3.198461in}{2.564379in}}%
\pgfpathlineto{\pgfqpoint{3.256878in}{2.543430in}}%
\pgfpathlineto{\pgfqpoint{3.315295in}{2.520101in}}%
\pgfpathlineto{\pgfqpoint{3.373712in}{2.494605in}}%
\pgfpathlineto{\pgfqpoint{3.451602in}{2.457628in}}%
\pgfpathlineto{\pgfqpoint{3.529491in}{2.417737in}}%
\pgfpathlineto{\pgfqpoint{3.607381in}{2.375473in}}%
\pgfpathlineto{\pgfqpoint{3.704742in}{2.320107in}}%
\pgfpathlineto{\pgfqpoint{3.841049in}{2.239567in}}%
\pgfpathlineto{\pgfqpoint{4.211024in}{2.019009in}}%
\pgfpathlineto{\pgfqpoint{4.327858in}{1.952417in}}%
\pgfpathlineto{\pgfqpoint{4.444692in}{1.888394in}}%
\pgfpathlineto{\pgfqpoint{4.542054in}{1.837265in}}%
\pgfpathlineto{\pgfqpoint{4.639416in}{1.788301in}}%
\pgfpathlineto{\pgfqpoint{4.736778in}{1.741574in}}%
\pgfpathlineto{\pgfqpoint{4.756250in}{1.732503in}}%
\pgfpathlineto{\pgfqpoint{4.756250in}{1.732503in}}%
\pgfusepath{stroke}%
\end{pgfscope}%
\begin{pgfscope}%
\pgfsetrectcap%
\pgfsetmiterjoin%
\pgfsetlinewidth{0.803000pt}%
\definecolor{currentstroke}{rgb}{0.000000,0.000000,0.000000}%
\pgfsetstrokecolor{currentstroke}%
\pgfsetdash{}{0pt}%
\pgfpathmoveto{\pgfqpoint{0.687500in}{0.385000in}}%
\pgfpathlineto{\pgfqpoint{0.687500in}{3.080000in}}%
\pgfusepath{stroke}%
\end{pgfscope}%
\begin{pgfscope}%
\pgfsetrectcap%
\pgfsetmiterjoin%
\pgfsetlinewidth{0.803000pt}%
\definecolor{currentstroke}{rgb}{0.000000,0.000000,0.000000}%
\pgfsetstrokecolor{currentstroke}%
\pgfsetdash{}{0pt}%
\pgfpathmoveto{\pgfqpoint{4.950000in}{0.385000in}}%
\pgfpathlineto{\pgfqpoint{4.950000in}{3.080000in}}%
\pgfusepath{stroke}%
\end{pgfscope}%
\begin{pgfscope}%
\pgfsetrectcap%
\pgfsetmiterjoin%
\pgfsetlinewidth{0.803000pt}%
\definecolor{currentstroke}{rgb}{0.000000,0.000000,0.000000}%
\pgfsetstrokecolor{currentstroke}%
\pgfsetdash{}{0pt}%
\pgfpathmoveto{\pgfqpoint{0.687500in}{0.385000in}}%
\pgfpathlineto{\pgfqpoint{4.950000in}{0.385000in}}%
\pgfusepath{stroke}%
\end{pgfscope}%
\begin{pgfscope}%
\pgfsetrectcap%
\pgfsetmiterjoin%
\pgfsetlinewidth{0.803000pt}%
\definecolor{currentstroke}{rgb}{0.000000,0.000000,0.000000}%
\pgfsetstrokecolor{currentstroke}%
\pgfsetdash{}{0pt}%
\pgfpathmoveto{\pgfqpoint{0.687500in}{3.080000in}}%
\pgfpathlineto{\pgfqpoint{4.950000in}{3.080000in}}%
\pgfusepath{stroke}%
\end{pgfscope}%
\begin{pgfscope}%
\definecolor{textcolor}{rgb}{0.000000,0.000000,0.000000}%
\pgfsetstrokecolor{textcolor}%
\pgfsetfillcolor{textcolor}%
\pgftext[x=2.818750in,y=3.163333in,,base]{\color{textcolor}\rmfamily\fontsize{12.000000}{14.400000}\selectfont N=6, 8, 16, 32}%
\end{pgfscope}%
\begin{pgfscope}%
\pgfsetbuttcap%
\pgfsetmiterjoin%
\definecolor{currentfill}{rgb}{1.000000,1.000000,1.000000}%
\pgfsetfillcolor{currentfill}%
\pgfsetfillopacity{0.800000}%
\pgfsetlinewidth{1.003750pt}%
\definecolor{currentstroke}{rgb}{0.800000,0.800000,0.800000}%
\pgfsetstrokecolor{currentstroke}%
\pgfsetstrokeopacity{0.800000}%
\pgfsetdash{}{0pt}%
\pgfpathmoveto{\pgfqpoint{0.784722in}{0.454444in}}%
\pgfpathlineto{\pgfqpoint{2.050468in}{0.454444in}}%
\pgfpathquadraticcurveto{\pgfqpoint{2.078246in}{0.454444in}}{\pgfqpoint{2.078246in}{0.482222in}}%
\pgfpathlineto{\pgfqpoint{2.078246in}{1.661422in}}%
\pgfpathquadraticcurveto{\pgfqpoint{2.078246in}{1.689199in}}{\pgfqpoint{2.050468in}{1.689199in}}%
\pgfpathlineto{\pgfqpoint{0.784722in}{1.689199in}}%
\pgfpathquadraticcurveto{\pgfqpoint{0.756944in}{1.689199in}}{\pgfqpoint{0.756944in}{1.661422in}}%
\pgfpathlineto{\pgfqpoint{0.756944in}{0.482222in}}%
\pgfpathquadraticcurveto{\pgfqpoint{0.756944in}{0.454444in}}{\pgfqpoint{0.784722in}{0.454444in}}%
\pgfpathclose%
\pgfusepath{stroke,fill}%
\end{pgfscope}%
\begin{pgfscope}%
\pgfsetrectcap%
\pgfsetroundjoin%
\pgfsetlinewidth{1.505625pt}%
\definecolor{currentstroke}{rgb}{0.121569,0.466667,0.705882}%
\pgfsetstrokecolor{currentstroke}%
\pgfsetdash{}{0pt}%
\pgfpathmoveto{\pgfqpoint{0.812500in}{1.498789in}}%
\pgfpathlineto{\pgfqpoint{1.090278in}{1.498789in}}%
\pgfusepath{stroke}%
\end{pgfscope}%
\begin{pgfscope}%
\definecolor{textcolor}{rgb}{0.000000,0.000000,0.000000}%
\pgfsetstrokecolor{textcolor}%
\pgfsetfillcolor{textcolor}%
\pgftext[x=1.201389in,y=1.450178in,left,base]{\color{textcolor}\rmfamily\fontsize{10.000000}{12.000000}\selectfont \(\displaystyle y(x)=\)\(\displaystyle \frac{1}{1+x^{2}}\)}%
\end{pgfscope}%
\begin{pgfscope}%
\pgfsetrectcap%
\pgfsetroundjoin%
\pgfsetlinewidth{1.505625pt}%
\definecolor{currentstroke}{rgb}{0.172549,0.627451,0.172549}%
\pgfsetstrokecolor{currentstroke}%
\pgfsetdash{}{0pt}%
\pgfpathmoveto{\pgfqpoint{0.812500in}{1.218333in}}%
\pgfpathlineto{\pgfqpoint{1.090278in}{1.218333in}}%
\pgfusepath{stroke}%
\end{pgfscope}%
\begin{pgfscope}%
\definecolor{textcolor}{rgb}{0.000000,0.000000,0.000000}%
\pgfsetstrokecolor{textcolor}%
\pgfsetfillcolor{textcolor}%
\pgftext[x=1.201389in,y=1.169722in,left,base]{\color{textcolor}\rmfamily\fontsize{10.000000}{12.000000}\selectfont W6(x)}%
\end{pgfscope}%
\begin{pgfscope}%
\pgfsetrectcap%
\pgfsetroundjoin%
\pgfsetlinewidth{1.505625pt}%
\definecolor{currentstroke}{rgb}{0.580392,0.403922,0.741176}%
\pgfsetstrokecolor{currentstroke}%
\pgfsetdash{}{0pt}%
\pgfpathmoveto{\pgfqpoint{0.812500in}{1.010000in}}%
\pgfpathlineto{\pgfqpoint{1.090278in}{1.010000in}}%
\pgfusepath{stroke}%
\end{pgfscope}%
\begin{pgfscope}%
\definecolor{textcolor}{rgb}{0.000000,0.000000,0.000000}%
\pgfsetstrokecolor{textcolor}%
\pgfsetfillcolor{textcolor}%
\pgftext[x=1.201389in,y=0.961389in,left,base]{\color{textcolor}\rmfamily\fontsize{10.000000}{12.000000}\selectfont W8(x)}%
\end{pgfscope}%
\begin{pgfscope}%
\pgfsetrectcap%
\pgfsetroundjoin%
\pgfsetlinewidth{1.505625pt}%
\definecolor{currentstroke}{rgb}{0.890196,0.466667,0.760784}%
\pgfsetstrokecolor{currentstroke}%
\pgfsetdash{}{0pt}%
\pgfpathmoveto{\pgfqpoint{0.812500in}{0.801667in}}%
\pgfpathlineto{\pgfqpoint{1.090278in}{0.801667in}}%
\pgfusepath{stroke}%
\end{pgfscope}%
\begin{pgfscope}%
\definecolor{textcolor}{rgb}{0.000000,0.000000,0.000000}%
\pgfsetstrokecolor{textcolor}%
\pgfsetfillcolor{textcolor}%
\pgftext[x=1.201389in,y=0.753056in,left,base]{\color{textcolor}\rmfamily\fontsize{10.000000}{12.000000}\selectfont W16(x)}%
\end{pgfscope}%
\begin{pgfscope}%
\pgfsetrectcap%
\pgfsetroundjoin%
\pgfsetlinewidth{1.505625pt}%
\definecolor{currentstroke}{rgb}{0.737255,0.741176,0.133333}%
\pgfsetstrokecolor{currentstroke}%
\pgfsetdash{}{0pt}%
\pgfpathmoveto{\pgfqpoint{0.812500in}{0.593333in}}%
\pgfpathlineto{\pgfqpoint{1.090278in}{0.593333in}}%
\pgfusepath{stroke}%
\end{pgfscope}%
\begin{pgfscope}%
\definecolor{textcolor}{rgb}{0.000000,0.000000,0.000000}%
\pgfsetstrokecolor{textcolor}%
\pgfsetfillcolor{textcolor}%
\pgftext[x=1.201389in,y=0.544722in,left,base]{\color{textcolor}\rmfamily\fontsize{10.000000}{12.000000}\selectfont W32(x)}%
\end{pgfscope}%
\end{pgfpicture}%
\makeatother%
\endgroup%
        
    \end{center}
    \caption{Węzły jednorodne, funkcja \(\tilde{y}\), \(N=6,8,16,32\)}
\end{figure}

\begin{figure}[h]
    \begin{center}
        %% Creator: Matplotlib, PGF backend
%%
%% To include the figure in your LaTeX document, write
%%   \input{<filename>.pgf}
%%
%% Make sure the required packages are loaded in your preamble
%%   \usepackage{pgf}
%%
%% Figures using additional raster images can only be included by \input if
%% they are in the same directory as the main LaTeX file. For loading figures
%% from other directories you can use the `import` package
%%   \usepackage{import}
%% and then include the figures with
%%   \import{<path to file>}{<filename>.pgf}
%%
%% Matplotlib used the following preamble
%%
\begingroup%
\makeatletter%
\begin{pgfpicture}%
\pgfpathrectangle{\pgfpointorigin}{\pgfqpoint{5.500000in}{3.500000in}}%
\pgfusepath{use as bounding box, clip}%
\begin{pgfscope}%
\pgfsetbuttcap%
\pgfsetmiterjoin%
\definecolor{currentfill}{rgb}{1.000000,1.000000,1.000000}%
\pgfsetfillcolor{currentfill}%
\pgfsetlinewidth{0.000000pt}%
\definecolor{currentstroke}{rgb}{1.000000,1.000000,1.000000}%
\pgfsetstrokecolor{currentstroke}%
\pgfsetdash{}{0pt}%
\pgfpathmoveto{\pgfqpoint{0.000000in}{0.000000in}}%
\pgfpathlineto{\pgfqpoint{5.500000in}{0.000000in}}%
\pgfpathlineto{\pgfqpoint{5.500000in}{3.500000in}}%
\pgfpathlineto{\pgfqpoint{0.000000in}{3.500000in}}%
\pgfpathclose%
\pgfusepath{fill}%
\end{pgfscope}%
\begin{pgfscope}%
\pgfsetbuttcap%
\pgfsetmiterjoin%
\definecolor{currentfill}{rgb}{1.000000,1.000000,1.000000}%
\pgfsetfillcolor{currentfill}%
\pgfsetlinewidth{0.000000pt}%
\definecolor{currentstroke}{rgb}{0.000000,0.000000,0.000000}%
\pgfsetstrokecolor{currentstroke}%
\pgfsetstrokeopacity{0.000000}%
\pgfsetdash{}{0pt}%
\pgfpathmoveto{\pgfqpoint{0.687500in}{0.385000in}}%
\pgfpathlineto{\pgfqpoint{4.950000in}{0.385000in}}%
\pgfpathlineto{\pgfqpoint{4.950000in}{3.080000in}}%
\pgfpathlineto{\pgfqpoint{0.687500in}{3.080000in}}%
\pgfpathclose%
\pgfusepath{fill}%
\end{pgfscope}%
\begin{pgfscope}%
\pgfsetbuttcap%
\pgfsetroundjoin%
\definecolor{currentfill}{rgb}{0.000000,0.000000,0.000000}%
\pgfsetfillcolor{currentfill}%
\pgfsetlinewidth{0.803000pt}%
\definecolor{currentstroke}{rgb}{0.000000,0.000000,0.000000}%
\pgfsetstrokecolor{currentstroke}%
\pgfsetdash{}{0pt}%
\pgfsys@defobject{currentmarker}{\pgfqpoint{0.000000in}{-0.048611in}}{\pgfqpoint{0.000000in}{0.000000in}}{%
\pgfpathmoveto{\pgfqpoint{0.000000in}{0.000000in}}%
\pgfpathlineto{\pgfqpoint{0.000000in}{-0.048611in}}%
\pgfusepath{stroke,fill}%
}%
\begin{pgfscope}%
\pgfsys@transformshift{0.881250in}{0.385000in}%
\pgfsys@useobject{currentmarker}{}%
\end{pgfscope}%
\end{pgfscope}%
\begin{pgfscope}%
\definecolor{textcolor}{rgb}{0.000000,0.000000,0.000000}%
\pgfsetstrokecolor{textcolor}%
\pgfsetfillcolor{textcolor}%
\pgftext[x=0.881250in,y=0.287778in,,top]{\color{textcolor}\rmfamily\fontsize{10.000000}{12.000000}\selectfont \(\displaystyle -1.00\)}%
\end{pgfscope}%
\begin{pgfscope}%
\pgfsetbuttcap%
\pgfsetroundjoin%
\definecolor{currentfill}{rgb}{0.000000,0.000000,0.000000}%
\pgfsetfillcolor{currentfill}%
\pgfsetlinewidth{0.803000pt}%
\definecolor{currentstroke}{rgb}{0.000000,0.000000,0.000000}%
\pgfsetstrokecolor{currentstroke}%
\pgfsetdash{}{0pt}%
\pgfsys@defobject{currentmarker}{\pgfqpoint{0.000000in}{-0.048611in}}{\pgfqpoint{0.000000in}{0.000000in}}{%
\pgfpathmoveto{\pgfqpoint{0.000000in}{0.000000in}}%
\pgfpathlineto{\pgfqpoint{0.000000in}{-0.048611in}}%
\pgfusepath{stroke,fill}%
}%
\begin{pgfscope}%
\pgfsys@transformshift{1.365625in}{0.385000in}%
\pgfsys@useobject{currentmarker}{}%
\end{pgfscope}%
\end{pgfscope}%
\begin{pgfscope}%
\definecolor{textcolor}{rgb}{0.000000,0.000000,0.000000}%
\pgfsetstrokecolor{textcolor}%
\pgfsetfillcolor{textcolor}%
\pgftext[x=1.365625in,y=0.287778in,,top]{\color{textcolor}\rmfamily\fontsize{10.000000}{12.000000}\selectfont \(\displaystyle -0.75\)}%
\end{pgfscope}%
\begin{pgfscope}%
\pgfsetbuttcap%
\pgfsetroundjoin%
\definecolor{currentfill}{rgb}{0.000000,0.000000,0.000000}%
\pgfsetfillcolor{currentfill}%
\pgfsetlinewidth{0.803000pt}%
\definecolor{currentstroke}{rgb}{0.000000,0.000000,0.000000}%
\pgfsetstrokecolor{currentstroke}%
\pgfsetdash{}{0pt}%
\pgfsys@defobject{currentmarker}{\pgfqpoint{0.000000in}{-0.048611in}}{\pgfqpoint{0.000000in}{0.000000in}}{%
\pgfpathmoveto{\pgfqpoint{0.000000in}{0.000000in}}%
\pgfpathlineto{\pgfqpoint{0.000000in}{-0.048611in}}%
\pgfusepath{stroke,fill}%
}%
\begin{pgfscope}%
\pgfsys@transformshift{1.850000in}{0.385000in}%
\pgfsys@useobject{currentmarker}{}%
\end{pgfscope}%
\end{pgfscope}%
\begin{pgfscope}%
\definecolor{textcolor}{rgb}{0.000000,0.000000,0.000000}%
\pgfsetstrokecolor{textcolor}%
\pgfsetfillcolor{textcolor}%
\pgftext[x=1.850000in,y=0.287778in,,top]{\color{textcolor}\rmfamily\fontsize{10.000000}{12.000000}\selectfont \(\displaystyle -0.50\)}%
\end{pgfscope}%
\begin{pgfscope}%
\pgfsetbuttcap%
\pgfsetroundjoin%
\definecolor{currentfill}{rgb}{0.000000,0.000000,0.000000}%
\pgfsetfillcolor{currentfill}%
\pgfsetlinewidth{0.803000pt}%
\definecolor{currentstroke}{rgb}{0.000000,0.000000,0.000000}%
\pgfsetstrokecolor{currentstroke}%
\pgfsetdash{}{0pt}%
\pgfsys@defobject{currentmarker}{\pgfqpoint{0.000000in}{-0.048611in}}{\pgfqpoint{0.000000in}{0.000000in}}{%
\pgfpathmoveto{\pgfqpoint{0.000000in}{0.000000in}}%
\pgfpathlineto{\pgfqpoint{0.000000in}{-0.048611in}}%
\pgfusepath{stroke,fill}%
}%
\begin{pgfscope}%
\pgfsys@transformshift{2.334375in}{0.385000in}%
\pgfsys@useobject{currentmarker}{}%
\end{pgfscope}%
\end{pgfscope}%
\begin{pgfscope}%
\definecolor{textcolor}{rgb}{0.000000,0.000000,0.000000}%
\pgfsetstrokecolor{textcolor}%
\pgfsetfillcolor{textcolor}%
\pgftext[x=2.334375in,y=0.287778in,,top]{\color{textcolor}\rmfamily\fontsize{10.000000}{12.000000}\selectfont \(\displaystyle -0.25\)}%
\end{pgfscope}%
\begin{pgfscope}%
\pgfsetbuttcap%
\pgfsetroundjoin%
\definecolor{currentfill}{rgb}{0.000000,0.000000,0.000000}%
\pgfsetfillcolor{currentfill}%
\pgfsetlinewidth{0.803000pt}%
\definecolor{currentstroke}{rgb}{0.000000,0.000000,0.000000}%
\pgfsetstrokecolor{currentstroke}%
\pgfsetdash{}{0pt}%
\pgfsys@defobject{currentmarker}{\pgfqpoint{0.000000in}{-0.048611in}}{\pgfqpoint{0.000000in}{0.000000in}}{%
\pgfpathmoveto{\pgfqpoint{0.000000in}{0.000000in}}%
\pgfpathlineto{\pgfqpoint{0.000000in}{-0.048611in}}%
\pgfusepath{stroke,fill}%
}%
\begin{pgfscope}%
\pgfsys@transformshift{2.818750in}{0.385000in}%
\pgfsys@useobject{currentmarker}{}%
\end{pgfscope}%
\end{pgfscope}%
\begin{pgfscope}%
\definecolor{textcolor}{rgb}{0.000000,0.000000,0.000000}%
\pgfsetstrokecolor{textcolor}%
\pgfsetfillcolor{textcolor}%
\pgftext[x=2.818750in,y=0.287778in,,top]{\color{textcolor}\rmfamily\fontsize{10.000000}{12.000000}\selectfont \(\displaystyle 0.00\)}%
\end{pgfscope}%
\begin{pgfscope}%
\pgfsetbuttcap%
\pgfsetroundjoin%
\definecolor{currentfill}{rgb}{0.000000,0.000000,0.000000}%
\pgfsetfillcolor{currentfill}%
\pgfsetlinewidth{0.803000pt}%
\definecolor{currentstroke}{rgb}{0.000000,0.000000,0.000000}%
\pgfsetstrokecolor{currentstroke}%
\pgfsetdash{}{0pt}%
\pgfsys@defobject{currentmarker}{\pgfqpoint{0.000000in}{-0.048611in}}{\pgfqpoint{0.000000in}{0.000000in}}{%
\pgfpathmoveto{\pgfqpoint{0.000000in}{0.000000in}}%
\pgfpathlineto{\pgfqpoint{0.000000in}{-0.048611in}}%
\pgfusepath{stroke,fill}%
}%
\begin{pgfscope}%
\pgfsys@transformshift{3.303125in}{0.385000in}%
\pgfsys@useobject{currentmarker}{}%
\end{pgfscope}%
\end{pgfscope}%
\begin{pgfscope}%
\definecolor{textcolor}{rgb}{0.000000,0.000000,0.000000}%
\pgfsetstrokecolor{textcolor}%
\pgfsetfillcolor{textcolor}%
\pgftext[x=3.303125in,y=0.287778in,,top]{\color{textcolor}\rmfamily\fontsize{10.000000}{12.000000}\selectfont \(\displaystyle 0.25\)}%
\end{pgfscope}%
\begin{pgfscope}%
\pgfsetbuttcap%
\pgfsetroundjoin%
\definecolor{currentfill}{rgb}{0.000000,0.000000,0.000000}%
\pgfsetfillcolor{currentfill}%
\pgfsetlinewidth{0.803000pt}%
\definecolor{currentstroke}{rgb}{0.000000,0.000000,0.000000}%
\pgfsetstrokecolor{currentstroke}%
\pgfsetdash{}{0pt}%
\pgfsys@defobject{currentmarker}{\pgfqpoint{0.000000in}{-0.048611in}}{\pgfqpoint{0.000000in}{0.000000in}}{%
\pgfpathmoveto{\pgfqpoint{0.000000in}{0.000000in}}%
\pgfpathlineto{\pgfqpoint{0.000000in}{-0.048611in}}%
\pgfusepath{stroke,fill}%
}%
\begin{pgfscope}%
\pgfsys@transformshift{3.787500in}{0.385000in}%
\pgfsys@useobject{currentmarker}{}%
\end{pgfscope}%
\end{pgfscope}%
\begin{pgfscope}%
\definecolor{textcolor}{rgb}{0.000000,0.000000,0.000000}%
\pgfsetstrokecolor{textcolor}%
\pgfsetfillcolor{textcolor}%
\pgftext[x=3.787500in,y=0.287778in,,top]{\color{textcolor}\rmfamily\fontsize{10.000000}{12.000000}\selectfont \(\displaystyle 0.50\)}%
\end{pgfscope}%
\begin{pgfscope}%
\pgfsetbuttcap%
\pgfsetroundjoin%
\definecolor{currentfill}{rgb}{0.000000,0.000000,0.000000}%
\pgfsetfillcolor{currentfill}%
\pgfsetlinewidth{0.803000pt}%
\definecolor{currentstroke}{rgb}{0.000000,0.000000,0.000000}%
\pgfsetstrokecolor{currentstroke}%
\pgfsetdash{}{0pt}%
\pgfsys@defobject{currentmarker}{\pgfqpoint{0.000000in}{-0.048611in}}{\pgfqpoint{0.000000in}{0.000000in}}{%
\pgfpathmoveto{\pgfqpoint{0.000000in}{0.000000in}}%
\pgfpathlineto{\pgfqpoint{0.000000in}{-0.048611in}}%
\pgfusepath{stroke,fill}%
}%
\begin{pgfscope}%
\pgfsys@transformshift{4.271875in}{0.385000in}%
\pgfsys@useobject{currentmarker}{}%
\end{pgfscope}%
\end{pgfscope}%
\begin{pgfscope}%
\definecolor{textcolor}{rgb}{0.000000,0.000000,0.000000}%
\pgfsetstrokecolor{textcolor}%
\pgfsetfillcolor{textcolor}%
\pgftext[x=4.271875in,y=0.287778in,,top]{\color{textcolor}\rmfamily\fontsize{10.000000}{12.000000}\selectfont \(\displaystyle 0.75\)}%
\end{pgfscope}%
\begin{pgfscope}%
\pgfsetbuttcap%
\pgfsetroundjoin%
\definecolor{currentfill}{rgb}{0.000000,0.000000,0.000000}%
\pgfsetfillcolor{currentfill}%
\pgfsetlinewidth{0.803000pt}%
\definecolor{currentstroke}{rgb}{0.000000,0.000000,0.000000}%
\pgfsetstrokecolor{currentstroke}%
\pgfsetdash{}{0pt}%
\pgfsys@defobject{currentmarker}{\pgfqpoint{0.000000in}{-0.048611in}}{\pgfqpoint{0.000000in}{0.000000in}}{%
\pgfpathmoveto{\pgfqpoint{0.000000in}{0.000000in}}%
\pgfpathlineto{\pgfqpoint{0.000000in}{-0.048611in}}%
\pgfusepath{stroke,fill}%
}%
\begin{pgfscope}%
\pgfsys@transformshift{4.756250in}{0.385000in}%
\pgfsys@useobject{currentmarker}{}%
\end{pgfscope}%
\end{pgfscope}%
\begin{pgfscope}%
\definecolor{textcolor}{rgb}{0.000000,0.000000,0.000000}%
\pgfsetstrokecolor{textcolor}%
\pgfsetfillcolor{textcolor}%
\pgftext[x=4.756250in,y=0.287778in,,top]{\color{textcolor}\rmfamily\fontsize{10.000000}{12.000000}\selectfont \(\displaystyle 1.00\)}%
\end{pgfscope}%
\begin{pgfscope}%
\definecolor{textcolor}{rgb}{0.000000,0.000000,0.000000}%
\pgfsetstrokecolor{textcolor}%
\pgfsetfillcolor{textcolor}%
\pgftext[x=2.818750in,y=0.108766in,,top]{\color{textcolor}\rmfamily\fontsize{10.000000}{12.000000}\selectfont x}%
\end{pgfscope}%
\begin{pgfscope}%
\pgfsetbuttcap%
\pgfsetroundjoin%
\definecolor{currentfill}{rgb}{0.000000,0.000000,0.000000}%
\pgfsetfillcolor{currentfill}%
\pgfsetlinewidth{0.803000pt}%
\definecolor{currentstroke}{rgb}{0.000000,0.000000,0.000000}%
\pgfsetstrokecolor{currentstroke}%
\pgfsetdash{}{0pt}%
\pgfsys@defobject{currentmarker}{\pgfqpoint{-0.048611in}{0.000000in}}{\pgfqpoint{0.000000in}{0.000000in}}{%
\pgfpathmoveto{\pgfqpoint{0.000000in}{0.000000in}}%
\pgfpathlineto{\pgfqpoint{-0.048611in}{0.000000in}}%
\pgfusepath{stroke,fill}%
}%
\begin{pgfscope}%
\pgfsys@transformshift{0.687500in}{0.474833in}%
\pgfsys@useobject{currentmarker}{}%
\end{pgfscope}%
\end{pgfscope}%
\begin{pgfscope}%
\definecolor{textcolor}{rgb}{0.000000,0.000000,0.000000}%
\pgfsetstrokecolor{textcolor}%
\pgfsetfillcolor{textcolor}%
\pgftext[x=0.304783in,y=0.426608in,left,base]{\color{textcolor}\rmfamily\fontsize{10.000000}{12.000000}\selectfont \(\displaystyle -0.2\)}%
\end{pgfscope}%
\begin{pgfscope}%
\pgfsetbuttcap%
\pgfsetroundjoin%
\definecolor{currentfill}{rgb}{0.000000,0.000000,0.000000}%
\pgfsetfillcolor{currentfill}%
\pgfsetlinewidth{0.803000pt}%
\definecolor{currentstroke}{rgb}{0.000000,0.000000,0.000000}%
\pgfsetstrokecolor{currentstroke}%
\pgfsetdash{}{0pt}%
\pgfsys@defobject{currentmarker}{\pgfqpoint{-0.048611in}{0.000000in}}{\pgfqpoint{0.000000in}{0.000000in}}{%
\pgfpathmoveto{\pgfqpoint{0.000000in}{0.000000in}}%
\pgfpathlineto{\pgfqpoint{-0.048611in}{0.000000in}}%
\pgfusepath{stroke,fill}%
}%
\begin{pgfscope}%
\pgfsys@transformshift{0.687500in}{0.834167in}%
\pgfsys@useobject{currentmarker}{}%
\end{pgfscope}%
\end{pgfscope}%
\begin{pgfscope}%
\definecolor{textcolor}{rgb}{0.000000,0.000000,0.000000}%
\pgfsetstrokecolor{textcolor}%
\pgfsetfillcolor{textcolor}%
\pgftext[x=0.412808in,y=0.785941in,left,base]{\color{textcolor}\rmfamily\fontsize{10.000000}{12.000000}\selectfont \(\displaystyle 0.0\)}%
\end{pgfscope}%
\begin{pgfscope}%
\pgfsetbuttcap%
\pgfsetroundjoin%
\definecolor{currentfill}{rgb}{0.000000,0.000000,0.000000}%
\pgfsetfillcolor{currentfill}%
\pgfsetlinewidth{0.803000pt}%
\definecolor{currentstroke}{rgb}{0.000000,0.000000,0.000000}%
\pgfsetstrokecolor{currentstroke}%
\pgfsetdash{}{0pt}%
\pgfsys@defobject{currentmarker}{\pgfqpoint{-0.048611in}{0.000000in}}{\pgfqpoint{0.000000in}{0.000000in}}{%
\pgfpathmoveto{\pgfqpoint{0.000000in}{0.000000in}}%
\pgfpathlineto{\pgfqpoint{-0.048611in}{0.000000in}}%
\pgfusepath{stroke,fill}%
}%
\begin{pgfscope}%
\pgfsys@transformshift{0.687500in}{1.193500in}%
\pgfsys@useobject{currentmarker}{}%
\end{pgfscope}%
\end{pgfscope}%
\begin{pgfscope}%
\definecolor{textcolor}{rgb}{0.000000,0.000000,0.000000}%
\pgfsetstrokecolor{textcolor}%
\pgfsetfillcolor{textcolor}%
\pgftext[x=0.412808in,y=1.145275in,left,base]{\color{textcolor}\rmfamily\fontsize{10.000000}{12.000000}\selectfont \(\displaystyle 0.2\)}%
\end{pgfscope}%
\begin{pgfscope}%
\pgfsetbuttcap%
\pgfsetroundjoin%
\definecolor{currentfill}{rgb}{0.000000,0.000000,0.000000}%
\pgfsetfillcolor{currentfill}%
\pgfsetlinewidth{0.803000pt}%
\definecolor{currentstroke}{rgb}{0.000000,0.000000,0.000000}%
\pgfsetstrokecolor{currentstroke}%
\pgfsetdash{}{0pt}%
\pgfsys@defobject{currentmarker}{\pgfqpoint{-0.048611in}{0.000000in}}{\pgfqpoint{0.000000in}{0.000000in}}{%
\pgfpathmoveto{\pgfqpoint{0.000000in}{0.000000in}}%
\pgfpathlineto{\pgfqpoint{-0.048611in}{0.000000in}}%
\pgfusepath{stroke,fill}%
}%
\begin{pgfscope}%
\pgfsys@transformshift{0.687500in}{1.552833in}%
\pgfsys@useobject{currentmarker}{}%
\end{pgfscope}%
\end{pgfscope}%
\begin{pgfscope}%
\definecolor{textcolor}{rgb}{0.000000,0.000000,0.000000}%
\pgfsetstrokecolor{textcolor}%
\pgfsetfillcolor{textcolor}%
\pgftext[x=0.412808in,y=1.504608in,left,base]{\color{textcolor}\rmfamily\fontsize{10.000000}{12.000000}\selectfont \(\displaystyle 0.4\)}%
\end{pgfscope}%
\begin{pgfscope}%
\pgfsetbuttcap%
\pgfsetroundjoin%
\definecolor{currentfill}{rgb}{0.000000,0.000000,0.000000}%
\pgfsetfillcolor{currentfill}%
\pgfsetlinewidth{0.803000pt}%
\definecolor{currentstroke}{rgb}{0.000000,0.000000,0.000000}%
\pgfsetstrokecolor{currentstroke}%
\pgfsetdash{}{0pt}%
\pgfsys@defobject{currentmarker}{\pgfqpoint{-0.048611in}{0.000000in}}{\pgfqpoint{0.000000in}{0.000000in}}{%
\pgfpathmoveto{\pgfqpoint{0.000000in}{0.000000in}}%
\pgfpathlineto{\pgfqpoint{-0.048611in}{0.000000in}}%
\pgfusepath{stroke,fill}%
}%
\begin{pgfscope}%
\pgfsys@transformshift{0.687500in}{1.912167in}%
\pgfsys@useobject{currentmarker}{}%
\end{pgfscope}%
\end{pgfscope}%
\begin{pgfscope}%
\definecolor{textcolor}{rgb}{0.000000,0.000000,0.000000}%
\pgfsetstrokecolor{textcolor}%
\pgfsetfillcolor{textcolor}%
\pgftext[x=0.412808in,y=1.863941in,left,base]{\color{textcolor}\rmfamily\fontsize{10.000000}{12.000000}\selectfont \(\displaystyle 0.6\)}%
\end{pgfscope}%
\begin{pgfscope}%
\pgfsetbuttcap%
\pgfsetroundjoin%
\definecolor{currentfill}{rgb}{0.000000,0.000000,0.000000}%
\pgfsetfillcolor{currentfill}%
\pgfsetlinewidth{0.803000pt}%
\definecolor{currentstroke}{rgb}{0.000000,0.000000,0.000000}%
\pgfsetstrokecolor{currentstroke}%
\pgfsetdash{}{0pt}%
\pgfsys@defobject{currentmarker}{\pgfqpoint{-0.048611in}{0.000000in}}{\pgfqpoint{0.000000in}{0.000000in}}{%
\pgfpathmoveto{\pgfqpoint{0.000000in}{0.000000in}}%
\pgfpathlineto{\pgfqpoint{-0.048611in}{0.000000in}}%
\pgfusepath{stroke,fill}%
}%
\begin{pgfscope}%
\pgfsys@transformshift{0.687500in}{2.271500in}%
\pgfsys@useobject{currentmarker}{}%
\end{pgfscope}%
\end{pgfscope}%
\begin{pgfscope}%
\definecolor{textcolor}{rgb}{0.000000,0.000000,0.000000}%
\pgfsetstrokecolor{textcolor}%
\pgfsetfillcolor{textcolor}%
\pgftext[x=0.412808in,y=2.223275in,left,base]{\color{textcolor}\rmfamily\fontsize{10.000000}{12.000000}\selectfont \(\displaystyle 0.8\)}%
\end{pgfscope}%
\begin{pgfscope}%
\pgfsetbuttcap%
\pgfsetroundjoin%
\definecolor{currentfill}{rgb}{0.000000,0.000000,0.000000}%
\pgfsetfillcolor{currentfill}%
\pgfsetlinewidth{0.803000pt}%
\definecolor{currentstroke}{rgb}{0.000000,0.000000,0.000000}%
\pgfsetstrokecolor{currentstroke}%
\pgfsetdash{}{0pt}%
\pgfsys@defobject{currentmarker}{\pgfqpoint{-0.048611in}{0.000000in}}{\pgfqpoint{0.000000in}{0.000000in}}{%
\pgfpathmoveto{\pgfqpoint{0.000000in}{0.000000in}}%
\pgfpathlineto{\pgfqpoint{-0.048611in}{0.000000in}}%
\pgfusepath{stroke,fill}%
}%
\begin{pgfscope}%
\pgfsys@transformshift{0.687500in}{2.630833in}%
\pgfsys@useobject{currentmarker}{}%
\end{pgfscope}%
\end{pgfscope}%
\begin{pgfscope}%
\definecolor{textcolor}{rgb}{0.000000,0.000000,0.000000}%
\pgfsetstrokecolor{textcolor}%
\pgfsetfillcolor{textcolor}%
\pgftext[x=0.412808in,y=2.582608in,left,base]{\color{textcolor}\rmfamily\fontsize{10.000000}{12.000000}\selectfont \(\displaystyle 1.0\)}%
\end{pgfscope}%
\begin{pgfscope}%
\pgfsetbuttcap%
\pgfsetroundjoin%
\definecolor{currentfill}{rgb}{0.000000,0.000000,0.000000}%
\pgfsetfillcolor{currentfill}%
\pgfsetlinewidth{0.803000pt}%
\definecolor{currentstroke}{rgb}{0.000000,0.000000,0.000000}%
\pgfsetstrokecolor{currentstroke}%
\pgfsetdash{}{0pt}%
\pgfsys@defobject{currentmarker}{\pgfqpoint{-0.048611in}{0.000000in}}{\pgfqpoint{0.000000in}{0.000000in}}{%
\pgfpathmoveto{\pgfqpoint{0.000000in}{0.000000in}}%
\pgfpathlineto{\pgfqpoint{-0.048611in}{0.000000in}}%
\pgfusepath{stroke,fill}%
}%
\begin{pgfscope}%
\pgfsys@transformshift{0.687500in}{2.990167in}%
\pgfsys@useobject{currentmarker}{}%
\end{pgfscope}%
\end{pgfscope}%
\begin{pgfscope}%
\definecolor{textcolor}{rgb}{0.000000,0.000000,0.000000}%
\pgfsetstrokecolor{textcolor}%
\pgfsetfillcolor{textcolor}%
\pgftext[x=0.412808in,y=2.941941in,left,base]{\color{textcolor}\rmfamily\fontsize{10.000000}{12.000000}\selectfont \(\displaystyle 1.2\)}%
\end{pgfscope}%
\begin{pgfscope}%
\definecolor{textcolor}{rgb}{0.000000,0.000000,0.000000}%
\pgfsetstrokecolor{textcolor}%
\pgfsetfillcolor{textcolor}%
\pgftext[x=0.249228in,y=1.732500in,,bottom,rotate=90.000000]{\color{textcolor}\rmfamily\fontsize{10.000000}{12.000000}\selectfont y}%
\end{pgfscope}%
\begin{pgfscope}%
\pgfpathrectangle{\pgfqpoint{0.687500in}{0.385000in}}{\pgfqpoint{4.262500in}{2.695000in}}%
\pgfusepath{clip}%
\pgfsetrectcap%
\pgfsetroundjoin%
\pgfsetlinewidth{1.505625pt}%
\definecolor{currentstroke}{rgb}{0.121569,0.466667,0.705882}%
\pgfsetstrokecolor{currentstroke}%
\pgfsetdash{}{0pt}%
\pgfpathmoveto{\pgfqpoint{0.881250in}{1.732500in}}%
\pgfpathlineto{\pgfqpoint{0.978612in}{1.778775in}}%
\pgfpathlineto{\pgfqpoint{1.075974in}{1.827296in}}%
\pgfpathlineto{\pgfqpoint{1.173335in}{1.878000in}}%
\pgfpathlineto{\pgfqpoint{1.290170in}{1.941556in}}%
\pgfpathlineto{\pgfqpoint{1.407004in}{2.007752in}}%
\pgfpathlineto{\pgfqpoint{1.543310in}{2.087643in}}%
\pgfpathlineto{\pgfqpoint{1.796451in}{2.239567in}}%
\pgfpathlineto{\pgfqpoint{1.932758in}{2.320107in}}%
\pgfpathlineto{\pgfqpoint{2.030119in}{2.375473in}}%
\pgfpathlineto{\pgfqpoint{2.108009in}{2.417737in}}%
\pgfpathlineto{\pgfqpoint{2.185898in}{2.457628in}}%
\pgfpathlineto{\pgfqpoint{2.263788in}{2.494605in}}%
\pgfpathlineto{\pgfqpoint{2.322205in}{2.520101in}}%
\pgfpathlineto{\pgfqpoint{2.380622in}{2.543430in}}%
\pgfpathlineto{\pgfqpoint{2.439039in}{2.564379in}}%
\pgfpathlineto{\pgfqpoint{2.497456in}{2.582749in}}%
\pgfpathlineto{\pgfqpoint{2.555873in}{2.598357in}}%
\pgfpathlineto{\pgfqpoint{2.614290in}{2.611046in}}%
\pgfpathlineto{\pgfqpoint{2.672707in}{2.620683in}}%
\pgfpathlineto{\pgfqpoint{2.731124in}{2.627166in}}%
\pgfpathlineto{\pgfqpoint{2.789541in}{2.630425in}}%
\pgfpathlineto{\pgfqpoint{2.847959in}{2.630425in}}%
\pgfpathlineto{\pgfqpoint{2.906376in}{2.627166in}}%
\pgfpathlineto{\pgfqpoint{2.964793in}{2.620683in}}%
\pgfpathlineto{\pgfqpoint{3.023210in}{2.611046in}}%
\pgfpathlineto{\pgfqpoint{3.081627in}{2.598357in}}%
\pgfpathlineto{\pgfqpoint{3.140044in}{2.582749in}}%
\pgfpathlineto{\pgfqpoint{3.198461in}{2.564379in}}%
\pgfpathlineto{\pgfqpoint{3.256878in}{2.543430in}}%
\pgfpathlineto{\pgfqpoint{3.315295in}{2.520101in}}%
\pgfpathlineto{\pgfqpoint{3.373712in}{2.494605in}}%
\pgfpathlineto{\pgfqpoint{3.451602in}{2.457628in}}%
\pgfpathlineto{\pgfqpoint{3.529491in}{2.417737in}}%
\pgfpathlineto{\pgfqpoint{3.607381in}{2.375473in}}%
\pgfpathlineto{\pgfqpoint{3.704742in}{2.320107in}}%
\pgfpathlineto{\pgfqpoint{3.841049in}{2.239567in}}%
\pgfpathlineto{\pgfqpoint{4.211024in}{2.019009in}}%
\pgfpathlineto{\pgfqpoint{4.327858in}{1.952417in}}%
\pgfpathlineto{\pgfqpoint{4.444692in}{1.888394in}}%
\pgfpathlineto{\pgfqpoint{4.542054in}{1.837265in}}%
\pgfpathlineto{\pgfqpoint{4.639416in}{1.788301in}}%
\pgfpathlineto{\pgfqpoint{4.736778in}{1.741574in}}%
\pgfpathlineto{\pgfqpoint{4.756250in}{1.732500in}}%
\pgfpathlineto{\pgfqpoint{4.756250in}{1.732500in}}%
\pgfusepath{stroke}%
\end{pgfscope}%
\begin{pgfscope}%
\pgfpathrectangle{\pgfqpoint{0.687500in}{0.385000in}}{\pgfqpoint{4.262500in}{2.695000in}}%
\pgfusepath{clip}%
\pgfsetbuttcap%
\pgfsetroundjoin%
\definecolor{currentfill}{rgb}{1.000000,0.498039,0.054902}%
\pgfsetfillcolor{currentfill}%
\pgfsetlinewidth{1.003750pt}%
\definecolor{currentstroke}{rgb}{1.000000,0.498039,0.054902}%
\pgfsetstrokecolor{currentstroke}%
\pgfsetdash{}{0pt}%
\pgfsys@defobject{currentmarker}{\pgfqpoint{-0.020833in}{-0.020833in}}{\pgfqpoint{0.020833in}{0.020833in}}{%
\pgfpathmoveto{\pgfqpoint{0.000000in}{-0.020833in}}%
\pgfpathcurveto{\pgfqpoint{0.005525in}{-0.020833in}}{\pgfqpoint{0.010825in}{-0.018638in}}{\pgfqpoint{0.014731in}{-0.014731in}}%
\pgfpathcurveto{\pgfqpoint{0.018638in}{-0.010825in}}{\pgfqpoint{0.020833in}{-0.005525in}}{\pgfqpoint{0.020833in}{0.000000in}}%
\pgfpathcurveto{\pgfqpoint{0.020833in}{0.005525in}}{\pgfqpoint{0.018638in}{0.010825in}}{\pgfqpoint{0.014731in}{0.014731in}}%
\pgfpathcurveto{\pgfqpoint{0.010825in}{0.018638in}}{\pgfqpoint{0.005525in}{0.020833in}}{\pgfqpoint{0.000000in}{0.020833in}}%
\pgfpathcurveto{\pgfqpoint{-0.005525in}{0.020833in}}{\pgfqpoint{-0.010825in}{0.018638in}}{\pgfqpoint{-0.014731in}{0.014731in}}%
\pgfpathcurveto{\pgfqpoint{-0.018638in}{0.010825in}}{\pgfqpoint{-0.020833in}{0.005525in}}{\pgfqpoint{-0.020833in}{0.000000in}}%
\pgfpathcurveto{\pgfqpoint{-0.020833in}{-0.005525in}}{\pgfqpoint{-0.018638in}{-0.010825in}}{\pgfqpoint{-0.014731in}{-0.014731in}}%
\pgfpathcurveto{\pgfqpoint{-0.010825in}{-0.018638in}}{\pgfqpoint{-0.005525in}{-0.020833in}}{\pgfqpoint{0.000000in}{-0.020833in}}%
\pgfpathclose%
\pgfusepath{stroke,fill}%
}%
\begin{pgfscope}%
\pgfsys@transformshift{0.881250in}{1.732500in}%
\pgfsys@useobject{currentmarker}{}%
\end{pgfscope}%
\begin{pgfscope}%
\pgfsys@transformshift{1.434821in}{2.023851in}%
\pgfsys@useobject{currentmarker}{}%
\end{pgfscope}%
\begin{pgfscope}%
\pgfsys@transformshift{1.988393in}{2.352040in}%
\pgfsys@useobject{currentmarker}{}%
\end{pgfscope}%
\begin{pgfscope}%
\pgfsys@transformshift{2.541964in}{2.594900in}%
\pgfsys@useobject{currentmarker}{}%
\end{pgfscope}%
\begin{pgfscope}%
\pgfsys@transformshift{3.095536in}{2.594900in}%
\pgfsys@useobject{currentmarker}{}%
\end{pgfscope}%
\begin{pgfscope}%
\pgfsys@transformshift{3.649107in}{2.352040in}%
\pgfsys@useobject{currentmarker}{}%
\end{pgfscope}%
\begin{pgfscope}%
\pgfsys@transformshift{4.202679in}{2.023851in}%
\pgfsys@useobject{currentmarker}{}%
\end{pgfscope}%
\end{pgfscope}%
\begin{pgfscope}%
\pgfpathrectangle{\pgfqpoint{0.687500in}{0.385000in}}{\pgfqpoint{4.262500in}{2.695000in}}%
\pgfusepath{clip}%
\pgfsetrectcap%
\pgfsetroundjoin%
\pgfsetlinewidth{1.505625pt}%
\definecolor{currentstroke}{rgb}{0.172549,0.627451,0.172549}%
\pgfsetstrokecolor{currentstroke}%
\pgfsetdash{}{0pt}%
\pgfpathmoveto{\pgfqpoint{0.881250in}{1.732500in}}%
\pgfpathlineto{\pgfqpoint{0.959139in}{1.774854in}}%
\pgfpathlineto{\pgfqpoint{1.075974in}{1.834958in}}%
\pgfpathlineto{\pgfqpoint{1.270697in}{1.934685in}}%
\pgfpathlineto{\pgfqpoint{1.387531in}{1.997531in}}%
\pgfpathlineto{\pgfqpoint{1.484893in}{2.052272in}}%
\pgfpathlineto{\pgfqpoint{1.601727in}{2.120554in}}%
\pgfpathlineto{\pgfqpoint{1.776979in}{2.226193in}}%
\pgfpathlineto{\pgfqpoint{1.932758in}{2.319641in}}%
\pgfpathlineto{\pgfqpoint{2.030119in}{2.375791in}}%
\pgfpathlineto{\pgfqpoint{2.108009in}{2.418528in}}%
\pgfpathlineto{\pgfqpoint{2.185898in}{2.458693in}}%
\pgfpathlineto{\pgfqpoint{2.244315in}{2.486791in}}%
\pgfpathlineto{\pgfqpoint{2.302732in}{2.512901in}}%
\pgfpathlineto{\pgfqpoint{2.361149in}{2.536813in}}%
\pgfpathlineto{\pgfqpoint{2.419567in}{2.558332in}}%
\pgfpathlineto{\pgfqpoint{2.477984in}{2.577280in}}%
\pgfpathlineto{\pgfqpoint{2.536401in}{2.593502in}}%
\pgfpathlineto{\pgfqpoint{2.594818in}{2.606863in}}%
\pgfpathlineto{\pgfqpoint{2.653235in}{2.617252in}}%
\pgfpathlineto{\pgfqpoint{2.711652in}{2.624582in}}%
\pgfpathlineto{\pgfqpoint{2.770069in}{2.628792in}}%
\pgfpathlineto{\pgfqpoint{2.828486in}{2.629847in}}%
\pgfpathlineto{\pgfqpoint{2.886903in}{2.627738in}}%
\pgfpathlineto{\pgfqpoint{2.945320in}{2.622483in}}%
\pgfpathlineto{\pgfqpoint{3.003737in}{2.614125in}}%
\pgfpathlineto{\pgfqpoint{3.062155in}{2.602735in}}%
\pgfpathlineto{\pgfqpoint{3.120572in}{2.588407in}}%
\pgfpathlineto{\pgfqpoint{3.178989in}{2.571260in}}%
\pgfpathlineto{\pgfqpoint{3.237406in}{2.551436in}}%
\pgfpathlineto{\pgfqpoint{3.295823in}{2.529099in}}%
\pgfpathlineto{\pgfqpoint{3.354240in}{2.504432in}}%
\pgfpathlineto{\pgfqpoint{3.412657in}{2.477635in}}%
\pgfpathlineto{\pgfqpoint{3.490546in}{2.438968in}}%
\pgfpathlineto{\pgfqpoint{3.568436in}{2.397445in}}%
\pgfpathlineto{\pgfqpoint{3.665798in}{2.342400in}}%
\pgfpathlineto{\pgfqpoint{3.782632in}{2.273257in}}%
\pgfpathlineto{\pgfqpoint{4.133134in}{2.063472in}}%
\pgfpathlineto{\pgfqpoint{4.249969in}{1.997531in}}%
\pgfpathlineto{\pgfqpoint{4.347330in}{1.944954in}}%
\pgfpathlineto{\pgfqpoint{4.464165in}{1.884341in}}%
\pgfpathlineto{\pgfqpoint{4.697833in}{1.764512in}}%
\pgfpathlineto{\pgfqpoint{4.756250in}{1.732500in}}%
\pgfpathlineto{\pgfqpoint{4.756250in}{1.732500in}}%
\pgfusepath{stroke}%
\end{pgfscope}%
\begin{pgfscope}%
\pgfpathrectangle{\pgfqpoint{0.687500in}{0.385000in}}{\pgfqpoint{4.262500in}{2.695000in}}%
\pgfusepath{clip}%
\pgfsetbuttcap%
\pgfsetroundjoin%
\definecolor{currentfill}{rgb}{0.839216,0.152941,0.156863}%
\pgfsetfillcolor{currentfill}%
\pgfsetlinewidth{1.003750pt}%
\definecolor{currentstroke}{rgb}{0.839216,0.152941,0.156863}%
\pgfsetstrokecolor{currentstroke}%
\pgfsetdash{}{0pt}%
\pgfsys@defobject{currentmarker}{\pgfqpoint{-0.020833in}{-0.020833in}}{\pgfqpoint{0.020833in}{0.020833in}}{%
\pgfpathmoveto{\pgfqpoint{0.000000in}{-0.020833in}}%
\pgfpathcurveto{\pgfqpoint{0.005525in}{-0.020833in}}{\pgfqpoint{0.010825in}{-0.018638in}}{\pgfqpoint{0.014731in}{-0.014731in}}%
\pgfpathcurveto{\pgfqpoint{0.018638in}{-0.010825in}}{\pgfqpoint{0.020833in}{-0.005525in}}{\pgfqpoint{0.020833in}{0.000000in}}%
\pgfpathcurveto{\pgfqpoint{0.020833in}{0.005525in}}{\pgfqpoint{0.018638in}{0.010825in}}{\pgfqpoint{0.014731in}{0.014731in}}%
\pgfpathcurveto{\pgfqpoint{0.010825in}{0.018638in}}{\pgfqpoint{0.005525in}{0.020833in}}{\pgfqpoint{0.000000in}{0.020833in}}%
\pgfpathcurveto{\pgfqpoint{-0.005525in}{0.020833in}}{\pgfqpoint{-0.010825in}{0.018638in}}{\pgfqpoint{-0.014731in}{0.014731in}}%
\pgfpathcurveto{\pgfqpoint{-0.018638in}{0.010825in}}{\pgfqpoint{-0.020833in}{0.005525in}}{\pgfqpoint{-0.020833in}{0.000000in}}%
\pgfpathcurveto{\pgfqpoint{-0.020833in}{-0.005525in}}{\pgfqpoint{-0.018638in}{-0.010825in}}{\pgfqpoint{-0.014731in}{-0.014731in}}%
\pgfpathcurveto{\pgfqpoint{-0.010825in}{-0.018638in}}{\pgfqpoint{-0.005525in}{-0.020833in}}{\pgfqpoint{0.000000in}{-0.020833in}}%
\pgfpathclose%
\pgfusepath{stroke,fill}%
}%
\begin{pgfscope}%
\pgfsys@transformshift{0.881250in}{1.732500in}%
\pgfsys@useobject{currentmarker}{}%
\end{pgfscope}%
\begin{pgfscope}%
\pgfsys@transformshift{0.940865in}{1.760566in}%
\pgfsys@useobject{currentmarker}{}%
\end{pgfscope}%
\begin{pgfscope}%
\pgfsys@transformshift{1.000481in}{1.789480in}%
\pgfsys@useobject{currentmarker}{}%
\end{pgfscope}%
\begin{pgfscope}%
\pgfsys@transformshift{1.060096in}{1.819232in}%
\pgfsys@useobject{currentmarker}{}%
\end{pgfscope}%
\begin{pgfscope}%
\pgfsys@transformshift{1.119712in}{1.849810in}%
\pgfsys@useobject{currentmarker}{}%
\end{pgfscope}%
\begin{pgfscope}%
\pgfsys@transformshift{1.179327in}{1.881190in}%
\pgfsys@useobject{currentmarker}{}%
\end{pgfscope}%
\begin{pgfscope}%
\pgfsys@transformshift{1.238942in}{1.913342in}%
\pgfsys@useobject{currentmarker}{}%
\end{pgfscope}%
\begin{pgfscope}%
\pgfsys@transformshift{1.298558in}{1.946226in}%
\pgfsys@useobject{currentmarker}{}%
\end{pgfscope}%
\begin{pgfscope}%
\pgfsys@transformshift{1.358173in}{1.979792in}%
\pgfsys@useobject{currentmarker}{}%
\end{pgfscope}%
\begin{pgfscope}%
\pgfsys@transformshift{1.417788in}{2.013980in}%
\pgfsys@useobject{currentmarker}{}%
\end{pgfscope}%
\begin{pgfscope}%
\pgfsys@transformshift{1.477404in}{2.048713in}%
\pgfsys@useobject{currentmarker}{}%
\end{pgfscope}%
\begin{pgfscope}%
\pgfsys@transformshift{1.537019in}{2.083906in}%
\pgfsys@useobject{currentmarker}{}%
\end{pgfscope}%
\begin{pgfscope}%
\pgfsys@transformshift{1.596635in}{2.119456in}%
\pgfsys@useobject{currentmarker}{}%
\end{pgfscope}%
\begin{pgfscope}%
\pgfsys@transformshift{1.656250in}{2.155245in}%
\pgfsys@useobject{currentmarker}{}%
\end{pgfscope}%
\begin{pgfscope}%
\pgfsys@transformshift{1.715865in}{2.191141in}%
\pgfsys@useobject{currentmarker}{}%
\end{pgfscope}%
\begin{pgfscope}%
\pgfsys@transformshift{1.775481in}{2.226995in}%
\pgfsys@useobject{currentmarker}{}%
\end{pgfscope}%
\begin{pgfscope}%
\pgfsys@transformshift{1.835096in}{2.262642in}%
\pgfsys@useobject{currentmarker}{}%
\end{pgfscope}%
\begin{pgfscope}%
\pgfsys@transformshift{1.894712in}{2.297899in}%
\pgfsys@useobject{currentmarker}{}%
\end{pgfscope}%
\begin{pgfscope}%
\pgfsys@transformshift{1.954327in}{2.332571in}%
\pgfsys@useobject{currentmarker}{}%
\end{pgfscope}%
\begin{pgfscope}%
\pgfsys@transformshift{2.013942in}{2.366447in}%
\pgfsys@useobject{currentmarker}{}%
\end{pgfscope}%
\begin{pgfscope}%
\pgfsys@transformshift{2.073558in}{2.399304in}%
\pgfsys@useobject{currentmarker}{}%
\end{pgfscope}%
\begin{pgfscope}%
\pgfsys@transformshift{2.133173in}{2.430910in}%
\pgfsys@useobject{currentmarker}{}%
\end{pgfscope}%
\begin{pgfscope}%
\pgfsys@transformshift{2.192788in}{2.461024in}%
\pgfsys@useobject{currentmarker}{}%
\end{pgfscope}%
\begin{pgfscope}%
\pgfsys@transformshift{2.252404in}{2.489404in}%
\pgfsys@useobject{currentmarker}{}%
\end{pgfscope}%
\begin{pgfscope}%
\pgfsys@transformshift{2.312019in}{2.515805in}%
\pgfsys@useobject{currentmarker}{}%
\end{pgfscope}%
\begin{pgfscope}%
\pgfsys@transformshift{2.371635in}{2.539991in}%
\pgfsys@useobject{currentmarker}{}%
\end{pgfscope}%
\begin{pgfscope}%
\pgfsys@transformshift{2.431250in}{2.561731in}%
\pgfsys@useobject{currentmarker}{}%
\end{pgfscope}%
\begin{pgfscope}%
\pgfsys@transformshift{2.490865in}{2.580811in}%
\pgfsys@useobject{currentmarker}{}%
\end{pgfscope}%
\begin{pgfscope}%
\pgfsys@transformshift{2.550481in}{2.597036in}%
\pgfsys@useobject{currentmarker}{}%
\end{pgfscope}%
\begin{pgfscope}%
\pgfsys@transformshift{2.610096in}{2.610235in}%
\pgfsys@useobject{currentmarker}{}%
\end{pgfscope}%
\begin{pgfscope}%
\pgfsys@transformshift{2.669712in}{2.620265in}%
\pgfsys@useobject{currentmarker}{}%
\end{pgfscope}%
\begin{pgfscope}%
\pgfsys@transformshift{2.729327in}{2.627014in}%
\pgfsys@useobject{currentmarker}{}%
\end{pgfscope}%
\begin{pgfscope}%
\pgfsys@transformshift{2.788942in}{2.630408in}%
\pgfsys@useobject{currentmarker}{}%
\end{pgfscope}%
\begin{pgfscope}%
\pgfsys@transformshift{2.848558in}{2.630408in}%
\pgfsys@useobject{currentmarker}{}%
\end{pgfscope}%
\begin{pgfscope}%
\pgfsys@transformshift{2.908173in}{2.627014in}%
\pgfsys@useobject{currentmarker}{}%
\end{pgfscope}%
\begin{pgfscope}%
\pgfsys@transformshift{2.967788in}{2.620265in}%
\pgfsys@useobject{currentmarker}{}%
\end{pgfscope}%
\begin{pgfscope}%
\pgfsys@transformshift{3.027404in}{2.610235in}%
\pgfsys@useobject{currentmarker}{}%
\end{pgfscope}%
\begin{pgfscope}%
\pgfsys@transformshift{3.087019in}{2.597036in}%
\pgfsys@useobject{currentmarker}{}%
\end{pgfscope}%
\begin{pgfscope}%
\pgfsys@transformshift{3.146635in}{2.580811in}%
\pgfsys@useobject{currentmarker}{}%
\end{pgfscope}%
\begin{pgfscope}%
\pgfsys@transformshift{3.206250in}{2.561731in}%
\pgfsys@useobject{currentmarker}{}%
\end{pgfscope}%
\begin{pgfscope}%
\pgfsys@transformshift{3.265865in}{2.539991in}%
\pgfsys@useobject{currentmarker}{}%
\end{pgfscope}%
\begin{pgfscope}%
\pgfsys@transformshift{3.325481in}{2.515805in}%
\pgfsys@useobject{currentmarker}{}%
\end{pgfscope}%
\begin{pgfscope}%
\pgfsys@transformshift{3.385096in}{2.489404in}%
\pgfsys@useobject{currentmarker}{}%
\end{pgfscope}%
\begin{pgfscope}%
\pgfsys@transformshift{3.444712in}{2.461024in}%
\pgfsys@useobject{currentmarker}{}%
\end{pgfscope}%
\begin{pgfscope}%
\pgfsys@transformshift{3.504327in}{2.430910in}%
\pgfsys@useobject{currentmarker}{}%
\end{pgfscope}%
\begin{pgfscope}%
\pgfsys@transformshift{3.563942in}{2.399304in}%
\pgfsys@useobject{currentmarker}{}%
\end{pgfscope}%
\begin{pgfscope}%
\pgfsys@transformshift{3.623558in}{2.366447in}%
\pgfsys@useobject{currentmarker}{}%
\end{pgfscope}%
\begin{pgfscope}%
\pgfsys@transformshift{3.683173in}{2.332571in}%
\pgfsys@useobject{currentmarker}{}%
\end{pgfscope}%
\begin{pgfscope}%
\pgfsys@transformshift{3.742788in}{2.297899in}%
\pgfsys@useobject{currentmarker}{}%
\end{pgfscope}%
\begin{pgfscope}%
\pgfsys@transformshift{3.802404in}{2.262642in}%
\pgfsys@useobject{currentmarker}{}%
\end{pgfscope}%
\begin{pgfscope}%
\pgfsys@transformshift{3.862019in}{2.226995in}%
\pgfsys@useobject{currentmarker}{}%
\end{pgfscope}%
\begin{pgfscope}%
\pgfsys@transformshift{3.921635in}{2.191141in}%
\pgfsys@useobject{currentmarker}{}%
\end{pgfscope}%
\begin{pgfscope}%
\pgfsys@transformshift{3.981250in}{2.155245in}%
\pgfsys@useobject{currentmarker}{}%
\end{pgfscope}%
\begin{pgfscope}%
\pgfsys@transformshift{4.040865in}{2.119456in}%
\pgfsys@useobject{currentmarker}{}%
\end{pgfscope}%
\begin{pgfscope}%
\pgfsys@transformshift{4.100481in}{2.083906in}%
\pgfsys@useobject{currentmarker}{}%
\end{pgfscope}%
\begin{pgfscope}%
\pgfsys@transformshift{4.160096in}{2.048713in}%
\pgfsys@useobject{currentmarker}{}%
\end{pgfscope}%
\begin{pgfscope}%
\pgfsys@transformshift{4.219712in}{2.013980in}%
\pgfsys@useobject{currentmarker}{}%
\end{pgfscope}%
\begin{pgfscope}%
\pgfsys@transformshift{4.279327in}{1.979792in}%
\pgfsys@useobject{currentmarker}{}%
\end{pgfscope}%
\begin{pgfscope}%
\pgfsys@transformshift{4.338942in}{1.946226in}%
\pgfsys@useobject{currentmarker}{}%
\end{pgfscope}%
\begin{pgfscope}%
\pgfsys@transformshift{4.398558in}{1.913342in}%
\pgfsys@useobject{currentmarker}{}%
\end{pgfscope}%
\begin{pgfscope}%
\pgfsys@transformshift{4.458173in}{1.881190in}%
\pgfsys@useobject{currentmarker}{}%
\end{pgfscope}%
\begin{pgfscope}%
\pgfsys@transformshift{4.517788in}{1.849810in}%
\pgfsys@useobject{currentmarker}{}%
\end{pgfscope}%
\begin{pgfscope}%
\pgfsys@transformshift{4.577404in}{1.819232in}%
\pgfsys@useobject{currentmarker}{}%
\end{pgfscope}%
\begin{pgfscope}%
\pgfsys@transformshift{4.637019in}{1.789480in}%
\pgfsys@useobject{currentmarker}{}%
\end{pgfscope}%
\begin{pgfscope}%
\pgfsys@transformshift{4.696635in}{1.760566in}%
\pgfsys@useobject{currentmarker}{}%
\end{pgfscope}%
\end{pgfscope}%
\begin{pgfscope}%
\pgfpathrectangle{\pgfqpoint{0.687500in}{0.385000in}}{\pgfqpoint{4.262500in}{2.695000in}}%
\pgfusepath{clip}%
\pgfsetrectcap%
\pgfsetroundjoin%
\pgfsetlinewidth{1.505625pt}%
\definecolor{currentstroke}{rgb}{0.580392,0.403922,0.741176}%
\pgfsetstrokecolor{currentstroke}%
\pgfsetdash{}{0pt}%
\pgfpathmoveto{\pgfqpoint{0.881250in}{1.732500in}}%
\pgfpathlineto{\pgfqpoint{0.886139in}{3.090000in}}%
\pgfpathmoveto{\pgfqpoint{0.912908in}{3.090000in}}%
\pgfpathlineto{\pgfqpoint{0.920195in}{0.669036in}}%
\pgfpathlineto{\pgfqpoint{0.939667in}{1.812073in}}%
\pgfpathlineto{\pgfqpoint{0.959139in}{1.548868in}}%
\pgfpathlineto{\pgfqpoint{0.978612in}{1.787590in}}%
\pgfpathlineto{\pgfqpoint{0.998084in}{1.788182in}}%
\pgfpathlineto{\pgfqpoint{1.017557in}{1.802828in}}%
\pgfpathlineto{\pgfqpoint{1.037029in}{1.804486in}}%
\pgfpathlineto{\pgfqpoint{1.056501in}{1.817729in}}%
\pgfpathlineto{\pgfqpoint{1.251225in}{1.920058in}}%
\pgfpathlineto{\pgfqpoint{1.368059in}{1.985421in}}%
\pgfpathlineto{\pgfqpoint{1.504366in}{2.064579in}}%
\pgfpathlineto{\pgfqpoint{1.679617in}{2.169310in}}%
\pgfpathlineto{\pgfqpoint{1.913285in}{2.308774in}}%
\pgfpathlineto{\pgfqpoint{2.010647in}{2.364599in}}%
\pgfpathlineto{\pgfqpoint{2.108009in}{2.417737in}}%
\pgfpathlineto{\pgfqpoint{2.185898in}{2.457628in}}%
\pgfpathlineto{\pgfqpoint{2.263788in}{2.494605in}}%
\pgfpathlineto{\pgfqpoint{2.322205in}{2.520101in}}%
\pgfpathlineto{\pgfqpoint{2.380622in}{2.543430in}}%
\pgfpathlineto{\pgfqpoint{2.439039in}{2.564379in}}%
\pgfpathlineto{\pgfqpoint{2.497456in}{2.582749in}}%
\pgfpathlineto{\pgfqpoint{2.555873in}{2.598357in}}%
\pgfpathlineto{\pgfqpoint{2.614290in}{2.611046in}}%
\pgfpathlineto{\pgfqpoint{2.672707in}{2.620683in}}%
\pgfpathlineto{\pgfqpoint{2.731124in}{2.627166in}}%
\pgfpathlineto{\pgfqpoint{2.789541in}{2.630425in}}%
\pgfpathlineto{\pgfqpoint{2.847959in}{2.630425in}}%
\pgfpathlineto{\pgfqpoint{2.906376in}{2.627166in}}%
\pgfpathlineto{\pgfqpoint{2.964793in}{2.620683in}}%
\pgfpathlineto{\pgfqpoint{3.023210in}{2.611046in}}%
\pgfpathlineto{\pgfqpoint{3.081627in}{2.598357in}}%
\pgfpathlineto{\pgfqpoint{3.140044in}{2.582749in}}%
\pgfpathlineto{\pgfqpoint{3.198461in}{2.564379in}}%
\pgfpathlineto{\pgfqpoint{3.256878in}{2.543430in}}%
\pgfpathlineto{\pgfqpoint{3.315295in}{2.520101in}}%
\pgfpathlineto{\pgfqpoint{3.373712in}{2.494605in}}%
\pgfpathlineto{\pgfqpoint{3.451602in}{2.457628in}}%
\pgfpathlineto{\pgfqpoint{3.529491in}{2.417737in}}%
\pgfpathlineto{\pgfqpoint{3.607381in}{2.375473in}}%
\pgfpathlineto{\pgfqpoint{3.704742in}{2.320107in}}%
\pgfpathlineto{\pgfqpoint{3.841049in}{2.239567in}}%
\pgfpathlineto{\pgfqpoint{4.211024in}{2.019009in}}%
\pgfpathlineto{\pgfqpoint{4.327858in}{1.952417in}}%
\pgfpathlineto{\pgfqpoint{4.444692in}{1.888379in}}%
\pgfpathlineto{\pgfqpoint{4.522582in}{1.846562in}}%
\pgfpathlineto{\pgfqpoint{4.542054in}{1.834156in}}%
\pgfpathlineto{\pgfqpoint{4.580999in}{1.815057in}}%
\pgfpathlineto{\pgfqpoint{4.600471in}{1.794077in}}%
\pgfpathlineto{\pgfqpoint{4.619943in}{2.120388in}}%
\pgfpathlineto{\pgfqpoint{4.639416in}{1.733197in}}%
\pgfpathlineto{\pgfqpoint{4.654007in}{0.375000in}}%
\pgfpathmoveto{\pgfqpoint{4.690299in}{0.375000in}}%
\pgfpathlineto{\pgfqpoint{4.694823in}{3.090000in}}%
\pgfpathmoveto{\pgfqpoint{4.698060in}{3.090000in}}%
\pgfpathlineto{\pgfqpoint{4.698402in}{0.375000in}}%
\pgfpathmoveto{\pgfqpoint{4.721294in}{0.375000in}}%
\pgfpathlineto{\pgfqpoint{4.721367in}{3.090000in}}%
\pgfpathmoveto{\pgfqpoint{4.737697in}{3.090000in}}%
\pgfpathlineto{\pgfqpoint{4.737701in}{0.375000in}}%
\pgfpathlineto{\pgfqpoint{4.737701in}{0.375000in}}%
\pgfusepath{stroke}%
\end{pgfscope}%
\begin{pgfscope}%
\pgfsetrectcap%
\pgfsetmiterjoin%
\pgfsetlinewidth{0.803000pt}%
\definecolor{currentstroke}{rgb}{0.000000,0.000000,0.000000}%
\pgfsetstrokecolor{currentstroke}%
\pgfsetdash{}{0pt}%
\pgfpathmoveto{\pgfqpoint{0.687500in}{0.385000in}}%
\pgfpathlineto{\pgfqpoint{0.687500in}{3.080000in}}%
\pgfusepath{stroke}%
\end{pgfscope}%
\begin{pgfscope}%
\pgfsetrectcap%
\pgfsetmiterjoin%
\pgfsetlinewidth{0.803000pt}%
\definecolor{currentstroke}{rgb}{0.000000,0.000000,0.000000}%
\pgfsetstrokecolor{currentstroke}%
\pgfsetdash{}{0pt}%
\pgfpathmoveto{\pgfqpoint{4.950000in}{0.385000in}}%
\pgfpathlineto{\pgfqpoint{4.950000in}{3.080000in}}%
\pgfusepath{stroke}%
\end{pgfscope}%
\begin{pgfscope}%
\pgfsetrectcap%
\pgfsetmiterjoin%
\pgfsetlinewidth{0.803000pt}%
\definecolor{currentstroke}{rgb}{0.000000,0.000000,0.000000}%
\pgfsetstrokecolor{currentstroke}%
\pgfsetdash{}{0pt}%
\pgfpathmoveto{\pgfqpoint{0.687500in}{0.385000in}}%
\pgfpathlineto{\pgfqpoint{4.950000in}{0.385000in}}%
\pgfusepath{stroke}%
\end{pgfscope}%
\begin{pgfscope}%
\pgfsetrectcap%
\pgfsetmiterjoin%
\pgfsetlinewidth{0.803000pt}%
\definecolor{currentstroke}{rgb}{0.000000,0.000000,0.000000}%
\pgfsetstrokecolor{currentstroke}%
\pgfsetdash{}{0pt}%
\pgfpathmoveto{\pgfqpoint{0.687500in}{3.080000in}}%
\pgfpathlineto{\pgfqpoint{4.950000in}{3.080000in}}%
\pgfusepath{stroke}%
\end{pgfscope}%
\begin{pgfscope}%
\definecolor{textcolor}{rgb}{0.000000,0.000000,0.000000}%
\pgfsetstrokecolor{textcolor}%
\pgfsetfillcolor{textcolor}%
\pgftext[x=2.818750in,y=3.163333in,,base]{\color{textcolor}\rmfamily\fontsize{12.000000}{14.400000}\selectfont N=6, 64}%
\end{pgfscope}%
\begin{pgfscope}%
\pgfsetbuttcap%
\pgfsetmiterjoin%
\definecolor{currentfill}{rgb}{1.000000,1.000000,1.000000}%
\pgfsetfillcolor{currentfill}%
\pgfsetfillopacity{0.800000}%
\pgfsetlinewidth{1.003750pt}%
\definecolor{currentstroke}{rgb}{0.800000,0.800000,0.800000}%
\pgfsetstrokecolor{currentstroke}%
\pgfsetstrokeopacity{0.800000}%
\pgfsetdash{}{0pt}%
\pgfpathmoveto{\pgfqpoint{2.185877in}{0.454444in}}%
\pgfpathlineto{\pgfqpoint{3.451623in}{0.454444in}}%
\pgfpathquadraticcurveto{\pgfqpoint{3.479401in}{0.454444in}}{\pgfqpoint{3.479401in}{0.482222in}}%
\pgfpathlineto{\pgfqpoint{3.479401in}{1.244755in}}%
\pgfpathquadraticcurveto{\pgfqpoint{3.479401in}{1.272533in}}{\pgfqpoint{3.451623in}{1.272533in}}%
\pgfpathlineto{\pgfqpoint{2.185877in}{1.272533in}}%
\pgfpathquadraticcurveto{\pgfqpoint{2.158099in}{1.272533in}}{\pgfqpoint{2.158099in}{1.244755in}}%
\pgfpathlineto{\pgfqpoint{2.158099in}{0.482222in}}%
\pgfpathquadraticcurveto{\pgfqpoint{2.158099in}{0.454444in}}{\pgfqpoint{2.185877in}{0.454444in}}%
\pgfpathclose%
\pgfusepath{stroke,fill}%
\end{pgfscope}%
\begin{pgfscope}%
\pgfsetrectcap%
\pgfsetroundjoin%
\pgfsetlinewidth{1.505625pt}%
\definecolor{currentstroke}{rgb}{0.121569,0.466667,0.705882}%
\pgfsetstrokecolor{currentstroke}%
\pgfsetdash{}{0pt}%
\pgfpathmoveto{\pgfqpoint{2.213655in}{1.082123in}}%
\pgfpathlineto{\pgfqpoint{2.491432in}{1.082123in}}%
\pgfusepath{stroke}%
\end{pgfscope}%
\begin{pgfscope}%
\definecolor{textcolor}{rgb}{0.000000,0.000000,0.000000}%
\pgfsetstrokecolor{textcolor}%
\pgfsetfillcolor{textcolor}%
\pgftext[x=2.602544in,y=1.033512in,left,base]{\color{textcolor}\rmfamily\fontsize{10.000000}{12.000000}\selectfont \(\displaystyle y(x)=\)\(\displaystyle \frac{1}{1+x^{2}}\)}%
\end{pgfscope}%
\begin{pgfscope}%
\pgfsetrectcap%
\pgfsetroundjoin%
\pgfsetlinewidth{1.505625pt}%
\definecolor{currentstroke}{rgb}{0.172549,0.627451,0.172549}%
\pgfsetstrokecolor{currentstroke}%
\pgfsetdash{}{0pt}%
\pgfpathmoveto{\pgfqpoint{2.213655in}{0.801667in}}%
\pgfpathlineto{\pgfqpoint{2.491432in}{0.801667in}}%
\pgfusepath{stroke}%
\end{pgfscope}%
\begin{pgfscope}%
\definecolor{textcolor}{rgb}{0.000000,0.000000,0.000000}%
\pgfsetstrokecolor{textcolor}%
\pgfsetfillcolor{textcolor}%
\pgftext[x=2.602544in,y=0.753056in,left,base]{\color{textcolor}\rmfamily\fontsize{10.000000}{12.000000}\selectfont W6(x)}%
\end{pgfscope}%
\begin{pgfscope}%
\pgfsetrectcap%
\pgfsetroundjoin%
\pgfsetlinewidth{1.505625pt}%
\definecolor{currentstroke}{rgb}{0.580392,0.403922,0.741176}%
\pgfsetstrokecolor{currentstroke}%
\pgfsetdash{}{0pt}%
\pgfpathmoveto{\pgfqpoint{2.213655in}{0.593333in}}%
\pgfpathlineto{\pgfqpoint{2.491432in}{0.593333in}}%
\pgfusepath{stroke}%
\end{pgfscope}%
\begin{pgfscope}%
\definecolor{textcolor}{rgb}{0.000000,0.000000,0.000000}%
\pgfsetstrokecolor{textcolor}%
\pgfsetfillcolor{textcolor}%
\pgftext[x=2.602544in,y=0.544722in,left,base]{\color{textcolor}\rmfamily\fontsize{10.000000}{12.000000}\selectfont W64(x)}%
\end{pgfscope}%
\end{pgfpicture}%
\makeatother%
\endgroup%
        
    \end{center}
    \caption{Węzły jednorodne, funkcja \(\tilde{y}\), \(N=6,64\)}
\end{figure}

\begin{figure}[h]
    \begin{center}
        %% Creator: Matplotlib, PGF backend
%%
%% To include the figure in your LaTeX document, write
%%   \input{<filename>.pgf}
%%
%% Make sure the required packages are loaded in your preamble
%%   \usepackage{pgf}
%%
%% Figures using additional raster images can only be included by \input if
%% they are in the same directory as the main LaTeX file. For loading figures
%% from other directories you can use the `import` package
%%   \usepackage{import}
%% and then include the figures with
%%   \import{<path to file>}{<filename>.pgf}
%%
%% Matplotlib used the following preamble
%%
\begingroup%
\makeatletter%
\begin{pgfpicture}%
\pgfpathrectangle{\pgfpointorigin}{\pgfqpoint{5.500000in}{3.500000in}}%
\pgfusepath{use as bounding box, clip}%
\begin{pgfscope}%
\pgfsetbuttcap%
\pgfsetmiterjoin%
\definecolor{currentfill}{rgb}{1.000000,1.000000,1.000000}%
\pgfsetfillcolor{currentfill}%
\pgfsetlinewidth{0.000000pt}%
\definecolor{currentstroke}{rgb}{1.000000,1.000000,1.000000}%
\pgfsetstrokecolor{currentstroke}%
\pgfsetdash{}{0pt}%
\pgfpathmoveto{\pgfqpoint{0.000000in}{0.000000in}}%
\pgfpathlineto{\pgfqpoint{5.500000in}{0.000000in}}%
\pgfpathlineto{\pgfqpoint{5.500000in}{3.500000in}}%
\pgfpathlineto{\pgfqpoint{0.000000in}{3.500000in}}%
\pgfpathclose%
\pgfusepath{fill}%
\end{pgfscope}%
\begin{pgfscope}%
\pgfsetbuttcap%
\pgfsetmiterjoin%
\definecolor{currentfill}{rgb}{1.000000,1.000000,1.000000}%
\pgfsetfillcolor{currentfill}%
\pgfsetlinewidth{0.000000pt}%
\definecolor{currentstroke}{rgb}{0.000000,0.000000,0.000000}%
\pgfsetstrokecolor{currentstroke}%
\pgfsetstrokeopacity{0.000000}%
\pgfsetdash{}{0pt}%
\pgfpathmoveto{\pgfqpoint{0.687500in}{0.385000in}}%
\pgfpathlineto{\pgfqpoint{4.950000in}{0.385000in}}%
\pgfpathlineto{\pgfqpoint{4.950000in}{3.080000in}}%
\pgfpathlineto{\pgfqpoint{0.687500in}{3.080000in}}%
\pgfpathclose%
\pgfusepath{fill}%
\end{pgfscope}%
\begin{pgfscope}%
\pgfsetbuttcap%
\pgfsetroundjoin%
\definecolor{currentfill}{rgb}{0.000000,0.000000,0.000000}%
\pgfsetfillcolor{currentfill}%
\pgfsetlinewidth{0.803000pt}%
\definecolor{currentstroke}{rgb}{0.000000,0.000000,0.000000}%
\pgfsetstrokecolor{currentstroke}%
\pgfsetdash{}{0pt}%
\pgfsys@defobject{currentmarker}{\pgfqpoint{0.000000in}{-0.048611in}}{\pgfqpoint{0.000000in}{0.000000in}}{%
\pgfpathmoveto{\pgfqpoint{0.000000in}{0.000000in}}%
\pgfpathlineto{\pgfqpoint{0.000000in}{-0.048611in}}%
\pgfusepath{stroke,fill}%
}%
\begin{pgfscope}%
\pgfsys@transformshift{0.881250in}{0.385000in}%
\pgfsys@useobject{currentmarker}{}%
\end{pgfscope}%
\end{pgfscope}%
\begin{pgfscope}%
\definecolor{textcolor}{rgb}{0.000000,0.000000,0.000000}%
\pgfsetstrokecolor{textcolor}%
\pgfsetfillcolor{textcolor}%
\pgftext[x=0.881250in,y=0.287778in,,top]{\color{textcolor}\rmfamily\fontsize{10.000000}{12.000000}\selectfont \(\displaystyle -1.00\)}%
\end{pgfscope}%
\begin{pgfscope}%
\pgfsetbuttcap%
\pgfsetroundjoin%
\definecolor{currentfill}{rgb}{0.000000,0.000000,0.000000}%
\pgfsetfillcolor{currentfill}%
\pgfsetlinewidth{0.803000pt}%
\definecolor{currentstroke}{rgb}{0.000000,0.000000,0.000000}%
\pgfsetstrokecolor{currentstroke}%
\pgfsetdash{}{0pt}%
\pgfsys@defobject{currentmarker}{\pgfqpoint{0.000000in}{-0.048611in}}{\pgfqpoint{0.000000in}{0.000000in}}{%
\pgfpathmoveto{\pgfqpoint{0.000000in}{0.000000in}}%
\pgfpathlineto{\pgfqpoint{0.000000in}{-0.048611in}}%
\pgfusepath{stroke,fill}%
}%
\begin{pgfscope}%
\pgfsys@transformshift{1.365625in}{0.385000in}%
\pgfsys@useobject{currentmarker}{}%
\end{pgfscope}%
\end{pgfscope}%
\begin{pgfscope}%
\definecolor{textcolor}{rgb}{0.000000,0.000000,0.000000}%
\pgfsetstrokecolor{textcolor}%
\pgfsetfillcolor{textcolor}%
\pgftext[x=1.365625in,y=0.287778in,,top]{\color{textcolor}\rmfamily\fontsize{10.000000}{12.000000}\selectfont \(\displaystyle -0.75\)}%
\end{pgfscope}%
\begin{pgfscope}%
\pgfsetbuttcap%
\pgfsetroundjoin%
\definecolor{currentfill}{rgb}{0.000000,0.000000,0.000000}%
\pgfsetfillcolor{currentfill}%
\pgfsetlinewidth{0.803000pt}%
\definecolor{currentstroke}{rgb}{0.000000,0.000000,0.000000}%
\pgfsetstrokecolor{currentstroke}%
\pgfsetdash{}{0pt}%
\pgfsys@defobject{currentmarker}{\pgfqpoint{0.000000in}{-0.048611in}}{\pgfqpoint{0.000000in}{0.000000in}}{%
\pgfpathmoveto{\pgfqpoint{0.000000in}{0.000000in}}%
\pgfpathlineto{\pgfqpoint{0.000000in}{-0.048611in}}%
\pgfusepath{stroke,fill}%
}%
\begin{pgfscope}%
\pgfsys@transformshift{1.850000in}{0.385000in}%
\pgfsys@useobject{currentmarker}{}%
\end{pgfscope}%
\end{pgfscope}%
\begin{pgfscope}%
\definecolor{textcolor}{rgb}{0.000000,0.000000,0.000000}%
\pgfsetstrokecolor{textcolor}%
\pgfsetfillcolor{textcolor}%
\pgftext[x=1.850000in,y=0.287778in,,top]{\color{textcolor}\rmfamily\fontsize{10.000000}{12.000000}\selectfont \(\displaystyle -0.50\)}%
\end{pgfscope}%
\begin{pgfscope}%
\pgfsetbuttcap%
\pgfsetroundjoin%
\definecolor{currentfill}{rgb}{0.000000,0.000000,0.000000}%
\pgfsetfillcolor{currentfill}%
\pgfsetlinewidth{0.803000pt}%
\definecolor{currentstroke}{rgb}{0.000000,0.000000,0.000000}%
\pgfsetstrokecolor{currentstroke}%
\pgfsetdash{}{0pt}%
\pgfsys@defobject{currentmarker}{\pgfqpoint{0.000000in}{-0.048611in}}{\pgfqpoint{0.000000in}{0.000000in}}{%
\pgfpathmoveto{\pgfqpoint{0.000000in}{0.000000in}}%
\pgfpathlineto{\pgfqpoint{0.000000in}{-0.048611in}}%
\pgfusepath{stroke,fill}%
}%
\begin{pgfscope}%
\pgfsys@transformshift{2.334375in}{0.385000in}%
\pgfsys@useobject{currentmarker}{}%
\end{pgfscope}%
\end{pgfscope}%
\begin{pgfscope}%
\definecolor{textcolor}{rgb}{0.000000,0.000000,0.000000}%
\pgfsetstrokecolor{textcolor}%
\pgfsetfillcolor{textcolor}%
\pgftext[x=2.334375in,y=0.287778in,,top]{\color{textcolor}\rmfamily\fontsize{10.000000}{12.000000}\selectfont \(\displaystyle -0.25\)}%
\end{pgfscope}%
\begin{pgfscope}%
\pgfsetbuttcap%
\pgfsetroundjoin%
\definecolor{currentfill}{rgb}{0.000000,0.000000,0.000000}%
\pgfsetfillcolor{currentfill}%
\pgfsetlinewidth{0.803000pt}%
\definecolor{currentstroke}{rgb}{0.000000,0.000000,0.000000}%
\pgfsetstrokecolor{currentstroke}%
\pgfsetdash{}{0pt}%
\pgfsys@defobject{currentmarker}{\pgfqpoint{0.000000in}{-0.048611in}}{\pgfqpoint{0.000000in}{0.000000in}}{%
\pgfpathmoveto{\pgfqpoint{0.000000in}{0.000000in}}%
\pgfpathlineto{\pgfqpoint{0.000000in}{-0.048611in}}%
\pgfusepath{stroke,fill}%
}%
\begin{pgfscope}%
\pgfsys@transformshift{2.818750in}{0.385000in}%
\pgfsys@useobject{currentmarker}{}%
\end{pgfscope}%
\end{pgfscope}%
\begin{pgfscope}%
\definecolor{textcolor}{rgb}{0.000000,0.000000,0.000000}%
\pgfsetstrokecolor{textcolor}%
\pgfsetfillcolor{textcolor}%
\pgftext[x=2.818750in,y=0.287778in,,top]{\color{textcolor}\rmfamily\fontsize{10.000000}{12.000000}\selectfont \(\displaystyle 0.00\)}%
\end{pgfscope}%
\begin{pgfscope}%
\pgfsetbuttcap%
\pgfsetroundjoin%
\definecolor{currentfill}{rgb}{0.000000,0.000000,0.000000}%
\pgfsetfillcolor{currentfill}%
\pgfsetlinewidth{0.803000pt}%
\definecolor{currentstroke}{rgb}{0.000000,0.000000,0.000000}%
\pgfsetstrokecolor{currentstroke}%
\pgfsetdash{}{0pt}%
\pgfsys@defobject{currentmarker}{\pgfqpoint{0.000000in}{-0.048611in}}{\pgfqpoint{0.000000in}{0.000000in}}{%
\pgfpathmoveto{\pgfqpoint{0.000000in}{0.000000in}}%
\pgfpathlineto{\pgfqpoint{0.000000in}{-0.048611in}}%
\pgfusepath{stroke,fill}%
}%
\begin{pgfscope}%
\pgfsys@transformshift{3.303125in}{0.385000in}%
\pgfsys@useobject{currentmarker}{}%
\end{pgfscope}%
\end{pgfscope}%
\begin{pgfscope}%
\definecolor{textcolor}{rgb}{0.000000,0.000000,0.000000}%
\pgfsetstrokecolor{textcolor}%
\pgfsetfillcolor{textcolor}%
\pgftext[x=3.303125in,y=0.287778in,,top]{\color{textcolor}\rmfamily\fontsize{10.000000}{12.000000}\selectfont \(\displaystyle 0.25\)}%
\end{pgfscope}%
\begin{pgfscope}%
\pgfsetbuttcap%
\pgfsetroundjoin%
\definecolor{currentfill}{rgb}{0.000000,0.000000,0.000000}%
\pgfsetfillcolor{currentfill}%
\pgfsetlinewidth{0.803000pt}%
\definecolor{currentstroke}{rgb}{0.000000,0.000000,0.000000}%
\pgfsetstrokecolor{currentstroke}%
\pgfsetdash{}{0pt}%
\pgfsys@defobject{currentmarker}{\pgfqpoint{0.000000in}{-0.048611in}}{\pgfqpoint{0.000000in}{0.000000in}}{%
\pgfpathmoveto{\pgfqpoint{0.000000in}{0.000000in}}%
\pgfpathlineto{\pgfqpoint{0.000000in}{-0.048611in}}%
\pgfusepath{stroke,fill}%
}%
\begin{pgfscope}%
\pgfsys@transformshift{3.787500in}{0.385000in}%
\pgfsys@useobject{currentmarker}{}%
\end{pgfscope}%
\end{pgfscope}%
\begin{pgfscope}%
\definecolor{textcolor}{rgb}{0.000000,0.000000,0.000000}%
\pgfsetstrokecolor{textcolor}%
\pgfsetfillcolor{textcolor}%
\pgftext[x=3.787500in,y=0.287778in,,top]{\color{textcolor}\rmfamily\fontsize{10.000000}{12.000000}\selectfont \(\displaystyle 0.50\)}%
\end{pgfscope}%
\begin{pgfscope}%
\pgfsetbuttcap%
\pgfsetroundjoin%
\definecolor{currentfill}{rgb}{0.000000,0.000000,0.000000}%
\pgfsetfillcolor{currentfill}%
\pgfsetlinewidth{0.803000pt}%
\definecolor{currentstroke}{rgb}{0.000000,0.000000,0.000000}%
\pgfsetstrokecolor{currentstroke}%
\pgfsetdash{}{0pt}%
\pgfsys@defobject{currentmarker}{\pgfqpoint{0.000000in}{-0.048611in}}{\pgfqpoint{0.000000in}{0.000000in}}{%
\pgfpathmoveto{\pgfqpoint{0.000000in}{0.000000in}}%
\pgfpathlineto{\pgfqpoint{0.000000in}{-0.048611in}}%
\pgfusepath{stroke,fill}%
}%
\begin{pgfscope}%
\pgfsys@transformshift{4.271875in}{0.385000in}%
\pgfsys@useobject{currentmarker}{}%
\end{pgfscope}%
\end{pgfscope}%
\begin{pgfscope}%
\definecolor{textcolor}{rgb}{0.000000,0.000000,0.000000}%
\pgfsetstrokecolor{textcolor}%
\pgfsetfillcolor{textcolor}%
\pgftext[x=4.271875in,y=0.287778in,,top]{\color{textcolor}\rmfamily\fontsize{10.000000}{12.000000}\selectfont \(\displaystyle 0.75\)}%
\end{pgfscope}%
\begin{pgfscope}%
\pgfsetbuttcap%
\pgfsetroundjoin%
\definecolor{currentfill}{rgb}{0.000000,0.000000,0.000000}%
\pgfsetfillcolor{currentfill}%
\pgfsetlinewidth{0.803000pt}%
\definecolor{currentstroke}{rgb}{0.000000,0.000000,0.000000}%
\pgfsetstrokecolor{currentstroke}%
\pgfsetdash{}{0pt}%
\pgfsys@defobject{currentmarker}{\pgfqpoint{0.000000in}{-0.048611in}}{\pgfqpoint{0.000000in}{0.000000in}}{%
\pgfpathmoveto{\pgfqpoint{0.000000in}{0.000000in}}%
\pgfpathlineto{\pgfqpoint{0.000000in}{-0.048611in}}%
\pgfusepath{stroke,fill}%
}%
\begin{pgfscope}%
\pgfsys@transformshift{4.756250in}{0.385000in}%
\pgfsys@useobject{currentmarker}{}%
\end{pgfscope}%
\end{pgfscope}%
\begin{pgfscope}%
\definecolor{textcolor}{rgb}{0.000000,0.000000,0.000000}%
\pgfsetstrokecolor{textcolor}%
\pgfsetfillcolor{textcolor}%
\pgftext[x=4.756250in,y=0.287778in,,top]{\color{textcolor}\rmfamily\fontsize{10.000000}{12.000000}\selectfont \(\displaystyle 1.00\)}%
\end{pgfscope}%
\begin{pgfscope}%
\definecolor{textcolor}{rgb}{0.000000,0.000000,0.000000}%
\pgfsetstrokecolor{textcolor}%
\pgfsetfillcolor{textcolor}%
\pgftext[x=2.818750in,y=0.108766in,,top]{\color{textcolor}\rmfamily\fontsize{10.000000}{12.000000}\selectfont x}%
\end{pgfscope}%
\begin{pgfscope}%
\pgfsetbuttcap%
\pgfsetroundjoin%
\definecolor{currentfill}{rgb}{0.000000,0.000000,0.000000}%
\pgfsetfillcolor{currentfill}%
\pgfsetlinewidth{0.803000pt}%
\definecolor{currentstroke}{rgb}{0.000000,0.000000,0.000000}%
\pgfsetstrokecolor{currentstroke}%
\pgfsetdash{}{0pt}%
\pgfsys@defobject{currentmarker}{\pgfqpoint{-0.048611in}{0.000000in}}{\pgfqpoint{0.000000in}{0.000000in}}{%
\pgfpathmoveto{\pgfqpoint{0.000000in}{0.000000in}}%
\pgfpathlineto{\pgfqpoint{-0.048611in}{0.000000in}}%
\pgfusepath{stroke,fill}%
}%
\begin{pgfscope}%
\pgfsys@transformshift{0.687500in}{0.474833in}%
\pgfsys@useobject{currentmarker}{}%
\end{pgfscope}%
\end{pgfscope}%
\begin{pgfscope}%
\definecolor{textcolor}{rgb}{0.000000,0.000000,0.000000}%
\pgfsetstrokecolor{textcolor}%
\pgfsetfillcolor{textcolor}%
\pgftext[x=0.304783in,y=0.426608in,left,base]{\color{textcolor}\rmfamily\fontsize{10.000000}{12.000000}\selectfont \(\displaystyle -0.2\)}%
\end{pgfscope}%
\begin{pgfscope}%
\pgfsetbuttcap%
\pgfsetroundjoin%
\definecolor{currentfill}{rgb}{0.000000,0.000000,0.000000}%
\pgfsetfillcolor{currentfill}%
\pgfsetlinewidth{0.803000pt}%
\definecolor{currentstroke}{rgb}{0.000000,0.000000,0.000000}%
\pgfsetstrokecolor{currentstroke}%
\pgfsetdash{}{0pt}%
\pgfsys@defobject{currentmarker}{\pgfqpoint{-0.048611in}{0.000000in}}{\pgfqpoint{0.000000in}{0.000000in}}{%
\pgfpathmoveto{\pgfqpoint{0.000000in}{0.000000in}}%
\pgfpathlineto{\pgfqpoint{-0.048611in}{0.000000in}}%
\pgfusepath{stroke,fill}%
}%
\begin{pgfscope}%
\pgfsys@transformshift{0.687500in}{0.834167in}%
\pgfsys@useobject{currentmarker}{}%
\end{pgfscope}%
\end{pgfscope}%
\begin{pgfscope}%
\definecolor{textcolor}{rgb}{0.000000,0.000000,0.000000}%
\pgfsetstrokecolor{textcolor}%
\pgfsetfillcolor{textcolor}%
\pgftext[x=0.412808in,y=0.785941in,left,base]{\color{textcolor}\rmfamily\fontsize{10.000000}{12.000000}\selectfont \(\displaystyle 0.0\)}%
\end{pgfscope}%
\begin{pgfscope}%
\pgfsetbuttcap%
\pgfsetroundjoin%
\definecolor{currentfill}{rgb}{0.000000,0.000000,0.000000}%
\pgfsetfillcolor{currentfill}%
\pgfsetlinewidth{0.803000pt}%
\definecolor{currentstroke}{rgb}{0.000000,0.000000,0.000000}%
\pgfsetstrokecolor{currentstroke}%
\pgfsetdash{}{0pt}%
\pgfsys@defobject{currentmarker}{\pgfqpoint{-0.048611in}{0.000000in}}{\pgfqpoint{0.000000in}{0.000000in}}{%
\pgfpathmoveto{\pgfqpoint{0.000000in}{0.000000in}}%
\pgfpathlineto{\pgfqpoint{-0.048611in}{0.000000in}}%
\pgfusepath{stroke,fill}%
}%
\begin{pgfscope}%
\pgfsys@transformshift{0.687500in}{1.193500in}%
\pgfsys@useobject{currentmarker}{}%
\end{pgfscope}%
\end{pgfscope}%
\begin{pgfscope}%
\definecolor{textcolor}{rgb}{0.000000,0.000000,0.000000}%
\pgfsetstrokecolor{textcolor}%
\pgfsetfillcolor{textcolor}%
\pgftext[x=0.412808in,y=1.145275in,left,base]{\color{textcolor}\rmfamily\fontsize{10.000000}{12.000000}\selectfont \(\displaystyle 0.2\)}%
\end{pgfscope}%
\begin{pgfscope}%
\pgfsetbuttcap%
\pgfsetroundjoin%
\definecolor{currentfill}{rgb}{0.000000,0.000000,0.000000}%
\pgfsetfillcolor{currentfill}%
\pgfsetlinewidth{0.803000pt}%
\definecolor{currentstroke}{rgb}{0.000000,0.000000,0.000000}%
\pgfsetstrokecolor{currentstroke}%
\pgfsetdash{}{0pt}%
\pgfsys@defobject{currentmarker}{\pgfqpoint{-0.048611in}{0.000000in}}{\pgfqpoint{0.000000in}{0.000000in}}{%
\pgfpathmoveto{\pgfqpoint{0.000000in}{0.000000in}}%
\pgfpathlineto{\pgfqpoint{-0.048611in}{0.000000in}}%
\pgfusepath{stroke,fill}%
}%
\begin{pgfscope}%
\pgfsys@transformshift{0.687500in}{1.552833in}%
\pgfsys@useobject{currentmarker}{}%
\end{pgfscope}%
\end{pgfscope}%
\begin{pgfscope}%
\definecolor{textcolor}{rgb}{0.000000,0.000000,0.000000}%
\pgfsetstrokecolor{textcolor}%
\pgfsetfillcolor{textcolor}%
\pgftext[x=0.412808in,y=1.504608in,left,base]{\color{textcolor}\rmfamily\fontsize{10.000000}{12.000000}\selectfont \(\displaystyle 0.4\)}%
\end{pgfscope}%
\begin{pgfscope}%
\pgfsetbuttcap%
\pgfsetroundjoin%
\definecolor{currentfill}{rgb}{0.000000,0.000000,0.000000}%
\pgfsetfillcolor{currentfill}%
\pgfsetlinewidth{0.803000pt}%
\definecolor{currentstroke}{rgb}{0.000000,0.000000,0.000000}%
\pgfsetstrokecolor{currentstroke}%
\pgfsetdash{}{0pt}%
\pgfsys@defobject{currentmarker}{\pgfqpoint{-0.048611in}{0.000000in}}{\pgfqpoint{0.000000in}{0.000000in}}{%
\pgfpathmoveto{\pgfqpoint{0.000000in}{0.000000in}}%
\pgfpathlineto{\pgfqpoint{-0.048611in}{0.000000in}}%
\pgfusepath{stroke,fill}%
}%
\begin{pgfscope}%
\pgfsys@transformshift{0.687500in}{1.912167in}%
\pgfsys@useobject{currentmarker}{}%
\end{pgfscope}%
\end{pgfscope}%
\begin{pgfscope}%
\definecolor{textcolor}{rgb}{0.000000,0.000000,0.000000}%
\pgfsetstrokecolor{textcolor}%
\pgfsetfillcolor{textcolor}%
\pgftext[x=0.412808in,y=1.863941in,left,base]{\color{textcolor}\rmfamily\fontsize{10.000000}{12.000000}\selectfont \(\displaystyle 0.6\)}%
\end{pgfscope}%
\begin{pgfscope}%
\pgfsetbuttcap%
\pgfsetroundjoin%
\definecolor{currentfill}{rgb}{0.000000,0.000000,0.000000}%
\pgfsetfillcolor{currentfill}%
\pgfsetlinewidth{0.803000pt}%
\definecolor{currentstroke}{rgb}{0.000000,0.000000,0.000000}%
\pgfsetstrokecolor{currentstroke}%
\pgfsetdash{}{0pt}%
\pgfsys@defobject{currentmarker}{\pgfqpoint{-0.048611in}{0.000000in}}{\pgfqpoint{0.000000in}{0.000000in}}{%
\pgfpathmoveto{\pgfqpoint{0.000000in}{0.000000in}}%
\pgfpathlineto{\pgfqpoint{-0.048611in}{0.000000in}}%
\pgfusepath{stroke,fill}%
}%
\begin{pgfscope}%
\pgfsys@transformshift{0.687500in}{2.271500in}%
\pgfsys@useobject{currentmarker}{}%
\end{pgfscope}%
\end{pgfscope}%
\begin{pgfscope}%
\definecolor{textcolor}{rgb}{0.000000,0.000000,0.000000}%
\pgfsetstrokecolor{textcolor}%
\pgfsetfillcolor{textcolor}%
\pgftext[x=0.412808in,y=2.223275in,left,base]{\color{textcolor}\rmfamily\fontsize{10.000000}{12.000000}\selectfont \(\displaystyle 0.8\)}%
\end{pgfscope}%
\begin{pgfscope}%
\pgfsetbuttcap%
\pgfsetroundjoin%
\definecolor{currentfill}{rgb}{0.000000,0.000000,0.000000}%
\pgfsetfillcolor{currentfill}%
\pgfsetlinewidth{0.803000pt}%
\definecolor{currentstroke}{rgb}{0.000000,0.000000,0.000000}%
\pgfsetstrokecolor{currentstroke}%
\pgfsetdash{}{0pt}%
\pgfsys@defobject{currentmarker}{\pgfqpoint{-0.048611in}{0.000000in}}{\pgfqpoint{0.000000in}{0.000000in}}{%
\pgfpathmoveto{\pgfqpoint{0.000000in}{0.000000in}}%
\pgfpathlineto{\pgfqpoint{-0.048611in}{0.000000in}}%
\pgfusepath{stroke,fill}%
}%
\begin{pgfscope}%
\pgfsys@transformshift{0.687500in}{2.630833in}%
\pgfsys@useobject{currentmarker}{}%
\end{pgfscope}%
\end{pgfscope}%
\begin{pgfscope}%
\definecolor{textcolor}{rgb}{0.000000,0.000000,0.000000}%
\pgfsetstrokecolor{textcolor}%
\pgfsetfillcolor{textcolor}%
\pgftext[x=0.412808in,y=2.582608in,left,base]{\color{textcolor}\rmfamily\fontsize{10.000000}{12.000000}\selectfont \(\displaystyle 1.0\)}%
\end{pgfscope}%
\begin{pgfscope}%
\pgfsetbuttcap%
\pgfsetroundjoin%
\definecolor{currentfill}{rgb}{0.000000,0.000000,0.000000}%
\pgfsetfillcolor{currentfill}%
\pgfsetlinewidth{0.803000pt}%
\definecolor{currentstroke}{rgb}{0.000000,0.000000,0.000000}%
\pgfsetstrokecolor{currentstroke}%
\pgfsetdash{}{0pt}%
\pgfsys@defobject{currentmarker}{\pgfqpoint{-0.048611in}{0.000000in}}{\pgfqpoint{0.000000in}{0.000000in}}{%
\pgfpathmoveto{\pgfqpoint{0.000000in}{0.000000in}}%
\pgfpathlineto{\pgfqpoint{-0.048611in}{0.000000in}}%
\pgfusepath{stroke,fill}%
}%
\begin{pgfscope}%
\pgfsys@transformshift{0.687500in}{2.990167in}%
\pgfsys@useobject{currentmarker}{}%
\end{pgfscope}%
\end{pgfscope}%
\begin{pgfscope}%
\definecolor{textcolor}{rgb}{0.000000,0.000000,0.000000}%
\pgfsetstrokecolor{textcolor}%
\pgfsetfillcolor{textcolor}%
\pgftext[x=0.412808in,y=2.941941in,left,base]{\color{textcolor}\rmfamily\fontsize{10.000000}{12.000000}\selectfont \(\displaystyle 1.2\)}%
\end{pgfscope}%
\begin{pgfscope}%
\definecolor{textcolor}{rgb}{0.000000,0.000000,0.000000}%
\pgfsetstrokecolor{textcolor}%
\pgfsetfillcolor{textcolor}%
\pgftext[x=0.249228in,y=1.732500in,,bottom,rotate=90.000000]{\color{textcolor}\rmfamily\fontsize{10.000000}{12.000000}\selectfont y}%
\end{pgfscope}%
\begin{pgfscope}%
\pgfpathrectangle{\pgfqpoint{0.687500in}{0.385000in}}{\pgfqpoint{4.262500in}{2.695000in}}%
\pgfusepath{clip}%
\pgfsetrectcap%
\pgfsetroundjoin%
\pgfsetlinewidth{1.505625pt}%
\definecolor{currentstroke}{rgb}{0.121569,0.466667,0.705882}%
\pgfsetstrokecolor{currentstroke}%
\pgfsetdash{}{0pt}%
\pgfpathmoveto{\pgfqpoint{0.881250in}{1.732500in}}%
\pgfpathlineto{\pgfqpoint{0.978612in}{1.778775in}}%
\pgfpathlineto{\pgfqpoint{1.075974in}{1.827296in}}%
\pgfpathlineto{\pgfqpoint{1.173335in}{1.878000in}}%
\pgfpathlineto{\pgfqpoint{1.290170in}{1.941556in}}%
\pgfpathlineto{\pgfqpoint{1.407004in}{2.007752in}}%
\pgfpathlineto{\pgfqpoint{1.543310in}{2.087643in}}%
\pgfpathlineto{\pgfqpoint{1.796451in}{2.239567in}}%
\pgfpathlineto{\pgfqpoint{1.932758in}{2.320107in}}%
\pgfpathlineto{\pgfqpoint{2.030119in}{2.375473in}}%
\pgfpathlineto{\pgfqpoint{2.108009in}{2.417737in}}%
\pgfpathlineto{\pgfqpoint{2.185898in}{2.457628in}}%
\pgfpathlineto{\pgfqpoint{2.263788in}{2.494605in}}%
\pgfpathlineto{\pgfqpoint{2.322205in}{2.520101in}}%
\pgfpathlineto{\pgfqpoint{2.380622in}{2.543430in}}%
\pgfpathlineto{\pgfqpoint{2.439039in}{2.564379in}}%
\pgfpathlineto{\pgfqpoint{2.497456in}{2.582749in}}%
\pgfpathlineto{\pgfqpoint{2.555873in}{2.598357in}}%
\pgfpathlineto{\pgfqpoint{2.614290in}{2.611046in}}%
\pgfpathlineto{\pgfqpoint{2.672707in}{2.620683in}}%
\pgfpathlineto{\pgfqpoint{2.731124in}{2.627166in}}%
\pgfpathlineto{\pgfqpoint{2.789541in}{2.630425in}}%
\pgfpathlineto{\pgfqpoint{2.847959in}{2.630425in}}%
\pgfpathlineto{\pgfqpoint{2.906376in}{2.627166in}}%
\pgfpathlineto{\pgfqpoint{2.964793in}{2.620683in}}%
\pgfpathlineto{\pgfqpoint{3.023210in}{2.611046in}}%
\pgfpathlineto{\pgfqpoint{3.081627in}{2.598357in}}%
\pgfpathlineto{\pgfqpoint{3.140044in}{2.582749in}}%
\pgfpathlineto{\pgfqpoint{3.198461in}{2.564379in}}%
\pgfpathlineto{\pgfqpoint{3.256878in}{2.543430in}}%
\pgfpathlineto{\pgfqpoint{3.315295in}{2.520101in}}%
\pgfpathlineto{\pgfqpoint{3.373712in}{2.494605in}}%
\pgfpathlineto{\pgfqpoint{3.451602in}{2.457628in}}%
\pgfpathlineto{\pgfqpoint{3.529491in}{2.417737in}}%
\pgfpathlineto{\pgfqpoint{3.607381in}{2.375473in}}%
\pgfpathlineto{\pgfqpoint{3.704742in}{2.320107in}}%
\pgfpathlineto{\pgfqpoint{3.841049in}{2.239567in}}%
\pgfpathlineto{\pgfqpoint{4.211024in}{2.019009in}}%
\pgfpathlineto{\pgfqpoint{4.327858in}{1.952417in}}%
\pgfpathlineto{\pgfqpoint{4.444692in}{1.888394in}}%
\pgfpathlineto{\pgfqpoint{4.542054in}{1.837265in}}%
\pgfpathlineto{\pgfqpoint{4.639416in}{1.788301in}}%
\pgfpathlineto{\pgfqpoint{4.736778in}{1.741574in}}%
\pgfpathlineto{\pgfqpoint{4.756250in}{1.732500in}}%
\pgfpathlineto{\pgfqpoint{4.756250in}{1.732500in}}%
\pgfusepath{stroke}%
\end{pgfscope}%
\begin{pgfscope}%
\pgfpathrectangle{\pgfqpoint{0.687500in}{0.385000in}}{\pgfqpoint{4.262500in}{2.695000in}}%
\pgfusepath{clip}%
\pgfsetbuttcap%
\pgfsetroundjoin%
\definecolor{currentfill}{rgb}{1.000000,0.498039,0.054902}%
\pgfsetfillcolor{currentfill}%
\pgfsetlinewidth{1.003750pt}%
\definecolor{currentstroke}{rgb}{1.000000,0.498039,0.054902}%
\pgfsetstrokecolor{currentstroke}%
\pgfsetdash{}{0pt}%
\pgfsys@defobject{currentmarker}{\pgfqpoint{-0.020833in}{-0.020833in}}{\pgfqpoint{0.020833in}{0.020833in}}{%
\pgfpathmoveto{\pgfqpoint{0.000000in}{-0.020833in}}%
\pgfpathcurveto{\pgfqpoint{0.005525in}{-0.020833in}}{\pgfqpoint{0.010825in}{-0.018638in}}{\pgfqpoint{0.014731in}{-0.014731in}}%
\pgfpathcurveto{\pgfqpoint{0.018638in}{-0.010825in}}{\pgfqpoint{0.020833in}{-0.005525in}}{\pgfqpoint{0.020833in}{0.000000in}}%
\pgfpathcurveto{\pgfqpoint{0.020833in}{0.005525in}}{\pgfqpoint{0.018638in}{0.010825in}}{\pgfqpoint{0.014731in}{0.014731in}}%
\pgfpathcurveto{\pgfqpoint{0.010825in}{0.018638in}}{\pgfqpoint{0.005525in}{0.020833in}}{\pgfqpoint{0.000000in}{0.020833in}}%
\pgfpathcurveto{\pgfqpoint{-0.005525in}{0.020833in}}{\pgfqpoint{-0.010825in}{0.018638in}}{\pgfqpoint{-0.014731in}{0.014731in}}%
\pgfpathcurveto{\pgfqpoint{-0.018638in}{0.010825in}}{\pgfqpoint{-0.020833in}{0.005525in}}{\pgfqpoint{-0.020833in}{0.000000in}}%
\pgfpathcurveto{\pgfqpoint{-0.020833in}{-0.005525in}}{\pgfqpoint{-0.018638in}{-0.010825in}}{\pgfqpoint{-0.014731in}{-0.014731in}}%
\pgfpathcurveto{\pgfqpoint{-0.010825in}{-0.018638in}}{\pgfqpoint{-0.005525in}{-0.020833in}}{\pgfqpoint{0.000000in}{-0.020833in}}%
\pgfpathclose%
\pgfusepath{stroke,fill}%
}%
\begin{pgfscope}%
\pgfsys@transformshift{0.881250in}{1.732500in}%
\pgfsys@useobject{currentmarker}{}%
\end{pgfscope}%
\begin{pgfscope}%
\pgfsys@transformshift{1.527083in}{2.078013in}%
\pgfsys@useobject{currentmarker}{}%
\end{pgfscope}%
\begin{pgfscope}%
\pgfsys@transformshift{2.172917in}{2.451167in}%
\pgfsys@useobject{currentmarker}{}%
\end{pgfscope}%
\begin{pgfscope}%
\pgfsys@transformshift{2.818750in}{2.630833in}%
\pgfsys@useobject{currentmarker}{}%
\end{pgfscope}%
\begin{pgfscope}%
\pgfsys@transformshift{3.464583in}{2.451167in}%
\pgfsys@useobject{currentmarker}{}%
\end{pgfscope}%
\begin{pgfscope}%
\pgfsys@transformshift{4.110417in}{2.078013in}%
\pgfsys@useobject{currentmarker}{}%
\end{pgfscope}%
\end{pgfscope}%
\begin{pgfscope}%
\pgfpathrectangle{\pgfqpoint{0.687500in}{0.385000in}}{\pgfqpoint{4.262500in}{2.695000in}}%
\pgfusepath{clip}%
\pgfsetrectcap%
\pgfsetroundjoin%
\pgfsetlinewidth{1.505625pt}%
\definecolor{currentstroke}{rgb}{0.172549,0.627451,0.172549}%
\pgfsetstrokecolor{currentstroke}%
\pgfsetdash{}{0pt}%
\pgfpathmoveto{\pgfqpoint{0.881250in}{1.732500in}}%
\pgfpathlineto{\pgfqpoint{0.998084in}{1.787957in}}%
\pgfpathlineto{\pgfqpoint{1.095446in}{1.836465in}}%
\pgfpathlineto{\pgfqpoint{1.192808in}{1.887296in}}%
\pgfpathlineto{\pgfqpoint{1.290170in}{1.940453in}}%
\pgfpathlineto{\pgfqpoint{1.407004in}{2.007040in}}%
\pgfpathlineto{\pgfqpoint{1.543310in}{2.087750in}}%
\pgfpathlineto{\pgfqpoint{1.952230in}{2.332749in}}%
\pgfpathlineto{\pgfqpoint{2.049592in}{2.387163in}}%
\pgfpathlineto{\pgfqpoint{2.127481in}{2.428329in}}%
\pgfpathlineto{\pgfqpoint{2.205371in}{2.466884in}}%
\pgfpathlineto{\pgfqpoint{2.283260in}{2.502369in}}%
\pgfpathlineto{\pgfqpoint{2.341677in}{2.526698in}}%
\pgfpathlineto{\pgfqpoint{2.400094in}{2.548868in}}%
\pgfpathlineto{\pgfqpoint{2.458511in}{2.568707in}}%
\pgfpathlineto{\pgfqpoint{2.516928in}{2.586053in}}%
\pgfpathlineto{\pgfqpoint{2.575345in}{2.600755in}}%
\pgfpathlineto{\pgfqpoint{2.633763in}{2.612679in}}%
\pgfpathlineto{\pgfqpoint{2.692180in}{2.621703in}}%
\pgfpathlineto{\pgfqpoint{2.750597in}{2.627722in}}%
\pgfpathlineto{\pgfqpoint{2.809014in}{2.630651in}}%
\pgfpathlineto{\pgfqpoint{2.867431in}{2.630424in}}%
\pgfpathlineto{\pgfqpoint{2.925848in}{2.626999in}}%
\pgfpathlineto{\pgfqpoint{2.984265in}{2.620355in}}%
\pgfpathlineto{\pgfqpoint{3.042682in}{2.610497in}}%
\pgfpathlineto{\pgfqpoint{3.101099in}{2.597459in}}%
\pgfpathlineto{\pgfqpoint{3.159516in}{2.581301in}}%
\pgfpathlineto{\pgfqpoint{3.217933in}{2.562115in}}%
\pgfpathlineto{\pgfqpoint{3.276351in}{2.540023in}}%
\pgfpathlineto{\pgfqpoint{3.334768in}{2.515182in}}%
\pgfpathlineto{\pgfqpoint{3.393185in}{2.487785in}}%
\pgfpathlineto{\pgfqpoint{3.451602in}{2.458062in}}%
\pgfpathlineto{\pgfqpoint{3.510019in}{2.426279in}}%
\pgfpathlineto{\pgfqpoint{3.587908in}{2.381237in}}%
\pgfpathlineto{\pgfqpoint{3.685270in}{2.321869in}}%
\pgfpathlineto{\pgfqpoint{3.918938in}{2.177533in}}%
\pgfpathlineto{\pgfqpoint{3.977356in}{2.144111in}}%
\pgfpathlineto{\pgfqpoint{4.035773in}{2.113056in}}%
\pgfpathlineto{\pgfqpoint{4.094190in}{2.085115in}}%
\pgfpathlineto{\pgfqpoint{4.133134in}{2.068612in}}%
\pgfpathlineto{\pgfqpoint{4.172079in}{2.054101in}}%
\pgfpathlineto{\pgfqpoint{4.211024in}{2.041847in}}%
\pgfpathlineto{\pgfqpoint{4.249969in}{2.032127in}}%
\pgfpathlineto{\pgfqpoint{4.288913in}{2.025228in}}%
\pgfpathlineto{\pgfqpoint{4.327858in}{2.021452in}}%
\pgfpathlineto{\pgfqpoint{4.366803in}{2.021113in}}%
\pgfpathlineto{\pgfqpoint{4.405747in}{2.024537in}}%
\pgfpathlineto{\pgfqpoint{4.444692in}{2.032065in}}%
\pgfpathlineto{\pgfqpoint{4.464165in}{2.037478in}}%
\pgfpathlineto{\pgfqpoint{4.483637in}{2.044050in}}%
\pgfpathlineto{\pgfqpoint{4.503109in}{2.051828in}}%
\pgfpathlineto{\pgfqpoint{4.522582in}{2.060859in}}%
\pgfpathlineto{\pgfqpoint{4.542054in}{2.071190in}}%
\pgfpathlineto{\pgfqpoint{4.561526in}{2.082872in}}%
\pgfpathlineto{\pgfqpoint{4.580999in}{2.095952in}}%
\pgfpathlineto{\pgfqpoint{4.600471in}{2.110484in}}%
\pgfpathlineto{\pgfqpoint{4.619943in}{2.126516in}}%
\pgfpathlineto{\pgfqpoint{4.639416in}{2.144104in}}%
\pgfpathlineto{\pgfqpoint{4.658888in}{2.163299in}}%
\pgfpathlineto{\pgfqpoint{4.678361in}{2.184155in}}%
\pgfpathlineto{\pgfqpoint{4.697833in}{2.206729in}}%
\pgfpathlineto{\pgfqpoint{4.717305in}{2.231077in}}%
\pgfpathlineto{\pgfqpoint{4.736778in}{2.257254in}}%
\pgfpathlineto{\pgfqpoint{4.756250in}{2.285321in}}%
\pgfpathlineto{\pgfqpoint{4.756250in}{2.285321in}}%
\pgfusepath{stroke}%
\end{pgfscope}%
\begin{pgfscope}%
\pgfpathrectangle{\pgfqpoint{0.687500in}{0.385000in}}{\pgfqpoint{4.262500in}{2.695000in}}%
\pgfusepath{clip}%
\pgfsetbuttcap%
\pgfsetroundjoin%
\definecolor{currentfill}{rgb}{0.839216,0.152941,0.156863}%
\pgfsetfillcolor{currentfill}%
\pgfsetlinewidth{1.003750pt}%
\definecolor{currentstroke}{rgb}{0.839216,0.152941,0.156863}%
\pgfsetstrokecolor{currentstroke}%
\pgfsetdash{}{0pt}%
\pgfsys@defobject{currentmarker}{\pgfqpoint{-0.020833in}{-0.020833in}}{\pgfqpoint{0.020833in}{0.020833in}}{%
\pgfpathmoveto{\pgfqpoint{0.000000in}{-0.020833in}}%
\pgfpathcurveto{\pgfqpoint{0.005525in}{-0.020833in}}{\pgfqpoint{0.010825in}{-0.018638in}}{\pgfqpoint{0.014731in}{-0.014731in}}%
\pgfpathcurveto{\pgfqpoint{0.018638in}{-0.010825in}}{\pgfqpoint{0.020833in}{-0.005525in}}{\pgfqpoint{0.020833in}{0.000000in}}%
\pgfpathcurveto{\pgfqpoint{0.020833in}{0.005525in}}{\pgfqpoint{0.018638in}{0.010825in}}{\pgfqpoint{0.014731in}{0.014731in}}%
\pgfpathcurveto{\pgfqpoint{0.010825in}{0.018638in}}{\pgfqpoint{0.005525in}{0.020833in}}{\pgfqpoint{0.000000in}{0.020833in}}%
\pgfpathcurveto{\pgfqpoint{-0.005525in}{0.020833in}}{\pgfqpoint{-0.010825in}{0.018638in}}{\pgfqpoint{-0.014731in}{0.014731in}}%
\pgfpathcurveto{\pgfqpoint{-0.018638in}{0.010825in}}{\pgfqpoint{-0.020833in}{0.005525in}}{\pgfqpoint{-0.020833in}{0.000000in}}%
\pgfpathcurveto{\pgfqpoint{-0.020833in}{-0.005525in}}{\pgfqpoint{-0.018638in}{-0.010825in}}{\pgfqpoint{-0.014731in}{-0.014731in}}%
\pgfpathcurveto{\pgfqpoint{-0.010825in}{-0.018638in}}{\pgfqpoint{-0.005525in}{-0.020833in}}{\pgfqpoint{0.000000in}{-0.020833in}}%
\pgfpathclose%
\pgfusepath{stroke,fill}%
}%
\begin{pgfscope}%
\pgfsys@transformshift{0.881250in}{1.732500in}%
\pgfsys@useobject{currentmarker}{}%
\end{pgfscope}%
\begin{pgfscope}%
\pgfsys@transformshift{1.365625in}{1.984033in}%
\pgfsys@useobject{currentmarker}{}%
\end{pgfscope}%
\begin{pgfscope}%
\pgfsys@transformshift{1.850000in}{2.271500in}%
\pgfsys@useobject{currentmarker}{}%
\end{pgfscope}%
\begin{pgfscope}%
\pgfsys@transformshift{2.334375in}{2.525147in}%
\pgfsys@useobject{currentmarker}{}%
\end{pgfscope}%
\begin{pgfscope}%
\pgfsys@transformshift{2.818750in}{2.630833in}%
\pgfsys@useobject{currentmarker}{}%
\end{pgfscope}%
\begin{pgfscope}%
\pgfsys@transformshift{3.303125in}{2.525147in}%
\pgfsys@useobject{currentmarker}{}%
\end{pgfscope}%
\begin{pgfscope}%
\pgfsys@transformshift{3.787500in}{2.271500in}%
\pgfsys@useobject{currentmarker}{}%
\end{pgfscope}%
\begin{pgfscope}%
\pgfsys@transformshift{4.271875in}{1.984033in}%
\pgfsys@useobject{currentmarker}{}%
\end{pgfscope}%
\end{pgfscope}%
\begin{pgfscope}%
\pgfpathrectangle{\pgfqpoint{0.687500in}{0.385000in}}{\pgfqpoint{4.262500in}{2.695000in}}%
\pgfusepath{clip}%
\pgfsetrectcap%
\pgfsetroundjoin%
\pgfsetlinewidth{1.505625pt}%
\definecolor{currentstroke}{rgb}{0.580392,0.403922,0.741176}%
\pgfsetstrokecolor{currentstroke}%
\pgfsetdash{}{0pt}%
\pgfpathmoveto{\pgfqpoint{0.881250in}{1.732500in}}%
\pgfpathlineto{\pgfqpoint{0.978612in}{1.778877in}}%
\pgfpathlineto{\pgfqpoint{1.075974in}{1.827520in}}%
\pgfpathlineto{\pgfqpoint{1.192808in}{1.888622in}}%
\pgfpathlineto{\pgfqpoint{1.309642in}{1.952505in}}%
\pgfpathlineto{\pgfqpoint{1.426476in}{2.018919in}}%
\pgfpathlineto{\pgfqpoint{1.562783in}{2.099020in}}%
\pgfpathlineto{\pgfqpoint{1.991175in}{2.353775in}}%
\pgfpathlineto{\pgfqpoint{2.088536in}{2.407573in}}%
\pgfpathlineto{\pgfqpoint{2.166426in}{2.448086in}}%
\pgfpathlineto{\pgfqpoint{2.244315in}{2.485778in}}%
\pgfpathlineto{\pgfqpoint{2.302732in}{2.511875in}}%
\pgfpathlineto{\pgfqpoint{2.361149in}{2.535870in}}%
\pgfpathlineto{\pgfqpoint{2.419567in}{2.557556in}}%
\pgfpathlineto{\pgfqpoint{2.477984in}{2.576739in}}%
\pgfpathlineto{\pgfqpoint{2.536401in}{2.593241in}}%
\pgfpathlineto{\pgfqpoint{2.594818in}{2.606903in}}%
\pgfpathlineto{\pgfqpoint{2.653235in}{2.617588in}}%
\pgfpathlineto{\pgfqpoint{2.711652in}{2.625186in}}%
\pgfpathlineto{\pgfqpoint{2.770069in}{2.629610in}}%
\pgfpathlineto{\pgfqpoint{2.828486in}{2.630807in}}%
\pgfpathlineto{\pgfqpoint{2.886903in}{2.628752in}}%
\pgfpathlineto{\pgfqpoint{2.945320in}{2.623454in}}%
\pgfpathlineto{\pgfqpoint{3.003737in}{2.614953in}}%
\pgfpathlineto{\pgfqpoint{3.062155in}{2.603324in}}%
\pgfpathlineto{\pgfqpoint{3.120572in}{2.588675in}}%
\pgfpathlineto{\pgfqpoint{3.178989in}{2.571146in}}%
\pgfpathlineto{\pgfqpoint{3.237406in}{2.550907in}}%
\pgfpathlineto{\pgfqpoint{3.295823in}{2.528157in}}%
\pgfpathlineto{\pgfqpoint{3.354240in}{2.503119in}}%
\pgfpathlineto{\pgfqpoint{3.412657in}{2.476038in}}%
\pgfpathlineto{\pgfqpoint{3.490546in}{2.437207in}}%
\pgfpathlineto{\pgfqpoint{3.568436in}{2.395856in}}%
\pgfpathlineto{\pgfqpoint{3.665798in}{2.341613in}}%
\pgfpathlineto{\pgfqpoint{3.821577in}{2.251652in}}%
\pgfpathlineto{\pgfqpoint{4.172079in}{2.046118in}}%
\pgfpathlineto{\pgfqpoint{4.249969in}{1.998059in}}%
\pgfpathlineto{\pgfqpoint{4.308386in}{1.959971in}}%
\pgfpathlineto{\pgfqpoint{4.366803in}{1.919172in}}%
\pgfpathlineto{\pgfqpoint{4.405747in}{1.889927in}}%
\pgfpathlineto{\pgfqpoint{4.444692in}{1.858599in}}%
\pgfpathlineto{\pgfqpoint{4.483637in}{1.824740in}}%
\pgfpathlineto{\pgfqpoint{4.522582in}{1.787849in}}%
\pgfpathlineto{\pgfqpoint{4.561526in}{1.747359in}}%
\pgfpathlineto{\pgfqpoint{4.600471in}{1.702636in}}%
\pgfpathlineto{\pgfqpoint{4.639416in}{1.652977in}}%
\pgfpathlineto{\pgfqpoint{4.678361in}{1.597602in}}%
\pgfpathlineto{\pgfqpoint{4.717305in}{1.535649in}}%
\pgfpathlineto{\pgfqpoint{4.756250in}{1.466171in}}%
\pgfpathlineto{\pgfqpoint{4.756250in}{1.466171in}}%
\pgfusepath{stroke}%
\end{pgfscope}%
\begin{pgfscope}%
\pgfpathrectangle{\pgfqpoint{0.687500in}{0.385000in}}{\pgfqpoint{4.262500in}{2.695000in}}%
\pgfusepath{clip}%
\pgfsetbuttcap%
\pgfsetroundjoin%
\definecolor{currentfill}{rgb}{0.549020,0.337255,0.294118}%
\pgfsetfillcolor{currentfill}%
\pgfsetlinewidth{1.003750pt}%
\definecolor{currentstroke}{rgb}{0.549020,0.337255,0.294118}%
\pgfsetstrokecolor{currentstroke}%
\pgfsetdash{}{0pt}%
\pgfsys@defobject{currentmarker}{\pgfqpoint{-0.020833in}{-0.020833in}}{\pgfqpoint{0.020833in}{0.020833in}}{%
\pgfpathmoveto{\pgfqpoint{0.000000in}{-0.020833in}}%
\pgfpathcurveto{\pgfqpoint{0.005525in}{-0.020833in}}{\pgfqpoint{0.010825in}{-0.018638in}}{\pgfqpoint{0.014731in}{-0.014731in}}%
\pgfpathcurveto{\pgfqpoint{0.018638in}{-0.010825in}}{\pgfqpoint{0.020833in}{-0.005525in}}{\pgfqpoint{0.020833in}{0.000000in}}%
\pgfpathcurveto{\pgfqpoint{0.020833in}{0.005525in}}{\pgfqpoint{0.018638in}{0.010825in}}{\pgfqpoint{0.014731in}{0.014731in}}%
\pgfpathcurveto{\pgfqpoint{0.010825in}{0.018638in}}{\pgfqpoint{0.005525in}{0.020833in}}{\pgfqpoint{0.000000in}{0.020833in}}%
\pgfpathcurveto{\pgfqpoint{-0.005525in}{0.020833in}}{\pgfqpoint{-0.010825in}{0.018638in}}{\pgfqpoint{-0.014731in}{0.014731in}}%
\pgfpathcurveto{\pgfqpoint{-0.018638in}{0.010825in}}{\pgfqpoint{-0.020833in}{0.005525in}}{\pgfqpoint{-0.020833in}{0.000000in}}%
\pgfpathcurveto{\pgfqpoint{-0.020833in}{-0.005525in}}{\pgfqpoint{-0.018638in}{-0.010825in}}{\pgfqpoint{-0.014731in}{-0.014731in}}%
\pgfpathcurveto{\pgfqpoint{-0.010825in}{-0.018638in}}{\pgfqpoint{-0.005525in}{-0.020833in}}{\pgfqpoint{0.000000in}{-0.020833in}}%
\pgfpathclose%
\pgfusepath{stroke,fill}%
}%
\begin{pgfscope}%
\pgfsys@transformshift{0.881250in}{1.732500in}%
\pgfsys@useobject{currentmarker}{}%
\end{pgfscope}%
\begin{pgfscope}%
\pgfsys@transformshift{1.123438in}{1.851748in}%
\pgfsys@useobject{currentmarker}{}%
\end{pgfscope}%
\begin{pgfscope}%
\pgfsys@transformshift{1.365625in}{1.984033in}%
\pgfsys@useobject{currentmarker}{}%
\end{pgfscope}%
\begin{pgfscope}%
\pgfsys@transformshift{1.607813in}{2.126152in}%
\pgfsys@useobject{currentmarker}{}%
\end{pgfscope}%
\begin{pgfscope}%
\pgfsys@transformshift{1.850000in}{2.271500in}%
\pgfsys@useobject{currentmarker}{}%
\end{pgfscope}%
\begin{pgfscope}%
\pgfsys@transformshift{2.092188in}{2.409326in}%
\pgfsys@useobject{currentmarker}{}%
\end{pgfscope}%
\begin{pgfscope}%
\pgfsys@transformshift{2.334375in}{2.525147in}%
\pgfsys@useobject{currentmarker}{}%
\end{pgfscope}%
\begin{pgfscope}%
\pgfsys@transformshift{2.576563in}{2.603192in}%
\pgfsys@useobject{currentmarker}{}%
\end{pgfscope}%
\begin{pgfscope}%
\pgfsys@transformshift{2.818750in}{2.630833in}%
\pgfsys@useobject{currentmarker}{}%
\end{pgfscope}%
\begin{pgfscope}%
\pgfsys@transformshift{3.060938in}{2.603192in}%
\pgfsys@useobject{currentmarker}{}%
\end{pgfscope}%
\begin{pgfscope}%
\pgfsys@transformshift{3.303125in}{2.525147in}%
\pgfsys@useobject{currentmarker}{}%
\end{pgfscope}%
\begin{pgfscope}%
\pgfsys@transformshift{3.545313in}{2.409326in}%
\pgfsys@useobject{currentmarker}{}%
\end{pgfscope}%
\begin{pgfscope}%
\pgfsys@transformshift{3.787500in}{2.271500in}%
\pgfsys@useobject{currentmarker}{}%
\end{pgfscope}%
\begin{pgfscope}%
\pgfsys@transformshift{4.029687in}{2.126152in}%
\pgfsys@useobject{currentmarker}{}%
\end{pgfscope}%
\begin{pgfscope}%
\pgfsys@transformshift{4.271875in}{1.984033in}%
\pgfsys@useobject{currentmarker}{}%
\end{pgfscope}%
\begin{pgfscope}%
\pgfsys@transformshift{4.514063in}{1.851748in}%
\pgfsys@useobject{currentmarker}{}%
\end{pgfscope}%
\end{pgfscope}%
\begin{pgfscope}%
\pgfpathrectangle{\pgfqpoint{0.687500in}{0.385000in}}{\pgfqpoint{4.262500in}{2.695000in}}%
\pgfusepath{clip}%
\pgfsetrectcap%
\pgfsetroundjoin%
\pgfsetlinewidth{1.505625pt}%
\definecolor{currentstroke}{rgb}{0.890196,0.466667,0.760784}%
\pgfsetstrokecolor{currentstroke}%
\pgfsetdash{}{0pt}%
\pgfpathmoveto{\pgfqpoint{0.881250in}{1.732500in}}%
\pgfpathlineto{\pgfqpoint{0.978612in}{1.778777in}}%
\pgfpathlineto{\pgfqpoint{1.075974in}{1.827297in}}%
\pgfpathlineto{\pgfqpoint{1.173335in}{1.878000in}}%
\pgfpathlineto{\pgfqpoint{1.290170in}{1.941556in}}%
\pgfpathlineto{\pgfqpoint{1.407004in}{2.007752in}}%
\pgfpathlineto{\pgfqpoint{1.543310in}{2.087643in}}%
\pgfpathlineto{\pgfqpoint{1.796451in}{2.239567in}}%
\pgfpathlineto{\pgfqpoint{1.932758in}{2.320107in}}%
\pgfpathlineto{\pgfqpoint{2.030119in}{2.375473in}}%
\pgfpathlineto{\pgfqpoint{2.108009in}{2.417737in}}%
\pgfpathlineto{\pgfqpoint{2.185898in}{2.457628in}}%
\pgfpathlineto{\pgfqpoint{2.263788in}{2.494605in}}%
\pgfpathlineto{\pgfqpoint{2.322205in}{2.520101in}}%
\pgfpathlineto{\pgfqpoint{2.380622in}{2.543430in}}%
\pgfpathlineto{\pgfqpoint{2.439039in}{2.564379in}}%
\pgfpathlineto{\pgfqpoint{2.497456in}{2.582749in}}%
\pgfpathlineto{\pgfqpoint{2.555873in}{2.598357in}}%
\pgfpathlineto{\pgfqpoint{2.614290in}{2.611046in}}%
\pgfpathlineto{\pgfqpoint{2.672707in}{2.620683in}}%
\pgfpathlineto{\pgfqpoint{2.731124in}{2.627166in}}%
\pgfpathlineto{\pgfqpoint{2.789541in}{2.630425in}}%
\pgfpathlineto{\pgfqpoint{2.847959in}{2.630425in}}%
\pgfpathlineto{\pgfqpoint{2.906376in}{2.627166in}}%
\pgfpathlineto{\pgfqpoint{2.964793in}{2.620683in}}%
\pgfpathlineto{\pgfqpoint{3.023210in}{2.611046in}}%
\pgfpathlineto{\pgfqpoint{3.081627in}{2.598357in}}%
\pgfpathlineto{\pgfqpoint{3.140044in}{2.582748in}}%
\pgfpathlineto{\pgfqpoint{3.198461in}{2.564379in}}%
\pgfpathlineto{\pgfqpoint{3.256878in}{2.543430in}}%
\pgfpathlineto{\pgfqpoint{3.315295in}{2.520101in}}%
\pgfpathlineto{\pgfqpoint{3.373712in}{2.494605in}}%
\pgfpathlineto{\pgfqpoint{3.451602in}{2.457628in}}%
\pgfpathlineto{\pgfqpoint{3.529491in}{2.417737in}}%
\pgfpathlineto{\pgfqpoint{3.607381in}{2.375473in}}%
\pgfpathlineto{\pgfqpoint{3.704742in}{2.320107in}}%
\pgfpathlineto{\pgfqpoint{3.841049in}{2.239567in}}%
\pgfpathlineto{\pgfqpoint{4.211024in}{2.019002in}}%
\pgfpathlineto{\pgfqpoint{4.327858in}{1.952436in}}%
\pgfpathlineto{\pgfqpoint{4.444692in}{1.888472in}}%
\pgfpathlineto{\pgfqpoint{4.580999in}{1.816930in}}%
\pgfpathlineto{\pgfqpoint{4.697833in}{1.755174in}}%
\pgfpathlineto{\pgfqpoint{4.756250in}{1.721187in}}%
\pgfpathlineto{\pgfqpoint{4.756250in}{1.721187in}}%
\pgfusepath{stroke}%
\end{pgfscope}%
\begin{pgfscope}%
\pgfpathrectangle{\pgfqpoint{0.687500in}{0.385000in}}{\pgfqpoint{4.262500in}{2.695000in}}%
\pgfusepath{clip}%
\pgfsetbuttcap%
\pgfsetroundjoin%
\definecolor{currentfill}{rgb}{0.498039,0.498039,0.498039}%
\pgfsetfillcolor{currentfill}%
\pgfsetlinewidth{1.003750pt}%
\definecolor{currentstroke}{rgb}{0.498039,0.498039,0.498039}%
\pgfsetstrokecolor{currentstroke}%
\pgfsetdash{}{0pt}%
\pgfsys@defobject{currentmarker}{\pgfqpoint{-0.020833in}{-0.020833in}}{\pgfqpoint{0.020833in}{0.020833in}}{%
\pgfpathmoveto{\pgfqpoint{0.000000in}{-0.020833in}}%
\pgfpathcurveto{\pgfqpoint{0.005525in}{-0.020833in}}{\pgfqpoint{0.010825in}{-0.018638in}}{\pgfqpoint{0.014731in}{-0.014731in}}%
\pgfpathcurveto{\pgfqpoint{0.018638in}{-0.010825in}}{\pgfqpoint{0.020833in}{-0.005525in}}{\pgfqpoint{0.020833in}{0.000000in}}%
\pgfpathcurveto{\pgfqpoint{0.020833in}{0.005525in}}{\pgfqpoint{0.018638in}{0.010825in}}{\pgfqpoint{0.014731in}{0.014731in}}%
\pgfpathcurveto{\pgfqpoint{0.010825in}{0.018638in}}{\pgfqpoint{0.005525in}{0.020833in}}{\pgfqpoint{0.000000in}{0.020833in}}%
\pgfpathcurveto{\pgfqpoint{-0.005525in}{0.020833in}}{\pgfqpoint{-0.010825in}{0.018638in}}{\pgfqpoint{-0.014731in}{0.014731in}}%
\pgfpathcurveto{\pgfqpoint{-0.018638in}{0.010825in}}{\pgfqpoint{-0.020833in}{0.005525in}}{\pgfqpoint{-0.020833in}{0.000000in}}%
\pgfpathcurveto{\pgfqpoint{-0.020833in}{-0.005525in}}{\pgfqpoint{-0.018638in}{-0.010825in}}{\pgfqpoint{-0.014731in}{-0.014731in}}%
\pgfpathcurveto{\pgfqpoint{-0.010825in}{-0.018638in}}{\pgfqpoint{-0.005525in}{-0.020833in}}{\pgfqpoint{0.000000in}{-0.020833in}}%
\pgfpathclose%
\pgfusepath{stroke,fill}%
}%
\begin{pgfscope}%
\pgfsys@transformshift{0.881250in}{1.732500in}%
\pgfsys@useobject{currentmarker}{}%
\end{pgfscope}%
\begin{pgfscope}%
\pgfsys@transformshift{1.002344in}{1.790397in}%
\pgfsys@useobject{currentmarker}{}%
\end{pgfscope}%
\begin{pgfscope}%
\pgfsys@transformshift{1.123438in}{1.851748in}%
\pgfsys@useobject{currentmarker}{}%
\end{pgfscope}%
\begin{pgfscope}%
\pgfsys@transformshift{1.244531in}{1.916394in}%
\pgfsys@useobject{currentmarker}{}%
\end{pgfscope}%
\begin{pgfscope}%
\pgfsys@transformshift{1.365625in}{1.984033in}%
\pgfsys@useobject{currentmarker}{}%
\end{pgfscope}%
\begin{pgfscope}%
\pgfsys@transformshift{1.486719in}{2.054184in}%
\pgfsys@useobject{currentmarker}{}%
\end{pgfscope}%
\begin{pgfscope}%
\pgfsys@transformshift{1.607813in}{2.126152in}%
\pgfsys@useobject{currentmarker}{}%
\end{pgfscope}%
\begin{pgfscope}%
\pgfsys@transformshift{1.728906in}{2.198994in}%
\pgfsys@useobject{currentmarker}{}%
\end{pgfscope}%
\begin{pgfscope}%
\pgfsys@transformshift{1.850000in}{2.271500in}%
\pgfsys@useobject{currentmarker}{}%
\end{pgfscope}%
\begin{pgfscope}%
\pgfsys@transformshift{1.971094in}{2.342189in}%
\pgfsys@useobject{currentmarker}{}%
\end{pgfscope}%
\begin{pgfscope}%
\pgfsys@transformshift{2.092188in}{2.409326in}%
\pgfsys@useobject{currentmarker}{}%
\end{pgfscope}%
\begin{pgfscope}%
\pgfsys@transformshift{2.213281in}{2.470988in}%
\pgfsys@useobject{currentmarker}{}%
\end{pgfscope}%
\begin{pgfscope}%
\pgfsys@transformshift{2.334375in}{2.525147in}%
\pgfsys@useobject{currentmarker}{}%
\end{pgfscope}%
\begin{pgfscope}%
\pgfsys@transformshift{2.455469in}{2.569814in}%
\pgfsys@useobject{currentmarker}{}%
\end{pgfscope}%
\begin{pgfscope}%
\pgfsys@transformshift{2.576563in}{2.603192in}%
\pgfsys@useobject{currentmarker}{}%
\end{pgfscope}%
\begin{pgfscope}%
\pgfsys@transformshift{2.697656in}{2.623842in}%
\pgfsys@useobject{currentmarker}{}%
\end{pgfscope}%
\begin{pgfscope}%
\pgfsys@transformshift{2.818750in}{2.630833in}%
\pgfsys@useobject{currentmarker}{}%
\end{pgfscope}%
\begin{pgfscope}%
\pgfsys@transformshift{2.939844in}{2.623842in}%
\pgfsys@useobject{currentmarker}{}%
\end{pgfscope}%
\begin{pgfscope}%
\pgfsys@transformshift{3.060938in}{2.603192in}%
\pgfsys@useobject{currentmarker}{}%
\end{pgfscope}%
\begin{pgfscope}%
\pgfsys@transformshift{3.182031in}{2.569814in}%
\pgfsys@useobject{currentmarker}{}%
\end{pgfscope}%
\begin{pgfscope}%
\pgfsys@transformshift{3.303125in}{2.525147in}%
\pgfsys@useobject{currentmarker}{}%
\end{pgfscope}%
\begin{pgfscope}%
\pgfsys@transformshift{3.424219in}{2.470988in}%
\pgfsys@useobject{currentmarker}{}%
\end{pgfscope}%
\begin{pgfscope}%
\pgfsys@transformshift{3.545313in}{2.409326in}%
\pgfsys@useobject{currentmarker}{}%
\end{pgfscope}%
\begin{pgfscope}%
\pgfsys@transformshift{3.666406in}{2.342189in}%
\pgfsys@useobject{currentmarker}{}%
\end{pgfscope}%
\begin{pgfscope}%
\pgfsys@transformshift{3.787500in}{2.271500in}%
\pgfsys@useobject{currentmarker}{}%
\end{pgfscope}%
\begin{pgfscope}%
\pgfsys@transformshift{3.908594in}{2.198994in}%
\pgfsys@useobject{currentmarker}{}%
\end{pgfscope}%
\begin{pgfscope}%
\pgfsys@transformshift{4.029687in}{2.126152in}%
\pgfsys@useobject{currentmarker}{}%
\end{pgfscope}%
\begin{pgfscope}%
\pgfsys@transformshift{4.150781in}{2.054184in}%
\pgfsys@useobject{currentmarker}{}%
\end{pgfscope}%
\begin{pgfscope}%
\pgfsys@transformshift{4.271875in}{1.984033in}%
\pgfsys@useobject{currentmarker}{}%
\end{pgfscope}%
\begin{pgfscope}%
\pgfsys@transformshift{4.392969in}{1.916394in}%
\pgfsys@useobject{currentmarker}{}%
\end{pgfscope}%
\begin{pgfscope}%
\pgfsys@transformshift{4.514063in}{1.851748in}%
\pgfsys@useobject{currentmarker}{}%
\end{pgfscope}%
\begin{pgfscope}%
\pgfsys@transformshift{4.635156in}{1.790397in}%
\pgfsys@useobject{currentmarker}{}%
\end{pgfscope}%
\end{pgfscope}%
\begin{pgfscope}%
\pgfpathrectangle{\pgfqpoint{0.687500in}{0.385000in}}{\pgfqpoint{4.262500in}{2.695000in}}%
\pgfusepath{clip}%
\pgfsetrectcap%
\pgfsetroundjoin%
\pgfsetlinewidth{1.505625pt}%
\definecolor{currentstroke}{rgb}{0.737255,0.741176,0.133333}%
\pgfsetstrokecolor{currentstroke}%
\pgfsetdash{}{0pt}%
\pgfpathmoveto{\pgfqpoint{0.881250in}{1.732500in}}%
\pgfpathlineto{\pgfqpoint{0.978612in}{1.778775in}}%
\pgfpathlineto{\pgfqpoint{1.075974in}{1.827296in}}%
\pgfpathlineto{\pgfqpoint{1.173335in}{1.878000in}}%
\pgfpathlineto{\pgfqpoint{1.290170in}{1.941556in}}%
\pgfpathlineto{\pgfqpoint{1.407004in}{2.007752in}}%
\pgfpathlineto{\pgfqpoint{1.543310in}{2.087643in}}%
\pgfpathlineto{\pgfqpoint{1.796451in}{2.239567in}}%
\pgfpathlineto{\pgfqpoint{1.932758in}{2.320107in}}%
\pgfpathlineto{\pgfqpoint{2.030119in}{2.375473in}}%
\pgfpathlineto{\pgfqpoint{2.108009in}{2.417737in}}%
\pgfpathlineto{\pgfqpoint{2.185898in}{2.457628in}}%
\pgfpathlineto{\pgfqpoint{2.263788in}{2.494605in}}%
\pgfpathlineto{\pgfqpoint{2.322205in}{2.520101in}}%
\pgfpathlineto{\pgfqpoint{2.380622in}{2.543430in}}%
\pgfpathlineto{\pgfqpoint{2.439039in}{2.564379in}}%
\pgfpathlineto{\pgfqpoint{2.497456in}{2.582749in}}%
\pgfpathlineto{\pgfqpoint{2.555873in}{2.598357in}}%
\pgfpathlineto{\pgfqpoint{2.614290in}{2.611046in}}%
\pgfpathlineto{\pgfqpoint{2.672707in}{2.620683in}}%
\pgfpathlineto{\pgfqpoint{2.731124in}{2.627166in}}%
\pgfpathlineto{\pgfqpoint{2.789541in}{2.630425in}}%
\pgfpathlineto{\pgfqpoint{2.847959in}{2.630425in}}%
\pgfpathlineto{\pgfqpoint{2.906376in}{2.627166in}}%
\pgfpathlineto{\pgfqpoint{2.964793in}{2.620683in}}%
\pgfpathlineto{\pgfqpoint{3.023210in}{2.611046in}}%
\pgfpathlineto{\pgfqpoint{3.081627in}{2.598357in}}%
\pgfpathlineto{\pgfqpoint{3.140044in}{2.582749in}}%
\pgfpathlineto{\pgfqpoint{3.198461in}{2.564379in}}%
\pgfpathlineto{\pgfqpoint{3.256878in}{2.543430in}}%
\pgfpathlineto{\pgfqpoint{3.315295in}{2.520101in}}%
\pgfpathlineto{\pgfqpoint{3.373712in}{2.494605in}}%
\pgfpathlineto{\pgfqpoint{3.451602in}{2.457628in}}%
\pgfpathlineto{\pgfqpoint{3.529491in}{2.417737in}}%
\pgfpathlineto{\pgfqpoint{3.607381in}{2.375473in}}%
\pgfpathlineto{\pgfqpoint{3.704742in}{2.320107in}}%
\pgfpathlineto{\pgfqpoint{3.841049in}{2.239567in}}%
\pgfpathlineto{\pgfqpoint{4.211024in}{2.019009in}}%
\pgfpathlineto{\pgfqpoint{4.327858in}{1.952417in}}%
\pgfpathlineto{\pgfqpoint{4.444692in}{1.888394in}}%
\pgfpathlineto{\pgfqpoint{4.542054in}{1.837265in}}%
\pgfpathlineto{\pgfqpoint{4.639416in}{1.788301in}}%
\pgfpathlineto{\pgfqpoint{4.736778in}{1.741567in}}%
\pgfpathlineto{\pgfqpoint{4.756250in}{1.732485in}}%
\pgfpathlineto{\pgfqpoint{4.756250in}{1.732485in}}%
\pgfusepath{stroke}%
\end{pgfscope}%
\begin{pgfscope}%
\pgfsetrectcap%
\pgfsetmiterjoin%
\pgfsetlinewidth{0.803000pt}%
\definecolor{currentstroke}{rgb}{0.000000,0.000000,0.000000}%
\pgfsetstrokecolor{currentstroke}%
\pgfsetdash{}{0pt}%
\pgfpathmoveto{\pgfqpoint{0.687500in}{0.385000in}}%
\pgfpathlineto{\pgfqpoint{0.687500in}{3.080000in}}%
\pgfusepath{stroke}%
\end{pgfscope}%
\begin{pgfscope}%
\pgfsetrectcap%
\pgfsetmiterjoin%
\pgfsetlinewidth{0.803000pt}%
\definecolor{currentstroke}{rgb}{0.000000,0.000000,0.000000}%
\pgfsetstrokecolor{currentstroke}%
\pgfsetdash{}{0pt}%
\pgfpathmoveto{\pgfqpoint{4.950000in}{0.385000in}}%
\pgfpathlineto{\pgfqpoint{4.950000in}{3.080000in}}%
\pgfusepath{stroke}%
\end{pgfscope}%
\begin{pgfscope}%
\pgfsetrectcap%
\pgfsetmiterjoin%
\pgfsetlinewidth{0.803000pt}%
\definecolor{currentstroke}{rgb}{0.000000,0.000000,0.000000}%
\pgfsetstrokecolor{currentstroke}%
\pgfsetdash{}{0pt}%
\pgfpathmoveto{\pgfqpoint{0.687500in}{0.385000in}}%
\pgfpathlineto{\pgfqpoint{4.950000in}{0.385000in}}%
\pgfusepath{stroke}%
\end{pgfscope}%
\begin{pgfscope}%
\pgfsetrectcap%
\pgfsetmiterjoin%
\pgfsetlinewidth{0.803000pt}%
\definecolor{currentstroke}{rgb}{0.000000,0.000000,0.000000}%
\pgfsetstrokecolor{currentstroke}%
\pgfsetdash{}{0pt}%
\pgfpathmoveto{\pgfqpoint{0.687500in}{3.080000in}}%
\pgfpathlineto{\pgfqpoint{4.950000in}{3.080000in}}%
\pgfusepath{stroke}%
\end{pgfscope}%
\begin{pgfscope}%
\definecolor{textcolor}{rgb}{0.000000,0.000000,0.000000}%
\pgfsetstrokecolor{textcolor}%
\pgfsetfillcolor{textcolor}%
\pgftext[x=2.818750in,y=3.163333in,,base]{\color{textcolor}\rmfamily\fontsize{12.000000}{14.400000}\selectfont N=5, 7, 15, 31}%
\end{pgfscope}%
\begin{pgfscope}%
\pgfsetbuttcap%
\pgfsetmiterjoin%
\definecolor{currentfill}{rgb}{1.000000,1.000000,1.000000}%
\pgfsetfillcolor{currentfill}%
\pgfsetfillopacity{0.800000}%
\pgfsetlinewidth{1.003750pt}%
\definecolor{currentstroke}{rgb}{0.800000,0.800000,0.800000}%
\pgfsetstrokecolor{currentstroke}%
\pgfsetstrokeopacity{0.800000}%
\pgfsetdash{}{0pt}%
\pgfpathmoveto{\pgfqpoint{0.784722in}{0.454444in}}%
\pgfpathlineto{\pgfqpoint{2.050468in}{0.454444in}}%
\pgfpathquadraticcurveto{\pgfqpoint{2.078246in}{0.454444in}}{\pgfqpoint{2.078246in}{0.482222in}}%
\pgfpathlineto{\pgfqpoint{2.078246in}{1.661422in}}%
\pgfpathquadraticcurveto{\pgfqpoint{2.078246in}{1.689199in}}{\pgfqpoint{2.050468in}{1.689199in}}%
\pgfpathlineto{\pgfqpoint{0.784722in}{1.689199in}}%
\pgfpathquadraticcurveto{\pgfqpoint{0.756944in}{1.689199in}}{\pgfqpoint{0.756944in}{1.661422in}}%
\pgfpathlineto{\pgfqpoint{0.756944in}{0.482222in}}%
\pgfpathquadraticcurveto{\pgfqpoint{0.756944in}{0.454444in}}{\pgfqpoint{0.784722in}{0.454444in}}%
\pgfpathclose%
\pgfusepath{stroke,fill}%
\end{pgfscope}%
\begin{pgfscope}%
\pgfsetrectcap%
\pgfsetroundjoin%
\pgfsetlinewidth{1.505625pt}%
\definecolor{currentstroke}{rgb}{0.121569,0.466667,0.705882}%
\pgfsetstrokecolor{currentstroke}%
\pgfsetdash{}{0pt}%
\pgfpathmoveto{\pgfqpoint{0.812500in}{1.498789in}}%
\pgfpathlineto{\pgfqpoint{1.090278in}{1.498789in}}%
\pgfusepath{stroke}%
\end{pgfscope}%
\begin{pgfscope}%
\definecolor{textcolor}{rgb}{0.000000,0.000000,0.000000}%
\pgfsetstrokecolor{textcolor}%
\pgfsetfillcolor{textcolor}%
\pgftext[x=1.201389in,y=1.450178in,left,base]{\color{textcolor}\rmfamily\fontsize{10.000000}{12.000000}\selectfont \(\displaystyle y(x)=\)\(\displaystyle \frac{1}{1+x^{2}}\)}%
\end{pgfscope}%
\begin{pgfscope}%
\pgfsetrectcap%
\pgfsetroundjoin%
\pgfsetlinewidth{1.505625pt}%
\definecolor{currentstroke}{rgb}{0.172549,0.627451,0.172549}%
\pgfsetstrokecolor{currentstroke}%
\pgfsetdash{}{0pt}%
\pgfpathmoveto{\pgfqpoint{0.812500in}{1.218333in}}%
\pgfpathlineto{\pgfqpoint{1.090278in}{1.218333in}}%
\pgfusepath{stroke}%
\end{pgfscope}%
\begin{pgfscope}%
\definecolor{textcolor}{rgb}{0.000000,0.000000,0.000000}%
\pgfsetstrokecolor{textcolor}%
\pgfsetfillcolor{textcolor}%
\pgftext[x=1.201389in,y=1.169722in,left,base]{\color{textcolor}\rmfamily\fontsize{10.000000}{12.000000}\selectfont W5(x)}%
\end{pgfscope}%
\begin{pgfscope}%
\pgfsetrectcap%
\pgfsetroundjoin%
\pgfsetlinewidth{1.505625pt}%
\definecolor{currentstroke}{rgb}{0.580392,0.403922,0.741176}%
\pgfsetstrokecolor{currentstroke}%
\pgfsetdash{}{0pt}%
\pgfpathmoveto{\pgfqpoint{0.812500in}{1.010000in}}%
\pgfpathlineto{\pgfqpoint{1.090278in}{1.010000in}}%
\pgfusepath{stroke}%
\end{pgfscope}%
\begin{pgfscope}%
\definecolor{textcolor}{rgb}{0.000000,0.000000,0.000000}%
\pgfsetstrokecolor{textcolor}%
\pgfsetfillcolor{textcolor}%
\pgftext[x=1.201389in,y=0.961389in,left,base]{\color{textcolor}\rmfamily\fontsize{10.000000}{12.000000}\selectfont W7(x)}%
\end{pgfscope}%
\begin{pgfscope}%
\pgfsetrectcap%
\pgfsetroundjoin%
\pgfsetlinewidth{1.505625pt}%
\definecolor{currentstroke}{rgb}{0.890196,0.466667,0.760784}%
\pgfsetstrokecolor{currentstroke}%
\pgfsetdash{}{0pt}%
\pgfpathmoveto{\pgfqpoint{0.812500in}{0.801667in}}%
\pgfpathlineto{\pgfqpoint{1.090278in}{0.801667in}}%
\pgfusepath{stroke}%
\end{pgfscope}%
\begin{pgfscope}%
\definecolor{textcolor}{rgb}{0.000000,0.000000,0.000000}%
\pgfsetstrokecolor{textcolor}%
\pgfsetfillcolor{textcolor}%
\pgftext[x=1.201389in,y=0.753056in,left,base]{\color{textcolor}\rmfamily\fontsize{10.000000}{12.000000}\selectfont W15(x)}%
\end{pgfscope}%
\begin{pgfscope}%
\pgfsetrectcap%
\pgfsetroundjoin%
\pgfsetlinewidth{1.505625pt}%
\definecolor{currentstroke}{rgb}{0.737255,0.741176,0.133333}%
\pgfsetstrokecolor{currentstroke}%
\pgfsetdash{}{0pt}%
\pgfpathmoveto{\pgfqpoint{0.812500in}{0.593333in}}%
\pgfpathlineto{\pgfqpoint{1.090278in}{0.593333in}}%
\pgfusepath{stroke}%
\end{pgfscope}%
\begin{pgfscope}%
\definecolor{textcolor}{rgb}{0.000000,0.000000,0.000000}%
\pgfsetstrokecolor{textcolor}%
\pgfsetfillcolor{textcolor}%
\pgftext[x=1.201389in,y=0.544722in,left,base]{\color{textcolor}\rmfamily\fontsize{10.000000}{12.000000}\selectfont W31(x)}%
\end{pgfscope}%
\end{pgfpicture}%
\makeatother%
\endgroup%
        
    \end{center}
    \caption{Węzły jednorodne, funkcja \(\tilde{y}\), \(N=5,7,15,31\)}
\end{figure}



% ------------------------------------------------------------------------------


\begin{figure}[h]
    \begin{center}
        %% Creator: Matplotlib, PGF backend
%%
%% To include the figure in your LaTeX document, write
%%   \input{<filename>.pgf}
%%
%% Make sure the required packages are loaded in your preamble
%%   \usepackage{pgf}
%%
%% Figures using additional raster images can only be included by \input if
%% they are in the same directory as the main LaTeX file. For loading figures
%% from other directories you can use the `import` package
%%   \usepackage{import}
%% and then include the figures with
%%   \import{<path to file>}{<filename>.pgf}
%%
%% Matplotlib used the following preamble
%%
\begingroup%
\makeatletter%
\begin{pgfpicture}%
\pgfpathrectangle{\pgfpointorigin}{\pgfqpoint{6.400000in}{4.800000in}}%
\pgfusepath{use as bounding box, clip}%
\begin{pgfscope}%
\pgfsetbuttcap%
\pgfsetmiterjoin%
\definecolor{currentfill}{rgb}{1.000000,1.000000,1.000000}%
\pgfsetfillcolor{currentfill}%
\pgfsetlinewidth{0.000000pt}%
\definecolor{currentstroke}{rgb}{1.000000,1.000000,1.000000}%
\pgfsetstrokecolor{currentstroke}%
\pgfsetdash{}{0pt}%
\pgfpathmoveto{\pgfqpoint{0.000000in}{0.000000in}}%
\pgfpathlineto{\pgfqpoint{6.400000in}{0.000000in}}%
\pgfpathlineto{\pgfqpoint{6.400000in}{4.800000in}}%
\pgfpathlineto{\pgfqpoint{0.000000in}{4.800000in}}%
\pgfpathclose%
\pgfusepath{fill}%
\end{pgfscope}%
\begin{pgfscope}%
\pgfsetbuttcap%
\pgfsetmiterjoin%
\definecolor{currentfill}{rgb}{1.000000,1.000000,1.000000}%
\pgfsetfillcolor{currentfill}%
\pgfsetlinewidth{0.000000pt}%
\definecolor{currentstroke}{rgb}{0.000000,0.000000,0.000000}%
\pgfsetstrokecolor{currentstroke}%
\pgfsetstrokeopacity{0.000000}%
\pgfsetdash{}{0pt}%
\pgfpathmoveto{\pgfqpoint{0.800000in}{0.528000in}}%
\pgfpathlineto{\pgfqpoint{5.760000in}{0.528000in}}%
\pgfpathlineto{\pgfqpoint{5.760000in}{4.224000in}}%
\pgfpathlineto{\pgfqpoint{0.800000in}{4.224000in}}%
\pgfpathclose%
\pgfusepath{fill}%
\end{pgfscope}%
\begin{pgfscope}%
\pgfsetbuttcap%
\pgfsetroundjoin%
\definecolor{currentfill}{rgb}{0.000000,0.000000,0.000000}%
\pgfsetfillcolor{currentfill}%
\pgfsetlinewidth{0.803000pt}%
\definecolor{currentstroke}{rgb}{0.000000,0.000000,0.000000}%
\pgfsetstrokecolor{currentstroke}%
\pgfsetdash{}{0pt}%
\pgfsys@defobject{currentmarker}{\pgfqpoint{0.000000in}{-0.048611in}}{\pgfqpoint{0.000000in}{0.000000in}}{%
\pgfpathmoveto{\pgfqpoint{0.000000in}{0.000000in}}%
\pgfpathlineto{\pgfqpoint{0.000000in}{-0.048611in}}%
\pgfusepath{stroke,fill}%
}%
\begin{pgfscope}%
\pgfsys@transformshift{1.025455in}{0.528000in}%
\pgfsys@useobject{currentmarker}{}%
\end{pgfscope}%
\end{pgfscope}%
\begin{pgfscope}%
\definecolor{textcolor}{rgb}{0.000000,0.000000,0.000000}%
\pgfsetstrokecolor{textcolor}%
\pgfsetfillcolor{textcolor}%
\pgftext[x=1.025455in,y=0.430778in,,top]{\color{textcolor}\rmfamily\fontsize{10.000000}{12.000000}\selectfont \(\displaystyle -1.00\)}%
\end{pgfscope}%
\begin{pgfscope}%
\pgfsetbuttcap%
\pgfsetroundjoin%
\definecolor{currentfill}{rgb}{0.000000,0.000000,0.000000}%
\pgfsetfillcolor{currentfill}%
\pgfsetlinewidth{0.803000pt}%
\definecolor{currentstroke}{rgb}{0.000000,0.000000,0.000000}%
\pgfsetstrokecolor{currentstroke}%
\pgfsetdash{}{0pt}%
\pgfsys@defobject{currentmarker}{\pgfqpoint{0.000000in}{-0.048611in}}{\pgfqpoint{0.000000in}{0.000000in}}{%
\pgfpathmoveto{\pgfqpoint{0.000000in}{0.000000in}}%
\pgfpathlineto{\pgfqpoint{0.000000in}{-0.048611in}}%
\pgfusepath{stroke,fill}%
}%
\begin{pgfscope}%
\pgfsys@transformshift{1.589091in}{0.528000in}%
\pgfsys@useobject{currentmarker}{}%
\end{pgfscope}%
\end{pgfscope}%
\begin{pgfscope}%
\definecolor{textcolor}{rgb}{0.000000,0.000000,0.000000}%
\pgfsetstrokecolor{textcolor}%
\pgfsetfillcolor{textcolor}%
\pgftext[x=1.589091in,y=0.430778in,,top]{\color{textcolor}\rmfamily\fontsize{10.000000}{12.000000}\selectfont \(\displaystyle -0.75\)}%
\end{pgfscope}%
\begin{pgfscope}%
\pgfsetbuttcap%
\pgfsetroundjoin%
\definecolor{currentfill}{rgb}{0.000000,0.000000,0.000000}%
\pgfsetfillcolor{currentfill}%
\pgfsetlinewidth{0.803000pt}%
\definecolor{currentstroke}{rgb}{0.000000,0.000000,0.000000}%
\pgfsetstrokecolor{currentstroke}%
\pgfsetdash{}{0pt}%
\pgfsys@defobject{currentmarker}{\pgfqpoint{0.000000in}{-0.048611in}}{\pgfqpoint{0.000000in}{0.000000in}}{%
\pgfpathmoveto{\pgfqpoint{0.000000in}{0.000000in}}%
\pgfpathlineto{\pgfqpoint{0.000000in}{-0.048611in}}%
\pgfusepath{stroke,fill}%
}%
\begin{pgfscope}%
\pgfsys@transformshift{2.152727in}{0.528000in}%
\pgfsys@useobject{currentmarker}{}%
\end{pgfscope}%
\end{pgfscope}%
\begin{pgfscope}%
\definecolor{textcolor}{rgb}{0.000000,0.000000,0.000000}%
\pgfsetstrokecolor{textcolor}%
\pgfsetfillcolor{textcolor}%
\pgftext[x=2.152727in,y=0.430778in,,top]{\color{textcolor}\rmfamily\fontsize{10.000000}{12.000000}\selectfont \(\displaystyle -0.50\)}%
\end{pgfscope}%
\begin{pgfscope}%
\pgfsetbuttcap%
\pgfsetroundjoin%
\definecolor{currentfill}{rgb}{0.000000,0.000000,0.000000}%
\pgfsetfillcolor{currentfill}%
\pgfsetlinewidth{0.803000pt}%
\definecolor{currentstroke}{rgb}{0.000000,0.000000,0.000000}%
\pgfsetstrokecolor{currentstroke}%
\pgfsetdash{}{0pt}%
\pgfsys@defobject{currentmarker}{\pgfqpoint{0.000000in}{-0.048611in}}{\pgfqpoint{0.000000in}{0.000000in}}{%
\pgfpathmoveto{\pgfqpoint{0.000000in}{0.000000in}}%
\pgfpathlineto{\pgfqpoint{0.000000in}{-0.048611in}}%
\pgfusepath{stroke,fill}%
}%
\begin{pgfscope}%
\pgfsys@transformshift{2.716364in}{0.528000in}%
\pgfsys@useobject{currentmarker}{}%
\end{pgfscope}%
\end{pgfscope}%
\begin{pgfscope}%
\definecolor{textcolor}{rgb}{0.000000,0.000000,0.000000}%
\pgfsetstrokecolor{textcolor}%
\pgfsetfillcolor{textcolor}%
\pgftext[x=2.716364in,y=0.430778in,,top]{\color{textcolor}\rmfamily\fontsize{10.000000}{12.000000}\selectfont \(\displaystyle -0.25\)}%
\end{pgfscope}%
\begin{pgfscope}%
\pgfsetbuttcap%
\pgfsetroundjoin%
\definecolor{currentfill}{rgb}{0.000000,0.000000,0.000000}%
\pgfsetfillcolor{currentfill}%
\pgfsetlinewidth{0.803000pt}%
\definecolor{currentstroke}{rgb}{0.000000,0.000000,0.000000}%
\pgfsetstrokecolor{currentstroke}%
\pgfsetdash{}{0pt}%
\pgfsys@defobject{currentmarker}{\pgfqpoint{0.000000in}{-0.048611in}}{\pgfqpoint{0.000000in}{0.000000in}}{%
\pgfpathmoveto{\pgfqpoint{0.000000in}{0.000000in}}%
\pgfpathlineto{\pgfqpoint{0.000000in}{-0.048611in}}%
\pgfusepath{stroke,fill}%
}%
\begin{pgfscope}%
\pgfsys@transformshift{3.280000in}{0.528000in}%
\pgfsys@useobject{currentmarker}{}%
\end{pgfscope}%
\end{pgfscope}%
\begin{pgfscope}%
\definecolor{textcolor}{rgb}{0.000000,0.000000,0.000000}%
\pgfsetstrokecolor{textcolor}%
\pgfsetfillcolor{textcolor}%
\pgftext[x=3.280000in,y=0.430778in,,top]{\color{textcolor}\rmfamily\fontsize{10.000000}{12.000000}\selectfont \(\displaystyle 0.00\)}%
\end{pgfscope}%
\begin{pgfscope}%
\pgfsetbuttcap%
\pgfsetroundjoin%
\definecolor{currentfill}{rgb}{0.000000,0.000000,0.000000}%
\pgfsetfillcolor{currentfill}%
\pgfsetlinewidth{0.803000pt}%
\definecolor{currentstroke}{rgb}{0.000000,0.000000,0.000000}%
\pgfsetstrokecolor{currentstroke}%
\pgfsetdash{}{0pt}%
\pgfsys@defobject{currentmarker}{\pgfqpoint{0.000000in}{-0.048611in}}{\pgfqpoint{0.000000in}{0.000000in}}{%
\pgfpathmoveto{\pgfqpoint{0.000000in}{0.000000in}}%
\pgfpathlineto{\pgfqpoint{0.000000in}{-0.048611in}}%
\pgfusepath{stroke,fill}%
}%
\begin{pgfscope}%
\pgfsys@transformshift{3.843636in}{0.528000in}%
\pgfsys@useobject{currentmarker}{}%
\end{pgfscope}%
\end{pgfscope}%
\begin{pgfscope}%
\definecolor{textcolor}{rgb}{0.000000,0.000000,0.000000}%
\pgfsetstrokecolor{textcolor}%
\pgfsetfillcolor{textcolor}%
\pgftext[x=3.843636in,y=0.430778in,,top]{\color{textcolor}\rmfamily\fontsize{10.000000}{12.000000}\selectfont \(\displaystyle 0.25\)}%
\end{pgfscope}%
\begin{pgfscope}%
\pgfsetbuttcap%
\pgfsetroundjoin%
\definecolor{currentfill}{rgb}{0.000000,0.000000,0.000000}%
\pgfsetfillcolor{currentfill}%
\pgfsetlinewidth{0.803000pt}%
\definecolor{currentstroke}{rgb}{0.000000,0.000000,0.000000}%
\pgfsetstrokecolor{currentstroke}%
\pgfsetdash{}{0pt}%
\pgfsys@defobject{currentmarker}{\pgfqpoint{0.000000in}{-0.048611in}}{\pgfqpoint{0.000000in}{0.000000in}}{%
\pgfpathmoveto{\pgfqpoint{0.000000in}{0.000000in}}%
\pgfpathlineto{\pgfqpoint{0.000000in}{-0.048611in}}%
\pgfusepath{stroke,fill}%
}%
\begin{pgfscope}%
\pgfsys@transformshift{4.407273in}{0.528000in}%
\pgfsys@useobject{currentmarker}{}%
\end{pgfscope}%
\end{pgfscope}%
\begin{pgfscope}%
\definecolor{textcolor}{rgb}{0.000000,0.000000,0.000000}%
\pgfsetstrokecolor{textcolor}%
\pgfsetfillcolor{textcolor}%
\pgftext[x=4.407273in,y=0.430778in,,top]{\color{textcolor}\rmfamily\fontsize{10.000000}{12.000000}\selectfont \(\displaystyle 0.50\)}%
\end{pgfscope}%
\begin{pgfscope}%
\pgfsetbuttcap%
\pgfsetroundjoin%
\definecolor{currentfill}{rgb}{0.000000,0.000000,0.000000}%
\pgfsetfillcolor{currentfill}%
\pgfsetlinewidth{0.803000pt}%
\definecolor{currentstroke}{rgb}{0.000000,0.000000,0.000000}%
\pgfsetstrokecolor{currentstroke}%
\pgfsetdash{}{0pt}%
\pgfsys@defobject{currentmarker}{\pgfqpoint{0.000000in}{-0.048611in}}{\pgfqpoint{0.000000in}{0.000000in}}{%
\pgfpathmoveto{\pgfqpoint{0.000000in}{0.000000in}}%
\pgfpathlineto{\pgfqpoint{0.000000in}{-0.048611in}}%
\pgfusepath{stroke,fill}%
}%
\begin{pgfscope}%
\pgfsys@transformshift{4.970909in}{0.528000in}%
\pgfsys@useobject{currentmarker}{}%
\end{pgfscope}%
\end{pgfscope}%
\begin{pgfscope}%
\definecolor{textcolor}{rgb}{0.000000,0.000000,0.000000}%
\pgfsetstrokecolor{textcolor}%
\pgfsetfillcolor{textcolor}%
\pgftext[x=4.970909in,y=0.430778in,,top]{\color{textcolor}\rmfamily\fontsize{10.000000}{12.000000}\selectfont \(\displaystyle 0.75\)}%
\end{pgfscope}%
\begin{pgfscope}%
\pgfsetbuttcap%
\pgfsetroundjoin%
\definecolor{currentfill}{rgb}{0.000000,0.000000,0.000000}%
\pgfsetfillcolor{currentfill}%
\pgfsetlinewidth{0.803000pt}%
\definecolor{currentstroke}{rgb}{0.000000,0.000000,0.000000}%
\pgfsetstrokecolor{currentstroke}%
\pgfsetdash{}{0pt}%
\pgfsys@defobject{currentmarker}{\pgfqpoint{0.000000in}{-0.048611in}}{\pgfqpoint{0.000000in}{0.000000in}}{%
\pgfpathmoveto{\pgfqpoint{0.000000in}{0.000000in}}%
\pgfpathlineto{\pgfqpoint{0.000000in}{-0.048611in}}%
\pgfusepath{stroke,fill}%
}%
\begin{pgfscope}%
\pgfsys@transformshift{5.534545in}{0.528000in}%
\pgfsys@useobject{currentmarker}{}%
\end{pgfscope}%
\end{pgfscope}%
\begin{pgfscope}%
\definecolor{textcolor}{rgb}{0.000000,0.000000,0.000000}%
\pgfsetstrokecolor{textcolor}%
\pgfsetfillcolor{textcolor}%
\pgftext[x=5.534545in,y=0.430778in,,top]{\color{textcolor}\rmfamily\fontsize{10.000000}{12.000000}\selectfont \(\displaystyle 1.00\)}%
\end{pgfscope}%
\begin{pgfscope}%
\definecolor{textcolor}{rgb}{0.000000,0.000000,0.000000}%
\pgfsetstrokecolor{textcolor}%
\pgfsetfillcolor{textcolor}%
\pgftext[x=3.280000in,y=0.251766in,,top]{\color{textcolor}\rmfamily\fontsize{10.000000}{12.000000}\selectfont x}%
\end{pgfscope}%
\begin{pgfscope}%
\pgfsetbuttcap%
\pgfsetroundjoin%
\definecolor{currentfill}{rgb}{0.000000,0.000000,0.000000}%
\pgfsetfillcolor{currentfill}%
\pgfsetlinewidth{0.803000pt}%
\definecolor{currentstroke}{rgb}{0.000000,0.000000,0.000000}%
\pgfsetstrokecolor{currentstroke}%
\pgfsetdash{}{0pt}%
\pgfsys@defobject{currentmarker}{\pgfqpoint{-0.048611in}{0.000000in}}{\pgfqpoint{0.000000in}{0.000000in}}{%
\pgfpathmoveto{\pgfqpoint{0.000000in}{0.000000in}}%
\pgfpathlineto{\pgfqpoint{-0.048611in}{0.000000in}}%
\pgfusepath{stroke,fill}%
}%
\begin{pgfscope}%
\pgfsys@transformshift{0.800000in}{0.528000in}%
\pgfsys@useobject{currentmarker}{}%
\end{pgfscope}%
\end{pgfscope}%
\begin{pgfscope}%
\definecolor{textcolor}{rgb}{0.000000,0.000000,0.000000}%
\pgfsetstrokecolor{textcolor}%
\pgfsetfillcolor{textcolor}%
\pgftext[x=0.347838in,y=0.479775in,left,base]{\color{textcolor}\rmfamily\fontsize{10.000000}{12.000000}\selectfont \(\displaystyle -0.50\)}%
\end{pgfscope}%
\begin{pgfscope}%
\pgfsetbuttcap%
\pgfsetroundjoin%
\definecolor{currentfill}{rgb}{0.000000,0.000000,0.000000}%
\pgfsetfillcolor{currentfill}%
\pgfsetlinewidth{0.803000pt}%
\definecolor{currentstroke}{rgb}{0.000000,0.000000,0.000000}%
\pgfsetstrokecolor{currentstroke}%
\pgfsetdash{}{0pt}%
\pgfsys@defobject{currentmarker}{\pgfqpoint{-0.048611in}{0.000000in}}{\pgfqpoint{0.000000in}{0.000000in}}{%
\pgfpathmoveto{\pgfqpoint{0.000000in}{0.000000in}}%
\pgfpathlineto{\pgfqpoint{-0.048611in}{0.000000in}}%
\pgfusepath{stroke,fill}%
}%
\begin{pgfscope}%
\pgfsys@transformshift{0.800000in}{0.990000in}%
\pgfsys@useobject{currentmarker}{}%
\end{pgfscope}%
\end{pgfscope}%
\begin{pgfscope}%
\definecolor{textcolor}{rgb}{0.000000,0.000000,0.000000}%
\pgfsetstrokecolor{textcolor}%
\pgfsetfillcolor{textcolor}%
\pgftext[x=0.347838in,y=0.941775in,left,base]{\color{textcolor}\rmfamily\fontsize{10.000000}{12.000000}\selectfont \(\displaystyle -0.25\)}%
\end{pgfscope}%
\begin{pgfscope}%
\pgfsetbuttcap%
\pgfsetroundjoin%
\definecolor{currentfill}{rgb}{0.000000,0.000000,0.000000}%
\pgfsetfillcolor{currentfill}%
\pgfsetlinewidth{0.803000pt}%
\definecolor{currentstroke}{rgb}{0.000000,0.000000,0.000000}%
\pgfsetstrokecolor{currentstroke}%
\pgfsetdash{}{0pt}%
\pgfsys@defobject{currentmarker}{\pgfqpoint{-0.048611in}{0.000000in}}{\pgfqpoint{0.000000in}{0.000000in}}{%
\pgfpathmoveto{\pgfqpoint{0.000000in}{0.000000in}}%
\pgfpathlineto{\pgfqpoint{-0.048611in}{0.000000in}}%
\pgfusepath{stroke,fill}%
}%
\begin{pgfscope}%
\pgfsys@transformshift{0.800000in}{1.452000in}%
\pgfsys@useobject{currentmarker}{}%
\end{pgfscope}%
\end{pgfscope}%
\begin{pgfscope}%
\definecolor{textcolor}{rgb}{0.000000,0.000000,0.000000}%
\pgfsetstrokecolor{textcolor}%
\pgfsetfillcolor{textcolor}%
\pgftext[x=0.455863in,y=1.403775in,left,base]{\color{textcolor}\rmfamily\fontsize{10.000000}{12.000000}\selectfont \(\displaystyle 0.00\)}%
\end{pgfscope}%
\begin{pgfscope}%
\pgfsetbuttcap%
\pgfsetroundjoin%
\definecolor{currentfill}{rgb}{0.000000,0.000000,0.000000}%
\pgfsetfillcolor{currentfill}%
\pgfsetlinewidth{0.803000pt}%
\definecolor{currentstroke}{rgb}{0.000000,0.000000,0.000000}%
\pgfsetstrokecolor{currentstroke}%
\pgfsetdash{}{0pt}%
\pgfsys@defobject{currentmarker}{\pgfqpoint{-0.048611in}{0.000000in}}{\pgfqpoint{0.000000in}{0.000000in}}{%
\pgfpathmoveto{\pgfqpoint{0.000000in}{0.000000in}}%
\pgfpathlineto{\pgfqpoint{-0.048611in}{0.000000in}}%
\pgfusepath{stroke,fill}%
}%
\begin{pgfscope}%
\pgfsys@transformshift{0.800000in}{1.914000in}%
\pgfsys@useobject{currentmarker}{}%
\end{pgfscope}%
\end{pgfscope}%
\begin{pgfscope}%
\definecolor{textcolor}{rgb}{0.000000,0.000000,0.000000}%
\pgfsetstrokecolor{textcolor}%
\pgfsetfillcolor{textcolor}%
\pgftext[x=0.455863in,y=1.865775in,left,base]{\color{textcolor}\rmfamily\fontsize{10.000000}{12.000000}\selectfont \(\displaystyle 0.25\)}%
\end{pgfscope}%
\begin{pgfscope}%
\pgfsetbuttcap%
\pgfsetroundjoin%
\definecolor{currentfill}{rgb}{0.000000,0.000000,0.000000}%
\pgfsetfillcolor{currentfill}%
\pgfsetlinewidth{0.803000pt}%
\definecolor{currentstroke}{rgb}{0.000000,0.000000,0.000000}%
\pgfsetstrokecolor{currentstroke}%
\pgfsetdash{}{0pt}%
\pgfsys@defobject{currentmarker}{\pgfqpoint{-0.048611in}{0.000000in}}{\pgfqpoint{0.000000in}{0.000000in}}{%
\pgfpathmoveto{\pgfqpoint{0.000000in}{0.000000in}}%
\pgfpathlineto{\pgfqpoint{-0.048611in}{0.000000in}}%
\pgfusepath{stroke,fill}%
}%
\begin{pgfscope}%
\pgfsys@transformshift{0.800000in}{2.376000in}%
\pgfsys@useobject{currentmarker}{}%
\end{pgfscope}%
\end{pgfscope}%
\begin{pgfscope}%
\definecolor{textcolor}{rgb}{0.000000,0.000000,0.000000}%
\pgfsetstrokecolor{textcolor}%
\pgfsetfillcolor{textcolor}%
\pgftext[x=0.455863in,y=2.327775in,left,base]{\color{textcolor}\rmfamily\fontsize{10.000000}{12.000000}\selectfont \(\displaystyle 0.50\)}%
\end{pgfscope}%
\begin{pgfscope}%
\pgfsetbuttcap%
\pgfsetroundjoin%
\definecolor{currentfill}{rgb}{0.000000,0.000000,0.000000}%
\pgfsetfillcolor{currentfill}%
\pgfsetlinewidth{0.803000pt}%
\definecolor{currentstroke}{rgb}{0.000000,0.000000,0.000000}%
\pgfsetstrokecolor{currentstroke}%
\pgfsetdash{}{0pt}%
\pgfsys@defobject{currentmarker}{\pgfqpoint{-0.048611in}{0.000000in}}{\pgfqpoint{0.000000in}{0.000000in}}{%
\pgfpathmoveto{\pgfqpoint{0.000000in}{0.000000in}}%
\pgfpathlineto{\pgfqpoint{-0.048611in}{0.000000in}}%
\pgfusepath{stroke,fill}%
}%
\begin{pgfscope}%
\pgfsys@transformshift{0.800000in}{2.838000in}%
\pgfsys@useobject{currentmarker}{}%
\end{pgfscope}%
\end{pgfscope}%
\begin{pgfscope}%
\definecolor{textcolor}{rgb}{0.000000,0.000000,0.000000}%
\pgfsetstrokecolor{textcolor}%
\pgfsetfillcolor{textcolor}%
\pgftext[x=0.455863in,y=2.789775in,left,base]{\color{textcolor}\rmfamily\fontsize{10.000000}{12.000000}\selectfont \(\displaystyle 0.75\)}%
\end{pgfscope}%
\begin{pgfscope}%
\pgfsetbuttcap%
\pgfsetroundjoin%
\definecolor{currentfill}{rgb}{0.000000,0.000000,0.000000}%
\pgfsetfillcolor{currentfill}%
\pgfsetlinewidth{0.803000pt}%
\definecolor{currentstroke}{rgb}{0.000000,0.000000,0.000000}%
\pgfsetstrokecolor{currentstroke}%
\pgfsetdash{}{0pt}%
\pgfsys@defobject{currentmarker}{\pgfqpoint{-0.048611in}{0.000000in}}{\pgfqpoint{0.000000in}{0.000000in}}{%
\pgfpathmoveto{\pgfqpoint{0.000000in}{0.000000in}}%
\pgfpathlineto{\pgfqpoint{-0.048611in}{0.000000in}}%
\pgfusepath{stroke,fill}%
}%
\begin{pgfscope}%
\pgfsys@transformshift{0.800000in}{3.300000in}%
\pgfsys@useobject{currentmarker}{}%
\end{pgfscope}%
\end{pgfscope}%
\begin{pgfscope}%
\definecolor{textcolor}{rgb}{0.000000,0.000000,0.000000}%
\pgfsetstrokecolor{textcolor}%
\pgfsetfillcolor{textcolor}%
\pgftext[x=0.455863in,y=3.251775in,left,base]{\color{textcolor}\rmfamily\fontsize{10.000000}{12.000000}\selectfont \(\displaystyle 1.00\)}%
\end{pgfscope}%
\begin{pgfscope}%
\pgfsetbuttcap%
\pgfsetroundjoin%
\definecolor{currentfill}{rgb}{0.000000,0.000000,0.000000}%
\pgfsetfillcolor{currentfill}%
\pgfsetlinewidth{0.803000pt}%
\definecolor{currentstroke}{rgb}{0.000000,0.000000,0.000000}%
\pgfsetstrokecolor{currentstroke}%
\pgfsetdash{}{0pt}%
\pgfsys@defobject{currentmarker}{\pgfqpoint{-0.048611in}{0.000000in}}{\pgfqpoint{0.000000in}{0.000000in}}{%
\pgfpathmoveto{\pgfqpoint{0.000000in}{0.000000in}}%
\pgfpathlineto{\pgfqpoint{-0.048611in}{0.000000in}}%
\pgfusepath{stroke,fill}%
}%
\begin{pgfscope}%
\pgfsys@transformshift{0.800000in}{3.762000in}%
\pgfsys@useobject{currentmarker}{}%
\end{pgfscope}%
\end{pgfscope}%
\begin{pgfscope}%
\definecolor{textcolor}{rgb}{0.000000,0.000000,0.000000}%
\pgfsetstrokecolor{textcolor}%
\pgfsetfillcolor{textcolor}%
\pgftext[x=0.455863in,y=3.713775in,left,base]{\color{textcolor}\rmfamily\fontsize{10.000000}{12.000000}\selectfont \(\displaystyle 1.25\)}%
\end{pgfscope}%
\begin{pgfscope}%
\pgfsetbuttcap%
\pgfsetroundjoin%
\definecolor{currentfill}{rgb}{0.000000,0.000000,0.000000}%
\pgfsetfillcolor{currentfill}%
\pgfsetlinewidth{0.803000pt}%
\definecolor{currentstroke}{rgb}{0.000000,0.000000,0.000000}%
\pgfsetstrokecolor{currentstroke}%
\pgfsetdash{}{0pt}%
\pgfsys@defobject{currentmarker}{\pgfqpoint{-0.048611in}{0.000000in}}{\pgfqpoint{0.000000in}{0.000000in}}{%
\pgfpathmoveto{\pgfqpoint{0.000000in}{0.000000in}}%
\pgfpathlineto{\pgfqpoint{-0.048611in}{0.000000in}}%
\pgfusepath{stroke,fill}%
}%
\begin{pgfscope}%
\pgfsys@transformshift{0.800000in}{4.224000in}%
\pgfsys@useobject{currentmarker}{}%
\end{pgfscope}%
\end{pgfscope}%
\begin{pgfscope}%
\definecolor{textcolor}{rgb}{0.000000,0.000000,0.000000}%
\pgfsetstrokecolor{textcolor}%
\pgfsetfillcolor{textcolor}%
\pgftext[x=0.455863in,y=4.175775in,left,base]{\color{textcolor}\rmfamily\fontsize{10.000000}{12.000000}\selectfont \(\displaystyle 1.50\)}%
\end{pgfscope}%
\begin{pgfscope}%
\definecolor{textcolor}{rgb}{0.000000,0.000000,0.000000}%
\pgfsetstrokecolor{textcolor}%
\pgfsetfillcolor{textcolor}%
\pgftext[x=0.292283in,y=2.376000in,,bottom,rotate=90.000000]{\color{textcolor}\rmfamily\fontsize{10.000000}{12.000000}\selectfont y}%
\end{pgfscope}%
\begin{pgfscope}%
\pgfpathrectangle{\pgfqpoint{0.800000in}{0.528000in}}{\pgfqpoint{4.960000in}{3.696000in}}%
\pgfusepath{clip}%
\pgfsetrectcap%
\pgfsetroundjoin%
\pgfsetlinewidth{1.505625pt}%
\definecolor{currentstroke}{rgb}{0.121569,0.466667,0.705882}%
\pgfsetstrokecolor{currentstroke}%
\pgfsetdash{}{0pt}%
\pgfpathmoveto{\pgfqpoint{1.025455in}{1.523077in}}%
\pgfpathlineto{\pgfqpoint{1.184066in}{1.533748in}}%
\pgfpathlineto{\pgfqpoint{1.320018in}{1.544892in}}%
\pgfpathlineto{\pgfqpoint{1.455971in}{1.558428in}}%
\pgfpathlineto{\pgfqpoint{1.569265in}{1.572045in}}%
\pgfpathlineto{\pgfqpoint{1.659899in}{1.584860in}}%
\pgfpathlineto{\pgfqpoint{1.750534in}{1.599776in}}%
\pgfpathlineto{\pgfqpoint{1.841169in}{1.617263in}}%
\pgfpathlineto{\pgfqpoint{1.909146in}{1.632419in}}%
\pgfpathlineto{\pgfqpoint{1.977122in}{1.649670in}}%
\pgfpathlineto{\pgfqpoint{2.045098in}{1.669400in}}%
\pgfpathlineto{\pgfqpoint{2.113074in}{1.692080in}}%
\pgfpathlineto{\pgfqpoint{2.158392in}{1.709119in}}%
\pgfpathlineto{\pgfqpoint{2.203709in}{1.727926in}}%
\pgfpathlineto{\pgfqpoint{2.249027in}{1.748735in}}%
\pgfpathlineto{\pgfqpoint{2.294344in}{1.771818in}}%
\pgfpathlineto{\pgfqpoint{2.339662in}{1.797485in}}%
\pgfpathlineto{\pgfqpoint{2.384979in}{1.826095in}}%
\pgfpathlineto{\pgfqpoint{2.430297in}{1.858060in}}%
\pgfpathlineto{\pgfqpoint{2.475614in}{1.893855in}}%
\pgfpathlineto{\pgfqpoint{2.520932in}{1.934017in}}%
\pgfpathlineto{\pgfqpoint{2.566249in}{1.979154in}}%
\pgfpathlineto{\pgfqpoint{2.611567in}{2.029943in}}%
\pgfpathlineto{\pgfqpoint{2.656884in}{2.087123in}}%
\pgfpathlineto{\pgfqpoint{2.702202in}{2.151469in}}%
\pgfpathlineto{\pgfqpoint{2.724861in}{2.186574in}}%
\pgfpathlineto{\pgfqpoint{2.747519in}{2.223757in}}%
\pgfpathlineto{\pgfqpoint{2.770178in}{2.263104in}}%
\pgfpathlineto{\pgfqpoint{2.815496in}{2.348561in}}%
\pgfpathlineto{\pgfqpoint{2.860813in}{2.443286in}}%
\pgfpathlineto{\pgfqpoint{2.906131in}{2.547122in}}%
\pgfpathlineto{\pgfqpoint{2.951448in}{2.659117in}}%
\pgfpathlineto{\pgfqpoint{3.019424in}{2.837353in}}%
\pgfpathlineto{\pgfqpoint{3.087401in}{3.014864in}}%
\pgfpathlineto{\pgfqpoint{3.110059in}{3.070154in}}%
\pgfpathlineto{\pgfqpoint{3.132718in}{3.121845in}}%
\pgfpathlineto{\pgfqpoint{3.155377in}{3.168855in}}%
\pgfpathlineto{\pgfqpoint{3.178036in}{3.210099in}}%
\pgfpathlineto{\pgfqpoint{3.200694in}{3.244550in}}%
\pgfpathlineto{\pgfqpoint{3.223353in}{3.271287in}}%
\pgfpathlineto{\pgfqpoint{3.246012in}{3.289560in}}%
\pgfpathlineto{\pgfqpoint{3.268671in}{3.298834in}}%
\pgfpathlineto{\pgfqpoint{3.291329in}{3.298834in}}%
\pgfpathlineto{\pgfqpoint{3.313988in}{3.289560in}}%
\pgfpathlineto{\pgfqpoint{3.336647in}{3.271287in}}%
\pgfpathlineto{\pgfqpoint{3.359306in}{3.244550in}}%
\pgfpathlineto{\pgfqpoint{3.381964in}{3.210099in}}%
\pgfpathlineto{\pgfqpoint{3.404623in}{3.168855in}}%
\pgfpathlineto{\pgfqpoint{3.427282in}{3.121845in}}%
\pgfpathlineto{\pgfqpoint{3.449941in}{3.070154in}}%
\pgfpathlineto{\pgfqpoint{3.495258in}{2.957011in}}%
\pgfpathlineto{\pgfqpoint{3.631211in}{2.602200in}}%
\pgfpathlineto{\pgfqpoint{3.676528in}{2.494102in}}%
\pgfpathlineto{\pgfqpoint{3.721846in}{2.394759in}}%
\pgfpathlineto{\pgfqpoint{3.767163in}{2.304686in}}%
\pgfpathlineto{\pgfqpoint{3.812481in}{2.223757in}}%
\pgfpathlineto{\pgfqpoint{3.857798in}{2.151469in}}%
\pgfpathlineto{\pgfqpoint{3.903116in}{2.087123in}}%
\pgfpathlineto{\pgfqpoint{3.948433in}{2.029943in}}%
\pgfpathlineto{\pgfqpoint{3.993751in}{1.979154in}}%
\pgfpathlineto{\pgfqpoint{4.039068in}{1.934017in}}%
\pgfpathlineto{\pgfqpoint{4.084386in}{1.893855in}}%
\pgfpathlineto{\pgfqpoint{4.129703in}{1.858060in}}%
\pgfpathlineto{\pgfqpoint{4.175021in}{1.826095in}}%
\pgfpathlineto{\pgfqpoint{4.220338in}{1.797485in}}%
\pgfpathlineto{\pgfqpoint{4.265656in}{1.771818in}}%
\pgfpathlineto{\pgfqpoint{4.310973in}{1.748735in}}%
\pgfpathlineto{\pgfqpoint{4.356291in}{1.727926in}}%
\pgfpathlineto{\pgfqpoint{4.401608in}{1.709119in}}%
\pgfpathlineto{\pgfqpoint{4.446926in}{1.692080in}}%
\pgfpathlineto{\pgfqpoint{4.514902in}{1.669400in}}%
\pgfpathlineto{\pgfqpoint{4.582878in}{1.649670in}}%
\pgfpathlineto{\pgfqpoint{4.650854in}{1.632419in}}%
\pgfpathlineto{\pgfqpoint{4.718831in}{1.617263in}}%
\pgfpathlineto{\pgfqpoint{4.809466in}{1.599776in}}%
\pgfpathlineto{\pgfqpoint{4.900101in}{1.584860in}}%
\pgfpathlineto{\pgfqpoint{4.990735in}{1.572045in}}%
\pgfpathlineto{\pgfqpoint{5.104029in}{1.558428in}}%
\pgfpathlineto{\pgfqpoint{5.217323in}{1.546965in}}%
\pgfpathlineto{\pgfqpoint{5.353275in}{1.535463in}}%
\pgfpathlineto{\pgfqpoint{5.511887in}{1.524471in}}%
\pgfpathlineto{\pgfqpoint{5.534545in}{1.523077in}}%
\pgfpathlineto{\pgfqpoint{5.534545in}{1.523077in}}%
\pgfusepath{stroke}%
\end{pgfscope}%
\begin{pgfscope}%
\pgfpathrectangle{\pgfqpoint{0.800000in}{0.528000in}}{\pgfqpoint{4.960000in}{3.696000in}}%
\pgfusepath{clip}%
\pgfsetbuttcap%
\pgfsetroundjoin%
\definecolor{currentfill}{rgb}{1.000000,0.498039,0.054902}%
\pgfsetfillcolor{currentfill}%
\pgfsetlinewidth{1.003750pt}%
\definecolor{currentstroke}{rgb}{1.000000,0.498039,0.054902}%
\pgfsetstrokecolor{currentstroke}%
\pgfsetdash{}{0pt}%
\pgfsys@defobject{currentmarker}{\pgfqpoint{-0.020833in}{-0.020833in}}{\pgfqpoint{0.020833in}{0.020833in}}{%
\pgfpathmoveto{\pgfqpoint{0.000000in}{-0.020833in}}%
\pgfpathcurveto{\pgfqpoint{0.005525in}{-0.020833in}}{\pgfqpoint{0.010825in}{-0.018638in}}{\pgfqpoint{0.014731in}{-0.014731in}}%
\pgfpathcurveto{\pgfqpoint{0.018638in}{-0.010825in}}{\pgfqpoint{0.020833in}{-0.005525in}}{\pgfqpoint{0.020833in}{0.000000in}}%
\pgfpathcurveto{\pgfqpoint{0.020833in}{0.005525in}}{\pgfqpoint{0.018638in}{0.010825in}}{\pgfqpoint{0.014731in}{0.014731in}}%
\pgfpathcurveto{\pgfqpoint{0.010825in}{0.018638in}}{\pgfqpoint{0.005525in}{0.020833in}}{\pgfqpoint{0.000000in}{0.020833in}}%
\pgfpathcurveto{\pgfqpoint{-0.005525in}{0.020833in}}{\pgfqpoint{-0.010825in}{0.018638in}}{\pgfqpoint{-0.014731in}{0.014731in}}%
\pgfpathcurveto{\pgfqpoint{-0.018638in}{0.010825in}}{\pgfqpoint{-0.020833in}{0.005525in}}{\pgfqpoint{-0.020833in}{0.000000in}}%
\pgfpathcurveto{\pgfqpoint{-0.020833in}{-0.005525in}}{\pgfqpoint{-0.018638in}{-0.010825in}}{\pgfqpoint{-0.014731in}{-0.014731in}}%
\pgfpathcurveto{\pgfqpoint{-0.010825in}{-0.018638in}}{\pgfqpoint{-0.005525in}{-0.020833in}}{\pgfqpoint{0.000000in}{-0.020833in}}%
\pgfpathclose%
\pgfusepath{stroke,fill}%
}%
\begin{pgfscope}%
\pgfsys@transformshift{1.025455in}{1.523077in}%
\pgfsys@useobject{currentmarker}{}%
\end{pgfscope}%
\begin{pgfscope}%
\pgfsys@transformshift{2.528485in}{1.941176in}%
\pgfsys@useobject{currentmarker}{}%
\end{pgfscope}%
\begin{pgfscope}%
\pgfsys@transformshift{4.031515in}{1.941176in}%
\pgfsys@useobject{currentmarker}{}%
\end{pgfscope}%
\end{pgfscope}%
\begin{pgfscope}%
\pgfpathrectangle{\pgfqpoint{0.800000in}{0.528000in}}{\pgfqpoint{4.960000in}{3.696000in}}%
\pgfusepath{clip}%
\pgfsetrectcap%
\pgfsetroundjoin%
\pgfsetlinewidth{1.505625pt}%
\definecolor{currentstroke}{rgb}{0.172549,0.627451,0.172549}%
\pgfsetstrokecolor{currentstroke}%
\pgfsetdash{}{0pt}%
\pgfpathmoveto{\pgfqpoint{1.025455in}{1.523077in}}%
\pgfpathlineto{\pgfqpoint{1.138748in}{1.569162in}}%
\pgfpathlineto{\pgfqpoint{1.252042in}{1.612871in}}%
\pgfpathlineto{\pgfqpoint{1.365336in}{1.654205in}}%
\pgfpathlineto{\pgfqpoint{1.478630in}{1.693163in}}%
\pgfpathlineto{\pgfqpoint{1.591923in}{1.729746in}}%
\pgfpathlineto{\pgfqpoint{1.705217in}{1.763953in}}%
\pgfpathlineto{\pgfqpoint{1.818511in}{1.795785in}}%
\pgfpathlineto{\pgfqpoint{1.931804in}{1.825241in}}%
\pgfpathlineto{\pgfqpoint{2.045098in}{1.852322in}}%
\pgfpathlineto{\pgfqpoint{2.158392in}{1.877027in}}%
\pgfpathlineto{\pgfqpoint{2.271686in}{1.899357in}}%
\pgfpathlineto{\pgfqpoint{2.384979in}{1.919311in}}%
\pgfpathlineto{\pgfqpoint{2.498273in}{1.936890in}}%
\pgfpathlineto{\pgfqpoint{2.611567in}{1.952093in}}%
\pgfpathlineto{\pgfqpoint{2.724861in}{1.964921in}}%
\pgfpathlineto{\pgfqpoint{2.838154in}{1.975373in}}%
\pgfpathlineto{\pgfqpoint{2.951448in}{1.983450in}}%
\pgfpathlineto{\pgfqpoint{3.064742in}{1.989151in}}%
\pgfpathlineto{\pgfqpoint{3.178036in}{1.992477in}}%
\pgfpathlineto{\pgfqpoint{3.291329in}{1.993427in}}%
\pgfpathlineto{\pgfqpoint{3.404623in}{1.992002in}}%
\pgfpathlineto{\pgfqpoint{3.517917in}{1.988201in}}%
\pgfpathlineto{\pgfqpoint{3.631211in}{1.982025in}}%
\pgfpathlineto{\pgfqpoint{3.744504in}{1.973473in}}%
\pgfpathlineto{\pgfqpoint{3.857798in}{1.962545in}}%
\pgfpathlineto{\pgfqpoint{3.971092in}{1.949243in}}%
\pgfpathlineto{\pgfqpoint{4.084386in}{1.933564in}}%
\pgfpathlineto{\pgfqpoint{4.197679in}{1.915510in}}%
\pgfpathlineto{\pgfqpoint{4.310973in}{1.895081in}}%
\pgfpathlineto{\pgfqpoint{4.424267in}{1.872276in}}%
\pgfpathlineto{\pgfqpoint{4.537561in}{1.847096in}}%
\pgfpathlineto{\pgfqpoint{4.650854in}{1.819540in}}%
\pgfpathlineto{\pgfqpoint{4.764148in}{1.789609in}}%
\pgfpathlineto{\pgfqpoint{4.877442in}{1.757302in}}%
\pgfpathlineto{\pgfqpoint{4.990735in}{1.722619in}}%
\pgfpathlineto{\pgfqpoint{5.104029in}{1.685562in}}%
\pgfpathlineto{\pgfqpoint{5.217323in}{1.646128in}}%
\pgfpathlineto{\pgfqpoint{5.330617in}{1.604319in}}%
\pgfpathlineto{\pgfqpoint{5.443910in}{1.560135in}}%
\pgfpathlineto{\pgfqpoint{5.534545in}{1.523077in}}%
\pgfpathlineto{\pgfqpoint{5.534545in}{1.523077in}}%
\pgfusepath{stroke}%
\end{pgfscope}%
\begin{pgfscope}%
\pgfpathrectangle{\pgfqpoint{0.800000in}{0.528000in}}{\pgfqpoint{4.960000in}{3.696000in}}%
\pgfusepath{clip}%
\pgfsetbuttcap%
\pgfsetroundjoin%
\definecolor{currentfill}{rgb}{0.839216,0.152941,0.156863}%
\pgfsetfillcolor{currentfill}%
\pgfsetlinewidth{1.003750pt}%
\definecolor{currentstroke}{rgb}{0.839216,0.152941,0.156863}%
\pgfsetstrokecolor{currentstroke}%
\pgfsetdash{}{0pt}%
\pgfsys@defobject{currentmarker}{\pgfqpoint{-0.020833in}{-0.020833in}}{\pgfqpoint{0.020833in}{0.020833in}}{%
\pgfpathmoveto{\pgfqpoint{0.000000in}{-0.020833in}}%
\pgfpathcurveto{\pgfqpoint{0.005525in}{-0.020833in}}{\pgfqpoint{0.010825in}{-0.018638in}}{\pgfqpoint{0.014731in}{-0.014731in}}%
\pgfpathcurveto{\pgfqpoint{0.018638in}{-0.010825in}}{\pgfqpoint{0.020833in}{-0.005525in}}{\pgfqpoint{0.020833in}{0.000000in}}%
\pgfpathcurveto{\pgfqpoint{0.020833in}{0.005525in}}{\pgfqpoint{0.018638in}{0.010825in}}{\pgfqpoint{0.014731in}{0.014731in}}%
\pgfpathcurveto{\pgfqpoint{0.010825in}{0.018638in}}{\pgfqpoint{0.005525in}{0.020833in}}{\pgfqpoint{0.000000in}{0.020833in}}%
\pgfpathcurveto{\pgfqpoint{-0.005525in}{0.020833in}}{\pgfqpoint{-0.010825in}{0.018638in}}{\pgfqpoint{-0.014731in}{0.014731in}}%
\pgfpathcurveto{\pgfqpoint{-0.018638in}{0.010825in}}{\pgfqpoint{-0.020833in}{0.005525in}}{\pgfqpoint{-0.020833in}{0.000000in}}%
\pgfpathcurveto{\pgfqpoint{-0.020833in}{-0.005525in}}{\pgfqpoint{-0.018638in}{-0.010825in}}{\pgfqpoint{-0.014731in}{-0.014731in}}%
\pgfpathcurveto{\pgfqpoint{-0.010825in}{-0.018638in}}{\pgfqpoint{-0.005525in}{-0.020833in}}{\pgfqpoint{0.000000in}{-0.020833in}}%
\pgfpathclose%
\pgfusepath{stroke,fill}%
}%
\begin{pgfscope}%
\pgfsys@transformshift{1.025455in}{1.523077in}%
\pgfsys@useobject{currentmarker}{}%
\end{pgfscope}%
\begin{pgfscope}%
\pgfsys@transformshift{1.927273in}{1.636800in}%
\pgfsys@useobject{currentmarker}{}%
\end{pgfscope}%
\begin{pgfscope}%
\pgfsys@transformshift{2.829091in}{2.376000in}%
\pgfsys@useobject{currentmarker}{}%
\end{pgfscope}%
\begin{pgfscope}%
\pgfsys@transformshift{3.730909in}{2.376000in}%
\pgfsys@useobject{currentmarker}{}%
\end{pgfscope}%
\begin{pgfscope}%
\pgfsys@transformshift{4.632727in}{1.636800in}%
\pgfsys@useobject{currentmarker}{}%
\end{pgfscope}%
\end{pgfscope}%
\begin{pgfscope}%
\pgfpathrectangle{\pgfqpoint{0.800000in}{0.528000in}}{\pgfqpoint{4.960000in}{3.696000in}}%
\pgfusepath{clip}%
\pgfsetrectcap%
\pgfsetroundjoin%
\pgfsetlinewidth{1.505625pt}%
\definecolor{currentstroke}{rgb}{0.580392,0.403922,0.741176}%
\pgfsetstrokecolor{currentstroke}%
\pgfsetdash{}{0pt}%
\pgfpathmoveto{\pgfqpoint{1.025455in}{1.523077in}}%
\pgfpathlineto{\pgfqpoint{1.048113in}{1.499089in}}%
\pgfpathlineto{\pgfqpoint{1.070772in}{1.477093in}}%
\pgfpathlineto{\pgfqpoint{1.093431in}{1.457037in}}%
\pgfpathlineto{\pgfqpoint{1.116090in}{1.438867in}}%
\pgfpathlineto{\pgfqpoint{1.138748in}{1.422530in}}%
\pgfpathlineto{\pgfqpoint{1.161407in}{1.407976in}}%
\pgfpathlineto{\pgfqpoint{1.184066in}{1.395154in}}%
\pgfpathlineto{\pgfqpoint{1.206725in}{1.384012in}}%
\pgfpathlineto{\pgfqpoint{1.229383in}{1.374500in}}%
\pgfpathlineto{\pgfqpoint{1.252042in}{1.366570in}}%
\pgfpathlineto{\pgfqpoint{1.274701in}{1.360172in}}%
\pgfpathlineto{\pgfqpoint{1.297360in}{1.355258in}}%
\pgfpathlineto{\pgfqpoint{1.320018in}{1.351779in}}%
\pgfpathlineto{\pgfqpoint{1.342677in}{1.349690in}}%
\pgfpathlineto{\pgfqpoint{1.365336in}{1.348942in}}%
\pgfpathlineto{\pgfqpoint{1.387995in}{1.349489in}}%
\pgfpathlineto{\pgfqpoint{1.410653in}{1.351287in}}%
\pgfpathlineto{\pgfqpoint{1.433312in}{1.354289in}}%
\pgfpathlineto{\pgfqpoint{1.478630in}{1.363731in}}%
\pgfpathlineto{\pgfqpoint{1.523947in}{1.377463in}}%
\pgfpathlineto{\pgfqpoint{1.569265in}{1.395145in}}%
\pgfpathlineto{\pgfqpoint{1.614582in}{1.416443in}}%
\pgfpathlineto{\pgfqpoint{1.659899in}{1.441034in}}%
\pgfpathlineto{\pgfqpoint{1.705217in}{1.468601in}}%
\pgfpathlineto{\pgfqpoint{1.750534in}{1.498839in}}%
\pgfpathlineto{\pgfqpoint{1.795852in}{1.531450in}}%
\pgfpathlineto{\pgfqpoint{1.841169in}{1.566143in}}%
\pgfpathlineto{\pgfqpoint{1.909146in}{1.621476in}}%
\pgfpathlineto{\pgfqpoint{1.977122in}{1.679957in}}%
\pgfpathlineto{\pgfqpoint{2.067757in}{1.761334in}}%
\pgfpathlineto{\pgfqpoint{2.407638in}{2.071304in}}%
\pgfpathlineto{\pgfqpoint{2.475614in}{2.129229in}}%
\pgfpathlineto{\pgfqpoint{2.543591in}{2.184426in}}%
\pgfpathlineto{\pgfqpoint{2.611567in}{2.236396in}}%
\pgfpathlineto{\pgfqpoint{2.679543in}{2.284685in}}%
\pgfpathlineto{\pgfqpoint{2.724861in}{2.314627in}}%
\pgfpathlineto{\pgfqpoint{2.770178in}{2.342639in}}%
\pgfpathlineto{\pgfqpoint{2.815496in}{2.368617in}}%
\pgfpathlineto{\pgfqpoint{2.860813in}{2.392469in}}%
\pgfpathlineto{\pgfqpoint{2.906131in}{2.414109in}}%
\pgfpathlineto{\pgfqpoint{2.951448in}{2.433461in}}%
\pgfpathlineto{\pgfqpoint{2.996766in}{2.450458in}}%
\pgfpathlineto{\pgfqpoint{3.042083in}{2.465042in}}%
\pgfpathlineto{\pgfqpoint{3.087401in}{2.477161in}}%
\pgfpathlineto{\pgfqpoint{3.132718in}{2.486775in}}%
\pgfpathlineto{\pgfqpoint{3.178036in}{2.493852in}}%
\pgfpathlineto{\pgfqpoint{3.223353in}{2.498366in}}%
\pgfpathlineto{\pgfqpoint{3.268671in}{2.500304in}}%
\pgfpathlineto{\pgfqpoint{3.313988in}{2.499658in}}%
\pgfpathlineto{\pgfqpoint{3.359306in}{2.496430in}}%
\pgfpathlineto{\pgfqpoint{3.404623in}{2.490633in}}%
\pgfpathlineto{\pgfqpoint{3.449941in}{2.482284in}}%
\pgfpathlineto{\pgfqpoint{3.495258in}{2.471412in}}%
\pgfpathlineto{\pgfqpoint{3.540576in}{2.458055in}}%
\pgfpathlineto{\pgfqpoint{3.585893in}{2.442258in}}%
\pgfpathlineto{\pgfqpoint{3.631211in}{2.424075in}}%
\pgfpathlineto{\pgfqpoint{3.676528in}{2.403570in}}%
\pgfpathlineto{\pgfqpoint{3.721846in}{2.380814in}}%
\pgfpathlineto{\pgfqpoint{3.767163in}{2.355888in}}%
\pgfpathlineto{\pgfqpoint{3.812481in}{2.328881in}}%
\pgfpathlineto{\pgfqpoint{3.857798in}{2.299891in}}%
\pgfpathlineto{\pgfqpoint{3.903116in}{2.269024in}}%
\pgfpathlineto{\pgfqpoint{3.971092in}{2.219460in}}%
\pgfpathlineto{\pgfqpoint{4.039068in}{2.166361in}}%
\pgfpathlineto{\pgfqpoint{4.107044in}{2.110198in}}%
\pgfpathlineto{\pgfqpoint{4.175021in}{2.051483in}}%
\pgfpathlineto{\pgfqpoint{4.265656in}{1.970199in}}%
\pgfpathlineto{\pgfqpoint{4.424267in}{1.823863in}}%
\pgfpathlineto{\pgfqpoint{4.537561in}{1.720262in}}%
\pgfpathlineto{\pgfqpoint{4.628196in}{1.640664in}}%
\pgfpathlineto{\pgfqpoint{4.696172in}{1.584182in}}%
\pgfpathlineto{\pgfqpoint{4.764148in}{1.531450in}}%
\pgfpathlineto{\pgfqpoint{4.809466in}{1.498839in}}%
\pgfpathlineto{\pgfqpoint{4.854783in}{1.468601in}}%
\pgfpathlineto{\pgfqpoint{4.900101in}{1.441034in}}%
\pgfpathlineto{\pgfqpoint{4.945418in}{1.416443in}}%
\pgfpathlineto{\pgfqpoint{4.990735in}{1.395145in}}%
\pgfpathlineto{\pgfqpoint{5.036053in}{1.377463in}}%
\pgfpathlineto{\pgfqpoint{5.081370in}{1.363731in}}%
\pgfpathlineto{\pgfqpoint{5.126688in}{1.354289in}}%
\pgfpathlineto{\pgfqpoint{5.149347in}{1.351287in}}%
\pgfpathlineto{\pgfqpoint{5.172005in}{1.349489in}}%
\pgfpathlineto{\pgfqpoint{5.194664in}{1.348942in}}%
\pgfpathlineto{\pgfqpoint{5.217323in}{1.349690in}}%
\pgfpathlineto{\pgfqpoint{5.239982in}{1.351779in}}%
\pgfpathlineto{\pgfqpoint{5.262640in}{1.355258in}}%
\pgfpathlineto{\pgfqpoint{5.285299in}{1.360172in}}%
\pgfpathlineto{\pgfqpoint{5.307958in}{1.366570in}}%
\pgfpathlineto{\pgfqpoint{5.330617in}{1.374500in}}%
\pgfpathlineto{\pgfqpoint{5.353275in}{1.384012in}}%
\pgfpathlineto{\pgfqpoint{5.375934in}{1.395154in}}%
\pgfpathlineto{\pgfqpoint{5.398593in}{1.407976in}}%
\pgfpathlineto{\pgfqpoint{5.421252in}{1.422530in}}%
\pgfpathlineto{\pgfqpoint{5.443910in}{1.438867in}}%
\pgfpathlineto{\pgfqpoint{5.466569in}{1.457037in}}%
\pgfpathlineto{\pgfqpoint{5.489228in}{1.477093in}}%
\pgfpathlineto{\pgfqpoint{5.511887in}{1.499089in}}%
\pgfpathlineto{\pgfqpoint{5.534545in}{1.523077in}}%
\pgfpathlineto{\pgfqpoint{5.534545in}{1.523077in}}%
\pgfusepath{stroke}%
\end{pgfscope}%
\begin{pgfscope}%
\pgfpathrectangle{\pgfqpoint{0.800000in}{0.528000in}}{\pgfqpoint{4.960000in}{3.696000in}}%
\pgfusepath{clip}%
\pgfsetrectcap%
\pgfsetroundjoin%
\pgfsetlinewidth{1.505625pt}%
\definecolor{currentstroke}{rgb}{0.549020,0.337255,0.294118}%
\pgfsetstrokecolor{currentstroke}%
\pgfsetdash{}{0pt}%
\pgfpathmoveto{\pgfqpoint{1.025455in}{1.523077in}}%
\pgfpathlineto{\pgfqpoint{1.184066in}{1.533748in}}%
\pgfpathlineto{\pgfqpoint{1.320018in}{1.544892in}}%
\pgfpathlineto{\pgfqpoint{1.455971in}{1.558428in}}%
\pgfpathlineto{\pgfqpoint{1.569265in}{1.572045in}}%
\pgfpathlineto{\pgfqpoint{1.659899in}{1.584860in}}%
\pgfpathlineto{\pgfqpoint{1.750534in}{1.599776in}}%
\pgfpathlineto{\pgfqpoint{1.841169in}{1.617263in}}%
\pgfpathlineto{\pgfqpoint{1.909146in}{1.632419in}}%
\pgfpathlineto{\pgfqpoint{1.977122in}{1.649670in}}%
\pgfpathlineto{\pgfqpoint{2.045098in}{1.669400in}}%
\pgfpathlineto{\pgfqpoint{2.113074in}{1.692080in}}%
\pgfpathlineto{\pgfqpoint{2.158392in}{1.709119in}}%
\pgfpathlineto{\pgfqpoint{2.203709in}{1.727926in}}%
\pgfpathlineto{\pgfqpoint{2.249027in}{1.748735in}}%
\pgfpathlineto{\pgfqpoint{2.294344in}{1.771818in}}%
\pgfpathlineto{\pgfqpoint{2.339662in}{1.797485in}}%
\pgfpathlineto{\pgfqpoint{2.384979in}{1.826095in}}%
\pgfpathlineto{\pgfqpoint{2.430297in}{1.858060in}}%
\pgfpathlineto{\pgfqpoint{2.475614in}{1.893855in}}%
\pgfpathlineto{\pgfqpoint{2.520932in}{1.934017in}}%
\pgfpathlineto{\pgfqpoint{2.566249in}{1.979154in}}%
\pgfpathlineto{\pgfqpoint{2.611567in}{2.029943in}}%
\pgfpathlineto{\pgfqpoint{2.656884in}{2.087123in}}%
\pgfpathlineto{\pgfqpoint{2.702202in}{2.151469in}}%
\pgfpathlineto{\pgfqpoint{2.724861in}{2.186574in}}%
\pgfpathlineto{\pgfqpoint{2.747519in}{2.223757in}}%
\pgfpathlineto{\pgfqpoint{2.770178in}{2.263104in}}%
\pgfpathlineto{\pgfqpoint{2.815496in}{2.348561in}}%
\pgfpathlineto{\pgfqpoint{2.860813in}{2.443286in}}%
\pgfpathlineto{\pgfqpoint{2.906131in}{2.547122in}}%
\pgfpathlineto{\pgfqpoint{2.951448in}{2.659117in}}%
\pgfpathlineto{\pgfqpoint{3.019424in}{2.837353in}}%
\pgfpathlineto{\pgfqpoint{3.087401in}{3.014864in}}%
\pgfpathlineto{\pgfqpoint{3.110059in}{3.070154in}}%
\pgfpathlineto{\pgfqpoint{3.132718in}{3.121845in}}%
\pgfpathlineto{\pgfqpoint{3.155377in}{3.168855in}}%
\pgfpathlineto{\pgfqpoint{3.178036in}{3.210099in}}%
\pgfpathlineto{\pgfqpoint{3.200694in}{3.244550in}}%
\pgfpathlineto{\pgfqpoint{3.223353in}{3.271287in}}%
\pgfpathlineto{\pgfqpoint{3.246012in}{3.289560in}}%
\pgfpathlineto{\pgfqpoint{3.268671in}{3.298834in}}%
\pgfpathlineto{\pgfqpoint{3.291329in}{3.298834in}}%
\pgfpathlineto{\pgfqpoint{3.313988in}{3.289560in}}%
\pgfpathlineto{\pgfqpoint{3.336647in}{3.271287in}}%
\pgfpathlineto{\pgfqpoint{3.359306in}{3.244550in}}%
\pgfpathlineto{\pgfqpoint{3.381964in}{3.210099in}}%
\pgfpathlineto{\pgfqpoint{3.404623in}{3.168855in}}%
\pgfpathlineto{\pgfqpoint{3.427282in}{3.121845in}}%
\pgfpathlineto{\pgfqpoint{3.449941in}{3.070154in}}%
\pgfpathlineto{\pgfqpoint{3.495258in}{2.957011in}}%
\pgfpathlineto{\pgfqpoint{3.631211in}{2.602200in}}%
\pgfpathlineto{\pgfqpoint{3.676528in}{2.494102in}}%
\pgfpathlineto{\pgfqpoint{3.721846in}{2.394759in}}%
\pgfpathlineto{\pgfqpoint{3.767163in}{2.304686in}}%
\pgfpathlineto{\pgfqpoint{3.812481in}{2.223757in}}%
\pgfpathlineto{\pgfqpoint{3.857798in}{2.151469in}}%
\pgfpathlineto{\pgfqpoint{3.903116in}{2.087123in}}%
\pgfpathlineto{\pgfqpoint{3.948433in}{2.029943in}}%
\pgfpathlineto{\pgfqpoint{3.993751in}{1.979154in}}%
\pgfpathlineto{\pgfqpoint{4.039068in}{1.934017in}}%
\pgfpathlineto{\pgfqpoint{4.084386in}{1.893855in}}%
\pgfpathlineto{\pgfqpoint{4.129703in}{1.858060in}}%
\pgfpathlineto{\pgfqpoint{4.175021in}{1.826095in}}%
\pgfpathlineto{\pgfqpoint{4.220338in}{1.797485in}}%
\pgfpathlineto{\pgfqpoint{4.265656in}{1.771818in}}%
\pgfpathlineto{\pgfqpoint{4.310973in}{1.748735in}}%
\pgfpathlineto{\pgfqpoint{4.356291in}{1.727926in}}%
\pgfpathlineto{\pgfqpoint{4.401608in}{1.709119in}}%
\pgfpathlineto{\pgfqpoint{4.446926in}{1.692080in}}%
\pgfpathlineto{\pgfqpoint{4.514902in}{1.669400in}}%
\pgfpathlineto{\pgfqpoint{4.582878in}{1.649670in}}%
\pgfpathlineto{\pgfqpoint{4.650854in}{1.632419in}}%
\pgfpathlineto{\pgfqpoint{4.718831in}{1.617263in}}%
\pgfpathlineto{\pgfqpoint{4.809466in}{1.599776in}}%
\pgfpathlineto{\pgfqpoint{4.900101in}{1.584860in}}%
\pgfpathlineto{\pgfqpoint{4.990735in}{1.572045in}}%
\pgfpathlineto{\pgfqpoint{5.104029in}{1.558428in}}%
\pgfpathlineto{\pgfqpoint{5.217323in}{1.546965in}}%
\pgfpathlineto{\pgfqpoint{5.353275in}{1.535463in}}%
\pgfpathlineto{\pgfqpoint{5.511887in}{1.524471in}}%
\pgfpathlineto{\pgfqpoint{5.534545in}{1.523077in}}%
\pgfpathlineto{\pgfqpoint{5.534545in}{1.523077in}}%
\pgfusepath{stroke}%
\end{pgfscope}%
\begin{pgfscope}%
\pgfpathrectangle{\pgfqpoint{0.800000in}{0.528000in}}{\pgfqpoint{4.960000in}{3.696000in}}%
\pgfusepath{clip}%
\pgfsetbuttcap%
\pgfsetroundjoin%
\definecolor{currentfill}{rgb}{0.890196,0.466667,0.760784}%
\pgfsetfillcolor{currentfill}%
\pgfsetlinewidth{1.003750pt}%
\definecolor{currentstroke}{rgb}{0.890196,0.466667,0.760784}%
\pgfsetstrokecolor{currentstroke}%
\pgfsetdash{}{0pt}%
\pgfsys@defobject{currentmarker}{\pgfqpoint{-0.020833in}{-0.020833in}}{\pgfqpoint{0.020833in}{0.020833in}}{%
\pgfpathmoveto{\pgfqpoint{0.000000in}{-0.020833in}}%
\pgfpathcurveto{\pgfqpoint{0.005525in}{-0.020833in}}{\pgfqpoint{0.010825in}{-0.018638in}}{\pgfqpoint{0.014731in}{-0.014731in}}%
\pgfpathcurveto{\pgfqpoint{0.018638in}{-0.010825in}}{\pgfqpoint{0.020833in}{-0.005525in}}{\pgfqpoint{0.020833in}{0.000000in}}%
\pgfpathcurveto{\pgfqpoint{0.020833in}{0.005525in}}{\pgfqpoint{0.018638in}{0.010825in}}{\pgfqpoint{0.014731in}{0.014731in}}%
\pgfpathcurveto{\pgfqpoint{0.010825in}{0.018638in}}{\pgfqpoint{0.005525in}{0.020833in}}{\pgfqpoint{0.000000in}{0.020833in}}%
\pgfpathcurveto{\pgfqpoint{-0.005525in}{0.020833in}}{\pgfqpoint{-0.010825in}{0.018638in}}{\pgfqpoint{-0.014731in}{0.014731in}}%
\pgfpathcurveto{\pgfqpoint{-0.018638in}{0.010825in}}{\pgfqpoint{-0.020833in}{0.005525in}}{\pgfqpoint{-0.020833in}{0.000000in}}%
\pgfpathcurveto{\pgfqpoint{-0.020833in}{-0.005525in}}{\pgfqpoint{-0.018638in}{-0.010825in}}{\pgfqpoint{-0.014731in}{-0.014731in}}%
\pgfpathcurveto{\pgfqpoint{-0.010825in}{-0.018638in}}{\pgfqpoint{-0.005525in}{-0.020833in}}{\pgfqpoint{0.000000in}{-0.020833in}}%
\pgfpathclose%
\pgfusepath{stroke,fill}%
}%
\begin{pgfscope}%
\pgfsys@transformshift{5.232494in}{1.545570in}%
\pgfsys@useobject{currentmarker}{}%
\end{pgfscope}%
\begin{pgfscope}%
\pgfsys@transformshift{3.280000in}{3.300000in}%
\pgfsys@useobject{currentmarker}{}%
\end{pgfscope}%
\begin{pgfscope}%
\pgfsys@transformshift{1.327506in}{1.545570in}%
\pgfsys@useobject{currentmarker}{}%
\end{pgfscope}%
\end{pgfscope}%
\begin{pgfscope}%
\pgfpathrectangle{\pgfqpoint{0.800000in}{0.528000in}}{\pgfqpoint{4.960000in}{3.696000in}}%
\pgfusepath{clip}%
\pgfsetrectcap%
\pgfsetroundjoin%
\pgfsetlinewidth{1.505625pt}%
\definecolor{currentstroke}{rgb}{0.498039,0.498039,0.498039}%
\pgfsetstrokecolor{currentstroke}%
\pgfsetdash{}{0pt}%
\pgfpathmoveto{\pgfqpoint{1.025455in}{0.960759in}}%
\pgfpathlineto{\pgfqpoint{1.093431in}{1.099693in}}%
\pgfpathlineto{\pgfqpoint{1.161407in}{1.234373in}}%
\pgfpathlineto{\pgfqpoint{1.229383in}{1.364800in}}%
\pgfpathlineto{\pgfqpoint{1.297360in}{1.490974in}}%
\pgfpathlineto{\pgfqpoint{1.365336in}{1.612895in}}%
\pgfpathlineto{\pgfqpoint{1.433312in}{1.730563in}}%
\pgfpathlineto{\pgfqpoint{1.501288in}{1.843978in}}%
\pgfpathlineto{\pgfqpoint{1.569265in}{1.953139in}}%
\pgfpathlineto{\pgfqpoint{1.637241in}{2.058048in}}%
\pgfpathlineto{\pgfqpoint{1.705217in}{2.158704in}}%
\pgfpathlineto{\pgfqpoint{1.773193in}{2.255107in}}%
\pgfpathlineto{\pgfqpoint{1.841169in}{2.347256in}}%
\pgfpathlineto{\pgfqpoint{1.909146in}{2.435153in}}%
\pgfpathlineto{\pgfqpoint{1.977122in}{2.518796in}}%
\pgfpathlineto{\pgfqpoint{2.045098in}{2.598186in}}%
\pgfpathlineto{\pgfqpoint{2.113074in}{2.673324in}}%
\pgfpathlineto{\pgfqpoint{2.181051in}{2.744208in}}%
\pgfpathlineto{\pgfqpoint{2.249027in}{2.810839in}}%
\pgfpathlineto{\pgfqpoint{2.317003in}{2.873218in}}%
\pgfpathlineto{\pgfqpoint{2.384979in}{2.931343in}}%
\pgfpathlineto{\pgfqpoint{2.452956in}{2.985215in}}%
\pgfpathlineto{\pgfqpoint{2.498273in}{3.018767in}}%
\pgfpathlineto{\pgfqpoint{2.543591in}{3.050428in}}%
\pgfpathlineto{\pgfqpoint{2.588908in}{3.080200in}}%
\pgfpathlineto{\pgfqpoint{2.634226in}{3.108081in}}%
\pgfpathlineto{\pgfqpoint{2.679543in}{3.134072in}}%
\pgfpathlineto{\pgfqpoint{2.724861in}{3.158172in}}%
\pgfpathlineto{\pgfqpoint{2.770178in}{3.180383in}}%
\pgfpathlineto{\pgfqpoint{2.815496in}{3.200703in}}%
\pgfpathlineto{\pgfqpoint{2.860813in}{3.219133in}}%
\pgfpathlineto{\pgfqpoint{2.906131in}{3.235673in}}%
\pgfpathlineto{\pgfqpoint{2.951448in}{3.250322in}}%
\pgfpathlineto{\pgfqpoint{2.996766in}{3.263081in}}%
\pgfpathlineto{\pgfqpoint{3.042083in}{3.273950in}}%
\pgfpathlineto{\pgfqpoint{3.087401in}{3.282929in}}%
\pgfpathlineto{\pgfqpoint{3.132718in}{3.290017in}}%
\pgfpathlineto{\pgfqpoint{3.178036in}{3.295215in}}%
\pgfpathlineto{\pgfqpoint{3.223353in}{3.298523in}}%
\pgfpathlineto{\pgfqpoint{3.268671in}{3.299941in}}%
\pgfpathlineto{\pgfqpoint{3.313988in}{3.299468in}}%
\pgfpathlineto{\pgfqpoint{3.359306in}{3.297106in}}%
\pgfpathlineto{\pgfqpoint{3.404623in}{3.292853in}}%
\pgfpathlineto{\pgfqpoint{3.449941in}{3.286709in}}%
\pgfpathlineto{\pgfqpoint{3.495258in}{3.278676in}}%
\pgfpathlineto{\pgfqpoint{3.540576in}{3.268752in}}%
\pgfpathlineto{\pgfqpoint{3.585893in}{3.256938in}}%
\pgfpathlineto{\pgfqpoint{3.631211in}{3.243234in}}%
\pgfpathlineto{\pgfqpoint{3.676528in}{3.227639in}}%
\pgfpathlineto{\pgfqpoint{3.721846in}{3.210154in}}%
\pgfpathlineto{\pgfqpoint{3.767163in}{3.190779in}}%
\pgfpathlineto{\pgfqpoint{3.812481in}{3.169514in}}%
\pgfpathlineto{\pgfqpoint{3.857798in}{3.146358in}}%
\pgfpathlineto{\pgfqpoint{3.903116in}{3.121313in}}%
\pgfpathlineto{\pgfqpoint{3.948433in}{3.094377in}}%
\pgfpathlineto{\pgfqpoint{3.993751in}{3.065550in}}%
\pgfpathlineto{\pgfqpoint{4.039068in}{3.034834in}}%
\pgfpathlineto{\pgfqpoint{4.084386in}{3.002227in}}%
\pgfpathlineto{\pgfqpoint{4.129703in}{2.967730in}}%
\pgfpathlineto{\pgfqpoint{4.197679in}{2.912440in}}%
\pgfpathlineto{\pgfqpoint{4.265656in}{2.852897in}}%
\pgfpathlineto{\pgfqpoint{4.333632in}{2.789102in}}%
\pgfpathlineto{\pgfqpoint{4.401608in}{2.721053in}}%
\pgfpathlineto{\pgfqpoint{4.469584in}{2.648751in}}%
\pgfpathlineto{\pgfqpoint{4.537561in}{2.572196in}}%
\pgfpathlineto{\pgfqpoint{4.605537in}{2.491388in}}%
\pgfpathlineto{\pgfqpoint{4.673513in}{2.406326in}}%
\pgfpathlineto{\pgfqpoint{4.741489in}{2.317012in}}%
\pgfpathlineto{\pgfqpoint{4.809466in}{2.223445in}}%
\pgfpathlineto{\pgfqpoint{4.877442in}{2.125625in}}%
\pgfpathlineto{\pgfqpoint{4.945418in}{2.023551in}}%
\pgfpathlineto{\pgfqpoint{5.013394in}{1.917225in}}%
\pgfpathlineto{\pgfqpoint{5.081370in}{1.806645in}}%
\pgfpathlineto{\pgfqpoint{5.149347in}{1.691813in}}%
\pgfpathlineto{\pgfqpoint{5.217323in}{1.572727in}}%
\pgfpathlineto{\pgfqpoint{5.285299in}{1.449389in}}%
\pgfpathlineto{\pgfqpoint{5.353275in}{1.321797in}}%
\pgfpathlineto{\pgfqpoint{5.421252in}{1.189952in}}%
\pgfpathlineto{\pgfqpoint{5.489228in}{1.053854in}}%
\pgfpathlineto{\pgfqpoint{5.534545in}{0.960759in}}%
\pgfpathlineto{\pgfqpoint{5.534545in}{0.960759in}}%
\pgfusepath{stroke}%
\end{pgfscope}%
\begin{pgfscope}%
\pgfpathrectangle{\pgfqpoint{0.800000in}{0.528000in}}{\pgfqpoint{4.960000in}{3.696000in}}%
\pgfusepath{clip}%
\pgfsetbuttcap%
\pgfsetroundjoin%
\definecolor{currentfill}{rgb}{0.737255,0.741176,0.133333}%
\pgfsetfillcolor{currentfill}%
\pgfsetlinewidth{1.003750pt}%
\definecolor{currentstroke}{rgb}{0.737255,0.741176,0.133333}%
\pgfsetstrokecolor{currentstroke}%
\pgfsetdash{}{0pt}%
\pgfsys@defobject{currentmarker}{\pgfqpoint{-0.020833in}{-0.020833in}}{\pgfqpoint{0.020833in}{0.020833in}}{%
\pgfpathmoveto{\pgfqpoint{0.000000in}{-0.020833in}}%
\pgfpathcurveto{\pgfqpoint{0.005525in}{-0.020833in}}{\pgfqpoint{0.010825in}{-0.018638in}}{\pgfqpoint{0.014731in}{-0.014731in}}%
\pgfpathcurveto{\pgfqpoint{0.018638in}{-0.010825in}}{\pgfqpoint{0.020833in}{-0.005525in}}{\pgfqpoint{0.020833in}{0.000000in}}%
\pgfpathcurveto{\pgfqpoint{0.020833in}{0.005525in}}{\pgfqpoint{0.018638in}{0.010825in}}{\pgfqpoint{0.014731in}{0.014731in}}%
\pgfpathcurveto{\pgfqpoint{0.010825in}{0.018638in}}{\pgfqpoint{0.005525in}{0.020833in}}{\pgfqpoint{0.000000in}{0.020833in}}%
\pgfpathcurveto{\pgfqpoint{-0.005525in}{0.020833in}}{\pgfqpoint{-0.010825in}{0.018638in}}{\pgfqpoint{-0.014731in}{0.014731in}}%
\pgfpathcurveto{\pgfqpoint{-0.018638in}{0.010825in}}{\pgfqpoint{-0.020833in}{0.005525in}}{\pgfqpoint{-0.020833in}{0.000000in}}%
\pgfpathcurveto{\pgfqpoint{-0.020833in}{-0.005525in}}{\pgfqpoint{-0.018638in}{-0.010825in}}{\pgfqpoint{-0.014731in}{-0.014731in}}%
\pgfpathcurveto{\pgfqpoint{-0.010825in}{-0.018638in}}{\pgfqpoint{-0.005525in}{-0.020833in}}{\pgfqpoint{0.000000in}{-0.020833in}}%
\pgfpathclose%
\pgfusepath{stroke,fill}%
}%
\begin{pgfscope}%
\pgfsys@transformshift{5.424200in}{1.530263in}%
\pgfsys@useobject{currentmarker}{}%
\end{pgfscope}%
\begin{pgfscope}%
\pgfsys@transformshift{4.605189in}{1.643755in}%
\pgfsys@useobject{currentmarker}{}%
\end{pgfscope}%
\begin{pgfscope}%
\pgfsys@transformshift{3.280000in}{3.300000in}%
\pgfsys@useobject{currentmarker}{}%
\end{pgfscope}%
\begin{pgfscope}%
\pgfsys@transformshift{1.954811in}{1.643755in}%
\pgfsys@useobject{currentmarker}{}%
\end{pgfscope}%
\begin{pgfscope}%
\pgfsys@transformshift{1.135800in}{1.530263in}%
\pgfsys@useobject{currentmarker}{}%
\end{pgfscope}%
\end{pgfscope}%
\begin{pgfscope}%
\pgfpathrectangle{\pgfqpoint{0.800000in}{0.528000in}}{\pgfqpoint{4.960000in}{3.696000in}}%
\pgfusepath{clip}%
\pgfsetrectcap%
\pgfsetroundjoin%
\pgfsetlinewidth{1.505625pt}%
\definecolor{currentstroke}{rgb}{0.090196,0.745098,0.811765}%
\pgfsetstrokecolor{currentstroke}%
\pgfsetdash{}{0pt}%
\pgfpathmoveto{\pgfqpoint{1.025455in}{1.828097in}}%
\pgfpathlineto{\pgfqpoint{1.048113in}{1.758056in}}%
\pgfpathlineto{\pgfqpoint{1.070772in}{1.692722in}}%
\pgfpathlineto{\pgfqpoint{1.093431in}{1.631972in}}%
\pgfpathlineto{\pgfqpoint{1.116090in}{1.575687in}}%
\pgfpathlineto{\pgfqpoint{1.138748in}{1.523746in}}%
\pgfpathlineto{\pgfqpoint{1.161407in}{1.476032in}}%
\pgfpathlineto{\pgfqpoint{1.184066in}{1.432427in}}%
\pgfpathlineto{\pgfqpoint{1.206725in}{1.392817in}}%
\pgfpathlineto{\pgfqpoint{1.229383in}{1.357087in}}%
\pgfpathlineto{\pgfqpoint{1.252042in}{1.325124in}}%
\pgfpathlineto{\pgfqpoint{1.274701in}{1.296816in}}%
\pgfpathlineto{\pgfqpoint{1.297360in}{1.272052in}}%
\pgfpathlineto{\pgfqpoint{1.320018in}{1.250723in}}%
\pgfpathlineto{\pgfqpoint{1.342677in}{1.232721in}}%
\pgfpathlineto{\pgfqpoint{1.365336in}{1.217939in}}%
\pgfpathlineto{\pgfqpoint{1.387995in}{1.206271in}}%
\pgfpathlineto{\pgfqpoint{1.410653in}{1.197613in}}%
\pgfpathlineto{\pgfqpoint{1.433312in}{1.191862in}}%
\pgfpathlineto{\pgfqpoint{1.455971in}{1.188916in}}%
\pgfpathlineto{\pgfqpoint{1.478630in}{1.188675in}}%
\pgfpathlineto{\pgfqpoint{1.501288in}{1.191038in}}%
\pgfpathlineto{\pgfqpoint{1.523947in}{1.195908in}}%
\pgfpathlineto{\pgfqpoint{1.546606in}{1.203187in}}%
\pgfpathlineto{\pgfqpoint{1.569265in}{1.212780in}}%
\pgfpathlineto{\pgfqpoint{1.591923in}{1.224593in}}%
\pgfpathlineto{\pgfqpoint{1.614582in}{1.238533in}}%
\pgfpathlineto{\pgfqpoint{1.637241in}{1.254506in}}%
\pgfpathlineto{\pgfqpoint{1.659899in}{1.272424in}}%
\pgfpathlineto{\pgfqpoint{1.682558in}{1.292195in}}%
\pgfpathlineto{\pgfqpoint{1.705217in}{1.313733in}}%
\pgfpathlineto{\pgfqpoint{1.727876in}{1.336949in}}%
\pgfpathlineto{\pgfqpoint{1.773193in}{1.388077in}}%
\pgfpathlineto{\pgfqpoint{1.818511in}{1.444907in}}%
\pgfpathlineto{\pgfqpoint{1.863828in}{1.506789in}}%
\pgfpathlineto{\pgfqpoint{1.909146in}{1.573092in}}%
\pgfpathlineto{\pgfqpoint{1.954463in}{1.643203in}}%
\pgfpathlineto{\pgfqpoint{1.999781in}{1.716532in}}%
\pgfpathlineto{\pgfqpoint{2.067757in}{1.831312in}}%
\pgfpathlineto{\pgfqpoint{2.158392in}{1.990444in}}%
\pgfpathlineto{\pgfqpoint{2.407638in}{2.433500in}}%
\pgfpathlineto{\pgfqpoint{2.475614in}{2.548790in}}%
\pgfpathlineto{\pgfqpoint{2.543591in}{2.659232in}}%
\pgfpathlineto{\pgfqpoint{2.588908in}{2.729600in}}%
\pgfpathlineto{\pgfqpoint{2.634226in}{2.796991in}}%
\pgfpathlineto{\pgfqpoint{2.679543in}{2.861111in}}%
\pgfpathlineto{\pgfqpoint{2.724861in}{2.921688in}}%
\pgfpathlineto{\pgfqpoint{2.770178in}{2.978468in}}%
\pgfpathlineto{\pgfqpoint{2.815496in}{3.031217in}}%
\pgfpathlineto{\pgfqpoint{2.860813in}{3.079722in}}%
\pgfpathlineto{\pgfqpoint{2.906131in}{3.123788in}}%
\pgfpathlineto{\pgfqpoint{2.951448in}{3.163242in}}%
\pgfpathlineto{\pgfqpoint{2.996766in}{3.197930in}}%
\pgfpathlineto{\pgfqpoint{3.042083in}{3.227717in}}%
\pgfpathlineto{\pgfqpoint{3.087401in}{3.252488in}}%
\pgfpathlineto{\pgfqpoint{3.110059in}{3.262963in}}%
\pgfpathlineto{\pgfqpoint{3.132718in}{3.272151in}}%
\pgfpathlineto{\pgfqpoint{3.155377in}{3.280042in}}%
\pgfpathlineto{\pgfqpoint{3.178036in}{3.286629in}}%
\pgfpathlineto{\pgfqpoint{3.200694in}{3.291906in}}%
\pgfpathlineto{\pgfqpoint{3.223353in}{3.295869in}}%
\pgfpathlineto{\pgfqpoint{3.246012in}{3.298512in}}%
\pgfpathlineto{\pgfqpoint{3.268671in}{3.299835in}}%
\pgfpathlineto{\pgfqpoint{3.291329in}{3.299835in}}%
\pgfpathlineto{\pgfqpoint{3.313988in}{3.298512in}}%
\pgfpathlineto{\pgfqpoint{3.336647in}{3.295869in}}%
\pgfpathlineto{\pgfqpoint{3.359306in}{3.291906in}}%
\pgfpathlineto{\pgfqpoint{3.381964in}{3.286629in}}%
\pgfpathlineto{\pgfqpoint{3.404623in}{3.280042in}}%
\pgfpathlineto{\pgfqpoint{3.427282in}{3.272151in}}%
\pgfpathlineto{\pgfqpoint{3.449941in}{3.262963in}}%
\pgfpathlineto{\pgfqpoint{3.472599in}{3.252488in}}%
\pgfpathlineto{\pgfqpoint{3.495258in}{3.240736in}}%
\pgfpathlineto{\pgfqpoint{3.540576in}{3.213443in}}%
\pgfpathlineto{\pgfqpoint{3.585893in}{3.181191in}}%
\pgfpathlineto{\pgfqpoint{3.631211in}{3.144102in}}%
\pgfpathlineto{\pgfqpoint{3.676528in}{3.102321in}}%
\pgfpathlineto{\pgfqpoint{3.721846in}{3.056013in}}%
\pgfpathlineto{\pgfqpoint{3.767163in}{3.005360in}}%
\pgfpathlineto{\pgfqpoint{3.812481in}{2.950568in}}%
\pgfpathlineto{\pgfqpoint{3.857798in}{2.891859in}}%
\pgfpathlineto{\pgfqpoint{3.903116in}{2.829477in}}%
\pgfpathlineto{\pgfqpoint{3.948433in}{2.763686in}}%
\pgfpathlineto{\pgfqpoint{3.993751in}{2.694769in}}%
\pgfpathlineto{\pgfqpoint{4.061727in}{2.586201in}}%
\pgfpathlineto{\pgfqpoint{4.129703in}{2.472394in}}%
\pgfpathlineto{\pgfqpoint{4.197679in}{2.354556in}}%
\pgfpathlineto{\pgfqpoint{4.310973in}{2.152800in}}%
\pgfpathlineto{\pgfqpoint{4.446926in}{1.910226in}}%
\pgfpathlineto{\pgfqpoint{4.514902in}{1.792506in}}%
\pgfpathlineto{\pgfqpoint{4.582878in}{1.679502in}}%
\pgfpathlineto{\pgfqpoint{4.628196in}{1.607709in}}%
\pgfpathlineto{\pgfqpoint{4.673513in}{1.539427in}}%
\pgfpathlineto{\pgfqpoint{4.718831in}{1.475257in}}%
\pgfpathlineto{\pgfqpoint{4.764148in}{1.415820in}}%
\pgfpathlineto{\pgfqpoint{4.809466in}{1.361758in}}%
\pgfpathlineto{\pgfqpoint{4.854783in}{1.313733in}}%
\pgfpathlineto{\pgfqpoint{4.877442in}{1.292195in}}%
\pgfpathlineto{\pgfqpoint{4.900101in}{1.272424in}}%
\pgfpathlineto{\pgfqpoint{4.922759in}{1.254506in}}%
\pgfpathlineto{\pgfqpoint{4.945418in}{1.238533in}}%
\pgfpathlineto{\pgfqpoint{4.968077in}{1.224593in}}%
\pgfpathlineto{\pgfqpoint{4.990735in}{1.212780in}}%
\pgfpathlineto{\pgfqpoint{5.013394in}{1.203187in}}%
\pgfpathlineto{\pgfqpoint{5.036053in}{1.195908in}}%
\pgfpathlineto{\pgfqpoint{5.058712in}{1.191038in}}%
\pgfpathlineto{\pgfqpoint{5.081370in}{1.188675in}}%
\pgfpathlineto{\pgfqpoint{5.104029in}{1.188916in}}%
\pgfpathlineto{\pgfqpoint{5.126688in}{1.191862in}}%
\pgfpathlineto{\pgfqpoint{5.149347in}{1.197613in}}%
\pgfpathlineto{\pgfqpoint{5.172005in}{1.206271in}}%
\pgfpathlineto{\pgfqpoint{5.194664in}{1.217939in}}%
\pgfpathlineto{\pgfqpoint{5.217323in}{1.232721in}}%
\pgfpathlineto{\pgfqpoint{5.239982in}{1.250723in}}%
\pgfpathlineto{\pgfqpoint{5.262640in}{1.272052in}}%
\pgfpathlineto{\pgfqpoint{5.285299in}{1.296816in}}%
\pgfpathlineto{\pgfqpoint{5.307958in}{1.325124in}}%
\pgfpathlineto{\pgfqpoint{5.330617in}{1.357087in}}%
\pgfpathlineto{\pgfqpoint{5.353275in}{1.392817in}}%
\pgfpathlineto{\pgfqpoint{5.375934in}{1.432427in}}%
\pgfpathlineto{\pgfqpoint{5.398593in}{1.476032in}}%
\pgfpathlineto{\pgfqpoint{5.421252in}{1.523746in}}%
\pgfpathlineto{\pgfqpoint{5.443910in}{1.575687in}}%
\pgfpathlineto{\pgfqpoint{5.466569in}{1.631972in}}%
\pgfpathlineto{\pgfqpoint{5.489228in}{1.692722in}}%
\pgfpathlineto{\pgfqpoint{5.511887in}{1.758056in}}%
\pgfpathlineto{\pgfqpoint{5.534545in}{1.828097in}}%
\pgfpathlineto{\pgfqpoint{5.534545in}{1.828097in}}%
\pgfusepath{stroke}%
\end{pgfscope}%
\begin{pgfscope}%
\pgfsetrectcap%
\pgfsetmiterjoin%
\pgfsetlinewidth{0.803000pt}%
\definecolor{currentstroke}{rgb}{0.000000,0.000000,0.000000}%
\pgfsetstrokecolor{currentstroke}%
\pgfsetdash{}{0pt}%
\pgfpathmoveto{\pgfqpoint{0.800000in}{0.528000in}}%
\pgfpathlineto{\pgfqpoint{0.800000in}{4.224000in}}%
\pgfusepath{stroke}%
\end{pgfscope}%
\begin{pgfscope}%
\pgfsetrectcap%
\pgfsetmiterjoin%
\pgfsetlinewidth{0.803000pt}%
\definecolor{currentstroke}{rgb}{0.000000,0.000000,0.000000}%
\pgfsetstrokecolor{currentstroke}%
\pgfsetdash{}{0pt}%
\pgfpathmoveto{\pgfqpoint{5.760000in}{0.528000in}}%
\pgfpathlineto{\pgfqpoint{5.760000in}{4.224000in}}%
\pgfusepath{stroke}%
\end{pgfscope}%
\begin{pgfscope}%
\pgfsetrectcap%
\pgfsetmiterjoin%
\pgfsetlinewidth{0.803000pt}%
\definecolor{currentstroke}{rgb}{0.000000,0.000000,0.000000}%
\pgfsetstrokecolor{currentstroke}%
\pgfsetdash{}{0pt}%
\pgfpathmoveto{\pgfqpoint{0.800000in}{0.528000in}}%
\pgfpathlineto{\pgfqpoint{5.760000in}{0.528000in}}%
\pgfusepath{stroke}%
\end{pgfscope}%
\begin{pgfscope}%
\pgfsetrectcap%
\pgfsetmiterjoin%
\pgfsetlinewidth{0.803000pt}%
\definecolor{currentstroke}{rgb}{0.000000,0.000000,0.000000}%
\pgfsetstrokecolor{currentstroke}%
\pgfsetdash{}{0pt}%
\pgfpathmoveto{\pgfqpoint{0.800000in}{4.224000in}}%
\pgfpathlineto{\pgfqpoint{5.760000in}{4.224000in}}%
\pgfusepath{stroke}%
\end{pgfscope}%
\begin{pgfscope}%
\definecolor{textcolor}{rgb}{0.000000,0.000000,0.000000}%
\pgfsetstrokecolor{textcolor}%
\pgfsetfillcolor{textcolor}%
\pgftext[x=3.280000in,y=4.307333in,,base]{\color{textcolor}\rmfamily\fontsize{12.000000}{14.400000}\selectfont Wielomiany Lagrange'a dla N=2, 4}%
\end{pgfscope}%
\begin{pgfscope}%
\pgfsetbuttcap%
\pgfsetmiterjoin%
\definecolor{currentfill}{rgb}{1.000000,1.000000,1.000000}%
\pgfsetfillcolor{currentfill}%
\pgfsetfillopacity{0.800000}%
\pgfsetlinewidth{1.003750pt}%
\definecolor{currentstroke}{rgb}{0.800000,0.800000,0.800000}%
\pgfsetstrokecolor{currentstroke}%
\pgfsetstrokeopacity{0.800000}%
\pgfsetdash{}{0pt}%
\pgfpathmoveto{\pgfqpoint{4.258142in}{2.560046in}}%
\pgfpathlineto{\pgfqpoint{5.662778in}{2.560046in}}%
\pgfpathquadraticcurveto{\pgfqpoint{5.690556in}{2.560046in}}{\pgfqpoint{5.690556in}{2.587823in}}%
\pgfpathlineto{\pgfqpoint{5.690556in}{4.126778in}}%
\pgfpathquadraticcurveto{\pgfqpoint{5.690556in}{4.154556in}}{\pgfqpoint{5.662778in}{4.154556in}}%
\pgfpathlineto{\pgfqpoint{4.258142in}{4.154556in}}%
\pgfpathquadraticcurveto{\pgfqpoint{4.230364in}{4.154556in}}{\pgfqpoint{4.230364in}{4.126778in}}%
\pgfpathlineto{\pgfqpoint{4.230364in}{2.587823in}}%
\pgfpathquadraticcurveto{\pgfqpoint{4.230364in}{2.560046in}}{\pgfqpoint{4.258142in}{2.560046in}}%
\pgfpathclose%
\pgfusepath{stroke,fill}%
\end{pgfscope}%
\begin{pgfscope}%
\pgfsetrectcap%
\pgfsetroundjoin%
\pgfsetlinewidth{1.505625pt}%
\definecolor{currentstroke}{rgb}{0.121569,0.466667,0.705882}%
\pgfsetstrokecolor{currentstroke}%
\pgfsetdash{}{0pt}%
\pgfpathmoveto{\pgfqpoint{4.285920in}{3.964146in}}%
\pgfpathlineto{\pgfqpoint{4.563698in}{3.964146in}}%
\pgfusepath{stroke}%
\end{pgfscope}%
\begin{pgfscope}%
\definecolor{textcolor}{rgb}{0.000000,0.000000,0.000000}%
\pgfsetstrokecolor{textcolor}%
\pgfsetfillcolor{textcolor}%
\pgftext[x=4.674809in,y=3.915534in,left,base]{\color{textcolor}\rmfamily\fontsize{10.000000}{12.000000}\selectfont \(\displaystyle y(x)=\)\(\displaystyle \frac{1}{1+25x^{2}}\)}%
\end{pgfscope}%
\begin{pgfscope}%
\pgfsetrectcap%
\pgfsetroundjoin%
\pgfsetlinewidth{1.505625pt}%
\definecolor{currentstroke}{rgb}{0.172549,0.627451,0.172549}%
\pgfsetstrokecolor{currentstroke}%
\pgfsetdash{}{0pt}%
\pgfpathmoveto{\pgfqpoint{4.285920in}{3.683689in}}%
\pgfpathlineto{\pgfqpoint{4.563698in}{3.683689in}}%
\pgfusepath{stroke}%
\end{pgfscope}%
\begin{pgfscope}%
\definecolor{textcolor}{rgb}{0.000000,0.000000,0.000000}%
\pgfsetstrokecolor{textcolor}%
\pgfsetfillcolor{textcolor}%
\pgftext[x=4.674809in,y=3.635078in,left,base]{\color{textcolor}\rmfamily\fontsize{10.000000}{12.000000}\selectfont W2(x)}%
\end{pgfscope}%
\begin{pgfscope}%
\pgfsetrectcap%
\pgfsetroundjoin%
\pgfsetlinewidth{1.505625pt}%
\definecolor{currentstroke}{rgb}{0.580392,0.403922,0.741176}%
\pgfsetstrokecolor{currentstroke}%
\pgfsetdash{}{0pt}%
\pgfpathmoveto{\pgfqpoint{4.285920in}{3.475356in}}%
\pgfpathlineto{\pgfqpoint{4.563698in}{3.475356in}}%
\pgfusepath{stroke}%
\end{pgfscope}%
\begin{pgfscope}%
\definecolor{textcolor}{rgb}{0.000000,0.000000,0.000000}%
\pgfsetstrokecolor{textcolor}%
\pgfsetfillcolor{textcolor}%
\pgftext[x=4.674809in,y=3.426745in,left,base]{\color{textcolor}\rmfamily\fontsize{10.000000}{12.000000}\selectfont W4(x)}%
\end{pgfscope}%
\begin{pgfscope}%
\pgfsetrectcap%
\pgfsetroundjoin%
\pgfsetlinewidth{1.505625pt}%
\definecolor{currentstroke}{rgb}{0.549020,0.337255,0.294118}%
\pgfsetstrokecolor{currentstroke}%
\pgfsetdash{}{0pt}%
\pgfpathmoveto{\pgfqpoint{4.285920in}{3.187724in}}%
\pgfpathlineto{\pgfqpoint{4.563698in}{3.187724in}}%
\pgfusepath{stroke}%
\end{pgfscope}%
\begin{pgfscope}%
\definecolor{textcolor}{rgb}{0.000000,0.000000,0.000000}%
\pgfsetstrokecolor{textcolor}%
\pgfsetfillcolor{textcolor}%
\pgftext[x=4.674809in,y=3.139113in,left,base]{\color{textcolor}\rmfamily\fontsize{10.000000}{12.000000}\selectfont \(\displaystyle y(x)=\)\(\displaystyle \frac{1}{1+25x^{2}}\)}%
\end{pgfscope}%
\begin{pgfscope}%
\pgfsetrectcap%
\pgfsetroundjoin%
\pgfsetlinewidth{1.505625pt}%
\definecolor{currentstroke}{rgb}{0.498039,0.498039,0.498039}%
\pgfsetstrokecolor{currentstroke}%
\pgfsetdash{}{0pt}%
\pgfpathmoveto{\pgfqpoint{4.285920in}{2.907268in}}%
\pgfpathlineto{\pgfqpoint{4.563698in}{2.907268in}}%
\pgfusepath{stroke}%
\end{pgfscope}%
\begin{pgfscope}%
\definecolor{textcolor}{rgb}{0.000000,0.000000,0.000000}%
\pgfsetstrokecolor{textcolor}%
\pgfsetfillcolor{textcolor}%
\pgftext[x=4.674809in,y=2.858657in,left,base]{\color{textcolor}\rmfamily\fontsize{10.000000}{12.000000}\selectfont W2(x)}%
\end{pgfscope}%
\begin{pgfscope}%
\pgfsetrectcap%
\pgfsetroundjoin%
\pgfsetlinewidth{1.505625pt}%
\definecolor{currentstroke}{rgb}{0.090196,0.745098,0.811765}%
\pgfsetstrokecolor{currentstroke}%
\pgfsetdash{}{0pt}%
\pgfpathmoveto{\pgfqpoint{4.285920in}{2.698934in}}%
\pgfpathlineto{\pgfqpoint{4.563698in}{2.698934in}}%
\pgfusepath{stroke}%
\end{pgfscope}%
\begin{pgfscope}%
\definecolor{textcolor}{rgb}{0.000000,0.000000,0.000000}%
\pgfsetstrokecolor{textcolor}%
\pgfsetfillcolor{textcolor}%
\pgftext[x=4.674809in,y=2.650323in,left,base]{\color{textcolor}\rmfamily\fontsize{10.000000}{12.000000}\selectfont W4(x)}%
\end{pgfscope}%
\end{pgfpicture}%
\makeatother%
\endgroup%
        
    \end{center}
    \caption{Węzły cosinus, funkcja \(y\), \(N=2,4\)}
\end{figure}

\begin{figure}[h]
    \begin{center}
        %% Creator: Matplotlib, PGF backend
%%
%% To include the figure in your LaTeX document, write
%%   \input{<filename>.pgf}
%%
%% Make sure the required packages are loaded in your preamble
%%   \usepackage{pgf}
%%
%% Figures using additional raster images can only be included by \input if
%% they are in the same directory as the main LaTeX file. For loading figures
%% from other directories you can use the `import` package
%%   \usepackage{import}
%% and then include the figures with
%%   \import{<path to file>}{<filename>.pgf}
%%
%% Matplotlib used the following preamble
%%
\begingroup%
\makeatletter%
\begin{pgfpicture}%
\pgfpathrectangle{\pgfpointorigin}{\pgfqpoint{5.500000in}{3.500000in}}%
\pgfusepath{use as bounding box, clip}%
\begin{pgfscope}%
\pgfsetbuttcap%
\pgfsetmiterjoin%
\definecolor{currentfill}{rgb}{1.000000,1.000000,1.000000}%
\pgfsetfillcolor{currentfill}%
\pgfsetlinewidth{0.000000pt}%
\definecolor{currentstroke}{rgb}{1.000000,1.000000,1.000000}%
\pgfsetstrokecolor{currentstroke}%
\pgfsetdash{}{0pt}%
\pgfpathmoveto{\pgfqpoint{0.000000in}{0.000000in}}%
\pgfpathlineto{\pgfqpoint{5.500000in}{0.000000in}}%
\pgfpathlineto{\pgfqpoint{5.500000in}{3.500000in}}%
\pgfpathlineto{\pgfqpoint{0.000000in}{3.500000in}}%
\pgfpathclose%
\pgfusepath{fill}%
\end{pgfscope}%
\begin{pgfscope}%
\pgfsetbuttcap%
\pgfsetmiterjoin%
\definecolor{currentfill}{rgb}{1.000000,1.000000,1.000000}%
\pgfsetfillcolor{currentfill}%
\pgfsetlinewidth{0.000000pt}%
\definecolor{currentstroke}{rgb}{0.000000,0.000000,0.000000}%
\pgfsetstrokecolor{currentstroke}%
\pgfsetstrokeopacity{0.000000}%
\pgfsetdash{}{0pt}%
\pgfpathmoveto{\pgfqpoint{0.687500in}{0.385000in}}%
\pgfpathlineto{\pgfqpoint{4.950000in}{0.385000in}}%
\pgfpathlineto{\pgfqpoint{4.950000in}{3.080000in}}%
\pgfpathlineto{\pgfqpoint{0.687500in}{3.080000in}}%
\pgfpathclose%
\pgfusepath{fill}%
\end{pgfscope}%
\begin{pgfscope}%
\pgfsetbuttcap%
\pgfsetroundjoin%
\definecolor{currentfill}{rgb}{0.000000,0.000000,0.000000}%
\pgfsetfillcolor{currentfill}%
\pgfsetlinewidth{0.803000pt}%
\definecolor{currentstroke}{rgb}{0.000000,0.000000,0.000000}%
\pgfsetstrokecolor{currentstroke}%
\pgfsetdash{}{0pt}%
\pgfsys@defobject{currentmarker}{\pgfqpoint{0.000000in}{-0.048611in}}{\pgfqpoint{0.000000in}{0.000000in}}{%
\pgfpathmoveto{\pgfqpoint{0.000000in}{0.000000in}}%
\pgfpathlineto{\pgfqpoint{0.000000in}{-0.048611in}}%
\pgfusepath{stroke,fill}%
}%
\begin{pgfscope}%
\pgfsys@transformshift{0.881250in}{0.385000in}%
\pgfsys@useobject{currentmarker}{}%
\end{pgfscope}%
\end{pgfscope}%
\begin{pgfscope}%
\definecolor{textcolor}{rgb}{0.000000,0.000000,0.000000}%
\pgfsetstrokecolor{textcolor}%
\pgfsetfillcolor{textcolor}%
\pgftext[x=0.881250in,y=0.287778in,,top]{\color{textcolor}\rmfamily\fontsize{10.000000}{12.000000}\selectfont \(\displaystyle -1.00\)}%
\end{pgfscope}%
\begin{pgfscope}%
\pgfsetbuttcap%
\pgfsetroundjoin%
\definecolor{currentfill}{rgb}{0.000000,0.000000,0.000000}%
\pgfsetfillcolor{currentfill}%
\pgfsetlinewidth{0.803000pt}%
\definecolor{currentstroke}{rgb}{0.000000,0.000000,0.000000}%
\pgfsetstrokecolor{currentstroke}%
\pgfsetdash{}{0pt}%
\pgfsys@defobject{currentmarker}{\pgfqpoint{0.000000in}{-0.048611in}}{\pgfqpoint{0.000000in}{0.000000in}}{%
\pgfpathmoveto{\pgfqpoint{0.000000in}{0.000000in}}%
\pgfpathlineto{\pgfqpoint{0.000000in}{-0.048611in}}%
\pgfusepath{stroke,fill}%
}%
\begin{pgfscope}%
\pgfsys@transformshift{1.365625in}{0.385000in}%
\pgfsys@useobject{currentmarker}{}%
\end{pgfscope}%
\end{pgfscope}%
\begin{pgfscope}%
\definecolor{textcolor}{rgb}{0.000000,0.000000,0.000000}%
\pgfsetstrokecolor{textcolor}%
\pgfsetfillcolor{textcolor}%
\pgftext[x=1.365625in,y=0.287778in,,top]{\color{textcolor}\rmfamily\fontsize{10.000000}{12.000000}\selectfont \(\displaystyle -0.75\)}%
\end{pgfscope}%
\begin{pgfscope}%
\pgfsetbuttcap%
\pgfsetroundjoin%
\definecolor{currentfill}{rgb}{0.000000,0.000000,0.000000}%
\pgfsetfillcolor{currentfill}%
\pgfsetlinewidth{0.803000pt}%
\definecolor{currentstroke}{rgb}{0.000000,0.000000,0.000000}%
\pgfsetstrokecolor{currentstroke}%
\pgfsetdash{}{0pt}%
\pgfsys@defobject{currentmarker}{\pgfqpoint{0.000000in}{-0.048611in}}{\pgfqpoint{0.000000in}{0.000000in}}{%
\pgfpathmoveto{\pgfqpoint{0.000000in}{0.000000in}}%
\pgfpathlineto{\pgfqpoint{0.000000in}{-0.048611in}}%
\pgfusepath{stroke,fill}%
}%
\begin{pgfscope}%
\pgfsys@transformshift{1.850000in}{0.385000in}%
\pgfsys@useobject{currentmarker}{}%
\end{pgfscope}%
\end{pgfscope}%
\begin{pgfscope}%
\definecolor{textcolor}{rgb}{0.000000,0.000000,0.000000}%
\pgfsetstrokecolor{textcolor}%
\pgfsetfillcolor{textcolor}%
\pgftext[x=1.850000in,y=0.287778in,,top]{\color{textcolor}\rmfamily\fontsize{10.000000}{12.000000}\selectfont \(\displaystyle -0.50\)}%
\end{pgfscope}%
\begin{pgfscope}%
\pgfsetbuttcap%
\pgfsetroundjoin%
\definecolor{currentfill}{rgb}{0.000000,0.000000,0.000000}%
\pgfsetfillcolor{currentfill}%
\pgfsetlinewidth{0.803000pt}%
\definecolor{currentstroke}{rgb}{0.000000,0.000000,0.000000}%
\pgfsetstrokecolor{currentstroke}%
\pgfsetdash{}{0pt}%
\pgfsys@defobject{currentmarker}{\pgfqpoint{0.000000in}{-0.048611in}}{\pgfqpoint{0.000000in}{0.000000in}}{%
\pgfpathmoveto{\pgfqpoint{0.000000in}{0.000000in}}%
\pgfpathlineto{\pgfqpoint{0.000000in}{-0.048611in}}%
\pgfusepath{stroke,fill}%
}%
\begin{pgfscope}%
\pgfsys@transformshift{2.334375in}{0.385000in}%
\pgfsys@useobject{currentmarker}{}%
\end{pgfscope}%
\end{pgfscope}%
\begin{pgfscope}%
\definecolor{textcolor}{rgb}{0.000000,0.000000,0.000000}%
\pgfsetstrokecolor{textcolor}%
\pgfsetfillcolor{textcolor}%
\pgftext[x=2.334375in,y=0.287778in,,top]{\color{textcolor}\rmfamily\fontsize{10.000000}{12.000000}\selectfont \(\displaystyle -0.25\)}%
\end{pgfscope}%
\begin{pgfscope}%
\pgfsetbuttcap%
\pgfsetroundjoin%
\definecolor{currentfill}{rgb}{0.000000,0.000000,0.000000}%
\pgfsetfillcolor{currentfill}%
\pgfsetlinewidth{0.803000pt}%
\definecolor{currentstroke}{rgb}{0.000000,0.000000,0.000000}%
\pgfsetstrokecolor{currentstroke}%
\pgfsetdash{}{0pt}%
\pgfsys@defobject{currentmarker}{\pgfqpoint{0.000000in}{-0.048611in}}{\pgfqpoint{0.000000in}{0.000000in}}{%
\pgfpathmoveto{\pgfqpoint{0.000000in}{0.000000in}}%
\pgfpathlineto{\pgfqpoint{0.000000in}{-0.048611in}}%
\pgfusepath{stroke,fill}%
}%
\begin{pgfscope}%
\pgfsys@transformshift{2.818750in}{0.385000in}%
\pgfsys@useobject{currentmarker}{}%
\end{pgfscope}%
\end{pgfscope}%
\begin{pgfscope}%
\definecolor{textcolor}{rgb}{0.000000,0.000000,0.000000}%
\pgfsetstrokecolor{textcolor}%
\pgfsetfillcolor{textcolor}%
\pgftext[x=2.818750in,y=0.287778in,,top]{\color{textcolor}\rmfamily\fontsize{10.000000}{12.000000}\selectfont \(\displaystyle 0.00\)}%
\end{pgfscope}%
\begin{pgfscope}%
\pgfsetbuttcap%
\pgfsetroundjoin%
\definecolor{currentfill}{rgb}{0.000000,0.000000,0.000000}%
\pgfsetfillcolor{currentfill}%
\pgfsetlinewidth{0.803000pt}%
\definecolor{currentstroke}{rgb}{0.000000,0.000000,0.000000}%
\pgfsetstrokecolor{currentstroke}%
\pgfsetdash{}{0pt}%
\pgfsys@defobject{currentmarker}{\pgfqpoint{0.000000in}{-0.048611in}}{\pgfqpoint{0.000000in}{0.000000in}}{%
\pgfpathmoveto{\pgfqpoint{0.000000in}{0.000000in}}%
\pgfpathlineto{\pgfqpoint{0.000000in}{-0.048611in}}%
\pgfusepath{stroke,fill}%
}%
\begin{pgfscope}%
\pgfsys@transformshift{3.303125in}{0.385000in}%
\pgfsys@useobject{currentmarker}{}%
\end{pgfscope}%
\end{pgfscope}%
\begin{pgfscope}%
\definecolor{textcolor}{rgb}{0.000000,0.000000,0.000000}%
\pgfsetstrokecolor{textcolor}%
\pgfsetfillcolor{textcolor}%
\pgftext[x=3.303125in,y=0.287778in,,top]{\color{textcolor}\rmfamily\fontsize{10.000000}{12.000000}\selectfont \(\displaystyle 0.25\)}%
\end{pgfscope}%
\begin{pgfscope}%
\pgfsetbuttcap%
\pgfsetroundjoin%
\definecolor{currentfill}{rgb}{0.000000,0.000000,0.000000}%
\pgfsetfillcolor{currentfill}%
\pgfsetlinewidth{0.803000pt}%
\definecolor{currentstroke}{rgb}{0.000000,0.000000,0.000000}%
\pgfsetstrokecolor{currentstroke}%
\pgfsetdash{}{0pt}%
\pgfsys@defobject{currentmarker}{\pgfqpoint{0.000000in}{-0.048611in}}{\pgfqpoint{0.000000in}{0.000000in}}{%
\pgfpathmoveto{\pgfqpoint{0.000000in}{0.000000in}}%
\pgfpathlineto{\pgfqpoint{0.000000in}{-0.048611in}}%
\pgfusepath{stroke,fill}%
}%
\begin{pgfscope}%
\pgfsys@transformshift{3.787500in}{0.385000in}%
\pgfsys@useobject{currentmarker}{}%
\end{pgfscope}%
\end{pgfscope}%
\begin{pgfscope}%
\definecolor{textcolor}{rgb}{0.000000,0.000000,0.000000}%
\pgfsetstrokecolor{textcolor}%
\pgfsetfillcolor{textcolor}%
\pgftext[x=3.787500in,y=0.287778in,,top]{\color{textcolor}\rmfamily\fontsize{10.000000}{12.000000}\selectfont \(\displaystyle 0.50\)}%
\end{pgfscope}%
\begin{pgfscope}%
\pgfsetbuttcap%
\pgfsetroundjoin%
\definecolor{currentfill}{rgb}{0.000000,0.000000,0.000000}%
\pgfsetfillcolor{currentfill}%
\pgfsetlinewidth{0.803000pt}%
\definecolor{currentstroke}{rgb}{0.000000,0.000000,0.000000}%
\pgfsetstrokecolor{currentstroke}%
\pgfsetdash{}{0pt}%
\pgfsys@defobject{currentmarker}{\pgfqpoint{0.000000in}{-0.048611in}}{\pgfqpoint{0.000000in}{0.000000in}}{%
\pgfpathmoveto{\pgfqpoint{0.000000in}{0.000000in}}%
\pgfpathlineto{\pgfqpoint{0.000000in}{-0.048611in}}%
\pgfusepath{stroke,fill}%
}%
\begin{pgfscope}%
\pgfsys@transformshift{4.271875in}{0.385000in}%
\pgfsys@useobject{currentmarker}{}%
\end{pgfscope}%
\end{pgfscope}%
\begin{pgfscope}%
\definecolor{textcolor}{rgb}{0.000000,0.000000,0.000000}%
\pgfsetstrokecolor{textcolor}%
\pgfsetfillcolor{textcolor}%
\pgftext[x=4.271875in,y=0.287778in,,top]{\color{textcolor}\rmfamily\fontsize{10.000000}{12.000000}\selectfont \(\displaystyle 0.75\)}%
\end{pgfscope}%
\begin{pgfscope}%
\pgfsetbuttcap%
\pgfsetroundjoin%
\definecolor{currentfill}{rgb}{0.000000,0.000000,0.000000}%
\pgfsetfillcolor{currentfill}%
\pgfsetlinewidth{0.803000pt}%
\definecolor{currentstroke}{rgb}{0.000000,0.000000,0.000000}%
\pgfsetstrokecolor{currentstroke}%
\pgfsetdash{}{0pt}%
\pgfsys@defobject{currentmarker}{\pgfqpoint{0.000000in}{-0.048611in}}{\pgfqpoint{0.000000in}{0.000000in}}{%
\pgfpathmoveto{\pgfqpoint{0.000000in}{0.000000in}}%
\pgfpathlineto{\pgfqpoint{0.000000in}{-0.048611in}}%
\pgfusepath{stroke,fill}%
}%
\begin{pgfscope}%
\pgfsys@transformshift{4.756250in}{0.385000in}%
\pgfsys@useobject{currentmarker}{}%
\end{pgfscope}%
\end{pgfscope}%
\begin{pgfscope}%
\definecolor{textcolor}{rgb}{0.000000,0.000000,0.000000}%
\pgfsetstrokecolor{textcolor}%
\pgfsetfillcolor{textcolor}%
\pgftext[x=4.756250in,y=0.287778in,,top]{\color{textcolor}\rmfamily\fontsize{10.000000}{12.000000}\selectfont \(\displaystyle 1.00\)}%
\end{pgfscope}%
\begin{pgfscope}%
\definecolor{textcolor}{rgb}{0.000000,0.000000,0.000000}%
\pgfsetstrokecolor{textcolor}%
\pgfsetfillcolor{textcolor}%
\pgftext[x=2.818750in,y=0.108766in,,top]{\color{textcolor}\rmfamily\fontsize{10.000000}{12.000000}\selectfont x}%
\end{pgfscope}%
\begin{pgfscope}%
\pgfsetbuttcap%
\pgfsetroundjoin%
\definecolor{currentfill}{rgb}{0.000000,0.000000,0.000000}%
\pgfsetfillcolor{currentfill}%
\pgfsetlinewidth{0.803000pt}%
\definecolor{currentstroke}{rgb}{0.000000,0.000000,0.000000}%
\pgfsetstrokecolor{currentstroke}%
\pgfsetdash{}{0pt}%
\pgfsys@defobject{currentmarker}{\pgfqpoint{-0.048611in}{0.000000in}}{\pgfqpoint{0.000000in}{0.000000in}}{%
\pgfpathmoveto{\pgfqpoint{0.000000in}{0.000000in}}%
\pgfpathlineto{\pgfqpoint{-0.048611in}{0.000000in}}%
\pgfusepath{stroke,fill}%
}%
\begin{pgfscope}%
\pgfsys@transformshift{0.687500in}{0.474833in}%
\pgfsys@useobject{currentmarker}{}%
\end{pgfscope}%
\end{pgfscope}%
\begin{pgfscope}%
\definecolor{textcolor}{rgb}{0.000000,0.000000,0.000000}%
\pgfsetstrokecolor{textcolor}%
\pgfsetfillcolor{textcolor}%
\pgftext[x=0.304783in,y=0.426608in,left,base]{\color{textcolor}\rmfamily\fontsize{10.000000}{12.000000}\selectfont \(\displaystyle -0.2\)}%
\end{pgfscope}%
\begin{pgfscope}%
\pgfsetbuttcap%
\pgfsetroundjoin%
\definecolor{currentfill}{rgb}{0.000000,0.000000,0.000000}%
\pgfsetfillcolor{currentfill}%
\pgfsetlinewidth{0.803000pt}%
\definecolor{currentstroke}{rgb}{0.000000,0.000000,0.000000}%
\pgfsetstrokecolor{currentstroke}%
\pgfsetdash{}{0pt}%
\pgfsys@defobject{currentmarker}{\pgfqpoint{-0.048611in}{0.000000in}}{\pgfqpoint{0.000000in}{0.000000in}}{%
\pgfpathmoveto{\pgfqpoint{0.000000in}{0.000000in}}%
\pgfpathlineto{\pgfqpoint{-0.048611in}{0.000000in}}%
\pgfusepath{stroke,fill}%
}%
\begin{pgfscope}%
\pgfsys@transformshift{0.687500in}{0.834167in}%
\pgfsys@useobject{currentmarker}{}%
\end{pgfscope}%
\end{pgfscope}%
\begin{pgfscope}%
\definecolor{textcolor}{rgb}{0.000000,0.000000,0.000000}%
\pgfsetstrokecolor{textcolor}%
\pgfsetfillcolor{textcolor}%
\pgftext[x=0.412808in,y=0.785941in,left,base]{\color{textcolor}\rmfamily\fontsize{10.000000}{12.000000}\selectfont \(\displaystyle 0.0\)}%
\end{pgfscope}%
\begin{pgfscope}%
\pgfsetbuttcap%
\pgfsetroundjoin%
\definecolor{currentfill}{rgb}{0.000000,0.000000,0.000000}%
\pgfsetfillcolor{currentfill}%
\pgfsetlinewidth{0.803000pt}%
\definecolor{currentstroke}{rgb}{0.000000,0.000000,0.000000}%
\pgfsetstrokecolor{currentstroke}%
\pgfsetdash{}{0pt}%
\pgfsys@defobject{currentmarker}{\pgfqpoint{-0.048611in}{0.000000in}}{\pgfqpoint{0.000000in}{0.000000in}}{%
\pgfpathmoveto{\pgfqpoint{0.000000in}{0.000000in}}%
\pgfpathlineto{\pgfqpoint{-0.048611in}{0.000000in}}%
\pgfusepath{stroke,fill}%
}%
\begin{pgfscope}%
\pgfsys@transformshift{0.687500in}{1.193500in}%
\pgfsys@useobject{currentmarker}{}%
\end{pgfscope}%
\end{pgfscope}%
\begin{pgfscope}%
\definecolor{textcolor}{rgb}{0.000000,0.000000,0.000000}%
\pgfsetstrokecolor{textcolor}%
\pgfsetfillcolor{textcolor}%
\pgftext[x=0.412808in,y=1.145275in,left,base]{\color{textcolor}\rmfamily\fontsize{10.000000}{12.000000}\selectfont \(\displaystyle 0.2\)}%
\end{pgfscope}%
\begin{pgfscope}%
\pgfsetbuttcap%
\pgfsetroundjoin%
\definecolor{currentfill}{rgb}{0.000000,0.000000,0.000000}%
\pgfsetfillcolor{currentfill}%
\pgfsetlinewidth{0.803000pt}%
\definecolor{currentstroke}{rgb}{0.000000,0.000000,0.000000}%
\pgfsetstrokecolor{currentstroke}%
\pgfsetdash{}{0pt}%
\pgfsys@defobject{currentmarker}{\pgfqpoint{-0.048611in}{0.000000in}}{\pgfqpoint{0.000000in}{0.000000in}}{%
\pgfpathmoveto{\pgfqpoint{0.000000in}{0.000000in}}%
\pgfpathlineto{\pgfqpoint{-0.048611in}{0.000000in}}%
\pgfusepath{stroke,fill}%
}%
\begin{pgfscope}%
\pgfsys@transformshift{0.687500in}{1.552833in}%
\pgfsys@useobject{currentmarker}{}%
\end{pgfscope}%
\end{pgfscope}%
\begin{pgfscope}%
\definecolor{textcolor}{rgb}{0.000000,0.000000,0.000000}%
\pgfsetstrokecolor{textcolor}%
\pgfsetfillcolor{textcolor}%
\pgftext[x=0.412808in,y=1.504608in,left,base]{\color{textcolor}\rmfamily\fontsize{10.000000}{12.000000}\selectfont \(\displaystyle 0.4\)}%
\end{pgfscope}%
\begin{pgfscope}%
\pgfsetbuttcap%
\pgfsetroundjoin%
\definecolor{currentfill}{rgb}{0.000000,0.000000,0.000000}%
\pgfsetfillcolor{currentfill}%
\pgfsetlinewidth{0.803000pt}%
\definecolor{currentstroke}{rgb}{0.000000,0.000000,0.000000}%
\pgfsetstrokecolor{currentstroke}%
\pgfsetdash{}{0pt}%
\pgfsys@defobject{currentmarker}{\pgfqpoint{-0.048611in}{0.000000in}}{\pgfqpoint{0.000000in}{0.000000in}}{%
\pgfpathmoveto{\pgfqpoint{0.000000in}{0.000000in}}%
\pgfpathlineto{\pgfqpoint{-0.048611in}{0.000000in}}%
\pgfusepath{stroke,fill}%
}%
\begin{pgfscope}%
\pgfsys@transformshift{0.687500in}{1.912167in}%
\pgfsys@useobject{currentmarker}{}%
\end{pgfscope}%
\end{pgfscope}%
\begin{pgfscope}%
\definecolor{textcolor}{rgb}{0.000000,0.000000,0.000000}%
\pgfsetstrokecolor{textcolor}%
\pgfsetfillcolor{textcolor}%
\pgftext[x=0.412808in,y=1.863941in,left,base]{\color{textcolor}\rmfamily\fontsize{10.000000}{12.000000}\selectfont \(\displaystyle 0.6\)}%
\end{pgfscope}%
\begin{pgfscope}%
\pgfsetbuttcap%
\pgfsetroundjoin%
\definecolor{currentfill}{rgb}{0.000000,0.000000,0.000000}%
\pgfsetfillcolor{currentfill}%
\pgfsetlinewidth{0.803000pt}%
\definecolor{currentstroke}{rgb}{0.000000,0.000000,0.000000}%
\pgfsetstrokecolor{currentstroke}%
\pgfsetdash{}{0pt}%
\pgfsys@defobject{currentmarker}{\pgfqpoint{-0.048611in}{0.000000in}}{\pgfqpoint{0.000000in}{0.000000in}}{%
\pgfpathmoveto{\pgfqpoint{0.000000in}{0.000000in}}%
\pgfpathlineto{\pgfqpoint{-0.048611in}{0.000000in}}%
\pgfusepath{stroke,fill}%
}%
\begin{pgfscope}%
\pgfsys@transformshift{0.687500in}{2.271500in}%
\pgfsys@useobject{currentmarker}{}%
\end{pgfscope}%
\end{pgfscope}%
\begin{pgfscope}%
\definecolor{textcolor}{rgb}{0.000000,0.000000,0.000000}%
\pgfsetstrokecolor{textcolor}%
\pgfsetfillcolor{textcolor}%
\pgftext[x=0.412808in,y=2.223275in,left,base]{\color{textcolor}\rmfamily\fontsize{10.000000}{12.000000}\selectfont \(\displaystyle 0.8\)}%
\end{pgfscope}%
\begin{pgfscope}%
\pgfsetbuttcap%
\pgfsetroundjoin%
\definecolor{currentfill}{rgb}{0.000000,0.000000,0.000000}%
\pgfsetfillcolor{currentfill}%
\pgfsetlinewidth{0.803000pt}%
\definecolor{currentstroke}{rgb}{0.000000,0.000000,0.000000}%
\pgfsetstrokecolor{currentstroke}%
\pgfsetdash{}{0pt}%
\pgfsys@defobject{currentmarker}{\pgfqpoint{-0.048611in}{0.000000in}}{\pgfqpoint{0.000000in}{0.000000in}}{%
\pgfpathmoveto{\pgfqpoint{0.000000in}{0.000000in}}%
\pgfpathlineto{\pgfqpoint{-0.048611in}{0.000000in}}%
\pgfusepath{stroke,fill}%
}%
\begin{pgfscope}%
\pgfsys@transformshift{0.687500in}{2.630833in}%
\pgfsys@useobject{currentmarker}{}%
\end{pgfscope}%
\end{pgfscope}%
\begin{pgfscope}%
\definecolor{textcolor}{rgb}{0.000000,0.000000,0.000000}%
\pgfsetstrokecolor{textcolor}%
\pgfsetfillcolor{textcolor}%
\pgftext[x=0.412808in,y=2.582608in,left,base]{\color{textcolor}\rmfamily\fontsize{10.000000}{12.000000}\selectfont \(\displaystyle 1.0\)}%
\end{pgfscope}%
\begin{pgfscope}%
\pgfsetbuttcap%
\pgfsetroundjoin%
\definecolor{currentfill}{rgb}{0.000000,0.000000,0.000000}%
\pgfsetfillcolor{currentfill}%
\pgfsetlinewidth{0.803000pt}%
\definecolor{currentstroke}{rgb}{0.000000,0.000000,0.000000}%
\pgfsetstrokecolor{currentstroke}%
\pgfsetdash{}{0pt}%
\pgfsys@defobject{currentmarker}{\pgfqpoint{-0.048611in}{0.000000in}}{\pgfqpoint{0.000000in}{0.000000in}}{%
\pgfpathmoveto{\pgfqpoint{0.000000in}{0.000000in}}%
\pgfpathlineto{\pgfqpoint{-0.048611in}{0.000000in}}%
\pgfusepath{stroke,fill}%
}%
\begin{pgfscope}%
\pgfsys@transformshift{0.687500in}{2.990167in}%
\pgfsys@useobject{currentmarker}{}%
\end{pgfscope}%
\end{pgfscope}%
\begin{pgfscope}%
\definecolor{textcolor}{rgb}{0.000000,0.000000,0.000000}%
\pgfsetstrokecolor{textcolor}%
\pgfsetfillcolor{textcolor}%
\pgftext[x=0.412808in,y=2.941941in,left,base]{\color{textcolor}\rmfamily\fontsize{10.000000}{12.000000}\selectfont \(\displaystyle 1.2\)}%
\end{pgfscope}%
\begin{pgfscope}%
\definecolor{textcolor}{rgb}{0.000000,0.000000,0.000000}%
\pgfsetstrokecolor{textcolor}%
\pgfsetfillcolor{textcolor}%
\pgftext[x=0.249228in,y=1.732500in,,bottom,rotate=90.000000]{\color{textcolor}\rmfamily\fontsize{10.000000}{12.000000}\selectfont y}%
\end{pgfscope}%
\begin{pgfscope}%
\pgfpathrectangle{\pgfqpoint{0.687500in}{0.385000in}}{\pgfqpoint{4.262500in}{2.695000in}}%
\pgfusepath{clip}%
\pgfsetrectcap%
\pgfsetroundjoin%
\pgfsetlinewidth{1.505625pt}%
\definecolor{currentstroke}{rgb}{0.121569,0.466667,0.705882}%
\pgfsetstrokecolor{currentstroke}%
\pgfsetdash{}{0pt}%
\pgfpathmoveto{\pgfqpoint{0.881250in}{0.903269in}}%
\pgfpathlineto{\pgfqpoint{1.017557in}{0.913644in}}%
\pgfpathlineto{\pgfqpoint{1.134391in}{0.924478in}}%
\pgfpathlineto{\pgfqpoint{1.251225in}{0.937638in}}%
\pgfpathlineto{\pgfqpoint{1.348587in}{0.950877in}}%
\pgfpathlineto{\pgfqpoint{1.426476in}{0.963336in}}%
\pgfpathlineto{\pgfqpoint{1.504366in}{0.977838in}}%
\pgfpathlineto{\pgfqpoint{1.582255in}{0.994839in}}%
\pgfpathlineto{\pgfqpoint{1.640672in}{1.009574in}}%
\pgfpathlineto{\pgfqpoint{1.699089in}{1.026346in}}%
\pgfpathlineto{\pgfqpoint{1.757506in}{1.045528in}}%
\pgfpathlineto{\pgfqpoint{1.815923in}{1.067578in}}%
\pgfpathlineto{\pgfqpoint{1.854868in}{1.084143in}}%
\pgfpathlineto{\pgfqpoint{1.893813in}{1.102428in}}%
\pgfpathlineto{\pgfqpoint{1.932758in}{1.122659in}}%
\pgfpathlineto{\pgfqpoint{1.971702in}{1.145101in}}%
\pgfpathlineto{\pgfqpoint{2.010647in}{1.170055in}}%
\pgfpathlineto{\pgfqpoint{2.049592in}{1.197870in}}%
\pgfpathlineto{\pgfqpoint{2.088536in}{1.228948in}}%
\pgfpathlineto{\pgfqpoint{2.127481in}{1.263748in}}%
\pgfpathlineto{\pgfqpoint{2.166426in}{1.302794in}}%
\pgfpathlineto{\pgfqpoint{2.205371in}{1.346677in}}%
\pgfpathlineto{\pgfqpoint{2.244315in}{1.396056in}}%
\pgfpathlineto{\pgfqpoint{2.283260in}{1.451647in}}%
\pgfpathlineto{\pgfqpoint{2.322205in}{1.514206in}}%
\pgfpathlineto{\pgfqpoint{2.361149in}{1.584486in}}%
\pgfpathlineto{\pgfqpoint{2.400094in}{1.663167in}}%
\pgfpathlineto{\pgfqpoint{2.439039in}{1.750738in}}%
\pgfpathlineto{\pgfqpoint{2.477984in}{1.847321in}}%
\pgfpathlineto{\pgfqpoint{2.516928in}{1.952417in}}%
\pgfpathlineto{\pgfqpoint{2.555873in}{2.064579in}}%
\pgfpathlineto{\pgfqpoint{2.653235in}{2.353617in}}%
\pgfpathlineto{\pgfqpoint{2.672707in}{2.407372in}}%
\pgfpathlineto{\pgfqpoint{2.692180in}{2.457628in}}%
\pgfpathlineto{\pgfqpoint{2.711652in}{2.503331in}}%
\pgfpathlineto{\pgfqpoint{2.731124in}{2.543430in}}%
\pgfpathlineto{\pgfqpoint{2.750597in}{2.576924in}}%
\pgfpathlineto{\pgfqpoint{2.770069in}{2.602918in}}%
\pgfpathlineto{\pgfqpoint{2.789541in}{2.620683in}}%
\pgfpathlineto{\pgfqpoint{2.809014in}{2.629700in}}%
\pgfpathlineto{\pgfqpoint{2.828486in}{2.629700in}}%
\pgfpathlineto{\pgfqpoint{2.847959in}{2.620683in}}%
\pgfpathlineto{\pgfqpoint{2.867431in}{2.602918in}}%
\pgfpathlineto{\pgfqpoint{2.886903in}{2.576924in}}%
\pgfpathlineto{\pgfqpoint{2.906376in}{2.543430in}}%
\pgfpathlineto{\pgfqpoint{2.925848in}{2.503331in}}%
\pgfpathlineto{\pgfqpoint{2.945320in}{2.457628in}}%
\pgfpathlineto{\pgfqpoint{2.964793in}{2.407372in}}%
\pgfpathlineto{\pgfqpoint{3.003737in}{2.297371in}}%
\pgfpathlineto{\pgfqpoint{3.120572in}{1.952417in}}%
\pgfpathlineto{\pgfqpoint{3.159516in}{1.847321in}}%
\pgfpathlineto{\pgfqpoint{3.198461in}{1.750738in}}%
\pgfpathlineto{\pgfqpoint{3.237406in}{1.663167in}}%
\pgfpathlineto{\pgfqpoint{3.276351in}{1.584486in}}%
\pgfpathlineto{\pgfqpoint{3.315295in}{1.514206in}}%
\pgfpathlineto{\pgfqpoint{3.354240in}{1.451647in}}%
\pgfpathlineto{\pgfqpoint{3.393185in}{1.396056in}}%
\pgfpathlineto{\pgfqpoint{3.432129in}{1.346677in}}%
\pgfpathlineto{\pgfqpoint{3.471074in}{1.302794in}}%
\pgfpathlineto{\pgfqpoint{3.510019in}{1.263748in}}%
\pgfpathlineto{\pgfqpoint{3.548964in}{1.228948in}}%
\pgfpathlineto{\pgfqpoint{3.587908in}{1.197870in}}%
\pgfpathlineto{\pgfqpoint{3.626853in}{1.170055in}}%
\pgfpathlineto{\pgfqpoint{3.665798in}{1.145101in}}%
\pgfpathlineto{\pgfqpoint{3.704742in}{1.122659in}}%
\pgfpathlineto{\pgfqpoint{3.743687in}{1.102428in}}%
\pgfpathlineto{\pgfqpoint{3.782632in}{1.084143in}}%
\pgfpathlineto{\pgfqpoint{3.841049in}{1.059877in}}%
\pgfpathlineto{\pgfqpoint{3.899466in}{1.038840in}}%
\pgfpathlineto{\pgfqpoint{3.957883in}{1.020508in}}%
\pgfpathlineto{\pgfqpoint{4.016300in}{1.004452in}}%
\pgfpathlineto{\pgfqpoint{4.074717in}{0.990325in}}%
\pgfpathlineto{\pgfqpoint{4.152607in}{0.973998in}}%
\pgfpathlineto{\pgfqpoint{4.230496in}{0.960045in}}%
\pgfpathlineto{\pgfqpoint{4.327858in}{0.945299in}}%
\pgfpathlineto{\pgfqpoint{4.425220in}{0.932955in}}%
\pgfpathlineto{\pgfqpoint{4.542054in}{0.920637in}}%
\pgfpathlineto{\pgfqpoint{4.678361in}{0.908933in}}%
\pgfpathlineto{\pgfqpoint{4.756250in}{0.903269in}}%
\pgfpathlineto{\pgfqpoint{4.756250in}{0.903269in}}%
\pgfusepath{stroke}%
\end{pgfscope}%
\begin{pgfscope}%
\pgfpathrectangle{\pgfqpoint{0.687500in}{0.385000in}}{\pgfqpoint{4.262500in}{2.695000in}}%
\pgfusepath{clip}%
\pgfsetbuttcap%
\pgfsetroundjoin%
\definecolor{currentfill}{rgb}{1.000000,0.498039,0.054902}%
\pgfsetfillcolor{currentfill}%
\pgfsetlinewidth{1.003750pt}%
\definecolor{currentstroke}{rgb}{1.000000,0.498039,0.054902}%
\pgfsetstrokecolor{currentstroke}%
\pgfsetdash{}{0pt}%
\pgfsys@defobject{currentmarker}{\pgfqpoint{-0.020833in}{-0.020833in}}{\pgfqpoint{0.020833in}{0.020833in}}{%
\pgfpathmoveto{\pgfqpoint{0.000000in}{-0.020833in}}%
\pgfpathcurveto{\pgfqpoint{0.005525in}{-0.020833in}}{\pgfqpoint{0.010825in}{-0.018638in}}{\pgfqpoint{0.014731in}{-0.014731in}}%
\pgfpathcurveto{\pgfqpoint{0.018638in}{-0.010825in}}{\pgfqpoint{0.020833in}{-0.005525in}}{\pgfqpoint{0.020833in}{0.000000in}}%
\pgfpathcurveto{\pgfqpoint{0.020833in}{0.005525in}}{\pgfqpoint{0.018638in}{0.010825in}}{\pgfqpoint{0.014731in}{0.014731in}}%
\pgfpathcurveto{\pgfqpoint{0.010825in}{0.018638in}}{\pgfqpoint{0.005525in}{0.020833in}}{\pgfqpoint{0.000000in}{0.020833in}}%
\pgfpathcurveto{\pgfqpoint{-0.005525in}{0.020833in}}{\pgfqpoint{-0.010825in}{0.018638in}}{\pgfqpoint{-0.014731in}{0.014731in}}%
\pgfpathcurveto{\pgfqpoint{-0.018638in}{0.010825in}}{\pgfqpoint{-0.020833in}{0.005525in}}{\pgfqpoint{-0.020833in}{0.000000in}}%
\pgfpathcurveto{\pgfqpoint{-0.020833in}{-0.005525in}}{\pgfqpoint{-0.018638in}{-0.010825in}}{\pgfqpoint{-0.014731in}{-0.014731in}}%
\pgfpathcurveto{\pgfqpoint{-0.010825in}{-0.018638in}}{\pgfqpoint{-0.005525in}{-0.020833in}}{\pgfqpoint{0.000000in}{-0.020833in}}%
\pgfpathclose%
\pgfusepath{stroke,fill}%
}%
\begin{pgfscope}%
\pgfsys@transformshift{4.707673in}{0.906724in}%
\pgfsys@useobject{currentmarker}{}%
\end{pgfscope}%
\begin{pgfscope}%
\pgfsys@transformshift{4.333548in}{0.944517in}%
\pgfsys@useobject{currentmarker}{}%
\end{pgfscope}%
\begin{pgfscope}%
\pgfsys@transformshift{3.659400in}{1.149019in}%
\pgfsys@useobject{currentmarker}{}%
\end{pgfscope}%
\begin{pgfscope}%
\pgfsys@transformshift{2.818750in}{2.630833in}%
\pgfsys@useobject{currentmarker}{}%
\end{pgfscope}%
\begin{pgfscope}%
\pgfsys@transformshift{1.978100in}{1.149019in}%
\pgfsys@useobject{currentmarker}{}%
\end{pgfscope}%
\begin{pgfscope}%
\pgfsys@transformshift{1.303952in}{0.944517in}%
\pgfsys@useobject{currentmarker}{}%
\end{pgfscope}%
\begin{pgfscope}%
\pgfsys@transformshift{0.929827in}{0.906724in}%
\pgfsys@useobject{currentmarker}{}%
\end{pgfscope}%
\end{pgfscope}%
\begin{pgfscope}%
\pgfpathrectangle{\pgfqpoint{0.687500in}{0.385000in}}{\pgfqpoint{4.262500in}{2.695000in}}%
\pgfusepath{clip}%
\pgfsetrectcap%
\pgfsetroundjoin%
\pgfsetlinewidth{1.505625pt}%
\definecolor{currentstroke}{rgb}{0.172549,0.627451,0.172549}%
\pgfsetstrokecolor{currentstroke}%
\pgfsetdash{}{0pt}%
\pgfpathmoveto{\pgfqpoint{0.881250in}{0.719940in}}%
\pgfpathlineto{\pgfqpoint{0.900722in}{0.803598in}}%
\pgfpathlineto{\pgfqpoint{0.920195in}{0.875352in}}%
\pgfpathlineto{\pgfqpoint{0.939667in}{0.936100in}}%
\pgfpathlineto{\pgfqpoint{0.959139in}{0.986699in}}%
\pgfpathlineto{\pgfqpoint{0.978612in}{1.027972in}}%
\pgfpathlineto{\pgfqpoint{0.998084in}{1.060704in}}%
\pgfpathlineto{\pgfqpoint{1.017557in}{1.085645in}}%
\pgfpathlineto{\pgfqpoint{1.037029in}{1.103514in}}%
\pgfpathlineto{\pgfqpoint{1.056501in}{1.114991in}}%
\pgfpathlineto{\pgfqpoint{1.075974in}{1.120729in}}%
\pgfpathlineto{\pgfqpoint{1.095446in}{1.121346in}}%
\pgfpathlineto{\pgfqpoint{1.114918in}{1.117429in}}%
\pgfpathlineto{\pgfqpoint{1.134391in}{1.109537in}}%
\pgfpathlineto{\pgfqpoint{1.153863in}{1.098197in}}%
\pgfpathlineto{\pgfqpoint{1.173335in}{1.083911in}}%
\pgfpathlineto{\pgfqpoint{1.192808in}{1.067150in}}%
\pgfpathlineto{\pgfqpoint{1.212280in}{1.048359in}}%
\pgfpathlineto{\pgfqpoint{1.251225in}{1.006339in}}%
\pgfpathlineto{\pgfqpoint{1.309642in}{0.937750in}}%
\pgfpathlineto{\pgfqpoint{1.368059in}{0.870084in}}%
\pgfpathlineto{\pgfqpoint{1.407004in}{0.828977in}}%
\pgfpathlineto{\pgfqpoint{1.445948in}{0.793045in}}%
\pgfpathlineto{\pgfqpoint{1.465421in}{0.777455in}}%
\pgfpathlineto{\pgfqpoint{1.484893in}{0.763644in}}%
\pgfpathlineto{\pgfqpoint{1.504366in}{0.751741in}}%
\pgfpathlineto{\pgfqpoint{1.523838in}{0.741856in}}%
\pgfpathlineto{\pgfqpoint{1.543310in}{0.734084in}}%
\pgfpathlineto{\pgfqpoint{1.562783in}{0.728507in}}%
\pgfpathlineto{\pgfqpoint{1.582255in}{0.725191in}}%
\pgfpathlineto{\pgfqpoint{1.601727in}{0.724190in}}%
\pgfpathlineto{\pgfqpoint{1.621200in}{0.725543in}}%
\pgfpathlineto{\pgfqpoint{1.640672in}{0.729280in}}%
\pgfpathlineto{\pgfqpoint{1.660144in}{0.735417in}}%
\pgfpathlineto{\pgfqpoint{1.679617in}{0.743958in}}%
\pgfpathlineto{\pgfqpoint{1.699089in}{0.754899in}}%
\pgfpathlineto{\pgfqpoint{1.718562in}{0.768222in}}%
\pgfpathlineto{\pgfqpoint{1.738034in}{0.783904in}}%
\pgfpathlineto{\pgfqpoint{1.757506in}{0.801909in}}%
\pgfpathlineto{\pgfqpoint{1.776979in}{0.822195in}}%
\pgfpathlineto{\pgfqpoint{1.796451in}{0.844711in}}%
\pgfpathlineto{\pgfqpoint{1.815923in}{0.869397in}}%
\pgfpathlineto{\pgfqpoint{1.835396in}{0.896188in}}%
\pgfpathlineto{\pgfqpoint{1.874340in}{0.955788in}}%
\pgfpathlineto{\pgfqpoint{1.913285in}{1.022858in}}%
\pgfpathlineto{\pgfqpoint{1.952230in}{1.096663in}}%
\pgfpathlineto{\pgfqpoint{1.991175in}{1.176394in}}%
\pgfpathlineto{\pgfqpoint{2.030119in}{1.261185in}}%
\pgfpathlineto{\pgfqpoint{2.088536in}{1.395857in}}%
\pgfpathlineto{\pgfqpoint{2.166426in}{1.584376in}}%
\pgfpathlineto{\pgfqpoint{2.302732in}{1.917050in}}%
\pgfpathlineto{\pgfqpoint{2.361149in}{2.052360in}}%
\pgfpathlineto{\pgfqpoint{2.400094in}{2.137894in}}%
\pgfpathlineto{\pgfqpoint{2.439039in}{2.218720in}}%
\pgfpathlineto{\pgfqpoint{2.477984in}{2.294066in}}%
\pgfpathlineto{\pgfqpoint{2.516928in}{2.363218in}}%
\pgfpathlineto{\pgfqpoint{2.555873in}{2.425525in}}%
\pgfpathlineto{\pgfqpoint{2.594818in}{2.480404in}}%
\pgfpathlineto{\pgfqpoint{2.614290in}{2.504895in}}%
\pgfpathlineto{\pgfqpoint{2.633763in}{2.527343in}}%
\pgfpathlineto{\pgfqpoint{2.653235in}{2.547697in}}%
\pgfpathlineto{\pgfqpoint{2.672707in}{2.565908in}}%
\pgfpathlineto{\pgfqpoint{2.692180in}{2.581936in}}%
\pgfpathlineto{\pgfqpoint{2.711652in}{2.595744in}}%
\pgfpathlineto{\pgfqpoint{2.731124in}{2.607299in}}%
\pgfpathlineto{\pgfqpoint{2.750597in}{2.616574in}}%
\pgfpathlineto{\pgfqpoint{2.770069in}{2.623550in}}%
\pgfpathlineto{\pgfqpoint{2.789541in}{2.628209in}}%
\pgfpathlineto{\pgfqpoint{2.809014in}{2.630542in}}%
\pgfpathlineto{\pgfqpoint{2.828486in}{2.630542in}}%
\pgfpathlineto{\pgfqpoint{2.847959in}{2.628209in}}%
\pgfpathlineto{\pgfqpoint{2.867431in}{2.623550in}}%
\pgfpathlineto{\pgfqpoint{2.886903in}{2.616574in}}%
\pgfpathlineto{\pgfqpoint{2.906376in}{2.607299in}}%
\pgfpathlineto{\pgfqpoint{2.925848in}{2.595744in}}%
\pgfpathlineto{\pgfqpoint{2.945320in}{2.581936in}}%
\pgfpathlineto{\pgfqpoint{2.964793in}{2.565908in}}%
\pgfpathlineto{\pgfqpoint{2.984265in}{2.547697in}}%
\pgfpathlineto{\pgfqpoint{3.003737in}{2.527343in}}%
\pgfpathlineto{\pgfqpoint{3.023210in}{2.504895in}}%
\pgfpathlineto{\pgfqpoint{3.042682in}{2.480404in}}%
\pgfpathlineto{\pgfqpoint{3.062155in}{2.453927in}}%
\pgfpathlineto{\pgfqpoint{3.101099in}{2.395265in}}%
\pgfpathlineto{\pgfqpoint{3.140044in}{2.329459in}}%
\pgfpathlineto{\pgfqpoint{3.178989in}{2.257124in}}%
\pgfpathlineto{\pgfqpoint{3.217933in}{2.178945in}}%
\pgfpathlineto{\pgfqpoint{3.256878in}{2.095665in}}%
\pgfpathlineto{\pgfqpoint{3.295823in}{2.008084in}}%
\pgfpathlineto{\pgfqpoint{3.354240in}{1.870515in}}%
\pgfpathlineto{\pgfqpoint{3.432129in}{1.680281in}}%
\pgfpathlineto{\pgfqpoint{3.529491in}{1.442269in}}%
\pgfpathlineto{\pgfqpoint{3.587908in}{1.305195in}}%
\pgfpathlineto{\pgfqpoint{3.626853in}{1.218213in}}%
\pgfpathlineto{\pgfqpoint{3.665798in}{1.135841in}}%
\pgfpathlineto{\pgfqpoint{3.704742in}{1.058968in}}%
\pgfpathlineto{\pgfqpoint{3.743687in}{0.988433in}}%
\pgfpathlineto{\pgfqpoint{3.782632in}{0.925011in}}%
\pgfpathlineto{\pgfqpoint{3.802104in}{0.896188in}}%
\pgfpathlineto{\pgfqpoint{3.821577in}{0.869397in}}%
\pgfpathlineto{\pgfqpoint{3.841049in}{0.844711in}}%
\pgfpathlineto{\pgfqpoint{3.860521in}{0.822195in}}%
\pgfpathlineto{\pgfqpoint{3.879994in}{0.801909in}}%
\pgfpathlineto{\pgfqpoint{3.899466in}{0.783904in}}%
\pgfpathlineto{\pgfqpoint{3.918938in}{0.768222in}}%
\pgfpathlineto{\pgfqpoint{3.938411in}{0.754899in}}%
\pgfpathlineto{\pgfqpoint{3.957883in}{0.743958in}}%
\pgfpathlineto{\pgfqpoint{3.977356in}{0.735417in}}%
\pgfpathlineto{\pgfqpoint{3.996828in}{0.729280in}}%
\pgfpathlineto{\pgfqpoint{4.016300in}{0.725543in}}%
\pgfpathlineto{\pgfqpoint{4.035773in}{0.724190in}}%
\pgfpathlineto{\pgfqpoint{4.055245in}{0.725191in}}%
\pgfpathlineto{\pgfqpoint{4.074717in}{0.728507in}}%
\pgfpathlineto{\pgfqpoint{4.094190in}{0.734084in}}%
\pgfpathlineto{\pgfqpoint{4.113662in}{0.741856in}}%
\pgfpathlineto{\pgfqpoint{4.133134in}{0.751741in}}%
\pgfpathlineto{\pgfqpoint{4.152607in}{0.763644in}}%
\pgfpathlineto{\pgfqpoint{4.172079in}{0.777455in}}%
\pgfpathlineto{\pgfqpoint{4.191552in}{0.793045in}}%
\pgfpathlineto{\pgfqpoint{4.230496in}{0.828977in}}%
\pgfpathlineto{\pgfqpoint{4.269441in}{0.870084in}}%
\pgfpathlineto{\pgfqpoint{4.308386in}{0.914715in}}%
\pgfpathlineto{\pgfqpoint{4.386275in}{1.006339in}}%
\pgfpathlineto{\pgfqpoint{4.425220in}{1.048359in}}%
\pgfpathlineto{\pgfqpoint{4.444692in}{1.067150in}}%
\pgfpathlineto{\pgfqpoint{4.464165in}{1.083911in}}%
\pgfpathlineto{\pgfqpoint{4.483637in}{1.098197in}}%
\pgfpathlineto{\pgfqpoint{4.503109in}{1.109537in}}%
\pgfpathlineto{\pgfqpoint{4.522582in}{1.117429in}}%
\pgfpathlineto{\pgfqpoint{4.542054in}{1.121346in}}%
\pgfpathlineto{\pgfqpoint{4.561526in}{1.120729in}}%
\pgfpathlineto{\pgfqpoint{4.580999in}{1.114991in}}%
\pgfpathlineto{\pgfqpoint{4.600471in}{1.103514in}}%
\pgfpathlineto{\pgfqpoint{4.619943in}{1.085645in}}%
\pgfpathlineto{\pgfqpoint{4.639416in}{1.060704in}}%
\pgfpathlineto{\pgfqpoint{4.658888in}{1.027972in}}%
\pgfpathlineto{\pgfqpoint{4.678361in}{0.986699in}}%
\pgfpathlineto{\pgfqpoint{4.697833in}{0.936100in}}%
\pgfpathlineto{\pgfqpoint{4.717305in}{0.875352in}}%
\pgfpathlineto{\pgfqpoint{4.736778in}{0.803598in}}%
\pgfpathlineto{\pgfqpoint{4.756250in}{0.719940in}}%
\pgfpathlineto{\pgfqpoint{4.756250in}{0.719940in}}%
\pgfusepath{stroke}%
\end{pgfscope}%
\begin{pgfscope}%
\pgfpathrectangle{\pgfqpoint{0.687500in}{0.385000in}}{\pgfqpoint{4.262500in}{2.695000in}}%
\pgfusepath{clip}%
\pgfsetbuttcap%
\pgfsetroundjoin%
\definecolor{currentfill}{rgb}{0.839216,0.152941,0.156863}%
\pgfsetfillcolor{currentfill}%
\pgfsetlinewidth{1.003750pt}%
\definecolor{currentstroke}{rgb}{0.839216,0.152941,0.156863}%
\pgfsetstrokecolor{currentstroke}%
\pgfsetdash{}{0pt}%
\pgfsys@defobject{currentmarker}{\pgfqpoint{-0.020833in}{-0.020833in}}{\pgfqpoint{0.020833in}{0.020833in}}{%
\pgfpathmoveto{\pgfqpoint{0.000000in}{-0.020833in}}%
\pgfpathcurveto{\pgfqpoint{0.005525in}{-0.020833in}}{\pgfqpoint{0.010825in}{-0.018638in}}{\pgfqpoint{0.014731in}{-0.014731in}}%
\pgfpathcurveto{\pgfqpoint{0.018638in}{-0.010825in}}{\pgfqpoint{0.020833in}{-0.005525in}}{\pgfqpoint{0.020833in}{0.000000in}}%
\pgfpathcurveto{\pgfqpoint{0.020833in}{0.005525in}}{\pgfqpoint{0.018638in}{0.010825in}}{\pgfqpoint{0.014731in}{0.014731in}}%
\pgfpathcurveto{\pgfqpoint{0.010825in}{0.018638in}}{\pgfqpoint{0.005525in}{0.020833in}}{\pgfqpoint{0.000000in}{0.020833in}}%
\pgfpathcurveto{\pgfqpoint{-0.005525in}{0.020833in}}{\pgfqpoint{-0.010825in}{0.018638in}}{\pgfqpoint{-0.014731in}{0.014731in}}%
\pgfpathcurveto{\pgfqpoint{-0.018638in}{0.010825in}}{\pgfqpoint{-0.020833in}{0.005525in}}{\pgfqpoint{-0.020833in}{0.000000in}}%
\pgfpathcurveto{\pgfqpoint{-0.020833in}{-0.005525in}}{\pgfqpoint{-0.018638in}{-0.010825in}}{\pgfqpoint{-0.014731in}{-0.014731in}}%
\pgfpathcurveto{\pgfqpoint{-0.010825in}{-0.018638in}}{\pgfqpoint{-0.005525in}{-0.020833in}}{\pgfqpoint{0.000000in}{-0.020833in}}%
\pgfpathclose%
\pgfusepath{stroke,fill}%
}%
\begin{pgfscope}%
\pgfsys@transformshift{4.726815in}{0.905333in}%
\pgfsys@useobject{currentmarker}{}%
\end{pgfscope}%
\begin{pgfscope}%
\pgfsys@transformshift{4.496674in}{0.925137in}%
\pgfsys@useobject{currentmarker}{}%
\end{pgfscope}%
\begin{pgfscope}%
\pgfsys@transformshift{4.064151in}{0.992751in}%
\pgfsys@useobject{currentmarker}{}%
\end{pgfscope}%
\begin{pgfscope}%
\pgfsys@transformshift{3.481414in}{1.291981in}%
\pgfsys@useobject{currentmarker}{}%
\end{pgfscope}%
\begin{pgfscope}%
\pgfsys@transformshift{2.818750in}{2.630833in}%
\pgfsys@useobject{currentmarker}{}%
\end{pgfscope}%
\begin{pgfscope}%
\pgfsys@transformshift{2.156086in}{1.291981in}%
\pgfsys@useobject{currentmarker}{}%
\end{pgfscope}%
\begin{pgfscope}%
\pgfsys@transformshift{1.573349in}{0.992751in}%
\pgfsys@useobject{currentmarker}{}%
\end{pgfscope}%
\begin{pgfscope}%
\pgfsys@transformshift{1.140826in}{0.925137in}%
\pgfsys@useobject{currentmarker}{}%
\end{pgfscope}%
\begin{pgfscope}%
\pgfsys@transformshift{0.910685in}{0.905333in}%
\pgfsys@useobject{currentmarker}{}%
\end{pgfscope}%
\end{pgfscope}%
\begin{pgfscope}%
\pgfpathrectangle{\pgfqpoint{0.687500in}{0.385000in}}{\pgfqpoint{4.262500in}{2.695000in}}%
\pgfusepath{clip}%
\pgfsetrectcap%
\pgfsetroundjoin%
\pgfsetlinewidth{1.505625pt}%
\definecolor{currentstroke}{rgb}{0.580392,0.403922,0.741176}%
\pgfsetstrokecolor{currentstroke}%
\pgfsetdash{}{0pt}%
\pgfpathmoveto{\pgfqpoint{0.881250in}{1.022176in}}%
\pgfpathlineto{\pgfqpoint{0.900722in}{0.939374in}}%
\pgfpathlineto{\pgfqpoint{0.920195in}{0.877602in}}%
\pgfpathlineto{\pgfqpoint{0.939667in}{0.833937in}}%
\pgfpathlineto{\pgfqpoint{0.959139in}{0.805713in}}%
\pgfpathlineto{\pgfqpoint{0.978612in}{0.790508in}}%
\pgfpathlineto{\pgfqpoint{0.998084in}{0.786130in}}%
\pgfpathlineto{\pgfqpoint{1.017557in}{0.790600in}}%
\pgfpathlineto{\pgfqpoint{1.037029in}{0.802144in}}%
\pgfpathlineto{\pgfqpoint{1.056501in}{0.819173in}}%
\pgfpathlineto{\pgfqpoint{1.075974in}{0.840279in}}%
\pgfpathlineto{\pgfqpoint{1.114918in}{0.889892in}}%
\pgfpathlineto{\pgfqpoint{1.173335in}{0.968526in}}%
\pgfpathlineto{\pgfqpoint{1.212280in}{1.015601in}}%
\pgfpathlineto{\pgfqpoint{1.231753in}{1.036095in}}%
\pgfpathlineto{\pgfqpoint{1.251225in}{1.054139in}}%
\pgfpathlineto{\pgfqpoint{1.270697in}{1.069519in}}%
\pgfpathlineto{\pgfqpoint{1.290170in}{1.082092in}}%
\pgfpathlineto{\pgfqpoint{1.309642in}{1.091782in}}%
\pgfpathlineto{\pgfqpoint{1.329114in}{1.098570in}}%
\pgfpathlineto{\pgfqpoint{1.348587in}{1.102490in}}%
\pgfpathlineto{\pgfqpoint{1.368059in}{1.103619in}}%
\pgfpathlineto{\pgfqpoint{1.387531in}{1.102079in}}%
\pgfpathlineto{\pgfqpoint{1.407004in}{1.098023in}}%
\pgfpathlineto{\pgfqpoint{1.426476in}{1.091637in}}%
\pgfpathlineto{\pgfqpoint{1.445948in}{1.083131in}}%
\pgfpathlineto{\pgfqpoint{1.465421in}{1.072737in}}%
\pgfpathlineto{\pgfqpoint{1.484893in}{1.060702in}}%
\pgfpathlineto{\pgfqpoint{1.523838in}{1.032767in}}%
\pgfpathlineto{\pgfqpoint{1.582255in}{0.985338in}}%
\pgfpathlineto{\pgfqpoint{1.640672in}{0.937958in}}%
\pgfpathlineto{\pgfqpoint{1.679617in}{0.909961in}}%
\pgfpathlineto{\pgfqpoint{1.699089in}{0.897782in}}%
\pgfpathlineto{\pgfqpoint{1.718562in}{0.887116in}}%
\pgfpathlineto{\pgfqpoint{1.738034in}{0.878171in}}%
\pgfpathlineto{\pgfqpoint{1.757506in}{0.871140in}}%
\pgfpathlineto{\pgfqpoint{1.776979in}{0.866199in}}%
\pgfpathlineto{\pgfqpoint{1.796451in}{0.863505in}}%
\pgfpathlineto{\pgfqpoint{1.815923in}{0.863200in}}%
\pgfpathlineto{\pgfqpoint{1.835396in}{0.865405in}}%
\pgfpathlineto{\pgfqpoint{1.854868in}{0.870226in}}%
\pgfpathlineto{\pgfqpoint{1.874340in}{0.877746in}}%
\pgfpathlineto{\pgfqpoint{1.893813in}{0.888033in}}%
\pgfpathlineto{\pgfqpoint{1.913285in}{0.901133in}}%
\pgfpathlineto{\pgfqpoint{1.932758in}{0.917075in}}%
\pgfpathlineto{\pgfqpoint{1.952230in}{0.935870in}}%
\pgfpathlineto{\pgfqpoint{1.971702in}{0.957509in}}%
\pgfpathlineto{\pgfqpoint{1.991175in}{0.981967in}}%
\pgfpathlineto{\pgfqpoint{2.010647in}{1.009201in}}%
\pgfpathlineto{\pgfqpoint{2.030119in}{1.039151in}}%
\pgfpathlineto{\pgfqpoint{2.049592in}{1.071742in}}%
\pgfpathlineto{\pgfqpoint{2.069064in}{1.106882in}}%
\pgfpathlineto{\pgfqpoint{2.108009in}{1.184374in}}%
\pgfpathlineto{\pgfqpoint{2.146954in}{1.270617in}}%
\pgfpathlineto{\pgfqpoint{2.185898in}{1.364407in}}%
\pgfpathlineto{\pgfqpoint{2.224843in}{1.464377in}}%
\pgfpathlineto{\pgfqpoint{2.283260in}{1.622596in}}%
\pgfpathlineto{\pgfqpoint{2.439039in}{2.053391in}}%
\pgfpathlineto{\pgfqpoint{2.477984in}{2.153880in}}%
\pgfpathlineto{\pgfqpoint{2.516928in}{2.248192in}}%
\pgfpathlineto{\pgfqpoint{2.555873in}{2.334822in}}%
\pgfpathlineto{\pgfqpoint{2.594818in}{2.412386in}}%
\pgfpathlineto{\pgfqpoint{2.614290in}{2.447375in}}%
\pgfpathlineto{\pgfqpoint{2.633763in}{2.479645in}}%
\pgfpathlineto{\pgfqpoint{2.653235in}{2.509068in}}%
\pgfpathlineto{\pgfqpoint{2.672707in}{2.535524in}}%
\pgfpathlineto{\pgfqpoint{2.692180in}{2.558910in}}%
\pgfpathlineto{\pgfqpoint{2.711652in}{2.579131in}}%
\pgfpathlineto{\pgfqpoint{2.731124in}{2.596107in}}%
\pgfpathlineto{\pgfqpoint{2.750597in}{2.609770in}}%
\pgfpathlineto{\pgfqpoint{2.770069in}{2.620065in}}%
\pgfpathlineto{\pgfqpoint{2.789541in}{2.626952in}}%
\pgfpathlineto{\pgfqpoint{2.809014in}{2.630402in}}%
\pgfpathlineto{\pgfqpoint{2.828486in}{2.630402in}}%
\pgfpathlineto{\pgfqpoint{2.847959in}{2.626952in}}%
\pgfpathlineto{\pgfqpoint{2.867431in}{2.620065in}}%
\pgfpathlineto{\pgfqpoint{2.886903in}{2.609770in}}%
\pgfpathlineto{\pgfqpoint{2.906376in}{2.596107in}}%
\pgfpathlineto{\pgfqpoint{2.925848in}{2.579131in}}%
\pgfpathlineto{\pgfqpoint{2.945320in}{2.558910in}}%
\pgfpathlineto{\pgfqpoint{2.964793in}{2.535524in}}%
\pgfpathlineto{\pgfqpoint{2.984265in}{2.509068in}}%
\pgfpathlineto{\pgfqpoint{3.003737in}{2.479645in}}%
\pgfpathlineto{\pgfqpoint{3.023210in}{2.447375in}}%
\pgfpathlineto{\pgfqpoint{3.042682in}{2.412386in}}%
\pgfpathlineto{\pgfqpoint{3.062155in}{2.374819in}}%
\pgfpathlineto{\pgfqpoint{3.101099in}{2.292557in}}%
\pgfpathlineto{\pgfqpoint{3.140044in}{2.201905in}}%
\pgfpathlineto{\pgfqpoint{3.178989in}{2.104309in}}%
\pgfpathlineto{\pgfqpoint{3.217933in}{2.001328in}}%
\pgfpathlineto{\pgfqpoint{3.276351in}{1.840365in}}%
\pgfpathlineto{\pgfqpoint{3.393185in}{1.516213in}}%
\pgfpathlineto{\pgfqpoint{3.432129in}{1.413711in}}%
\pgfpathlineto{\pgfqpoint{3.471074in}{1.316651in}}%
\pgfpathlineto{\pgfqpoint{3.510019in}{1.226472in}}%
\pgfpathlineto{\pgfqpoint{3.548964in}{1.144466in}}%
\pgfpathlineto{\pgfqpoint{3.568436in}{1.106882in}}%
\pgfpathlineto{\pgfqpoint{3.587908in}{1.071742in}}%
\pgfpathlineto{\pgfqpoint{3.607381in}{1.039151in}}%
\pgfpathlineto{\pgfqpoint{3.626853in}{1.009201in}}%
\pgfpathlineto{\pgfqpoint{3.646325in}{0.981967in}}%
\pgfpathlineto{\pgfqpoint{3.665798in}{0.957509in}}%
\pgfpathlineto{\pgfqpoint{3.685270in}{0.935870in}}%
\pgfpathlineto{\pgfqpoint{3.704742in}{0.917075in}}%
\pgfpathlineto{\pgfqpoint{3.724215in}{0.901133in}}%
\pgfpathlineto{\pgfqpoint{3.743687in}{0.888033in}}%
\pgfpathlineto{\pgfqpoint{3.763160in}{0.877746in}}%
\pgfpathlineto{\pgfqpoint{3.782632in}{0.870226in}}%
\pgfpathlineto{\pgfqpoint{3.802104in}{0.865405in}}%
\pgfpathlineto{\pgfqpoint{3.821577in}{0.863200in}}%
\pgfpathlineto{\pgfqpoint{3.841049in}{0.863505in}}%
\pgfpathlineto{\pgfqpoint{3.860521in}{0.866199in}}%
\pgfpathlineto{\pgfqpoint{3.879994in}{0.871140in}}%
\pgfpathlineto{\pgfqpoint{3.899466in}{0.878171in}}%
\pgfpathlineto{\pgfqpoint{3.918938in}{0.887116in}}%
\pgfpathlineto{\pgfqpoint{3.938411in}{0.897782in}}%
\pgfpathlineto{\pgfqpoint{3.977356in}{0.923432in}}%
\pgfpathlineto{\pgfqpoint{4.016300in}{0.953291in}}%
\pgfpathlineto{\pgfqpoint{4.133134in}{1.047288in}}%
\pgfpathlineto{\pgfqpoint{4.152607in}{1.060702in}}%
\pgfpathlineto{\pgfqpoint{4.172079in}{1.072737in}}%
\pgfpathlineto{\pgfqpoint{4.191552in}{1.083131in}}%
\pgfpathlineto{\pgfqpoint{4.211024in}{1.091637in}}%
\pgfpathlineto{\pgfqpoint{4.230496in}{1.098023in}}%
\pgfpathlineto{\pgfqpoint{4.249969in}{1.102079in}}%
\pgfpathlineto{\pgfqpoint{4.269441in}{1.103619in}}%
\pgfpathlineto{\pgfqpoint{4.288913in}{1.102490in}}%
\pgfpathlineto{\pgfqpoint{4.308386in}{1.098570in}}%
\pgfpathlineto{\pgfqpoint{4.327858in}{1.091782in}}%
\pgfpathlineto{\pgfqpoint{4.347330in}{1.082092in}}%
\pgfpathlineto{\pgfqpoint{4.366803in}{1.069519in}}%
\pgfpathlineto{\pgfqpoint{4.386275in}{1.054139in}}%
\pgfpathlineto{\pgfqpoint{4.405747in}{1.036095in}}%
\pgfpathlineto{\pgfqpoint{4.425220in}{1.015601in}}%
\pgfpathlineto{\pgfqpoint{4.444692in}{0.992951in}}%
\pgfpathlineto{\pgfqpoint{4.483637in}{0.942802in}}%
\pgfpathlineto{\pgfqpoint{4.542054in}{0.864215in}}%
\pgfpathlineto{\pgfqpoint{4.561526in}{0.840279in}}%
\pgfpathlineto{\pgfqpoint{4.580999in}{0.819173in}}%
\pgfpathlineto{\pgfqpoint{4.600471in}{0.802144in}}%
\pgfpathlineto{\pgfqpoint{4.619943in}{0.790600in}}%
\pgfpathlineto{\pgfqpoint{4.639416in}{0.786130in}}%
\pgfpathlineto{\pgfqpoint{4.658888in}{0.790508in}}%
\pgfpathlineto{\pgfqpoint{4.678361in}{0.805713in}}%
\pgfpathlineto{\pgfqpoint{4.697833in}{0.833937in}}%
\pgfpathlineto{\pgfqpoint{4.717305in}{0.877602in}}%
\pgfpathlineto{\pgfqpoint{4.736778in}{0.939374in}}%
\pgfpathlineto{\pgfqpoint{4.756250in}{1.022176in}}%
\pgfpathlineto{\pgfqpoint{4.756250in}{1.022176in}}%
\pgfusepath{stroke}%
\end{pgfscope}%
\begin{pgfscope}%
\pgfpathrectangle{\pgfqpoint{0.687500in}{0.385000in}}{\pgfqpoint{4.262500in}{2.695000in}}%
\pgfusepath{clip}%
\pgfsetbuttcap%
\pgfsetroundjoin%
\definecolor{currentfill}{rgb}{0.549020,0.337255,0.294118}%
\pgfsetfillcolor{currentfill}%
\pgfsetlinewidth{1.003750pt}%
\definecolor{currentstroke}{rgb}{0.549020,0.337255,0.294118}%
\pgfsetstrokecolor{currentstroke}%
\pgfsetdash{}{0pt}%
\pgfsys@defobject{currentmarker}{\pgfqpoint{-0.020833in}{-0.020833in}}{\pgfqpoint{0.020833in}{0.020833in}}{%
\pgfpathmoveto{\pgfqpoint{0.000000in}{-0.020833in}}%
\pgfpathcurveto{\pgfqpoint{0.005525in}{-0.020833in}}{\pgfqpoint{0.010825in}{-0.018638in}}{\pgfqpoint{0.014731in}{-0.014731in}}%
\pgfpathcurveto{\pgfqpoint{0.018638in}{-0.010825in}}{\pgfqpoint{0.020833in}{-0.005525in}}{\pgfqpoint{0.020833in}{0.000000in}}%
\pgfpathcurveto{\pgfqpoint{0.020833in}{0.005525in}}{\pgfqpoint{0.018638in}{0.010825in}}{\pgfqpoint{0.014731in}{0.014731in}}%
\pgfpathcurveto{\pgfqpoint{0.010825in}{0.018638in}}{\pgfqpoint{0.005525in}{0.020833in}}{\pgfqpoint{0.000000in}{0.020833in}}%
\pgfpathcurveto{\pgfqpoint{-0.005525in}{0.020833in}}{\pgfqpoint{-0.010825in}{0.018638in}}{\pgfqpoint{-0.014731in}{0.014731in}}%
\pgfpathcurveto{\pgfqpoint{-0.018638in}{0.010825in}}{\pgfqpoint{-0.020833in}{0.005525in}}{\pgfqpoint{-0.020833in}{0.000000in}}%
\pgfpathcurveto{\pgfqpoint{-0.020833in}{-0.005525in}}{\pgfqpoint{-0.018638in}{-0.010825in}}{\pgfqpoint{-0.014731in}{-0.014731in}}%
\pgfpathcurveto{\pgfqpoint{-0.010825in}{-0.018638in}}{\pgfqpoint{-0.005525in}{-0.020833in}}{\pgfqpoint{0.000000in}{-0.020833in}}%
\pgfpathclose%
\pgfusepath{stroke,fill}%
}%
\begin{pgfscope}%
\pgfsys@transformshift{4.747985in}{0.903840in}%
\pgfsys@useobject{currentmarker}{}%
\end{pgfscope}%
\begin{pgfscope}%
\pgfsys@transformshift{4.682287in}{0.908632in}%
\pgfsys@useobject{currentmarker}{}%
\end{pgfscope}%
\begin{pgfscope}%
\pgfsys@transformshift{4.553129in}{0.919588in}%
\pgfsys@useobject{currentmarker}{}%
\end{pgfscope}%
\begin{pgfscope}%
\pgfsys@transformshift{4.364908in}{0.940348in}%
\pgfsys@useobject{currentmarker}{}%
\end{pgfscope}%
\begin{pgfscope}%
\pgfsys@transformshift{4.124035in}{0.979685in}%
\pgfsys@useobject{currentmarker}{}%
\end{pgfscope}%
\begin{pgfscope}%
\pgfsys@transformshift{3.838712in}{1.060782in}%
\pgfsys@useobject{currentmarker}{}%
\end{pgfscope}%
\begin{pgfscope}%
\pgfsys@transformshift{3.518656in}{1.255683in}%
\pgfsys@useobject{currentmarker}{}%
\end{pgfscope}%
\begin{pgfscope}%
\pgfsys@transformshift{3.174765in}{1.808447in}%
\pgfsys@useobject{currentmarker}{}%
\end{pgfscope}%
\begin{pgfscope}%
\pgfsys@transformshift{2.818750in}{2.630833in}%
\pgfsys@useobject{currentmarker}{}%
\end{pgfscope}%
\begin{pgfscope}%
\pgfsys@transformshift{2.462735in}{1.808447in}%
\pgfsys@useobject{currentmarker}{}%
\end{pgfscope}%
\begin{pgfscope}%
\pgfsys@transformshift{2.118844in}{1.255683in}%
\pgfsys@useobject{currentmarker}{}%
\end{pgfscope}%
\begin{pgfscope}%
\pgfsys@transformshift{1.798788in}{1.060782in}%
\pgfsys@useobject{currentmarker}{}%
\end{pgfscope}%
\begin{pgfscope}%
\pgfsys@transformshift{1.513465in}{0.979685in}%
\pgfsys@useobject{currentmarker}{}%
\end{pgfscope}%
\begin{pgfscope}%
\pgfsys@transformshift{1.272592in}{0.940348in}%
\pgfsys@useobject{currentmarker}{}%
\end{pgfscope}%
\begin{pgfscope}%
\pgfsys@transformshift{1.084371in}{0.919588in}%
\pgfsys@useobject{currentmarker}{}%
\end{pgfscope}%
\begin{pgfscope}%
\pgfsys@transformshift{0.955213in}{0.908632in}%
\pgfsys@useobject{currentmarker}{}%
\end{pgfscope}%
\begin{pgfscope}%
\pgfsys@transformshift{0.889515in}{0.903840in}%
\pgfsys@useobject{currentmarker}{}%
\end{pgfscope}%
\end{pgfscope}%
\begin{pgfscope}%
\pgfpathrectangle{\pgfqpoint{0.687500in}{0.385000in}}{\pgfqpoint{4.262500in}{2.695000in}}%
\pgfusepath{clip}%
\pgfsetrectcap%
\pgfsetroundjoin%
\pgfsetlinewidth{1.505625pt}%
\definecolor{currentstroke}{rgb}{0.890196,0.466667,0.760784}%
\pgfsetstrokecolor{currentstroke}%
\pgfsetdash{}{0pt}%
\pgfpathmoveto{\pgfqpoint{0.881250in}{0.926878in}}%
\pgfpathlineto{\pgfqpoint{0.900722in}{0.886857in}}%
\pgfpathlineto{\pgfqpoint{0.920195in}{0.882853in}}%
\pgfpathlineto{\pgfqpoint{0.939667in}{0.895236in}}%
\pgfpathlineto{\pgfqpoint{0.959139in}{0.911973in}}%
\pgfpathlineto{\pgfqpoint{0.978612in}{0.926400in}}%
\pgfpathlineto{\pgfqpoint{0.998084in}{0.935512in}}%
\pgfpathlineto{\pgfqpoint{1.017557in}{0.938668in}}%
\pgfpathlineto{\pgfqpoint{1.037029in}{0.936630in}}%
\pgfpathlineto{\pgfqpoint{1.056501in}{0.930878in}}%
\pgfpathlineto{\pgfqpoint{1.114918in}{0.907946in}}%
\pgfpathlineto{\pgfqpoint{1.134391in}{0.902918in}}%
\pgfpathlineto{\pgfqpoint{1.153863in}{0.900579in}}%
\pgfpathlineto{\pgfqpoint{1.173335in}{0.901198in}}%
\pgfpathlineto{\pgfqpoint{1.192808in}{0.904724in}}%
\pgfpathlineto{\pgfqpoint{1.212280in}{0.910847in}}%
\pgfpathlineto{\pgfqpoint{1.231753in}{0.919070in}}%
\pgfpathlineto{\pgfqpoint{1.270697in}{0.939305in}}%
\pgfpathlineto{\pgfqpoint{1.309642in}{0.960214in}}%
\pgfpathlineto{\pgfqpoint{1.329114in}{0.969462in}}%
\pgfpathlineto{\pgfqpoint{1.348587in}{0.977318in}}%
\pgfpathlineto{\pgfqpoint{1.368059in}{0.983496in}}%
\pgfpathlineto{\pgfqpoint{1.387531in}{0.987843in}}%
\pgfpathlineto{\pgfqpoint{1.407004in}{0.990333in}}%
\pgfpathlineto{\pgfqpoint{1.426476in}{0.991059in}}%
\pgfpathlineto{\pgfqpoint{1.445948in}{0.990219in}}%
\pgfpathlineto{\pgfqpoint{1.484893in}{0.985040in}}%
\pgfpathlineto{\pgfqpoint{1.562783in}{0.971466in}}%
\pgfpathlineto{\pgfqpoint{1.582255in}{0.969711in}}%
\pgfpathlineto{\pgfqpoint{1.601727in}{0.969291in}}%
\pgfpathlineto{\pgfqpoint{1.621200in}{0.970454in}}%
\pgfpathlineto{\pgfqpoint{1.640672in}{0.973379in}}%
\pgfpathlineto{\pgfqpoint{1.660144in}{0.978173in}}%
\pgfpathlineto{\pgfqpoint{1.679617in}{0.984872in}}%
\pgfpathlineto{\pgfqpoint{1.699089in}{0.993441in}}%
\pgfpathlineto{\pgfqpoint{1.718562in}{1.003778in}}%
\pgfpathlineto{\pgfqpoint{1.738034in}{1.015721in}}%
\pgfpathlineto{\pgfqpoint{1.776979in}{1.043545in}}%
\pgfpathlineto{\pgfqpoint{1.835396in}{1.091006in}}%
\pgfpathlineto{\pgfqpoint{1.893813in}{1.138376in}}%
\pgfpathlineto{\pgfqpoint{1.932758in}{1.166644in}}%
\pgfpathlineto{\pgfqpoint{1.971702in}{1.190934in}}%
\pgfpathlineto{\pgfqpoint{2.010647in}{1.211107in}}%
\pgfpathlineto{\pgfqpoint{2.049592in}{1.227986in}}%
\pgfpathlineto{\pgfqpoint{2.127481in}{1.259501in}}%
\pgfpathlineto{\pgfqpoint{2.146954in}{1.268854in}}%
\pgfpathlineto{\pgfqpoint{2.166426in}{1.279565in}}%
\pgfpathlineto{\pgfqpoint{2.185898in}{1.292031in}}%
\pgfpathlineto{\pgfqpoint{2.205371in}{1.306641in}}%
\pgfpathlineto{\pgfqpoint{2.224843in}{1.323763in}}%
\pgfpathlineto{\pgfqpoint{2.244315in}{1.343740in}}%
\pgfpathlineto{\pgfqpoint{2.263788in}{1.366870in}}%
\pgfpathlineto{\pgfqpoint{2.283260in}{1.393409in}}%
\pgfpathlineto{\pgfqpoint{2.302732in}{1.423553in}}%
\pgfpathlineto{\pgfqpoint{2.322205in}{1.457438in}}%
\pgfpathlineto{\pgfqpoint{2.341677in}{1.495129in}}%
\pgfpathlineto{\pgfqpoint{2.361149in}{1.536621in}}%
\pgfpathlineto{\pgfqpoint{2.380622in}{1.581833in}}%
\pgfpathlineto{\pgfqpoint{2.400094in}{1.630608in}}%
\pgfpathlineto{\pgfqpoint{2.439039in}{1.737839in}}%
\pgfpathlineto{\pgfqpoint{2.477984in}{1.855576in}}%
\pgfpathlineto{\pgfqpoint{2.536401in}{2.043362in}}%
\pgfpathlineto{\pgfqpoint{2.594818in}{2.230055in}}%
\pgfpathlineto{\pgfqpoint{2.633763in}{2.345029in}}%
\pgfpathlineto{\pgfqpoint{2.653235in}{2.397662in}}%
\pgfpathlineto{\pgfqpoint{2.672707in}{2.446215in}}%
\pgfpathlineto{\pgfqpoint{2.692180in}{2.490103in}}%
\pgfpathlineto{\pgfqpoint{2.711652in}{2.528790in}}%
\pgfpathlineto{\pgfqpoint{2.731124in}{2.561798in}}%
\pgfpathlineto{\pgfqpoint{2.750597in}{2.588717in}}%
\pgfpathlineto{\pgfqpoint{2.770069in}{2.609209in}}%
\pgfpathlineto{\pgfqpoint{2.789541in}{2.623015in}}%
\pgfpathlineto{\pgfqpoint{2.809014in}{2.629963in}}%
\pgfpathlineto{\pgfqpoint{2.828486in}{2.629963in}}%
\pgfpathlineto{\pgfqpoint{2.847959in}{2.623015in}}%
\pgfpathlineto{\pgfqpoint{2.867431in}{2.609209in}}%
\pgfpathlineto{\pgfqpoint{2.886903in}{2.588717in}}%
\pgfpathlineto{\pgfqpoint{2.906376in}{2.561798in}}%
\pgfpathlineto{\pgfqpoint{2.925848in}{2.528790in}}%
\pgfpathlineto{\pgfqpoint{2.945320in}{2.490103in}}%
\pgfpathlineto{\pgfqpoint{2.964793in}{2.446215in}}%
\pgfpathlineto{\pgfqpoint{2.984265in}{2.397662in}}%
\pgfpathlineto{\pgfqpoint{3.003737in}{2.345029in}}%
\pgfpathlineto{\pgfqpoint{3.042682in}{2.230055in}}%
\pgfpathlineto{\pgfqpoint{3.081627in}{2.106586in}}%
\pgfpathlineto{\pgfqpoint{3.159516in}{1.855576in}}%
\pgfpathlineto{\pgfqpoint{3.198461in}{1.737839in}}%
\pgfpathlineto{\pgfqpoint{3.237406in}{1.630608in}}%
\pgfpathlineto{\pgfqpoint{3.256878in}{1.581833in}}%
\pgfpathlineto{\pgfqpoint{3.276351in}{1.536621in}}%
\pgfpathlineto{\pgfqpoint{3.295823in}{1.495129in}}%
\pgfpathlineto{\pgfqpoint{3.315295in}{1.457438in}}%
\pgfpathlineto{\pgfqpoint{3.334768in}{1.423553in}}%
\pgfpathlineto{\pgfqpoint{3.354240in}{1.393409in}}%
\pgfpathlineto{\pgfqpoint{3.373712in}{1.366870in}}%
\pgfpathlineto{\pgfqpoint{3.393185in}{1.343740in}}%
\pgfpathlineto{\pgfqpoint{3.412657in}{1.323763in}}%
\pgfpathlineto{\pgfqpoint{3.432129in}{1.306641in}}%
\pgfpathlineto{\pgfqpoint{3.451602in}{1.292031in}}%
\pgfpathlineto{\pgfqpoint{3.471074in}{1.279565in}}%
\pgfpathlineto{\pgfqpoint{3.490546in}{1.268854in}}%
\pgfpathlineto{\pgfqpoint{3.529491in}{1.251109in}}%
\pgfpathlineto{\pgfqpoint{3.626853in}{1.211107in}}%
\pgfpathlineto{\pgfqpoint{3.665798in}{1.190934in}}%
\pgfpathlineto{\pgfqpoint{3.704742in}{1.166644in}}%
\pgfpathlineto{\pgfqpoint{3.743687in}{1.138376in}}%
\pgfpathlineto{\pgfqpoint{3.802104in}{1.091006in}}%
\pgfpathlineto{\pgfqpoint{3.860521in}{1.043545in}}%
\pgfpathlineto{\pgfqpoint{3.899466in}{1.015721in}}%
\pgfpathlineto{\pgfqpoint{3.918938in}{1.003778in}}%
\pgfpathlineto{\pgfqpoint{3.938411in}{0.993441in}}%
\pgfpathlineto{\pgfqpoint{3.957883in}{0.984872in}}%
\pgfpathlineto{\pgfqpoint{3.977356in}{0.978173in}}%
\pgfpathlineto{\pgfqpoint{3.996828in}{0.973379in}}%
\pgfpathlineto{\pgfqpoint{4.016300in}{0.970454in}}%
\pgfpathlineto{\pgfqpoint{4.035773in}{0.969291in}}%
\pgfpathlineto{\pgfqpoint{4.055245in}{0.969711in}}%
\pgfpathlineto{\pgfqpoint{4.094190in}{0.974250in}}%
\pgfpathlineto{\pgfqpoint{4.172079in}{0.988097in}}%
\pgfpathlineto{\pgfqpoint{4.191552in}{0.990219in}}%
\pgfpathlineto{\pgfqpoint{4.211024in}{0.991059in}}%
\pgfpathlineto{\pgfqpoint{4.230496in}{0.990333in}}%
\pgfpathlineto{\pgfqpoint{4.249969in}{0.987843in}}%
\pgfpathlineto{\pgfqpoint{4.269441in}{0.983496in}}%
\pgfpathlineto{\pgfqpoint{4.288913in}{0.977318in}}%
\pgfpathlineto{\pgfqpoint{4.308386in}{0.969462in}}%
\pgfpathlineto{\pgfqpoint{4.347330in}{0.949987in}}%
\pgfpathlineto{\pgfqpoint{4.405747in}{0.919070in}}%
\pgfpathlineto{\pgfqpoint{4.425220in}{0.910847in}}%
\pgfpathlineto{\pgfqpoint{4.444692in}{0.904724in}}%
\pgfpathlineto{\pgfqpoint{4.464165in}{0.901198in}}%
\pgfpathlineto{\pgfqpoint{4.483637in}{0.900579in}}%
\pgfpathlineto{\pgfqpoint{4.503109in}{0.902918in}}%
\pgfpathlineto{\pgfqpoint{4.522582in}{0.907946in}}%
\pgfpathlineto{\pgfqpoint{4.561526in}{0.923128in}}%
\pgfpathlineto{\pgfqpoint{4.580999in}{0.930878in}}%
\pgfpathlineto{\pgfqpoint{4.600471in}{0.936630in}}%
\pgfpathlineto{\pgfqpoint{4.619943in}{0.938668in}}%
\pgfpathlineto{\pgfqpoint{4.639416in}{0.935512in}}%
\pgfpathlineto{\pgfqpoint{4.658888in}{0.926400in}}%
\pgfpathlineto{\pgfqpoint{4.678361in}{0.911973in}}%
\pgfpathlineto{\pgfqpoint{4.697833in}{0.895236in}}%
\pgfpathlineto{\pgfqpoint{4.717305in}{0.882853in}}%
\pgfpathlineto{\pgfqpoint{4.736778in}{0.886857in}}%
\pgfpathlineto{\pgfqpoint{4.756250in}{0.926878in}}%
\pgfpathlineto{\pgfqpoint{4.756250in}{0.926878in}}%
\pgfusepath{stroke}%
\end{pgfscope}%
\begin{pgfscope}%
\pgfpathrectangle{\pgfqpoint{0.687500in}{0.385000in}}{\pgfqpoint{4.262500in}{2.695000in}}%
\pgfusepath{clip}%
\pgfsetbuttcap%
\pgfsetroundjoin%
\definecolor{currentfill}{rgb}{0.498039,0.498039,0.498039}%
\pgfsetfillcolor{currentfill}%
\pgfsetlinewidth{1.003750pt}%
\definecolor{currentstroke}{rgb}{0.498039,0.498039,0.498039}%
\pgfsetstrokecolor{currentstroke}%
\pgfsetdash{}{0pt}%
\pgfsys@defobject{currentmarker}{\pgfqpoint{-0.020833in}{-0.020833in}}{\pgfqpoint{0.020833in}{0.020833in}}{%
\pgfpathmoveto{\pgfqpoint{0.000000in}{-0.020833in}}%
\pgfpathcurveto{\pgfqpoint{0.005525in}{-0.020833in}}{\pgfqpoint{0.010825in}{-0.018638in}}{\pgfqpoint{0.014731in}{-0.014731in}}%
\pgfpathcurveto{\pgfqpoint{0.018638in}{-0.010825in}}{\pgfqpoint{0.020833in}{-0.005525in}}{\pgfqpoint{0.020833in}{0.000000in}}%
\pgfpathcurveto{\pgfqpoint{0.020833in}{0.005525in}}{\pgfqpoint{0.018638in}{0.010825in}}{\pgfqpoint{0.014731in}{0.014731in}}%
\pgfpathcurveto{\pgfqpoint{0.010825in}{0.018638in}}{\pgfqpoint{0.005525in}{0.020833in}}{\pgfqpoint{0.000000in}{0.020833in}}%
\pgfpathcurveto{\pgfqpoint{-0.005525in}{0.020833in}}{\pgfqpoint{-0.010825in}{0.018638in}}{\pgfqpoint{-0.014731in}{0.014731in}}%
\pgfpathcurveto{\pgfqpoint{-0.018638in}{0.010825in}}{\pgfqpoint{-0.020833in}{0.005525in}}{\pgfqpoint{-0.020833in}{0.000000in}}%
\pgfpathcurveto{\pgfqpoint{-0.020833in}{-0.005525in}}{\pgfqpoint{-0.018638in}{-0.010825in}}{\pgfqpoint{-0.014731in}{-0.014731in}}%
\pgfpathcurveto{\pgfqpoint{-0.010825in}{-0.018638in}}{\pgfqpoint{-0.005525in}{-0.020833in}}{\pgfqpoint{0.000000in}{-0.020833in}}%
\pgfpathclose%
\pgfusepath{stroke,fill}%
}%
\begin{pgfscope}%
\pgfsys@transformshift{4.754055in}{0.903420in}%
\pgfsys@useobject{currentmarker}{}%
\end{pgfscope}%
\begin{pgfscope}%
\pgfsys@transformshift{4.736529in}{0.904642in}%
\pgfsys@useobject{currentmarker}{}%
\end{pgfscope}%
\begin{pgfscope}%
\pgfsys@transformshift{4.701635in}{0.907171in}%
\pgfsys@useobject{currentmarker}{}%
\end{pgfscope}%
\begin{pgfscope}%
\pgfsys@transformshift{4.649689in}{0.911192in}%
\pgfsys@useobject{currentmarker}{}%
\end{pgfscope}%
\begin{pgfscope}%
\pgfsys@transformshift{4.581162in}{0.917017in}%
\pgfsys@useobject{currentmarker}{}%
\end{pgfscope}%
\begin{pgfscope}%
\pgfsys@transformshift{4.496674in}{0.925137in}%
\pgfsys@useobject{currentmarker}{}%
\end{pgfscope}%
\begin{pgfscope}%
\pgfsys@transformshift{4.396991in}{0.936318in}%
\pgfsys@useobject{currentmarker}{}%
\end{pgfscope}%
\begin{pgfscope}%
\pgfsys@transformshift{4.283015in}{0.951758in}%
\pgfsys@useobject{currentmarker}{}%
\end{pgfscope}%
\begin{pgfscope}%
\pgfsys@transformshift{4.155778in}{0.973387in}%
\pgfsys@useobject{currentmarker}{}%
\end{pgfscope}%
\begin{pgfscope}%
\pgfsys@transformshift{4.016433in}{1.004418in}%
\pgfsys@useobject{currentmarker}{}%
\end{pgfscope}%
\begin{pgfscope}%
\pgfsys@transformshift{3.866242in}{1.050442in}%
\pgfsys@useobject{currentmarker}{}%
\end{pgfscope}%
\begin{pgfscope}%
\pgfsys@transformshift{3.706564in}{1.121666in}%
\pgfsys@useobject{currentmarker}{}%
\end{pgfscope}%
\begin{pgfscope}%
\pgfsys@transformshift{3.538846in}{1.237611in}%
\pgfsys@useobject{currentmarker}{}%
\end{pgfscope}%
\begin{pgfscope}%
\pgfsys@transformshift{3.364607in}{1.436200in}%
\pgfsys@useobject{currentmarker}{}%
\end{pgfscope}%
\begin{pgfscope}%
\pgfsys@transformshift{3.185424in}{1.782075in}%
\pgfsys@useobject{currentmarker}{}%
\end{pgfscope}%
\begin{pgfscope}%
\pgfsys@transformshift{3.002921in}{2.299767in}%
\pgfsys@useobject{currentmarker}{}%
\end{pgfscope}%
\begin{pgfscope}%
\pgfsys@transformshift{2.818750in}{2.630833in}%
\pgfsys@useobject{currentmarker}{}%
\end{pgfscope}%
\begin{pgfscope}%
\pgfsys@transformshift{2.634579in}{2.299767in}%
\pgfsys@useobject{currentmarker}{}%
\end{pgfscope}%
\begin{pgfscope}%
\pgfsys@transformshift{2.452076in}{1.782075in}%
\pgfsys@useobject{currentmarker}{}%
\end{pgfscope}%
\begin{pgfscope}%
\pgfsys@transformshift{2.272893in}{1.436200in}%
\pgfsys@useobject{currentmarker}{}%
\end{pgfscope}%
\begin{pgfscope}%
\pgfsys@transformshift{2.098654in}{1.237611in}%
\pgfsys@useobject{currentmarker}{}%
\end{pgfscope}%
\begin{pgfscope}%
\pgfsys@transformshift{1.930936in}{1.121666in}%
\pgfsys@useobject{currentmarker}{}%
\end{pgfscope}%
\begin{pgfscope}%
\pgfsys@transformshift{1.771258in}{1.050442in}%
\pgfsys@useobject{currentmarker}{}%
\end{pgfscope}%
\begin{pgfscope}%
\pgfsys@transformshift{1.621067in}{1.004418in}%
\pgfsys@useobject{currentmarker}{}%
\end{pgfscope}%
\begin{pgfscope}%
\pgfsys@transformshift{1.481722in}{0.973387in}%
\pgfsys@useobject{currentmarker}{}%
\end{pgfscope}%
\begin{pgfscope}%
\pgfsys@transformshift{1.354485in}{0.951758in}%
\pgfsys@useobject{currentmarker}{}%
\end{pgfscope}%
\begin{pgfscope}%
\pgfsys@transformshift{1.240509in}{0.936318in}%
\pgfsys@useobject{currentmarker}{}%
\end{pgfscope}%
\begin{pgfscope}%
\pgfsys@transformshift{1.140826in}{0.925137in}%
\pgfsys@useobject{currentmarker}{}%
\end{pgfscope}%
\begin{pgfscope}%
\pgfsys@transformshift{1.056338in}{0.917017in}%
\pgfsys@useobject{currentmarker}{}%
\end{pgfscope}%
\begin{pgfscope}%
\pgfsys@transformshift{0.987811in}{0.911192in}%
\pgfsys@useobject{currentmarker}{}%
\end{pgfscope}%
\begin{pgfscope}%
\pgfsys@transformshift{0.935865in}{0.907171in}%
\pgfsys@useobject{currentmarker}{}%
\end{pgfscope}%
\begin{pgfscope}%
\pgfsys@transformshift{0.900971in}{0.904642in}%
\pgfsys@useobject{currentmarker}{}%
\end{pgfscope}%
\begin{pgfscope}%
\pgfsys@transformshift{0.883445in}{0.903420in}%
\pgfsys@useobject{currentmarker}{}%
\end{pgfscope}%
\end{pgfscope}%
\begin{pgfscope}%
\pgfpathrectangle{\pgfqpoint{0.687500in}{0.385000in}}{\pgfqpoint{4.262500in}{2.695000in}}%
\pgfusepath{clip}%
\pgfsetrectcap%
\pgfsetroundjoin%
\pgfsetlinewidth{1.505625pt}%
\definecolor{currentstroke}{rgb}{0.737255,0.741176,0.133333}%
\pgfsetstrokecolor{currentstroke}%
\pgfsetdash{}{0pt}%
\pgfpathmoveto{\pgfqpoint{0.881250in}{0.904251in}}%
\pgfpathlineto{\pgfqpoint{0.900722in}{0.904595in}}%
\pgfpathlineto{\pgfqpoint{0.920195in}{0.906960in}}%
\pgfpathlineto{\pgfqpoint{0.959139in}{0.907915in}}%
\pgfpathlineto{\pgfqpoint{1.037029in}{0.916082in}}%
\pgfpathlineto{\pgfqpoint{1.095446in}{0.919545in}}%
\pgfpathlineto{\pgfqpoint{1.153863in}{0.926973in}}%
\pgfpathlineto{\pgfqpoint{1.212280in}{0.933824in}}%
\pgfpathlineto{\pgfqpoint{1.290170in}{0.941454in}}%
\pgfpathlineto{\pgfqpoint{1.329114in}{0.947268in}}%
\pgfpathlineto{\pgfqpoint{1.562783in}{0.988946in}}%
\pgfpathlineto{\pgfqpoint{1.621200in}{1.004457in}}%
\pgfpathlineto{\pgfqpoint{1.738034in}{1.039856in}}%
\pgfpathlineto{\pgfqpoint{1.796451in}{1.059055in}}%
\pgfpathlineto{\pgfqpoint{1.835396in}{1.073989in}}%
\pgfpathlineto{\pgfqpoint{1.874340in}{1.091464in}}%
\pgfpathlineto{\pgfqpoint{1.913285in}{1.111670in}}%
\pgfpathlineto{\pgfqpoint{1.952230in}{1.134338in}}%
\pgfpathlineto{\pgfqpoint{1.991175in}{1.159028in}}%
\pgfpathlineto{\pgfqpoint{2.030119in}{1.185509in}}%
\pgfpathlineto{\pgfqpoint{2.069064in}{1.214052in}}%
\pgfpathlineto{\pgfqpoint{2.108009in}{1.245485in}}%
\pgfpathlineto{\pgfqpoint{2.146954in}{1.281000in}}%
\pgfpathlineto{\pgfqpoint{2.185898in}{1.321818in}}%
\pgfpathlineto{\pgfqpoint{2.224843in}{1.368872in}}%
\pgfpathlineto{\pgfqpoint{2.263788in}{1.422642in}}%
\pgfpathlineto{\pgfqpoint{2.302732in}{1.483244in}}%
\pgfpathlineto{\pgfqpoint{2.341677in}{1.550677in}}%
\pgfpathlineto{\pgfqpoint{2.380622in}{1.625139in}}%
\pgfpathlineto{\pgfqpoint{2.419567in}{1.707188in}}%
\pgfpathlineto{\pgfqpoint{2.458511in}{1.797633in}}%
\pgfpathlineto{\pgfqpoint{2.497456in}{1.897096in}}%
\pgfpathlineto{\pgfqpoint{2.536401in}{2.005339in}}%
\pgfpathlineto{\pgfqpoint{2.575345in}{2.120552in}}%
\pgfpathlineto{\pgfqpoint{2.653235in}{2.354196in}}%
\pgfpathlineto{\pgfqpoint{2.672707in}{2.408394in}}%
\pgfpathlineto{\pgfqpoint{2.692180in}{2.458882in}}%
\pgfpathlineto{\pgfqpoint{2.711652in}{2.504599in}}%
\pgfpathlineto{\pgfqpoint{2.731124in}{2.544525in}}%
\pgfpathlineto{\pgfqpoint{2.750597in}{2.577722in}}%
\pgfpathlineto{\pgfqpoint{2.770069in}{2.603384in}}%
\pgfpathlineto{\pgfqpoint{2.789541in}{2.620866in}}%
\pgfpathlineto{\pgfqpoint{2.809014in}{2.629721in}}%
\pgfpathlineto{\pgfqpoint{2.828486in}{2.629721in}}%
\pgfpathlineto{\pgfqpoint{2.847959in}{2.620866in}}%
\pgfpathlineto{\pgfqpoint{2.867431in}{2.603384in}}%
\pgfpathlineto{\pgfqpoint{2.886903in}{2.577722in}}%
\pgfpathlineto{\pgfqpoint{2.906376in}{2.544525in}}%
\pgfpathlineto{\pgfqpoint{2.925848in}{2.504599in}}%
\pgfpathlineto{\pgfqpoint{2.945320in}{2.458882in}}%
\pgfpathlineto{\pgfqpoint{2.964793in}{2.408394in}}%
\pgfpathlineto{\pgfqpoint{3.003737in}{2.297344in}}%
\pgfpathlineto{\pgfqpoint{3.101099in}{2.005339in}}%
\pgfpathlineto{\pgfqpoint{3.140044in}{1.897096in}}%
\pgfpathlineto{\pgfqpoint{3.178989in}{1.797633in}}%
\pgfpathlineto{\pgfqpoint{3.217933in}{1.707188in}}%
\pgfpathlineto{\pgfqpoint{3.256878in}{1.625139in}}%
\pgfpathlineto{\pgfqpoint{3.295823in}{1.550677in}}%
\pgfpathlineto{\pgfqpoint{3.334768in}{1.483244in}}%
\pgfpathlineto{\pgfqpoint{3.373712in}{1.422642in}}%
\pgfpathlineto{\pgfqpoint{3.412657in}{1.368872in}}%
\pgfpathlineto{\pgfqpoint{3.451602in}{1.321818in}}%
\pgfpathlineto{\pgfqpoint{3.490546in}{1.281000in}}%
\pgfpathlineto{\pgfqpoint{3.529491in}{1.245485in}}%
\pgfpathlineto{\pgfqpoint{3.568436in}{1.214052in}}%
\pgfpathlineto{\pgfqpoint{3.607381in}{1.185509in}}%
\pgfpathlineto{\pgfqpoint{3.665798in}{1.146454in}}%
\pgfpathlineto{\pgfqpoint{3.704742in}{1.122725in}}%
\pgfpathlineto{\pgfqpoint{3.743687in}{1.101234in}}%
\pgfpathlineto{\pgfqpoint{3.782632in}{1.082385in}}%
\pgfpathlineto{\pgfqpoint{3.821577in}{1.066237in}}%
\pgfpathlineto{\pgfqpoint{3.860521in}{1.052342in}}%
\pgfpathlineto{\pgfqpoint{3.938411in}{1.027918in}}%
\pgfpathlineto{\pgfqpoint{4.035773in}{0.998941in}}%
\pgfpathlineto{\pgfqpoint{4.074717in}{0.988946in}}%
\pgfpathlineto{\pgfqpoint{4.113662in}{0.980671in}}%
\pgfpathlineto{\pgfqpoint{4.172079in}{0.970814in}}%
\pgfpathlineto{\pgfqpoint{4.249969in}{0.957840in}}%
\pgfpathlineto{\pgfqpoint{4.327858in}{0.944157in}}%
\pgfpathlineto{\pgfqpoint{4.366803in}{0.939184in}}%
\pgfpathlineto{\pgfqpoint{4.483637in}{0.926973in}}%
\pgfpathlineto{\pgfqpoint{4.542054in}{0.919545in}}%
\pgfpathlineto{\pgfqpoint{4.600471in}{0.916082in}}%
\pgfpathlineto{\pgfqpoint{4.639416in}{0.912545in}}%
\pgfpathlineto{\pgfqpoint{4.678361in}{0.907915in}}%
\pgfpathlineto{\pgfqpoint{4.717305in}{0.906960in}}%
\pgfpathlineto{\pgfqpoint{4.736778in}{0.904595in}}%
\pgfpathlineto{\pgfqpoint{4.756250in}{0.904251in}}%
\pgfpathlineto{\pgfqpoint{4.756250in}{0.904251in}}%
\pgfusepath{stroke}%
\end{pgfscope}%
\begin{pgfscope}%
\pgfsetrectcap%
\pgfsetmiterjoin%
\pgfsetlinewidth{0.803000pt}%
\definecolor{currentstroke}{rgb}{0.000000,0.000000,0.000000}%
\pgfsetstrokecolor{currentstroke}%
\pgfsetdash{}{0pt}%
\pgfpathmoveto{\pgfqpoint{0.687500in}{0.385000in}}%
\pgfpathlineto{\pgfqpoint{0.687500in}{3.080000in}}%
\pgfusepath{stroke}%
\end{pgfscope}%
\begin{pgfscope}%
\pgfsetrectcap%
\pgfsetmiterjoin%
\pgfsetlinewidth{0.803000pt}%
\definecolor{currentstroke}{rgb}{0.000000,0.000000,0.000000}%
\pgfsetstrokecolor{currentstroke}%
\pgfsetdash{}{0pt}%
\pgfpathmoveto{\pgfqpoint{4.950000in}{0.385000in}}%
\pgfpathlineto{\pgfqpoint{4.950000in}{3.080000in}}%
\pgfusepath{stroke}%
\end{pgfscope}%
\begin{pgfscope}%
\pgfsetrectcap%
\pgfsetmiterjoin%
\pgfsetlinewidth{0.803000pt}%
\definecolor{currentstroke}{rgb}{0.000000,0.000000,0.000000}%
\pgfsetstrokecolor{currentstroke}%
\pgfsetdash{}{0pt}%
\pgfpathmoveto{\pgfqpoint{0.687500in}{0.385000in}}%
\pgfpathlineto{\pgfqpoint{4.950000in}{0.385000in}}%
\pgfusepath{stroke}%
\end{pgfscope}%
\begin{pgfscope}%
\pgfsetrectcap%
\pgfsetmiterjoin%
\pgfsetlinewidth{0.803000pt}%
\definecolor{currentstroke}{rgb}{0.000000,0.000000,0.000000}%
\pgfsetstrokecolor{currentstroke}%
\pgfsetdash{}{0pt}%
\pgfpathmoveto{\pgfqpoint{0.687500in}{3.080000in}}%
\pgfpathlineto{\pgfqpoint{4.950000in}{3.080000in}}%
\pgfusepath{stroke}%
\end{pgfscope}%
\begin{pgfscope}%
\definecolor{textcolor}{rgb}{0.000000,0.000000,0.000000}%
\pgfsetstrokecolor{textcolor}%
\pgfsetfillcolor{textcolor}%
\pgftext[x=2.818750in,y=3.163333in,,base]{\color{textcolor}\rmfamily\fontsize{12.000000}{14.400000}\selectfont N=6, 8, 16, 32}%
\end{pgfscope}%
\begin{pgfscope}%
\pgfsetbuttcap%
\pgfsetmiterjoin%
\definecolor{currentfill}{rgb}{1.000000,1.000000,1.000000}%
\pgfsetfillcolor{currentfill}%
\pgfsetfillopacity{0.800000}%
\pgfsetlinewidth{1.003750pt}%
\definecolor{currentstroke}{rgb}{0.800000,0.800000,0.800000}%
\pgfsetstrokecolor{currentstroke}%
\pgfsetstrokeopacity{0.800000}%
\pgfsetdash{}{0pt}%
\pgfpathmoveto{\pgfqpoint{3.448142in}{1.775801in}}%
\pgfpathlineto{\pgfqpoint{4.852778in}{1.775801in}}%
\pgfpathquadraticcurveto{\pgfqpoint{4.880556in}{1.775801in}}{\pgfqpoint{4.880556in}{1.803578in}}%
\pgfpathlineto{\pgfqpoint{4.880556in}{2.982778in}}%
\pgfpathquadraticcurveto{\pgfqpoint{4.880556in}{3.010556in}}{\pgfqpoint{4.852778in}{3.010556in}}%
\pgfpathlineto{\pgfqpoint{3.448142in}{3.010556in}}%
\pgfpathquadraticcurveto{\pgfqpoint{3.420364in}{3.010556in}}{\pgfqpoint{3.420364in}{2.982778in}}%
\pgfpathlineto{\pgfqpoint{3.420364in}{1.803578in}}%
\pgfpathquadraticcurveto{\pgfqpoint{3.420364in}{1.775801in}}{\pgfqpoint{3.448142in}{1.775801in}}%
\pgfpathclose%
\pgfusepath{stroke,fill}%
\end{pgfscope}%
\begin{pgfscope}%
\pgfsetrectcap%
\pgfsetroundjoin%
\pgfsetlinewidth{1.505625pt}%
\definecolor{currentstroke}{rgb}{0.121569,0.466667,0.705882}%
\pgfsetstrokecolor{currentstroke}%
\pgfsetdash{}{0pt}%
\pgfpathmoveto{\pgfqpoint{3.475920in}{2.820146in}}%
\pgfpathlineto{\pgfqpoint{3.753698in}{2.820146in}}%
\pgfusepath{stroke}%
\end{pgfscope}%
\begin{pgfscope}%
\definecolor{textcolor}{rgb}{0.000000,0.000000,0.000000}%
\pgfsetstrokecolor{textcolor}%
\pgfsetfillcolor{textcolor}%
\pgftext[x=3.864809in,y=2.771534in,left,base]{\color{textcolor}\rmfamily\fontsize{10.000000}{12.000000}\selectfont \(\displaystyle y(x)=\)\(\displaystyle \frac{1}{1+25x^{2}}\)}%
\end{pgfscope}%
\begin{pgfscope}%
\pgfsetrectcap%
\pgfsetroundjoin%
\pgfsetlinewidth{1.505625pt}%
\definecolor{currentstroke}{rgb}{0.172549,0.627451,0.172549}%
\pgfsetstrokecolor{currentstroke}%
\pgfsetdash{}{0pt}%
\pgfpathmoveto{\pgfqpoint{3.475920in}{2.539689in}}%
\pgfpathlineto{\pgfqpoint{3.753698in}{2.539689in}}%
\pgfusepath{stroke}%
\end{pgfscope}%
\begin{pgfscope}%
\definecolor{textcolor}{rgb}{0.000000,0.000000,0.000000}%
\pgfsetstrokecolor{textcolor}%
\pgfsetfillcolor{textcolor}%
\pgftext[x=3.864809in,y=2.491078in,left,base]{\color{textcolor}\rmfamily\fontsize{10.000000}{12.000000}\selectfont W6(x)}%
\end{pgfscope}%
\begin{pgfscope}%
\pgfsetrectcap%
\pgfsetroundjoin%
\pgfsetlinewidth{1.505625pt}%
\definecolor{currentstroke}{rgb}{0.580392,0.403922,0.741176}%
\pgfsetstrokecolor{currentstroke}%
\pgfsetdash{}{0pt}%
\pgfpathmoveto{\pgfqpoint{3.475920in}{2.331356in}}%
\pgfpathlineto{\pgfqpoint{3.753698in}{2.331356in}}%
\pgfusepath{stroke}%
\end{pgfscope}%
\begin{pgfscope}%
\definecolor{textcolor}{rgb}{0.000000,0.000000,0.000000}%
\pgfsetstrokecolor{textcolor}%
\pgfsetfillcolor{textcolor}%
\pgftext[x=3.864809in,y=2.282745in,left,base]{\color{textcolor}\rmfamily\fontsize{10.000000}{12.000000}\selectfont W8(x)}%
\end{pgfscope}%
\begin{pgfscope}%
\pgfsetrectcap%
\pgfsetroundjoin%
\pgfsetlinewidth{1.505625pt}%
\definecolor{currentstroke}{rgb}{0.890196,0.466667,0.760784}%
\pgfsetstrokecolor{currentstroke}%
\pgfsetdash{}{0pt}%
\pgfpathmoveto{\pgfqpoint{3.475920in}{2.123023in}}%
\pgfpathlineto{\pgfqpoint{3.753698in}{2.123023in}}%
\pgfusepath{stroke}%
\end{pgfscope}%
\begin{pgfscope}%
\definecolor{textcolor}{rgb}{0.000000,0.000000,0.000000}%
\pgfsetstrokecolor{textcolor}%
\pgfsetfillcolor{textcolor}%
\pgftext[x=3.864809in,y=2.074412in,left,base]{\color{textcolor}\rmfamily\fontsize{10.000000}{12.000000}\selectfont W16(x)}%
\end{pgfscope}%
\begin{pgfscope}%
\pgfsetrectcap%
\pgfsetroundjoin%
\pgfsetlinewidth{1.505625pt}%
\definecolor{currentstroke}{rgb}{0.737255,0.741176,0.133333}%
\pgfsetstrokecolor{currentstroke}%
\pgfsetdash{}{0pt}%
\pgfpathmoveto{\pgfqpoint{3.475920in}{1.914689in}}%
\pgfpathlineto{\pgfqpoint{3.753698in}{1.914689in}}%
\pgfusepath{stroke}%
\end{pgfscope}%
\begin{pgfscope}%
\definecolor{textcolor}{rgb}{0.000000,0.000000,0.000000}%
\pgfsetstrokecolor{textcolor}%
\pgfsetfillcolor{textcolor}%
\pgftext[x=3.864809in,y=1.866078in,left,base]{\color{textcolor}\rmfamily\fontsize{10.000000}{12.000000}\selectfont W32(x)}%
\end{pgfscope}%
\end{pgfpicture}%
\makeatother%
\endgroup%
        
    \end{center}
    \caption{Węzły cosinus, funkcja \(y\), \(N=6,8,16,32\)}
\end{figure}

\begin{figure}[h]
    \begin{center}
        %% Creator: Matplotlib, PGF backend
%%
%% To include the figure in your LaTeX document, write
%%   \input{<filename>.pgf}
%%
%% Make sure the required packages are loaded in your preamble
%%   \usepackage{pgf}
%%
%% Figures using additional raster images can only be included by \input if
%% they are in the same directory as the main LaTeX file. For loading figures
%% from other directories you can use the `import` package
%%   \usepackage{import}
%% and then include the figures with
%%   \import{<path to file>}{<filename>.pgf}
%%
%% Matplotlib used the following preamble
%%
\begingroup%
\makeatletter%
\begin{pgfpicture}%
\pgfpathrectangle{\pgfpointorigin}{\pgfqpoint{6.400000in}{4.800000in}}%
\pgfusepath{use as bounding box, clip}%
\begin{pgfscope}%
\pgfsetbuttcap%
\pgfsetmiterjoin%
\definecolor{currentfill}{rgb}{1.000000,1.000000,1.000000}%
\pgfsetfillcolor{currentfill}%
\pgfsetlinewidth{0.000000pt}%
\definecolor{currentstroke}{rgb}{1.000000,1.000000,1.000000}%
\pgfsetstrokecolor{currentstroke}%
\pgfsetdash{}{0pt}%
\pgfpathmoveto{\pgfqpoint{0.000000in}{0.000000in}}%
\pgfpathlineto{\pgfqpoint{6.400000in}{0.000000in}}%
\pgfpathlineto{\pgfqpoint{6.400000in}{4.800000in}}%
\pgfpathlineto{\pgfqpoint{0.000000in}{4.800000in}}%
\pgfpathclose%
\pgfusepath{fill}%
\end{pgfscope}%
\begin{pgfscope}%
\pgfsetbuttcap%
\pgfsetmiterjoin%
\definecolor{currentfill}{rgb}{1.000000,1.000000,1.000000}%
\pgfsetfillcolor{currentfill}%
\pgfsetlinewidth{0.000000pt}%
\definecolor{currentstroke}{rgb}{0.000000,0.000000,0.000000}%
\pgfsetstrokecolor{currentstroke}%
\pgfsetstrokeopacity{0.000000}%
\pgfsetdash{}{0pt}%
\pgfpathmoveto{\pgfqpoint{0.800000in}{0.528000in}}%
\pgfpathlineto{\pgfqpoint{5.760000in}{0.528000in}}%
\pgfpathlineto{\pgfqpoint{5.760000in}{4.224000in}}%
\pgfpathlineto{\pgfqpoint{0.800000in}{4.224000in}}%
\pgfpathclose%
\pgfusepath{fill}%
\end{pgfscope}%
\begin{pgfscope}%
\pgfsetbuttcap%
\pgfsetroundjoin%
\definecolor{currentfill}{rgb}{0.000000,0.000000,0.000000}%
\pgfsetfillcolor{currentfill}%
\pgfsetlinewidth{0.803000pt}%
\definecolor{currentstroke}{rgb}{0.000000,0.000000,0.000000}%
\pgfsetstrokecolor{currentstroke}%
\pgfsetdash{}{0pt}%
\pgfsys@defobject{currentmarker}{\pgfqpoint{0.000000in}{-0.048611in}}{\pgfqpoint{0.000000in}{0.000000in}}{%
\pgfpathmoveto{\pgfqpoint{0.000000in}{0.000000in}}%
\pgfpathlineto{\pgfqpoint{0.000000in}{-0.048611in}}%
\pgfusepath{stroke,fill}%
}%
\begin{pgfscope}%
\pgfsys@transformshift{1.025455in}{0.528000in}%
\pgfsys@useobject{currentmarker}{}%
\end{pgfscope}%
\end{pgfscope}%
\begin{pgfscope}%
\definecolor{textcolor}{rgb}{0.000000,0.000000,0.000000}%
\pgfsetstrokecolor{textcolor}%
\pgfsetfillcolor{textcolor}%
\pgftext[x=1.025455in,y=0.430778in,,top]{\color{textcolor}\rmfamily\fontsize{10.000000}{12.000000}\selectfont \(\displaystyle -1.00\)}%
\end{pgfscope}%
\begin{pgfscope}%
\pgfsetbuttcap%
\pgfsetroundjoin%
\definecolor{currentfill}{rgb}{0.000000,0.000000,0.000000}%
\pgfsetfillcolor{currentfill}%
\pgfsetlinewidth{0.803000pt}%
\definecolor{currentstroke}{rgb}{0.000000,0.000000,0.000000}%
\pgfsetstrokecolor{currentstroke}%
\pgfsetdash{}{0pt}%
\pgfsys@defobject{currentmarker}{\pgfqpoint{0.000000in}{-0.048611in}}{\pgfqpoint{0.000000in}{0.000000in}}{%
\pgfpathmoveto{\pgfqpoint{0.000000in}{0.000000in}}%
\pgfpathlineto{\pgfqpoint{0.000000in}{-0.048611in}}%
\pgfusepath{stroke,fill}%
}%
\begin{pgfscope}%
\pgfsys@transformshift{1.589091in}{0.528000in}%
\pgfsys@useobject{currentmarker}{}%
\end{pgfscope}%
\end{pgfscope}%
\begin{pgfscope}%
\definecolor{textcolor}{rgb}{0.000000,0.000000,0.000000}%
\pgfsetstrokecolor{textcolor}%
\pgfsetfillcolor{textcolor}%
\pgftext[x=1.589091in,y=0.430778in,,top]{\color{textcolor}\rmfamily\fontsize{10.000000}{12.000000}\selectfont \(\displaystyle -0.75\)}%
\end{pgfscope}%
\begin{pgfscope}%
\pgfsetbuttcap%
\pgfsetroundjoin%
\definecolor{currentfill}{rgb}{0.000000,0.000000,0.000000}%
\pgfsetfillcolor{currentfill}%
\pgfsetlinewidth{0.803000pt}%
\definecolor{currentstroke}{rgb}{0.000000,0.000000,0.000000}%
\pgfsetstrokecolor{currentstroke}%
\pgfsetdash{}{0pt}%
\pgfsys@defobject{currentmarker}{\pgfqpoint{0.000000in}{-0.048611in}}{\pgfqpoint{0.000000in}{0.000000in}}{%
\pgfpathmoveto{\pgfqpoint{0.000000in}{0.000000in}}%
\pgfpathlineto{\pgfqpoint{0.000000in}{-0.048611in}}%
\pgfusepath{stroke,fill}%
}%
\begin{pgfscope}%
\pgfsys@transformshift{2.152727in}{0.528000in}%
\pgfsys@useobject{currentmarker}{}%
\end{pgfscope}%
\end{pgfscope}%
\begin{pgfscope}%
\definecolor{textcolor}{rgb}{0.000000,0.000000,0.000000}%
\pgfsetstrokecolor{textcolor}%
\pgfsetfillcolor{textcolor}%
\pgftext[x=2.152727in,y=0.430778in,,top]{\color{textcolor}\rmfamily\fontsize{10.000000}{12.000000}\selectfont \(\displaystyle -0.50\)}%
\end{pgfscope}%
\begin{pgfscope}%
\pgfsetbuttcap%
\pgfsetroundjoin%
\definecolor{currentfill}{rgb}{0.000000,0.000000,0.000000}%
\pgfsetfillcolor{currentfill}%
\pgfsetlinewidth{0.803000pt}%
\definecolor{currentstroke}{rgb}{0.000000,0.000000,0.000000}%
\pgfsetstrokecolor{currentstroke}%
\pgfsetdash{}{0pt}%
\pgfsys@defobject{currentmarker}{\pgfqpoint{0.000000in}{-0.048611in}}{\pgfqpoint{0.000000in}{0.000000in}}{%
\pgfpathmoveto{\pgfqpoint{0.000000in}{0.000000in}}%
\pgfpathlineto{\pgfqpoint{0.000000in}{-0.048611in}}%
\pgfusepath{stroke,fill}%
}%
\begin{pgfscope}%
\pgfsys@transformshift{2.716364in}{0.528000in}%
\pgfsys@useobject{currentmarker}{}%
\end{pgfscope}%
\end{pgfscope}%
\begin{pgfscope}%
\definecolor{textcolor}{rgb}{0.000000,0.000000,0.000000}%
\pgfsetstrokecolor{textcolor}%
\pgfsetfillcolor{textcolor}%
\pgftext[x=2.716364in,y=0.430778in,,top]{\color{textcolor}\rmfamily\fontsize{10.000000}{12.000000}\selectfont \(\displaystyle -0.25\)}%
\end{pgfscope}%
\begin{pgfscope}%
\pgfsetbuttcap%
\pgfsetroundjoin%
\definecolor{currentfill}{rgb}{0.000000,0.000000,0.000000}%
\pgfsetfillcolor{currentfill}%
\pgfsetlinewidth{0.803000pt}%
\definecolor{currentstroke}{rgb}{0.000000,0.000000,0.000000}%
\pgfsetstrokecolor{currentstroke}%
\pgfsetdash{}{0pt}%
\pgfsys@defobject{currentmarker}{\pgfqpoint{0.000000in}{-0.048611in}}{\pgfqpoint{0.000000in}{0.000000in}}{%
\pgfpathmoveto{\pgfqpoint{0.000000in}{0.000000in}}%
\pgfpathlineto{\pgfqpoint{0.000000in}{-0.048611in}}%
\pgfusepath{stroke,fill}%
}%
\begin{pgfscope}%
\pgfsys@transformshift{3.280000in}{0.528000in}%
\pgfsys@useobject{currentmarker}{}%
\end{pgfscope}%
\end{pgfscope}%
\begin{pgfscope}%
\definecolor{textcolor}{rgb}{0.000000,0.000000,0.000000}%
\pgfsetstrokecolor{textcolor}%
\pgfsetfillcolor{textcolor}%
\pgftext[x=3.280000in,y=0.430778in,,top]{\color{textcolor}\rmfamily\fontsize{10.000000}{12.000000}\selectfont \(\displaystyle 0.00\)}%
\end{pgfscope}%
\begin{pgfscope}%
\pgfsetbuttcap%
\pgfsetroundjoin%
\definecolor{currentfill}{rgb}{0.000000,0.000000,0.000000}%
\pgfsetfillcolor{currentfill}%
\pgfsetlinewidth{0.803000pt}%
\definecolor{currentstroke}{rgb}{0.000000,0.000000,0.000000}%
\pgfsetstrokecolor{currentstroke}%
\pgfsetdash{}{0pt}%
\pgfsys@defobject{currentmarker}{\pgfqpoint{0.000000in}{-0.048611in}}{\pgfqpoint{0.000000in}{0.000000in}}{%
\pgfpathmoveto{\pgfqpoint{0.000000in}{0.000000in}}%
\pgfpathlineto{\pgfqpoint{0.000000in}{-0.048611in}}%
\pgfusepath{stroke,fill}%
}%
\begin{pgfscope}%
\pgfsys@transformshift{3.843636in}{0.528000in}%
\pgfsys@useobject{currentmarker}{}%
\end{pgfscope}%
\end{pgfscope}%
\begin{pgfscope}%
\definecolor{textcolor}{rgb}{0.000000,0.000000,0.000000}%
\pgfsetstrokecolor{textcolor}%
\pgfsetfillcolor{textcolor}%
\pgftext[x=3.843636in,y=0.430778in,,top]{\color{textcolor}\rmfamily\fontsize{10.000000}{12.000000}\selectfont \(\displaystyle 0.25\)}%
\end{pgfscope}%
\begin{pgfscope}%
\pgfsetbuttcap%
\pgfsetroundjoin%
\definecolor{currentfill}{rgb}{0.000000,0.000000,0.000000}%
\pgfsetfillcolor{currentfill}%
\pgfsetlinewidth{0.803000pt}%
\definecolor{currentstroke}{rgb}{0.000000,0.000000,0.000000}%
\pgfsetstrokecolor{currentstroke}%
\pgfsetdash{}{0pt}%
\pgfsys@defobject{currentmarker}{\pgfqpoint{0.000000in}{-0.048611in}}{\pgfqpoint{0.000000in}{0.000000in}}{%
\pgfpathmoveto{\pgfqpoint{0.000000in}{0.000000in}}%
\pgfpathlineto{\pgfqpoint{0.000000in}{-0.048611in}}%
\pgfusepath{stroke,fill}%
}%
\begin{pgfscope}%
\pgfsys@transformshift{4.407273in}{0.528000in}%
\pgfsys@useobject{currentmarker}{}%
\end{pgfscope}%
\end{pgfscope}%
\begin{pgfscope}%
\definecolor{textcolor}{rgb}{0.000000,0.000000,0.000000}%
\pgfsetstrokecolor{textcolor}%
\pgfsetfillcolor{textcolor}%
\pgftext[x=4.407273in,y=0.430778in,,top]{\color{textcolor}\rmfamily\fontsize{10.000000}{12.000000}\selectfont \(\displaystyle 0.50\)}%
\end{pgfscope}%
\begin{pgfscope}%
\pgfsetbuttcap%
\pgfsetroundjoin%
\definecolor{currentfill}{rgb}{0.000000,0.000000,0.000000}%
\pgfsetfillcolor{currentfill}%
\pgfsetlinewidth{0.803000pt}%
\definecolor{currentstroke}{rgb}{0.000000,0.000000,0.000000}%
\pgfsetstrokecolor{currentstroke}%
\pgfsetdash{}{0pt}%
\pgfsys@defobject{currentmarker}{\pgfqpoint{0.000000in}{-0.048611in}}{\pgfqpoint{0.000000in}{0.000000in}}{%
\pgfpathmoveto{\pgfqpoint{0.000000in}{0.000000in}}%
\pgfpathlineto{\pgfqpoint{0.000000in}{-0.048611in}}%
\pgfusepath{stroke,fill}%
}%
\begin{pgfscope}%
\pgfsys@transformshift{4.970909in}{0.528000in}%
\pgfsys@useobject{currentmarker}{}%
\end{pgfscope}%
\end{pgfscope}%
\begin{pgfscope}%
\definecolor{textcolor}{rgb}{0.000000,0.000000,0.000000}%
\pgfsetstrokecolor{textcolor}%
\pgfsetfillcolor{textcolor}%
\pgftext[x=4.970909in,y=0.430778in,,top]{\color{textcolor}\rmfamily\fontsize{10.000000}{12.000000}\selectfont \(\displaystyle 0.75\)}%
\end{pgfscope}%
\begin{pgfscope}%
\pgfsetbuttcap%
\pgfsetroundjoin%
\definecolor{currentfill}{rgb}{0.000000,0.000000,0.000000}%
\pgfsetfillcolor{currentfill}%
\pgfsetlinewidth{0.803000pt}%
\definecolor{currentstroke}{rgb}{0.000000,0.000000,0.000000}%
\pgfsetstrokecolor{currentstroke}%
\pgfsetdash{}{0pt}%
\pgfsys@defobject{currentmarker}{\pgfqpoint{0.000000in}{-0.048611in}}{\pgfqpoint{0.000000in}{0.000000in}}{%
\pgfpathmoveto{\pgfqpoint{0.000000in}{0.000000in}}%
\pgfpathlineto{\pgfqpoint{0.000000in}{-0.048611in}}%
\pgfusepath{stroke,fill}%
}%
\begin{pgfscope}%
\pgfsys@transformshift{5.534545in}{0.528000in}%
\pgfsys@useobject{currentmarker}{}%
\end{pgfscope}%
\end{pgfscope}%
\begin{pgfscope}%
\definecolor{textcolor}{rgb}{0.000000,0.000000,0.000000}%
\pgfsetstrokecolor{textcolor}%
\pgfsetfillcolor{textcolor}%
\pgftext[x=5.534545in,y=0.430778in,,top]{\color{textcolor}\rmfamily\fontsize{10.000000}{12.000000}\selectfont \(\displaystyle 1.00\)}%
\end{pgfscope}%
\begin{pgfscope}%
\definecolor{textcolor}{rgb}{0.000000,0.000000,0.000000}%
\pgfsetstrokecolor{textcolor}%
\pgfsetfillcolor{textcolor}%
\pgftext[x=3.280000in,y=0.251766in,,top]{\color{textcolor}\rmfamily\fontsize{10.000000}{12.000000}\selectfont x}%
\end{pgfscope}%
\begin{pgfscope}%
\pgfsetbuttcap%
\pgfsetroundjoin%
\definecolor{currentfill}{rgb}{0.000000,0.000000,0.000000}%
\pgfsetfillcolor{currentfill}%
\pgfsetlinewidth{0.803000pt}%
\definecolor{currentstroke}{rgb}{0.000000,0.000000,0.000000}%
\pgfsetstrokecolor{currentstroke}%
\pgfsetdash{}{0pt}%
\pgfsys@defobject{currentmarker}{\pgfqpoint{-0.048611in}{0.000000in}}{\pgfqpoint{0.000000in}{0.000000in}}{%
\pgfpathmoveto{\pgfqpoint{0.000000in}{0.000000in}}%
\pgfpathlineto{\pgfqpoint{-0.048611in}{0.000000in}}%
\pgfusepath{stroke,fill}%
}%
\begin{pgfscope}%
\pgfsys@transformshift{0.800000in}{0.528000in}%
\pgfsys@useobject{currentmarker}{}%
\end{pgfscope}%
\end{pgfscope}%
\begin{pgfscope}%
\definecolor{textcolor}{rgb}{0.000000,0.000000,0.000000}%
\pgfsetstrokecolor{textcolor}%
\pgfsetfillcolor{textcolor}%
\pgftext[x=0.347838in,y=0.479775in,left,base]{\color{textcolor}\rmfamily\fontsize{10.000000}{12.000000}\selectfont \(\displaystyle -0.50\)}%
\end{pgfscope}%
\begin{pgfscope}%
\pgfsetbuttcap%
\pgfsetroundjoin%
\definecolor{currentfill}{rgb}{0.000000,0.000000,0.000000}%
\pgfsetfillcolor{currentfill}%
\pgfsetlinewidth{0.803000pt}%
\definecolor{currentstroke}{rgb}{0.000000,0.000000,0.000000}%
\pgfsetstrokecolor{currentstroke}%
\pgfsetdash{}{0pt}%
\pgfsys@defobject{currentmarker}{\pgfqpoint{-0.048611in}{0.000000in}}{\pgfqpoint{0.000000in}{0.000000in}}{%
\pgfpathmoveto{\pgfqpoint{0.000000in}{0.000000in}}%
\pgfpathlineto{\pgfqpoint{-0.048611in}{0.000000in}}%
\pgfusepath{stroke,fill}%
}%
\begin{pgfscope}%
\pgfsys@transformshift{0.800000in}{0.990000in}%
\pgfsys@useobject{currentmarker}{}%
\end{pgfscope}%
\end{pgfscope}%
\begin{pgfscope}%
\definecolor{textcolor}{rgb}{0.000000,0.000000,0.000000}%
\pgfsetstrokecolor{textcolor}%
\pgfsetfillcolor{textcolor}%
\pgftext[x=0.347838in,y=0.941775in,left,base]{\color{textcolor}\rmfamily\fontsize{10.000000}{12.000000}\selectfont \(\displaystyle -0.25\)}%
\end{pgfscope}%
\begin{pgfscope}%
\pgfsetbuttcap%
\pgfsetroundjoin%
\definecolor{currentfill}{rgb}{0.000000,0.000000,0.000000}%
\pgfsetfillcolor{currentfill}%
\pgfsetlinewidth{0.803000pt}%
\definecolor{currentstroke}{rgb}{0.000000,0.000000,0.000000}%
\pgfsetstrokecolor{currentstroke}%
\pgfsetdash{}{0pt}%
\pgfsys@defobject{currentmarker}{\pgfqpoint{-0.048611in}{0.000000in}}{\pgfqpoint{0.000000in}{0.000000in}}{%
\pgfpathmoveto{\pgfqpoint{0.000000in}{0.000000in}}%
\pgfpathlineto{\pgfqpoint{-0.048611in}{0.000000in}}%
\pgfusepath{stroke,fill}%
}%
\begin{pgfscope}%
\pgfsys@transformshift{0.800000in}{1.452000in}%
\pgfsys@useobject{currentmarker}{}%
\end{pgfscope}%
\end{pgfscope}%
\begin{pgfscope}%
\definecolor{textcolor}{rgb}{0.000000,0.000000,0.000000}%
\pgfsetstrokecolor{textcolor}%
\pgfsetfillcolor{textcolor}%
\pgftext[x=0.455863in,y=1.403775in,left,base]{\color{textcolor}\rmfamily\fontsize{10.000000}{12.000000}\selectfont \(\displaystyle 0.00\)}%
\end{pgfscope}%
\begin{pgfscope}%
\pgfsetbuttcap%
\pgfsetroundjoin%
\definecolor{currentfill}{rgb}{0.000000,0.000000,0.000000}%
\pgfsetfillcolor{currentfill}%
\pgfsetlinewidth{0.803000pt}%
\definecolor{currentstroke}{rgb}{0.000000,0.000000,0.000000}%
\pgfsetstrokecolor{currentstroke}%
\pgfsetdash{}{0pt}%
\pgfsys@defobject{currentmarker}{\pgfqpoint{-0.048611in}{0.000000in}}{\pgfqpoint{0.000000in}{0.000000in}}{%
\pgfpathmoveto{\pgfqpoint{0.000000in}{0.000000in}}%
\pgfpathlineto{\pgfqpoint{-0.048611in}{0.000000in}}%
\pgfusepath{stroke,fill}%
}%
\begin{pgfscope}%
\pgfsys@transformshift{0.800000in}{1.914000in}%
\pgfsys@useobject{currentmarker}{}%
\end{pgfscope}%
\end{pgfscope}%
\begin{pgfscope}%
\definecolor{textcolor}{rgb}{0.000000,0.000000,0.000000}%
\pgfsetstrokecolor{textcolor}%
\pgfsetfillcolor{textcolor}%
\pgftext[x=0.455863in,y=1.865775in,left,base]{\color{textcolor}\rmfamily\fontsize{10.000000}{12.000000}\selectfont \(\displaystyle 0.25\)}%
\end{pgfscope}%
\begin{pgfscope}%
\pgfsetbuttcap%
\pgfsetroundjoin%
\definecolor{currentfill}{rgb}{0.000000,0.000000,0.000000}%
\pgfsetfillcolor{currentfill}%
\pgfsetlinewidth{0.803000pt}%
\definecolor{currentstroke}{rgb}{0.000000,0.000000,0.000000}%
\pgfsetstrokecolor{currentstroke}%
\pgfsetdash{}{0pt}%
\pgfsys@defobject{currentmarker}{\pgfqpoint{-0.048611in}{0.000000in}}{\pgfqpoint{0.000000in}{0.000000in}}{%
\pgfpathmoveto{\pgfqpoint{0.000000in}{0.000000in}}%
\pgfpathlineto{\pgfqpoint{-0.048611in}{0.000000in}}%
\pgfusepath{stroke,fill}%
}%
\begin{pgfscope}%
\pgfsys@transformshift{0.800000in}{2.376000in}%
\pgfsys@useobject{currentmarker}{}%
\end{pgfscope}%
\end{pgfscope}%
\begin{pgfscope}%
\definecolor{textcolor}{rgb}{0.000000,0.000000,0.000000}%
\pgfsetstrokecolor{textcolor}%
\pgfsetfillcolor{textcolor}%
\pgftext[x=0.455863in,y=2.327775in,left,base]{\color{textcolor}\rmfamily\fontsize{10.000000}{12.000000}\selectfont \(\displaystyle 0.50\)}%
\end{pgfscope}%
\begin{pgfscope}%
\pgfsetbuttcap%
\pgfsetroundjoin%
\definecolor{currentfill}{rgb}{0.000000,0.000000,0.000000}%
\pgfsetfillcolor{currentfill}%
\pgfsetlinewidth{0.803000pt}%
\definecolor{currentstroke}{rgb}{0.000000,0.000000,0.000000}%
\pgfsetstrokecolor{currentstroke}%
\pgfsetdash{}{0pt}%
\pgfsys@defobject{currentmarker}{\pgfqpoint{-0.048611in}{0.000000in}}{\pgfqpoint{0.000000in}{0.000000in}}{%
\pgfpathmoveto{\pgfqpoint{0.000000in}{0.000000in}}%
\pgfpathlineto{\pgfqpoint{-0.048611in}{0.000000in}}%
\pgfusepath{stroke,fill}%
}%
\begin{pgfscope}%
\pgfsys@transformshift{0.800000in}{2.838000in}%
\pgfsys@useobject{currentmarker}{}%
\end{pgfscope}%
\end{pgfscope}%
\begin{pgfscope}%
\definecolor{textcolor}{rgb}{0.000000,0.000000,0.000000}%
\pgfsetstrokecolor{textcolor}%
\pgfsetfillcolor{textcolor}%
\pgftext[x=0.455863in,y=2.789775in,left,base]{\color{textcolor}\rmfamily\fontsize{10.000000}{12.000000}\selectfont \(\displaystyle 0.75\)}%
\end{pgfscope}%
\begin{pgfscope}%
\pgfsetbuttcap%
\pgfsetroundjoin%
\definecolor{currentfill}{rgb}{0.000000,0.000000,0.000000}%
\pgfsetfillcolor{currentfill}%
\pgfsetlinewidth{0.803000pt}%
\definecolor{currentstroke}{rgb}{0.000000,0.000000,0.000000}%
\pgfsetstrokecolor{currentstroke}%
\pgfsetdash{}{0pt}%
\pgfsys@defobject{currentmarker}{\pgfqpoint{-0.048611in}{0.000000in}}{\pgfqpoint{0.000000in}{0.000000in}}{%
\pgfpathmoveto{\pgfqpoint{0.000000in}{0.000000in}}%
\pgfpathlineto{\pgfqpoint{-0.048611in}{0.000000in}}%
\pgfusepath{stroke,fill}%
}%
\begin{pgfscope}%
\pgfsys@transformshift{0.800000in}{3.300000in}%
\pgfsys@useobject{currentmarker}{}%
\end{pgfscope}%
\end{pgfscope}%
\begin{pgfscope}%
\definecolor{textcolor}{rgb}{0.000000,0.000000,0.000000}%
\pgfsetstrokecolor{textcolor}%
\pgfsetfillcolor{textcolor}%
\pgftext[x=0.455863in,y=3.251775in,left,base]{\color{textcolor}\rmfamily\fontsize{10.000000}{12.000000}\selectfont \(\displaystyle 1.00\)}%
\end{pgfscope}%
\begin{pgfscope}%
\pgfsetbuttcap%
\pgfsetroundjoin%
\definecolor{currentfill}{rgb}{0.000000,0.000000,0.000000}%
\pgfsetfillcolor{currentfill}%
\pgfsetlinewidth{0.803000pt}%
\definecolor{currentstroke}{rgb}{0.000000,0.000000,0.000000}%
\pgfsetstrokecolor{currentstroke}%
\pgfsetdash{}{0pt}%
\pgfsys@defobject{currentmarker}{\pgfqpoint{-0.048611in}{0.000000in}}{\pgfqpoint{0.000000in}{0.000000in}}{%
\pgfpathmoveto{\pgfqpoint{0.000000in}{0.000000in}}%
\pgfpathlineto{\pgfqpoint{-0.048611in}{0.000000in}}%
\pgfusepath{stroke,fill}%
}%
\begin{pgfscope}%
\pgfsys@transformshift{0.800000in}{3.762000in}%
\pgfsys@useobject{currentmarker}{}%
\end{pgfscope}%
\end{pgfscope}%
\begin{pgfscope}%
\definecolor{textcolor}{rgb}{0.000000,0.000000,0.000000}%
\pgfsetstrokecolor{textcolor}%
\pgfsetfillcolor{textcolor}%
\pgftext[x=0.455863in,y=3.713775in,left,base]{\color{textcolor}\rmfamily\fontsize{10.000000}{12.000000}\selectfont \(\displaystyle 1.25\)}%
\end{pgfscope}%
\begin{pgfscope}%
\pgfsetbuttcap%
\pgfsetroundjoin%
\definecolor{currentfill}{rgb}{0.000000,0.000000,0.000000}%
\pgfsetfillcolor{currentfill}%
\pgfsetlinewidth{0.803000pt}%
\definecolor{currentstroke}{rgb}{0.000000,0.000000,0.000000}%
\pgfsetstrokecolor{currentstroke}%
\pgfsetdash{}{0pt}%
\pgfsys@defobject{currentmarker}{\pgfqpoint{-0.048611in}{0.000000in}}{\pgfqpoint{0.000000in}{0.000000in}}{%
\pgfpathmoveto{\pgfqpoint{0.000000in}{0.000000in}}%
\pgfpathlineto{\pgfqpoint{-0.048611in}{0.000000in}}%
\pgfusepath{stroke,fill}%
}%
\begin{pgfscope}%
\pgfsys@transformshift{0.800000in}{4.224000in}%
\pgfsys@useobject{currentmarker}{}%
\end{pgfscope}%
\end{pgfscope}%
\begin{pgfscope}%
\definecolor{textcolor}{rgb}{0.000000,0.000000,0.000000}%
\pgfsetstrokecolor{textcolor}%
\pgfsetfillcolor{textcolor}%
\pgftext[x=0.455863in,y=4.175775in,left,base]{\color{textcolor}\rmfamily\fontsize{10.000000}{12.000000}\selectfont \(\displaystyle 1.50\)}%
\end{pgfscope}%
\begin{pgfscope}%
\definecolor{textcolor}{rgb}{0.000000,0.000000,0.000000}%
\pgfsetstrokecolor{textcolor}%
\pgfsetfillcolor{textcolor}%
\pgftext[x=0.292283in,y=2.376000in,,bottom,rotate=90.000000]{\color{textcolor}\rmfamily\fontsize{10.000000}{12.000000}\selectfont y}%
\end{pgfscope}%
\begin{pgfscope}%
\pgfpathrectangle{\pgfqpoint{0.800000in}{0.528000in}}{\pgfqpoint{4.960000in}{3.696000in}}%
\pgfusepath{clip}%
\pgfsetrectcap%
\pgfsetroundjoin%
\pgfsetlinewidth{1.505625pt}%
\definecolor{currentstroke}{rgb}{0.121569,0.466667,0.705882}%
\pgfsetstrokecolor{currentstroke}%
\pgfsetdash{}{0pt}%
\pgfpathmoveto{\pgfqpoint{1.025455in}{1.523077in}}%
\pgfpathlineto{\pgfqpoint{1.184066in}{1.533748in}}%
\pgfpathlineto{\pgfqpoint{1.320018in}{1.544892in}}%
\pgfpathlineto{\pgfqpoint{1.455971in}{1.558428in}}%
\pgfpathlineto{\pgfqpoint{1.569265in}{1.572045in}}%
\pgfpathlineto{\pgfqpoint{1.659899in}{1.584860in}}%
\pgfpathlineto{\pgfqpoint{1.750534in}{1.599776in}}%
\pgfpathlineto{\pgfqpoint{1.841169in}{1.617263in}}%
\pgfpathlineto{\pgfqpoint{1.909146in}{1.632419in}}%
\pgfpathlineto{\pgfqpoint{1.977122in}{1.649670in}}%
\pgfpathlineto{\pgfqpoint{2.045098in}{1.669400in}}%
\pgfpathlineto{\pgfqpoint{2.113074in}{1.692080in}}%
\pgfpathlineto{\pgfqpoint{2.158392in}{1.709119in}}%
\pgfpathlineto{\pgfqpoint{2.203709in}{1.727926in}}%
\pgfpathlineto{\pgfqpoint{2.249027in}{1.748735in}}%
\pgfpathlineto{\pgfqpoint{2.294344in}{1.771818in}}%
\pgfpathlineto{\pgfqpoint{2.339662in}{1.797485in}}%
\pgfpathlineto{\pgfqpoint{2.384979in}{1.826095in}}%
\pgfpathlineto{\pgfqpoint{2.430297in}{1.858060in}}%
\pgfpathlineto{\pgfqpoint{2.475614in}{1.893855in}}%
\pgfpathlineto{\pgfqpoint{2.520932in}{1.934017in}}%
\pgfpathlineto{\pgfqpoint{2.566249in}{1.979154in}}%
\pgfpathlineto{\pgfqpoint{2.611567in}{2.029943in}}%
\pgfpathlineto{\pgfqpoint{2.656884in}{2.087123in}}%
\pgfpathlineto{\pgfqpoint{2.702202in}{2.151469in}}%
\pgfpathlineto{\pgfqpoint{2.724861in}{2.186574in}}%
\pgfpathlineto{\pgfqpoint{2.747519in}{2.223757in}}%
\pgfpathlineto{\pgfqpoint{2.770178in}{2.263104in}}%
\pgfpathlineto{\pgfqpoint{2.815496in}{2.348561in}}%
\pgfpathlineto{\pgfqpoint{2.860813in}{2.443286in}}%
\pgfpathlineto{\pgfqpoint{2.906131in}{2.547122in}}%
\pgfpathlineto{\pgfqpoint{2.951448in}{2.659117in}}%
\pgfpathlineto{\pgfqpoint{3.019424in}{2.837353in}}%
\pgfpathlineto{\pgfqpoint{3.087401in}{3.014864in}}%
\pgfpathlineto{\pgfqpoint{3.110059in}{3.070154in}}%
\pgfpathlineto{\pgfqpoint{3.132718in}{3.121845in}}%
\pgfpathlineto{\pgfqpoint{3.155377in}{3.168855in}}%
\pgfpathlineto{\pgfqpoint{3.178036in}{3.210099in}}%
\pgfpathlineto{\pgfqpoint{3.200694in}{3.244550in}}%
\pgfpathlineto{\pgfqpoint{3.223353in}{3.271287in}}%
\pgfpathlineto{\pgfqpoint{3.246012in}{3.289560in}}%
\pgfpathlineto{\pgfqpoint{3.268671in}{3.298834in}}%
\pgfpathlineto{\pgfqpoint{3.291329in}{3.298834in}}%
\pgfpathlineto{\pgfqpoint{3.313988in}{3.289560in}}%
\pgfpathlineto{\pgfqpoint{3.336647in}{3.271287in}}%
\pgfpathlineto{\pgfqpoint{3.359306in}{3.244550in}}%
\pgfpathlineto{\pgfqpoint{3.381964in}{3.210099in}}%
\pgfpathlineto{\pgfqpoint{3.404623in}{3.168855in}}%
\pgfpathlineto{\pgfqpoint{3.427282in}{3.121845in}}%
\pgfpathlineto{\pgfqpoint{3.449941in}{3.070154in}}%
\pgfpathlineto{\pgfqpoint{3.495258in}{2.957011in}}%
\pgfpathlineto{\pgfqpoint{3.631211in}{2.602200in}}%
\pgfpathlineto{\pgfqpoint{3.676528in}{2.494102in}}%
\pgfpathlineto{\pgfqpoint{3.721846in}{2.394759in}}%
\pgfpathlineto{\pgfqpoint{3.767163in}{2.304686in}}%
\pgfpathlineto{\pgfqpoint{3.812481in}{2.223757in}}%
\pgfpathlineto{\pgfqpoint{3.857798in}{2.151469in}}%
\pgfpathlineto{\pgfqpoint{3.903116in}{2.087123in}}%
\pgfpathlineto{\pgfqpoint{3.948433in}{2.029943in}}%
\pgfpathlineto{\pgfqpoint{3.993751in}{1.979154in}}%
\pgfpathlineto{\pgfqpoint{4.039068in}{1.934017in}}%
\pgfpathlineto{\pgfqpoint{4.084386in}{1.893855in}}%
\pgfpathlineto{\pgfqpoint{4.129703in}{1.858060in}}%
\pgfpathlineto{\pgfqpoint{4.175021in}{1.826095in}}%
\pgfpathlineto{\pgfqpoint{4.220338in}{1.797485in}}%
\pgfpathlineto{\pgfqpoint{4.265656in}{1.771818in}}%
\pgfpathlineto{\pgfqpoint{4.310973in}{1.748735in}}%
\pgfpathlineto{\pgfqpoint{4.356291in}{1.727926in}}%
\pgfpathlineto{\pgfqpoint{4.401608in}{1.709119in}}%
\pgfpathlineto{\pgfqpoint{4.446926in}{1.692080in}}%
\pgfpathlineto{\pgfqpoint{4.514902in}{1.669400in}}%
\pgfpathlineto{\pgfqpoint{4.582878in}{1.649670in}}%
\pgfpathlineto{\pgfqpoint{4.650854in}{1.632419in}}%
\pgfpathlineto{\pgfqpoint{4.718831in}{1.617263in}}%
\pgfpathlineto{\pgfqpoint{4.809466in}{1.599776in}}%
\pgfpathlineto{\pgfqpoint{4.900101in}{1.584860in}}%
\pgfpathlineto{\pgfqpoint{4.990735in}{1.572045in}}%
\pgfpathlineto{\pgfqpoint{5.104029in}{1.558428in}}%
\pgfpathlineto{\pgfqpoint{5.217323in}{1.546965in}}%
\pgfpathlineto{\pgfqpoint{5.353275in}{1.535463in}}%
\pgfpathlineto{\pgfqpoint{5.511887in}{1.524471in}}%
\pgfpathlineto{\pgfqpoint{5.534545in}{1.523077in}}%
\pgfpathlineto{\pgfqpoint{5.534545in}{1.523077in}}%
\pgfusepath{stroke}%
\end{pgfscope}%
\begin{pgfscope}%
\pgfpathrectangle{\pgfqpoint{0.800000in}{0.528000in}}{\pgfqpoint{4.960000in}{3.696000in}}%
\pgfusepath{clip}%
\pgfsetbuttcap%
\pgfsetroundjoin%
\definecolor{currentfill}{rgb}{1.000000,0.498039,0.054902}%
\pgfsetfillcolor{currentfill}%
\pgfsetlinewidth{1.003750pt}%
\definecolor{currentstroke}{rgb}{1.000000,0.498039,0.054902}%
\pgfsetstrokecolor{currentstroke}%
\pgfsetdash{}{0pt}%
\pgfsys@defobject{currentmarker}{\pgfqpoint{-0.020833in}{-0.020833in}}{\pgfqpoint{0.020833in}{0.020833in}}{%
\pgfpathmoveto{\pgfqpoint{0.000000in}{-0.020833in}}%
\pgfpathcurveto{\pgfqpoint{0.005525in}{-0.020833in}}{\pgfqpoint{0.010825in}{-0.018638in}}{\pgfqpoint{0.014731in}{-0.014731in}}%
\pgfpathcurveto{\pgfqpoint{0.018638in}{-0.010825in}}{\pgfqpoint{0.020833in}{-0.005525in}}{\pgfqpoint{0.020833in}{0.000000in}}%
\pgfpathcurveto{\pgfqpoint{0.020833in}{0.005525in}}{\pgfqpoint{0.018638in}{0.010825in}}{\pgfqpoint{0.014731in}{0.014731in}}%
\pgfpathcurveto{\pgfqpoint{0.010825in}{0.018638in}}{\pgfqpoint{0.005525in}{0.020833in}}{\pgfqpoint{0.000000in}{0.020833in}}%
\pgfpathcurveto{\pgfqpoint{-0.005525in}{0.020833in}}{\pgfqpoint{-0.010825in}{0.018638in}}{\pgfqpoint{-0.014731in}{0.014731in}}%
\pgfpathcurveto{\pgfqpoint{-0.018638in}{0.010825in}}{\pgfqpoint{-0.020833in}{0.005525in}}{\pgfqpoint{-0.020833in}{0.000000in}}%
\pgfpathcurveto{\pgfqpoint{-0.020833in}{-0.005525in}}{\pgfqpoint{-0.018638in}{-0.010825in}}{\pgfqpoint{-0.014731in}{-0.014731in}}%
\pgfpathcurveto{\pgfqpoint{-0.010825in}{-0.018638in}}{\pgfqpoint{-0.005525in}{-0.020833in}}{\pgfqpoint{0.000000in}{-0.020833in}}%
\pgfpathclose%
\pgfusepath{stroke,fill}%
}%
\begin{pgfscope}%
\pgfsys@transformshift{1.025455in}{1.523077in}%
\pgfsys@useobject{currentmarker}{}%
\end{pgfscope}%
\begin{pgfscope}%
\pgfsys@transformshift{2.528485in}{1.941176in}%
\pgfsys@useobject{currentmarker}{}%
\end{pgfscope}%
\begin{pgfscope}%
\pgfsys@transformshift{4.031515in}{1.941176in}%
\pgfsys@useobject{currentmarker}{}%
\end{pgfscope}%
\end{pgfscope}%
\begin{pgfscope}%
\pgfpathrectangle{\pgfqpoint{0.800000in}{0.528000in}}{\pgfqpoint{4.960000in}{3.696000in}}%
\pgfusepath{clip}%
\pgfsetrectcap%
\pgfsetroundjoin%
\pgfsetlinewidth{1.505625pt}%
\definecolor{currentstroke}{rgb}{0.172549,0.627451,0.172549}%
\pgfsetstrokecolor{currentstroke}%
\pgfsetdash{}{0pt}%
\pgfpathmoveto{\pgfqpoint{1.025455in}{1.523077in}}%
\pgfpathlineto{\pgfqpoint{1.138748in}{1.569162in}}%
\pgfpathlineto{\pgfqpoint{1.252042in}{1.612871in}}%
\pgfpathlineto{\pgfqpoint{1.365336in}{1.654205in}}%
\pgfpathlineto{\pgfqpoint{1.478630in}{1.693163in}}%
\pgfpathlineto{\pgfqpoint{1.591923in}{1.729746in}}%
\pgfpathlineto{\pgfqpoint{1.705217in}{1.763953in}}%
\pgfpathlineto{\pgfqpoint{1.818511in}{1.795785in}}%
\pgfpathlineto{\pgfqpoint{1.931804in}{1.825241in}}%
\pgfpathlineto{\pgfqpoint{2.045098in}{1.852322in}}%
\pgfpathlineto{\pgfqpoint{2.158392in}{1.877027in}}%
\pgfpathlineto{\pgfqpoint{2.271686in}{1.899357in}}%
\pgfpathlineto{\pgfqpoint{2.384979in}{1.919311in}}%
\pgfpathlineto{\pgfqpoint{2.498273in}{1.936890in}}%
\pgfpathlineto{\pgfqpoint{2.611567in}{1.952093in}}%
\pgfpathlineto{\pgfqpoint{2.724861in}{1.964921in}}%
\pgfpathlineto{\pgfqpoint{2.838154in}{1.975373in}}%
\pgfpathlineto{\pgfqpoint{2.951448in}{1.983450in}}%
\pgfpathlineto{\pgfqpoint{3.064742in}{1.989151in}}%
\pgfpathlineto{\pgfqpoint{3.178036in}{1.992477in}}%
\pgfpathlineto{\pgfqpoint{3.291329in}{1.993427in}}%
\pgfpathlineto{\pgfqpoint{3.404623in}{1.992002in}}%
\pgfpathlineto{\pgfqpoint{3.517917in}{1.988201in}}%
\pgfpathlineto{\pgfqpoint{3.631211in}{1.982025in}}%
\pgfpathlineto{\pgfqpoint{3.744504in}{1.973473in}}%
\pgfpathlineto{\pgfqpoint{3.857798in}{1.962545in}}%
\pgfpathlineto{\pgfqpoint{3.971092in}{1.949243in}}%
\pgfpathlineto{\pgfqpoint{4.084386in}{1.933564in}}%
\pgfpathlineto{\pgfqpoint{4.197679in}{1.915510in}}%
\pgfpathlineto{\pgfqpoint{4.310973in}{1.895081in}}%
\pgfpathlineto{\pgfqpoint{4.424267in}{1.872276in}}%
\pgfpathlineto{\pgfqpoint{4.537561in}{1.847096in}}%
\pgfpathlineto{\pgfqpoint{4.650854in}{1.819540in}}%
\pgfpathlineto{\pgfqpoint{4.764148in}{1.789609in}}%
\pgfpathlineto{\pgfqpoint{4.877442in}{1.757302in}}%
\pgfpathlineto{\pgfqpoint{4.990735in}{1.722619in}}%
\pgfpathlineto{\pgfqpoint{5.104029in}{1.685562in}}%
\pgfpathlineto{\pgfqpoint{5.217323in}{1.646128in}}%
\pgfpathlineto{\pgfqpoint{5.330617in}{1.604319in}}%
\pgfpathlineto{\pgfqpoint{5.443910in}{1.560135in}}%
\pgfpathlineto{\pgfqpoint{5.534545in}{1.523077in}}%
\pgfpathlineto{\pgfqpoint{5.534545in}{1.523077in}}%
\pgfusepath{stroke}%
\end{pgfscope}%
\begin{pgfscope}%
\pgfpathrectangle{\pgfqpoint{0.800000in}{0.528000in}}{\pgfqpoint{4.960000in}{3.696000in}}%
\pgfusepath{clip}%
\pgfsetbuttcap%
\pgfsetroundjoin%
\definecolor{currentfill}{rgb}{0.839216,0.152941,0.156863}%
\pgfsetfillcolor{currentfill}%
\pgfsetlinewidth{1.003750pt}%
\definecolor{currentstroke}{rgb}{0.839216,0.152941,0.156863}%
\pgfsetstrokecolor{currentstroke}%
\pgfsetdash{}{0pt}%
\pgfsys@defobject{currentmarker}{\pgfqpoint{-0.020833in}{-0.020833in}}{\pgfqpoint{0.020833in}{0.020833in}}{%
\pgfpathmoveto{\pgfqpoint{0.000000in}{-0.020833in}}%
\pgfpathcurveto{\pgfqpoint{0.005525in}{-0.020833in}}{\pgfqpoint{0.010825in}{-0.018638in}}{\pgfqpoint{0.014731in}{-0.014731in}}%
\pgfpathcurveto{\pgfqpoint{0.018638in}{-0.010825in}}{\pgfqpoint{0.020833in}{-0.005525in}}{\pgfqpoint{0.020833in}{0.000000in}}%
\pgfpathcurveto{\pgfqpoint{0.020833in}{0.005525in}}{\pgfqpoint{0.018638in}{0.010825in}}{\pgfqpoint{0.014731in}{0.014731in}}%
\pgfpathcurveto{\pgfqpoint{0.010825in}{0.018638in}}{\pgfqpoint{0.005525in}{0.020833in}}{\pgfqpoint{0.000000in}{0.020833in}}%
\pgfpathcurveto{\pgfqpoint{-0.005525in}{0.020833in}}{\pgfqpoint{-0.010825in}{0.018638in}}{\pgfqpoint{-0.014731in}{0.014731in}}%
\pgfpathcurveto{\pgfqpoint{-0.018638in}{0.010825in}}{\pgfqpoint{-0.020833in}{0.005525in}}{\pgfqpoint{-0.020833in}{0.000000in}}%
\pgfpathcurveto{\pgfqpoint{-0.020833in}{-0.005525in}}{\pgfqpoint{-0.018638in}{-0.010825in}}{\pgfqpoint{-0.014731in}{-0.014731in}}%
\pgfpathcurveto{\pgfqpoint{-0.010825in}{-0.018638in}}{\pgfqpoint{-0.005525in}{-0.020833in}}{\pgfqpoint{0.000000in}{-0.020833in}}%
\pgfpathclose%
\pgfusepath{stroke,fill}%
}%
\begin{pgfscope}%
\pgfsys@transformshift{1.025455in}{1.523077in}%
\pgfsys@useobject{currentmarker}{}%
\end{pgfscope}%
\begin{pgfscope}%
\pgfsys@transformshift{1.927273in}{1.636800in}%
\pgfsys@useobject{currentmarker}{}%
\end{pgfscope}%
\begin{pgfscope}%
\pgfsys@transformshift{2.829091in}{2.376000in}%
\pgfsys@useobject{currentmarker}{}%
\end{pgfscope}%
\begin{pgfscope}%
\pgfsys@transformshift{3.730909in}{2.376000in}%
\pgfsys@useobject{currentmarker}{}%
\end{pgfscope}%
\begin{pgfscope}%
\pgfsys@transformshift{4.632727in}{1.636800in}%
\pgfsys@useobject{currentmarker}{}%
\end{pgfscope}%
\end{pgfscope}%
\begin{pgfscope}%
\pgfpathrectangle{\pgfqpoint{0.800000in}{0.528000in}}{\pgfqpoint{4.960000in}{3.696000in}}%
\pgfusepath{clip}%
\pgfsetrectcap%
\pgfsetroundjoin%
\pgfsetlinewidth{1.505625pt}%
\definecolor{currentstroke}{rgb}{0.580392,0.403922,0.741176}%
\pgfsetstrokecolor{currentstroke}%
\pgfsetdash{}{0pt}%
\pgfpathmoveto{\pgfqpoint{1.025455in}{1.523077in}}%
\pgfpathlineto{\pgfqpoint{1.048113in}{1.499089in}}%
\pgfpathlineto{\pgfqpoint{1.070772in}{1.477093in}}%
\pgfpathlineto{\pgfqpoint{1.093431in}{1.457037in}}%
\pgfpathlineto{\pgfqpoint{1.116090in}{1.438867in}}%
\pgfpathlineto{\pgfqpoint{1.138748in}{1.422530in}}%
\pgfpathlineto{\pgfqpoint{1.161407in}{1.407976in}}%
\pgfpathlineto{\pgfqpoint{1.184066in}{1.395154in}}%
\pgfpathlineto{\pgfqpoint{1.206725in}{1.384012in}}%
\pgfpathlineto{\pgfqpoint{1.229383in}{1.374500in}}%
\pgfpathlineto{\pgfqpoint{1.252042in}{1.366570in}}%
\pgfpathlineto{\pgfqpoint{1.274701in}{1.360172in}}%
\pgfpathlineto{\pgfqpoint{1.297360in}{1.355258in}}%
\pgfpathlineto{\pgfqpoint{1.320018in}{1.351779in}}%
\pgfpathlineto{\pgfqpoint{1.342677in}{1.349690in}}%
\pgfpathlineto{\pgfqpoint{1.365336in}{1.348942in}}%
\pgfpathlineto{\pgfqpoint{1.387995in}{1.349489in}}%
\pgfpathlineto{\pgfqpoint{1.410653in}{1.351287in}}%
\pgfpathlineto{\pgfqpoint{1.433312in}{1.354289in}}%
\pgfpathlineto{\pgfqpoint{1.478630in}{1.363731in}}%
\pgfpathlineto{\pgfqpoint{1.523947in}{1.377463in}}%
\pgfpathlineto{\pgfqpoint{1.569265in}{1.395145in}}%
\pgfpathlineto{\pgfqpoint{1.614582in}{1.416443in}}%
\pgfpathlineto{\pgfqpoint{1.659899in}{1.441034in}}%
\pgfpathlineto{\pgfqpoint{1.705217in}{1.468601in}}%
\pgfpathlineto{\pgfqpoint{1.750534in}{1.498839in}}%
\pgfpathlineto{\pgfqpoint{1.795852in}{1.531450in}}%
\pgfpathlineto{\pgfqpoint{1.841169in}{1.566143in}}%
\pgfpathlineto{\pgfqpoint{1.909146in}{1.621476in}}%
\pgfpathlineto{\pgfqpoint{1.977122in}{1.679957in}}%
\pgfpathlineto{\pgfqpoint{2.067757in}{1.761334in}}%
\pgfpathlineto{\pgfqpoint{2.407638in}{2.071304in}}%
\pgfpathlineto{\pgfqpoint{2.475614in}{2.129229in}}%
\pgfpathlineto{\pgfqpoint{2.543591in}{2.184426in}}%
\pgfpathlineto{\pgfqpoint{2.611567in}{2.236396in}}%
\pgfpathlineto{\pgfqpoint{2.679543in}{2.284685in}}%
\pgfpathlineto{\pgfqpoint{2.724861in}{2.314627in}}%
\pgfpathlineto{\pgfqpoint{2.770178in}{2.342639in}}%
\pgfpathlineto{\pgfqpoint{2.815496in}{2.368617in}}%
\pgfpathlineto{\pgfqpoint{2.860813in}{2.392469in}}%
\pgfpathlineto{\pgfqpoint{2.906131in}{2.414109in}}%
\pgfpathlineto{\pgfqpoint{2.951448in}{2.433461in}}%
\pgfpathlineto{\pgfqpoint{2.996766in}{2.450458in}}%
\pgfpathlineto{\pgfqpoint{3.042083in}{2.465042in}}%
\pgfpathlineto{\pgfqpoint{3.087401in}{2.477161in}}%
\pgfpathlineto{\pgfqpoint{3.132718in}{2.486775in}}%
\pgfpathlineto{\pgfqpoint{3.178036in}{2.493852in}}%
\pgfpathlineto{\pgfqpoint{3.223353in}{2.498366in}}%
\pgfpathlineto{\pgfqpoint{3.268671in}{2.500304in}}%
\pgfpathlineto{\pgfqpoint{3.313988in}{2.499658in}}%
\pgfpathlineto{\pgfqpoint{3.359306in}{2.496430in}}%
\pgfpathlineto{\pgfqpoint{3.404623in}{2.490633in}}%
\pgfpathlineto{\pgfqpoint{3.449941in}{2.482284in}}%
\pgfpathlineto{\pgfqpoint{3.495258in}{2.471412in}}%
\pgfpathlineto{\pgfqpoint{3.540576in}{2.458055in}}%
\pgfpathlineto{\pgfqpoint{3.585893in}{2.442258in}}%
\pgfpathlineto{\pgfqpoint{3.631211in}{2.424075in}}%
\pgfpathlineto{\pgfqpoint{3.676528in}{2.403570in}}%
\pgfpathlineto{\pgfqpoint{3.721846in}{2.380814in}}%
\pgfpathlineto{\pgfqpoint{3.767163in}{2.355888in}}%
\pgfpathlineto{\pgfqpoint{3.812481in}{2.328881in}}%
\pgfpathlineto{\pgfqpoint{3.857798in}{2.299891in}}%
\pgfpathlineto{\pgfqpoint{3.903116in}{2.269024in}}%
\pgfpathlineto{\pgfqpoint{3.971092in}{2.219460in}}%
\pgfpathlineto{\pgfqpoint{4.039068in}{2.166361in}}%
\pgfpathlineto{\pgfqpoint{4.107044in}{2.110198in}}%
\pgfpathlineto{\pgfqpoint{4.175021in}{2.051483in}}%
\pgfpathlineto{\pgfqpoint{4.265656in}{1.970199in}}%
\pgfpathlineto{\pgfqpoint{4.424267in}{1.823863in}}%
\pgfpathlineto{\pgfqpoint{4.537561in}{1.720262in}}%
\pgfpathlineto{\pgfqpoint{4.628196in}{1.640664in}}%
\pgfpathlineto{\pgfqpoint{4.696172in}{1.584182in}}%
\pgfpathlineto{\pgfqpoint{4.764148in}{1.531450in}}%
\pgfpathlineto{\pgfqpoint{4.809466in}{1.498839in}}%
\pgfpathlineto{\pgfqpoint{4.854783in}{1.468601in}}%
\pgfpathlineto{\pgfqpoint{4.900101in}{1.441034in}}%
\pgfpathlineto{\pgfqpoint{4.945418in}{1.416443in}}%
\pgfpathlineto{\pgfqpoint{4.990735in}{1.395145in}}%
\pgfpathlineto{\pgfqpoint{5.036053in}{1.377463in}}%
\pgfpathlineto{\pgfqpoint{5.081370in}{1.363731in}}%
\pgfpathlineto{\pgfqpoint{5.126688in}{1.354289in}}%
\pgfpathlineto{\pgfqpoint{5.149347in}{1.351287in}}%
\pgfpathlineto{\pgfqpoint{5.172005in}{1.349489in}}%
\pgfpathlineto{\pgfqpoint{5.194664in}{1.348942in}}%
\pgfpathlineto{\pgfqpoint{5.217323in}{1.349690in}}%
\pgfpathlineto{\pgfqpoint{5.239982in}{1.351779in}}%
\pgfpathlineto{\pgfqpoint{5.262640in}{1.355258in}}%
\pgfpathlineto{\pgfqpoint{5.285299in}{1.360172in}}%
\pgfpathlineto{\pgfqpoint{5.307958in}{1.366570in}}%
\pgfpathlineto{\pgfqpoint{5.330617in}{1.374500in}}%
\pgfpathlineto{\pgfqpoint{5.353275in}{1.384012in}}%
\pgfpathlineto{\pgfqpoint{5.375934in}{1.395154in}}%
\pgfpathlineto{\pgfqpoint{5.398593in}{1.407976in}}%
\pgfpathlineto{\pgfqpoint{5.421252in}{1.422530in}}%
\pgfpathlineto{\pgfqpoint{5.443910in}{1.438867in}}%
\pgfpathlineto{\pgfqpoint{5.466569in}{1.457037in}}%
\pgfpathlineto{\pgfqpoint{5.489228in}{1.477093in}}%
\pgfpathlineto{\pgfqpoint{5.511887in}{1.499089in}}%
\pgfpathlineto{\pgfqpoint{5.534545in}{1.523077in}}%
\pgfpathlineto{\pgfqpoint{5.534545in}{1.523077in}}%
\pgfusepath{stroke}%
\end{pgfscope}%
\begin{pgfscope}%
\pgfpathrectangle{\pgfqpoint{0.800000in}{0.528000in}}{\pgfqpoint{4.960000in}{3.696000in}}%
\pgfusepath{clip}%
\pgfsetrectcap%
\pgfsetroundjoin%
\pgfsetlinewidth{1.505625pt}%
\definecolor{currentstroke}{rgb}{0.549020,0.337255,0.294118}%
\pgfsetstrokecolor{currentstroke}%
\pgfsetdash{}{0pt}%
\pgfpathmoveto{\pgfqpoint{1.025455in}{1.523077in}}%
\pgfpathlineto{\pgfqpoint{1.184066in}{1.533748in}}%
\pgfpathlineto{\pgfqpoint{1.320018in}{1.544892in}}%
\pgfpathlineto{\pgfqpoint{1.455971in}{1.558428in}}%
\pgfpathlineto{\pgfqpoint{1.569265in}{1.572045in}}%
\pgfpathlineto{\pgfqpoint{1.659899in}{1.584860in}}%
\pgfpathlineto{\pgfqpoint{1.750534in}{1.599776in}}%
\pgfpathlineto{\pgfqpoint{1.841169in}{1.617263in}}%
\pgfpathlineto{\pgfqpoint{1.909146in}{1.632419in}}%
\pgfpathlineto{\pgfqpoint{1.977122in}{1.649670in}}%
\pgfpathlineto{\pgfqpoint{2.045098in}{1.669400in}}%
\pgfpathlineto{\pgfqpoint{2.113074in}{1.692080in}}%
\pgfpathlineto{\pgfqpoint{2.158392in}{1.709119in}}%
\pgfpathlineto{\pgfqpoint{2.203709in}{1.727926in}}%
\pgfpathlineto{\pgfqpoint{2.249027in}{1.748735in}}%
\pgfpathlineto{\pgfqpoint{2.294344in}{1.771818in}}%
\pgfpathlineto{\pgfqpoint{2.339662in}{1.797485in}}%
\pgfpathlineto{\pgfqpoint{2.384979in}{1.826095in}}%
\pgfpathlineto{\pgfqpoint{2.430297in}{1.858060in}}%
\pgfpathlineto{\pgfqpoint{2.475614in}{1.893855in}}%
\pgfpathlineto{\pgfqpoint{2.520932in}{1.934017in}}%
\pgfpathlineto{\pgfqpoint{2.566249in}{1.979154in}}%
\pgfpathlineto{\pgfqpoint{2.611567in}{2.029943in}}%
\pgfpathlineto{\pgfqpoint{2.656884in}{2.087123in}}%
\pgfpathlineto{\pgfqpoint{2.702202in}{2.151469in}}%
\pgfpathlineto{\pgfqpoint{2.724861in}{2.186574in}}%
\pgfpathlineto{\pgfqpoint{2.747519in}{2.223757in}}%
\pgfpathlineto{\pgfqpoint{2.770178in}{2.263104in}}%
\pgfpathlineto{\pgfqpoint{2.815496in}{2.348561in}}%
\pgfpathlineto{\pgfqpoint{2.860813in}{2.443286in}}%
\pgfpathlineto{\pgfqpoint{2.906131in}{2.547122in}}%
\pgfpathlineto{\pgfqpoint{2.951448in}{2.659117in}}%
\pgfpathlineto{\pgfqpoint{3.019424in}{2.837353in}}%
\pgfpathlineto{\pgfqpoint{3.087401in}{3.014864in}}%
\pgfpathlineto{\pgfqpoint{3.110059in}{3.070154in}}%
\pgfpathlineto{\pgfqpoint{3.132718in}{3.121845in}}%
\pgfpathlineto{\pgfqpoint{3.155377in}{3.168855in}}%
\pgfpathlineto{\pgfqpoint{3.178036in}{3.210099in}}%
\pgfpathlineto{\pgfqpoint{3.200694in}{3.244550in}}%
\pgfpathlineto{\pgfqpoint{3.223353in}{3.271287in}}%
\pgfpathlineto{\pgfqpoint{3.246012in}{3.289560in}}%
\pgfpathlineto{\pgfqpoint{3.268671in}{3.298834in}}%
\pgfpathlineto{\pgfqpoint{3.291329in}{3.298834in}}%
\pgfpathlineto{\pgfqpoint{3.313988in}{3.289560in}}%
\pgfpathlineto{\pgfqpoint{3.336647in}{3.271287in}}%
\pgfpathlineto{\pgfqpoint{3.359306in}{3.244550in}}%
\pgfpathlineto{\pgfqpoint{3.381964in}{3.210099in}}%
\pgfpathlineto{\pgfqpoint{3.404623in}{3.168855in}}%
\pgfpathlineto{\pgfqpoint{3.427282in}{3.121845in}}%
\pgfpathlineto{\pgfqpoint{3.449941in}{3.070154in}}%
\pgfpathlineto{\pgfqpoint{3.495258in}{2.957011in}}%
\pgfpathlineto{\pgfqpoint{3.631211in}{2.602200in}}%
\pgfpathlineto{\pgfqpoint{3.676528in}{2.494102in}}%
\pgfpathlineto{\pgfqpoint{3.721846in}{2.394759in}}%
\pgfpathlineto{\pgfqpoint{3.767163in}{2.304686in}}%
\pgfpathlineto{\pgfqpoint{3.812481in}{2.223757in}}%
\pgfpathlineto{\pgfqpoint{3.857798in}{2.151469in}}%
\pgfpathlineto{\pgfqpoint{3.903116in}{2.087123in}}%
\pgfpathlineto{\pgfqpoint{3.948433in}{2.029943in}}%
\pgfpathlineto{\pgfqpoint{3.993751in}{1.979154in}}%
\pgfpathlineto{\pgfqpoint{4.039068in}{1.934017in}}%
\pgfpathlineto{\pgfqpoint{4.084386in}{1.893855in}}%
\pgfpathlineto{\pgfqpoint{4.129703in}{1.858060in}}%
\pgfpathlineto{\pgfqpoint{4.175021in}{1.826095in}}%
\pgfpathlineto{\pgfqpoint{4.220338in}{1.797485in}}%
\pgfpathlineto{\pgfqpoint{4.265656in}{1.771818in}}%
\pgfpathlineto{\pgfqpoint{4.310973in}{1.748735in}}%
\pgfpathlineto{\pgfqpoint{4.356291in}{1.727926in}}%
\pgfpathlineto{\pgfqpoint{4.401608in}{1.709119in}}%
\pgfpathlineto{\pgfqpoint{4.446926in}{1.692080in}}%
\pgfpathlineto{\pgfqpoint{4.514902in}{1.669400in}}%
\pgfpathlineto{\pgfqpoint{4.582878in}{1.649670in}}%
\pgfpathlineto{\pgfqpoint{4.650854in}{1.632419in}}%
\pgfpathlineto{\pgfqpoint{4.718831in}{1.617263in}}%
\pgfpathlineto{\pgfqpoint{4.809466in}{1.599776in}}%
\pgfpathlineto{\pgfqpoint{4.900101in}{1.584860in}}%
\pgfpathlineto{\pgfqpoint{4.990735in}{1.572045in}}%
\pgfpathlineto{\pgfqpoint{5.104029in}{1.558428in}}%
\pgfpathlineto{\pgfqpoint{5.217323in}{1.546965in}}%
\pgfpathlineto{\pgfqpoint{5.353275in}{1.535463in}}%
\pgfpathlineto{\pgfqpoint{5.511887in}{1.524471in}}%
\pgfpathlineto{\pgfqpoint{5.534545in}{1.523077in}}%
\pgfpathlineto{\pgfqpoint{5.534545in}{1.523077in}}%
\pgfusepath{stroke}%
\end{pgfscope}%
\begin{pgfscope}%
\pgfpathrectangle{\pgfqpoint{0.800000in}{0.528000in}}{\pgfqpoint{4.960000in}{3.696000in}}%
\pgfusepath{clip}%
\pgfsetbuttcap%
\pgfsetroundjoin%
\definecolor{currentfill}{rgb}{0.890196,0.466667,0.760784}%
\pgfsetfillcolor{currentfill}%
\pgfsetlinewidth{1.003750pt}%
\definecolor{currentstroke}{rgb}{0.890196,0.466667,0.760784}%
\pgfsetstrokecolor{currentstroke}%
\pgfsetdash{}{0pt}%
\pgfsys@defobject{currentmarker}{\pgfqpoint{-0.020833in}{-0.020833in}}{\pgfqpoint{0.020833in}{0.020833in}}{%
\pgfpathmoveto{\pgfqpoint{0.000000in}{-0.020833in}}%
\pgfpathcurveto{\pgfqpoint{0.005525in}{-0.020833in}}{\pgfqpoint{0.010825in}{-0.018638in}}{\pgfqpoint{0.014731in}{-0.014731in}}%
\pgfpathcurveto{\pgfqpoint{0.018638in}{-0.010825in}}{\pgfqpoint{0.020833in}{-0.005525in}}{\pgfqpoint{0.020833in}{0.000000in}}%
\pgfpathcurveto{\pgfqpoint{0.020833in}{0.005525in}}{\pgfqpoint{0.018638in}{0.010825in}}{\pgfqpoint{0.014731in}{0.014731in}}%
\pgfpathcurveto{\pgfqpoint{0.010825in}{0.018638in}}{\pgfqpoint{0.005525in}{0.020833in}}{\pgfqpoint{0.000000in}{0.020833in}}%
\pgfpathcurveto{\pgfqpoint{-0.005525in}{0.020833in}}{\pgfqpoint{-0.010825in}{0.018638in}}{\pgfqpoint{-0.014731in}{0.014731in}}%
\pgfpathcurveto{\pgfqpoint{-0.018638in}{0.010825in}}{\pgfqpoint{-0.020833in}{0.005525in}}{\pgfqpoint{-0.020833in}{0.000000in}}%
\pgfpathcurveto{\pgfqpoint{-0.020833in}{-0.005525in}}{\pgfqpoint{-0.018638in}{-0.010825in}}{\pgfqpoint{-0.014731in}{-0.014731in}}%
\pgfpathcurveto{\pgfqpoint{-0.010825in}{-0.018638in}}{\pgfqpoint{-0.005525in}{-0.020833in}}{\pgfqpoint{0.000000in}{-0.020833in}}%
\pgfpathclose%
\pgfusepath{stroke,fill}%
}%
\begin{pgfscope}%
\pgfsys@transformshift{5.232494in}{1.545570in}%
\pgfsys@useobject{currentmarker}{}%
\end{pgfscope}%
\begin{pgfscope}%
\pgfsys@transformshift{3.280000in}{3.300000in}%
\pgfsys@useobject{currentmarker}{}%
\end{pgfscope}%
\begin{pgfscope}%
\pgfsys@transformshift{1.327506in}{1.545570in}%
\pgfsys@useobject{currentmarker}{}%
\end{pgfscope}%
\end{pgfscope}%
\begin{pgfscope}%
\pgfpathrectangle{\pgfqpoint{0.800000in}{0.528000in}}{\pgfqpoint{4.960000in}{3.696000in}}%
\pgfusepath{clip}%
\pgfsetrectcap%
\pgfsetroundjoin%
\pgfsetlinewidth{1.505625pt}%
\definecolor{currentstroke}{rgb}{0.498039,0.498039,0.498039}%
\pgfsetstrokecolor{currentstroke}%
\pgfsetdash{}{0pt}%
\pgfpathmoveto{\pgfqpoint{1.025455in}{0.960759in}}%
\pgfpathlineto{\pgfqpoint{1.093431in}{1.099693in}}%
\pgfpathlineto{\pgfqpoint{1.161407in}{1.234373in}}%
\pgfpathlineto{\pgfqpoint{1.229383in}{1.364800in}}%
\pgfpathlineto{\pgfqpoint{1.297360in}{1.490974in}}%
\pgfpathlineto{\pgfqpoint{1.365336in}{1.612895in}}%
\pgfpathlineto{\pgfqpoint{1.433312in}{1.730563in}}%
\pgfpathlineto{\pgfqpoint{1.501288in}{1.843978in}}%
\pgfpathlineto{\pgfqpoint{1.569265in}{1.953139in}}%
\pgfpathlineto{\pgfqpoint{1.637241in}{2.058048in}}%
\pgfpathlineto{\pgfqpoint{1.705217in}{2.158704in}}%
\pgfpathlineto{\pgfqpoint{1.773193in}{2.255107in}}%
\pgfpathlineto{\pgfqpoint{1.841169in}{2.347256in}}%
\pgfpathlineto{\pgfqpoint{1.909146in}{2.435153in}}%
\pgfpathlineto{\pgfqpoint{1.977122in}{2.518796in}}%
\pgfpathlineto{\pgfqpoint{2.045098in}{2.598186in}}%
\pgfpathlineto{\pgfqpoint{2.113074in}{2.673324in}}%
\pgfpathlineto{\pgfqpoint{2.181051in}{2.744208in}}%
\pgfpathlineto{\pgfqpoint{2.249027in}{2.810839in}}%
\pgfpathlineto{\pgfqpoint{2.317003in}{2.873218in}}%
\pgfpathlineto{\pgfqpoint{2.384979in}{2.931343in}}%
\pgfpathlineto{\pgfqpoint{2.452956in}{2.985215in}}%
\pgfpathlineto{\pgfqpoint{2.498273in}{3.018767in}}%
\pgfpathlineto{\pgfqpoint{2.543591in}{3.050428in}}%
\pgfpathlineto{\pgfqpoint{2.588908in}{3.080200in}}%
\pgfpathlineto{\pgfqpoint{2.634226in}{3.108081in}}%
\pgfpathlineto{\pgfqpoint{2.679543in}{3.134072in}}%
\pgfpathlineto{\pgfqpoint{2.724861in}{3.158172in}}%
\pgfpathlineto{\pgfqpoint{2.770178in}{3.180383in}}%
\pgfpathlineto{\pgfqpoint{2.815496in}{3.200703in}}%
\pgfpathlineto{\pgfqpoint{2.860813in}{3.219133in}}%
\pgfpathlineto{\pgfqpoint{2.906131in}{3.235673in}}%
\pgfpathlineto{\pgfqpoint{2.951448in}{3.250322in}}%
\pgfpathlineto{\pgfqpoint{2.996766in}{3.263081in}}%
\pgfpathlineto{\pgfqpoint{3.042083in}{3.273950in}}%
\pgfpathlineto{\pgfqpoint{3.087401in}{3.282929in}}%
\pgfpathlineto{\pgfqpoint{3.132718in}{3.290017in}}%
\pgfpathlineto{\pgfqpoint{3.178036in}{3.295215in}}%
\pgfpathlineto{\pgfqpoint{3.223353in}{3.298523in}}%
\pgfpathlineto{\pgfqpoint{3.268671in}{3.299941in}}%
\pgfpathlineto{\pgfqpoint{3.313988in}{3.299468in}}%
\pgfpathlineto{\pgfqpoint{3.359306in}{3.297106in}}%
\pgfpathlineto{\pgfqpoint{3.404623in}{3.292853in}}%
\pgfpathlineto{\pgfqpoint{3.449941in}{3.286709in}}%
\pgfpathlineto{\pgfqpoint{3.495258in}{3.278676in}}%
\pgfpathlineto{\pgfqpoint{3.540576in}{3.268752in}}%
\pgfpathlineto{\pgfqpoint{3.585893in}{3.256938in}}%
\pgfpathlineto{\pgfqpoint{3.631211in}{3.243234in}}%
\pgfpathlineto{\pgfqpoint{3.676528in}{3.227639in}}%
\pgfpathlineto{\pgfqpoint{3.721846in}{3.210154in}}%
\pgfpathlineto{\pgfqpoint{3.767163in}{3.190779in}}%
\pgfpathlineto{\pgfqpoint{3.812481in}{3.169514in}}%
\pgfpathlineto{\pgfqpoint{3.857798in}{3.146358in}}%
\pgfpathlineto{\pgfqpoint{3.903116in}{3.121313in}}%
\pgfpathlineto{\pgfqpoint{3.948433in}{3.094377in}}%
\pgfpathlineto{\pgfqpoint{3.993751in}{3.065550in}}%
\pgfpathlineto{\pgfqpoint{4.039068in}{3.034834in}}%
\pgfpathlineto{\pgfqpoint{4.084386in}{3.002227in}}%
\pgfpathlineto{\pgfqpoint{4.129703in}{2.967730in}}%
\pgfpathlineto{\pgfqpoint{4.197679in}{2.912440in}}%
\pgfpathlineto{\pgfqpoint{4.265656in}{2.852897in}}%
\pgfpathlineto{\pgfqpoint{4.333632in}{2.789102in}}%
\pgfpathlineto{\pgfqpoint{4.401608in}{2.721053in}}%
\pgfpathlineto{\pgfqpoint{4.469584in}{2.648751in}}%
\pgfpathlineto{\pgfqpoint{4.537561in}{2.572196in}}%
\pgfpathlineto{\pgfqpoint{4.605537in}{2.491388in}}%
\pgfpathlineto{\pgfqpoint{4.673513in}{2.406326in}}%
\pgfpathlineto{\pgfqpoint{4.741489in}{2.317012in}}%
\pgfpathlineto{\pgfqpoint{4.809466in}{2.223445in}}%
\pgfpathlineto{\pgfqpoint{4.877442in}{2.125625in}}%
\pgfpathlineto{\pgfqpoint{4.945418in}{2.023551in}}%
\pgfpathlineto{\pgfqpoint{5.013394in}{1.917225in}}%
\pgfpathlineto{\pgfqpoint{5.081370in}{1.806645in}}%
\pgfpathlineto{\pgfqpoint{5.149347in}{1.691813in}}%
\pgfpathlineto{\pgfqpoint{5.217323in}{1.572727in}}%
\pgfpathlineto{\pgfqpoint{5.285299in}{1.449389in}}%
\pgfpathlineto{\pgfqpoint{5.353275in}{1.321797in}}%
\pgfpathlineto{\pgfqpoint{5.421252in}{1.189952in}}%
\pgfpathlineto{\pgfqpoint{5.489228in}{1.053854in}}%
\pgfpathlineto{\pgfqpoint{5.534545in}{0.960759in}}%
\pgfpathlineto{\pgfqpoint{5.534545in}{0.960759in}}%
\pgfusepath{stroke}%
\end{pgfscope}%
\begin{pgfscope}%
\pgfpathrectangle{\pgfqpoint{0.800000in}{0.528000in}}{\pgfqpoint{4.960000in}{3.696000in}}%
\pgfusepath{clip}%
\pgfsetbuttcap%
\pgfsetroundjoin%
\definecolor{currentfill}{rgb}{0.737255,0.741176,0.133333}%
\pgfsetfillcolor{currentfill}%
\pgfsetlinewidth{1.003750pt}%
\definecolor{currentstroke}{rgb}{0.737255,0.741176,0.133333}%
\pgfsetstrokecolor{currentstroke}%
\pgfsetdash{}{0pt}%
\pgfsys@defobject{currentmarker}{\pgfqpoint{-0.020833in}{-0.020833in}}{\pgfqpoint{0.020833in}{0.020833in}}{%
\pgfpathmoveto{\pgfqpoint{0.000000in}{-0.020833in}}%
\pgfpathcurveto{\pgfqpoint{0.005525in}{-0.020833in}}{\pgfqpoint{0.010825in}{-0.018638in}}{\pgfqpoint{0.014731in}{-0.014731in}}%
\pgfpathcurveto{\pgfqpoint{0.018638in}{-0.010825in}}{\pgfqpoint{0.020833in}{-0.005525in}}{\pgfqpoint{0.020833in}{0.000000in}}%
\pgfpathcurveto{\pgfqpoint{0.020833in}{0.005525in}}{\pgfqpoint{0.018638in}{0.010825in}}{\pgfqpoint{0.014731in}{0.014731in}}%
\pgfpathcurveto{\pgfqpoint{0.010825in}{0.018638in}}{\pgfqpoint{0.005525in}{0.020833in}}{\pgfqpoint{0.000000in}{0.020833in}}%
\pgfpathcurveto{\pgfqpoint{-0.005525in}{0.020833in}}{\pgfqpoint{-0.010825in}{0.018638in}}{\pgfqpoint{-0.014731in}{0.014731in}}%
\pgfpathcurveto{\pgfqpoint{-0.018638in}{0.010825in}}{\pgfqpoint{-0.020833in}{0.005525in}}{\pgfqpoint{-0.020833in}{0.000000in}}%
\pgfpathcurveto{\pgfqpoint{-0.020833in}{-0.005525in}}{\pgfqpoint{-0.018638in}{-0.010825in}}{\pgfqpoint{-0.014731in}{-0.014731in}}%
\pgfpathcurveto{\pgfqpoint{-0.010825in}{-0.018638in}}{\pgfqpoint{-0.005525in}{-0.020833in}}{\pgfqpoint{0.000000in}{-0.020833in}}%
\pgfpathclose%
\pgfusepath{stroke,fill}%
}%
\begin{pgfscope}%
\pgfsys@transformshift{5.424200in}{1.530263in}%
\pgfsys@useobject{currentmarker}{}%
\end{pgfscope}%
\begin{pgfscope}%
\pgfsys@transformshift{4.605189in}{1.643755in}%
\pgfsys@useobject{currentmarker}{}%
\end{pgfscope}%
\begin{pgfscope}%
\pgfsys@transformshift{3.280000in}{3.300000in}%
\pgfsys@useobject{currentmarker}{}%
\end{pgfscope}%
\begin{pgfscope}%
\pgfsys@transformshift{1.954811in}{1.643755in}%
\pgfsys@useobject{currentmarker}{}%
\end{pgfscope}%
\begin{pgfscope}%
\pgfsys@transformshift{1.135800in}{1.530263in}%
\pgfsys@useobject{currentmarker}{}%
\end{pgfscope}%
\end{pgfscope}%
\begin{pgfscope}%
\pgfpathrectangle{\pgfqpoint{0.800000in}{0.528000in}}{\pgfqpoint{4.960000in}{3.696000in}}%
\pgfusepath{clip}%
\pgfsetrectcap%
\pgfsetroundjoin%
\pgfsetlinewidth{1.505625pt}%
\definecolor{currentstroke}{rgb}{0.090196,0.745098,0.811765}%
\pgfsetstrokecolor{currentstroke}%
\pgfsetdash{}{0pt}%
\pgfpathmoveto{\pgfqpoint{1.025455in}{1.828097in}}%
\pgfpathlineto{\pgfqpoint{1.048113in}{1.758056in}}%
\pgfpathlineto{\pgfqpoint{1.070772in}{1.692722in}}%
\pgfpathlineto{\pgfqpoint{1.093431in}{1.631972in}}%
\pgfpathlineto{\pgfqpoint{1.116090in}{1.575687in}}%
\pgfpathlineto{\pgfqpoint{1.138748in}{1.523746in}}%
\pgfpathlineto{\pgfqpoint{1.161407in}{1.476032in}}%
\pgfpathlineto{\pgfqpoint{1.184066in}{1.432427in}}%
\pgfpathlineto{\pgfqpoint{1.206725in}{1.392817in}}%
\pgfpathlineto{\pgfqpoint{1.229383in}{1.357087in}}%
\pgfpathlineto{\pgfqpoint{1.252042in}{1.325124in}}%
\pgfpathlineto{\pgfqpoint{1.274701in}{1.296816in}}%
\pgfpathlineto{\pgfqpoint{1.297360in}{1.272052in}}%
\pgfpathlineto{\pgfqpoint{1.320018in}{1.250723in}}%
\pgfpathlineto{\pgfqpoint{1.342677in}{1.232721in}}%
\pgfpathlineto{\pgfqpoint{1.365336in}{1.217939in}}%
\pgfpathlineto{\pgfqpoint{1.387995in}{1.206271in}}%
\pgfpathlineto{\pgfqpoint{1.410653in}{1.197613in}}%
\pgfpathlineto{\pgfqpoint{1.433312in}{1.191862in}}%
\pgfpathlineto{\pgfqpoint{1.455971in}{1.188916in}}%
\pgfpathlineto{\pgfqpoint{1.478630in}{1.188675in}}%
\pgfpathlineto{\pgfqpoint{1.501288in}{1.191038in}}%
\pgfpathlineto{\pgfqpoint{1.523947in}{1.195908in}}%
\pgfpathlineto{\pgfqpoint{1.546606in}{1.203187in}}%
\pgfpathlineto{\pgfqpoint{1.569265in}{1.212780in}}%
\pgfpathlineto{\pgfqpoint{1.591923in}{1.224593in}}%
\pgfpathlineto{\pgfqpoint{1.614582in}{1.238533in}}%
\pgfpathlineto{\pgfqpoint{1.637241in}{1.254506in}}%
\pgfpathlineto{\pgfqpoint{1.659899in}{1.272424in}}%
\pgfpathlineto{\pgfqpoint{1.682558in}{1.292195in}}%
\pgfpathlineto{\pgfqpoint{1.705217in}{1.313733in}}%
\pgfpathlineto{\pgfqpoint{1.727876in}{1.336949in}}%
\pgfpathlineto{\pgfqpoint{1.773193in}{1.388077in}}%
\pgfpathlineto{\pgfqpoint{1.818511in}{1.444907in}}%
\pgfpathlineto{\pgfqpoint{1.863828in}{1.506789in}}%
\pgfpathlineto{\pgfqpoint{1.909146in}{1.573092in}}%
\pgfpathlineto{\pgfqpoint{1.954463in}{1.643203in}}%
\pgfpathlineto{\pgfqpoint{1.999781in}{1.716532in}}%
\pgfpathlineto{\pgfqpoint{2.067757in}{1.831312in}}%
\pgfpathlineto{\pgfqpoint{2.158392in}{1.990444in}}%
\pgfpathlineto{\pgfqpoint{2.407638in}{2.433500in}}%
\pgfpathlineto{\pgfqpoint{2.475614in}{2.548790in}}%
\pgfpathlineto{\pgfqpoint{2.543591in}{2.659232in}}%
\pgfpathlineto{\pgfqpoint{2.588908in}{2.729600in}}%
\pgfpathlineto{\pgfqpoint{2.634226in}{2.796991in}}%
\pgfpathlineto{\pgfqpoint{2.679543in}{2.861111in}}%
\pgfpathlineto{\pgfqpoint{2.724861in}{2.921688in}}%
\pgfpathlineto{\pgfqpoint{2.770178in}{2.978468in}}%
\pgfpathlineto{\pgfqpoint{2.815496in}{3.031217in}}%
\pgfpathlineto{\pgfqpoint{2.860813in}{3.079722in}}%
\pgfpathlineto{\pgfqpoint{2.906131in}{3.123788in}}%
\pgfpathlineto{\pgfqpoint{2.951448in}{3.163242in}}%
\pgfpathlineto{\pgfqpoint{2.996766in}{3.197930in}}%
\pgfpathlineto{\pgfqpoint{3.042083in}{3.227717in}}%
\pgfpathlineto{\pgfqpoint{3.087401in}{3.252488in}}%
\pgfpathlineto{\pgfqpoint{3.110059in}{3.262963in}}%
\pgfpathlineto{\pgfqpoint{3.132718in}{3.272151in}}%
\pgfpathlineto{\pgfqpoint{3.155377in}{3.280042in}}%
\pgfpathlineto{\pgfqpoint{3.178036in}{3.286629in}}%
\pgfpathlineto{\pgfqpoint{3.200694in}{3.291906in}}%
\pgfpathlineto{\pgfqpoint{3.223353in}{3.295869in}}%
\pgfpathlineto{\pgfqpoint{3.246012in}{3.298512in}}%
\pgfpathlineto{\pgfqpoint{3.268671in}{3.299835in}}%
\pgfpathlineto{\pgfqpoint{3.291329in}{3.299835in}}%
\pgfpathlineto{\pgfqpoint{3.313988in}{3.298512in}}%
\pgfpathlineto{\pgfqpoint{3.336647in}{3.295869in}}%
\pgfpathlineto{\pgfqpoint{3.359306in}{3.291906in}}%
\pgfpathlineto{\pgfqpoint{3.381964in}{3.286629in}}%
\pgfpathlineto{\pgfqpoint{3.404623in}{3.280042in}}%
\pgfpathlineto{\pgfqpoint{3.427282in}{3.272151in}}%
\pgfpathlineto{\pgfqpoint{3.449941in}{3.262963in}}%
\pgfpathlineto{\pgfqpoint{3.472599in}{3.252488in}}%
\pgfpathlineto{\pgfqpoint{3.495258in}{3.240736in}}%
\pgfpathlineto{\pgfqpoint{3.540576in}{3.213443in}}%
\pgfpathlineto{\pgfqpoint{3.585893in}{3.181191in}}%
\pgfpathlineto{\pgfqpoint{3.631211in}{3.144102in}}%
\pgfpathlineto{\pgfqpoint{3.676528in}{3.102321in}}%
\pgfpathlineto{\pgfqpoint{3.721846in}{3.056013in}}%
\pgfpathlineto{\pgfqpoint{3.767163in}{3.005360in}}%
\pgfpathlineto{\pgfqpoint{3.812481in}{2.950568in}}%
\pgfpathlineto{\pgfqpoint{3.857798in}{2.891859in}}%
\pgfpathlineto{\pgfqpoint{3.903116in}{2.829477in}}%
\pgfpathlineto{\pgfqpoint{3.948433in}{2.763686in}}%
\pgfpathlineto{\pgfqpoint{3.993751in}{2.694769in}}%
\pgfpathlineto{\pgfqpoint{4.061727in}{2.586201in}}%
\pgfpathlineto{\pgfqpoint{4.129703in}{2.472394in}}%
\pgfpathlineto{\pgfqpoint{4.197679in}{2.354556in}}%
\pgfpathlineto{\pgfqpoint{4.310973in}{2.152800in}}%
\pgfpathlineto{\pgfqpoint{4.446926in}{1.910226in}}%
\pgfpathlineto{\pgfqpoint{4.514902in}{1.792506in}}%
\pgfpathlineto{\pgfqpoint{4.582878in}{1.679502in}}%
\pgfpathlineto{\pgfqpoint{4.628196in}{1.607709in}}%
\pgfpathlineto{\pgfqpoint{4.673513in}{1.539427in}}%
\pgfpathlineto{\pgfqpoint{4.718831in}{1.475257in}}%
\pgfpathlineto{\pgfqpoint{4.764148in}{1.415820in}}%
\pgfpathlineto{\pgfqpoint{4.809466in}{1.361758in}}%
\pgfpathlineto{\pgfqpoint{4.854783in}{1.313733in}}%
\pgfpathlineto{\pgfqpoint{4.877442in}{1.292195in}}%
\pgfpathlineto{\pgfqpoint{4.900101in}{1.272424in}}%
\pgfpathlineto{\pgfqpoint{4.922759in}{1.254506in}}%
\pgfpathlineto{\pgfqpoint{4.945418in}{1.238533in}}%
\pgfpathlineto{\pgfqpoint{4.968077in}{1.224593in}}%
\pgfpathlineto{\pgfqpoint{4.990735in}{1.212780in}}%
\pgfpathlineto{\pgfqpoint{5.013394in}{1.203187in}}%
\pgfpathlineto{\pgfqpoint{5.036053in}{1.195908in}}%
\pgfpathlineto{\pgfqpoint{5.058712in}{1.191038in}}%
\pgfpathlineto{\pgfqpoint{5.081370in}{1.188675in}}%
\pgfpathlineto{\pgfqpoint{5.104029in}{1.188916in}}%
\pgfpathlineto{\pgfqpoint{5.126688in}{1.191862in}}%
\pgfpathlineto{\pgfqpoint{5.149347in}{1.197613in}}%
\pgfpathlineto{\pgfqpoint{5.172005in}{1.206271in}}%
\pgfpathlineto{\pgfqpoint{5.194664in}{1.217939in}}%
\pgfpathlineto{\pgfqpoint{5.217323in}{1.232721in}}%
\pgfpathlineto{\pgfqpoint{5.239982in}{1.250723in}}%
\pgfpathlineto{\pgfqpoint{5.262640in}{1.272052in}}%
\pgfpathlineto{\pgfqpoint{5.285299in}{1.296816in}}%
\pgfpathlineto{\pgfqpoint{5.307958in}{1.325124in}}%
\pgfpathlineto{\pgfqpoint{5.330617in}{1.357087in}}%
\pgfpathlineto{\pgfqpoint{5.353275in}{1.392817in}}%
\pgfpathlineto{\pgfqpoint{5.375934in}{1.432427in}}%
\pgfpathlineto{\pgfqpoint{5.398593in}{1.476032in}}%
\pgfpathlineto{\pgfqpoint{5.421252in}{1.523746in}}%
\pgfpathlineto{\pgfqpoint{5.443910in}{1.575687in}}%
\pgfpathlineto{\pgfqpoint{5.466569in}{1.631972in}}%
\pgfpathlineto{\pgfqpoint{5.489228in}{1.692722in}}%
\pgfpathlineto{\pgfqpoint{5.511887in}{1.758056in}}%
\pgfpathlineto{\pgfqpoint{5.534545in}{1.828097in}}%
\pgfpathlineto{\pgfqpoint{5.534545in}{1.828097in}}%
\pgfusepath{stroke}%
\end{pgfscope}%
\begin{pgfscope}%
\pgfpathrectangle{\pgfqpoint{0.800000in}{0.528000in}}{\pgfqpoint{4.960000in}{3.696000in}}%
\pgfusepath{clip}%
\pgfsetrectcap%
\pgfsetroundjoin%
\pgfsetlinewidth{1.505625pt}%
\definecolor{currentstroke}{rgb}{0.121569,0.466667,0.705882}%
\pgfsetstrokecolor{currentstroke}%
\pgfsetdash{}{0pt}%
\pgfpathmoveto{\pgfqpoint{1.025455in}{2.376000in}}%
\pgfpathlineto{\pgfqpoint{1.138748in}{2.423597in}}%
\pgfpathlineto{\pgfqpoint{1.252042in}{2.473505in}}%
\pgfpathlineto{\pgfqpoint{1.365336in}{2.525658in}}%
\pgfpathlineto{\pgfqpoint{1.478630in}{2.579935in}}%
\pgfpathlineto{\pgfqpoint{1.614582in}{2.647600in}}%
\pgfpathlineto{\pgfqpoint{1.773193in}{2.729407in}}%
\pgfpathlineto{\pgfqpoint{1.999781in}{2.849415in}}%
\pgfpathlineto{\pgfqpoint{2.226368in}{2.968739in}}%
\pgfpathlineto{\pgfqpoint{2.339662in}{3.026159in}}%
\pgfpathlineto{\pgfqpoint{2.430297in}{3.070154in}}%
\pgfpathlineto{\pgfqpoint{2.520932in}{3.111847in}}%
\pgfpathlineto{\pgfqpoint{2.611567in}{3.150683in}}%
\pgfpathlineto{\pgfqpoint{2.679543in}{3.177599in}}%
\pgfpathlineto{\pgfqpoint{2.747519in}{3.202362in}}%
\pgfpathlineto{\pgfqpoint{2.815496in}{3.224750in}}%
\pgfpathlineto{\pgfqpoint{2.883472in}{3.244550in}}%
\pgfpathlineto{\pgfqpoint{2.951448in}{3.261570in}}%
\pgfpathlineto{\pgfqpoint{3.019424in}{3.275639in}}%
\pgfpathlineto{\pgfqpoint{3.087401in}{3.286611in}}%
\pgfpathlineto{\pgfqpoint{3.155377in}{3.294371in}}%
\pgfpathlineto{\pgfqpoint{3.223353in}{3.298834in}}%
\pgfpathlineto{\pgfqpoint{3.268671in}{3.299953in}}%
\pgfpathlineto{\pgfqpoint{3.313988in}{3.299580in}}%
\pgfpathlineto{\pgfqpoint{3.359306in}{3.297716in}}%
\pgfpathlineto{\pgfqpoint{3.427282in}{3.292147in}}%
\pgfpathlineto{\pgfqpoint{3.495258in}{3.283306in}}%
\pgfpathlineto{\pgfqpoint{3.563234in}{3.271287in}}%
\pgfpathlineto{\pgfqpoint{3.631211in}{3.256217in}}%
\pgfpathlineto{\pgfqpoint{3.699187in}{3.238250in}}%
\pgfpathlineto{\pgfqpoint{3.767163in}{3.217564in}}%
\pgfpathlineto{\pgfqpoint{3.835139in}{3.194361in}}%
\pgfpathlineto{\pgfqpoint{3.903116in}{3.168855in}}%
\pgfpathlineto{\pgfqpoint{3.971092in}{3.141272in}}%
\pgfpathlineto{\pgfqpoint{4.061727in}{3.101670in}}%
\pgfpathlineto{\pgfqpoint{4.152362in}{3.059350in}}%
\pgfpathlineto{\pgfqpoint{4.265656in}{3.003466in}}%
\pgfpathlineto{\pgfqpoint{4.401608in}{2.933370in}}%
\pgfpathlineto{\pgfqpoint{4.628196in}{2.813234in}}%
\pgfpathlineto{\pgfqpoint{4.854783in}{2.694026in}}%
\pgfpathlineto{\pgfqpoint{4.990735in}{2.624761in}}%
\pgfpathlineto{\pgfqpoint{5.126688in}{2.557979in}}%
\pgfpathlineto{\pgfqpoint{5.239982in}{2.504533in}}%
\pgfpathlineto{\pgfqpoint{5.353275in}{2.453268in}}%
\pgfpathlineto{\pgfqpoint{5.466569in}{2.404279in}}%
\pgfpathlineto{\pgfqpoint{5.534545in}{2.376000in}}%
\pgfpathlineto{\pgfqpoint{5.534545in}{2.376000in}}%
\pgfusepath{stroke}%
\end{pgfscope}%
\begin{pgfscope}%
\pgfpathrectangle{\pgfqpoint{0.800000in}{0.528000in}}{\pgfqpoint{4.960000in}{3.696000in}}%
\pgfusepath{clip}%
\pgfsetbuttcap%
\pgfsetroundjoin%
\definecolor{currentfill}{rgb}{1.000000,0.498039,0.054902}%
\pgfsetfillcolor{currentfill}%
\pgfsetlinewidth{1.003750pt}%
\definecolor{currentstroke}{rgb}{1.000000,0.498039,0.054902}%
\pgfsetstrokecolor{currentstroke}%
\pgfsetdash{}{0pt}%
\pgfsys@defobject{currentmarker}{\pgfqpoint{-0.020833in}{-0.020833in}}{\pgfqpoint{0.020833in}{0.020833in}}{%
\pgfpathmoveto{\pgfqpoint{0.000000in}{-0.020833in}}%
\pgfpathcurveto{\pgfqpoint{0.005525in}{-0.020833in}}{\pgfqpoint{0.010825in}{-0.018638in}}{\pgfqpoint{0.014731in}{-0.014731in}}%
\pgfpathcurveto{\pgfqpoint{0.018638in}{-0.010825in}}{\pgfqpoint{0.020833in}{-0.005525in}}{\pgfqpoint{0.020833in}{0.000000in}}%
\pgfpathcurveto{\pgfqpoint{0.020833in}{0.005525in}}{\pgfqpoint{0.018638in}{0.010825in}}{\pgfqpoint{0.014731in}{0.014731in}}%
\pgfpathcurveto{\pgfqpoint{0.010825in}{0.018638in}}{\pgfqpoint{0.005525in}{0.020833in}}{\pgfqpoint{0.000000in}{0.020833in}}%
\pgfpathcurveto{\pgfqpoint{-0.005525in}{0.020833in}}{\pgfqpoint{-0.010825in}{0.018638in}}{\pgfqpoint{-0.014731in}{0.014731in}}%
\pgfpathcurveto{\pgfqpoint{-0.018638in}{0.010825in}}{\pgfqpoint{-0.020833in}{0.005525in}}{\pgfqpoint{-0.020833in}{0.000000in}}%
\pgfpathcurveto{\pgfqpoint{-0.020833in}{-0.005525in}}{\pgfqpoint{-0.018638in}{-0.010825in}}{\pgfqpoint{-0.014731in}{-0.014731in}}%
\pgfpathcurveto{\pgfqpoint{-0.010825in}{-0.018638in}}{\pgfqpoint{-0.005525in}{-0.020833in}}{\pgfqpoint{0.000000in}{-0.020833in}}%
\pgfpathclose%
\pgfusepath{stroke,fill}%
}%
\begin{pgfscope}%
\pgfsys@transformshift{1.025455in}{2.376000in}%
\pgfsys@useobject{currentmarker}{}%
\end{pgfscope}%
\begin{pgfscope}%
\pgfsys@transformshift{2.528485in}{3.115200in}%
\pgfsys@useobject{currentmarker}{}%
\end{pgfscope}%
\begin{pgfscope}%
\pgfsys@transformshift{4.031515in}{3.115200in}%
\pgfsys@useobject{currentmarker}{}%
\end{pgfscope}%
\end{pgfscope}%
\begin{pgfscope}%
\pgfpathrectangle{\pgfqpoint{0.800000in}{0.528000in}}{\pgfqpoint{4.960000in}{3.696000in}}%
\pgfusepath{clip}%
\pgfsetrectcap%
\pgfsetroundjoin%
\pgfsetlinewidth{1.505625pt}%
\definecolor{currentstroke}{rgb}{0.172549,0.627451,0.172549}%
\pgfsetstrokecolor{currentstroke}%
\pgfsetdash{}{0pt}%
\pgfpathmoveto{\pgfqpoint{1.025455in}{2.376000in}}%
\pgfpathlineto{\pgfqpoint{1.116090in}{2.441518in}}%
\pgfpathlineto{\pgfqpoint{1.206725in}{2.504349in}}%
\pgfpathlineto{\pgfqpoint{1.297360in}{2.564491in}}%
\pgfpathlineto{\pgfqpoint{1.387995in}{2.621946in}}%
\pgfpathlineto{\pgfqpoint{1.478630in}{2.676712in}}%
\pgfpathlineto{\pgfqpoint{1.569265in}{2.728791in}}%
\pgfpathlineto{\pgfqpoint{1.659899in}{2.778182in}}%
\pgfpathlineto{\pgfqpoint{1.750534in}{2.824885in}}%
\pgfpathlineto{\pgfqpoint{1.841169in}{2.868900in}}%
\pgfpathlineto{\pgfqpoint{1.931804in}{2.910227in}}%
\pgfpathlineto{\pgfqpoint{2.022439in}{2.948866in}}%
\pgfpathlineto{\pgfqpoint{2.113074in}{2.984817in}}%
\pgfpathlineto{\pgfqpoint{2.203709in}{3.018080in}}%
\pgfpathlineto{\pgfqpoint{2.294344in}{3.048655in}}%
\pgfpathlineto{\pgfqpoint{2.384979in}{3.076542in}}%
\pgfpathlineto{\pgfqpoint{2.475614in}{3.101742in}}%
\pgfpathlineto{\pgfqpoint{2.566249in}{3.124253in}}%
\pgfpathlineto{\pgfqpoint{2.656884in}{3.144077in}}%
\pgfpathlineto{\pgfqpoint{2.747519in}{3.161212in}}%
\pgfpathlineto{\pgfqpoint{2.838154in}{3.175660in}}%
\pgfpathlineto{\pgfqpoint{2.928789in}{3.187420in}}%
\pgfpathlineto{\pgfqpoint{3.019424in}{3.196491in}}%
\pgfpathlineto{\pgfqpoint{3.110059in}{3.202875in}}%
\pgfpathlineto{\pgfqpoint{3.200694in}{3.206571in}}%
\pgfpathlineto{\pgfqpoint{3.291329in}{3.207579in}}%
\pgfpathlineto{\pgfqpoint{3.381964in}{3.205899in}}%
\pgfpathlineto{\pgfqpoint{3.472599in}{3.201531in}}%
\pgfpathlineto{\pgfqpoint{3.563234in}{3.194475in}}%
\pgfpathlineto{\pgfqpoint{3.653869in}{3.184732in}}%
\pgfpathlineto{\pgfqpoint{3.744504in}{3.172300in}}%
\pgfpathlineto{\pgfqpoint{3.835139in}{3.157180in}}%
\pgfpathlineto{\pgfqpoint{3.925774in}{3.139373in}}%
\pgfpathlineto{\pgfqpoint{4.016409in}{3.118877in}}%
\pgfpathlineto{\pgfqpoint{4.107044in}{3.095694in}}%
\pgfpathlineto{\pgfqpoint{4.197679in}{3.069822in}}%
\pgfpathlineto{\pgfqpoint{4.288314in}{3.041263in}}%
\pgfpathlineto{\pgfqpoint{4.378949in}{3.010016in}}%
\pgfpathlineto{\pgfqpoint{4.469584in}{2.976081in}}%
\pgfpathlineto{\pgfqpoint{4.560219in}{2.939458in}}%
\pgfpathlineto{\pgfqpoint{4.650854in}{2.900147in}}%
\pgfpathlineto{\pgfqpoint{4.741489in}{2.858148in}}%
\pgfpathlineto{\pgfqpoint{4.832124in}{2.813461in}}%
\pgfpathlineto{\pgfqpoint{4.922759in}{2.766086in}}%
\pgfpathlineto{\pgfqpoint{5.013394in}{2.716023in}}%
\pgfpathlineto{\pgfqpoint{5.104029in}{2.663273in}}%
\pgfpathlineto{\pgfqpoint{5.194664in}{2.607834in}}%
\pgfpathlineto{\pgfqpoint{5.285299in}{2.549708in}}%
\pgfpathlineto{\pgfqpoint{5.375934in}{2.488893in}}%
\pgfpathlineto{\pgfqpoint{5.466569in}{2.425391in}}%
\pgfpathlineto{\pgfqpoint{5.534545in}{2.376000in}}%
\pgfpathlineto{\pgfqpoint{5.534545in}{2.376000in}}%
\pgfusepath{stroke}%
\end{pgfscope}%
\begin{pgfscope}%
\pgfpathrectangle{\pgfqpoint{0.800000in}{0.528000in}}{\pgfqpoint{4.960000in}{3.696000in}}%
\pgfusepath{clip}%
\pgfsetbuttcap%
\pgfsetroundjoin%
\definecolor{currentfill}{rgb}{0.839216,0.152941,0.156863}%
\pgfsetfillcolor{currentfill}%
\pgfsetlinewidth{1.003750pt}%
\definecolor{currentstroke}{rgb}{0.839216,0.152941,0.156863}%
\pgfsetstrokecolor{currentstroke}%
\pgfsetdash{}{0pt}%
\pgfsys@defobject{currentmarker}{\pgfqpoint{-0.020833in}{-0.020833in}}{\pgfqpoint{0.020833in}{0.020833in}}{%
\pgfpathmoveto{\pgfqpoint{0.000000in}{-0.020833in}}%
\pgfpathcurveto{\pgfqpoint{0.005525in}{-0.020833in}}{\pgfqpoint{0.010825in}{-0.018638in}}{\pgfqpoint{0.014731in}{-0.014731in}}%
\pgfpathcurveto{\pgfqpoint{0.018638in}{-0.010825in}}{\pgfqpoint{0.020833in}{-0.005525in}}{\pgfqpoint{0.020833in}{0.000000in}}%
\pgfpathcurveto{\pgfqpoint{0.020833in}{0.005525in}}{\pgfqpoint{0.018638in}{0.010825in}}{\pgfqpoint{0.014731in}{0.014731in}}%
\pgfpathcurveto{\pgfqpoint{0.010825in}{0.018638in}}{\pgfqpoint{0.005525in}{0.020833in}}{\pgfqpoint{0.000000in}{0.020833in}}%
\pgfpathcurveto{\pgfqpoint{-0.005525in}{0.020833in}}{\pgfqpoint{-0.010825in}{0.018638in}}{\pgfqpoint{-0.014731in}{0.014731in}}%
\pgfpathcurveto{\pgfqpoint{-0.018638in}{0.010825in}}{\pgfqpoint{-0.020833in}{0.005525in}}{\pgfqpoint{-0.020833in}{0.000000in}}%
\pgfpathcurveto{\pgfqpoint{-0.020833in}{-0.005525in}}{\pgfqpoint{-0.018638in}{-0.010825in}}{\pgfqpoint{-0.014731in}{-0.014731in}}%
\pgfpathcurveto{\pgfqpoint{-0.010825in}{-0.018638in}}{\pgfqpoint{-0.005525in}{-0.020833in}}{\pgfqpoint{0.000000in}{-0.020833in}}%
\pgfpathclose%
\pgfusepath{stroke,fill}%
}%
\begin{pgfscope}%
\pgfsys@transformshift{1.025455in}{2.376000in}%
\pgfsys@useobject{currentmarker}{}%
\end{pgfscope}%
\begin{pgfscope}%
\pgfsys@transformshift{1.927273in}{2.810824in}%
\pgfsys@useobject{currentmarker}{}%
\end{pgfscope}%
\begin{pgfscope}%
\pgfsys@transformshift{2.829091in}{3.228923in}%
\pgfsys@useobject{currentmarker}{}%
\end{pgfscope}%
\begin{pgfscope}%
\pgfsys@transformshift{3.730909in}{3.228923in}%
\pgfsys@useobject{currentmarker}{}%
\end{pgfscope}%
\begin{pgfscope}%
\pgfsys@transformshift{4.632727in}{2.810824in}%
\pgfsys@useobject{currentmarker}{}%
\end{pgfscope}%
\end{pgfscope}%
\begin{pgfscope}%
\pgfpathrectangle{\pgfqpoint{0.800000in}{0.528000in}}{\pgfqpoint{4.960000in}{3.696000in}}%
\pgfusepath{clip}%
\pgfsetrectcap%
\pgfsetroundjoin%
\pgfsetlinewidth{1.505625pt}%
\definecolor{currentstroke}{rgb}{0.580392,0.403922,0.741176}%
\pgfsetstrokecolor{currentstroke}%
\pgfsetdash{}{0pt}%
\pgfpathmoveto{\pgfqpoint{1.025455in}{2.376000in}}%
\pgfpathlineto{\pgfqpoint{1.070772in}{2.387434in}}%
\pgfpathlineto{\pgfqpoint{1.116090in}{2.400643in}}%
\pgfpathlineto{\pgfqpoint{1.161407in}{2.415503in}}%
\pgfpathlineto{\pgfqpoint{1.229383in}{2.440624in}}%
\pgfpathlineto{\pgfqpoint{1.297360in}{2.468792in}}%
\pgfpathlineto{\pgfqpoint{1.365336in}{2.499621in}}%
\pgfpathlineto{\pgfqpoint{1.433312in}{2.532741in}}%
\pgfpathlineto{\pgfqpoint{1.523947in}{2.579845in}}%
\pgfpathlineto{\pgfqpoint{1.614582in}{2.629571in}}%
\pgfpathlineto{\pgfqpoint{1.727876in}{2.694242in}}%
\pgfpathlineto{\pgfqpoint{2.135733in}{2.930066in}}%
\pgfpathlineto{\pgfqpoint{2.226368in}{2.979325in}}%
\pgfpathlineto{\pgfqpoint{2.317003in}{3.026287in}}%
\pgfpathlineto{\pgfqpoint{2.407638in}{3.070497in}}%
\pgfpathlineto{\pgfqpoint{2.475614in}{3.101596in}}%
\pgfpathlineto{\pgfqpoint{2.543591in}{3.130754in}}%
\pgfpathlineto{\pgfqpoint{2.611567in}{3.157822in}}%
\pgfpathlineto{\pgfqpoint{2.679543in}{3.182666in}}%
\pgfpathlineto{\pgfqpoint{2.747519in}{3.205167in}}%
\pgfpathlineto{\pgfqpoint{2.815496in}{3.225216in}}%
\pgfpathlineto{\pgfqpoint{2.883472in}{3.242718in}}%
\pgfpathlineto{\pgfqpoint{2.951448in}{3.257591in}}%
\pgfpathlineto{\pgfqpoint{3.019424in}{3.269765in}}%
\pgfpathlineto{\pgfqpoint{3.087401in}{3.279186in}}%
\pgfpathlineto{\pgfqpoint{3.155377in}{3.285808in}}%
\pgfpathlineto{\pgfqpoint{3.223353in}{3.289603in}}%
\pgfpathlineto{\pgfqpoint{3.291329in}{3.290553in}}%
\pgfpathlineto{\pgfqpoint{3.359306in}{3.288654in}}%
\pgfpathlineto{\pgfqpoint{3.427282in}{3.283914in}}%
\pgfpathlineto{\pgfqpoint{3.495258in}{3.276354in}}%
\pgfpathlineto{\pgfqpoint{3.563234in}{3.266011in}}%
\pgfpathlineto{\pgfqpoint{3.631211in}{3.252930in}}%
\pgfpathlineto{\pgfqpoint{3.699187in}{3.237172in}}%
\pgfpathlineto{\pgfqpoint{3.767163in}{3.218812in}}%
\pgfpathlineto{\pgfqpoint{3.835139in}{3.197934in}}%
\pgfpathlineto{\pgfqpoint{3.903116in}{3.174640in}}%
\pgfpathlineto{\pgfqpoint{3.971092in}{3.149040in}}%
\pgfpathlineto{\pgfqpoint{4.039068in}{3.121260in}}%
\pgfpathlineto{\pgfqpoint{4.107044in}{3.091438in}}%
\pgfpathlineto{\pgfqpoint{4.197679in}{3.048763in}}%
\pgfpathlineto{\pgfqpoint{4.288314in}{3.003123in}}%
\pgfpathlineto{\pgfqpoint{4.378949in}{2.954952in}}%
\pgfpathlineto{\pgfqpoint{4.492243in}{2.891910in}}%
\pgfpathlineto{\pgfqpoint{4.628196in}{2.813470in}}%
\pgfpathlineto{\pgfqpoint{4.922759in}{2.642321in}}%
\pgfpathlineto{\pgfqpoint{5.036053in}{2.579845in}}%
\pgfpathlineto{\pgfqpoint{5.126688in}{2.532741in}}%
\pgfpathlineto{\pgfqpoint{5.194664in}{2.499621in}}%
\pgfpathlineto{\pgfqpoint{5.262640in}{2.468792in}}%
\pgfpathlineto{\pgfqpoint{5.330617in}{2.440624in}}%
\pgfpathlineto{\pgfqpoint{5.398593in}{2.415503in}}%
\pgfpathlineto{\pgfqpoint{5.443910in}{2.400643in}}%
\pgfpathlineto{\pgfqpoint{5.489228in}{2.387434in}}%
\pgfpathlineto{\pgfqpoint{5.534545in}{2.376000in}}%
\pgfpathlineto{\pgfqpoint{5.534545in}{2.376000in}}%
\pgfusepath{stroke}%
\end{pgfscope}%
\begin{pgfscope}%
\pgfpathrectangle{\pgfqpoint{0.800000in}{0.528000in}}{\pgfqpoint{4.960000in}{3.696000in}}%
\pgfusepath{clip}%
\pgfsetrectcap%
\pgfsetroundjoin%
\pgfsetlinewidth{1.505625pt}%
\definecolor{currentstroke}{rgb}{0.549020,0.337255,0.294118}%
\pgfsetstrokecolor{currentstroke}%
\pgfsetdash{}{0pt}%
\pgfpathmoveto{\pgfqpoint{1.025455in}{2.376000in}}%
\pgfpathlineto{\pgfqpoint{1.138748in}{2.423597in}}%
\pgfpathlineto{\pgfqpoint{1.252042in}{2.473505in}}%
\pgfpathlineto{\pgfqpoint{1.365336in}{2.525658in}}%
\pgfpathlineto{\pgfqpoint{1.478630in}{2.579935in}}%
\pgfpathlineto{\pgfqpoint{1.614582in}{2.647600in}}%
\pgfpathlineto{\pgfqpoint{1.773193in}{2.729407in}}%
\pgfpathlineto{\pgfqpoint{1.999781in}{2.849415in}}%
\pgfpathlineto{\pgfqpoint{2.226368in}{2.968739in}}%
\pgfpathlineto{\pgfqpoint{2.339662in}{3.026159in}}%
\pgfpathlineto{\pgfqpoint{2.430297in}{3.070154in}}%
\pgfpathlineto{\pgfqpoint{2.520932in}{3.111847in}}%
\pgfpathlineto{\pgfqpoint{2.611567in}{3.150683in}}%
\pgfpathlineto{\pgfqpoint{2.679543in}{3.177599in}}%
\pgfpathlineto{\pgfqpoint{2.747519in}{3.202362in}}%
\pgfpathlineto{\pgfqpoint{2.815496in}{3.224750in}}%
\pgfpathlineto{\pgfqpoint{2.883472in}{3.244550in}}%
\pgfpathlineto{\pgfqpoint{2.951448in}{3.261570in}}%
\pgfpathlineto{\pgfqpoint{3.019424in}{3.275639in}}%
\pgfpathlineto{\pgfqpoint{3.087401in}{3.286611in}}%
\pgfpathlineto{\pgfqpoint{3.155377in}{3.294371in}}%
\pgfpathlineto{\pgfqpoint{3.223353in}{3.298834in}}%
\pgfpathlineto{\pgfqpoint{3.268671in}{3.299953in}}%
\pgfpathlineto{\pgfqpoint{3.313988in}{3.299580in}}%
\pgfpathlineto{\pgfqpoint{3.359306in}{3.297716in}}%
\pgfpathlineto{\pgfqpoint{3.427282in}{3.292147in}}%
\pgfpathlineto{\pgfqpoint{3.495258in}{3.283306in}}%
\pgfpathlineto{\pgfqpoint{3.563234in}{3.271287in}}%
\pgfpathlineto{\pgfqpoint{3.631211in}{3.256217in}}%
\pgfpathlineto{\pgfqpoint{3.699187in}{3.238250in}}%
\pgfpathlineto{\pgfqpoint{3.767163in}{3.217564in}}%
\pgfpathlineto{\pgfqpoint{3.835139in}{3.194361in}}%
\pgfpathlineto{\pgfqpoint{3.903116in}{3.168855in}}%
\pgfpathlineto{\pgfqpoint{3.971092in}{3.141272in}}%
\pgfpathlineto{\pgfqpoint{4.061727in}{3.101670in}}%
\pgfpathlineto{\pgfqpoint{4.152362in}{3.059350in}}%
\pgfpathlineto{\pgfqpoint{4.265656in}{3.003466in}}%
\pgfpathlineto{\pgfqpoint{4.401608in}{2.933370in}}%
\pgfpathlineto{\pgfqpoint{4.628196in}{2.813234in}}%
\pgfpathlineto{\pgfqpoint{4.854783in}{2.694026in}}%
\pgfpathlineto{\pgfqpoint{4.990735in}{2.624761in}}%
\pgfpathlineto{\pgfqpoint{5.126688in}{2.557979in}}%
\pgfpathlineto{\pgfqpoint{5.239982in}{2.504533in}}%
\pgfpathlineto{\pgfqpoint{5.353275in}{2.453268in}}%
\pgfpathlineto{\pgfqpoint{5.466569in}{2.404279in}}%
\pgfpathlineto{\pgfqpoint{5.534545in}{2.376000in}}%
\pgfpathlineto{\pgfqpoint{5.534545in}{2.376000in}}%
\pgfusepath{stroke}%
\end{pgfscope}%
\begin{pgfscope}%
\pgfpathrectangle{\pgfqpoint{0.800000in}{0.528000in}}{\pgfqpoint{4.960000in}{3.696000in}}%
\pgfusepath{clip}%
\pgfsetbuttcap%
\pgfsetroundjoin%
\definecolor{currentfill}{rgb}{0.890196,0.466667,0.760784}%
\pgfsetfillcolor{currentfill}%
\pgfsetlinewidth{1.003750pt}%
\definecolor{currentstroke}{rgb}{0.890196,0.466667,0.760784}%
\pgfsetstrokecolor{currentstroke}%
\pgfsetdash{}{0pt}%
\pgfsys@defobject{currentmarker}{\pgfqpoint{-0.020833in}{-0.020833in}}{\pgfqpoint{0.020833in}{0.020833in}}{%
\pgfpathmoveto{\pgfqpoint{0.000000in}{-0.020833in}}%
\pgfpathcurveto{\pgfqpoint{0.005525in}{-0.020833in}}{\pgfqpoint{0.010825in}{-0.018638in}}{\pgfqpoint{0.014731in}{-0.014731in}}%
\pgfpathcurveto{\pgfqpoint{0.018638in}{-0.010825in}}{\pgfqpoint{0.020833in}{-0.005525in}}{\pgfqpoint{0.020833in}{0.000000in}}%
\pgfpathcurveto{\pgfqpoint{0.020833in}{0.005525in}}{\pgfqpoint{0.018638in}{0.010825in}}{\pgfqpoint{0.014731in}{0.014731in}}%
\pgfpathcurveto{\pgfqpoint{0.010825in}{0.018638in}}{\pgfqpoint{0.005525in}{0.020833in}}{\pgfqpoint{0.000000in}{0.020833in}}%
\pgfpathcurveto{\pgfqpoint{-0.005525in}{0.020833in}}{\pgfqpoint{-0.010825in}{0.018638in}}{\pgfqpoint{-0.014731in}{0.014731in}}%
\pgfpathcurveto{\pgfqpoint{-0.018638in}{0.010825in}}{\pgfqpoint{-0.020833in}{0.005525in}}{\pgfqpoint{-0.020833in}{0.000000in}}%
\pgfpathcurveto{\pgfqpoint{-0.020833in}{-0.005525in}}{\pgfqpoint{-0.018638in}{-0.010825in}}{\pgfqpoint{-0.014731in}{-0.014731in}}%
\pgfpathcurveto{\pgfqpoint{-0.010825in}{-0.018638in}}{\pgfqpoint{-0.005525in}{-0.020833in}}{\pgfqpoint{0.000000in}{-0.020833in}}%
\pgfpathclose%
\pgfusepath{stroke,fill}%
}%
\begin{pgfscope}%
\pgfsys@transformshift{5.232494in}{2.508000in}%
\pgfsys@useobject{currentmarker}{}%
\end{pgfscope}%
\begin{pgfscope}%
\pgfsys@transformshift{3.280000in}{3.300000in}%
\pgfsys@useobject{currentmarker}{}%
\end{pgfscope}%
\begin{pgfscope}%
\pgfsys@transformshift{1.327506in}{2.508000in}%
\pgfsys@useobject{currentmarker}{}%
\end{pgfscope}%
\end{pgfscope}%
\begin{pgfscope}%
\pgfpathrectangle{\pgfqpoint{0.800000in}{0.528000in}}{\pgfqpoint{4.960000in}{3.696000in}}%
\pgfusepath{clip}%
\pgfsetrectcap%
\pgfsetroundjoin%
\pgfsetlinewidth{1.505625pt}%
\definecolor{currentstroke}{rgb}{0.498039,0.498039,0.498039}%
\pgfsetstrokecolor{currentstroke}%
\pgfsetdash{}{0pt}%
\pgfpathmoveto{\pgfqpoint{1.025455in}{2.244000in}}%
\pgfpathlineto{\pgfqpoint{1.116090in}{2.327198in}}%
\pgfpathlineto{\pgfqpoint{1.206725in}{2.406983in}}%
\pgfpathlineto{\pgfqpoint{1.297360in}{2.483354in}}%
\pgfpathlineto{\pgfqpoint{1.387995in}{2.556312in}}%
\pgfpathlineto{\pgfqpoint{1.478630in}{2.625857in}}%
\pgfpathlineto{\pgfqpoint{1.569265in}{2.691989in}}%
\pgfpathlineto{\pgfqpoint{1.659899in}{2.754707in}}%
\pgfpathlineto{\pgfqpoint{1.750534in}{2.814012in}}%
\pgfpathlineto{\pgfqpoint{1.841169in}{2.869904in}}%
\pgfpathlineto{\pgfqpoint{1.931804in}{2.922383in}}%
\pgfpathlineto{\pgfqpoint{1.999781in}{2.959502in}}%
\pgfpathlineto{\pgfqpoint{2.067757in}{2.994701in}}%
\pgfpathlineto{\pgfqpoint{2.135733in}{3.027980in}}%
\pgfpathlineto{\pgfqpoint{2.203709in}{3.059339in}}%
\pgfpathlineto{\pgfqpoint{2.271686in}{3.088779in}}%
\pgfpathlineto{\pgfqpoint{2.339662in}{3.116298in}}%
\pgfpathlineto{\pgfqpoint{2.407638in}{3.141897in}}%
\pgfpathlineto{\pgfqpoint{2.475614in}{3.165577in}}%
\pgfpathlineto{\pgfqpoint{2.543591in}{3.187336in}}%
\pgfpathlineto{\pgfqpoint{2.611567in}{3.207176in}}%
\pgfpathlineto{\pgfqpoint{2.679543in}{3.225095in}}%
\pgfpathlineto{\pgfqpoint{2.747519in}{3.241095in}}%
\pgfpathlineto{\pgfqpoint{2.815496in}{3.255174in}}%
\pgfpathlineto{\pgfqpoint{2.883472in}{3.267334in}}%
\pgfpathlineto{\pgfqpoint{2.951448in}{3.277574in}}%
\pgfpathlineto{\pgfqpoint{3.019424in}{3.285894in}}%
\pgfpathlineto{\pgfqpoint{3.087401in}{3.292294in}}%
\pgfpathlineto{\pgfqpoint{3.155377in}{3.296773in}}%
\pgfpathlineto{\pgfqpoint{3.223353in}{3.299333in}}%
\pgfpathlineto{\pgfqpoint{3.291329in}{3.299973in}}%
\pgfpathlineto{\pgfqpoint{3.359306in}{3.298693in}}%
\pgfpathlineto{\pgfqpoint{3.427282in}{3.295493in}}%
\pgfpathlineto{\pgfqpoint{3.495258in}{3.290374in}}%
\pgfpathlineto{\pgfqpoint{3.563234in}{3.283334in}}%
\pgfpathlineto{\pgfqpoint{3.631211in}{3.274374in}}%
\pgfpathlineto{\pgfqpoint{3.699187in}{3.263494in}}%
\pgfpathlineto{\pgfqpoint{3.767163in}{3.250695in}}%
\pgfpathlineto{\pgfqpoint{3.835139in}{3.235975in}}%
\pgfpathlineto{\pgfqpoint{3.903116in}{3.219335in}}%
\pgfpathlineto{\pgfqpoint{3.971092in}{3.200776in}}%
\pgfpathlineto{\pgfqpoint{4.039068in}{3.180296in}}%
\pgfpathlineto{\pgfqpoint{4.107044in}{3.157897in}}%
\pgfpathlineto{\pgfqpoint{4.175021in}{3.133578in}}%
\pgfpathlineto{\pgfqpoint{4.242997in}{3.107338in}}%
\pgfpathlineto{\pgfqpoint{4.310973in}{3.079179in}}%
\pgfpathlineto{\pgfqpoint{4.378949in}{3.049100in}}%
\pgfpathlineto{\pgfqpoint{4.446926in}{3.017100in}}%
\pgfpathlineto{\pgfqpoint{4.514902in}{2.983181in}}%
\pgfpathlineto{\pgfqpoint{4.582878in}{2.947342in}}%
\pgfpathlineto{\pgfqpoint{4.650854in}{2.909583in}}%
\pgfpathlineto{\pgfqpoint{4.741489in}{2.856251in}}%
\pgfpathlineto{\pgfqpoint{4.832124in}{2.799506in}}%
\pgfpathlineto{\pgfqpoint{4.922759in}{2.739347in}}%
\pgfpathlineto{\pgfqpoint{5.013394in}{2.675776in}}%
\pgfpathlineto{\pgfqpoint{5.104029in}{2.608791in}}%
\pgfpathlineto{\pgfqpoint{5.194664in}{2.538393in}}%
\pgfpathlineto{\pgfqpoint{5.285299in}{2.464581in}}%
\pgfpathlineto{\pgfqpoint{5.375934in}{2.387356in}}%
\pgfpathlineto{\pgfqpoint{5.466569in}{2.306718in}}%
\pgfpathlineto{\pgfqpoint{5.534545in}{2.244000in}}%
\pgfpathlineto{\pgfqpoint{5.534545in}{2.244000in}}%
\pgfusepath{stroke}%
\end{pgfscope}%
\begin{pgfscope}%
\pgfpathrectangle{\pgfqpoint{0.800000in}{0.528000in}}{\pgfqpoint{4.960000in}{3.696000in}}%
\pgfusepath{clip}%
\pgfsetbuttcap%
\pgfsetroundjoin%
\definecolor{currentfill}{rgb}{0.737255,0.741176,0.133333}%
\pgfsetfillcolor{currentfill}%
\pgfsetlinewidth{1.003750pt}%
\definecolor{currentstroke}{rgb}{0.737255,0.741176,0.133333}%
\pgfsetstrokecolor{currentstroke}%
\pgfsetdash{}{0pt}%
\pgfsys@defobject{currentmarker}{\pgfqpoint{-0.020833in}{-0.020833in}}{\pgfqpoint{0.020833in}{0.020833in}}{%
\pgfpathmoveto{\pgfqpoint{0.000000in}{-0.020833in}}%
\pgfpathcurveto{\pgfqpoint{0.005525in}{-0.020833in}}{\pgfqpoint{0.010825in}{-0.018638in}}{\pgfqpoint{0.014731in}{-0.014731in}}%
\pgfpathcurveto{\pgfqpoint{0.018638in}{-0.010825in}}{\pgfqpoint{0.020833in}{-0.005525in}}{\pgfqpoint{0.020833in}{0.000000in}}%
\pgfpathcurveto{\pgfqpoint{0.020833in}{0.005525in}}{\pgfqpoint{0.018638in}{0.010825in}}{\pgfqpoint{0.014731in}{0.014731in}}%
\pgfpathcurveto{\pgfqpoint{0.010825in}{0.018638in}}{\pgfqpoint{0.005525in}{0.020833in}}{\pgfqpoint{0.000000in}{0.020833in}}%
\pgfpathcurveto{\pgfqpoint{-0.005525in}{0.020833in}}{\pgfqpoint{-0.010825in}{0.018638in}}{\pgfqpoint{-0.014731in}{0.014731in}}%
\pgfpathcurveto{\pgfqpoint{-0.018638in}{0.010825in}}{\pgfqpoint{-0.020833in}{0.005525in}}{\pgfqpoint{-0.020833in}{0.000000in}}%
\pgfpathcurveto{\pgfqpoint{-0.020833in}{-0.005525in}}{\pgfqpoint{-0.018638in}{-0.010825in}}{\pgfqpoint{-0.014731in}{-0.014731in}}%
\pgfpathcurveto{\pgfqpoint{-0.010825in}{-0.018638in}}{\pgfqpoint{-0.005525in}{-0.020833in}}{\pgfqpoint{0.000000in}{-0.020833in}}%
\pgfpathclose%
\pgfusepath{stroke,fill}%
}%
\begin{pgfscope}%
\pgfsys@transformshift{5.424200in}{2.422329in}%
\pgfsys@useobject{currentmarker}{}%
\end{pgfscope}%
\begin{pgfscope}%
\pgfsys@transformshift{4.605189in}{2.825476in}%
\pgfsys@useobject{currentmarker}{}%
\end{pgfscope}%
\begin{pgfscope}%
\pgfsys@transformshift{3.280000in}{3.300000in}%
\pgfsys@useobject{currentmarker}{}%
\end{pgfscope}%
\begin{pgfscope}%
\pgfsys@transformshift{1.954811in}{2.825476in}%
\pgfsys@useobject{currentmarker}{}%
\end{pgfscope}%
\begin{pgfscope}%
\pgfsys@transformshift{1.135800in}{2.422329in}%
\pgfsys@useobject{currentmarker}{}%
\end{pgfscope}%
\end{pgfscope}%
\begin{pgfscope}%
\pgfpathrectangle{\pgfqpoint{0.800000in}{0.528000in}}{\pgfqpoint{4.960000in}{3.696000in}}%
\pgfusepath{clip}%
\pgfsetrectcap%
\pgfsetroundjoin%
\pgfsetlinewidth{1.505625pt}%
\definecolor{currentstroke}{rgb}{0.090196,0.745098,0.811765}%
\pgfsetstrokecolor{currentstroke}%
\pgfsetdash{}{0pt}%
\pgfpathmoveto{\pgfqpoint{1.025455in}{2.398537in}}%
\pgfpathlineto{\pgfqpoint{1.070772in}{2.406854in}}%
\pgfpathlineto{\pgfqpoint{1.116090in}{2.417218in}}%
\pgfpathlineto{\pgfqpoint{1.161407in}{2.429491in}}%
\pgfpathlineto{\pgfqpoint{1.206725in}{2.443542in}}%
\pgfpathlineto{\pgfqpoint{1.274701in}{2.467663in}}%
\pgfpathlineto{\pgfqpoint{1.342677in}{2.495059in}}%
\pgfpathlineto{\pgfqpoint{1.410653in}{2.525316in}}%
\pgfpathlineto{\pgfqpoint{1.478630in}{2.558032in}}%
\pgfpathlineto{\pgfqpoint{1.569265in}{2.604813in}}%
\pgfpathlineto{\pgfqpoint{1.659899in}{2.654407in}}%
\pgfpathlineto{\pgfqpoint{1.773193in}{2.719091in}}%
\pgfpathlineto{\pgfqpoint{2.181051in}{2.955181in}}%
\pgfpathlineto{\pgfqpoint{2.271686in}{3.004293in}}%
\pgfpathlineto{\pgfqpoint{2.362321in}{3.050961in}}%
\pgfpathlineto{\pgfqpoint{2.452956in}{3.094706in}}%
\pgfpathlineto{\pgfqpoint{2.520932in}{3.125332in}}%
\pgfpathlineto{\pgfqpoint{2.588908in}{3.153901in}}%
\pgfpathlineto{\pgfqpoint{2.656884in}{3.180260in}}%
\pgfpathlineto{\pgfqpoint{2.724861in}{3.204271in}}%
\pgfpathlineto{\pgfqpoint{2.792837in}{3.225810in}}%
\pgfpathlineto{\pgfqpoint{2.860813in}{3.244768in}}%
\pgfpathlineto{\pgfqpoint{2.928789in}{3.261048in}}%
\pgfpathlineto{\pgfqpoint{2.996766in}{3.274571in}}%
\pgfpathlineto{\pgfqpoint{3.064742in}{3.285268in}}%
\pgfpathlineto{\pgfqpoint{3.132718in}{3.293088in}}%
\pgfpathlineto{\pgfqpoint{3.200694in}{3.297993in}}%
\pgfpathlineto{\pgfqpoint{3.268671in}{3.299959in}}%
\pgfpathlineto{\pgfqpoint{3.336647in}{3.298976in}}%
\pgfpathlineto{\pgfqpoint{3.404623in}{3.295049in}}%
\pgfpathlineto{\pgfqpoint{3.472599in}{3.288197in}}%
\pgfpathlineto{\pgfqpoint{3.540576in}{3.278453in}}%
\pgfpathlineto{\pgfqpoint{3.608552in}{3.265866in}}%
\pgfpathlineto{\pgfqpoint{3.676528in}{3.250496in}}%
\pgfpathlineto{\pgfqpoint{3.744504in}{3.232421in}}%
\pgfpathlineto{\pgfqpoint{3.812481in}{3.211731in}}%
\pgfpathlineto{\pgfqpoint{3.880457in}{3.188531in}}%
\pgfpathlineto{\pgfqpoint{3.948433in}{3.162940in}}%
\pgfpathlineto{\pgfqpoint{4.016409in}{3.135091in}}%
\pgfpathlineto{\pgfqpoint{4.084386in}{3.105133in}}%
\pgfpathlineto{\pgfqpoint{4.152362in}{3.073227in}}%
\pgfpathlineto{\pgfqpoint{4.242997in}{3.027964in}}%
\pgfpathlineto{\pgfqpoint{4.333632in}{2.980011in}}%
\pgfpathlineto{\pgfqpoint{4.446926in}{2.917058in}}%
\pgfpathlineto{\pgfqpoint{4.582878in}{2.838541in}}%
\pgfpathlineto{\pgfqpoint{4.877442in}{2.667146in}}%
\pgfpathlineto{\pgfqpoint{4.968077in}{2.616981in}}%
\pgfpathlineto{\pgfqpoint{5.058712in}{2.569416in}}%
\pgfpathlineto{\pgfqpoint{5.126688in}{2.535967in}}%
\pgfpathlineto{\pgfqpoint{5.194664in}{2.504847in}}%
\pgfpathlineto{\pgfqpoint{5.262640in}{2.476452in}}%
\pgfpathlineto{\pgfqpoint{5.330617in}{2.451192in}}%
\pgfpathlineto{\pgfqpoint{5.375934in}{2.436303in}}%
\pgfpathlineto{\pgfqpoint{5.421252in}{2.423124in}}%
\pgfpathlineto{\pgfqpoint{5.466569in}{2.411789in}}%
\pgfpathlineto{\pgfqpoint{5.511887in}{2.402431in}}%
\pgfpathlineto{\pgfqpoint{5.534545in}{2.398537in}}%
\pgfpathlineto{\pgfqpoint{5.534545in}{2.398537in}}%
\pgfusepath{stroke}%
\end{pgfscope}%
\begin{pgfscope}%
\pgfpathrectangle{\pgfqpoint{0.800000in}{0.528000in}}{\pgfqpoint{4.960000in}{3.696000in}}%
\pgfusepath{clip}%
\pgfsetrectcap%
\pgfsetroundjoin%
\pgfsetlinewidth{1.505625pt}%
\definecolor{currentstroke}{rgb}{0.121569,0.466667,0.705882}%
\pgfsetstrokecolor{currentstroke}%
\pgfsetdash{}{0pt}%
\pgfpathmoveto{\pgfqpoint{1.025455in}{1.523077in}}%
\pgfpathlineto{\pgfqpoint{1.184066in}{1.533748in}}%
\pgfpathlineto{\pgfqpoint{1.320018in}{1.544892in}}%
\pgfpathlineto{\pgfqpoint{1.455971in}{1.558428in}}%
\pgfpathlineto{\pgfqpoint{1.569265in}{1.572045in}}%
\pgfpathlineto{\pgfqpoint{1.659899in}{1.584860in}}%
\pgfpathlineto{\pgfqpoint{1.750534in}{1.599776in}}%
\pgfpathlineto{\pgfqpoint{1.841169in}{1.617263in}}%
\pgfpathlineto{\pgfqpoint{1.909146in}{1.632419in}}%
\pgfpathlineto{\pgfqpoint{1.977122in}{1.649670in}}%
\pgfpathlineto{\pgfqpoint{2.045098in}{1.669400in}}%
\pgfpathlineto{\pgfqpoint{2.113074in}{1.692080in}}%
\pgfpathlineto{\pgfqpoint{2.158392in}{1.709119in}}%
\pgfpathlineto{\pgfqpoint{2.203709in}{1.727926in}}%
\pgfpathlineto{\pgfqpoint{2.249027in}{1.748735in}}%
\pgfpathlineto{\pgfqpoint{2.294344in}{1.771818in}}%
\pgfpathlineto{\pgfqpoint{2.339662in}{1.797485in}}%
\pgfpathlineto{\pgfqpoint{2.384979in}{1.826095in}}%
\pgfpathlineto{\pgfqpoint{2.430297in}{1.858060in}}%
\pgfpathlineto{\pgfqpoint{2.475614in}{1.893855in}}%
\pgfpathlineto{\pgfqpoint{2.520932in}{1.934017in}}%
\pgfpathlineto{\pgfqpoint{2.566249in}{1.979154in}}%
\pgfpathlineto{\pgfqpoint{2.611567in}{2.029943in}}%
\pgfpathlineto{\pgfqpoint{2.656884in}{2.087123in}}%
\pgfpathlineto{\pgfqpoint{2.702202in}{2.151469in}}%
\pgfpathlineto{\pgfqpoint{2.724861in}{2.186574in}}%
\pgfpathlineto{\pgfqpoint{2.747519in}{2.223757in}}%
\pgfpathlineto{\pgfqpoint{2.770178in}{2.263104in}}%
\pgfpathlineto{\pgfqpoint{2.815496in}{2.348561in}}%
\pgfpathlineto{\pgfqpoint{2.860813in}{2.443286in}}%
\pgfpathlineto{\pgfqpoint{2.906131in}{2.547122in}}%
\pgfpathlineto{\pgfqpoint{2.951448in}{2.659117in}}%
\pgfpathlineto{\pgfqpoint{3.019424in}{2.837353in}}%
\pgfpathlineto{\pgfqpoint{3.087401in}{3.014864in}}%
\pgfpathlineto{\pgfqpoint{3.110059in}{3.070154in}}%
\pgfpathlineto{\pgfqpoint{3.132718in}{3.121845in}}%
\pgfpathlineto{\pgfqpoint{3.155377in}{3.168855in}}%
\pgfpathlineto{\pgfqpoint{3.178036in}{3.210099in}}%
\pgfpathlineto{\pgfqpoint{3.200694in}{3.244550in}}%
\pgfpathlineto{\pgfqpoint{3.223353in}{3.271287in}}%
\pgfpathlineto{\pgfqpoint{3.246012in}{3.289560in}}%
\pgfpathlineto{\pgfqpoint{3.268671in}{3.298834in}}%
\pgfpathlineto{\pgfqpoint{3.291329in}{3.298834in}}%
\pgfpathlineto{\pgfqpoint{3.313988in}{3.289560in}}%
\pgfpathlineto{\pgfqpoint{3.336647in}{3.271287in}}%
\pgfpathlineto{\pgfqpoint{3.359306in}{3.244550in}}%
\pgfpathlineto{\pgfqpoint{3.381964in}{3.210099in}}%
\pgfpathlineto{\pgfqpoint{3.404623in}{3.168855in}}%
\pgfpathlineto{\pgfqpoint{3.427282in}{3.121845in}}%
\pgfpathlineto{\pgfqpoint{3.449941in}{3.070154in}}%
\pgfpathlineto{\pgfqpoint{3.495258in}{2.957011in}}%
\pgfpathlineto{\pgfqpoint{3.631211in}{2.602200in}}%
\pgfpathlineto{\pgfqpoint{3.676528in}{2.494102in}}%
\pgfpathlineto{\pgfqpoint{3.721846in}{2.394759in}}%
\pgfpathlineto{\pgfqpoint{3.767163in}{2.304686in}}%
\pgfpathlineto{\pgfqpoint{3.812481in}{2.223757in}}%
\pgfpathlineto{\pgfqpoint{3.857798in}{2.151469in}}%
\pgfpathlineto{\pgfqpoint{3.903116in}{2.087123in}}%
\pgfpathlineto{\pgfqpoint{3.948433in}{2.029943in}}%
\pgfpathlineto{\pgfqpoint{3.993751in}{1.979154in}}%
\pgfpathlineto{\pgfqpoint{4.039068in}{1.934017in}}%
\pgfpathlineto{\pgfqpoint{4.084386in}{1.893855in}}%
\pgfpathlineto{\pgfqpoint{4.129703in}{1.858060in}}%
\pgfpathlineto{\pgfqpoint{4.175021in}{1.826095in}}%
\pgfpathlineto{\pgfqpoint{4.220338in}{1.797485in}}%
\pgfpathlineto{\pgfqpoint{4.265656in}{1.771818in}}%
\pgfpathlineto{\pgfqpoint{4.310973in}{1.748735in}}%
\pgfpathlineto{\pgfqpoint{4.356291in}{1.727926in}}%
\pgfpathlineto{\pgfqpoint{4.401608in}{1.709119in}}%
\pgfpathlineto{\pgfqpoint{4.446926in}{1.692080in}}%
\pgfpathlineto{\pgfqpoint{4.514902in}{1.669400in}}%
\pgfpathlineto{\pgfqpoint{4.582878in}{1.649670in}}%
\pgfpathlineto{\pgfqpoint{4.650854in}{1.632419in}}%
\pgfpathlineto{\pgfqpoint{4.718831in}{1.617263in}}%
\pgfpathlineto{\pgfqpoint{4.809466in}{1.599776in}}%
\pgfpathlineto{\pgfqpoint{4.900101in}{1.584860in}}%
\pgfpathlineto{\pgfqpoint{4.990735in}{1.572045in}}%
\pgfpathlineto{\pgfqpoint{5.104029in}{1.558428in}}%
\pgfpathlineto{\pgfqpoint{5.217323in}{1.546965in}}%
\pgfpathlineto{\pgfqpoint{5.353275in}{1.535463in}}%
\pgfpathlineto{\pgfqpoint{5.511887in}{1.524471in}}%
\pgfpathlineto{\pgfqpoint{5.534545in}{1.523077in}}%
\pgfpathlineto{\pgfqpoint{5.534545in}{1.523077in}}%
\pgfusepath{stroke}%
\end{pgfscope}%
\begin{pgfscope}%
\pgfpathrectangle{\pgfqpoint{0.800000in}{0.528000in}}{\pgfqpoint{4.960000in}{3.696000in}}%
\pgfusepath{clip}%
\pgfsetbuttcap%
\pgfsetroundjoin%
\definecolor{currentfill}{rgb}{1.000000,0.498039,0.054902}%
\pgfsetfillcolor{currentfill}%
\pgfsetlinewidth{1.003750pt}%
\definecolor{currentstroke}{rgb}{1.000000,0.498039,0.054902}%
\pgfsetstrokecolor{currentstroke}%
\pgfsetdash{}{0pt}%
\pgfsys@defobject{currentmarker}{\pgfqpoint{-0.020833in}{-0.020833in}}{\pgfqpoint{0.020833in}{0.020833in}}{%
\pgfpathmoveto{\pgfqpoint{0.000000in}{-0.020833in}}%
\pgfpathcurveto{\pgfqpoint{0.005525in}{-0.020833in}}{\pgfqpoint{0.010825in}{-0.018638in}}{\pgfqpoint{0.014731in}{-0.014731in}}%
\pgfpathcurveto{\pgfqpoint{0.018638in}{-0.010825in}}{\pgfqpoint{0.020833in}{-0.005525in}}{\pgfqpoint{0.020833in}{0.000000in}}%
\pgfpathcurveto{\pgfqpoint{0.020833in}{0.005525in}}{\pgfqpoint{0.018638in}{0.010825in}}{\pgfqpoint{0.014731in}{0.014731in}}%
\pgfpathcurveto{\pgfqpoint{0.010825in}{0.018638in}}{\pgfqpoint{0.005525in}{0.020833in}}{\pgfqpoint{0.000000in}{0.020833in}}%
\pgfpathcurveto{\pgfqpoint{-0.005525in}{0.020833in}}{\pgfqpoint{-0.010825in}{0.018638in}}{\pgfqpoint{-0.014731in}{0.014731in}}%
\pgfpathcurveto{\pgfqpoint{-0.018638in}{0.010825in}}{\pgfqpoint{-0.020833in}{0.005525in}}{\pgfqpoint{-0.020833in}{0.000000in}}%
\pgfpathcurveto{\pgfqpoint{-0.020833in}{-0.005525in}}{\pgfqpoint{-0.018638in}{-0.010825in}}{\pgfqpoint{-0.014731in}{-0.014731in}}%
\pgfpathcurveto{\pgfqpoint{-0.010825in}{-0.018638in}}{\pgfqpoint{-0.005525in}{-0.020833in}}{\pgfqpoint{0.000000in}{-0.020833in}}%
\pgfpathclose%
\pgfusepath{stroke,fill}%
}%
\begin{pgfscope}%
\pgfsys@transformshift{1.025455in}{1.523077in}%
\pgfsys@useobject{currentmarker}{}%
\end{pgfscope}%
\begin{pgfscope}%
\pgfsys@transformshift{1.669610in}{1.586350in}%
\pgfsys@useobject{currentmarker}{}%
\end{pgfscope}%
\begin{pgfscope}%
\pgfsys@transformshift{2.313766in}{1.782482in}%
\pgfsys@useobject{currentmarker}{}%
\end{pgfscope}%
\begin{pgfscope}%
\pgfsys@transformshift{2.957922in}{2.675676in}%
\pgfsys@useobject{currentmarker}{}%
\end{pgfscope}%
\begin{pgfscope}%
\pgfsys@transformshift{3.602078in}{2.675676in}%
\pgfsys@useobject{currentmarker}{}%
\end{pgfscope}%
\begin{pgfscope}%
\pgfsys@transformshift{4.246234in}{1.782482in}%
\pgfsys@useobject{currentmarker}{}%
\end{pgfscope}%
\begin{pgfscope}%
\pgfsys@transformshift{4.890390in}{1.586350in}%
\pgfsys@useobject{currentmarker}{}%
\end{pgfscope}%
\end{pgfscope}%
\begin{pgfscope}%
\pgfpathrectangle{\pgfqpoint{0.800000in}{0.528000in}}{\pgfqpoint{4.960000in}{3.696000in}}%
\pgfusepath{clip}%
\pgfsetrectcap%
\pgfsetroundjoin%
\pgfsetlinewidth{1.505625pt}%
\definecolor{currentstroke}{rgb}{0.172549,0.627451,0.172549}%
\pgfsetstrokecolor{currentstroke}%
\pgfsetdash{}{0pt}%
\pgfpathmoveto{\pgfqpoint{1.025455in}{1.523077in}}%
\pgfpathlineto{\pgfqpoint{1.048113in}{1.591768in}}%
\pgfpathlineto{\pgfqpoint{1.070772in}{1.650833in}}%
\pgfpathlineto{\pgfqpoint{1.093431in}{1.700981in}}%
\pgfpathlineto{\pgfqpoint{1.116090in}{1.742892in}}%
\pgfpathlineto{\pgfqpoint{1.138748in}{1.777220in}}%
\pgfpathlineto{\pgfqpoint{1.161407in}{1.804587in}}%
\pgfpathlineto{\pgfqpoint{1.184066in}{1.825591in}}%
\pgfpathlineto{\pgfqpoint{1.206725in}{1.840798in}}%
\pgfpathlineto{\pgfqpoint{1.229383in}{1.850754in}}%
\pgfpathlineto{\pgfqpoint{1.252042in}{1.855975in}}%
\pgfpathlineto{\pgfqpoint{1.274701in}{1.856953in}}%
\pgfpathlineto{\pgfqpoint{1.297360in}{1.854156in}}%
\pgfpathlineto{\pgfqpoint{1.320018in}{1.848029in}}%
\pgfpathlineto{\pgfqpoint{1.342677in}{1.838994in}}%
\pgfpathlineto{\pgfqpoint{1.365336in}{1.827449in}}%
\pgfpathlineto{\pgfqpoint{1.387995in}{1.813771in}}%
\pgfpathlineto{\pgfqpoint{1.410653in}{1.798316in}}%
\pgfpathlineto{\pgfqpoint{1.455971in}{1.763396in}}%
\pgfpathlineto{\pgfqpoint{1.523947in}{1.705431in}}%
\pgfpathlineto{\pgfqpoint{1.591923in}{1.646873in}}%
\pgfpathlineto{\pgfqpoint{1.637241in}{1.610324in}}%
\pgfpathlineto{\pgfqpoint{1.682558in}{1.577371in}}%
\pgfpathlineto{\pgfqpoint{1.727876in}{1.549122in}}%
\pgfpathlineto{\pgfqpoint{1.750534in}{1.537050in}}%
\pgfpathlineto{\pgfqpoint{1.773193in}{1.526473in}}%
\pgfpathlineto{\pgfqpoint{1.795852in}{1.517472in}}%
\pgfpathlineto{\pgfqpoint{1.818511in}{1.510116in}}%
\pgfpathlineto{\pgfqpoint{1.841169in}{1.504462in}}%
\pgfpathlineto{\pgfqpoint{1.863828in}{1.500559in}}%
\pgfpathlineto{\pgfqpoint{1.886487in}{1.498442in}}%
\pgfpathlineto{\pgfqpoint{1.909146in}{1.498139in}}%
\pgfpathlineto{\pgfqpoint{1.931804in}{1.499668in}}%
\pgfpathlineto{\pgfqpoint{1.954463in}{1.503038in}}%
\pgfpathlineto{\pgfqpoint{1.977122in}{1.508249in}}%
\pgfpathlineto{\pgfqpoint{1.999781in}{1.515295in}}%
\pgfpathlineto{\pgfqpoint{2.022439in}{1.524159in}}%
\pgfpathlineto{\pgfqpoint{2.045098in}{1.534820in}}%
\pgfpathlineto{\pgfqpoint{2.067757in}{1.547249in}}%
\pgfpathlineto{\pgfqpoint{2.090416in}{1.561409in}}%
\pgfpathlineto{\pgfqpoint{2.113074in}{1.577260in}}%
\pgfpathlineto{\pgfqpoint{2.135733in}{1.594754in}}%
\pgfpathlineto{\pgfqpoint{2.158392in}{1.613839in}}%
\pgfpathlineto{\pgfqpoint{2.203709in}{1.656545in}}%
\pgfpathlineto{\pgfqpoint{2.249027in}{1.704865in}}%
\pgfpathlineto{\pgfqpoint{2.294344in}{1.758225in}}%
\pgfpathlineto{\pgfqpoint{2.339662in}{1.815996in}}%
\pgfpathlineto{\pgfqpoint{2.384979in}{1.877510in}}%
\pgfpathlineto{\pgfqpoint{2.430297in}{1.942066in}}%
\pgfpathlineto{\pgfqpoint{2.498273in}{2.043014in}}%
\pgfpathlineto{\pgfqpoint{2.724861in}{2.385149in}}%
\pgfpathlineto{\pgfqpoint{2.770178in}{2.449206in}}%
\pgfpathlineto{\pgfqpoint{2.815496in}{2.510228in}}%
\pgfpathlineto{\pgfqpoint{2.860813in}{2.567602in}}%
\pgfpathlineto{\pgfqpoint{2.906131in}{2.620759in}}%
\pgfpathlineto{\pgfqpoint{2.951448in}{2.669174in}}%
\pgfpathlineto{\pgfqpoint{2.996766in}{2.712375in}}%
\pgfpathlineto{\pgfqpoint{3.042083in}{2.749942in}}%
\pgfpathlineto{\pgfqpoint{3.064742in}{2.766496in}}%
\pgfpathlineto{\pgfqpoint{3.087401in}{2.781511in}}%
\pgfpathlineto{\pgfqpoint{3.110059in}{2.794948in}}%
\pgfpathlineto{\pgfqpoint{3.132718in}{2.806778in}}%
\pgfpathlineto{\pgfqpoint{3.155377in}{2.816970in}}%
\pgfpathlineto{\pgfqpoint{3.178036in}{2.825501in}}%
\pgfpathlineto{\pgfqpoint{3.200694in}{2.832351in}}%
\pgfpathlineto{\pgfqpoint{3.223353in}{2.837502in}}%
\pgfpathlineto{\pgfqpoint{3.246012in}{2.840943in}}%
\pgfpathlineto{\pgfqpoint{3.268671in}{2.842666in}}%
\pgfpathlineto{\pgfqpoint{3.291329in}{2.842666in}}%
\pgfpathlineto{\pgfqpoint{3.313988in}{2.840943in}}%
\pgfpathlineto{\pgfqpoint{3.336647in}{2.837502in}}%
\pgfpathlineto{\pgfqpoint{3.359306in}{2.832351in}}%
\pgfpathlineto{\pgfqpoint{3.381964in}{2.825501in}}%
\pgfpathlineto{\pgfqpoint{3.404623in}{2.816970in}}%
\pgfpathlineto{\pgfqpoint{3.427282in}{2.806778in}}%
\pgfpathlineto{\pgfqpoint{3.449941in}{2.794948in}}%
\pgfpathlineto{\pgfqpoint{3.472599in}{2.781511in}}%
\pgfpathlineto{\pgfqpoint{3.495258in}{2.766496in}}%
\pgfpathlineto{\pgfqpoint{3.517917in}{2.749942in}}%
\pgfpathlineto{\pgfqpoint{3.540576in}{2.731887in}}%
\pgfpathlineto{\pgfqpoint{3.585893in}{2.691454in}}%
\pgfpathlineto{\pgfqpoint{3.631211in}{2.645590in}}%
\pgfpathlineto{\pgfqpoint{3.676528in}{2.594741in}}%
\pgfpathlineto{\pgfqpoint{3.721846in}{2.539407in}}%
\pgfpathlineto{\pgfqpoint{3.767163in}{2.480136in}}%
\pgfpathlineto{\pgfqpoint{3.812481in}{2.417517in}}%
\pgfpathlineto{\pgfqpoint{3.880457in}{2.318705in}}%
\pgfpathlineto{\pgfqpoint{3.971092in}{2.181403in}}%
\pgfpathlineto{\pgfqpoint{4.084386in}{2.008939in}}%
\pgfpathlineto{\pgfqpoint{4.152362in}{1.909452in}}%
\pgfpathlineto{\pgfqpoint{4.197679in}{1.846328in}}%
\pgfpathlineto{\pgfqpoint{4.242997in}{1.786600in}}%
\pgfpathlineto{\pgfqpoint{4.288314in}{1.730953in}}%
\pgfpathlineto{\pgfqpoint{4.333632in}{1.680038in}}%
\pgfpathlineto{\pgfqpoint{4.378949in}{1.634456in}}%
\pgfpathlineto{\pgfqpoint{4.401608in}{1.613839in}}%
\pgfpathlineto{\pgfqpoint{4.424267in}{1.594754in}}%
\pgfpathlineto{\pgfqpoint{4.446926in}{1.577260in}}%
\pgfpathlineto{\pgfqpoint{4.469584in}{1.561409in}}%
\pgfpathlineto{\pgfqpoint{4.492243in}{1.547249in}}%
\pgfpathlineto{\pgfqpoint{4.514902in}{1.534820in}}%
\pgfpathlineto{\pgfqpoint{4.537561in}{1.524159in}}%
\pgfpathlineto{\pgfqpoint{4.560219in}{1.515295in}}%
\pgfpathlineto{\pgfqpoint{4.582878in}{1.508249in}}%
\pgfpathlineto{\pgfqpoint{4.605537in}{1.503038in}}%
\pgfpathlineto{\pgfqpoint{4.628196in}{1.499668in}}%
\pgfpathlineto{\pgfqpoint{4.650854in}{1.498139in}}%
\pgfpathlineto{\pgfqpoint{4.673513in}{1.498442in}}%
\pgfpathlineto{\pgfqpoint{4.696172in}{1.500559in}}%
\pgfpathlineto{\pgfqpoint{4.718831in}{1.504462in}}%
\pgfpathlineto{\pgfqpoint{4.741489in}{1.510116in}}%
\pgfpathlineto{\pgfqpoint{4.764148in}{1.517472in}}%
\pgfpathlineto{\pgfqpoint{4.786807in}{1.526473in}}%
\pgfpathlineto{\pgfqpoint{4.809466in}{1.537050in}}%
\pgfpathlineto{\pgfqpoint{4.832124in}{1.549122in}}%
\pgfpathlineto{\pgfqpoint{4.854783in}{1.562597in}}%
\pgfpathlineto{\pgfqpoint{4.900101in}{1.593323in}}%
\pgfpathlineto{\pgfqpoint{4.945418in}{1.628228in}}%
\pgfpathlineto{\pgfqpoint{4.990735in}{1.666085in}}%
\pgfpathlineto{\pgfqpoint{5.104029in}{1.763396in}}%
\pgfpathlineto{\pgfqpoint{5.149347in}{1.798316in}}%
\pgfpathlineto{\pgfqpoint{5.172005in}{1.813771in}}%
\pgfpathlineto{\pgfqpoint{5.194664in}{1.827449in}}%
\pgfpathlineto{\pgfqpoint{5.217323in}{1.838994in}}%
\pgfpathlineto{\pgfqpoint{5.239982in}{1.848029in}}%
\pgfpathlineto{\pgfqpoint{5.262640in}{1.854156in}}%
\pgfpathlineto{\pgfqpoint{5.285299in}{1.856953in}}%
\pgfpathlineto{\pgfqpoint{5.307958in}{1.855975in}}%
\pgfpathlineto{\pgfqpoint{5.330617in}{1.850754in}}%
\pgfpathlineto{\pgfqpoint{5.353275in}{1.840798in}}%
\pgfpathlineto{\pgfqpoint{5.375934in}{1.825591in}}%
\pgfpathlineto{\pgfqpoint{5.398593in}{1.804587in}}%
\pgfpathlineto{\pgfqpoint{5.421252in}{1.777220in}}%
\pgfpathlineto{\pgfqpoint{5.443910in}{1.742892in}}%
\pgfpathlineto{\pgfqpoint{5.466569in}{1.700981in}}%
\pgfpathlineto{\pgfqpoint{5.489228in}{1.650833in}}%
\pgfpathlineto{\pgfqpoint{5.511887in}{1.591768in}}%
\pgfpathlineto{\pgfqpoint{5.534545in}{1.523077in}}%
\pgfpathlineto{\pgfqpoint{5.534545in}{1.523077in}}%
\pgfusepath{stroke}%
\end{pgfscope}%
\begin{pgfscope}%
\pgfpathrectangle{\pgfqpoint{0.800000in}{0.528000in}}{\pgfqpoint{4.960000in}{3.696000in}}%
\pgfusepath{clip}%
\pgfsetbuttcap%
\pgfsetroundjoin%
\definecolor{currentfill}{rgb}{0.839216,0.152941,0.156863}%
\pgfsetfillcolor{currentfill}%
\pgfsetlinewidth{1.003750pt}%
\definecolor{currentstroke}{rgb}{0.839216,0.152941,0.156863}%
\pgfsetstrokecolor{currentstroke}%
\pgfsetdash{}{0pt}%
\pgfsys@defobject{currentmarker}{\pgfqpoint{-0.020833in}{-0.020833in}}{\pgfqpoint{0.020833in}{0.020833in}}{%
\pgfpathmoveto{\pgfqpoint{0.000000in}{-0.020833in}}%
\pgfpathcurveto{\pgfqpoint{0.005525in}{-0.020833in}}{\pgfqpoint{0.010825in}{-0.018638in}}{\pgfqpoint{0.014731in}{-0.014731in}}%
\pgfpathcurveto{\pgfqpoint{0.018638in}{-0.010825in}}{\pgfqpoint{0.020833in}{-0.005525in}}{\pgfqpoint{0.020833in}{0.000000in}}%
\pgfpathcurveto{\pgfqpoint{0.020833in}{0.005525in}}{\pgfqpoint{0.018638in}{0.010825in}}{\pgfqpoint{0.014731in}{0.014731in}}%
\pgfpathcurveto{\pgfqpoint{0.010825in}{0.018638in}}{\pgfqpoint{0.005525in}{0.020833in}}{\pgfqpoint{0.000000in}{0.020833in}}%
\pgfpathcurveto{\pgfqpoint{-0.005525in}{0.020833in}}{\pgfqpoint{-0.010825in}{0.018638in}}{\pgfqpoint{-0.014731in}{0.014731in}}%
\pgfpathcurveto{\pgfqpoint{-0.018638in}{0.010825in}}{\pgfqpoint{-0.020833in}{0.005525in}}{\pgfqpoint{-0.020833in}{0.000000in}}%
\pgfpathcurveto{\pgfqpoint{-0.020833in}{-0.005525in}}{\pgfqpoint{-0.018638in}{-0.010825in}}{\pgfqpoint{-0.014731in}{-0.014731in}}%
\pgfpathcurveto{\pgfqpoint{-0.010825in}{-0.018638in}}{\pgfqpoint{-0.005525in}{-0.020833in}}{\pgfqpoint{0.000000in}{-0.020833in}}%
\pgfpathclose%
\pgfusepath{stroke,fill}%
}%
\begin{pgfscope}%
\pgfsys@transformshift{1.025455in}{1.523077in}%
\pgfsys@useobject{currentmarker}{}%
\end{pgfscope}%
\begin{pgfscope}%
\pgfsys@transformshift{1.526465in}{1.566616in}%
\pgfsys@useobject{currentmarker}{}%
\end{pgfscope}%
\begin{pgfscope}%
\pgfsys@transformshift{2.027475in}{1.664023in}%
\pgfsys@useobject{currentmarker}{}%
\end{pgfscope}%
\begin{pgfscope}%
\pgfsys@transformshift{2.528485in}{1.941176in}%
\pgfsys@useobject{currentmarker}{}%
\end{pgfscope}%
\begin{pgfscope}%
\pgfsys@transformshift{3.029495in}{2.864151in}%
\pgfsys@useobject{currentmarker}{}%
\end{pgfscope}%
\begin{pgfscope}%
\pgfsys@transformshift{3.530505in}{2.864151in}%
\pgfsys@useobject{currentmarker}{}%
\end{pgfscope}%
\begin{pgfscope}%
\pgfsys@transformshift{4.031515in}{1.941176in}%
\pgfsys@useobject{currentmarker}{}%
\end{pgfscope}%
\begin{pgfscope}%
\pgfsys@transformshift{4.532525in}{1.664023in}%
\pgfsys@useobject{currentmarker}{}%
\end{pgfscope}%
\begin{pgfscope}%
\pgfsys@transformshift{5.033535in}{1.566616in}%
\pgfsys@useobject{currentmarker}{}%
\end{pgfscope}%
\end{pgfscope}%
\begin{pgfscope}%
\pgfpathrectangle{\pgfqpoint{0.800000in}{0.528000in}}{\pgfqpoint{4.960000in}{3.696000in}}%
\pgfusepath{clip}%
\pgfsetrectcap%
\pgfsetroundjoin%
\pgfsetlinewidth{1.505625pt}%
\definecolor{currentstroke}{rgb}{0.580392,0.403922,0.741176}%
\pgfsetstrokecolor{currentstroke}%
\pgfsetdash{}{0pt}%
\pgfpathmoveto{\pgfqpoint{1.025455in}{1.523077in}}%
\pgfpathlineto{\pgfqpoint{1.048113in}{1.358583in}}%
\pgfpathlineto{\pgfqpoint{1.070772in}{1.229681in}}%
\pgfpathlineto{\pgfqpoint{1.093431in}{1.132034in}}%
\pgfpathlineto{\pgfqpoint{1.116090in}{1.061656in}}%
\pgfpathlineto{\pgfqpoint{1.138748in}{1.014895in}}%
\pgfpathlineto{\pgfqpoint{1.161407in}{0.988414in}}%
\pgfpathlineto{\pgfqpoint{1.184066in}{0.979170in}}%
\pgfpathlineto{\pgfqpoint{1.206725in}{0.984403in}}%
\pgfpathlineto{\pgfqpoint{1.229383in}{1.001611in}}%
\pgfpathlineto{\pgfqpoint{1.252042in}{1.028541in}}%
\pgfpathlineto{\pgfqpoint{1.274701in}{1.063168in}}%
\pgfpathlineto{\pgfqpoint{1.297360in}{1.103686in}}%
\pgfpathlineto{\pgfqpoint{1.320018in}{1.148487in}}%
\pgfpathlineto{\pgfqpoint{1.365336in}{1.245442in}}%
\pgfpathlineto{\pgfqpoint{1.433312in}{1.392964in}}%
\pgfpathlineto{\pgfqpoint{1.478630in}{1.483392in}}%
\pgfpathlineto{\pgfqpoint{1.501288in}{1.524580in}}%
\pgfpathlineto{\pgfqpoint{1.523947in}{1.562599in}}%
\pgfpathlineto{\pgfqpoint{1.546606in}{1.597198in}}%
\pgfpathlineto{\pgfqpoint{1.569265in}{1.628209in}}%
\pgfpathlineto{\pgfqpoint{1.591923in}{1.655537in}}%
\pgfpathlineto{\pgfqpoint{1.614582in}{1.679150in}}%
\pgfpathlineto{\pgfqpoint{1.637241in}{1.699078in}}%
\pgfpathlineto{\pgfqpoint{1.659899in}{1.715400in}}%
\pgfpathlineto{\pgfqpoint{1.682558in}{1.728241in}}%
\pgfpathlineto{\pgfqpoint{1.705217in}{1.737765in}}%
\pgfpathlineto{\pgfqpoint{1.727876in}{1.744170in}}%
\pgfpathlineto{\pgfqpoint{1.750534in}{1.747682in}}%
\pgfpathlineto{\pgfqpoint{1.773193in}{1.748549in}}%
\pgfpathlineto{\pgfqpoint{1.795852in}{1.747041in}}%
\pgfpathlineto{\pgfqpoint{1.818511in}{1.743439in}}%
\pgfpathlineto{\pgfqpoint{1.841169in}{1.738036in}}%
\pgfpathlineto{\pgfqpoint{1.863828in}{1.731132in}}%
\pgfpathlineto{\pgfqpoint{1.909146in}{1.714035in}}%
\pgfpathlineto{\pgfqpoint{2.045098in}{1.657644in}}%
\pgfpathlineto{\pgfqpoint{2.067757in}{1.650373in}}%
\pgfpathlineto{\pgfqpoint{2.090416in}{1.644355in}}%
\pgfpathlineto{\pgfqpoint{2.113074in}{1.639797in}}%
\pgfpathlineto{\pgfqpoint{2.135733in}{1.636888in}}%
\pgfpathlineto{\pgfqpoint{2.158392in}{1.635798in}}%
\pgfpathlineto{\pgfqpoint{2.181051in}{1.636677in}}%
\pgfpathlineto{\pgfqpoint{2.203709in}{1.639658in}}%
\pgfpathlineto{\pgfqpoint{2.226368in}{1.644851in}}%
\pgfpathlineto{\pgfqpoint{2.249027in}{1.652349in}}%
\pgfpathlineto{\pgfqpoint{2.271686in}{1.662223in}}%
\pgfpathlineto{\pgfqpoint{2.294344in}{1.674525in}}%
\pgfpathlineto{\pgfqpoint{2.317003in}{1.689287in}}%
\pgfpathlineto{\pgfqpoint{2.339662in}{1.706522in}}%
\pgfpathlineto{\pgfqpoint{2.362321in}{1.726223in}}%
\pgfpathlineto{\pgfqpoint{2.384979in}{1.748366in}}%
\pgfpathlineto{\pgfqpoint{2.407638in}{1.772909in}}%
\pgfpathlineto{\pgfqpoint{2.430297in}{1.799793in}}%
\pgfpathlineto{\pgfqpoint{2.452956in}{1.828941in}}%
\pgfpathlineto{\pgfqpoint{2.475614in}{1.860263in}}%
\pgfpathlineto{\pgfqpoint{2.520932in}{1.928988in}}%
\pgfpathlineto{\pgfqpoint{2.566249in}{2.004959in}}%
\pgfpathlineto{\pgfqpoint{2.611567in}{2.086980in}}%
\pgfpathlineto{\pgfqpoint{2.656884in}{2.173696in}}%
\pgfpathlineto{\pgfqpoint{2.724861in}{2.309322in}}%
\pgfpathlineto{\pgfqpoint{2.838154in}{2.536931in}}%
\pgfpathlineto{\pgfqpoint{2.883472in}{2.623803in}}%
\pgfpathlineto{\pgfqpoint{2.928789in}{2.705926in}}%
\pgfpathlineto{\pgfqpoint{2.974107in}{2.781823in}}%
\pgfpathlineto{\pgfqpoint{2.996766in}{2.817006in}}%
\pgfpathlineto{\pgfqpoint{3.019424in}{2.850130in}}%
\pgfpathlineto{\pgfqpoint{3.042083in}{2.881045in}}%
\pgfpathlineto{\pgfqpoint{3.064742in}{2.909613in}}%
\pgfpathlineto{\pgfqpoint{3.087401in}{2.935705in}}%
\pgfpathlineto{\pgfqpoint{3.110059in}{2.959203in}}%
\pgfpathlineto{\pgfqpoint{3.132718in}{2.980001in}}%
\pgfpathlineto{\pgfqpoint{3.155377in}{2.998005in}}%
\pgfpathlineto{\pgfqpoint{3.178036in}{3.013134in}}%
\pgfpathlineto{\pgfqpoint{3.200694in}{3.025320in}}%
\pgfpathlineto{\pgfqpoint{3.223353in}{3.034508in}}%
\pgfpathlineto{\pgfqpoint{3.246012in}{3.040656in}}%
\pgfpathlineto{\pgfqpoint{3.268671in}{3.043737in}}%
\pgfpathlineto{\pgfqpoint{3.291329in}{3.043737in}}%
\pgfpathlineto{\pgfqpoint{3.313988in}{3.040656in}}%
\pgfpathlineto{\pgfqpoint{3.336647in}{3.034508in}}%
\pgfpathlineto{\pgfqpoint{3.359306in}{3.025320in}}%
\pgfpathlineto{\pgfqpoint{3.381964in}{3.013134in}}%
\pgfpathlineto{\pgfqpoint{3.404623in}{2.998005in}}%
\pgfpathlineto{\pgfqpoint{3.427282in}{2.980001in}}%
\pgfpathlineto{\pgfqpoint{3.449941in}{2.959203in}}%
\pgfpathlineto{\pgfqpoint{3.472599in}{2.935705in}}%
\pgfpathlineto{\pgfqpoint{3.495258in}{2.909613in}}%
\pgfpathlineto{\pgfqpoint{3.517917in}{2.881045in}}%
\pgfpathlineto{\pgfqpoint{3.540576in}{2.850130in}}%
\pgfpathlineto{\pgfqpoint{3.563234in}{2.817006in}}%
\pgfpathlineto{\pgfqpoint{3.608552in}{2.744741in}}%
\pgfpathlineto{\pgfqpoint{3.653869in}{2.665552in}}%
\pgfpathlineto{\pgfqpoint{3.699187in}{2.580865in}}%
\pgfpathlineto{\pgfqpoint{3.767163in}{2.446870in}}%
\pgfpathlineto{\pgfqpoint{3.903116in}{2.173696in}}%
\pgfpathlineto{\pgfqpoint{3.948433in}{2.086980in}}%
\pgfpathlineto{\pgfqpoint{3.993751in}{2.004959in}}%
\pgfpathlineto{\pgfqpoint{4.039068in}{1.928988in}}%
\pgfpathlineto{\pgfqpoint{4.084386in}{1.860263in}}%
\pgfpathlineto{\pgfqpoint{4.107044in}{1.828941in}}%
\pgfpathlineto{\pgfqpoint{4.129703in}{1.799793in}}%
\pgfpathlineto{\pgfqpoint{4.152362in}{1.772909in}}%
\pgfpathlineto{\pgfqpoint{4.175021in}{1.748366in}}%
\pgfpathlineto{\pgfqpoint{4.197679in}{1.726223in}}%
\pgfpathlineto{\pgfqpoint{4.220338in}{1.706522in}}%
\pgfpathlineto{\pgfqpoint{4.242997in}{1.689287in}}%
\pgfpathlineto{\pgfqpoint{4.265656in}{1.674525in}}%
\pgfpathlineto{\pgfqpoint{4.288314in}{1.662223in}}%
\pgfpathlineto{\pgfqpoint{4.310973in}{1.652349in}}%
\pgfpathlineto{\pgfqpoint{4.333632in}{1.644851in}}%
\pgfpathlineto{\pgfqpoint{4.356291in}{1.639658in}}%
\pgfpathlineto{\pgfqpoint{4.378949in}{1.636677in}}%
\pgfpathlineto{\pgfqpoint{4.401608in}{1.635798in}}%
\pgfpathlineto{\pgfqpoint{4.424267in}{1.636888in}}%
\pgfpathlineto{\pgfqpoint{4.446926in}{1.639797in}}%
\pgfpathlineto{\pgfqpoint{4.469584in}{1.644355in}}%
\pgfpathlineto{\pgfqpoint{4.492243in}{1.650373in}}%
\pgfpathlineto{\pgfqpoint{4.537561in}{1.665944in}}%
\pgfpathlineto{\pgfqpoint{4.582878in}{1.684663in}}%
\pgfpathlineto{\pgfqpoint{4.650854in}{1.714035in}}%
\pgfpathlineto{\pgfqpoint{4.696172in}{1.731132in}}%
\pgfpathlineto{\pgfqpoint{4.718831in}{1.738036in}}%
\pgfpathlineto{\pgfqpoint{4.741489in}{1.743439in}}%
\pgfpathlineto{\pgfqpoint{4.764148in}{1.747041in}}%
\pgfpathlineto{\pgfqpoint{4.786807in}{1.748549in}}%
\pgfpathlineto{\pgfqpoint{4.809466in}{1.747682in}}%
\pgfpathlineto{\pgfqpoint{4.832124in}{1.744170in}}%
\pgfpathlineto{\pgfqpoint{4.854783in}{1.737765in}}%
\pgfpathlineto{\pgfqpoint{4.877442in}{1.728241in}}%
\pgfpathlineto{\pgfqpoint{4.900101in}{1.715400in}}%
\pgfpathlineto{\pgfqpoint{4.922759in}{1.699078in}}%
\pgfpathlineto{\pgfqpoint{4.945418in}{1.679150in}}%
\pgfpathlineto{\pgfqpoint{4.968077in}{1.655537in}}%
\pgfpathlineto{\pgfqpoint{4.990735in}{1.628209in}}%
\pgfpathlineto{\pgfqpoint{5.013394in}{1.597198in}}%
\pgfpathlineto{\pgfqpoint{5.036053in}{1.562599in}}%
\pgfpathlineto{\pgfqpoint{5.058712in}{1.524580in}}%
\pgfpathlineto{\pgfqpoint{5.081370in}{1.483392in}}%
\pgfpathlineto{\pgfqpoint{5.126688in}{1.392964in}}%
\pgfpathlineto{\pgfqpoint{5.172005in}{1.295270in}}%
\pgfpathlineto{\pgfqpoint{5.217323in}{1.196154in}}%
\pgfpathlineto{\pgfqpoint{5.239982in}{1.148487in}}%
\pgfpathlineto{\pgfqpoint{5.262640in}{1.103686in}}%
\pgfpathlineto{\pgfqpoint{5.285299in}{1.063168in}}%
\pgfpathlineto{\pgfqpoint{5.307958in}{1.028541in}}%
\pgfpathlineto{\pgfqpoint{5.330617in}{1.001611in}}%
\pgfpathlineto{\pgfqpoint{5.353275in}{0.984403in}}%
\pgfpathlineto{\pgfqpoint{5.375934in}{0.979170in}}%
\pgfpathlineto{\pgfqpoint{5.398593in}{0.988414in}}%
\pgfpathlineto{\pgfqpoint{5.421252in}{1.014895in}}%
\pgfpathlineto{\pgfqpoint{5.443910in}{1.061656in}}%
\pgfpathlineto{\pgfqpoint{5.466569in}{1.132034in}}%
\pgfpathlineto{\pgfqpoint{5.489228in}{1.229681in}}%
\pgfpathlineto{\pgfqpoint{5.511887in}{1.358583in}}%
\pgfpathlineto{\pgfqpoint{5.534545in}{1.523077in}}%
\pgfpathlineto{\pgfqpoint{5.534545in}{1.523077in}}%
\pgfusepath{stroke}%
\end{pgfscope}%
\begin{pgfscope}%
\pgfpathrectangle{\pgfqpoint{0.800000in}{0.528000in}}{\pgfqpoint{4.960000in}{3.696000in}}%
\pgfusepath{clip}%
\pgfsetbuttcap%
\pgfsetroundjoin%
\definecolor{currentfill}{rgb}{0.549020,0.337255,0.294118}%
\pgfsetfillcolor{currentfill}%
\pgfsetlinewidth{1.003750pt}%
\definecolor{currentstroke}{rgb}{0.549020,0.337255,0.294118}%
\pgfsetstrokecolor{currentstroke}%
\pgfsetdash{}{0pt}%
\pgfsys@defobject{currentmarker}{\pgfqpoint{-0.020833in}{-0.020833in}}{\pgfqpoint{0.020833in}{0.020833in}}{%
\pgfpathmoveto{\pgfqpoint{0.000000in}{-0.020833in}}%
\pgfpathcurveto{\pgfqpoint{0.005525in}{-0.020833in}}{\pgfqpoint{0.010825in}{-0.018638in}}{\pgfqpoint{0.014731in}{-0.014731in}}%
\pgfpathcurveto{\pgfqpoint{0.018638in}{-0.010825in}}{\pgfqpoint{0.020833in}{-0.005525in}}{\pgfqpoint{0.020833in}{0.000000in}}%
\pgfpathcurveto{\pgfqpoint{0.020833in}{0.005525in}}{\pgfqpoint{0.018638in}{0.010825in}}{\pgfqpoint{0.014731in}{0.014731in}}%
\pgfpathcurveto{\pgfqpoint{0.010825in}{0.018638in}}{\pgfqpoint{0.005525in}{0.020833in}}{\pgfqpoint{0.000000in}{0.020833in}}%
\pgfpathcurveto{\pgfqpoint{-0.005525in}{0.020833in}}{\pgfqpoint{-0.010825in}{0.018638in}}{\pgfqpoint{-0.014731in}{0.014731in}}%
\pgfpathcurveto{\pgfqpoint{-0.018638in}{0.010825in}}{\pgfqpoint{-0.020833in}{0.005525in}}{\pgfqpoint{-0.020833in}{0.000000in}}%
\pgfpathcurveto{\pgfqpoint{-0.020833in}{-0.005525in}}{\pgfqpoint{-0.018638in}{-0.010825in}}{\pgfqpoint{-0.014731in}{-0.014731in}}%
\pgfpathcurveto{\pgfqpoint{-0.010825in}{-0.018638in}}{\pgfqpoint{-0.005525in}{-0.020833in}}{\pgfqpoint{0.000000in}{-0.020833in}}%
\pgfpathclose%
\pgfusepath{stroke,fill}%
}%
\begin{pgfscope}%
\pgfsys@transformshift{1.025455in}{1.523077in}%
\pgfsys@useobject{currentmarker}{}%
\end{pgfscope}%
\begin{pgfscope}%
\pgfsys@transformshift{1.290695in}{1.542306in}%
\pgfsys@useobject{currentmarker}{}%
\end{pgfscope}%
\begin{pgfscope}%
\pgfsys@transformshift{1.555936in}{1.570315in}%
\pgfsys@useobject{currentmarker}{}%
\end{pgfscope}%
\begin{pgfscope}%
\pgfsys@transformshift{1.821176in}{1.613156in}%
\pgfsys@useobject{currentmarker}{}%
\end{pgfscope}%
\begin{pgfscope}%
\pgfsys@transformshift{2.086417in}{1.682800in}%
\pgfsys@useobject{currentmarker}{}%
\end{pgfscope}%
\begin{pgfscope}%
\pgfsys@transformshift{2.351658in}{1.804756in}%
\pgfsys@useobject{currentmarker}{}%
\end{pgfscope}%
\begin{pgfscope}%
\pgfsys@transformshift{2.616898in}{2.036324in}%
\pgfsys@useobject{currentmarker}{}%
\end{pgfscope}%
\begin{pgfscope}%
\pgfsys@transformshift{2.882139in}{2.491051in}%
\pgfsys@useobject{currentmarker}{}%
\end{pgfscope}%
\begin{pgfscope}%
\pgfsys@transformshift{3.147380in}{3.152866in}%
\pgfsys@useobject{currentmarker}{}%
\end{pgfscope}%
\begin{pgfscope}%
\pgfsys@transformshift{3.412620in}{3.152866in}%
\pgfsys@useobject{currentmarker}{}%
\end{pgfscope}%
\begin{pgfscope}%
\pgfsys@transformshift{3.677861in}{2.491051in}%
\pgfsys@useobject{currentmarker}{}%
\end{pgfscope}%
\begin{pgfscope}%
\pgfsys@transformshift{3.943102in}{2.036324in}%
\pgfsys@useobject{currentmarker}{}%
\end{pgfscope}%
\begin{pgfscope}%
\pgfsys@transformshift{4.208342in}{1.804756in}%
\pgfsys@useobject{currentmarker}{}%
\end{pgfscope}%
\begin{pgfscope}%
\pgfsys@transformshift{4.473583in}{1.682800in}%
\pgfsys@useobject{currentmarker}{}%
\end{pgfscope}%
\begin{pgfscope}%
\pgfsys@transformshift{4.738824in}{1.613156in}%
\pgfsys@useobject{currentmarker}{}%
\end{pgfscope}%
\begin{pgfscope}%
\pgfsys@transformshift{5.004064in}{1.570315in}%
\pgfsys@useobject{currentmarker}{}%
\end{pgfscope}%
\begin{pgfscope}%
\pgfsys@transformshift{5.269305in}{1.542306in}%
\pgfsys@useobject{currentmarker}{}%
\end{pgfscope}%
\end{pgfscope}%
\begin{pgfscope}%
\pgfpathrectangle{\pgfqpoint{0.800000in}{0.528000in}}{\pgfqpoint{4.960000in}{3.696000in}}%
\pgfusepath{clip}%
\pgfsetrectcap%
\pgfsetroundjoin%
\pgfsetlinewidth{1.505625pt}%
\definecolor{currentstroke}{rgb}{0.890196,0.466667,0.760784}%
\pgfsetstrokecolor{currentstroke}%
\pgfsetdash{}{0pt}%
\pgfpathmoveto{\pgfqpoint{1.025455in}{1.523077in}}%
\pgfpathlineto{\pgfqpoint{1.030252in}{0.518000in}}%
\pgfpathmoveto{\pgfqpoint{1.253434in}{0.518000in}}%
\pgfpathlineto{\pgfqpoint{1.274701in}{1.165653in}}%
\pgfpathlineto{\pgfqpoint{1.297360in}{1.673772in}}%
\pgfpathlineto{\pgfqpoint{1.320018in}{2.017973in}}%
\pgfpathlineto{\pgfqpoint{1.342677in}{2.222511in}}%
\pgfpathlineto{\pgfqpoint{1.365336in}{2.314119in}}%
\pgfpathlineto{\pgfqpoint{1.387995in}{2.319381in}}%
\pgfpathlineto{\pgfqpoint{1.410653in}{2.262991in}}%
\pgfpathlineto{\pgfqpoint{1.433312in}{2.166702in}}%
\pgfpathlineto{\pgfqpoint{1.455971in}{2.048790in}}%
\pgfpathlineto{\pgfqpoint{1.501288in}{1.803113in}}%
\pgfpathlineto{\pgfqpoint{1.523947in}{1.694314in}}%
\pgfpathlineto{\pgfqpoint{1.546606in}{1.602502in}}%
\pgfpathlineto{\pgfqpoint{1.569265in}{1.530282in}}%
\pgfpathlineto{\pgfqpoint{1.591923in}{1.478324in}}%
\pgfpathlineto{\pgfqpoint{1.614582in}{1.445808in}}%
\pgfpathlineto{\pgfqpoint{1.637241in}{1.430852in}}%
\pgfpathlineto{\pgfqpoint{1.659899in}{1.430875in}}%
\pgfpathlineto{\pgfqpoint{1.682558in}{1.442925in}}%
\pgfpathlineto{\pgfqpoint{1.705217in}{1.463936in}}%
\pgfpathlineto{\pgfqpoint{1.727876in}{1.490936in}}%
\pgfpathlineto{\pgfqpoint{1.795852in}{1.582518in}}%
\pgfpathlineto{\pgfqpoint{1.818511in}{1.610127in}}%
\pgfpathlineto{\pgfqpoint{1.841169in}{1.634164in}}%
\pgfpathlineto{\pgfqpoint{1.863828in}{1.654008in}}%
\pgfpathlineto{\pgfqpoint{1.886487in}{1.669419in}}%
\pgfpathlineto{\pgfqpoint{1.909146in}{1.680487in}}%
\pgfpathlineto{\pgfqpoint{1.931804in}{1.687566in}}%
\pgfpathlineto{\pgfqpoint{1.954463in}{1.691213in}}%
\pgfpathlineto{\pgfqpoint{1.977122in}{1.692122in}}%
\pgfpathlineto{\pgfqpoint{1.999781in}{1.691069in}}%
\pgfpathlineto{\pgfqpoint{2.090416in}{1.682662in}}%
\pgfpathlineto{\pgfqpoint{2.113074in}{1.682821in}}%
\pgfpathlineto{\pgfqpoint{2.135733in}{1.684845in}}%
\pgfpathlineto{\pgfqpoint{2.158392in}{1.688998in}}%
\pgfpathlineto{\pgfqpoint{2.181051in}{1.695418in}}%
\pgfpathlineto{\pgfqpoint{2.203709in}{1.704125in}}%
\pgfpathlineto{\pgfqpoint{2.226368in}{1.715036in}}%
\pgfpathlineto{\pgfqpoint{2.249027in}{1.727980in}}%
\pgfpathlineto{\pgfqpoint{2.271686in}{1.742720in}}%
\pgfpathlineto{\pgfqpoint{2.317003in}{1.776429in}}%
\pgfpathlineto{\pgfqpoint{2.362321in}{1.813731in}}%
\pgfpathlineto{\pgfqpoint{2.566249in}{1.988029in}}%
\pgfpathlineto{\pgfqpoint{2.611567in}{2.030926in}}%
\pgfpathlineto{\pgfqpoint{2.634226in}{2.054518in}}%
\pgfpathlineto{\pgfqpoint{2.656884in}{2.080025in}}%
\pgfpathlineto{\pgfqpoint{2.679543in}{2.107800in}}%
\pgfpathlineto{\pgfqpoint{2.702202in}{2.138171in}}%
\pgfpathlineto{\pgfqpoint{2.724861in}{2.171423in}}%
\pgfpathlineto{\pgfqpoint{2.747519in}{2.207785in}}%
\pgfpathlineto{\pgfqpoint{2.770178in}{2.247419in}}%
\pgfpathlineto{\pgfqpoint{2.792837in}{2.290405in}}%
\pgfpathlineto{\pgfqpoint{2.815496in}{2.336740in}}%
\pgfpathlineto{\pgfqpoint{2.838154in}{2.386326in}}%
\pgfpathlineto{\pgfqpoint{2.883472in}{2.494380in}}%
\pgfpathlineto{\pgfqpoint{2.928789in}{2.611877in}}%
\pgfpathlineto{\pgfqpoint{3.042083in}{2.917316in}}%
\pgfpathlineto{\pgfqpoint{3.064742in}{2.974728in}}%
\pgfpathlineto{\pgfqpoint{3.087401in}{3.029113in}}%
\pgfpathlineto{\pgfqpoint{3.110059in}{3.079707in}}%
\pgfpathlineto{\pgfqpoint{3.132718in}{3.125782in}}%
\pgfpathlineto{\pgfqpoint{3.155377in}{3.166661in}}%
\pgfpathlineto{\pgfqpoint{3.178036in}{3.201730in}}%
\pgfpathlineto{\pgfqpoint{3.200694in}{3.230458in}}%
\pgfpathlineto{\pgfqpoint{3.223353in}{3.252404in}}%
\pgfpathlineto{\pgfqpoint{3.246012in}{3.267227in}}%
\pgfpathlineto{\pgfqpoint{3.268671in}{3.274698in}}%
\pgfpathlineto{\pgfqpoint{3.291329in}{3.274698in}}%
\pgfpathlineto{\pgfqpoint{3.313988in}{3.267227in}}%
\pgfpathlineto{\pgfqpoint{3.336647in}{3.252404in}}%
\pgfpathlineto{\pgfqpoint{3.359306in}{3.230458in}}%
\pgfpathlineto{\pgfqpoint{3.381964in}{3.201730in}}%
\pgfpathlineto{\pgfqpoint{3.404623in}{3.166661in}}%
\pgfpathlineto{\pgfqpoint{3.427282in}{3.125782in}}%
\pgfpathlineto{\pgfqpoint{3.449941in}{3.079707in}}%
\pgfpathlineto{\pgfqpoint{3.472599in}{3.029113in}}%
\pgfpathlineto{\pgfqpoint{3.495258in}{2.974728in}}%
\pgfpathlineto{\pgfqpoint{3.540576in}{2.857657in}}%
\pgfpathlineto{\pgfqpoint{3.653869in}{2.552174in}}%
\pgfpathlineto{\pgfqpoint{3.699187in}{2.438969in}}%
\pgfpathlineto{\pgfqpoint{3.721846in}{2.386326in}}%
\pgfpathlineto{\pgfqpoint{3.744504in}{2.336740in}}%
\pgfpathlineto{\pgfqpoint{3.767163in}{2.290405in}}%
\pgfpathlineto{\pgfqpoint{3.789822in}{2.247419in}}%
\pgfpathlineto{\pgfqpoint{3.812481in}{2.207785in}}%
\pgfpathlineto{\pgfqpoint{3.835139in}{2.171423in}}%
\pgfpathlineto{\pgfqpoint{3.857798in}{2.138171in}}%
\pgfpathlineto{\pgfqpoint{3.880457in}{2.107800in}}%
\pgfpathlineto{\pgfqpoint{3.903116in}{2.080025in}}%
\pgfpathlineto{\pgfqpoint{3.925774in}{2.054518in}}%
\pgfpathlineto{\pgfqpoint{3.971092in}{2.008883in}}%
\pgfpathlineto{\pgfqpoint{4.016409in}{1.968022in}}%
\pgfpathlineto{\pgfqpoint{4.197679in}{1.813731in}}%
\pgfpathlineto{\pgfqpoint{4.242997in}{1.776429in}}%
\pgfpathlineto{\pgfqpoint{4.288314in}{1.742720in}}%
\pgfpathlineto{\pgfqpoint{4.310973in}{1.727980in}}%
\pgfpathlineto{\pgfqpoint{4.333632in}{1.715036in}}%
\pgfpathlineto{\pgfqpoint{4.356291in}{1.704125in}}%
\pgfpathlineto{\pgfqpoint{4.378949in}{1.695418in}}%
\pgfpathlineto{\pgfqpoint{4.401608in}{1.688998in}}%
\pgfpathlineto{\pgfqpoint{4.424267in}{1.684845in}}%
\pgfpathlineto{\pgfqpoint{4.446926in}{1.682821in}}%
\pgfpathlineto{\pgfqpoint{4.469584in}{1.682662in}}%
\pgfpathlineto{\pgfqpoint{4.514902in}{1.686249in}}%
\pgfpathlineto{\pgfqpoint{4.560219in}{1.691069in}}%
\pgfpathlineto{\pgfqpoint{4.582878in}{1.692122in}}%
\pgfpathlineto{\pgfqpoint{4.605537in}{1.691213in}}%
\pgfpathlineto{\pgfqpoint{4.628196in}{1.687566in}}%
\pgfpathlineto{\pgfqpoint{4.650854in}{1.680487in}}%
\pgfpathlineto{\pgfqpoint{4.673513in}{1.669419in}}%
\pgfpathlineto{\pgfqpoint{4.696172in}{1.654008in}}%
\pgfpathlineto{\pgfqpoint{4.718831in}{1.634164in}}%
\pgfpathlineto{\pgfqpoint{4.741489in}{1.610127in}}%
\pgfpathlineto{\pgfqpoint{4.764148in}{1.582518in}}%
\pgfpathlineto{\pgfqpoint{4.832124in}{1.490936in}}%
\pgfpathlineto{\pgfqpoint{4.854783in}{1.463936in}}%
\pgfpathlineto{\pgfqpoint{4.877442in}{1.442925in}}%
\pgfpathlineto{\pgfqpoint{4.900101in}{1.430875in}}%
\pgfpathlineto{\pgfqpoint{4.922759in}{1.430852in}}%
\pgfpathlineto{\pgfqpoint{4.945418in}{1.445808in}}%
\pgfpathlineto{\pgfqpoint{4.968077in}{1.478324in}}%
\pgfpathlineto{\pgfqpoint{4.990735in}{1.530282in}}%
\pgfpathlineto{\pgfqpoint{5.013394in}{1.602502in}}%
\pgfpathlineto{\pgfqpoint{5.036053in}{1.694314in}}%
\pgfpathlineto{\pgfqpoint{5.058712in}{1.803113in}}%
\pgfpathlineto{\pgfqpoint{5.126688in}{2.166702in}}%
\pgfpathlineto{\pgfqpoint{5.149347in}{2.262991in}}%
\pgfpathlineto{\pgfqpoint{5.172005in}{2.319381in}}%
\pgfpathlineto{\pgfqpoint{5.194664in}{2.314119in}}%
\pgfpathlineto{\pgfqpoint{5.217323in}{2.222511in}}%
\pgfpathlineto{\pgfqpoint{5.239982in}{2.017973in}}%
\pgfpathlineto{\pgfqpoint{5.262640in}{1.673772in}}%
\pgfpathlineto{\pgfqpoint{5.285299in}{1.165653in}}%
\pgfpathlineto{\pgfqpoint{5.306566in}{0.518000in}}%
\pgfpathmoveto{\pgfqpoint{5.529748in}{0.518000in}}%
\pgfpathlineto{\pgfqpoint{5.534545in}{1.523077in}}%
\pgfpathlineto{\pgfqpoint{5.534545in}{1.523077in}}%
\pgfusepath{stroke}%
\end{pgfscope}%
\begin{pgfscope}%
\pgfpathrectangle{\pgfqpoint{0.800000in}{0.528000in}}{\pgfqpoint{4.960000in}{3.696000in}}%
\pgfusepath{clip}%
\pgfsetbuttcap%
\pgfsetroundjoin%
\definecolor{currentfill}{rgb}{0.498039,0.498039,0.498039}%
\pgfsetfillcolor{currentfill}%
\pgfsetlinewidth{1.003750pt}%
\definecolor{currentstroke}{rgb}{0.498039,0.498039,0.498039}%
\pgfsetstrokecolor{currentstroke}%
\pgfsetdash{}{0pt}%
\pgfsys@defobject{currentmarker}{\pgfqpoint{-0.020833in}{-0.020833in}}{\pgfqpoint{0.020833in}{0.020833in}}{%
\pgfpathmoveto{\pgfqpoint{0.000000in}{-0.020833in}}%
\pgfpathcurveto{\pgfqpoint{0.005525in}{-0.020833in}}{\pgfqpoint{0.010825in}{-0.018638in}}{\pgfqpoint{0.014731in}{-0.014731in}}%
\pgfpathcurveto{\pgfqpoint{0.018638in}{-0.010825in}}{\pgfqpoint{0.020833in}{-0.005525in}}{\pgfqpoint{0.020833in}{0.000000in}}%
\pgfpathcurveto{\pgfqpoint{0.020833in}{0.005525in}}{\pgfqpoint{0.018638in}{0.010825in}}{\pgfqpoint{0.014731in}{0.014731in}}%
\pgfpathcurveto{\pgfqpoint{0.010825in}{0.018638in}}{\pgfqpoint{0.005525in}{0.020833in}}{\pgfqpoint{0.000000in}{0.020833in}}%
\pgfpathcurveto{\pgfqpoint{-0.005525in}{0.020833in}}{\pgfqpoint{-0.010825in}{0.018638in}}{\pgfqpoint{-0.014731in}{0.014731in}}%
\pgfpathcurveto{\pgfqpoint{-0.018638in}{0.010825in}}{\pgfqpoint{-0.020833in}{0.005525in}}{\pgfqpoint{-0.020833in}{0.000000in}}%
\pgfpathcurveto{\pgfqpoint{-0.020833in}{-0.005525in}}{\pgfqpoint{-0.018638in}{-0.010825in}}{\pgfqpoint{-0.014731in}{-0.014731in}}%
\pgfpathcurveto{\pgfqpoint{-0.010825in}{-0.018638in}}{\pgfqpoint{-0.005525in}{-0.020833in}}{\pgfqpoint{0.000000in}{-0.020833in}}%
\pgfpathclose%
\pgfusepath{stroke,fill}%
}%
\begin{pgfscope}%
\pgfsys@transformshift{1.025455in}{1.523077in}%
\pgfsys@useobject{currentmarker}{}%
\end{pgfscope}%
\begin{pgfscope}%
\pgfsys@transformshift{1.162094in}{1.532133in}%
\pgfsys@useobject{currentmarker}{}%
\end{pgfscope}%
\begin{pgfscope}%
\pgfsys@transformshift{1.298733in}{1.543004in}%
\pgfsys@useobject{currentmarker}{}%
\end{pgfscope}%
\begin{pgfscope}%
\pgfsys@transformshift{1.435372in}{1.556198in}%
\pgfsys@useobject{currentmarker}{}%
\end{pgfscope}%
\begin{pgfscope}%
\pgfsys@transformshift{1.572011in}{1.572406in}%
\pgfsys@useobject{currentmarker}{}%
\end{pgfscope}%
\begin{pgfscope}%
\pgfsys@transformshift{1.708650in}{1.592595in}%
\pgfsys@useobject{currentmarker}{}%
\end{pgfscope}%
\begin{pgfscope}%
\pgfsys@transformshift{1.845289in}{1.618128in}%
\pgfsys@useobject{currentmarker}{}%
\end{pgfscope}%
\begin{pgfscope}%
\pgfsys@transformshift{1.981928in}{1.650979in}%
\pgfsys@useobject{currentmarker}{}%
\end{pgfscope}%
\begin{pgfscope}%
\pgfsys@transformshift{2.118567in}{1.694058in}%
\pgfsys@useobject{currentmarker}{}%
\end{pgfscope}%
\begin{pgfscope}%
\pgfsys@transformshift{2.255207in}{1.751743in}%
\pgfsys@useobject{currentmarker}{}%
\end{pgfscope}%
\begin{pgfscope}%
\pgfsys@transformshift{2.391846in}{1.830711in}%
\pgfsys@useobject{currentmarker}{}%
\end{pgfscope}%
\begin{pgfscope}%
\pgfsys@transformshift{2.528485in}{1.941176in}%
\pgfsys@useobject{currentmarker}{}%
\end{pgfscope}%
\begin{pgfscope}%
\pgfsys@transformshift{2.665124in}{2.098266in}%
\pgfsys@useobject{currentmarker}{}%
\end{pgfscope}%
\begin{pgfscope}%
\pgfsys@transformshift{2.801763in}{2.321694in}%
\pgfsys@useobject{currentmarker}{}%
\end{pgfscope}%
\begin{pgfscope}%
\pgfsys@transformshift{2.938402in}{2.626138in}%
\pgfsys@useobject{currentmarker}{}%
\end{pgfscope}%
\begin{pgfscope}%
\pgfsys@transformshift{3.075041in}{2.983562in}%
\pgfsys@useobject{currentmarker}{}%
\end{pgfscope}%
\begin{pgfscope}%
\pgfsys@transformshift{3.211680in}{3.258528in}%
\pgfsys@useobject{currentmarker}{}%
\end{pgfscope}%
\begin{pgfscope}%
\pgfsys@transformshift{3.348320in}{3.258528in}%
\pgfsys@useobject{currentmarker}{}%
\end{pgfscope}%
\begin{pgfscope}%
\pgfsys@transformshift{3.484959in}{2.983562in}%
\pgfsys@useobject{currentmarker}{}%
\end{pgfscope}%
\begin{pgfscope}%
\pgfsys@transformshift{3.621598in}{2.626138in}%
\pgfsys@useobject{currentmarker}{}%
\end{pgfscope}%
\begin{pgfscope}%
\pgfsys@transformshift{3.758237in}{2.321694in}%
\pgfsys@useobject{currentmarker}{}%
\end{pgfscope}%
\begin{pgfscope}%
\pgfsys@transformshift{3.894876in}{2.098266in}%
\pgfsys@useobject{currentmarker}{}%
\end{pgfscope}%
\begin{pgfscope}%
\pgfsys@transformshift{4.031515in}{1.941176in}%
\pgfsys@useobject{currentmarker}{}%
\end{pgfscope}%
\begin{pgfscope}%
\pgfsys@transformshift{4.168154in}{1.830711in}%
\pgfsys@useobject{currentmarker}{}%
\end{pgfscope}%
\begin{pgfscope}%
\pgfsys@transformshift{4.304793in}{1.751743in}%
\pgfsys@useobject{currentmarker}{}%
\end{pgfscope}%
\begin{pgfscope}%
\pgfsys@transformshift{4.441433in}{1.694058in}%
\pgfsys@useobject{currentmarker}{}%
\end{pgfscope}%
\begin{pgfscope}%
\pgfsys@transformshift{4.578072in}{1.650979in}%
\pgfsys@useobject{currentmarker}{}%
\end{pgfscope}%
\begin{pgfscope}%
\pgfsys@transformshift{4.714711in}{1.618128in}%
\pgfsys@useobject{currentmarker}{}%
\end{pgfscope}%
\begin{pgfscope}%
\pgfsys@transformshift{4.851350in}{1.592595in}%
\pgfsys@useobject{currentmarker}{}%
\end{pgfscope}%
\begin{pgfscope}%
\pgfsys@transformshift{4.987989in}{1.572406in}%
\pgfsys@useobject{currentmarker}{}%
\end{pgfscope}%
\begin{pgfscope}%
\pgfsys@transformshift{5.124628in}{1.556198in}%
\pgfsys@useobject{currentmarker}{}%
\end{pgfscope}%
\begin{pgfscope}%
\pgfsys@transformshift{5.261267in}{1.543004in}%
\pgfsys@useobject{currentmarker}{}%
\end{pgfscope}%
\begin{pgfscope}%
\pgfsys@transformshift{5.397906in}{1.532133in}%
\pgfsys@useobject{currentmarker}{}%
\end{pgfscope}%
\end{pgfscope}%
\begin{pgfscope}%
\pgfpathrectangle{\pgfqpoint{0.800000in}{0.528000in}}{\pgfqpoint{4.960000in}{3.696000in}}%
\pgfusepath{clip}%
\pgfsetrectcap%
\pgfsetroundjoin%
\pgfsetlinewidth{1.505625pt}%
\definecolor{currentstroke}{rgb}{0.737255,0.741176,0.133333}%
\pgfsetstrokecolor{currentstroke}%
\pgfsetdash{}{0pt}%
\pgfpathmoveto{\pgfqpoint{1.025455in}{1.523077in}}%
\pgfpathlineto{\pgfqpoint{1.025463in}{0.518000in}}%
\pgfpathmoveto{\pgfqpoint{1.162330in}{0.518000in}}%
\pgfpathlineto{\pgfqpoint{1.163099in}{4.234000in}}%
\pgfpathmoveto{\pgfqpoint{1.295055in}{4.234000in}}%
\pgfpathlineto{\pgfqpoint{1.297360in}{2.327267in}}%
\pgfpathlineto{\pgfqpoint{1.302086in}{0.518000in}}%
\pgfpathmoveto{\pgfqpoint{1.422871in}{0.518000in}}%
\pgfpathlineto{\pgfqpoint{1.433312in}{1.424798in}}%
\pgfpathlineto{\pgfqpoint{1.455971in}{2.465484in}}%
\pgfpathlineto{\pgfqpoint{1.478630in}{2.768251in}}%
\pgfpathlineto{\pgfqpoint{1.501288in}{2.613062in}}%
\pgfpathlineto{\pgfqpoint{1.546606in}{1.885509in}}%
\pgfpathlineto{\pgfqpoint{1.569265in}{1.599102in}}%
\pgfpathlineto{\pgfqpoint{1.591923in}{1.432411in}}%
\pgfpathlineto{\pgfqpoint{1.614582in}{1.375137in}}%
\pgfpathlineto{\pgfqpoint{1.637241in}{1.395104in}}%
\pgfpathlineto{\pgfqpoint{1.659899in}{1.455535in}}%
\pgfpathlineto{\pgfqpoint{1.682558in}{1.525640in}}%
\pgfpathlineto{\pgfqpoint{1.705217in}{1.585205in}}%
\pgfpathlineto{\pgfqpoint{1.727876in}{1.624836in}}%
\pgfpathlineto{\pgfqpoint{1.750534in}{1.643673in}}%
\pgfpathlineto{\pgfqpoint{1.773193in}{1.646145in}}%
\pgfpathlineto{\pgfqpoint{1.795852in}{1.638874in}}%
\pgfpathlineto{\pgfqpoint{1.818511in}{1.628311in}}%
\pgfpathlineto{\pgfqpoint{1.841169in}{1.619337in}}%
\pgfpathlineto{\pgfqpoint{1.863828in}{1.614741in}}%
\pgfpathlineto{\pgfqpoint{1.886487in}{1.615353in}}%
\pgfpathlineto{\pgfqpoint{1.909146in}{1.620554in}}%
\pgfpathlineto{\pgfqpoint{1.931804in}{1.628926in}}%
\pgfpathlineto{\pgfqpoint{1.999781in}{1.658227in}}%
\pgfpathlineto{\pgfqpoint{2.045098in}{1.673441in}}%
\pgfpathlineto{\pgfqpoint{2.135733in}{1.699547in}}%
\pgfpathlineto{\pgfqpoint{2.181051in}{1.716642in}}%
\pgfpathlineto{\pgfqpoint{2.226368in}{1.737277in}}%
\pgfpathlineto{\pgfqpoint{2.271686in}{1.760345in}}%
\pgfpathlineto{\pgfqpoint{2.317003in}{1.785100in}}%
\pgfpathlineto{\pgfqpoint{2.362321in}{1.811807in}}%
\pgfpathlineto{\pgfqpoint{2.407638in}{1.841438in}}%
\pgfpathlineto{\pgfqpoint{2.452956in}{1.874998in}}%
\pgfpathlineto{\pgfqpoint{2.498273in}{1.913104in}}%
\pgfpathlineto{\pgfqpoint{2.543591in}{1.956038in}}%
\pgfpathlineto{\pgfqpoint{2.588908in}{2.004094in}}%
\pgfpathlineto{\pgfqpoint{2.634226in}{2.057864in}}%
\pgfpathlineto{\pgfqpoint{2.679543in}{2.118270in}}%
\pgfpathlineto{\pgfqpoint{2.702202in}{2.151288in}}%
\pgfpathlineto{\pgfqpoint{2.724861in}{2.186346in}}%
\pgfpathlineto{\pgfqpoint{2.747519in}{2.223543in}}%
\pgfpathlineto{\pgfqpoint{2.770178in}{2.262958in}}%
\pgfpathlineto{\pgfqpoint{2.815496in}{2.348626in}}%
\pgfpathlineto{\pgfqpoint{2.860813in}{2.443486in}}%
\pgfpathlineto{\pgfqpoint{2.906131in}{2.547259in}}%
\pgfpathlineto{\pgfqpoint{2.951448in}{2.659057in}}%
\pgfpathlineto{\pgfqpoint{3.019424in}{2.837157in}}%
\pgfpathlineto{\pgfqpoint{3.087401in}{3.014922in}}%
\pgfpathlineto{\pgfqpoint{3.110059in}{3.070306in}}%
\pgfpathlineto{\pgfqpoint{3.132718in}{3.122052in}}%
\pgfpathlineto{\pgfqpoint{3.155377in}{3.169061in}}%
\pgfpathlineto{\pgfqpoint{3.178036in}{3.210250in}}%
\pgfpathlineto{\pgfqpoint{3.200694in}{3.244604in}}%
\pgfpathlineto{\pgfqpoint{3.223353in}{3.271229in}}%
\pgfpathlineto{\pgfqpoint{3.246012in}{3.289404in}}%
\pgfpathlineto{\pgfqpoint{3.268671in}{3.298622in}}%
\pgfpathlineto{\pgfqpoint{3.291329in}{3.298622in}}%
\pgfpathlineto{\pgfqpoint{3.313988in}{3.289404in}}%
\pgfpathlineto{\pgfqpoint{3.336647in}{3.271229in}}%
\pgfpathlineto{\pgfqpoint{3.359306in}{3.244604in}}%
\pgfpathlineto{\pgfqpoint{3.381964in}{3.210250in}}%
\pgfpathlineto{\pgfqpoint{3.404623in}{3.169061in}}%
\pgfpathlineto{\pgfqpoint{3.427282in}{3.122052in}}%
\pgfpathlineto{\pgfqpoint{3.449941in}{3.070306in}}%
\pgfpathlineto{\pgfqpoint{3.495258in}{2.956962in}}%
\pgfpathlineto{\pgfqpoint{3.631211in}{2.602245in}}%
\pgfpathlineto{\pgfqpoint{3.676528in}{2.494296in}}%
\pgfpathlineto{\pgfqpoint{3.721846in}{2.394914in}}%
\pgfpathlineto{\pgfqpoint{3.767163in}{2.304642in}}%
\pgfpathlineto{\pgfqpoint{3.812481in}{2.223543in}}%
\pgfpathlineto{\pgfqpoint{3.835139in}{2.186346in}}%
\pgfpathlineto{\pgfqpoint{3.880457in}{2.118270in}}%
\pgfpathlineto{\pgfqpoint{3.925774in}{2.057864in}}%
\pgfpathlineto{\pgfqpoint{3.971092in}{2.004094in}}%
\pgfpathlineto{\pgfqpoint{4.016409in}{1.956038in}}%
\pgfpathlineto{\pgfqpoint{4.061727in}{1.913104in}}%
\pgfpathlineto{\pgfqpoint{4.107044in}{1.874998in}}%
\pgfpathlineto{\pgfqpoint{4.152362in}{1.841438in}}%
\pgfpathlineto{\pgfqpoint{4.197679in}{1.811807in}}%
\pgfpathlineto{\pgfqpoint{4.242997in}{1.785100in}}%
\pgfpathlineto{\pgfqpoint{4.310973in}{1.748575in}}%
\pgfpathlineto{\pgfqpoint{4.356291in}{1.726579in}}%
\pgfpathlineto{\pgfqpoint{4.401608in}{1.707608in}}%
\pgfpathlineto{\pgfqpoint{4.446926in}{1.692400in}}%
\pgfpathlineto{\pgfqpoint{4.537561in}{1.666378in}}%
\pgfpathlineto{\pgfqpoint{4.582878in}{1.648919in}}%
\pgfpathlineto{\pgfqpoint{4.628196in}{1.628926in}}%
\pgfpathlineto{\pgfqpoint{4.650854in}{1.620554in}}%
\pgfpathlineto{\pgfqpoint{4.673513in}{1.615353in}}%
\pgfpathlineto{\pgfqpoint{4.696172in}{1.614741in}}%
\pgfpathlineto{\pgfqpoint{4.718831in}{1.619337in}}%
\pgfpathlineto{\pgfqpoint{4.741489in}{1.628311in}}%
\pgfpathlineto{\pgfqpoint{4.764148in}{1.638874in}}%
\pgfpathlineto{\pgfqpoint{4.786807in}{1.646145in}}%
\pgfpathlineto{\pgfqpoint{4.809466in}{1.643673in}}%
\pgfpathlineto{\pgfqpoint{4.832124in}{1.624836in}}%
\pgfpathlineto{\pgfqpoint{4.854783in}{1.585205in}}%
\pgfpathlineto{\pgfqpoint{4.877442in}{1.525640in}}%
\pgfpathlineto{\pgfqpoint{4.900101in}{1.455535in}}%
\pgfpathlineto{\pgfqpoint{4.922759in}{1.395104in}}%
\pgfpathlineto{\pgfqpoint{4.945418in}{1.375137in}}%
\pgfpathlineto{\pgfqpoint{4.968077in}{1.432411in}}%
\pgfpathlineto{\pgfqpoint{4.990735in}{1.599102in}}%
\pgfpathlineto{\pgfqpoint{5.013394in}{1.885509in}}%
\pgfpathlineto{\pgfqpoint{5.058712in}{2.613062in}}%
\pgfpathlineto{\pgfqpoint{5.081370in}{2.768251in}}%
\pgfpathlineto{\pgfqpoint{5.104029in}{2.465484in}}%
\pgfpathlineto{\pgfqpoint{5.126688in}{1.424798in}}%
\pgfpathlineto{\pgfqpoint{5.137129in}{0.518000in}}%
\pgfpathmoveto{\pgfqpoint{5.257914in}{0.518000in}}%
\pgfpathlineto{\pgfqpoint{5.262640in}{2.327267in}}%
\pgfpathlineto{\pgfqpoint{5.264945in}{4.234000in}}%
\pgfpathmoveto{\pgfqpoint{5.396901in}{4.234000in}}%
\pgfpathlineto{\pgfqpoint{5.397670in}{0.518000in}}%
\pgfpathmoveto{\pgfqpoint{5.534537in}{0.518000in}}%
\pgfpathlineto{\pgfqpoint{5.534545in}{1.523079in}}%
\pgfpathlineto{\pgfqpoint{5.534545in}{1.523079in}}%
\pgfusepath{stroke}%
\end{pgfscope}%
\begin{pgfscope}%
\pgfpathrectangle{\pgfqpoint{0.800000in}{0.528000in}}{\pgfqpoint{4.960000in}{3.696000in}}%
\pgfusepath{clip}%
\pgfsetrectcap%
\pgfsetroundjoin%
\pgfsetlinewidth{1.505625pt}%
\definecolor{currentstroke}{rgb}{0.090196,0.745098,0.811765}%
\pgfsetstrokecolor{currentstroke}%
\pgfsetdash{}{0pt}%
\pgfpathmoveto{\pgfqpoint{1.025455in}{1.523077in}}%
\pgfpathlineto{\pgfqpoint{1.184066in}{1.533748in}}%
\pgfpathlineto{\pgfqpoint{1.320018in}{1.544892in}}%
\pgfpathlineto{\pgfqpoint{1.455971in}{1.558428in}}%
\pgfpathlineto{\pgfqpoint{1.569265in}{1.572045in}}%
\pgfpathlineto{\pgfqpoint{1.659899in}{1.584860in}}%
\pgfpathlineto{\pgfqpoint{1.750534in}{1.599776in}}%
\pgfpathlineto{\pgfqpoint{1.841169in}{1.617263in}}%
\pgfpathlineto{\pgfqpoint{1.909146in}{1.632419in}}%
\pgfpathlineto{\pgfqpoint{1.977122in}{1.649670in}}%
\pgfpathlineto{\pgfqpoint{2.045098in}{1.669400in}}%
\pgfpathlineto{\pgfqpoint{2.113074in}{1.692080in}}%
\pgfpathlineto{\pgfqpoint{2.158392in}{1.709119in}}%
\pgfpathlineto{\pgfqpoint{2.203709in}{1.727926in}}%
\pgfpathlineto{\pgfqpoint{2.249027in}{1.748735in}}%
\pgfpathlineto{\pgfqpoint{2.294344in}{1.771818in}}%
\pgfpathlineto{\pgfqpoint{2.339662in}{1.797485in}}%
\pgfpathlineto{\pgfqpoint{2.384979in}{1.826095in}}%
\pgfpathlineto{\pgfqpoint{2.430297in}{1.858060in}}%
\pgfpathlineto{\pgfqpoint{2.475614in}{1.893855in}}%
\pgfpathlineto{\pgfqpoint{2.520932in}{1.934017in}}%
\pgfpathlineto{\pgfqpoint{2.566249in}{1.979154in}}%
\pgfpathlineto{\pgfqpoint{2.611567in}{2.029943in}}%
\pgfpathlineto{\pgfqpoint{2.656884in}{2.087123in}}%
\pgfpathlineto{\pgfqpoint{2.702202in}{2.151469in}}%
\pgfpathlineto{\pgfqpoint{2.724861in}{2.186574in}}%
\pgfpathlineto{\pgfqpoint{2.747519in}{2.223757in}}%
\pgfpathlineto{\pgfqpoint{2.770178in}{2.263104in}}%
\pgfpathlineto{\pgfqpoint{2.815496in}{2.348561in}}%
\pgfpathlineto{\pgfqpoint{2.860813in}{2.443286in}}%
\pgfpathlineto{\pgfqpoint{2.906131in}{2.547122in}}%
\pgfpathlineto{\pgfqpoint{2.951448in}{2.659117in}}%
\pgfpathlineto{\pgfqpoint{3.019424in}{2.837353in}}%
\pgfpathlineto{\pgfqpoint{3.087401in}{3.014864in}}%
\pgfpathlineto{\pgfqpoint{3.110059in}{3.070154in}}%
\pgfpathlineto{\pgfqpoint{3.132718in}{3.121845in}}%
\pgfpathlineto{\pgfqpoint{3.155377in}{3.168855in}}%
\pgfpathlineto{\pgfqpoint{3.178036in}{3.210099in}}%
\pgfpathlineto{\pgfqpoint{3.200694in}{3.244550in}}%
\pgfpathlineto{\pgfqpoint{3.223353in}{3.271287in}}%
\pgfpathlineto{\pgfqpoint{3.246012in}{3.289560in}}%
\pgfpathlineto{\pgfqpoint{3.268671in}{3.298834in}}%
\pgfpathlineto{\pgfqpoint{3.291329in}{3.298834in}}%
\pgfpathlineto{\pgfqpoint{3.313988in}{3.289560in}}%
\pgfpathlineto{\pgfqpoint{3.336647in}{3.271287in}}%
\pgfpathlineto{\pgfqpoint{3.359306in}{3.244550in}}%
\pgfpathlineto{\pgfqpoint{3.381964in}{3.210099in}}%
\pgfpathlineto{\pgfqpoint{3.404623in}{3.168855in}}%
\pgfpathlineto{\pgfqpoint{3.427282in}{3.121845in}}%
\pgfpathlineto{\pgfqpoint{3.449941in}{3.070154in}}%
\pgfpathlineto{\pgfqpoint{3.495258in}{2.957011in}}%
\pgfpathlineto{\pgfqpoint{3.631211in}{2.602200in}}%
\pgfpathlineto{\pgfqpoint{3.676528in}{2.494102in}}%
\pgfpathlineto{\pgfqpoint{3.721846in}{2.394759in}}%
\pgfpathlineto{\pgfqpoint{3.767163in}{2.304686in}}%
\pgfpathlineto{\pgfqpoint{3.812481in}{2.223757in}}%
\pgfpathlineto{\pgfqpoint{3.857798in}{2.151469in}}%
\pgfpathlineto{\pgfqpoint{3.903116in}{2.087123in}}%
\pgfpathlineto{\pgfqpoint{3.948433in}{2.029943in}}%
\pgfpathlineto{\pgfqpoint{3.993751in}{1.979154in}}%
\pgfpathlineto{\pgfqpoint{4.039068in}{1.934017in}}%
\pgfpathlineto{\pgfqpoint{4.084386in}{1.893855in}}%
\pgfpathlineto{\pgfqpoint{4.129703in}{1.858060in}}%
\pgfpathlineto{\pgfqpoint{4.175021in}{1.826095in}}%
\pgfpathlineto{\pgfqpoint{4.220338in}{1.797485in}}%
\pgfpathlineto{\pgfqpoint{4.265656in}{1.771818in}}%
\pgfpathlineto{\pgfqpoint{4.310973in}{1.748735in}}%
\pgfpathlineto{\pgfqpoint{4.356291in}{1.727926in}}%
\pgfpathlineto{\pgfqpoint{4.401608in}{1.709119in}}%
\pgfpathlineto{\pgfqpoint{4.446926in}{1.692080in}}%
\pgfpathlineto{\pgfqpoint{4.514902in}{1.669400in}}%
\pgfpathlineto{\pgfqpoint{4.582878in}{1.649670in}}%
\pgfpathlineto{\pgfqpoint{4.650854in}{1.632419in}}%
\pgfpathlineto{\pgfqpoint{4.718831in}{1.617263in}}%
\pgfpathlineto{\pgfqpoint{4.809466in}{1.599776in}}%
\pgfpathlineto{\pgfqpoint{4.900101in}{1.584860in}}%
\pgfpathlineto{\pgfqpoint{4.990735in}{1.572045in}}%
\pgfpathlineto{\pgfqpoint{5.104029in}{1.558428in}}%
\pgfpathlineto{\pgfqpoint{5.217323in}{1.546965in}}%
\pgfpathlineto{\pgfqpoint{5.353275in}{1.535463in}}%
\pgfpathlineto{\pgfqpoint{5.511887in}{1.524471in}}%
\pgfpathlineto{\pgfqpoint{5.534545in}{1.523077in}}%
\pgfpathlineto{\pgfqpoint{5.534545in}{1.523077in}}%
\pgfusepath{stroke}%
\end{pgfscope}%
\begin{pgfscope}%
\pgfpathrectangle{\pgfqpoint{0.800000in}{0.528000in}}{\pgfqpoint{4.960000in}{3.696000in}}%
\pgfusepath{clip}%
\pgfsetbuttcap%
\pgfsetroundjoin%
\definecolor{currentfill}{rgb}{0.121569,0.466667,0.705882}%
\pgfsetfillcolor{currentfill}%
\pgfsetlinewidth{1.003750pt}%
\definecolor{currentstroke}{rgb}{0.121569,0.466667,0.705882}%
\pgfsetstrokecolor{currentstroke}%
\pgfsetdash{}{0pt}%
\pgfsys@defobject{currentmarker}{\pgfqpoint{-0.020833in}{-0.020833in}}{\pgfqpoint{0.020833in}{0.020833in}}{%
\pgfpathmoveto{\pgfqpoint{0.000000in}{-0.020833in}}%
\pgfpathcurveto{\pgfqpoint{0.005525in}{-0.020833in}}{\pgfqpoint{0.010825in}{-0.018638in}}{\pgfqpoint{0.014731in}{-0.014731in}}%
\pgfpathcurveto{\pgfqpoint{0.018638in}{-0.010825in}}{\pgfqpoint{0.020833in}{-0.005525in}}{\pgfqpoint{0.020833in}{0.000000in}}%
\pgfpathcurveto{\pgfqpoint{0.020833in}{0.005525in}}{\pgfqpoint{0.018638in}{0.010825in}}{\pgfqpoint{0.014731in}{0.014731in}}%
\pgfpathcurveto{\pgfqpoint{0.010825in}{0.018638in}}{\pgfqpoint{0.005525in}{0.020833in}}{\pgfqpoint{0.000000in}{0.020833in}}%
\pgfpathcurveto{\pgfqpoint{-0.005525in}{0.020833in}}{\pgfqpoint{-0.010825in}{0.018638in}}{\pgfqpoint{-0.014731in}{0.014731in}}%
\pgfpathcurveto{\pgfqpoint{-0.018638in}{0.010825in}}{\pgfqpoint{-0.020833in}{0.005525in}}{\pgfqpoint{-0.020833in}{0.000000in}}%
\pgfpathcurveto{\pgfqpoint{-0.020833in}{-0.005525in}}{\pgfqpoint{-0.018638in}{-0.010825in}}{\pgfqpoint{-0.014731in}{-0.014731in}}%
\pgfpathcurveto{\pgfqpoint{-0.010825in}{-0.018638in}}{\pgfqpoint{-0.005525in}{-0.020833in}}{\pgfqpoint{0.000000in}{-0.020833in}}%
\pgfpathclose%
\pgfusepath{stroke,fill}%
}%
\begin{pgfscope}%
\pgfsys@transformshift{5.478019in}{1.526630in}%
\pgfsys@useobject{currentmarker}{}%
\end{pgfscope}%
\begin{pgfscope}%
\pgfsys@transformshift{5.042675in}{1.565503in}%
\pgfsys@useobject{currentmarker}{}%
\end{pgfscope}%
\begin{pgfscope}%
\pgfsys@transformshift{4.258211in}{1.775848in}%
\pgfsys@useobject{currentmarker}{}%
\end{pgfscope}%
\begin{pgfscope}%
\pgfsys@transformshift{3.280000in}{3.300000in}%
\pgfsys@useobject{currentmarker}{}%
\end{pgfscope}%
\begin{pgfscope}%
\pgfsys@transformshift{2.301789in}{1.775848in}%
\pgfsys@useobject{currentmarker}{}%
\end{pgfscope}%
\begin{pgfscope}%
\pgfsys@transformshift{1.517325in}{1.565503in}%
\pgfsys@useobject{currentmarker}{}%
\end{pgfscope}%
\begin{pgfscope}%
\pgfsys@transformshift{1.081981in}{1.526630in}%
\pgfsys@useobject{currentmarker}{}%
\end{pgfscope}%
\end{pgfscope}%
\begin{pgfscope}%
\pgfpathrectangle{\pgfqpoint{0.800000in}{0.528000in}}{\pgfqpoint{4.960000in}{3.696000in}}%
\pgfusepath{clip}%
\pgfsetrectcap%
\pgfsetroundjoin%
\pgfsetlinewidth{1.505625pt}%
\definecolor{currentstroke}{rgb}{1.000000,0.498039,0.054902}%
\pgfsetstrokecolor{currentstroke}%
\pgfsetdash{}{0pt}%
\pgfpathmoveto{\pgfqpoint{1.025455in}{1.334510in}}%
\pgfpathlineto{\pgfqpoint{1.048113in}{1.420558in}}%
\pgfpathlineto{\pgfqpoint{1.070772in}{1.494362in}}%
\pgfpathlineto{\pgfqpoint{1.093431in}{1.556846in}}%
\pgfpathlineto{\pgfqpoint{1.116090in}{1.608890in}}%
\pgfpathlineto{\pgfqpoint{1.138748in}{1.651342in}}%
\pgfpathlineto{\pgfqpoint{1.161407in}{1.685009in}}%
\pgfpathlineto{\pgfqpoint{1.184066in}{1.710664in}}%
\pgfpathlineto{\pgfqpoint{1.206725in}{1.729043in}}%
\pgfpathlineto{\pgfqpoint{1.229383in}{1.740848in}}%
\pgfpathlineto{\pgfqpoint{1.252042in}{1.746750in}}%
\pgfpathlineto{\pgfqpoint{1.274701in}{1.747384in}}%
\pgfpathlineto{\pgfqpoint{1.297360in}{1.743355in}}%
\pgfpathlineto{\pgfqpoint{1.320018in}{1.735238in}}%
\pgfpathlineto{\pgfqpoint{1.342677in}{1.723574in}}%
\pgfpathlineto{\pgfqpoint{1.365336in}{1.708880in}}%
\pgfpathlineto{\pgfqpoint{1.387995in}{1.691640in}}%
\pgfpathlineto{\pgfqpoint{1.410653in}{1.672312in}}%
\pgfpathlineto{\pgfqpoint{1.455971in}{1.629092in}}%
\pgfpathlineto{\pgfqpoint{1.523947in}{1.558543in}}%
\pgfpathlineto{\pgfqpoint{1.591923in}{1.488944in}}%
\pgfpathlineto{\pgfqpoint{1.637241in}{1.446662in}}%
\pgfpathlineto{\pgfqpoint{1.659899in}{1.427424in}}%
\pgfpathlineto{\pgfqpoint{1.682558in}{1.409704in}}%
\pgfpathlineto{\pgfqpoint{1.705217in}{1.393668in}}%
\pgfpathlineto{\pgfqpoint{1.727876in}{1.379463in}}%
\pgfpathlineto{\pgfqpoint{1.750534in}{1.367219in}}%
\pgfpathlineto{\pgfqpoint{1.773193in}{1.357052in}}%
\pgfpathlineto{\pgfqpoint{1.795852in}{1.349058in}}%
\pgfpathlineto{\pgfqpoint{1.818511in}{1.343321in}}%
\pgfpathlineto{\pgfqpoint{1.841169in}{1.339911in}}%
\pgfpathlineto{\pgfqpoint{1.863828in}{1.338881in}}%
\pgfpathlineto{\pgfqpoint{1.886487in}{1.340273in}}%
\pgfpathlineto{\pgfqpoint{1.909146in}{1.344117in}}%
\pgfpathlineto{\pgfqpoint{1.931804in}{1.350429in}}%
\pgfpathlineto{\pgfqpoint{1.954463in}{1.359214in}}%
\pgfpathlineto{\pgfqpoint{1.977122in}{1.370467in}}%
\pgfpathlineto{\pgfqpoint{1.999781in}{1.384171in}}%
\pgfpathlineto{\pgfqpoint{2.022439in}{1.400301in}}%
\pgfpathlineto{\pgfqpoint{2.045098in}{1.418821in}}%
\pgfpathlineto{\pgfqpoint{2.067757in}{1.439687in}}%
\pgfpathlineto{\pgfqpoint{2.090416in}{1.462845in}}%
\pgfpathlineto{\pgfqpoint{2.113074in}{1.488237in}}%
\pgfpathlineto{\pgfqpoint{2.135733in}{1.515793in}}%
\pgfpathlineto{\pgfqpoint{2.158392in}{1.545440in}}%
\pgfpathlineto{\pgfqpoint{2.181051in}{1.577096in}}%
\pgfpathlineto{\pgfqpoint{2.226368in}{1.646083in}}%
\pgfpathlineto{\pgfqpoint{2.271686in}{1.721997in}}%
\pgfpathlineto{\pgfqpoint{2.317003in}{1.804006in}}%
\pgfpathlineto{\pgfqpoint{2.362321in}{1.891219in}}%
\pgfpathlineto{\pgfqpoint{2.407638in}{1.982700in}}%
\pgfpathlineto{\pgfqpoint{2.475614in}{2.125791in}}%
\pgfpathlineto{\pgfqpoint{2.702202in}{2.613028in}}%
\pgfpathlineto{\pgfqpoint{2.747519in}{2.704999in}}%
\pgfpathlineto{\pgfqpoint{2.792837in}{2.792976in}}%
\pgfpathlineto{\pgfqpoint{2.838154in}{2.876112in}}%
\pgfpathlineto{\pgfqpoint{2.883472in}{2.953611in}}%
\pgfpathlineto{\pgfqpoint{2.928789in}{3.024739in}}%
\pgfpathlineto{\pgfqpoint{2.974107in}{3.088826in}}%
\pgfpathlineto{\pgfqpoint{2.996766in}{3.118039in}}%
\pgfpathlineto{\pgfqpoint{3.019424in}{3.145272in}}%
\pgfpathlineto{\pgfqpoint{3.042083in}{3.170463in}}%
\pgfpathlineto{\pgfqpoint{3.064742in}{3.193553in}}%
\pgfpathlineto{\pgfqpoint{3.087401in}{3.214488in}}%
\pgfpathlineto{\pgfqpoint{3.110059in}{3.233220in}}%
\pgfpathlineto{\pgfqpoint{3.132718in}{3.249706in}}%
\pgfpathlineto{\pgfqpoint{3.155377in}{3.263908in}}%
\pgfpathlineto{\pgfqpoint{3.178036in}{3.275793in}}%
\pgfpathlineto{\pgfqpoint{3.200694in}{3.285334in}}%
\pgfpathlineto{\pgfqpoint{3.223353in}{3.292509in}}%
\pgfpathlineto{\pgfqpoint{3.246012in}{3.297301in}}%
\pgfpathlineto{\pgfqpoint{3.268671in}{3.299700in}}%
\pgfpathlineto{\pgfqpoint{3.291329in}{3.299700in}}%
\pgfpathlineto{\pgfqpoint{3.313988in}{3.297301in}}%
\pgfpathlineto{\pgfqpoint{3.336647in}{3.292509in}}%
\pgfpathlineto{\pgfqpoint{3.359306in}{3.285334in}}%
\pgfpathlineto{\pgfqpoint{3.381964in}{3.275793in}}%
\pgfpathlineto{\pgfqpoint{3.404623in}{3.263908in}}%
\pgfpathlineto{\pgfqpoint{3.427282in}{3.249706in}}%
\pgfpathlineto{\pgfqpoint{3.449941in}{3.233220in}}%
\pgfpathlineto{\pgfqpoint{3.472599in}{3.214488in}}%
\pgfpathlineto{\pgfqpoint{3.495258in}{3.193553in}}%
\pgfpathlineto{\pgfqpoint{3.517917in}{3.170463in}}%
\pgfpathlineto{\pgfqpoint{3.540576in}{3.145272in}}%
\pgfpathlineto{\pgfqpoint{3.563234in}{3.118039in}}%
\pgfpathlineto{\pgfqpoint{3.585893in}{3.088826in}}%
\pgfpathlineto{\pgfqpoint{3.631211in}{3.024739in}}%
\pgfpathlineto{\pgfqpoint{3.676528in}{2.953611in}}%
\pgfpathlineto{\pgfqpoint{3.721846in}{2.876112in}}%
\pgfpathlineto{\pgfqpoint{3.767163in}{2.792976in}}%
\pgfpathlineto{\pgfqpoint{3.812481in}{2.704999in}}%
\pgfpathlineto{\pgfqpoint{3.857798in}{2.613028in}}%
\pgfpathlineto{\pgfqpoint{3.925774in}{2.469551in}}%
\pgfpathlineto{\pgfqpoint{4.152362in}{1.982700in}}%
\pgfpathlineto{\pgfqpoint{4.197679in}{1.891219in}}%
\pgfpathlineto{\pgfqpoint{4.242997in}{1.804006in}}%
\pgfpathlineto{\pgfqpoint{4.288314in}{1.721997in}}%
\pgfpathlineto{\pgfqpoint{4.333632in}{1.646083in}}%
\pgfpathlineto{\pgfqpoint{4.378949in}{1.577096in}}%
\pgfpathlineto{\pgfqpoint{4.401608in}{1.545440in}}%
\pgfpathlineto{\pgfqpoint{4.424267in}{1.515793in}}%
\pgfpathlineto{\pgfqpoint{4.446926in}{1.488237in}}%
\pgfpathlineto{\pgfqpoint{4.469584in}{1.462845in}}%
\pgfpathlineto{\pgfqpoint{4.492243in}{1.439687in}}%
\pgfpathlineto{\pgfqpoint{4.514902in}{1.418821in}}%
\pgfpathlineto{\pgfqpoint{4.537561in}{1.400301in}}%
\pgfpathlineto{\pgfqpoint{4.560219in}{1.384171in}}%
\pgfpathlineto{\pgfqpoint{4.582878in}{1.370467in}}%
\pgfpathlineto{\pgfqpoint{4.605537in}{1.359214in}}%
\pgfpathlineto{\pgfqpoint{4.628196in}{1.350429in}}%
\pgfpathlineto{\pgfqpoint{4.650854in}{1.344117in}}%
\pgfpathlineto{\pgfqpoint{4.673513in}{1.340273in}}%
\pgfpathlineto{\pgfqpoint{4.696172in}{1.338881in}}%
\pgfpathlineto{\pgfqpoint{4.718831in}{1.339911in}}%
\pgfpathlineto{\pgfqpoint{4.741489in}{1.343321in}}%
\pgfpathlineto{\pgfqpoint{4.764148in}{1.349058in}}%
\pgfpathlineto{\pgfqpoint{4.786807in}{1.357052in}}%
\pgfpathlineto{\pgfqpoint{4.809466in}{1.367219in}}%
\pgfpathlineto{\pgfqpoint{4.832124in}{1.379463in}}%
\pgfpathlineto{\pgfqpoint{4.854783in}{1.393668in}}%
\pgfpathlineto{\pgfqpoint{4.877442in}{1.409704in}}%
\pgfpathlineto{\pgfqpoint{4.900101in}{1.427424in}}%
\pgfpathlineto{\pgfqpoint{4.945418in}{1.467237in}}%
\pgfpathlineto{\pgfqpoint{4.990735in}{1.511562in}}%
\pgfpathlineto{\pgfqpoint{5.126688in}{1.651328in}}%
\pgfpathlineto{\pgfqpoint{5.149347in}{1.672312in}}%
\pgfpathlineto{\pgfqpoint{5.172005in}{1.691640in}}%
\pgfpathlineto{\pgfqpoint{5.194664in}{1.708880in}}%
\pgfpathlineto{\pgfqpoint{5.217323in}{1.723574in}}%
\pgfpathlineto{\pgfqpoint{5.239982in}{1.735238in}}%
\pgfpathlineto{\pgfqpoint{5.262640in}{1.743355in}}%
\pgfpathlineto{\pgfqpoint{5.285299in}{1.747384in}}%
\pgfpathlineto{\pgfqpoint{5.307958in}{1.746750in}}%
\pgfpathlineto{\pgfqpoint{5.330617in}{1.740848in}}%
\pgfpathlineto{\pgfqpoint{5.353275in}{1.729043in}}%
\pgfpathlineto{\pgfqpoint{5.375934in}{1.710664in}}%
\pgfpathlineto{\pgfqpoint{5.398593in}{1.685009in}}%
\pgfpathlineto{\pgfqpoint{5.421252in}{1.651342in}}%
\pgfpathlineto{\pgfqpoint{5.443910in}{1.608890in}}%
\pgfpathlineto{\pgfqpoint{5.466569in}{1.556846in}}%
\pgfpathlineto{\pgfqpoint{5.489228in}{1.494362in}}%
\pgfpathlineto{\pgfqpoint{5.511887in}{1.420558in}}%
\pgfpathlineto{\pgfqpoint{5.534545in}{1.334510in}}%
\pgfpathlineto{\pgfqpoint{5.534545in}{1.334510in}}%
\pgfusepath{stroke}%
\end{pgfscope}%
\begin{pgfscope}%
\pgfpathrectangle{\pgfqpoint{0.800000in}{0.528000in}}{\pgfqpoint{4.960000in}{3.696000in}}%
\pgfusepath{clip}%
\pgfsetbuttcap%
\pgfsetroundjoin%
\definecolor{currentfill}{rgb}{0.172549,0.627451,0.172549}%
\pgfsetfillcolor{currentfill}%
\pgfsetlinewidth{1.003750pt}%
\definecolor{currentstroke}{rgb}{0.172549,0.627451,0.172549}%
\pgfsetstrokecolor{currentstroke}%
\pgfsetdash{}{0pt}%
\pgfsys@defobject{currentmarker}{\pgfqpoint{-0.020833in}{-0.020833in}}{\pgfqpoint{0.020833in}{0.020833in}}{%
\pgfpathmoveto{\pgfqpoint{0.000000in}{-0.020833in}}%
\pgfpathcurveto{\pgfqpoint{0.005525in}{-0.020833in}}{\pgfqpoint{0.010825in}{-0.018638in}}{\pgfqpoint{0.014731in}{-0.014731in}}%
\pgfpathcurveto{\pgfqpoint{0.018638in}{-0.010825in}}{\pgfqpoint{0.020833in}{-0.005525in}}{\pgfqpoint{0.020833in}{0.000000in}}%
\pgfpathcurveto{\pgfqpoint{0.020833in}{0.005525in}}{\pgfqpoint{0.018638in}{0.010825in}}{\pgfqpoint{0.014731in}{0.014731in}}%
\pgfpathcurveto{\pgfqpoint{0.010825in}{0.018638in}}{\pgfqpoint{0.005525in}{0.020833in}}{\pgfqpoint{0.000000in}{0.020833in}}%
\pgfpathcurveto{\pgfqpoint{-0.005525in}{0.020833in}}{\pgfqpoint{-0.010825in}{0.018638in}}{\pgfqpoint{-0.014731in}{0.014731in}}%
\pgfpathcurveto{\pgfqpoint{-0.018638in}{0.010825in}}{\pgfqpoint{-0.020833in}{0.005525in}}{\pgfqpoint{-0.020833in}{0.000000in}}%
\pgfpathcurveto{\pgfqpoint{-0.020833in}{-0.005525in}}{\pgfqpoint{-0.018638in}{-0.010825in}}{\pgfqpoint{-0.014731in}{-0.014731in}}%
\pgfpathcurveto{\pgfqpoint{-0.010825in}{-0.018638in}}{\pgfqpoint{-0.005525in}{-0.020833in}}{\pgfqpoint{0.000000in}{-0.020833in}}%
\pgfpathclose%
\pgfusepath{stroke,fill}%
}%
\begin{pgfscope}%
\pgfsys@transformshift{5.500294in}{1.525199in}%
\pgfsys@useobject{currentmarker}{}%
\end{pgfscope}%
\begin{pgfscope}%
\pgfsys@transformshift{5.232494in}{1.545570in}%
\pgfsys@useobject{currentmarker}{}%
\end{pgfscope}%
\begin{pgfscope}%
\pgfsys@transformshift{4.729194in}{1.615115in}%
\pgfsys@useobject{currentmarker}{}%
\end{pgfscope}%
\begin{pgfscope}%
\pgfsys@transformshift{4.051100in}{1.922895in}%
\pgfsys@useobject{currentmarker}{}%
\end{pgfscope}%
\begin{pgfscope}%
\pgfsys@transformshift{3.280000in}{3.300000in}%
\pgfsys@useobject{currentmarker}{}%
\end{pgfscope}%
\begin{pgfscope}%
\pgfsys@transformshift{2.508900in}{1.922895in}%
\pgfsys@useobject{currentmarker}{}%
\end{pgfscope}%
\begin{pgfscope}%
\pgfsys@transformshift{1.830806in}{1.615115in}%
\pgfsys@useobject{currentmarker}{}%
\end{pgfscope}%
\begin{pgfscope}%
\pgfsys@transformshift{1.327506in}{1.545570in}%
\pgfsys@useobject{currentmarker}{}%
\end{pgfscope}%
\begin{pgfscope}%
\pgfsys@transformshift{1.059706in}{1.525199in}%
\pgfsys@useobject{currentmarker}{}%
\end{pgfscope}%
\end{pgfscope}%
\begin{pgfscope}%
\pgfpathrectangle{\pgfqpoint{0.800000in}{0.528000in}}{\pgfqpoint{4.960000in}{3.696000in}}%
\pgfusepath{clip}%
\pgfsetrectcap%
\pgfsetroundjoin%
\pgfsetlinewidth{1.505625pt}%
\definecolor{currentstroke}{rgb}{0.839216,0.152941,0.156863}%
\pgfsetstrokecolor{currentstroke}%
\pgfsetdash{}{0pt}%
\pgfpathmoveto{\pgfqpoint{1.025455in}{1.645381in}}%
\pgfpathlineto{\pgfqpoint{1.048113in}{1.560213in}}%
\pgfpathlineto{\pgfqpoint{1.070772in}{1.496677in}}%
\pgfpathlineto{\pgfqpoint{1.093431in}{1.451764in}}%
\pgfpathlineto{\pgfqpoint{1.116090in}{1.422734in}}%
\pgfpathlineto{\pgfqpoint{1.138748in}{1.407094in}}%
\pgfpathlineto{\pgfqpoint{1.161407in}{1.402591in}}%
\pgfpathlineto{\pgfqpoint{1.184066in}{1.407189in}}%
\pgfpathlineto{\pgfqpoint{1.206725in}{1.419062in}}%
\pgfpathlineto{\pgfqpoint{1.229383in}{1.436578in}}%
\pgfpathlineto{\pgfqpoint{1.252042in}{1.458287in}}%
\pgfpathlineto{\pgfqpoint{1.274701in}{1.482907in}}%
\pgfpathlineto{\pgfqpoint{1.387995in}{1.615321in}}%
\pgfpathlineto{\pgfqpoint{1.410653in}{1.638619in}}%
\pgfpathlineto{\pgfqpoint{1.433312in}{1.659698in}}%
\pgfpathlineto{\pgfqpoint{1.455971in}{1.678257in}}%
\pgfpathlineto{\pgfqpoint{1.478630in}{1.694076in}}%
\pgfpathlineto{\pgfqpoint{1.501288in}{1.707009in}}%
\pgfpathlineto{\pgfqpoint{1.523947in}{1.716976in}}%
\pgfpathlineto{\pgfqpoint{1.546606in}{1.723958in}}%
\pgfpathlineto{\pgfqpoint{1.569265in}{1.727989in}}%
\pgfpathlineto{\pgfqpoint{1.591923in}{1.729151in}}%
\pgfpathlineto{\pgfqpoint{1.614582in}{1.727566in}}%
\pgfpathlineto{\pgfqpoint{1.637241in}{1.723395in}}%
\pgfpathlineto{\pgfqpoint{1.659899in}{1.716827in}}%
\pgfpathlineto{\pgfqpoint{1.682558in}{1.708078in}}%
\pgfpathlineto{\pgfqpoint{1.705217in}{1.697386in}}%
\pgfpathlineto{\pgfqpoint{1.727876in}{1.685008in}}%
\pgfpathlineto{\pgfqpoint{1.773193in}{1.656275in}}%
\pgfpathlineto{\pgfqpoint{1.818511in}{1.624126in}}%
\pgfpathlineto{\pgfqpoint{1.909146in}{1.558757in}}%
\pgfpathlineto{\pgfqpoint{1.954463in}{1.529960in}}%
\pgfpathlineto{\pgfqpoint{1.977122in}{1.517433in}}%
\pgfpathlineto{\pgfqpoint{1.999781in}{1.506462in}}%
\pgfpathlineto{\pgfqpoint{2.022439in}{1.497262in}}%
\pgfpathlineto{\pgfqpoint{2.045098in}{1.490030in}}%
\pgfpathlineto{\pgfqpoint{2.067757in}{1.484947in}}%
\pgfpathlineto{\pgfqpoint{2.090416in}{1.482176in}}%
\pgfpathlineto{\pgfqpoint{2.113074in}{1.481862in}}%
\pgfpathlineto{\pgfqpoint{2.135733in}{1.484131in}}%
\pgfpathlineto{\pgfqpoint{2.158392in}{1.489089in}}%
\pgfpathlineto{\pgfqpoint{2.181051in}{1.496825in}}%
\pgfpathlineto{\pgfqpoint{2.203709in}{1.507405in}}%
\pgfpathlineto{\pgfqpoint{2.226368in}{1.520879in}}%
\pgfpathlineto{\pgfqpoint{2.249027in}{1.537277in}}%
\pgfpathlineto{\pgfqpoint{2.271686in}{1.556609in}}%
\pgfpathlineto{\pgfqpoint{2.294344in}{1.578866in}}%
\pgfpathlineto{\pgfqpoint{2.317003in}{1.604023in}}%
\pgfpathlineto{\pgfqpoint{2.339662in}{1.632035in}}%
\pgfpathlineto{\pgfqpoint{2.362321in}{1.662841in}}%
\pgfpathlineto{\pgfqpoint{2.384979in}{1.696363in}}%
\pgfpathlineto{\pgfqpoint{2.407638in}{1.732507in}}%
\pgfpathlineto{\pgfqpoint{2.430297in}{1.771165in}}%
\pgfpathlineto{\pgfqpoint{2.475614in}{1.855514in}}%
\pgfpathlineto{\pgfqpoint{2.520932in}{1.948269in}}%
\pgfpathlineto{\pgfqpoint{2.566249in}{2.048103in}}%
\pgfpathlineto{\pgfqpoint{2.611567in}{2.153533in}}%
\pgfpathlineto{\pgfqpoint{2.679543in}{2.318637in}}%
\pgfpathlineto{\pgfqpoint{2.815496in}{2.652509in}}%
\pgfpathlineto{\pgfqpoint{2.860813in}{2.758432in}}%
\pgfpathlineto{\pgfqpoint{2.906131in}{2.858816in}}%
\pgfpathlineto{\pgfqpoint{2.951448in}{2.952059in}}%
\pgfpathlineto{\pgfqpoint{2.996766in}{3.036671in}}%
\pgfpathlineto{\pgfqpoint{3.019424in}{3.075312in}}%
\pgfpathlineto{\pgfqpoint{3.042083in}{3.111300in}}%
\pgfpathlineto{\pgfqpoint{3.064742in}{3.144492in}}%
\pgfpathlineto{\pgfqpoint{3.087401in}{3.174755in}}%
\pgfpathlineto{\pgfqpoint{3.110059in}{3.201968in}}%
\pgfpathlineto{\pgfqpoint{3.132718in}{3.226022in}}%
\pgfpathlineto{\pgfqpoint{3.155377in}{3.246821in}}%
\pgfpathlineto{\pgfqpoint{3.178036in}{3.264282in}}%
\pgfpathlineto{\pgfqpoint{3.200694in}{3.278335in}}%
\pgfpathlineto{\pgfqpoint{3.223353in}{3.288924in}}%
\pgfpathlineto{\pgfqpoint{3.246012in}{3.296007in}}%
\pgfpathlineto{\pgfqpoint{3.268671in}{3.299556in}}%
\pgfpathlineto{\pgfqpoint{3.291329in}{3.299556in}}%
\pgfpathlineto{\pgfqpoint{3.313988in}{3.296007in}}%
\pgfpathlineto{\pgfqpoint{3.336647in}{3.288924in}}%
\pgfpathlineto{\pgfqpoint{3.359306in}{3.278335in}}%
\pgfpathlineto{\pgfqpoint{3.381964in}{3.264282in}}%
\pgfpathlineto{\pgfqpoint{3.404623in}{3.246821in}}%
\pgfpathlineto{\pgfqpoint{3.427282in}{3.226022in}}%
\pgfpathlineto{\pgfqpoint{3.449941in}{3.201968in}}%
\pgfpathlineto{\pgfqpoint{3.472599in}{3.174755in}}%
\pgfpathlineto{\pgfqpoint{3.495258in}{3.144492in}}%
\pgfpathlineto{\pgfqpoint{3.517917in}{3.111300in}}%
\pgfpathlineto{\pgfqpoint{3.540576in}{3.075312in}}%
\pgfpathlineto{\pgfqpoint{3.563234in}{3.036671in}}%
\pgfpathlineto{\pgfqpoint{3.585893in}{2.995531in}}%
\pgfpathlineto{\pgfqpoint{3.631211in}{2.906426in}}%
\pgfpathlineto{\pgfqpoint{3.676528in}{2.809419in}}%
\pgfpathlineto{\pgfqpoint{3.721846in}{2.706059in}}%
\pgfpathlineto{\pgfqpoint{3.789822in}{2.542735in}}%
\pgfpathlineto{\pgfqpoint{3.948433in}{2.153533in}}%
\pgfpathlineto{\pgfqpoint{3.993751in}{2.048103in}}%
\pgfpathlineto{\pgfqpoint{4.039068in}{1.948269in}}%
\pgfpathlineto{\pgfqpoint{4.084386in}{1.855514in}}%
\pgfpathlineto{\pgfqpoint{4.129703in}{1.771165in}}%
\pgfpathlineto{\pgfqpoint{4.152362in}{1.732507in}}%
\pgfpathlineto{\pgfqpoint{4.175021in}{1.696363in}}%
\pgfpathlineto{\pgfqpoint{4.197679in}{1.662841in}}%
\pgfpathlineto{\pgfqpoint{4.220338in}{1.632035in}}%
\pgfpathlineto{\pgfqpoint{4.242997in}{1.604023in}}%
\pgfpathlineto{\pgfqpoint{4.265656in}{1.578866in}}%
\pgfpathlineto{\pgfqpoint{4.288314in}{1.556609in}}%
\pgfpathlineto{\pgfqpoint{4.310973in}{1.537277in}}%
\pgfpathlineto{\pgfqpoint{4.333632in}{1.520879in}}%
\pgfpathlineto{\pgfqpoint{4.356291in}{1.507405in}}%
\pgfpathlineto{\pgfqpoint{4.378949in}{1.496825in}}%
\pgfpathlineto{\pgfqpoint{4.401608in}{1.489089in}}%
\pgfpathlineto{\pgfqpoint{4.424267in}{1.484131in}}%
\pgfpathlineto{\pgfqpoint{4.446926in}{1.481862in}}%
\pgfpathlineto{\pgfqpoint{4.469584in}{1.482176in}}%
\pgfpathlineto{\pgfqpoint{4.492243in}{1.484947in}}%
\pgfpathlineto{\pgfqpoint{4.514902in}{1.490030in}}%
\pgfpathlineto{\pgfqpoint{4.537561in}{1.497262in}}%
\pgfpathlineto{\pgfqpoint{4.560219in}{1.506462in}}%
\pgfpathlineto{\pgfqpoint{4.582878in}{1.517433in}}%
\pgfpathlineto{\pgfqpoint{4.605537in}{1.529960in}}%
\pgfpathlineto{\pgfqpoint{4.650854in}{1.558757in}}%
\pgfpathlineto{\pgfqpoint{4.696172in}{1.590865in}}%
\pgfpathlineto{\pgfqpoint{4.786807in}{1.656275in}}%
\pgfpathlineto{\pgfqpoint{4.832124in}{1.685008in}}%
\pgfpathlineto{\pgfqpoint{4.854783in}{1.697386in}}%
\pgfpathlineto{\pgfqpoint{4.877442in}{1.708078in}}%
\pgfpathlineto{\pgfqpoint{4.900101in}{1.716827in}}%
\pgfpathlineto{\pgfqpoint{4.922759in}{1.723395in}}%
\pgfpathlineto{\pgfqpoint{4.945418in}{1.727566in}}%
\pgfpathlineto{\pgfqpoint{4.968077in}{1.729151in}}%
\pgfpathlineto{\pgfqpoint{4.990735in}{1.727989in}}%
\pgfpathlineto{\pgfqpoint{5.013394in}{1.723958in}}%
\pgfpathlineto{\pgfqpoint{5.036053in}{1.716976in}}%
\pgfpathlineto{\pgfqpoint{5.058712in}{1.707009in}}%
\pgfpathlineto{\pgfqpoint{5.081370in}{1.694076in}}%
\pgfpathlineto{\pgfqpoint{5.104029in}{1.678257in}}%
\pgfpathlineto{\pgfqpoint{5.126688in}{1.659698in}}%
\pgfpathlineto{\pgfqpoint{5.149347in}{1.638619in}}%
\pgfpathlineto{\pgfqpoint{5.172005in}{1.615321in}}%
\pgfpathlineto{\pgfqpoint{5.217323in}{1.563739in}}%
\pgfpathlineto{\pgfqpoint{5.285299in}{1.482907in}}%
\pgfpathlineto{\pgfqpoint{5.307958in}{1.458287in}}%
\pgfpathlineto{\pgfqpoint{5.330617in}{1.436578in}}%
\pgfpathlineto{\pgfqpoint{5.353275in}{1.419062in}}%
\pgfpathlineto{\pgfqpoint{5.375934in}{1.407189in}}%
\pgfpathlineto{\pgfqpoint{5.398593in}{1.402591in}}%
\pgfpathlineto{\pgfqpoint{5.421252in}{1.407094in}}%
\pgfpathlineto{\pgfqpoint{5.443910in}{1.422734in}}%
\pgfpathlineto{\pgfqpoint{5.466569in}{1.451764in}}%
\pgfpathlineto{\pgfqpoint{5.489228in}{1.496677in}}%
\pgfpathlineto{\pgfqpoint{5.511887in}{1.560213in}}%
\pgfpathlineto{\pgfqpoint{5.534545in}{1.645381in}}%
\pgfpathlineto{\pgfqpoint{5.534545in}{1.645381in}}%
\pgfusepath{stroke}%
\end{pgfscope}%
\begin{pgfscope}%
\pgfpathrectangle{\pgfqpoint{0.800000in}{0.528000in}}{\pgfqpoint{4.960000in}{3.696000in}}%
\pgfusepath{clip}%
\pgfsetbuttcap%
\pgfsetroundjoin%
\definecolor{currentfill}{rgb}{0.580392,0.403922,0.741176}%
\pgfsetfillcolor{currentfill}%
\pgfsetlinewidth{1.003750pt}%
\definecolor{currentstroke}{rgb}{0.580392,0.403922,0.741176}%
\pgfsetstrokecolor{currentstroke}%
\pgfsetdash{}{0pt}%
\pgfsys@defobject{currentmarker}{\pgfqpoint{-0.020833in}{-0.020833in}}{\pgfqpoint{0.020833in}{0.020833in}}{%
\pgfpathmoveto{\pgfqpoint{0.000000in}{-0.020833in}}%
\pgfpathcurveto{\pgfqpoint{0.005525in}{-0.020833in}}{\pgfqpoint{0.010825in}{-0.018638in}}{\pgfqpoint{0.014731in}{-0.014731in}}%
\pgfpathcurveto{\pgfqpoint{0.018638in}{-0.010825in}}{\pgfqpoint{0.020833in}{-0.005525in}}{\pgfqpoint{0.020833in}{0.000000in}}%
\pgfpathcurveto{\pgfqpoint{0.020833in}{0.005525in}}{\pgfqpoint{0.018638in}{0.010825in}}{\pgfqpoint{0.014731in}{0.014731in}}%
\pgfpathcurveto{\pgfqpoint{0.010825in}{0.018638in}}{\pgfqpoint{0.005525in}{0.020833in}}{\pgfqpoint{0.000000in}{0.020833in}}%
\pgfpathcurveto{\pgfqpoint{-0.005525in}{0.020833in}}{\pgfqpoint{-0.010825in}{0.018638in}}{\pgfqpoint{-0.014731in}{0.014731in}}%
\pgfpathcurveto{\pgfqpoint{-0.018638in}{0.010825in}}{\pgfqpoint{-0.020833in}{0.005525in}}{\pgfqpoint{-0.020833in}{0.000000in}}%
\pgfpathcurveto{\pgfqpoint{-0.020833in}{-0.005525in}}{\pgfqpoint{-0.018638in}{-0.010825in}}{\pgfqpoint{-0.014731in}{-0.014731in}}%
\pgfpathcurveto{\pgfqpoint{-0.010825in}{-0.018638in}}{\pgfqpoint{-0.005525in}{-0.020833in}}{\pgfqpoint{0.000000in}{-0.020833in}}%
\pgfpathclose%
\pgfusepath{stroke,fill}%
}%
\begin{pgfscope}%
\pgfsys@transformshift{5.524928in}{1.523664in}%
\pgfsys@useobject{currentmarker}{}%
\end{pgfscope}%
\begin{pgfscope}%
\pgfsys@transformshift{5.448480in}{1.528592in}%
\pgfsys@useobject{currentmarker}{}%
\end{pgfscope}%
\begin{pgfscope}%
\pgfsys@transformshift{5.298186in}{1.539862in}%
\pgfsys@useobject{currentmarker}{}%
\end{pgfscope}%
\begin{pgfscope}%
\pgfsys@transformshift{5.079166in}{1.561215in}%
\pgfsys@useobject{currentmarker}{}%
\end{pgfscope}%
\begin{pgfscope}%
\pgfsys@transformshift{4.798877in}{1.601676in}%
\pgfsys@useobject{currentmarker}{}%
\end{pgfscope}%
\begin{pgfscope}%
\pgfsys@transformshift{4.466865in}{1.685090in}%
\pgfsys@useobject{currentmarker}{}%
\end{pgfscope}%
\begin{pgfscope}%
\pgfsys@transformshift{4.094436in}{1.885560in}%
\pgfsys@useobject{currentmarker}{}%
\end{pgfscope}%
\begin{pgfscope}%
\pgfsys@transformshift{3.694272in}{2.454116in}%
\pgfsys@useobject{currentmarker}{}%
\end{pgfscope}%
\begin{pgfscope}%
\pgfsys@transformshift{3.280000in}{3.300000in}%
\pgfsys@useobject{currentmarker}{}%
\end{pgfscope}%
\begin{pgfscope}%
\pgfsys@transformshift{2.865728in}{2.454116in}%
\pgfsys@useobject{currentmarker}{}%
\end{pgfscope}%
\begin{pgfscope}%
\pgfsys@transformshift{2.465564in}{1.885560in}%
\pgfsys@useobject{currentmarker}{}%
\end{pgfscope}%
\begin{pgfscope}%
\pgfsys@transformshift{2.093135in}{1.685090in}%
\pgfsys@useobject{currentmarker}{}%
\end{pgfscope}%
\begin{pgfscope}%
\pgfsys@transformshift{1.761123in}{1.601676in}%
\pgfsys@useobject{currentmarker}{}%
\end{pgfscope}%
\begin{pgfscope}%
\pgfsys@transformshift{1.480834in}{1.561215in}%
\pgfsys@useobject{currentmarker}{}%
\end{pgfscope}%
\begin{pgfscope}%
\pgfsys@transformshift{1.261814in}{1.539862in}%
\pgfsys@useobject{currentmarker}{}%
\end{pgfscope}%
\begin{pgfscope}%
\pgfsys@transformshift{1.111520in}{1.528592in}%
\pgfsys@useobject{currentmarker}{}%
\end{pgfscope}%
\begin{pgfscope}%
\pgfsys@transformshift{1.035072in}{1.523664in}%
\pgfsys@useobject{currentmarker}{}%
\end{pgfscope}%
\end{pgfscope}%
\begin{pgfscope}%
\pgfpathrectangle{\pgfqpoint{0.800000in}{0.528000in}}{\pgfqpoint{4.960000in}{3.696000in}}%
\pgfusepath{clip}%
\pgfsetrectcap%
\pgfsetroundjoin%
\pgfsetlinewidth{1.505625pt}%
\definecolor{currentstroke}{rgb}{0.549020,0.337255,0.294118}%
\pgfsetstrokecolor{currentstroke}%
\pgfsetdash{}{0pt}%
\pgfpathmoveto{\pgfqpoint{1.025455in}{1.547360in}}%
\pgfpathlineto{\pgfqpoint{1.048113in}{1.506196in}}%
\pgfpathlineto{\pgfqpoint{1.070772in}{1.502077in}}%
\pgfpathlineto{\pgfqpoint{1.093431in}{1.514814in}}%
\pgfpathlineto{\pgfqpoint{1.116090in}{1.532029in}}%
\pgfpathlineto{\pgfqpoint{1.138748in}{1.546868in}}%
\pgfpathlineto{\pgfqpoint{1.161407in}{1.556241in}}%
\pgfpathlineto{\pgfqpoint{1.184066in}{1.559487in}}%
\pgfpathlineto{\pgfqpoint{1.206725in}{1.557391in}}%
\pgfpathlineto{\pgfqpoint{1.229383in}{1.551474in}}%
\pgfpathlineto{\pgfqpoint{1.297360in}{1.527887in}}%
\pgfpathlineto{\pgfqpoint{1.320018in}{1.522716in}}%
\pgfpathlineto{\pgfqpoint{1.342677in}{1.520310in}}%
\pgfpathlineto{\pgfqpoint{1.365336in}{1.520946in}}%
\pgfpathlineto{\pgfqpoint{1.387995in}{1.524573in}}%
\pgfpathlineto{\pgfqpoint{1.410653in}{1.530871in}}%
\pgfpathlineto{\pgfqpoint{1.433312in}{1.539329in}}%
\pgfpathlineto{\pgfqpoint{1.478630in}{1.560142in}}%
\pgfpathlineto{\pgfqpoint{1.523947in}{1.581649in}}%
\pgfpathlineto{\pgfqpoint{1.546606in}{1.591161in}}%
\pgfpathlineto{\pgfqpoint{1.569265in}{1.599241in}}%
\pgfpathlineto{\pgfqpoint{1.591923in}{1.605596in}}%
\pgfpathlineto{\pgfqpoint{1.614582in}{1.610067in}}%
\pgfpathlineto{\pgfqpoint{1.637241in}{1.612628in}}%
\pgfpathlineto{\pgfqpoint{1.659899in}{1.613375in}}%
\pgfpathlineto{\pgfqpoint{1.682558in}{1.612511in}}%
\pgfpathlineto{\pgfqpoint{1.727876in}{1.607184in}}%
\pgfpathlineto{\pgfqpoint{1.818511in}{1.593223in}}%
\pgfpathlineto{\pgfqpoint{1.841169in}{1.591417in}}%
\pgfpathlineto{\pgfqpoint{1.863828in}{1.590985in}}%
\pgfpathlineto{\pgfqpoint{1.886487in}{1.592181in}}%
\pgfpathlineto{\pgfqpoint{1.909146in}{1.595189in}}%
\pgfpathlineto{\pgfqpoint{1.931804in}{1.600121in}}%
\pgfpathlineto{\pgfqpoint{1.954463in}{1.607012in}}%
\pgfpathlineto{\pgfqpoint{1.977122in}{1.615825in}}%
\pgfpathlineto{\pgfqpoint{1.999781in}{1.626457in}}%
\pgfpathlineto{\pgfqpoint{2.022439in}{1.638742in}}%
\pgfpathlineto{\pgfqpoint{2.067757in}{1.667361in}}%
\pgfpathlineto{\pgfqpoint{2.113074in}{1.699525in}}%
\pgfpathlineto{\pgfqpoint{2.203709in}{1.764901in}}%
\pgfpathlineto{\pgfqpoint{2.249027in}{1.793977in}}%
\pgfpathlineto{\pgfqpoint{2.294344in}{1.818961in}}%
\pgfpathlineto{\pgfqpoint{2.339662in}{1.839711in}}%
\pgfpathlineto{\pgfqpoint{2.384979in}{1.857071in}}%
\pgfpathlineto{\pgfqpoint{2.475614in}{1.889487in}}%
\pgfpathlineto{\pgfqpoint{2.498273in}{1.899107in}}%
\pgfpathlineto{\pgfqpoint{2.520932in}{1.910124in}}%
\pgfpathlineto{\pgfqpoint{2.543591in}{1.922946in}}%
\pgfpathlineto{\pgfqpoint{2.566249in}{1.937973in}}%
\pgfpathlineto{\pgfqpoint{2.588908in}{1.955585in}}%
\pgfpathlineto{\pgfqpoint{2.611567in}{1.976132in}}%
\pgfpathlineto{\pgfqpoint{2.634226in}{1.999924in}}%
\pgfpathlineto{\pgfqpoint{2.656884in}{2.027221in}}%
\pgfpathlineto{\pgfqpoint{2.679543in}{2.058226in}}%
\pgfpathlineto{\pgfqpoint{2.702202in}{2.093079in}}%
\pgfpathlineto{\pgfqpoint{2.724861in}{2.131847in}}%
\pgfpathlineto{\pgfqpoint{2.747519in}{2.174524in}}%
\pgfpathlineto{\pgfqpoint{2.770178in}{2.221028in}}%
\pgfpathlineto{\pgfqpoint{2.792837in}{2.271197in}}%
\pgfpathlineto{\pgfqpoint{2.815496in}{2.324790in}}%
\pgfpathlineto{\pgfqpoint{2.860813in}{2.440911in}}%
\pgfpathlineto{\pgfqpoint{2.906131in}{2.566015in}}%
\pgfpathlineto{\pgfqpoint{3.019424in}{2.887771in}}%
\pgfpathlineto{\pgfqpoint{3.064742in}{3.006030in}}%
\pgfpathlineto{\pgfqpoint{3.087401in}{3.060166in}}%
\pgfpathlineto{\pgfqpoint{3.110059in}{3.110107in}}%
\pgfpathlineto{\pgfqpoint{3.132718in}{3.155249in}}%
\pgfpathlineto{\pgfqpoint{3.155377in}{3.195041in}}%
\pgfpathlineto{\pgfqpoint{3.178036in}{3.228992in}}%
\pgfpathlineto{\pgfqpoint{3.200694in}{3.256680in}}%
\pgfpathlineto{\pgfqpoint{3.223353in}{3.277757in}}%
\pgfpathlineto{\pgfqpoint{3.246012in}{3.291959in}}%
\pgfpathlineto{\pgfqpoint{3.268671in}{3.299105in}}%
\pgfpathlineto{\pgfqpoint{3.291329in}{3.299105in}}%
\pgfpathlineto{\pgfqpoint{3.313988in}{3.291959in}}%
\pgfpathlineto{\pgfqpoint{3.336647in}{3.277757in}}%
\pgfpathlineto{\pgfqpoint{3.359306in}{3.256680in}}%
\pgfpathlineto{\pgfqpoint{3.381964in}{3.228992in}}%
\pgfpathlineto{\pgfqpoint{3.404623in}{3.195041in}}%
\pgfpathlineto{\pgfqpoint{3.427282in}{3.155249in}}%
\pgfpathlineto{\pgfqpoint{3.449941in}{3.110107in}}%
\pgfpathlineto{\pgfqpoint{3.472599in}{3.060166in}}%
\pgfpathlineto{\pgfqpoint{3.495258in}{3.006030in}}%
\pgfpathlineto{\pgfqpoint{3.540576in}{2.887771in}}%
\pgfpathlineto{\pgfqpoint{3.585893in}{2.760774in}}%
\pgfpathlineto{\pgfqpoint{3.676528in}{2.502592in}}%
\pgfpathlineto{\pgfqpoint{3.721846in}{2.381492in}}%
\pgfpathlineto{\pgfqpoint{3.744504in}{2.324790in}}%
\pgfpathlineto{\pgfqpoint{3.767163in}{2.271197in}}%
\pgfpathlineto{\pgfqpoint{3.789822in}{2.221028in}}%
\pgfpathlineto{\pgfqpoint{3.812481in}{2.174524in}}%
\pgfpathlineto{\pgfqpoint{3.835139in}{2.131847in}}%
\pgfpathlineto{\pgfqpoint{3.857798in}{2.093079in}}%
\pgfpathlineto{\pgfqpoint{3.880457in}{2.058226in}}%
\pgfpathlineto{\pgfqpoint{3.903116in}{2.027221in}}%
\pgfpathlineto{\pgfqpoint{3.925774in}{1.999924in}}%
\pgfpathlineto{\pgfqpoint{3.948433in}{1.976132in}}%
\pgfpathlineto{\pgfqpoint{3.971092in}{1.955585in}}%
\pgfpathlineto{\pgfqpoint{3.993751in}{1.937973in}}%
\pgfpathlineto{\pgfqpoint{4.016409in}{1.922946in}}%
\pgfpathlineto{\pgfqpoint{4.039068in}{1.910124in}}%
\pgfpathlineto{\pgfqpoint{4.061727in}{1.899107in}}%
\pgfpathlineto{\pgfqpoint{4.107044in}{1.880855in}}%
\pgfpathlineto{\pgfqpoint{4.220338in}{1.839711in}}%
\pgfpathlineto{\pgfqpoint{4.265656in}{1.818961in}}%
\pgfpathlineto{\pgfqpoint{4.310973in}{1.793977in}}%
\pgfpathlineto{\pgfqpoint{4.356291in}{1.764901in}}%
\pgfpathlineto{\pgfqpoint{4.401608in}{1.732807in}}%
\pgfpathlineto{\pgfqpoint{4.492243in}{1.667361in}}%
\pgfpathlineto{\pgfqpoint{4.537561in}{1.638742in}}%
\pgfpathlineto{\pgfqpoint{4.560219in}{1.626457in}}%
\pgfpathlineto{\pgfqpoint{4.582878in}{1.615825in}}%
\pgfpathlineto{\pgfqpoint{4.605537in}{1.607012in}}%
\pgfpathlineto{\pgfqpoint{4.628196in}{1.600121in}}%
\pgfpathlineto{\pgfqpoint{4.650854in}{1.595189in}}%
\pgfpathlineto{\pgfqpoint{4.673513in}{1.592181in}}%
\pgfpathlineto{\pgfqpoint{4.696172in}{1.590985in}}%
\pgfpathlineto{\pgfqpoint{4.718831in}{1.591417in}}%
\pgfpathlineto{\pgfqpoint{4.764148in}{1.596086in}}%
\pgfpathlineto{\pgfqpoint{4.854783in}{1.610329in}}%
\pgfpathlineto{\pgfqpoint{4.877442in}{1.612511in}}%
\pgfpathlineto{\pgfqpoint{4.900101in}{1.613375in}}%
\pgfpathlineto{\pgfqpoint{4.922759in}{1.612628in}}%
\pgfpathlineto{\pgfqpoint{4.945418in}{1.610067in}}%
\pgfpathlineto{\pgfqpoint{4.968077in}{1.605596in}}%
\pgfpathlineto{\pgfqpoint{4.990735in}{1.599241in}}%
\pgfpathlineto{\pgfqpoint{5.013394in}{1.591161in}}%
\pgfpathlineto{\pgfqpoint{5.058712in}{1.571130in}}%
\pgfpathlineto{\pgfqpoint{5.126688in}{1.539329in}}%
\pgfpathlineto{\pgfqpoint{5.149347in}{1.530871in}}%
\pgfpathlineto{\pgfqpoint{5.172005in}{1.524573in}}%
\pgfpathlineto{\pgfqpoint{5.194664in}{1.520946in}}%
\pgfpathlineto{\pgfqpoint{5.217323in}{1.520310in}}%
\pgfpathlineto{\pgfqpoint{5.239982in}{1.522716in}}%
\pgfpathlineto{\pgfqpoint{5.262640in}{1.527887in}}%
\pgfpathlineto{\pgfqpoint{5.307958in}{1.543503in}}%
\pgfpathlineto{\pgfqpoint{5.330617in}{1.551474in}}%
\pgfpathlineto{\pgfqpoint{5.353275in}{1.557391in}}%
\pgfpathlineto{\pgfqpoint{5.375934in}{1.559487in}}%
\pgfpathlineto{\pgfqpoint{5.398593in}{1.556241in}}%
\pgfpathlineto{\pgfqpoint{5.421252in}{1.546868in}}%
\pgfpathlineto{\pgfqpoint{5.443910in}{1.532029in}}%
\pgfpathlineto{\pgfqpoint{5.466569in}{1.514814in}}%
\pgfpathlineto{\pgfqpoint{5.489228in}{1.502077in}}%
\pgfpathlineto{\pgfqpoint{5.511887in}{1.506196in}}%
\pgfpathlineto{\pgfqpoint{5.534545in}{1.547360in}}%
\pgfpathlineto{\pgfqpoint{5.534545in}{1.547360in}}%
\pgfusepath{stroke}%
\end{pgfscope}%
\begin{pgfscope}%
\pgfpathrectangle{\pgfqpoint{0.800000in}{0.528000in}}{\pgfqpoint{4.960000in}{3.696000in}}%
\pgfusepath{clip}%
\pgfsetbuttcap%
\pgfsetroundjoin%
\definecolor{currentfill}{rgb}{0.890196,0.466667,0.760784}%
\pgfsetfillcolor{currentfill}%
\pgfsetlinewidth{1.003750pt}%
\definecolor{currentstroke}{rgb}{0.890196,0.466667,0.760784}%
\pgfsetstrokecolor{currentstroke}%
\pgfsetdash{}{0pt}%
\pgfsys@defobject{currentmarker}{\pgfqpoint{-0.020833in}{-0.020833in}}{\pgfqpoint{0.020833in}{0.020833in}}{%
\pgfpathmoveto{\pgfqpoint{0.000000in}{-0.020833in}}%
\pgfpathcurveto{\pgfqpoint{0.005525in}{-0.020833in}}{\pgfqpoint{0.010825in}{-0.018638in}}{\pgfqpoint{0.014731in}{-0.014731in}}%
\pgfpathcurveto{\pgfqpoint{0.018638in}{-0.010825in}}{\pgfqpoint{0.020833in}{-0.005525in}}{\pgfqpoint{0.020833in}{0.000000in}}%
\pgfpathcurveto{\pgfqpoint{0.020833in}{0.005525in}}{\pgfqpoint{0.018638in}{0.010825in}}{\pgfqpoint{0.014731in}{0.014731in}}%
\pgfpathcurveto{\pgfqpoint{0.010825in}{0.018638in}}{\pgfqpoint{0.005525in}{0.020833in}}{\pgfqpoint{0.000000in}{0.020833in}}%
\pgfpathcurveto{\pgfqpoint{-0.005525in}{0.020833in}}{\pgfqpoint{-0.010825in}{0.018638in}}{\pgfqpoint{-0.014731in}{0.014731in}}%
\pgfpathcurveto{\pgfqpoint{-0.018638in}{0.010825in}}{\pgfqpoint{-0.020833in}{0.005525in}}{\pgfqpoint{-0.020833in}{0.000000in}}%
\pgfpathcurveto{\pgfqpoint{-0.020833in}{-0.005525in}}{\pgfqpoint{-0.018638in}{-0.010825in}}{\pgfqpoint{-0.014731in}{-0.014731in}}%
\pgfpathcurveto{\pgfqpoint{-0.010825in}{-0.018638in}}{\pgfqpoint{-0.005525in}{-0.020833in}}{\pgfqpoint{0.000000in}{-0.020833in}}%
\pgfpathclose%
\pgfusepath{stroke,fill}%
}%
\begin{pgfscope}%
\pgfsys@transformshift{5.531992in}{1.523232in}%
\pgfsys@useobject{currentmarker}{}%
\end{pgfscope}%
\begin{pgfscope}%
\pgfsys@transformshift{5.511597in}{1.524489in}%
\pgfsys@useobject{currentmarker}{}%
\end{pgfscope}%
\begin{pgfscope}%
\pgfsys@transformshift{5.470993in}{1.527090in}%
\pgfsys@useobject{currentmarker}{}%
\end{pgfscope}%
\begin{pgfscope}%
\pgfsys@transformshift{5.410547in}{1.531226in}%
\pgfsys@useobject{currentmarker}{}%
\end{pgfscope}%
\begin{pgfscope}%
\pgfsys@transformshift{5.330807in}{1.537217in}%
\pgfsys@useobject{currentmarker}{}%
\end{pgfscope}%
\begin{pgfscope}%
\pgfsys@transformshift{5.232494in}{1.545570in}%
\pgfsys@useobject{currentmarker}{}%
\end{pgfscope}%
\begin{pgfscope}%
\pgfsys@transformshift{5.116499in}{1.557070in}%
\pgfsys@useobject{currentmarker}{}%
\end{pgfscope}%
\begin{pgfscope}%
\pgfsys@transformshift{4.983872in}{1.572951in}%
\pgfsys@useobject{currentmarker}{}%
\end{pgfscope}%
\begin{pgfscope}%
\pgfsys@transformshift{4.835814in}{1.595198in}%
\pgfsys@useobject{currentmarker}{}%
\end{pgfscope}%
\begin{pgfscope}%
\pgfsys@transformshift{4.673668in}{1.627116in}%
\pgfsys@useobject{currentmarker}{}%
\end{pgfscope}%
\begin{pgfscope}%
\pgfsys@transformshift{4.498899in}{1.674455in}%
\pgfsys@useobject{currentmarker}{}%
\end{pgfscope}%
\begin{pgfscope}%
\pgfsys@transformshift{4.313093in}{1.747714in}%
\pgfsys@useobject{currentmarker}{}%
\end{pgfscope}%
\begin{pgfscope}%
\pgfsys@transformshift{4.117930in}{1.866971in}%
\pgfsys@useobject{currentmarker}{}%
\end{pgfscope}%
\begin{pgfscope}%
\pgfsys@transformshift{3.915179in}{2.071234in}%
\pgfsys@useobject{currentmarker}{}%
\end{pgfscope}%
\begin{pgfscope}%
\pgfsys@transformshift{3.706676in}{2.426992in}%
\pgfsys@useobject{currentmarker}{}%
\end{pgfscope}%
\begin{pgfscope}%
\pgfsys@transformshift{3.494308in}{2.959475in}%
\pgfsys@useobject{currentmarker}{}%
\end{pgfscope}%
\begin{pgfscope}%
\pgfsys@transformshift{3.280000in}{3.300000in}%
\pgfsys@useobject{currentmarker}{}%
\end{pgfscope}%
\begin{pgfscope}%
\pgfsys@transformshift{3.065692in}{2.959475in}%
\pgfsys@useobject{currentmarker}{}%
\end{pgfscope}%
\begin{pgfscope}%
\pgfsys@transformshift{2.853324in}{2.426992in}%
\pgfsys@useobject{currentmarker}{}%
\end{pgfscope}%
\begin{pgfscope}%
\pgfsys@transformshift{2.644821in}{2.071234in}%
\pgfsys@useobject{currentmarker}{}%
\end{pgfscope}%
\begin{pgfscope}%
\pgfsys@transformshift{2.442070in}{1.866971in}%
\pgfsys@useobject{currentmarker}{}%
\end{pgfscope}%
\begin{pgfscope}%
\pgfsys@transformshift{2.246907in}{1.747714in}%
\pgfsys@useobject{currentmarker}{}%
\end{pgfscope}%
\begin{pgfscope}%
\pgfsys@transformshift{2.061101in}{1.674455in}%
\pgfsys@useobject{currentmarker}{}%
\end{pgfscope}%
\begin{pgfscope}%
\pgfsys@transformshift{1.886332in}{1.627116in}%
\pgfsys@useobject{currentmarker}{}%
\end{pgfscope}%
\begin{pgfscope}%
\pgfsys@transformshift{1.724186in}{1.595198in}%
\pgfsys@useobject{currentmarker}{}%
\end{pgfscope}%
\begin{pgfscope}%
\pgfsys@transformshift{1.576128in}{1.572951in}%
\pgfsys@useobject{currentmarker}{}%
\end{pgfscope}%
\begin{pgfscope}%
\pgfsys@transformshift{1.443501in}{1.557070in}%
\pgfsys@useobject{currentmarker}{}%
\end{pgfscope}%
\begin{pgfscope}%
\pgfsys@transformshift{1.327506in}{1.545570in}%
\pgfsys@useobject{currentmarker}{}%
\end{pgfscope}%
\begin{pgfscope}%
\pgfsys@transformshift{1.229193in}{1.537217in}%
\pgfsys@useobject{currentmarker}{}%
\end{pgfscope}%
\begin{pgfscope}%
\pgfsys@transformshift{1.149453in}{1.531226in}%
\pgfsys@useobject{currentmarker}{}%
\end{pgfscope}%
\begin{pgfscope}%
\pgfsys@transformshift{1.089007in}{1.527090in}%
\pgfsys@useobject{currentmarker}{}%
\end{pgfscope}%
\begin{pgfscope}%
\pgfsys@transformshift{1.048403in}{1.524489in}%
\pgfsys@useobject{currentmarker}{}%
\end{pgfscope}%
\begin{pgfscope}%
\pgfsys@transformshift{1.028008in}{1.523232in}%
\pgfsys@useobject{currentmarker}{}%
\end{pgfscope}%
\end{pgfscope}%
\begin{pgfscope}%
\pgfpathrectangle{\pgfqpoint{0.800000in}{0.528000in}}{\pgfqpoint{4.960000in}{3.696000in}}%
\pgfusepath{clip}%
\pgfsetrectcap%
\pgfsetroundjoin%
\pgfsetlinewidth{1.505625pt}%
\definecolor{currentstroke}{rgb}{0.498039,0.498039,0.498039}%
\pgfsetstrokecolor{currentstroke}%
\pgfsetdash{}{0pt}%
\pgfpathmoveto{\pgfqpoint{1.025455in}{1.524087in}}%
\pgfpathlineto{\pgfqpoint{1.048113in}{1.524440in}}%
\pgfpathlineto{\pgfqpoint{1.070772in}{1.526873in}}%
\pgfpathlineto{\pgfqpoint{1.116090in}{1.527855in}}%
\pgfpathlineto{\pgfqpoint{1.206725in}{1.536256in}}%
\pgfpathlineto{\pgfqpoint{1.274701in}{1.539818in}}%
\pgfpathlineto{\pgfqpoint{1.320018in}{1.544641in}}%
\pgfpathlineto{\pgfqpoint{1.387995in}{1.552496in}}%
\pgfpathlineto{\pgfqpoint{1.523947in}{1.565133in}}%
\pgfpathlineto{\pgfqpoint{1.591923in}{1.575523in}}%
\pgfpathlineto{\pgfqpoint{1.682558in}{1.589409in}}%
\pgfpathlineto{\pgfqpoint{1.773193in}{1.602691in}}%
\pgfpathlineto{\pgfqpoint{1.818511in}{1.611202in}}%
\pgfpathlineto{\pgfqpoint{1.863828in}{1.621482in}}%
\pgfpathlineto{\pgfqpoint{1.954463in}{1.645184in}}%
\pgfpathlineto{\pgfqpoint{2.067757in}{1.676409in}}%
\pgfpathlineto{\pgfqpoint{2.113074in}{1.690701in}}%
\pgfpathlineto{\pgfqpoint{2.158392in}{1.707310in}}%
\pgfpathlineto{\pgfqpoint{2.203709in}{1.726698in}}%
\pgfpathlineto{\pgfqpoint{2.249027in}{1.748803in}}%
\pgfpathlineto{\pgfqpoint{2.294344in}{1.773210in}}%
\pgfpathlineto{\pgfqpoint{2.339662in}{1.799528in}}%
\pgfpathlineto{\pgfqpoint{2.384979in}{1.827754in}}%
\pgfpathlineto{\pgfqpoint{2.430297in}{1.858462in}}%
\pgfpathlineto{\pgfqpoint{2.475614in}{1.892728in}}%
\pgfpathlineto{\pgfqpoint{2.520932in}{1.931841in}}%
\pgfpathlineto{\pgfqpoint{2.543591in}{1.953585in}}%
\pgfpathlineto{\pgfqpoint{2.566249in}{1.976940in}}%
\pgfpathlineto{\pgfqpoint{2.588908in}{2.001982in}}%
\pgfpathlineto{\pgfqpoint{2.611567in}{2.028758in}}%
\pgfpathlineto{\pgfqpoint{2.656884in}{2.087579in}}%
\pgfpathlineto{\pgfqpoint{2.702202in}{2.153419in}}%
\pgfpathlineto{\pgfqpoint{2.747519in}{2.226346in}}%
\pgfpathlineto{\pgfqpoint{2.792837in}{2.306743in}}%
\pgfpathlineto{\pgfqpoint{2.838154in}{2.395348in}}%
\pgfpathlineto{\pgfqpoint{2.883472in}{2.492972in}}%
\pgfpathlineto{\pgfqpoint{2.928789in}{2.599908in}}%
\pgfpathlineto{\pgfqpoint{2.974107in}{2.715187in}}%
\pgfpathlineto{\pgfqpoint{3.087401in}{3.015458in}}%
\pgfpathlineto{\pgfqpoint{3.110059in}{3.071205in}}%
\pgfpathlineto{\pgfqpoint{3.132718in}{3.123136in}}%
\pgfpathlineto{\pgfqpoint{3.155377in}{3.170159in}}%
\pgfpathlineto{\pgfqpoint{3.178036in}{3.211225in}}%
\pgfpathlineto{\pgfqpoint{3.200694in}{3.245372in}}%
\pgfpathlineto{\pgfqpoint{3.223353in}{3.271766in}}%
\pgfpathlineto{\pgfqpoint{3.246012in}{3.289747in}}%
\pgfpathlineto{\pgfqpoint{3.268671in}{3.298856in}}%
\pgfpathlineto{\pgfqpoint{3.291329in}{3.298856in}}%
\pgfpathlineto{\pgfqpoint{3.313988in}{3.289747in}}%
\pgfpathlineto{\pgfqpoint{3.336647in}{3.271766in}}%
\pgfpathlineto{\pgfqpoint{3.359306in}{3.245372in}}%
\pgfpathlineto{\pgfqpoint{3.381964in}{3.211225in}}%
\pgfpathlineto{\pgfqpoint{3.404623in}{3.170159in}}%
\pgfpathlineto{\pgfqpoint{3.427282in}{3.123136in}}%
\pgfpathlineto{\pgfqpoint{3.449941in}{3.071205in}}%
\pgfpathlineto{\pgfqpoint{3.495258in}{2.956982in}}%
\pgfpathlineto{\pgfqpoint{3.608552in}{2.656634in}}%
\pgfpathlineto{\pgfqpoint{3.653869in}{2.545298in}}%
\pgfpathlineto{\pgfqpoint{3.699187in}{2.442994in}}%
\pgfpathlineto{\pgfqpoint{3.744504in}{2.349965in}}%
\pgfpathlineto{\pgfqpoint{3.789822in}{2.265571in}}%
\pgfpathlineto{\pgfqpoint{3.835139in}{2.188982in}}%
\pgfpathlineto{\pgfqpoint{3.880457in}{2.119622in}}%
\pgfpathlineto{\pgfqpoint{3.925774in}{2.057289in}}%
\pgfpathlineto{\pgfqpoint{3.971092in}{2.001982in}}%
\pgfpathlineto{\pgfqpoint{3.993751in}{1.976940in}}%
\pgfpathlineto{\pgfqpoint{4.016409in}{1.953585in}}%
\pgfpathlineto{\pgfqpoint{4.061727in}{1.911600in}}%
\pgfpathlineto{\pgfqpoint{4.107044in}{1.875071in}}%
\pgfpathlineto{\pgfqpoint{4.152362in}{1.842740in}}%
\pgfpathlineto{\pgfqpoint{4.197679in}{1.813380in}}%
\pgfpathlineto{\pgfqpoint{4.242997in}{1.786143in}}%
\pgfpathlineto{\pgfqpoint{4.288314in}{1.760748in}}%
\pgfpathlineto{\pgfqpoint{4.333632in}{1.737432in}}%
\pgfpathlineto{\pgfqpoint{4.378949in}{1.716649in}}%
\pgfpathlineto{\pgfqpoint{4.424267in}{1.698675in}}%
\pgfpathlineto{\pgfqpoint{4.469584in}{1.683314in}}%
\pgfpathlineto{\pgfqpoint{4.537561in}{1.663566in}}%
\pgfpathlineto{\pgfqpoint{4.696172in}{1.621482in}}%
\pgfpathlineto{\pgfqpoint{4.741489in}{1.611202in}}%
\pgfpathlineto{\pgfqpoint{4.786807in}{1.602691in}}%
\pgfpathlineto{\pgfqpoint{4.854783in}{1.592552in}}%
\pgfpathlineto{\pgfqpoint{4.945418in}{1.579207in}}%
\pgfpathlineto{\pgfqpoint{5.036053in}{1.565133in}}%
\pgfpathlineto{\pgfqpoint{5.081370in}{1.560018in}}%
\pgfpathlineto{\pgfqpoint{5.217323in}{1.547458in}}%
\pgfpathlineto{\pgfqpoint{5.285299in}{1.539818in}}%
\pgfpathlineto{\pgfqpoint{5.353275in}{1.536256in}}%
\pgfpathlineto{\pgfqpoint{5.398593in}{1.532618in}}%
\pgfpathlineto{\pgfqpoint{5.443910in}{1.527855in}}%
\pgfpathlineto{\pgfqpoint{5.489228in}{1.526873in}}%
\pgfpathlineto{\pgfqpoint{5.511887in}{1.524440in}}%
\pgfpathlineto{\pgfqpoint{5.534545in}{1.524087in}}%
\pgfpathlineto{\pgfqpoint{5.534545in}{1.524087in}}%
\pgfusepath{stroke}%
\end{pgfscope}%
\begin{pgfscope}%
\pgfpathrectangle{\pgfqpoint{0.800000in}{0.528000in}}{\pgfqpoint{4.960000in}{3.696000in}}%
\pgfusepath{clip}%
\pgfsetrectcap%
\pgfsetroundjoin%
\pgfsetlinewidth{1.505625pt}%
\definecolor{currentstroke}{rgb}{0.737255,0.741176,0.133333}%
\pgfsetstrokecolor{currentstroke}%
\pgfsetdash{}{0pt}%
\pgfpathmoveto{\pgfqpoint{1.025455in}{2.376000in}}%
\pgfpathlineto{\pgfqpoint{1.138748in}{2.423597in}}%
\pgfpathlineto{\pgfqpoint{1.252042in}{2.473505in}}%
\pgfpathlineto{\pgfqpoint{1.365336in}{2.525658in}}%
\pgfpathlineto{\pgfqpoint{1.478630in}{2.579935in}}%
\pgfpathlineto{\pgfqpoint{1.614582in}{2.647600in}}%
\pgfpathlineto{\pgfqpoint{1.773193in}{2.729407in}}%
\pgfpathlineto{\pgfqpoint{1.999781in}{2.849415in}}%
\pgfpathlineto{\pgfqpoint{2.226368in}{2.968739in}}%
\pgfpathlineto{\pgfqpoint{2.339662in}{3.026159in}}%
\pgfpathlineto{\pgfqpoint{2.430297in}{3.070154in}}%
\pgfpathlineto{\pgfqpoint{2.520932in}{3.111847in}}%
\pgfpathlineto{\pgfqpoint{2.611567in}{3.150683in}}%
\pgfpathlineto{\pgfqpoint{2.679543in}{3.177599in}}%
\pgfpathlineto{\pgfqpoint{2.747519in}{3.202362in}}%
\pgfpathlineto{\pgfqpoint{2.815496in}{3.224750in}}%
\pgfpathlineto{\pgfqpoint{2.883472in}{3.244550in}}%
\pgfpathlineto{\pgfqpoint{2.951448in}{3.261570in}}%
\pgfpathlineto{\pgfqpoint{3.019424in}{3.275639in}}%
\pgfpathlineto{\pgfqpoint{3.087401in}{3.286611in}}%
\pgfpathlineto{\pgfqpoint{3.155377in}{3.294371in}}%
\pgfpathlineto{\pgfqpoint{3.223353in}{3.298834in}}%
\pgfpathlineto{\pgfqpoint{3.268671in}{3.299953in}}%
\pgfpathlineto{\pgfqpoint{3.313988in}{3.299580in}}%
\pgfpathlineto{\pgfqpoint{3.359306in}{3.297716in}}%
\pgfpathlineto{\pgfqpoint{3.427282in}{3.292147in}}%
\pgfpathlineto{\pgfqpoint{3.495258in}{3.283306in}}%
\pgfpathlineto{\pgfqpoint{3.563234in}{3.271287in}}%
\pgfpathlineto{\pgfqpoint{3.631211in}{3.256217in}}%
\pgfpathlineto{\pgfqpoint{3.699187in}{3.238250in}}%
\pgfpathlineto{\pgfqpoint{3.767163in}{3.217564in}}%
\pgfpathlineto{\pgfqpoint{3.835139in}{3.194361in}}%
\pgfpathlineto{\pgfqpoint{3.903116in}{3.168855in}}%
\pgfpathlineto{\pgfqpoint{3.971092in}{3.141272in}}%
\pgfpathlineto{\pgfqpoint{4.061727in}{3.101670in}}%
\pgfpathlineto{\pgfqpoint{4.152362in}{3.059350in}}%
\pgfpathlineto{\pgfqpoint{4.265656in}{3.003466in}}%
\pgfpathlineto{\pgfqpoint{4.401608in}{2.933370in}}%
\pgfpathlineto{\pgfqpoint{4.628196in}{2.813234in}}%
\pgfpathlineto{\pgfqpoint{4.854783in}{2.694026in}}%
\pgfpathlineto{\pgfqpoint{4.990735in}{2.624761in}}%
\pgfpathlineto{\pgfqpoint{5.126688in}{2.557979in}}%
\pgfpathlineto{\pgfqpoint{5.239982in}{2.504533in}}%
\pgfpathlineto{\pgfqpoint{5.353275in}{2.453268in}}%
\pgfpathlineto{\pgfqpoint{5.466569in}{2.404279in}}%
\pgfpathlineto{\pgfqpoint{5.534545in}{2.376000in}}%
\pgfpathlineto{\pgfqpoint{5.534545in}{2.376000in}}%
\pgfusepath{stroke}%
\end{pgfscope}%
\begin{pgfscope}%
\pgfpathrectangle{\pgfqpoint{0.800000in}{0.528000in}}{\pgfqpoint{4.960000in}{3.696000in}}%
\pgfusepath{clip}%
\pgfsetbuttcap%
\pgfsetroundjoin%
\definecolor{currentfill}{rgb}{0.090196,0.745098,0.811765}%
\pgfsetfillcolor{currentfill}%
\pgfsetlinewidth{1.003750pt}%
\definecolor{currentstroke}{rgb}{0.090196,0.745098,0.811765}%
\pgfsetstrokecolor{currentstroke}%
\pgfsetdash{}{0pt}%
\pgfsys@defobject{currentmarker}{\pgfqpoint{-0.020833in}{-0.020833in}}{\pgfqpoint{0.020833in}{0.020833in}}{%
\pgfpathmoveto{\pgfqpoint{0.000000in}{-0.020833in}}%
\pgfpathcurveto{\pgfqpoint{0.005525in}{-0.020833in}}{\pgfqpoint{0.010825in}{-0.018638in}}{\pgfqpoint{0.014731in}{-0.014731in}}%
\pgfpathcurveto{\pgfqpoint{0.018638in}{-0.010825in}}{\pgfqpoint{0.020833in}{-0.005525in}}{\pgfqpoint{0.020833in}{0.000000in}}%
\pgfpathcurveto{\pgfqpoint{0.020833in}{0.005525in}}{\pgfqpoint{0.018638in}{0.010825in}}{\pgfqpoint{0.014731in}{0.014731in}}%
\pgfpathcurveto{\pgfqpoint{0.010825in}{0.018638in}}{\pgfqpoint{0.005525in}{0.020833in}}{\pgfqpoint{0.000000in}{0.020833in}}%
\pgfpathcurveto{\pgfqpoint{-0.005525in}{0.020833in}}{\pgfqpoint{-0.010825in}{0.018638in}}{\pgfqpoint{-0.014731in}{0.014731in}}%
\pgfpathcurveto{\pgfqpoint{-0.018638in}{0.010825in}}{\pgfqpoint{-0.020833in}{0.005525in}}{\pgfqpoint{-0.020833in}{0.000000in}}%
\pgfpathcurveto{\pgfqpoint{-0.020833in}{-0.005525in}}{\pgfqpoint{-0.018638in}{-0.010825in}}{\pgfqpoint{-0.014731in}{-0.014731in}}%
\pgfpathcurveto{\pgfqpoint{-0.010825in}{-0.018638in}}{\pgfqpoint{-0.005525in}{-0.020833in}}{\pgfqpoint{0.000000in}{-0.020833in}}%
\pgfpathclose%
\pgfusepath{stroke,fill}%
}%
\begin{pgfscope}%
\pgfsys@transformshift{1.025455in}{2.376000in}%
\pgfsys@useobject{currentmarker}{}%
\end{pgfscope}%
\begin{pgfscope}%
\pgfsys@transformshift{1.669610in}{2.675676in}%
\pgfsys@useobject{currentmarker}{}%
\end{pgfscope}%
\begin{pgfscope}%
\pgfsys@transformshift{2.313766in}{3.013241in}%
\pgfsys@useobject{currentmarker}{}%
\end{pgfscope}%
\begin{pgfscope}%
\pgfsys@transformshift{2.957922in}{3.263040in}%
\pgfsys@useobject{currentmarker}{}%
\end{pgfscope}%
\begin{pgfscope}%
\pgfsys@transformshift{3.602078in}{3.263040in}%
\pgfsys@useobject{currentmarker}{}%
\end{pgfscope}%
\begin{pgfscope}%
\pgfsys@transformshift{4.246234in}{3.013241in}%
\pgfsys@useobject{currentmarker}{}%
\end{pgfscope}%
\begin{pgfscope}%
\pgfsys@transformshift{4.890390in}{2.675676in}%
\pgfsys@useobject{currentmarker}{}%
\end{pgfscope}%
\end{pgfscope}%
\begin{pgfscope}%
\pgfpathrectangle{\pgfqpoint{0.800000in}{0.528000in}}{\pgfqpoint{4.960000in}{3.696000in}}%
\pgfusepath{clip}%
\pgfsetrectcap%
\pgfsetroundjoin%
\pgfsetlinewidth{1.505625pt}%
\definecolor{currentstroke}{rgb}{0.121569,0.466667,0.705882}%
\pgfsetstrokecolor{currentstroke}%
\pgfsetdash{}{0pt}%
\pgfpathmoveto{\pgfqpoint{1.025455in}{2.376000in}}%
\pgfpathlineto{\pgfqpoint{1.116090in}{2.419564in}}%
\pgfpathlineto{\pgfqpoint{1.229383in}{2.471225in}}%
\pgfpathlineto{\pgfqpoint{1.478630in}{2.583962in}}%
\pgfpathlineto{\pgfqpoint{1.591923in}{2.637611in}}%
\pgfpathlineto{\pgfqpoint{1.705217in}{2.693473in}}%
\pgfpathlineto{\pgfqpoint{1.841169in}{2.763265in}}%
\pgfpathlineto{\pgfqpoint{2.022439in}{2.859510in}}%
\pgfpathlineto{\pgfqpoint{2.249027in}{2.979917in}}%
\pgfpathlineto{\pgfqpoint{2.362321in}{3.037671in}}%
\pgfpathlineto{\pgfqpoint{2.452956in}{3.081628in}}%
\pgfpathlineto{\pgfqpoint{2.543591in}{3.122941in}}%
\pgfpathlineto{\pgfqpoint{2.611567in}{3.151842in}}%
\pgfpathlineto{\pgfqpoint{2.679543in}{3.178698in}}%
\pgfpathlineto{\pgfqpoint{2.747519in}{3.203294in}}%
\pgfpathlineto{\pgfqpoint{2.815496in}{3.225427in}}%
\pgfpathlineto{\pgfqpoint{2.883472in}{3.244917in}}%
\pgfpathlineto{\pgfqpoint{2.951448in}{3.261602in}}%
\pgfpathlineto{\pgfqpoint{3.019424in}{3.275345in}}%
\pgfpathlineto{\pgfqpoint{3.087401in}{3.286031in}}%
\pgfpathlineto{\pgfqpoint{3.155377in}{3.293570in}}%
\pgfpathlineto{\pgfqpoint{3.223353in}{3.297901in}}%
\pgfpathlineto{\pgfqpoint{3.291329in}{3.298986in}}%
\pgfpathlineto{\pgfqpoint{3.359306in}{3.296817in}}%
\pgfpathlineto{\pgfqpoint{3.427282in}{3.291411in}}%
\pgfpathlineto{\pgfqpoint{3.495258in}{3.282815in}}%
\pgfpathlineto{\pgfqpoint{3.563234in}{3.271099in}}%
\pgfpathlineto{\pgfqpoint{3.631211in}{3.256361in}}%
\pgfpathlineto{\pgfqpoint{3.699187in}{3.238724in}}%
\pgfpathlineto{\pgfqpoint{3.767163in}{3.218334in}}%
\pgfpathlineto{\pgfqpoint{3.835139in}{3.195359in}}%
\pgfpathlineto{\pgfqpoint{3.903116in}{3.169987in}}%
\pgfpathlineto{\pgfqpoint{3.971092in}{3.142424in}}%
\pgfpathlineto{\pgfqpoint{4.039068in}{3.112893in}}%
\pgfpathlineto{\pgfqpoint{4.129703in}{3.070864in}}%
\pgfpathlineto{\pgfqpoint{4.220338in}{3.026338in}}%
\pgfpathlineto{\pgfqpoint{4.333632in}{2.968087in}}%
\pgfpathlineto{\pgfqpoint{4.514902in}{2.871654in}}%
\pgfpathlineto{\pgfqpoint{4.741489in}{2.751450in}}%
\pgfpathlineto{\pgfqpoint{4.877442in}{2.682122in}}%
\pgfpathlineto{\pgfqpoint{4.990735in}{2.626708in}}%
\pgfpathlineto{\pgfqpoint{5.104029in}{2.573476in}}%
\pgfpathlineto{\pgfqpoint{5.262640in}{2.501666in}}%
\pgfpathlineto{\pgfqpoint{5.443910in}{2.419564in}}%
\pgfpathlineto{\pgfqpoint{5.534545in}{2.376000in}}%
\pgfpathlineto{\pgfqpoint{5.534545in}{2.376000in}}%
\pgfusepath{stroke}%
\end{pgfscope}%
\begin{pgfscope}%
\pgfpathrectangle{\pgfqpoint{0.800000in}{0.528000in}}{\pgfqpoint{4.960000in}{3.696000in}}%
\pgfusepath{clip}%
\pgfsetbuttcap%
\pgfsetroundjoin%
\definecolor{currentfill}{rgb}{1.000000,0.498039,0.054902}%
\pgfsetfillcolor{currentfill}%
\pgfsetlinewidth{1.003750pt}%
\definecolor{currentstroke}{rgb}{1.000000,0.498039,0.054902}%
\pgfsetstrokecolor{currentstroke}%
\pgfsetdash{}{0pt}%
\pgfsys@defobject{currentmarker}{\pgfqpoint{-0.020833in}{-0.020833in}}{\pgfqpoint{0.020833in}{0.020833in}}{%
\pgfpathmoveto{\pgfqpoint{0.000000in}{-0.020833in}}%
\pgfpathcurveto{\pgfqpoint{0.005525in}{-0.020833in}}{\pgfqpoint{0.010825in}{-0.018638in}}{\pgfqpoint{0.014731in}{-0.014731in}}%
\pgfpathcurveto{\pgfqpoint{0.018638in}{-0.010825in}}{\pgfqpoint{0.020833in}{-0.005525in}}{\pgfqpoint{0.020833in}{0.000000in}}%
\pgfpathcurveto{\pgfqpoint{0.020833in}{0.005525in}}{\pgfqpoint{0.018638in}{0.010825in}}{\pgfqpoint{0.014731in}{0.014731in}}%
\pgfpathcurveto{\pgfqpoint{0.010825in}{0.018638in}}{\pgfqpoint{0.005525in}{0.020833in}}{\pgfqpoint{0.000000in}{0.020833in}}%
\pgfpathcurveto{\pgfqpoint{-0.005525in}{0.020833in}}{\pgfqpoint{-0.010825in}{0.018638in}}{\pgfqpoint{-0.014731in}{0.014731in}}%
\pgfpathcurveto{\pgfqpoint{-0.018638in}{0.010825in}}{\pgfqpoint{-0.020833in}{0.005525in}}{\pgfqpoint{-0.020833in}{0.000000in}}%
\pgfpathcurveto{\pgfqpoint{-0.020833in}{-0.005525in}}{\pgfqpoint{-0.018638in}{-0.010825in}}{\pgfqpoint{-0.014731in}{-0.014731in}}%
\pgfpathcurveto{\pgfqpoint{-0.010825in}{-0.018638in}}{\pgfqpoint{-0.005525in}{-0.020833in}}{\pgfqpoint{0.000000in}{-0.020833in}}%
\pgfpathclose%
\pgfusepath{stroke,fill}%
}%
\begin{pgfscope}%
\pgfsys@transformshift{1.025455in}{2.376000in}%
\pgfsys@useobject{currentmarker}{}%
\end{pgfscope}%
\begin{pgfscope}%
\pgfsys@transformshift{1.526465in}{2.603446in}%
\pgfsys@useobject{currentmarker}{}%
\end{pgfscope}%
\begin{pgfscope}%
\pgfsys@transformshift{2.027475in}{2.864151in}%
\pgfsys@useobject{currentmarker}{}%
\end{pgfscope}%
\begin{pgfscope}%
\pgfsys@transformshift{2.528485in}{3.115200in}%
\pgfsys@useobject{currentmarker}{}%
\end{pgfscope}%
\begin{pgfscope}%
\pgfsys@transformshift{3.029495in}{3.277463in}%
\pgfsys@useobject{currentmarker}{}%
\end{pgfscope}%
\begin{pgfscope}%
\pgfsys@transformshift{3.530505in}{3.277463in}%
\pgfsys@useobject{currentmarker}{}%
\end{pgfscope}%
\begin{pgfscope}%
\pgfsys@transformshift{4.031515in}{3.115200in}%
\pgfsys@useobject{currentmarker}{}%
\end{pgfscope}%
\begin{pgfscope}%
\pgfsys@transformshift{4.532525in}{2.864151in}%
\pgfsys@useobject{currentmarker}{}%
\end{pgfscope}%
\begin{pgfscope}%
\pgfsys@transformshift{5.033535in}{2.603446in}%
\pgfsys@useobject{currentmarker}{}%
\end{pgfscope}%
\end{pgfscope}%
\begin{pgfscope}%
\pgfpathrectangle{\pgfqpoint{0.800000in}{0.528000in}}{\pgfqpoint{4.960000in}{3.696000in}}%
\pgfusepath{clip}%
\pgfsetrectcap%
\pgfsetroundjoin%
\pgfsetlinewidth{1.505625pt}%
\definecolor{currentstroke}{rgb}{0.172549,0.627451,0.172549}%
\pgfsetstrokecolor{currentstroke}%
\pgfsetdash{}{0pt}%
\pgfpathmoveto{\pgfqpoint{1.025455in}{2.376000in}}%
\pgfpathlineto{\pgfqpoint{1.093431in}{2.402324in}}%
\pgfpathlineto{\pgfqpoint{1.161407in}{2.430787in}}%
\pgfpathlineto{\pgfqpoint{1.252042in}{2.471160in}}%
\pgfpathlineto{\pgfqpoint{1.365336in}{2.524343in}}%
\pgfpathlineto{\pgfqpoint{1.501288in}{2.590874in}}%
\pgfpathlineto{\pgfqpoint{1.659899in}{2.671164in}}%
\pgfpathlineto{\pgfqpoint{1.841169in}{2.765540in}}%
\pgfpathlineto{\pgfqpoint{2.294344in}{3.003281in}}%
\pgfpathlineto{\pgfqpoint{2.407638in}{3.059239in}}%
\pgfpathlineto{\pgfqpoint{2.498273in}{3.101642in}}%
\pgfpathlineto{\pgfqpoint{2.588908in}{3.141322in}}%
\pgfpathlineto{\pgfqpoint{2.656884in}{3.168946in}}%
\pgfpathlineto{\pgfqpoint{2.724861in}{3.194475in}}%
\pgfpathlineto{\pgfqpoint{2.792837in}{3.217680in}}%
\pgfpathlineto{\pgfqpoint{2.860813in}{3.238347in}}%
\pgfpathlineto{\pgfqpoint{2.928789in}{3.256281in}}%
\pgfpathlineto{\pgfqpoint{2.996766in}{3.271309in}}%
\pgfpathlineto{\pgfqpoint{3.064742in}{3.283283in}}%
\pgfpathlineto{\pgfqpoint{3.132718in}{3.292086in}}%
\pgfpathlineto{\pgfqpoint{3.200694in}{3.297628in}}%
\pgfpathlineto{\pgfqpoint{3.268671in}{3.299853in}}%
\pgfpathlineto{\pgfqpoint{3.313988in}{3.299482in}}%
\pgfpathlineto{\pgfqpoint{3.381964in}{3.296147in}}%
\pgfpathlineto{\pgfqpoint{3.449941in}{3.289510in}}%
\pgfpathlineto{\pgfqpoint{3.517917in}{3.279639in}}%
\pgfpathlineto{\pgfqpoint{3.585893in}{3.266632in}}%
\pgfpathlineto{\pgfqpoint{3.653869in}{3.250618in}}%
\pgfpathlineto{\pgfqpoint{3.721846in}{3.231752in}}%
\pgfpathlineto{\pgfqpoint{3.789822in}{3.210217in}}%
\pgfpathlineto{\pgfqpoint{3.857798in}{3.186213in}}%
\pgfpathlineto{\pgfqpoint{3.925774in}{3.159959in}}%
\pgfpathlineto{\pgfqpoint{3.993751in}{3.131689in}}%
\pgfpathlineto{\pgfqpoint{4.084386in}{3.091274in}}%
\pgfpathlineto{\pgfqpoint{4.175021in}{3.048281in}}%
\pgfpathlineto{\pgfqpoint{4.288314in}{2.991783in}}%
\pgfpathlineto{\pgfqpoint{4.424267in}{2.921325in}}%
\pgfpathlineto{\pgfqpoint{4.968077in}{2.636450in}}%
\pgfpathlineto{\pgfqpoint{5.126688in}{2.557300in}}%
\pgfpathlineto{\pgfqpoint{5.262640in}{2.492130in}}%
\pgfpathlineto{\pgfqpoint{5.353275in}{2.450680in}}%
\pgfpathlineto{\pgfqpoint{5.443910in}{2.411604in}}%
\pgfpathlineto{\pgfqpoint{5.511887in}{2.384497in}}%
\pgfpathlineto{\pgfqpoint{5.534545in}{2.376000in}}%
\pgfpathlineto{\pgfqpoint{5.534545in}{2.376000in}}%
\pgfusepath{stroke}%
\end{pgfscope}%
\begin{pgfscope}%
\pgfpathrectangle{\pgfqpoint{0.800000in}{0.528000in}}{\pgfqpoint{4.960000in}{3.696000in}}%
\pgfusepath{clip}%
\pgfsetbuttcap%
\pgfsetroundjoin%
\definecolor{currentfill}{rgb}{0.839216,0.152941,0.156863}%
\pgfsetfillcolor{currentfill}%
\pgfsetlinewidth{1.003750pt}%
\definecolor{currentstroke}{rgb}{0.839216,0.152941,0.156863}%
\pgfsetstrokecolor{currentstroke}%
\pgfsetdash{}{0pt}%
\pgfsys@defobject{currentmarker}{\pgfqpoint{-0.020833in}{-0.020833in}}{\pgfqpoint{0.020833in}{0.020833in}}{%
\pgfpathmoveto{\pgfqpoint{0.000000in}{-0.020833in}}%
\pgfpathcurveto{\pgfqpoint{0.005525in}{-0.020833in}}{\pgfqpoint{0.010825in}{-0.018638in}}{\pgfqpoint{0.014731in}{-0.014731in}}%
\pgfpathcurveto{\pgfqpoint{0.018638in}{-0.010825in}}{\pgfqpoint{0.020833in}{-0.005525in}}{\pgfqpoint{0.020833in}{0.000000in}}%
\pgfpathcurveto{\pgfqpoint{0.020833in}{0.005525in}}{\pgfqpoint{0.018638in}{0.010825in}}{\pgfqpoint{0.014731in}{0.014731in}}%
\pgfpathcurveto{\pgfqpoint{0.010825in}{0.018638in}}{\pgfqpoint{0.005525in}{0.020833in}}{\pgfqpoint{0.000000in}{0.020833in}}%
\pgfpathcurveto{\pgfqpoint{-0.005525in}{0.020833in}}{\pgfqpoint{-0.010825in}{0.018638in}}{\pgfqpoint{-0.014731in}{0.014731in}}%
\pgfpathcurveto{\pgfqpoint{-0.018638in}{0.010825in}}{\pgfqpoint{-0.020833in}{0.005525in}}{\pgfqpoint{-0.020833in}{0.000000in}}%
\pgfpathcurveto{\pgfqpoint{-0.020833in}{-0.005525in}}{\pgfqpoint{-0.018638in}{-0.010825in}}{\pgfqpoint{-0.014731in}{-0.014731in}}%
\pgfpathcurveto{\pgfqpoint{-0.010825in}{-0.018638in}}{\pgfqpoint{-0.005525in}{-0.020833in}}{\pgfqpoint{0.000000in}{-0.020833in}}%
\pgfpathclose%
\pgfusepath{stroke,fill}%
}%
\begin{pgfscope}%
\pgfsys@transformshift{1.025455in}{2.376000in}%
\pgfsys@useobject{currentmarker}{}%
\end{pgfscope}%
\begin{pgfscope}%
\pgfsys@transformshift{1.290695in}{2.491051in}%
\pgfsys@useobject{currentmarker}{}%
\end{pgfscope}%
\begin{pgfscope}%
\pgfsys@transformshift{1.555936in}{2.618096in}%
\pgfsys@useobject{currentmarker}{}%
\end{pgfscope}%
\begin{pgfscope}%
\pgfsys@transformshift{1.821176in}{2.754615in}%
\pgfsys@useobject{currentmarker}{}%
\end{pgfscope}%
\begin{pgfscope}%
\pgfsys@transformshift{2.086417in}{2.895438in}%
\pgfsys@useobject{currentmarker}{}%
\end{pgfscope}%
\begin{pgfscope}%
\pgfsys@transformshift{2.351658in}{3.032095in}%
\pgfsys@useobject{currentmarker}{}%
\end{pgfscope}%
\begin{pgfscope}%
\pgfsys@transformshift{2.616898in}{3.152866in}%
\pgfsys@useobject{currentmarker}{}%
\end{pgfscope}%
\begin{pgfscope}%
\pgfsys@transformshift{2.882139in}{3.244188in}%
\pgfsys@useobject{currentmarker}{}%
\end{pgfscope}%
\begin{pgfscope}%
\pgfsys@transformshift{3.147380in}{3.293628in}%
\pgfsys@useobject{currentmarker}{}%
\end{pgfscope}%
\begin{pgfscope}%
\pgfsys@transformshift{3.412620in}{3.293628in}%
\pgfsys@useobject{currentmarker}{}%
\end{pgfscope}%
\begin{pgfscope}%
\pgfsys@transformshift{3.677861in}{3.244188in}%
\pgfsys@useobject{currentmarker}{}%
\end{pgfscope}%
\begin{pgfscope}%
\pgfsys@transformshift{3.943102in}{3.152866in}%
\pgfsys@useobject{currentmarker}{}%
\end{pgfscope}%
\begin{pgfscope}%
\pgfsys@transformshift{4.208342in}{3.032095in}%
\pgfsys@useobject{currentmarker}{}%
\end{pgfscope}%
\begin{pgfscope}%
\pgfsys@transformshift{4.473583in}{2.895438in}%
\pgfsys@useobject{currentmarker}{}%
\end{pgfscope}%
\begin{pgfscope}%
\pgfsys@transformshift{4.738824in}{2.754615in}%
\pgfsys@useobject{currentmarker}{}%
\end{pgfscope}%
\begin{pgfscope}%
\pgfsys@transformshift{5.004064in}{2.618096in}%
\pgfsys@useobject{currentmarker}{}%
\end{pgfscope}%
\begin{pgfscope}%
\pgfsys@transformshift{5.269305in}{2.491051in}%
\pgfsys@useobject{currentmarker}{}%
\end{pgfscope}%
\end{pgfscope}%
\begin{pgfscope}%
\pgfpathrectangle{\pgfqpoint{0.800000in}{0.528000in}}{\pgfqpoint{4.960000in}{3.696000in}}%
\pgfusepath{clip}%
\pgfsetrectcap%
\pgfsetroundjoin%
\pgfsetlinewidth{1.505625pt}%
\definecolor{currentstroke}{rgb}{0.580392,0.403922,0.741176}%
\pgfsetstrokecolor{currentstroke}%
\pgfsetdash{}{0pt}%
\pgfpathmoveto{\pgfqpoint{1.025455in}{2.376000in}}%
\pgfpathlineto{\pgfqpoint{1.138748in}{2.423558in}}%
\pgfpathlineto{\pgfqpoint{1.252042in}{2.473499in}}%
\pgfpathlineto{\pgfqpoint{1.365336in}{2.525662in}}%
\pgfpathlineto{\pgfqpoint{1.501288in}{2.591031in}}%
\pgfpathlineto{\pgfqpoint{1.637241in}{2.659116in}}%
\pgfpathlineto{\pgfqpoint{1.795852in}{2.741289in}}%
\pgfpathlineto{\pgfqpoint{2.067757in}{2.885548in}}%
\pgfpathlineto{\pgfqpoint{2.249027in}{2.980396in}}%
\pgfpathlineto{\pgfqpoint{2.362321in}{3.037344in}}%
\pgfpathlineto{\pgfqpoint{2.452956in}{3.080815in}}%
\pgfpathlineto{\pgfqpoint{2.543591in}{3.121845in}}%
\pgfpathlineto{\pgfqpoint{2.611567in}{3.150683in}}%
\pgfpathlineto{\pgfqpoint{2.679543in}{3.177599in}}%
\pgfpathlineto{\pgfqpoint{2.747519in}{3.202362in}}%
\pgfpathlineto{\pgfqpoint{2.815496in}{3.224750in}}%
\pgfpathlineto{\pgfqpoint{2.883472in}{3.244550in}}%
\pgfpathlineto{\pgfqpoint{2.951448in}{3.261570in}}%
\pgfpathlineto{\pgfqpoint{3.019424in}{3.275639in}}%
\pgfpathlineto{\pgfqpoint{3.087401in}{3.286611in}}%
\pgfpathlineto{\pgfqpoint{3.155377in}{3.294371in}}%
\pgfpathlineto{\pgfqpoint{3.223353in}{3.298834in}}%
\pgfpathlineto{\pgfqpoint{3.268671in}{3.299953in}}%
\pgfpathlineto{\pgfqpoint{3.313988in}{3.299580in}}%
\pgfpathlineto{\pgfqpoint{3.359306in}{3.297716in}}%
\pgfpathlineto{\pgfqpoint{3.427282in}{3.292147in}}%
\pgfpathlineto{\pgfqpoint{3.495258in}{3.283306in}}%
\pgfpathlineto{\pgfqpoint{3.563234in}{3.271287in}}%
\pgfpathlineto{\pgfqpoint{3.631211in}{3.256217in}}%
\pgfpathlineto{\pgfqpoint{3.699187in}{3.238250in}}%
\pgfpathlineto{\pgfqpoint{3.767163in}{3.217564in}}%
\pgfpathlineto{\pgfqpoint{3.835139in}{3.194361in}}%
\pgfpathlineto{\pgfqpoint{3.903116in}{3.168855in}}%
\pgfpathlineto{\pgfqpoint{3.971092in}{3.141272in}}%
\pgfpathlineto{\pgfqpoint{4.061727in}{3.101670in}}%
\pgfpathlineto{\pgfqpoint{4.152362in}{3.059350in}}%
\pgfpathlineto{\pgfqpoint{4.265656in}{3.003466in}}%
\pgfpathlineto{\pgfqpoint{4.401608in}{2.933370in}}%
\pgfpathlineto{\pgfqpoint{4.628196in}{2.813234in}}%
\pgfpathlineto{\pgfqpoint{4.854783in}{2.694025in}}%
\pgfpathlineto{\pgfqpoint{4.990735in}{2.624761in}}%
\pgfpathlineto{\pgfqpoint{5.126688in}{2.557982in}}%
\pgfpathlineto{\pgfqpoint{5.239982in}{2.504536in}}%
\pgfpathlineto{\pgfqpoint{5.353275in}{2.453251in}}%
\pgfpathlineto{\pgfqpoint{5.466569in}{2.404232in}}%
\pgfpathlineto{\pgfqpoint{5.534545in}{2.376000in}}%
\pgfpathlineto{\pgfqpoint{5.534545in}{2.376000in}}%
\pgfusepath{stroke}%
\end{pgfscope}%
\begin{pgfscope}%
\pgfpathrectangle{\pgfqpoint{0.800000in}{0.528000in}}{\pgfqpoint{4.960000in}{3.696000in}}%
\pgfusepath{clip}%
\pgfsetbuttcap%
\pgfsetroundjoin%
\definecolor{currentfill}{rgb}{0.549020,0.337255,0.294118}%
\pgfsetfillcolor{currentfill}%
\pgfsetlinewidth{1.003750pt}%
\definecolor{currentstroke}{rgb}{0.549020,0.337255,0.294118}%
\pgfsetstrokecolor{currentstroke}%
\pgfsetdash{}{0pt}%
\pgfsys@defobject{currentmarker}{\pgfqpoint{-0.020833in}{-0.020833in}}{\pgfqpoint{0.020833in}{0.020833in}}{%
\pgfpathmoveto{\pgfqpoint{0.000000in}{-0.020833in}}%
\pgfpathcurveto{\pgfqpoint{0.005525in}{-0.020833in}}{\pgfqpoint{0.010825in}{-0.018638in}}{\pgfqpoint{0.014731in}{-0.014731in}}%
\pgfpathcurveto{\pgfqpoint{0.018638in}{-0.010825in}}{\pgfqpoint{0.020833in}{-0.005525in}}{\pgfqpoint{0.020833in}{0.000000in}}%
\pgfpathcurveto{\pgfqpoint{0.020833in}{0.005525in}}{\pgfqpoint{0.018638in}{0.010825in}}{\pgfqpoint{0.014731in}{0.014731in}}%
\pgfpathcurveto{\pgfqpoint{0.010825in}{0.018638in}}{\pgfqpoint{0.005525in}{0.020833in}}{\pgfqpoint{0.000000in}{0.020833in}}%
\pgfpathcurveto{\pgfqpoint{-0.005525in}{0.020833in}}{\pgfqpoint{-0.010825in}{0.018638in}}{\pgfqpoint{-0.014731in}{0.014731in}}%
\pgfpathcurveto{\pgfqpoint{-0.018638in}{0.010825in}}{\pgfqpoint{-0.020833in}{0.005525in}}{\pgfqpoint{-0.020833in}{0.000000in}}%
\pgfpathcurveto{\pgfqpoint{-0.020833in}{-0.005525in}}{\pgfqpoint{-0.018638in}{-0.010825in}}{\pgfqpoint{-0.014731in}{-0.014731in}}%
\pgfpathcurveto{\pgfqpoint{-0.010825in}{-0.018638in}}{\pgfqpoint{-0.005525in}{-0.020833in}}{\pgfqpoint{0.000000in}{-0.020833in}}%
\pgfpathclose%
\pgfusepath{stroke,fill}%
}%
\begin{pgfscope}%
\pgfsys@transformshift{1.025455in}{2.376000in}%
\pgfsys@useobject{currentmarker}{}%
\end{pgfscope}%
\begin{pgfscope}%
\pgfsys@transformshift{1.162094in}{2.433694in}%
\pgfsys@useobject{currentmarker}{}%
\end{pgfscope}%
\begin{pgfscope}%
\pgfsys@transformshift{1.298733in}{2.494732in}%
\pgfsys@useobject{currentmarker}{}%
\end{pgfscope}%
\begin{pgfscope}%
\pgfsys@transformshift{1.435372in}{2.558970in}%
\pgfsys@useobject{currentmarker}{}%
\end{pgfscope}%
\begin{pgfscope}%
\pgfsys@transformshift{1.572011in}{2.626138in}%
\pgfsys@useobject{currentmarker}{}%
\end{pgfscope}%
\begin{pgfscope}%
\pgfsys@transformshift{1.708650in}{2.695802in}%
\pgfsys@useobject{currentmarker}{}%
\end{pgfscope}%
\begin{pgfscope}%
\pgfsys@transformshift{1.845289in}{2.767341in}%
\pgfsys@useobject{currentmarker}{}%
\end{pgfscope}%
\begin{pgfscope}%
\pgfsys@transformshift{1.981928in}{2.839912in}%
\pgfsys@useobject{currentmarker}{}%
\end{pgfscope}%
\begin{pgfscope}%
\pgfsys@transformshift{2.118567in}{2.912430in}%
\pgfsys@useobject{currentmarker}{}%
\end{pgfscope}%
\begin{pgfscope}%
\pgfsys@transformshift{2.255207in}{2.983562in}%
\pgfsys@useobject{currentmarker}{}%
\end{pgfscope}%
\begin{pgfscope}%
\pgfsys@transformshift{2.391846in}{3.051739in}%
\pgfsys@useobject{currentmarker}{}%
\end{pgfscope}%
\begin{pgfscope}%
\pgfsys@transformshift{2.528485in}{3.115200in}%
\pgfsys@useobject{currentmarker}{}%
\end{pgfscope}%
\begin{pgfscope}%
\pgfsys@transformshift{2.665124in}{3.172062in}%
\pgfsys@useobject{currentmarker}{}%
\end{pgfscope}%
\begin{pgfscope}%
\pgfsys@transformshift{2.801763in}{3.220429in}%
\pgfsys@useobject{currentmarker}{}%
\end{pgfscope}%
\begin{pgfscope}%
\pgfsys@transformshift{2.938402in}{3.258528in}%
\pgfsys@useobject{currentmarker}{}%
\end{pgfscope}%
\begin{pgfscope}%
\pgfsys@transformshift{3.075041in}{3.284852in}%
\pgfsys@useobject{currentmarker}{}%
\end{pgfscope}%
\begin{pgfscope}%
\pgfsys@transformshift{3.211680in}{3.298305in}%
\pgfsys@useobject{currentmarker}{}%
\end{pgfscope}%
\begin{pgfscope}%
\pgfsys@transformshift{3.348320in}{3.298305in}%
\pgfsys@useobject{currentmarker}{}%
\end{pgfscope}%
\begin{pgfscope}%
\pgfsys@transformshift{3.484959in}{3.284852in}%
\pgfsys@useobject{currentmarker}{}%
\end{pgfscope}%
\begin{pgfscope}%
\pgfsys@transformshift{3.621598in}{3.258528in}%
\pgfsys@useobject{currentmarker}{}%
\end{pgfscope}%
\begin{pgfscope}%
\pgfsys@transformshift{3.758237in}{3.220429in}%
\pgfsys@useobject{currentmarker}{}%
\end{pgfscope}%
\begin{pgfscope}%
\pgfsys@transformshift{3.894876in}{3.172062in}%
\pgfsys@useobject{currentmarker}{}%
\end{pgfscope}%
\begin{pgfscope}%
\pgfsys@transformshift{4.031515in}{3.115200in}%
\pgfsys@useobject{currentmarker}{}%
\end{pgfscope}%
\begin{pgfscope}%
\pgfsys@transformshift{4.168154in}{3.051739in}%
\pgfsys@useobject{currentmarker}{}%
\end{pgfscope}%
\begin{pgfscope}%
\pgfsys@transformshift{4.304793in}{2.983562in}%
\pgfsys@useobject{currentmarker}{}%
\end{pgfscope}%
\begin{pgfscope}%
\pgfsys@transformshift{4.441433in}{2.912430in}%
\pgfsys@useobject{currentmarker}{}%
\end{pgfscope}%
\begin{pgfscope}%
\pgfsys@transformshift{4.578072in}{2.839912in}%
\pgfsys@useobject{currentmarker}{}%
\end{pgfscope}%
\begin{pgfscope}%
\pgfsys@transformshift{4.714711in}{2.767341in}%
\pgfsys@useobject{currentmarker}{}%
\end{pgfscope}%
\begin{pgfscope}%
\pgfsys@transformshift{4.851350in}{2.695802in}%
\pgfsys@useobject{currentmarker}{}%
\end{pgfscope}%
\begin{pgfscope}%
\pgfsys@transformshift{4.987989in}{2.626138in}%
\pgfsys@useobject{currentmarker}{}%
\end{pgfscope}%
\begin{pgfscope}%
\pgfsys@transformshift{5.124628in}{2.558970in}%
\pgfsys@useobject{currentmarker}{}%
\end{pgfscope}%
\begin{pgfscope}%
\pgfsys@transformshift{5.261267in}{2.494732in}%
\pgfsys@useobject{currentmarker}{}%
\end{pgfscope}%
\begin{pgfscope}%
\pgfsys@transformshift{5.397906in}{2.433694in}%
\pgfsys@useobject{currentmarker}{}%
\end{pgfscope}%
\end{pgfscope}%
\begin{pgfscope}%
\pgfpathrectangle{\pgfqpoint{0.800000in}{0.528000in}}{\pgfqpoint{4.960000in}{3.696000in}}%
\pgfusepath{clip}%
\pgfsetrectcap%
\pgfsetroundjoin%
\pgfsetlinewidth{1.505625pt}%
\definecolor{currentstroke}{rgb}{0.890196,0.466667,0.760784}%
\pgfsetstrokecolor{currentstroke}%
\pgfsetdash{}{0pt}%
\pgfpathmoveto{\pgfqpoint{1.025455in}{2.376000in}}%
\pgfpathlineto{\pgfqpoint{1.138748in}{2.423597in}}%
\pgfpathlineto{\pgfqpoint{1.252042in}{2.473505in}}%
\pgfpathlineto{\pgfqpoint{1.365336in}{2.525658in}}%
\pgfpathlineto{\pgfqpoint{1.478630in}{2.579935in}}%
\pgfpathlineto{\pgfqpoint{1.614582in}{2.647600in}}%
\pgfpathlineto{\pgfqpoint{1.773193in}{2.729407in}}%
\pgfpathlineto{\pgfqpoint{1.999781in}{2.849415in}}%
\pgfpathlineto{\pgfqpoint{2.226368in}{2.968739in}}%
\pgfpathlineto{\pgfqpoint{2.339662in}{3.026159in}}%
\pgfpathlineto{\pgfqpoint{2.430297in}{3.070154in}}%
\pgfpathlineto{\pgfqpoint{2.520932in}{3.111847in}}%
\pgfpathlineto{\pgfqpoint{2.611567in}{3.150683in}}%
\pgfpathlineto{\pgfqpoint{2.679543in}{3.177599in}}%
\pgfpathlineto{\pgfqpoint{2.747519in}{3.202362in}}%
\pgfpathlineto{\pgfqpoint{2.815496in}{3.224750in}}%
\pgfpathlineto{\pgfqpoint{2.883472in}{3.244550in}}%
\pgfpathlineto{\pgfqpoint{2.951448in}{3.261570in}}%
\pgfpathlineto{\pgfqpoint{3.019424in}{3.275639in}}%
\pgfpathlineto{\pgfqpoint{3.087401in}{3.286611in}}%
\pgfpathlineto{\pgfqpoint{3.155377in}{3.294371in}}%
\pgfpathlineto{\pgfqpoint{3.223353in}{3.298834in}}%
\pgfpathlineto{\pgfqpoint{3.268671in}{3.299953in}}%
\pgfpathlineto{\pgfqpoint{3.313988in}{3.299580in}}%
\pgfpathlineto{\pgfqpoint{3.359306in}{3.297716in}}%
\pgfpathlineto{\pgfqpoint{3.427282in}{3.292147in}}%
\pgfpathlineto{\pgfqpoint{3.495258in}{3.283306in}}%
\pgfpathlineto{\pgfqpoint{3.563234in}{3.271287in}}%
\pgfpathlineto{\pgfqpoint{3.631211in}{3.256217in}}%
\pgfpathlineto{\pgfqpoint{3.699187in}{3.238250in}}%
\pgfpathlineto{\pgfqpoint{3.767163in}{3.217564in}}%
\pgfpathlineto{\pgfqpoint{3.835139in}{3.194361in}}%
\pgfpathlineto{\pgfqpoint{3.903116in}{3.168855in}}%
\pgfpathlineto{\pgfqpoint{3.971092in}{3.141272in}}%
\pgfpathlineto{\pgfqpoint{4.061727in}{3.101670in}}%
\pgfpathlineto{\pgfqpoint{4.152362in}{3.059350in}}%
\pgfpathlineto{\pgfqpoint{4.265656in}{3.003466in}}%
\pgfpathlineto{\pgfqpoint{4.401608in}{2.933370in}}%
\pgfpathlineto{\pgfqpoint{4.628196in}{2.813234in}}%
\pgfpathlineto{\pgfqpoint{4.854783in}{2.694026in}}%
\pgfpathlineto{\pgfqpoint{4.990735in}{2.624761in}}%
\pgfpathlineto{\pgfqpoint{5.126688in}{2.557979in}}%
\pgfpathlineto{\pgfqpoint{5.239982in}{2.504533in}}%
\pgfpathlineto{\pgfqpoint{5.353275in}{2.453268in}}%
\pgfpathlineto{\pgfqpoint{5.466569in}{2.404279in}}%
\pgfpathlineto{\pgfqpoint{5.534545in}{2.376003in}}%
\pgfpathlineto{\pgfqpoint{5.534545in}{2.376003in}}%
\pgfusepath{stroke}%
\end{pgfscope}%
\begin{pgfscope}%
\pgfpathrectangle{\pgfqpoint{0.800000in}{0.528000in}}{\pgfqpoint{4.960000in}{3.696000in}}%
\pgfusepath{clip}%
\pgfsetrectcap%
\pgfsetroundjoin%
\pgfsetlinewidth{1.505625pt}%
\definecolor{currentstroke}{rgb}{0.498039,0.498039,0.498039}%
\pgfsetstrokecolor{currentstroke}%
\pgfsetdash{}{0pt}%
\pgfpathmoveto{\pgfqpoint{1.025455in}{2.376000in}}%
\pgfpathlineto{\pgfqpoint{1.138748in}{2.423597in}}%
\pgfpathlineto{\pgfqpoint{1.252042in}{2.473505in}}%
\pgfpathlineto{\pgfqpoint{1.365336in}{2.525658in}}%
\pgfpathlineto{\pgfqpoint{1.478630in}{2.579935in}}%
\pgfpathlineto{\pgfqpoint{1.614582in}{2.647600in}}%
\pgfpathlineto{\pgfqpoint{1.773193in}{2.729407in}}%
\pgfpathlineto{\pgfqpoint{1.999781in}{2.849415in}}%
\pgfpathlineto{\pgfqpoint{2.226368in}{2.968739in}}%
\pgfpathlineto{\pgfqpoint{2.339662in}{3.026159in}}%
\pgfpathlineto{\pgfqpoint{2.430297in}{3.070154in}}%
\pgfpathlineto{\pgfqpoint{2.520932in}{3.111847in}}%
\pgfpathlineto{\pgfqpoint{2.611567in}{3.150683in}}%
\pgfpathlineto{\pgfqpoint{2.679543in}{3.177599in}}%
\pgfpathlineto{\pgfqpoint{2.747519in}{3.202362in}}%
\pgfpathlineto{\pgfqpoint{2.815496in}{3.224750in}}%
\pgfpathlineto{\pgfqpoint{2.883472in}{3.244550in}}%
\pgfpathlineto{\pgfqpoint{2.951448in}{3.261570in}}%
\pgfpathlineto{\pgfqpoint{3.019424in}{3.275639in}}%
\pgfpathlineto{\pgfqpoint{3.087401in}{3.286611in}}%
\pgfpathlineto{\pgfqpoint{3.155377in}{3.294371in}}%
\pgfpathlineto{\pgfqpoint{3.223353in}{3.298834in}}%
\pgfpathlineto{\pgfqpoint{3.268671in}{3.299953in}}%
\pgfpathlineto{\pgfqpoint{3.313988in}{3.299580in}}%
\pgfpathlineto{\pgfqpoint{3.359306in}{3.297716in}}%
\pgfpathlineto{\pgfqpoint{3.427282in}{3.292147in}}%
\pgfpathlineto{\pgfqpoint{3.495258in}{3.283306in}}%
\pgfpathlineto{\pgfqpoint{3.563234in}{3.271287in}}%
\pgfpathlineto{\pgfqpoint{3.631211in}{3.256217in}}%
\pgfpathlineto{\pgfqpoint{3.699187in}{3.238250in}}%
\pgfpathlineto{\pgfqpoint{3.767163in}{3.217564in}}%
\pgfpathlineto{\pgfqpoint{3.835139in}{3.194361in}}%
\pgfpathlineto{\pgfqpoint{3.903116in}{3.168855in}}%
\pgfpathlineto{\pgfqpoint{3.971092in}{3.141272in}}%
\pgfpathlineto{\pgfqpoint{4.061727in}{3.101670in}}%
\pgfpathlineto{\pgfqpoint{4.152362in}{3.059350in}}%
\pgfpathlineto{\pgfqpoint{4.265656in}{3.003466in}}%
\pgfpathlineto{\pgfqpoint{4.401608in}{2.933370in}}%
\pgfpathlineto{\pgfqpoint{4.628196in}{2.813234in}}%
\pgfpathlineto{\pgfqpoint{4.854783in}{2.694026in}}%
\pgfpathlineto{\pgfqpoint{4.990735in}{2.624761in}}%
\pgfpathlineto{\pgfqpoint{5.126688in}{2.557979in}}%
\pgfpathlineto{\pgfqpoint{5.239982in}{2.504533in}}%
\pgfpathlineto{\pgfqpoint{5.353275in}{2.453268in}}%
\pgfpathlineto{\pgfqpoint{5.466569in}{2.404279in}}%
\pgfpathlineto{\pgfqpoint{5.534545in}{2.376000in}}%
\pgfpathlineto{\pgfqpoint{5.534545in}{2.376000in}}%
\pgfusepath{stroke}%
\end{pgfscope}%
\begin{pgfscope}%
\pgfpathrectangle{\pgfqpoint{0.800000in}{0.528000in}}{\pgfqpoint{4.960000in}{3.696000in}}%
\pgfusepath{clip}%
\pgfsetbuttcap%
\pgfsetroundjoin%
\definecolor{currentfill}{rgb}{0.737255,0.741176,0.133333}%
\pgfsetfillcolor{currentfill}%
\pgfsetlinewidth{1.003750pt}%
\definecolor{currentstroke}{rgb}{0.737255,0.741176,0.133333}%
\pgfsetstrokecolor{currentstroke}%
\pgfsetdash{}{0pt}%
\pgfsys@defobject{currentmarker}{\pgfqpoint{-0.020833in}{-0.020833in}}{\pgfqpoint{0.020833in}{0.020833in}}{%
\pgfpathmoveto{\pgfqpoint{0.000000in}{-0.020833in}}%
\pgfpathcurveto{\pgfqpoint{0.005525in}{-0.020833in}}{\pgfqpoint{0.010825in}{-0.018638in}}{\pgfqpoint{0.014731in}{-0.014731in}}%
\pgfpathcurveto{\pgfqpoint{0.018638in}{-0.010825in}}{\pgfqpoint{0.020833in}{-0.005525in}}{\pgfqpoint{0.020833in}{0.000000in}}%
\pgfpathcurveto{\pgfqpoint{0.020833in}{0.005525in}}{\pgfqpoint{0.018638in}{0.010825in}}{\pgfqpoint{0.014731in}{0.014731in}}%
\pgfpathcurveto{\pgfqpoint{0.010825in}{0.018638in}}{\pgfqpoint{0.005525in}{0.020833in}}{\pgfqpoint{0.000000in}{0.020833in}}%
\pgfpathcurveto{\pgfqpoint{-0.005525in}{0.020833in}}{\pgfqpoint{-0.010825in}{0.018638in}}{\pgfqpoint{-0.014731in}{0.014731in}}%
\pgfpathcurveto{\pgfqpoint{-0.018638in}{0.010825in}}{\pgfqpoint{-0.020833in}{0.005525in}}{\pgfqpoint{-0.020833in}{0.000000in}}%
\pgfpathcurveto{\pgfqpoint{-0.020833in}{-0.005525in}}{\pgfqpoint{-0.018638in}{-0.010825in}}{\pgfqpoint{-0.014731in}{-0.014731in}}%
\pgfpathcurveto{\pgfqpoint{-0.010825in}{-0.018638in}}{\pgfqpoint{-0.005525in}{-0.020833in}}{\pgfqpoint{0.000000in}{-0.020833in}}%
\pgfpathclose%
\pgfusepath{stroke,fill}%
}%
\begin{pgfscope}%
\pgfsys@transformshift{5.478019in}{2.399457in}%
\pgfsys@useobject{currentmarker}{}%
\end{pgfscope}%
\begin{pgfscope}%
\pgfsys@transformshift{5.042675in}{2.598928in}%
\pgfsys@useobject{currentmarker}{}%
\end{pgfscope}%
\begin{pgfscope}%
\pgfsys@transformshift{4.258211in}{3.007222in}%
\pgfsys@useobject{currentmarker}{}%
\end{pgfscope}%
\begin{pgfscope}%
\pgfsys@transformshift{3.280000in}{3.300000in}%
\pgfsys@useobject{currentmarker}{}%
\end{pgfscope}%
\begin{pgfscope}%
\pgfsys@transformshift{2.301789in}{3.007222in}%
\pgfsys@useobject{currentmarker}{}%
\end{pgfscope}%
\begin{pgfscope}%
\pgfsys@transformshift{1.517325in}{2.598928in}%
\pgfsys@useobject{currentmarker}{}%
\end{pgfscope}%
\begin{pgfscope}%
\pgfsys@transformshift{1.081981in}{2.399457in}%
\pgfsys@useobject{currentmarker}{}%
\end{pgfscope}%
\end{pgfscope}%
\begin{pgfscope}%
\pgfpathrectangle{\pgfqpoint{0.800000in}{0.528000in}}{\pgfqpoint{4.960000in}{3.696000in}}%
\pgfusepath{clip}%
\pgfsetrectcap%
\pgfsetroundjoin%
\pgfsetlinewidth{1.505625pt}%
\definecolor{currentstroke}{rgb}{0.090196,0.745098,0.811765}%
\pgfsetstrokecolor{currentstroke}%
\pgfsetdash{}{0pt}%
\pgfpathmoveto{\pgfqpoint{1.025455in}{2.372134in}}%
\pgfpathlineto{\pgfqpoint{1.116090in}{2.415470in}}%
\pgfpathlineto{\pgfqpoint{1.252042in}{2.477348in}}%
\pgfpathlineto{\pgfqpoint{1.455971in}{2.570087in}}%
\pgfpathlineto{\pgfqpoint{1.569265in}{2.623829in}}%
\pgfpathlineto{\pgfqpoint{1.682558in}{2.679790in}}%
\pgfpathlineto{\pgfqpoint{1.818511in}{2.749769in}}%
\pgfpathlineto{\pgfqpoint{1.977122in}{2.834267in}}%
\pgfpathlineto{\pgfqpoint{2.249027in}{2.979864in}}%
\pgfpathlineto{\pgfqpoint{2.362321in}{3.037899in}}%
\pgfpathlineto{\pgfqpoint{2.452956in}{3.082040in}}%
\pgfpathlineto{\pgfqpoint{2.543591in}{3.123502in}}%
\pgfpathlineto{\pgfqpoint{2.611567in}{3.152494in}}%
\pgfpathlineto{\pgfqpoint{2.679543in}{3.179426in}}%
\pgfpathlineto{\pgfqpoint{2.747519in}{3.204084in}}%
\pgfpathlineto{\pgfqpoint{2.815496in}{3.226267in}}%
\pgfpathlineto{\pgfqpoint{2.883472in}{3.245797in}}%
\pgfpathlineto{\pgfqpoint{2.951448in}{3.262513in}}%
\pgfpathlineto{\pgfqpoint{3.019424in}{3.276279in}}%
\pgfpathlineto{\pgfqpoint{3.087401in}{3.286981in}}%
\pgfpathlineto{\pgfqpoint{3.155377in}{3.294532in}}%
\pgfpathlineto{\pgfqpoint{3.223353in}{3.298868in}}%
\pgfpathlineto{\pgfqpoint{3.291329in}{3.299955in}}%
\pgfpathlineto{\pgfqpoint{3.359306in}{3.297782in}}%
\pgfpathlineto{\pgfqpoint{3.427282in}{3.292369in}}%
\pgfpathlineto{\pgfqpoint{3.495258in}{3.283760in}}%
\pgfpathlineto{\pgfqpoint{3.563234in}{3.272025in}}%
\pgfpathlineto{\pgfqpoint{3.631211in}{3.257262in}}%
\pgfpathlineto{\pgfqpoint{3.699187in}{3.239592in}}%
\pgfpathlineto{\pgfqpoint{3.767163in}{3.219159in}}%
\pgfpathlineto{\pgfqpoint{3.835139in}{3.196130in}}%
\pgfpathlineto{\pgfqpoint{3.903116in}{3.170691in}}%
\pgfpathlineto{\pgfqpoint{3.971092in}{3.143048in}}%
\pgfpathlineto{\pgfqpoint{4.039068in}{3.113420in}}%
\pgfpathlineto{\pgfqpoint{4.129703in}{3.071233in}}%
\pgfpathlineto{\pgfqpoint{4.220338in}{3.026515in}}%
\pgfpathlineto{\pgfqpoint{4.333632in}{2.967970in}}%
\pgfpathlineto{\pgfqpoint{4.492243in}{2.883164in}}%
\pgfpathlineto{\pgfqpoint{4.741489in}{2.749769in}}%
\pgfpathlineto{\pgfqpoint{4.877442in}{2.679790in}}%
\pgfpathlineto{\pgfqpoint{4.990735in}{2.623829in}}%
\pgfpathlineto{\pgfqpoint{5.104029in}{2.570087in}}%
\pgfpathlineto{\pgfqpoint{5.262640in}{2.497727in}}%
\pgfpathlineto{\pgfqpoint{5.466569in}{2.404866in}}%
\pgfpathlineto{\pgfqpoint{5.534545in}{2.372134in}}%
\pgfpathlineto{\pgfqpoint{5.534545in}{2.372134in}}%
\pgfusepath{stroke}%
\end{pgfscope}%
\begin{pgfscope}%
\pgfpathrectangle{\pgfqpoint{0.800000in}{0.528000in}}{\pgfqpoint{4.960000in}{3.696000in}}%
\pgfusepath{clip}%
\pgfsetbuttcap%
\pgfsetroundjoin%
\definecolor{currentfill}{rgb}{0.121569,0.466667,0.705882}%
\pgfsetfillcolor{currentfill}%
\pgfsetlinewidth{1.003750pt}%
\definecolor{currentstroke}{rgb}{0.121569,0.466667,0.705882}%
\pgfsetstrokecolor{currentstroke}%
\pgfsetdash{}{0pt}%
\pgfsys@defobject{currentmarker}{\pgfqpoint{-0.020833in}{-0.020833in}}{\pgfqpoint{0.020833in}{0.020833in}}{%
\pgfpathmoveto{\pgfqpoint{0.000000in}{-0.020833in}}%
\pgfpathcurveto{\pgfqpoint{0.005525in}{-0.020833in}}{\pgfqpoint{0.010825in}{-0.018638in}}{\pgfqpoint{0.014731in}{-0.014731in}}%
\pgfpathcurveto{\pgfqpoint{0.018638in}{-0.010825in}}{\pgfqpoint{0.020833in}{-0.005525in}}{\pgfqpoint{0.020833in}{0.000000in}}%
\pgfpathcurveto{\pgfqpoint{0.020833in}{0.005525in}}{\pgfqpoint{0.018638in}{0.010825in}}{\pgfqpoint{0.014731in}{0.014731in}}%
\pgfpathcurveto{\pgfqpoint{0.010825in}{0.018638in}}{\pgfqpoint{0.005525in}{0.020833in}}{\pgfqpoint{0.000000in}{0.020833in}}%
\pgfpathcurveto{\pgfqpoint{-0.005525in}{0.020833in}}{\pgfqpoint{-0.010825in}{0.018638in}}{\pgfqpoint{-0.014731in}{0.014731in}}%
\pgfpathcurveto{\pgfqpoint{-0.018638in}{0.010825in}}{\pgfqpoint{-0.020833in}{0.005525in}}{\pgfqpoint{-0.020833in}{0.000000in}}%
\pgfpathcurveto{\pgfqpoint{-0.020833in}{-0.005525in}}{\pgfqpoint{-0.018638in}{-0.010825in}}{\pgfqpoint{-0.014731in}{-0.014731in}}%
\pgfpathcurveto{\pgfqpoint{-0.010825in}{-0.018638in}}{\pgfqpoint{-0.005525in}{-0.020833in}}{\pgfqpoint{0.000000in}{-0.020833in}}%
\pgfpathclose%
\pgfusepath{stroke,fill}%
}%
\begin{pgfscope}%
\pgfsys@transformshift{5.500294in}{2.390144in}%
\pgfsys@useobject{currentmarker}{}%
\end{pgfscope}%
\begin{pgfscope}%
\pgfsys@transformshift{5.232494in}{2.508000in}%
\pgfsys@useobject{currentmarker}{}%
\end{pgfscope}%
\begin{pgfscope}%
\pgfsys@transformshift{4.729194in}{2.759693in}%
\pgfsys@useobject{currentmarker}{}%
\end{pgfscope}%
\begin{pgfscope}%
\pgfsys@transformshift{4.051100in}{3.106464in}%
\pgfsys@useobject{currentmarker}{}%
\end{pgfscope}%
\begin{pgfscope}%
\pgfsys@transformshift{3.280000in}{3.300000in}%
\pgfsys@useobject{currentmarker}{}%
\end{pgfscope}%
\begin{pgfscope}%
\pgfsys@transformshift{2.508900in}{3.106464in}%
\pgfsys@useobject{currentmarker}{}%
\end{pgfscope}%
\begin{pgfscope}%
\pgfsys@transformshift{1.830806in}{2.759693in}%
\pgfsys@useobject{currentmarker}{}%
\end{pgfscope}%
\begin{pgfscope}%
\pgfsys@transformshift{1.327506in}{2.508000in}%
\pgfsys@useobject{currentmarker}{}%
\end{pgfscope}%
\begin{pgfscope}%
\pgfsys@transformshift{1.059706in}{2.390144in}%
\pgfsys@useobject{currentmarker}{}%
\end{pgfscope}%
\end{pgfscope}%
\begin{pgfscope}%
\pgfpathrectangle{\pgfqpoint{0.800000in}{0.528000in}}{\pgfqpoint{4.960000in}{3.696000in}}%
\pgfusepath{clip}%
\pgfsetrectcap%
\pgfsetroundjoin%
\pgfsetlinewidth{1.505625pt}%
\definecolor{currentstroke}{rgb}{1.000000,0.498039,0.054902}%
\pgfsetstrokecolor{currentstroke}%
\pgfsetdash{}{0pt}%
\pgfpathmoveto{\pgfqpoint{1.025455in}{2.376663in}}%
\pgfpathlineto{\pgfqpoint{1.093431in}{2.403881in}}%
\pgfpathlineto{\pgfqpoint{1.184066in}{2.442645in}}%
\pgfpathlineto{\pgfqpoint{1.297360in}{2.493941in}}%
\pgfpathlineto{\pgfqpoint{1.433312in}{2.558440in}}%
\pgfpathlineto{\pgfqpoint{1.569265in}{2.625397in}}%
\pgfpathlineto{\pgfqpoint{1.727876in}{2.706096in}}%
\pgfpathlineto{\pgfqpoint{1.909146in}{2.800958in}}%
\pgfpathlineto{\pgfqpoint{2.271686in}{2.991550in}}%
\pgfpathlineto{\pgfqpoint{2.384979in}{3.048179in}}%
\pgfpathlineto{\pgfqpoint{2.475614in}{3.091263in}}%
\pgfpathlineto{\pgfqpoint{2.566249in}{3.131748in}}%
\pgfpathlineto{\pgfqpoint{2.634226in}{3.160056in}}%
\pgfpathlineto{\pgfqpoint{2.702202in}{3.186335in}}%
\pgfpathlineto{\pgfqpoint{2.770178in}{3.210352in}}%
\pgfpathlineto{\pgfqpoint{2.838154in}{3.231892in}}%
\pgfpathlineto{\pgfqpoint{2.906131in}{3.250755in}}%
\pgfpathlineto{\pgfqpoint{2.974107in}{3.266762in}}%
\pgfpathlineto{\pgfqpoint{3.042083in}{3.279760in}}%
\pgfpathlineto{\pgfqpoint{3.110059in}{3.289622in}}%
\pgfpathlineto{\pgfqpoint{3.178036in}{3.296252in}}%
\pgfpathlineto{\pgfqpoint{3.246012in}{3.299583in}}%
\pgfpathlineto{\pgfqpoint{3.291329in}{3.299954in}}%
\pgfpathlineto{\pgfqpoint{3.359306in}{3.297731in}}%
\pgfpathlineto{\pgfqpoint{3.427282in}{3.292195in}}%
\pgfpathlineto{\pgfqpoint{3.495258in}{3.283401in}}%
\pgfpathlineto{\pgfqpoint{3.563234in}{3.271436in}}%
\pgfpathlineto{\pgfqpoint{3.631211in}{3.256416in}}%
\pgfpathlineto{\pgfqpoint{3.699187in}{3.238487in}}%
\pgfpathlineto{\pgfqpoint{3.767163in}{3.217818in}}%
\pgfpathlineto{\pgfqpoint{3.835139in}{3.194603in}}%
\pgfpathlineto{\pgfqpoint{3.903116in}{3.169053in}}%
\pgfpathlineto{\pgfqpoint{3.971092in}{3.141395in}}%
\pgfpathlineto{\pgfqpoint{4.061727in}{3.101651in}}%
\pgfpathlineto{\pgfqpoint{4.152362in}{3.059162in}}%
\pgfpathlineto{\pgfqpoint{4.265656in}{3.003075in}}%
\pgfpathlineto{\pgfqpoint{4.401608in}{2.932841in}}%
\pgfpathlineto{\pgfqpoint{4.945418in}{2.648190in}}%
\pgfpathlineto{\pgfqpoint{5.104029in}{2.569442in}}%
\pgfpathlineto{\pgfqpoint{5.239982in}{2.504494in}}%
\pgfpathlineto{\pgfqpoint{5.353275in}{2.452685in}}%
\pgfpathlineto{\pgfqpoint{5.443910in}{2.413338in}}%
\pgfpathlineto{\pgfqpoint{5.534545in}{2.376663in}}%
\pgfpathlineto{\pgfqpoint{5.534545in}{2.376663in}}%
\pgfusepath{stroke}%
\end{pgfscope}%
\begin{pgfscope}%
\pgfpathrectangle{\pgfqpoint{0.800000in}{0.528000in}}{\pgfqpoint{4.960000in}{3.696000in}}%
\pgfusepath{clip}%
\pgfsetbuttcap%
\pgfsetroundjoin%
\definecolor{currentfill}{rgb}{0.172549,0.627451,0.172549}%
\pgfsetfillcolor{currentfill}%
\pgfsetlinewidth{1.003750pt}%
\definecolor{currentstroke}{rgb}{0.172549,0.627451,0.172549}%
\pgfsetstrokecolor{currentstroke}%
\pgfsetdash{}{0pt}%
\pgfsys@defobject{currentmarker}{\pgfqpoint{-0.020833in}{-0.020833in}}{\pgfqpoint{0.020833in}{0.020833in}}{%
\pgfpathmoveto{\pgfqpoint{0.000000in}{-0.020833in}}%
\pgfpathcurveto{\pgfqpoint{0.005525in}{-0.020833in}}{\pgfqpoint{0.010825in}{-0.018638in}}{\pgfqpoint{0.014731in}{-0.014731in}}%
\pgfpathcurveto{\pgfqpoint{0.018638in}{-0.010825in}}{\pgfqpoint{0.020833in}{-0.005525in}}{\pgfqpoint{0.020833in}{0.000000in}}%
\pgfpathcurveto{\pgfqpoint{0.020833in}{0.005525in}}{\pgfqpoint{0.018638in}{0.010825in}}{\pgfqpoint{0.014731in}{0.014731in}}%
\pgfpathcurveto{\pgfqpoint{0.010825in}{0.018638in}}{\pgfqpoint{0.005525in}{0.020833in}}{\pgfqpoint{0.000000in}{0.020833in}}%
\pgfpathcurveto{\pgfqpoint{-0.005525in}{0.020833in}}{\pgfqpoint{-0.010825in}{0.018638in}}{\pgfqpoint{-0.014731in}{0.014731in}}%
\pgfpathcurveto{\pgfqpoint{-0.018638in}{0.010825in}}{\pgfqpoint{-0.020833in}{0.005525in}}{\pgfqpoint{-0.020833in}{0.000000in}}%
\pgfpathcurveto{\pgfqpoint{-0.020833in}{-0.005525in}}{\pgfqpoint{-0.018638in}{-0.010825in}}{\pgfqpoint{-0.014731in}{-0.014731in}}%
\pgfpathcurveto{\pgfqpoint{-0.010825in}{-0.018638in}}{\pgfqpoint{-0.005525in}{-0.020833in}}{\pgfqpoint{0.000000in}{-0.020833in}}%
\pgfpathclose%
\pgfusepath{stroke,fill}%
}%
\begin{pgfscope}%
\pgfsys@transformshift{5.524928in}{2.379950in}%
\pgfsys@useobject{currentmarker}{}%
\end{pgfscope}%
\begin{pgfscope}%
\pgfsys@transformshift{5.448480in}{2.411946in}%
\pgfsys@useobject{currentmarker}{}%
\end{pgfscope}%
\begin{pgfscope}%
\pgfsys@transformshift{5.298186in}{2.477916in}%
\pgfsys@useobject{currentmarker}{}%
\end{pgfscope}%
\begin{pgfscope}%
\pgfsys@transformshift{5.079166in}{2.581011in}%
\pgfsys@useobject{currentmarker}{}%
\end{pgfscope}%
\begin{pgfscope}%
\pgfsys@transformshift{4.798877in}{2.723094in}%
\pgfsys@useobject{currentmarker}{}%
\end{pgfscope}%
\begin{pgfscope}%
\pgfsys@transformshift{4.466865in}{2.898994in}%
\pgfsys@useobject{currentmarker}{}%
\end{pgfscope}%
\begin{pgfscope}%
\pgfsys@transformshift{4.094436in}{3.086681in}%
\pgfsys@useobject{currentmarker}{}%
\end{pgfscope}%
\begin{pgfscope}%
\pgfsys@transformshift{3.694272in}{3.239642in}%
\pgfsys@useobject{currentmarker}{}%
\end{pgfscope}%
\begin{pgfscope}%
\pgfsys@transformshift{3.280000in}{3.300000in}%
\pgfsys@useobject{currentmarker}{}%
\end{pgfscope}%
\begin{pgfscope}%
\pgfsys@transformshift{2.865728in}{3.239642in}%
\pgfsys@useobject{currentmarker}{}%
\end{pgfscope}%
\begin{pgfscope}%
\pgfsys@transformshift{2.465564in}{3.086681in}%
\pgfsys@useobject{currentmarker}{}%
\end{pgfscope}%
\begin{pgfscope}%
\pgfsys@transformshift{2.093135in}{2.898994in}%
\pgfsys@useobject{currentmarker}{}%
\end{pgfscope}%
\begin{pgfscope}%
\pgfsys@transformshift{1.761123in}{2.723094in}%
\pgfsys@useobject{currentmarker}{}%
\end{pgfscope}%
\begin{pgfscope}%
\pgfsys@transformshift{1.480834in}{2.581011in}%
\pgfsys@useobject{currentmarker}{}%
\end{pgfscope}%
\begin{pgfscope}%
\pgfsys@transformshift{1.261814in}{2.477916in}%
\pgfsys@useobject{currentmarker}{}%
\end{pgfscope}%
\begin{pgfscope}%
\pgfsys@transformshift{1.111520in}{2.411946in}%
\pgfsys@useobject{currentmarker}{}%
\end{pgfscope}%
\begin{pgfscope}%
\pgfsys@transformshift{1.035072in}{2.379950in}%
\pgfsys@useobject{currentmarker}{}%
\end{pgfscope}%
\end{pgfscope}%
\begin{pgfscope}%
\pgfpathrectangle{\pgfqpoint{0.800000in}{0.528000in}}{\pgfqpoint{4.960000in}{3.696000in}}%
\pgfusepath{clip}%
\pgfsetrectcap%
\pgfsetroundjoin%
\pgfsetlinewidth{1.505625pt}%
\definecolor{currentstroke}{rgb}{0.839216,0.152941,0.156863}%
\pgfsetstrokecolor{currentstroke}%
\pgfsetdash{}{0pt}%
\pgfpathmoveto{\pgfqpoint{1.025455in}{2.376001in}}%
\pgfpathlineto{\pgfqpoint{1.138748in}{2.423598in}}%
\pgfpathlineto{\pgfqpoint{1.252042in}{2.473505in}}%
\pgfpathlineto{\pgfqpoint{1.365336in}{2.525657in}}%
\pgfpathlineto{\pgfqpoint{1.478630in}{2.579935in}}%
\pgfpathlineto{\pgfqpoint{1.614582in}{2.647600in}}%
\pgfpathlineto{\pgfqpoint{1.773193in}{2.729407in}}%
\pgfpathlineto{\pgfqpoint{1.999781in}{2.849415in}}%
\pgfpathlineto{\pgfqpoint{2.226368in}{2.968739in}}%
\pgfpathlineto{\pgfqpoint{2.339662in}{3.026159in}}%
\pgfpathlineto{\pgfqpoint{2.430297in}{3.070154in}}%
\pgfpathlineto{\pgfqpoint{2.520932in}{3.111847in}}%
\pgfpathlineto{\pgfqpoint{2.611567in}{3.150682in}}%
\pgfpathlineto{\pgfqpoint{2.679543in}{3.177599in}}%
\pgfpathlineto{\pgfqpoint{2.747519in}{3.202362in}}%
\pgfpathlineto{\pgfqpoint{2.815496in}{3.224749in}}%
\pgfpathlineto{\pgfqpoint{2.883472in}{3.244550in}}%
\pgfpathlineto{\pgfqpoint{2.951448in}{3.261571in}}%
\pgfpathlineto{\pgfqpoint{3.019424in}{3.275639in}}%
\pgfpathlineto{\pgfqpoint{3.087401in}{3.286611in}}%
\pgfpathlineto{\pgfqpoint{3.155377in}{3.294371in}}%
\pgfpathlineto{\pgfqpoint{3.223353in}{3.298834in}}%
\pgfpathlineto{\pgfqpoint{3.268671in}{3.299953in}}%
\pgfpathlineto{\pgfqpoint{3.313988in}{3.299580in}}%
\pgfpathlineto{\pgfqpoint{3.359306in}{3.297716in}}%
\pgfpathlineto{\pgfqpoint{3.427282in}{3.292147in}}%
\pgfpathlineto{\pgfqpoint{3.495258in}{3.283306in}}%
\pgfpathlineto{\pgfqpoint{3.563234in}{3.271287in}}%
\pgfpathlineto{\pgfqpoint{3.631211in}{3.256217in}}%
\pgfpathlineto{\pgfqpoint{3.699187in}{3.238250in}}%
\pgfpathlineto{\pgfqpoint{3.767163in}{3.217564in}}%
\pgfpathlineto{\pgfqpoint{3.835139in}{3.194361in}}%
\pgfpathlineto{\pgfqpoint{3.903116in}{3.168854in}}%
\pgfpathlineto{\pgfqpoint{3.971092in}{3.141272in}}%
\pgfpathlineto{\pgfqpoint{4.061727in}{3.101670in}}%
\pgfpathlineto{\pgfqpoint{4.152362in}{3.059350in}}%
\pgfpathlineto{\pgfqpoint{4.265656in}{3.003466in}}%
\pgfpathlineto{\pgfqpoint{4.401608in}{2.933370in}}%
\pgfpathlineto{\pgfqpoint{4.628196in}{2.813233in}}%
\pgfpathlineto{\pgfqpoint{4.854783in}{2.694026in}}%
\pgfpathlineto{\pgfqpoint{4.990735in}{2.624762in}}%
\pgfpathlineto{\pgfqpoint{5.126688in}{2.557979in}}%
\pgfpathlineto{\pgfqpoint{5.239982in}{2.504533in}}%
\pgfpathlineto{\pgfqpoint{5.353275in}{2.453268in}}%
\pgfpathlineto{\pgfqpoint{5.466569in}{2.404279in}}%
\pgfpathlineto{\pgfqpoint{5.534545in}{2.376001in}}%
\pgfpathlineto{\pgfqpoint{5.534545in}{2.376001in}}%
\pgfusepath{stroke}%
\end{pgfscope}%
\begin{pgfscope}%
\pgfpathrectangle{\pgfqpoint{0.800000in}{0.528000in}}{\pgfqpoint{4.960000in}{3.696000in}}%
\pgfusepath{clip}%
\pgfsetbuttcap%
\pgfsetroundjoin%
\definecolor{currentfill}{rgb}{0.580392,0.403922,0.741176}%
\pgfsetfillcolor{currentfill}%
\pgfsetlinewidth{1.003750pt}%
\definecolor{currentstroke}{rgb}{0.580392,0.403922,0.741176}%
\pgfsetstrokecolor{currentstroke}%
\pgfsetdash{}{0pt}%
\pgfsys@defobject{currentmarker}{\pgfqpoint{-0.020833in}{-0.020833in}}{\pgfqpoint{0.020833in}{0.020833in}}{%
\pgfpathmoveto{\pgfqpoint{0.000000in}{-0.020833in}}%
\pgfpathcurveto{\pgfqpoint{0.005525in}{-0.020833in}}{\pgfqpoint{0.010825in}{-0.018638in}}{\pgfqpoint{0.014731in}{-0.014731in}}%
\pgfpathcurveto{\pgfqpoint{0.018638in}{-0.010825in}}{\pgfqpoint{0.020833in}{-0.005525in}}{\pgfqpoint{0.020833in}{0.000000in}}%
\pgfpathcurveto{\pgfqpoint{0.020833in}{0.005525in}}{\pgfqpoint{0.018638in}{0.010825in}}{\pgfqpoint{0.014731in}{0.014731in}}%
\pgfpathcurveto{\pgfqpoint{0.010825in}{0.018638in}}{\pgfqpoint{0.005525in}{0.020833in}}{\pgfqpoint{0.000000in}{0.020833in}}%
\pgfpathcurveto{\pgfqpoint{-0.005525in}{0.020833in}}{\pgfqpoint{-0.010825in}{0.018638in}}{\pgfqpoint{-0.014731in}{0.014731in}}%
\pgfpathcurveto{\pgfqpoint{-0.018638in}{0.010825in}}{\pgfqpoint{-0.020833in}{0.005525in}}{\pgfqpoint{-0.020833in}{0.000000in}}%
\pgfpathcurveto{\pgfqpoint{-0.020833in}{-0.005525in}}{\pgfqpoint{-0.018638in}{-0.010825in}}{\pgfqpoint{-0.014731in}{-0.014731in}}%
\pgfpathcurveto{\pgfqpoint{-0.010825in}{-0.018638in}}{\pgfqpoint{-0.005525in}{-0.020833in}}{\pgfqpoint{0.000000in}{-0.020833in}}%
\pgfpathclose%
\pgfusepath{stroke,fill}%
}%
\begin{pgfscope}%
\pgfsys@transformshift{5.531992in}{2.377047in}%
\pgfsys@useobject{currentmarker}{}%
\end{pgfscope}%
\begin{pgfscope}%
\pgfsys@transformshift{5.511597in}{2.385453in}%
\pgfsys@useobject{currentmarker}{}%
\end{pgfscope}%
\begin{pgfscope}%
\pgfsys@transformshift{5.470993in}{2.402413in}%
\pgfsys@useobject{currentmarker}{}%
\end{pgfscope}%
\begin{pgfscope}%
\pgfsys@transformshift{5.410547in}{2.428215in}%
\pgfsys@useobject{currentmarker}{}%
\end{pgfscope}%
\begin{pgfscope}%
\pgfsys@transformshift{5.330807in}{2.463256in}%
\pgfsys@useobject{currentmarker}{}%
\end{pgfscope}%
\begin{pgfscope}%
\pgfsys@transformshift{5.232494in}{2.508000in}%
\pgfsys@useobject{currentmarker}{}%
\end{pgfscope}%
\begin{pgfscope}%
\pgfsys@transformshift{5.116499in}{2.562888in}%
\pgfsys@useobject{currentmarker}{}%
\end{pgfscope}%
\begin{pgfscope}%
\pgfsys@transformshift{4.983872in}{2.628203in}%
\pgfsys@useobject{currentmarker}{}%
\end{pgfscope}%
\begin{pgfscope}%
\pgfsys@transformshift{4.835814in}{2.703855in}%
\pgfsys@useobject{currentmarker}{}%
\end{pgfscope}%
\begin{pgfscope}%
\pgfsys@transformshift{4.673668in}{2.789076in}%
\pgfsys@useobject{currentmarker}{}%
\end{pgfscope}%
\begin{pgfscope}%
\pgfsys@transformshift{4.498899in}{2.882017in}%
\pgfsys@useobject{currentmarker}{}%
\end{pgfscope}%
\begin{pgfscope}%
\pgfsys@transformshift{4.313093in}{2.979309in}%
\pgfsys@useobject{currentmarker}{}%
\end{pgfscope}%
\begin{pgfscope}%
\pgfsys@transformshift{4.117930in}{3.075712in}%
\pgfsys@useobject{currentmarker}{}%
\end{pgfscope}%
\begin{pgfscope}%
\pgfsys@transformshift{3.915179in}{3.164105in}%
\pgfsys@useobject{currentmarker}{}%
\end{pgfscope}%
\begin{pgfscope}%
\pgfsys@transformshift{3.706676in}{3.236101in}%
\pgfsys@useobject{currentmarker}{}%
\end{pgfscope}%
\begin{pgfscope}%
\pgfsys@transformshift{3.494308in}{3.283452in}%
\pgfsys@useobject{currentmarker}{}%
\end{pgfscope}%
\begin{pgfscope}%
\pgfsys@transformshift{3.280000in}{3.300000in}%
\pgfsys@useobject{currentmarker}{}%
\end{pgfscope}%
\begin{pgfscope}%
\pgfsys@transformshift{3.065692in}{3.283452in}%
\pgfsys@useobject{currentmarker}{}%
\end{pgfscope}%
\begin{pgfscope}%
\pgfsys@transformshift{2.853324in}{3.236101in}%
\pgfsys@useobject{currentmarker}{}%
\end{pgfscope}%
\begin{pgfscope}%
\pgfsys@transformshift{2.644821in}{3.164105in}%
\pgfsys@useobject{currentmarker}{}%
\end{pgfscope}%
\begin{pgfscope}%
\pgfsys@transformshift{2.442070in}{3.075712in}%
\pgfsys@useobject{currentmarker}{}%
\end{pgfscope}%
\begin{pgfscope}%
\pgfsys@transformshift{2.246907in}{2.979309in}%
\pgfsys@useobject{currentmarker}{}%
\end{pgfscope}%
\begin{pgfscope}%
\pgfsys@transformshift{2.061101in}{2.882017in}%
\pgfsys@useobject{currentmarker}{}%
\end{pgfscope}%
\begin{pgfscope}%
\pgfsys@transformshift{1.886332in}{2.789076in}%
\pgfsys@useobject{currentmarker}{}%
\end{pgfscope}%
\begin{pgfscope}%
\pgfsys@transformshift{1.724186in}{2.703855in}%
\pgfsys@useobject{currentmarker}{}%
\end{pgfscope}%
\begin{pgfscope}%
\pgfsys@transformshift{1.576128in}{2.628203in}%
\pgfsys@useobject{currentmarker}{}%
\end{pgfscope}%
\begin{pgfscope}%
\pgfsys@transformshift{1.443501in}{2.562888in}%
\pgfsys@useobject{currentmarker}{}%
\end{pgfscope}%
\begin{pgfscope}%
\pgfsys@transformshift{1.327506in}{2.508000in}%
\pgfsys@useobject{currentmarker}{}%
\end{pgfscope}%
\begin{pgfscope}%
\pgfsys@transformshift{1.229193in}{2.463256in}%
\pgfsys@useobject{currentmarker}{}%
\end{pgfscope}%
\begin{pgfscope}%
\pgfsys@transformshift{1.149453in}{2.428215in}%
\pgfsys@useobject{currentmarker}{}%
\end{pgfscope}%
\begin{pgfscope}%
\pgfsys@transformshift{1.089007in}{2.402413in}%
\pgfsys@useobject{currentmarker}{}%
\end{pgfscope}%
\begin{pgfscope}%
\pgfsys@transformshift{1.048403in}{2.385453in}%
\pgfsys@useobject{currentmarker}{}%
\end{pgfscope}%
\begin{pgfscope}%
\pgfsys@transformshift{1.028008in}{2.377047in}%
\pgfsys@useobject{currentmarker}{}%
\end{pgfscope}%
\end{pgfscope}%
\begin{pgfscope}%
\pgfpathrectangle{\pgfqpoint{0.800000in}{0.528000in}}{\pgfqpoint{4.960000in}{3.696000in}}%
\pgfusepath{clip}%
\pgfsetrectcap%
\pgfsetroundjoin%
\pgfsetlinewidth{1.505625pt}%
\definecolor{currentstroke}{rgb}{0.549020,0.337255,0.294118}%
\pgfsetstrokecolor{currentstroke}%
\pgfsetdash{}{0pt}%
\pgfpathmoveto{\pgfqpoint{1.025455in}{2.376000in}}%
\pgfpathlineto{\pgfqpoint{1.138748in}{2.423597in}}%
\pgfpathlineto{\pgfqpoint{1.252042in}{2.473505in}}%
\pgfpathlineto{\pgfqpoint{1.365336in}{2.525658in}}%
\pgfpathlineto{\pgfqpoint{1.478630in}{2.579935in}}%
\pgfpathlineto{\pgfqpoint{1.614582in}{2.647600in}}%
\pgfpathlineto{\pgfqpoint{1.773193in}{2.729407in}}%
\pgfpathlineto{\pgfqpoint{1.999781in}{2.849415in}}%
\pgfpathlineto{\pgfqpoint{2.226368in}{2.968739in}}%
\pgfpathlineto{\pgfqpoint{2.339662in}{3.026159in}}%
\pgfpathlineto{\pgfqpoint{2.430297in}{3.070154in}}%
\pgfpathlineto{\pgfqpoint{2.520932in}{3.111847in}}%
\pgfpathlineto{\pgfqpoint{2.611567in}{3.150683in}}%
\pgfpathlineto{\pgfqpoint{2.679543in}{3.177599in}}%
\pgfpathlineto{\pgfqpoint{2.747519in}{3.202362in}}%
\pgfpathlineto{\pgfqpoint{2.815496in}{3.224750in}}%
\pgfpathlineto{\pgfqpoint{2.883472in}{3.244550in}}%
\pgfpathlineto{\pgfqpoint{2.951448in}{3.261570in}}%
\pgfpathlineto{\pgfqpoint{3.019424in}{3.275639in}}%
\pgfpathlineto{\pgfqpoint{3.087401in}{3.286611in}}%
\pgfpathlineto{\pgfqpoint{3.155377in}{3.294371in}}%
\pgfpathlineto{\pgfqpoint{3.223353in}{3.298834in}}%
\pgfpathlineto{\pgfqpoint{3.268671in}{3.299953in}}%
\pgfpathlineto{\pgfqpoint{3.313988in}{3.299580in}}%
\pgfpathlineto{\pgfqpoint{3.359306in}{3.297716in}}%
\pgfpathlineto{\pgfqpoint{3.427282in}{3.292147in}}%
\pgfpathlineto{\pgfqpoint{3.495258in}{3.283306in}}%
\pgfpathlineto{\pgfqpoint{3.563234in}{3.271287in}}%
\pgfpathlineto{\pgfqpoint{3.631211in}{3.256217in}}%
\pgfpathlineto{\pgfqpoint{3.699187in}{3.238250in}}%
\pgfpathlineto{\pgfqpoint{3.767163in}{3.217564in}}%
\pgfpathlineto{\pgfqpoint{3.835139in}{3.194361in}}%
\pgfpathlineto{\pgfqpoint{3.903116in}{3.168855in}}%
\pgfpathlineto{\pgfqpoint{3.971092in}{3.141272in}}%
\pgfpathlineto{\pgfqpoint{4.061727in}{3.101670in}}%
\pgfpathlineto{\pgfqpoint{4.152362in}{3.059350in}}%
\pgfpathlineto{\pgfqpoint{4.265656in}{3.003466in}}%
\pgfpathlineto{\pgfqpoint{4.401608in}{2.933370in}}%
\pgfpathlineto{\pgfqpoint{4.628196in}{2.813234in}}%
\pgfpathlineto{\pgfqpoint{4.854783in}{2.694026in}}%
\pgfpathlineto{\pgfqpoint{4.990735in}{2.624761in}}%
\pgfpathlineto{\pgfqpoint{5.126688in}{2.557979in}}%
\pgfpathlineto{\pgfqpoint{5.239982in}{2.504533in}}%
\pgfpathlineto{\pgfqpoint{5.353275in}{2.453268in}}%
\pgfpathlineto{\pgfqpoint{5.466569in}{2.404279in}}%
\pgfpathlineto{\pgfqpoint{5.534545in}{2.376000in}}%
\pgfpathlineto{\pgfqpoint{5.534545in}{2.376000in}}%
\pgfusepath{stroke}%
\end{pgfscope}%
\begin{pgfscope}%
\pgfpathrectangle{\pgfqpoint{0.800000in}{0.528000in}}{\pgfqpoint{4.960000in}{3.696000in}}%
\pgfusepath{clip}%
\pgfsetrectcap%
\pgfsetroundjoin%
\pgfsetlinewidth{1.505625pt}%
\definecolor{currentstroke}{rgb}{0.890196,0.466667,0.760784}%
\pgfsetstrokecolor{currentstroke}%
\pgfsetdash{}{0pt}%
\pgfpathmoveto{\pgfqpoint{1.025455in}{1.523077in}}%
\pgfpathlineto{\pgfqpoint{1.184066in}{1.533748in}}%
\pgfpathlineto{\pgfqpoint{1.320018in}{1.544892in}}%
\pgfpathlineto{\pgfqpoint{1.455971in}{1.558428in}}%
\pgfpathlineto{\pgfqpoint{1.569265in}{1.572045in}}%
\pgfpathlineto{\pgfqpoint{1.659899in}{1.584860in}}%
\pgfpathlineto{\pgfqpoint{1.750534in}{1.599776in}}%
\pgfpathlineto{\pgfqpoint{1.841169in}{1.617263in}}%
\pgfpathlineto{\pgfqpoint{1.909146in}{1.632419in}}%
\pgfpathlineto{\pgfqpoint{1.977122in}{1.649670in}}%
\pgfpathlineto{\pgfqpoint{2.045098in}{1.669400in}}%
\pgfpathlineto{\pgfqpoint{2.113074in}{1.692080in}}%
\pgfpathlineto{\pgfqpoint{2.158392in}{1.709119in}}%
\pgfpathlineto{\pgfqpoint{2.203709in}{1.727926in}}%
\pgfpathlineto{\pgfqpoint{2.249027in}{1.748735in}}%
\pgfpathlineto{\pgfqpoint{2.294344in}{1.771818in}}%
\pgfpathlineto{\pgfqpoint{2.339662in}{1.797485in}}%
\pgfpathlineto{\pgfqpoint{2.384979in}{1.826095in}}%
\pgfpathlineto{\pgfqpoint{2.430297in}{1.858060in}}%
\pgfpathlineto{\pgfqpoint{2.475614in}{1.893855in}}%
\pgfpathlineto{\pgfqpoint{2.520932in}{1.934017in}}%
\pgfpathlineto{\pgfqpoint{2.566249in}{1.979154in}}%
\pgfpathlineto{\pgfqpoint{2.611567in}{2.029943in}}%
\pgfpathlineto{\pgfqpoint{2.656884in}{2.087123in}}%
\pgfpathlineto{\pgfqpoint{2.702202in}{2.151469in}}%
\pgfpathlineto{\pgfqpoint{2.724861in}{2.186574in}}%
\pgfpathlineto{\pgfqpoint{2.747519in}{2.223757in}}%
\pgfpathlineto{\pgfqpoint{2.770178in}{2.263104in}}%
\pgfpathlineto{\pgfqpoint{2.815496in}{2.348561in}}%
\pgfpathlineto{\pgfqpoint{2.860813in}{2.443286in}}%
\pgfpathlineto{\pgfqpoint{2.906131in}{2.547122in}}%
\pgfpathlineto{\pgfqpoint{2.951448in}{2.659117in}}%
\pgfpathlineto{\pgfqpoint{3.019424in}{2.837353in}}%
\pgfpathlineto{\pgfqpoint{3.087401in}{3.014864in}}%
\pgfpathlineto{\pgfqpoint{3.110059in}{3.070154in}}%
\pgfpathlineto{\pgfqpoint{3.132718in}{3.121845in}}%
\pgfpathlineto{\pgfqpoint{3.155377in}{3.168855in}}%
\pgfpathlineto{\pgfqpoint{3.178036in}{3.210099in}}%
\pgfpathlineto{\pgfqpoint{3.200694in}{3.244550in}}%
\pgfpathlineto{\pgfqpoint{3.223353in}{3.271287in}}%
\pgfpathlineto{\pgfqpoint{3.246012in}{3.289560in}}%
\pgfpathlineto{\pgfqpoint{3.268671in}{3.298834in}}%
\pgfpathlineto{\pgfqpoint{3.291329in}{3.298834in}}%
\pgfpathlineto{\pgfqpoint{3.313988in}{3.289560in}}%
\pgfpathlineto{\pgfqpoint{3.336647in}{3.271287in}}%
\pgfpathlineto{\pgfqpoint{3.359306in}{3.244550in}}%
\pgfpathlineto{\pgfqpoint{3.381964in}{3.210099in}}%
\pgfpathlineto{\pgfqpoint{3.404623in}{3.168855in}}%
\pgfpathlineto{\pgfqpoint{3.427282in}{3.121845in}}%
\pgfpathlineto{\pgfqpoint{3.449941in}{3.070154in}}%
\pgfpathlineto{\pgfqpoint{3.495258in}{2.957011in}}%
\pgfpathlineto{\pgfqpoint{3.631211in}{2.602200in}}%
\pgfpathlineto{\pgfqpoint{3.676528in}{2.494102in}}%
\pgfpathlineto{\pgfqpoint{3.721846in}{2.394759in}}%
\pgfpathlineto{\pgfqpoint{3.767163in}{2.304686in}}%
\pgfpathlineto{\pgfqpoint{3.812481in}{2.223757in}}%
\pgfpathlineto{\pgfqpoint{3.857798in}{2.151469in}}%
\pgfpathlineto{\pgfqpoint{3.903116in}{2.087123in}}%
\pgfpathlineto{\pgfqpoint{3.948433in}{2.029943in}}%
\pgfpathlineto{\pgfqpoint{3.993751in}{1.979154in}}%
\pgfpathlineto{\pgfqpoint{4.039068in}{1.934017in}}%
\pgfpathlineto{\pgfqpoint{4.084386in}{1.893855in}}%
\pgfpathlineto{\pgfqpoint{4.129703in}{1.858060in}}%
\pgfpathlineto{\pgfqpoint{4.175021in}{1.826095in}}%
\pgfpathlineto{\pgfqpoint{4.220338in}{1.797485in}}%
\pgfpathlineto{\pgfqpoint{4.265656in}{1.771818in}}%
\pgfpathlineto{\pgfqpoint{4.310973in}{1.748735in}}%
\pgfpathlineto{\pgfqpoint{4.356291in}{1.727926in}}%
\pgfpathlineto{\pgfqpoint{4.401608in}{1.709119in}}%
\pgfpathlineto{\pgfqpoint{4.446926in}{1.692080in}}%
\pgfpathlineto{\pgfqpoint{4.514902in}{1.669400in}}%
\pgfpathlineto{\pgfqpoint{4.582878in}{1.649670in}}%
\pgfpathlineto{\pgfqpoint{4.650854in}{1.632419in}}%
\pgfpathlineto{\pgfqpoint{4.718831in}{1.617263in}}%
\pgfpathlineto{\pgfqpoint{4.809466in}{1.599776in}}%
\pgfpathlineto{\pgfqpoint{4.900101in}{1.584860in}}%
\pgfpathlineto{\pgfqpoint{4.990735in}{1.572045in}}%
\pgfpathlineto{\pgfqpoint{5.104029in}{1.558428in}}%
\pgfpathlineto{\pgfqpoint{5.217323in}{1.546965in}}%
\pgfpathlineto{\pgfqpoint{5.353275in}{1.535463in}}%
\pgfpathlineto{\pgfqpoint{5.511887in}{1.524471in}}%
\pgfpathlineto{\pgfqpoint{5.534545in}{1.523077in}}%
\pgfpathlineto{\pgfqpoint{5.534545in}{1.523077in}}%
\pgfusepath{stroke}%
\end{pgfscope}%
\begin{pgfscope}%
\pgfpathrectangle{\pgfqpoint{0.800000in}{0.528000in}}{\pgfqpoint{4.960000in}{3.696000in}}%
\pgfusepath{clip}%
\pgfsetbuttcap%
\pgfsetroundjoin%
\definecolor{currentfill}{rgb}{0.498039,0.498039,0.498039}%
\pgfsetfillcolor{currentfill}%
\pgfsetlinewidth{1.003750pt}%
\definecolor{currentstroke}{rgb}{0.498039,0.498039,0.498039}%
\pgfsetstrokecolor{currentstroke}%
\pgfsetdash{}{0pt}%
\pgfsys@defobject{currentmarker}{\pgfqpoint{-0.020833in}{-0.020833in}}{\pgfqpoint{0.020833in}{0.020833in}}{%
\pgfpathmoveto{\pgfqpoint{0.000000in}{-0.020833in}}%
\pgfpathcurveto{\pgfqpoint{0.005525in}{-0.020833in}}{\pgfqpoint{0.010825in}{-0.018638in}}{\pgfqpoint{0.014731in}{-0.014731in}}%
\pgfpathcurveto{\pgfqpoint{0.018638in}{-0.010825in}}{\pgfqpoint{0.020833in}{-0.005525in}}{\pgfqpoint{0.020833in}{0.000000in}}%
\pgfpathcurveto{\pgfqpoint{0.020833in}{0.005525in}}{\pgfqpoint{0.018638in}{0.010825in}}{\pgfqpoint{0.014731in}{0.014731in}}%
\pgfpathcurveto{\pgfqpoint{0.010825in}{0.018638in}}{\pgfqpoint{0.005525in}{0.020833in}}{\pgfqpoint{0.000000in}{0.020833in}}%
\pgfpathcurveto{\pgfqpoint{-0.005525in}{0.020833in}}{\pgfqpoint{-0.010825in}{0.018638in}}{\pgfqpoint{-0.014731in}{0.014731in}}%
\pgfpathcurveto{\pgfqpoint{-0.018638in}{0.010825in}}{\pgfqpoint{-0.020833in}{0.005525in}}{\pgfqpoint{-0.020833in}{0.000000in}}%
\pgfpathcurveto{\pgfqpoint{-0.020833in}{-0.005525in}}{\pgfqpoint{-0.018638in}{-0.010825in}}{\pgfqpoint{-0.014731in}{-0.014731in}}%
\pgfpathcurveto{\pgfqpoint{-0.010825in}{-0.018638in}}{\pgfqpoint{-0.005525in}{-0.020833in}}{\pgfqpoint{0.000000in}{-0.020833in}}%
\pgfpathclose%
\pgfusepath{stroke,fill}%
}%
\begin{pgfscope}%
\pgfsys@transformshift{1.025455in}{1.523077in}%
\pgfsys@useobject{currentmarker}{}%
\end{pgfscope}%
\begin{pgfscope}%
\pgfsys@transformshift{1.669610in}{1.586350in}%
\pgfsys@useobject{currentmarker}{}%
\end{pgfscope}%
\begin{pgfscope}%
\pgfsys@transformshift{2.313766in}{1.782482in}%
\pgfsys@useobject{currentmarker}{}%
\end{pgfscope}%
\begin{pgfscope}%
\pgfsys@transformshift{2.957922in}{2.675676in}%
\pgfsys@useobject{currentmarker}{}%
\end{pgfscope}%
\begin{pgfscope}%
\pgfsys@transformshift{3.602078in}{2.675676in}%
\pgfsys@useobject{currentmarker}{}%
\end{pgfscope}%
\begin{pgfscope}%
\pgfsys@transformshift{4.246234in}{1.782482in}%
\pgfsys@useobject{currentmarker}{}%
\end{pgfscope}%
\begin{pgfscope}%
\pgfsys@transformshift{4.890390in}{1.586350in}%
\pgfsys@useobject{currentmarker}{}%
\end{pgfscope}%
\end{pgfscope}%
\begin{pgfscope}%
\pgfpathrectangle{\pgfqpoint{0.800000in}{0.528000in}}{\pgfqpoint{4.960000in}{3.696000in}}%
\pgfusepath{clip}%
\pgfsetrectcap%
\pgfsetroundjoin%
\pgfsetlinewidth{1.505625pt}%
\definecolor{currentstroke}{rgb}{0.737255,0.741176,0.133333}%
\pgfsetstrokecolor{currentstroke}%
\pgfsetdash{}{0pt}%
\pgfpathmoveto{\pgfqpoint{1.025455in}{1.523077in}}%
\pgfpathlineto{\pgfqpoint{1.048113in}{1.591768in}}%
\pgfpathlineto{\pgfqpoint{1.070772in}{1.650833in}}%
\pgfpathlineto{\pgfqpoint{1.093431in}{1.700981in}}%
\pgfpathlineto{\pgfqpoint{1.116090in}{1.742892in}}%
\pgfpathlineto{\pgfqpoint{1.138748in}{1.777220in}}%
\pgfpathlineto{\pgfqpoint{1.161407in}{1.804587in}}%
\pgfpathlineto{\pgfqpoint{1.184066in}{1.825591in}}%
\pgfpathlineto{\pgfqpoint{1.206725in}{1.840798in}}%
\pgfpathlineto{\pgfqpoint{1.229383in}{1.850754in}}%
\pgfpathlineto{\pgfqpoint{1.252042in}{1.855975in}}%
\pgfpathlineto{\pgfqpoint{1.274701in}{1.856953in}}%
\pgfpathlineto{\pgfqpoint{1.297360in}{1.854156in}}%
\pgfpathlineto{\pgfqpoint{1.320018in}{1.848029in}}%
\pgfpathlineto{\pgfqpoint{1.342677in}{1.838994in}}%
\pgfpathlineto{\pgfqpoint{1.365336in}{1.827449in}}%
\pgfpathlineto{\pgfqpoint{1.387995in}{1.813771in}}%
\pgfpathlineto{\pgfqpoint{1.410653in}{1.798316in}}%
\pgfpathlineto{\pgfqpoint{1.455971in}{1.763396in}}%
\pgfpathlineto{\pgfqpoint{1.523947in}{1.705431in}}%
\pgfpathlineto{\pgfqpoint{1.591923in}{1.646873in}}%
\pgfpathlineto{\pgfqpoint{1.637241in}{1.610324in}}%
\pgfpathlineto{\pgfqpoint{1.682558in}{1.577371in}}%
\pgfpathlineto{\pgfqpoint{1.727876in}{1.549122in}}%
\pgfpathlineto{\pgfqpoint{1.750534in}{1.537050in}}%
\pgfpathlineto{\pgfqpoint{1.773193in}{1.526473in}}%
\pgfpathlineto{\pgfqpoint{1.795852in}{1.517472in}}%
\pgfpathlineto{\pgfqpoint{1.818511in}{1.510116in}}%
\pgfpathlineto{\pgfqpoint{1.841169in}{1.504462in}}%
\pgfpathlineto{\pgfqpoint{1.863828in}{1.500559in}}%
\pgfpathlineto{\pgfqpoint{1.886487in}{1.498442in}}%
\pgfpathlineto{\pgfqpoint{1.909146in}{1.498139in}}%
\pgfpathlineto{\pgfqpoint{1.931804in}{1.499668in}}%
\pgfpathlineto{\pgfqpoint{1.954463in}{1.503038in}}%
\pgfpathlineto{\pgfqpoint{1.977122in}{1.508249in}}%
\pgfpathlineto{\pgfqpoint{1.999781in}{1.515295in}}%
\pgfpathlineto{\pgfqpoint{2.022439in}{1.524159in}}%
\pgfpathlineto{\pgfqpoint{2.045098in}{1.534820in}}%
\pgfpathlineto{\pgfqpoint{2.067757in}{1.547249in}}%
\pgfpathlineto{\pgfqpoint{2.090416in}{1.561409in}}%
\pgfpathlineto{\pgfqpoint{2.113074in}{1.577260in}}%
\pgfpathlineto{\pgfqpoint{2.135733in}{1.594754in}}%
\pgfpathlineto{\pgfqpoint{2.158392in}{1.613839in}}%
\pgfpathlineto{\pgfqpoint{2.203709in}{1.656545in}}%
\pgfpathlineto{\pgfqpoint{2.249027in}{1.704865in}}%
\pgfpathlineto{\pgfqpoint{2.294344in}{1.758225in}}%
\pgfpathlineto{\pgfqpoint{2.339662in}{1.815996in}}%
\pgfpathlineto{\pgfqpoint{2.384979in}{1.877510in}}%
\pgfpathlineto{\pgfqpoint{2.430297in}{1.942066in}}%
\pgfpathlineto{\pgfqpoint{2.498273in}{2.043014in}}%
\pgfpathlineto{\pgfqpoint{2.724861in}{2.385149in}}%
\pgfpathlineto{\pgfqpoint{2.770178in}{2.449206in}}%
\pgfpathlineto{\pgfqpoint{2.815496in}{2.510228in}}%
\pgfpathlineto{\pgfqpoint{2.860813in}{2.567602in}}%
\pgfpathlineto{\pgfqpoint{2.906131in}{2.620759in}}%
\pgfpathlineto{\pgfqpoint{2.951448in}{2.669174in}}%
\pgfpathlineto{\pgfqpoint{2.996766in}{2.712375in}}%
\pgfpathlineto{\pgfqpoint{3.042083in}{2.749942in}}%
\pgfpathlineto{\pgfqpoint{3.064742in}{2.766496in}}%
\pgfpathlineto{\pgfqpoint{3.087401in}{2.781511in}}%
\pgfpathlineto{\pgfqpoint{3.110059in}{2.794948in}}%
\pgfpathlineto{\pgfqpoint{3.132718in}{2.806778in}}%
\pgfpathlineto{\pgfqpoint{3.155377in}{2.816970in}}%
\pgfpathlineto{\pgfqpoint{3.178036in}{2.825501in}}%
\pgfpathlineto{\pgfqpoint{3.200694in}{2.832351in}}%
\pgfpathlineto{\pgfqpoint{3.223353in}{2.837502in}}%
\pgfpathlineto{\pgfqpoint{3.246012in}{2.840943in}}%
\pgfpathlineto{\pgfqpoint{3.268671in}{2.842666in}}%
\pgfpathlineto{\pgfqpoint{3.291329in}{2.842666in}}%
\pgfpathlineto{\pgfqpoint{3.313988in}{2.840943in}}%
\pgfpathlineto{\pgfqpoint{3.336647in}{2.837502in}}%
\pgfpathlineto{\pgfqpoint{3.359306in}{2.832351in}}%
\pgfpathlineto{\pgfqpoint{3.381964in}{2.825501in}}%
\pgfpathlineto{\pgfqpoint{3.404623in}{2.816970in}}%
\pgfpathlineto{\pgfqpoint{3.427282in}{2.806778in}}%
\pgfpathlineto{\pgfqpoint{3.449941in}{2.794948in}}%
\pgfpathlineto{\pgfqpoint{3.472599in}{2.781511in}}%
\pgfpathlineto{\pgfqpoint{3.495258in}{2.766496in}}%
\pgfpathlineto{\pgfqpoint{3.517917in}{2.749942in}}%
\pgfpathlineto{\pgfqpoint{3.540576in}{2.731887in}}%
\pgfpathlineto{\pgfqpoint{3.585893in}{2.691454in}}%
\pgfpathlineto{\pgfqpoint{3.631211in}{2.645590in}}%
\pgfpathlineto{\pgfqpoint{3.676528in}{2.594741in}}%
\pgfpathlineto{\pgfqpoint{3.721846in}{2.539407in}}%
\pgfpathlineto{\pgfqpoint{3.767163in}{2.480136in}}%
\pgfpathlineto{\pgfqpoint{3.812481in}{2.417517in}}%
\pgfpathlineto{\pgfqpoint{3.880457in}{2.318705in}}%
\pgfpathlineto{\pgfqpoint{3.971092in}{2.181403in}}%
\pgfpathlineto{\pgfqpoint{4.084386in}{2.008939in}}%
\pgfpathlineto{\pgfqpoint{4.152362in}{1.909452in}}%
\pgfpathlineto{\pgfqpoint{4.197679in}{1.846328in}}%
\pgfpathlineto{\pgfqpoint{4.242997in}{1.786600in}}%
\pgfpathlineto{\pgfqpoint{4.288314in}{1.730953in}}%
\pgfpathlineto{\pgfqpoint{4.333632in}{1.680038in}}%
\pgfpathlineto{\pgfqpoint{4.378949in}{1.634456in}}%
\pgfpathlineto{\pgfqpoint{4.401608in}{1.613839in}}%
\pgfpathlineto{\pgfqpoint{4.424267in}{1.594754in}}%
\pgfpathlineto{\pgfqpoint{4.446926in}{1.577260in}}%
\pgfpathlineto{\pgfqpoint{4.469584in}{1.561409in}}%
\pgfpathlineto{\pgfqpoint{4.492243in}{1.547249in}}%
\pgfpathlineto{\pgfqpoint{4.514902in}{1.534820in}}%
\pgfpathlineto{\pgfqpoint{4.537561in}{1.524159in}}%
\pgfpathlineto{\pgfqpoint{4.560219in}{1.515295in}}%
\pgfpathlineto{\pgfqpoint{4.582878in}{1.508249in}}%
\pgfpathlineto{\pgfqpoint{4.605537in}{1.503038in}}%
\pgfpathlineto{\pgfqpoint{4.628196in}{1.499668in}}%
\pgfpathlineto{\pgfqpoint{4.650854in}{1.498139in}}%
\pgfpathlineto{\pgfqpoint{4.673513in}{1.498442in}}%
\pgfpathlineto{\pgfqpoint{4.696172in}{1.500559in}}%
\pgfpathlineto{\pgfqpoint{4.718831in}{1.504462in}}%
\pgfpathlineto{\pgfqpoint{4.741489in}{1.510116in}}%
\pgfpathlineto{\pgfqpoint{4.764148in}{1.517472in}}%
\pgfpathlineto{\pgfqpoint{4.786807in}{1.526473in}}%
\pgfpathlineto{\pgfqpoint{4.809466in}{1.537050in}}%
\pgfpathlineto{\pgfqpoint{4.832124in}{1.549122in}}%
\pgfpathlineto{\pgfqpoint{4.854783in}{1.562597in}}%
\pgfpathlineto{\pgfqpoint{4.900101in}{1.593323in}}%
\pgfpathlineto{\pgfqpoint{4.945418in}{1.628228in}}%
\pgfpathlineto{\pgfqpoint{4.990735in}{1.666085in}}%
\pgfpathlineto{\pgfqpoint{5.104029in}{1.763396in}}%
\pgfpathlineto{\pgfqpoint{5.149347in}{1.798316in}}%
\pgfpathlineto{\pgfqpoint{5.172005in}{1.813771in}}%
\pgfpathlineto{\pgfqpoint{5.194664in}{1.827449in}}%
\pgfpathlineto{\pgfqpoint{5.217323in}{1.838994in}}%
\pgfpathlineto{\pgfqpoint{5.239982in}{1.848029in}}%
\pgfpathlineto{\pgfqpoint{5.262640in}{1.854156in}}%
\pgfpathlineto{\pgfqpoint{5.285299in}{1.856953in}}%
\pgfpathlineto{\pgfqpoint{5.307958in}{1.855975in}}%
\pgfpathlineto{\pgfqpoint{5.330617in}{1.850754in}}%
\pgfpathlineto{\pgfqpoint{5.353275in}{1.840798in}}%
\pgfpathlineto{\pgfqpoint{5.375934in}{1.825591in}}%
\pgfpathlineto{\pgfqpoint{5.398593in}{1.804587in}}%
\pgfpathlineto{\pgfqpoint{5.421252in}{1.777220in}}%
\pgfpathlineto{\pgfqpoint{5.443910in}{1.742892in}}%
\pgfpathlineto{\pgfqpoint{5.466569in}{1.700981in}}%
\pgfpathlineto{\pgfqpoint{5.489228in}{1.650833in}}%
\pgfpathlineto{\pgfqpoint{5.511887in}{1.591768in}}%
\pgfpathlineto{\pgfqpoint{5.534545in}{1.523077in}}%
\pgfpathlineto{\pgfqpoint{5.534545in}{1.523077in}}%
\pgfusepath{stroke}%
\end{pgfscope}%
\begin{pgfscope}%
\pgfpathrectangle{\pgfqpoint{0.800000in}{0.528000in}}{\pgfqpoint{4.960000in}{3.696000in}}%
\pgfusepath{clip}%
\pgfsetbuttcap%
\pgfsetroundjoin%
\definecolor{currentfill}{rgb}{0.090196,0.745098,0.811765}%
\pgfsetfillcolor{currentfill}%
\pgfsetlinewidth{1.003750pt}%
\definecolor{currentstroke}{rgb}{0.090196,0.745098,0.811765}%
\pgfsetstrokecolor{currentstroke}%
\pgfsetdash{}{0pt}%
\pgfsys@defobject{currentmarker}{\pgfqpoint{-0.020833in}{-0.020833in}}{\pgfqpoint{0.020833in}{0.020833in}}{%
\pgfpathmoveto{\pgfqpoint{0.000000in}{-0.020833in}}%
\pgfpathcurveto{\pgfqpoint{0.005525in}{-0.020833in}}{\pgfqpoint{0.010825in}{-0.018638in}}{\pgfqpoint{0.014731in}{-0.014731in}}%
\pgfpathcurveto{\pgfqpoint{0.018638in}{-0.010825in}}{\pgfqpoint{0.020833in}{-0.005525in}}{\pgfqpoint{0.020833in}{0.000000in}}%
\pgfpathcurveto{\pgfqpoint{0.020833in}{0.005525in}}{\pgfqpoint{0.018638in}{0.010825in}}{\pgfqpoint{0.014731in}{0.014731in}}%
\pgfpathcurveto{\pgfqpoint{0.010825in}{0.018638in}}{\pgfqpoint{0.005525in}{0.020833in}}{\pgfqpoint{0.000000in}{0.020833in}}%
\pgfpathcurveto{\pgfqpoint{-0.005525in}{0.020833in}}{\pgfqpoint{-0.010825in}{0.018638in}}{\pgfqpoint{-0.014731in}{0.014731in}}%
\pgfpathcurveto{\pgfqpoint{-0.018638in}{0.010825in}}{\pgfqpoint{-0.020833in}{0.005525in}}{\pgfqpoint{-0.020833in}{0.000000in}}%
\pgfpathcurveto{\pgfqpoint{-0.020833in}{-0.005525in}}{\pgfqpoint{-0.018638in}{-0.010825in}}{\pgfqpoint{-0.014731in}{-0.014731in}}%
\pgfpathcurveto{\pgfqpoint{-0.010825in}{-0.018638in}}{\pgfqpoint{-0.005525in}{-0.020833in}}{\pgfqpoint{0.000000in}{-0.020833in}}%
\pgfpathclose%
\pgfusepath{stroke,fill}%
}%
\begin{pgfscope}%
\pgfsys@transformshift{1.025455in}{1.523077in}%
\pgfsys@useobject{currentmarker}{}%
\end{pgfscope}%
\begin{pgfscope}%
\pgfsys@transformshift{1.094825in}{1.527474in}%
\pgfsys@useobject{currentmarker}{}%
\end{pgfscope}%
\begin{pgfscope}%
\pgfsys@transformshift{1.164196in}{1.532286in}%
\pgfsys@useobject{currentmarker}{}%
\end{pgfscope}%
\begin{pgfscope}%
\pgfsys@transformshift{1.233566in}{1.537565in}%
\pgfsys@useobject{currentmarker}{}%
\end{pgfscope}%
\begin{pgfscope}%
\pgfsys@transformshift{1.302937in}{1.543373in}%
\pgfsys@useobject{currentmarker}{}%
\end{pgfscope}%
\begin{pgfscope}%
\pgfsys@transformshift{1.372308in}{1.549781in}%
\pgfsys@useobject{currentmarker}{}%
\end{pgfscope}%
\begin{pgfscope}%
\pgfsys@transformshift{1.441678in}{1.556873in}%
\pgfsys@useobject{currentmarker}{}%
\end{pgfscope}%
\begin{pgfscope}%
\pgfsys@transformshift{1.511049in}{1.564748in}%
\pgfsys@useobject{currentmarker}{}%
\end{pgfscope}%
\begin{pgfscope}%
\pgfsys@transformshift{1.580420in}{1.573522in}%
\pgfsys@useobject{currentmarker}{}%
\end{pgfscope}%
\begin{pgfscope}%
\pgfsys@transformshift{1.649790in}{1.583334in}%
\pgfsys@useobject{currentmarker}{}%
\end{pgfscope}%
\begin{pgfscope}%
\pgfsys@transformshift{1.719161in}{1.594348in}%
\pgfsys@useobject{currentmarker}{}%
\end{pgfscope}%
\begin{pgfscope}%
\pgfsys@transformshift{1.788531in}{1.606763in}%
\pgfsys@useobject{currentmarker}{}%
\end{pgfscope}%
\begin{pgfscope}%
\pgfsys@transformshift{1.857902in}{1.620817in}%
\pgfsys@useobject{currentmarker}{}%
\end{pgfscope}%
\begin{pgfscope}%
\pgfsys@transformshift{1.927273in}{1.636800in}%
\pgfsys@useobject{currentmarker}{}%
\end{pgfscope}%
\begin{pgfscope}%
\pgfsys@transformshift{1.996643in}{1.655064in}%
\pgfsys@useobject{currentmarker}{}%
\end{pgfscope}%
\begin{pgfscope}%
\pgfsys@transformshift{2.066014in}{1.676040in}%
\pgfsys@useobject{currentmarker}{}%
\end{pgfscope}%
\begin{pgfscope}%
\pgfsys@transformshift{2.135385in}{1.700261in}%
\pgfsys@useobject{currentmarker}{}%
\end{pgfscope}%
\begin{pgfscope}%
\pgfsys@transformshift{2.204755in}{1.728382in}%
\pgfsys@useobject{currentmarker}{}%
\end{pgfscope}%
\begin{pgfscope}%
\pgfsys@transformshift{2.274126in}{1.761220in}%
\pgfsys@useobject{currentmarker}{}%
\end{pgfscope}%
\begin{pgfscope}%
\pgfsys@transformshift{2.343497in}{1.799786in}%
\pgfsys@useobject{currentmarker}{}%
\end{pgfscope}%
\begin{pgfscope}%
\pgfsys@transformshift{2.412867in}{1.845340in}%
\pgfsys@useobject{currentmarker}{}%
\end{pgfscope}%
\begin{pgfscope}%
\pgfsys@transformshift{2.482238in}{1.899438in}%
\pgfsys@useobject{currentmarker}{}%
\end{pgfscope}%
\begin{pgfscope}%
\pgfsys@transformshift{2.551608in}{1.963987in}%
\pgfsys@useobject{currentmarker}{}%
\end{pgfscope}%
\begin{pgfscope}%
\pgfsys@transformshift{2.620979in}{2.041268in}%
\pgfsys@useobject{currentmarker}{}%
\end{pgfscope}%
\begin{pgfscope}%
\pgfsys@transformshift{2.690350in}{2.133904in}%
\pgfsys@useobject{currentmarker}{}%
\end{pgfscope}%
\begin{pgfscope}%
\pgfsys@transformshift{2.759720in}{2.244670in}%
\pgfsys@useobject{currentmarker}{}%
\end{pgfscope}%
\begin{pgfscope}%
\pgfsys@transformshift{2.829091in}{2.376000in}%
\pgfsys@useobject{currentmarker}{}%
\end{pgfscope}%
\begin{pgfscope}%
\pgfsys@transformshift{2.898462in}{2.528938in}%
\pgfsys@useobject{currentmarker}{}%
\end{pgfscope}%
\begin{pgfscope}%
\pgfsys@transformshift{2.967832in}{2.701248in}%
\pgfsys@useobject{currentmarker}{}%
\end{pgfscope}%
\begin{pgfscope}%
\pgfsys@transformshift{3.037203in}{2.884624in}%
\pgfsys@useobject{currentmarker}{}%
\end{pgfscope}%
\begin{pgfscope}%
\pgfsys@transformshift{3.106573in}{3.061856in}%
\pgfsys@useobject{currentmarker}{}%
\end{pgfscope}%
\begin{pgfscope}%
\pgfsys@transformshift{3.175944in}{3.206562in}%
\pgfsys@useobject{currentmarker}{}%
\end{pgfscope}%
\begin{pgfscope}%
\pgfsys@transformshift{3.245315in}{3.289129in}%
\pgfsys@useobject{currentmarker}{}%
\end{pgfscope}%
\begin{pgfscope}%
\pgfsys@transformshift{3.314685in}{3.289129in}%
\pgfsys@useobject{currentmarker}{}%
\end{pgfscope}%
\begin{pgfscope}%
\pgfsys@transformshift{3.384056in}{3.206562in}%
\pgfsys@useobject{currentmarker}{}%
\end{pgfscope}%
\begin{pgfscope}%
\pgfsys@transformshift{3.453427in}{3.061856in}%
\pgfsys@useobject{currentmarker}{}%
\end{pgfscope}%
\begin{pgfscope}%
\pgfsys@transformshift{3.522797in}{2.884624in}%
\pgfsys@useobject{currentmarker}{}%
\end{pgfscope}%
\begin{pgfscope}%
\pgfsys@transformshift{3.592168in}{2.701248in}%
\pgfsys@useobject{currentmarker}{}%
\end{pgfscope}%
\begin{pgfscope}%
\pgfsys@transformshift{3.661538in}{2.528938in}%
\pgfsys@useobject{currentmarker}{}%
\end{pgfscope}%
\begin{pgfscope}%
\pgfsys@transformshift{3.730909in}{2.376000in}%
\pgfsys@useobject{currentmarker}{}%
\end{pgfscope}%
\begin{pgfscope}%
\pgfsys@transformshift{3.800280in}{2.244670in}%
\pgfsys@useobject{currentmarker}{}%
\end{pgfscope}%
\begin{pgfscope}%
\pgfsys@transformshift{3.869650in}{2.133904in}%
\pgfsys@useobject{currentmarker}{}%
\end{pgfscope}%
\begin{pgfscope}%
\pgfsys@transformshift{3.939021in}{2.041268in}%
\pgfsys@useobject{currentmarker}{}%
\end{pgfscope}%
\begin{pgfscope}%
\pgfsys@transformshift{4.008392in}{1.963987in}%
\pgfsys@useobject{currentmarker}{}%
\end{pgfscope}%
\begin{pgfscope}%
\pgfsys@transformshift{4.077762in}{1.899438in}%
\pgfsys@useobject{currentmarker}{}%
\end{pgfscope}%
\begin{pgfscope}%
\pgfsys@transformshift{4.147133in}{1.845340in}%
\pgfsys@useobject{currentmarker}{}%
\end{pgfscope}%
\begin{pgfscope}%
\pgfsys@transformshift{4.216503in}{1.799786in}%
\pgfsys@useobject{currentmarker}{}%
\end{pgfscope}%
\begin{pgfscope}%
\pgfsys@transformshift{4.285874in}{1.761220in}%
\pgfsys@useobject{currentmarker}{}%
\end{pgfscope}%
\begin{pgfscope}%
\pgfsys@transformshift{4.355245in}{1.728382in}%
\pgfsys@useobject{currentmarker}{}%
\end{pgfscope}%
\begin{pgfscope}%
\pgfsys@transformshift{4.424615in}{1.700261in}%
\pgfsys@useobject{currentmarker}{}%
\end{pgfscope}%
\begin{pgfscope}%
\pgfsys@transformshift{4.493986in}{1.676040in}%
\pgfsys@useobject{currentmarker}{}%
\end{pgfscope}%
\begin{pgfscope}%
\pgfsys@transformshift{4.563357in}{1.655064in}%
\pgfsys@useobject{currentmarker}{}%
\end{pgfscope}%
\begin{pgfscope}%
\pgfsys@transformshift{4.632727in}{1.636800in}%
\pgfsys@useobject{currentmarker}{}%
\end{pgfscope}%
\begin{pgfscope}%
\pgfsys@transformshift{4.702098in}{1.620817in}%
\pgfsys@useobject{currentmarker}{}%
\end{pgfscope}%
\begin{pgfscope}%
\pgfsys@transformshift{4.771469in}{1.606763in}%
\pgfsys@useobject{currentmarker}{}%
\end{pgfscope}%
\begin{pgfscope}%
\pgfsys@transformshift{4.840839in}{1.594348in}%
\pgfsys@useobject{currentmarker}{}%
\end{pgfscope}%
\begin{pgfscope}%
\pgfsys@transformshift{4.910210in}{1.583334in}%
\pgfsys@useobject{currentmarker}{}%
\end{pgfscope}%
\begin{pgfscope}%
\pgfsys@transformshift{4.979580in}{1.573522in}%
\pgfsys@useobject{currentmarker}{}%
\end{pgfscope}%
\begin{pgfscope}%
\pgfsys@transformshift{5.048951in}{1.564748in}%
\pgfsys@useobject{currentmarker}{}%
\end{pgfscope}%
\begin{pgfscope}%
\pgfsys@transformshift{5.118322in}{1.556873in}%
\pgfsys@useobject{currentmarker}{}%
\end{pgfscope}%
\begin{pgfscope}%
\pgfsys@transformshift{5.187692in}{1.549781in}%
\pgfsys@useobject{currentmarker}{}%
\end{pgfscope}%
\begin{pgfscope}%
\pgfsys@transformshift{5.257063in}{1.543373in}%
\pgfsys@useobject{currentmarker}{}%
\end{pgfscope}%
\begin{pgfscope}%
\pgfsys@transformshift{5.326434in}{1.537565in}%
\pgfsys@useobject{currentmarker}{}%
\end{pgfscope}%
\begin{pgfscope}%
\pgfsys@transformshift{5.395804in}{1.532286in}%
\pgfsys@useobject{currentmarker}{}%
\end{pgfscope}%
\begin{pgfscope}%
\pgfsys@transformshift{5.465175in}{1.527474in}%
\pgfsys@useobject{currentmarker}{}%
\end{pgfscope}%
\end{pgfscope}%
\begin{pgfscope}%
\pgfpathrectangle{\pgfqpoint{0.800000in}{0.528000in}}{\pgfqpoint{4.960000in}{3.696000in}}%
\pgfusepath{clip}%
\pgfsetrectcap%
\pgfsetroundjoin%
\pgfsetlinewidth{1.505625pt}%
\definecolor{currentstroke}{rgb}{0.121569,0.466667,0.705882}%
\pgfsetstrokecolor{currentstroke}%
\pgfsetdash{}{0pt}%
\pgfpathmoveto{\pgfqpoint{1.025455in}{1.523077in}}%
\pgfpathlineto{\pgfqpoint{1.025455in}{0.518000in}}%
\pgfpathmoveto{\pgfqpoint{1.097943in}{0.518000in}}%
\pgfpathlineto{\pgfqpoint{1.097943in}{4.234000in}}%
\pgfpathmoveto{\pgfqpoint{1.168386in}{4.234000in}}%
\pgfpathlineto{\pgfqpoint{1.168386in}{0.518000in}}%
\pgfpathmoveto{\pgfqpoint{1.238138in}{0.518000in}}%
\pgfpathlineto{\pgfqpoint{1.238140in}{4.234000in}}%
\pgfpathmoveto{\pgfqpoint{1.307525in}{4.234000in}}%
\pgfpathlineto{\pgfqpoint{1.307551in}{0.518000in}}%
\pgfpathmoveto{\pgfqpoint{1.376674in}{0.518000in}}%
\pgfpathlineto{\pgfqpoint{1.376933in}{4.234000in}}%
\pgfpathmoveto{\pgfqpoint{1.444254in}{4.234000in}}%
\pgfpathlineto{\pgfqpoint{1.446426in}{0.518000in}}%
\pgfpathmoveto{\pgfqpoint{1.510415in}{0.518000in}}%
\pgfpathlineto{\pgfqpoint{1.523947in}{3.743851in}}%
\pgfpathlineto{\pgfqpoint{1.546606in}{3.671400in}}%
\pgfpathlineto{\pgfqpoint{1.569265in}{2.123000in}}%
\pgfpathlineto{\pgfqpoint{1.591923in}{1.263874in}}%
\pgfpathlineto{\pgfqpoint{1.614582in}{1.229909in}}%
\pgfpathlineto{\pgfqpoint{1.637241in}{1.475509in}}%
\pgfpathlineto{\pgfqpoint{1.659899in}{1.634709in}}%
\pgfpathlineto{\pgfqpoint{1.682558in}{1.653598in}}%
\pgfpathlineto{\pgfqpoint{1.705217in}{1.614755in}}%
\pgfpathlineto{\pgfqpoint{1.727876in}{1.587012in}}%
\pgfpathlineto{\pgfqpoint{1.750534in}{1.586111in}}%
\pgfpathlineto{\pgfqpoint{1.795852in}{1.609859in}}%
\pgfpathlineto{\pgfqpoint{1.818511in}{1.615797in}}%
\pgfpathlineto{\pgfqpoint{1.863828in}{1.621755in}}%
\pgfpathlineto{\pgfqpoint{1.909146in}{1.632015in}}%
\pgfpathlineto{\pgfqpoint{2.022439in}{1.662451in}}%
\pgfpathlineto{\pgfqpoint{2.090416in}{1.684183in}}%
\pgfpathlineto{\pgfqpoint{2.135733in}{1.700391in}}%
\pgfpathlineto{\pgfqpoint{2.181051in}{1.718281in}}%
\pgfpathlineto{\pgfqpoint{2.226368in}{1.738067in}}%
\pgfpathlineto{\pgfqpoint{2.271686in}{1.759974in}}%
\pgfpathlineto{\pgfqpoint{2.317003in}{1.784306in}}%
\pgfpathlineto{\pgfqpoint{2.362321in}{1.811398in}}%
\pgfpathlineto{\pgfqpoint{2.407638in}{1.841630in}}%
\pgfpathlineto{\pgfqpoint{2.452956in}{1.875447in}}%
\pgfpathlineto{\pgfqpoint{2.498273in}{1.913354in}}%
\pgfpathlineto{\pgfqpoint{2.543591in}{1.955923in}}%
\pgfpathlineto{\pgfqpoint{2.588908in}{2.003797in}}%
\pgfpathlineto{\pgfqpoint{2.634226in}{2.057686in}}%
\pgfpathlineto{\pgfqpoint{2.679543in}{2.118351in}}%
\pgfpathlineto{\pgfqpoint{2.702202in}{2.151469in}}%
\pgfpathlineto{\pgfqpoint{2.724861in}{2.186574in}}%
\pgfpathlineto{\pgfqpoint{2.747519in}{2.223757in}}%
\pgfpathlineto{\pgfqpoint{2.770178in}{2.263104in}}%
\pgfpathlineto{\pgfqpoint{2.815496in}{2.348561in}}%
\pgfpathlineto{\pgfqpoint{2.860813in}{2.443286in}}%
\pgfpathlineto{\pgfqpoint{2.906131in}{2.547122in}}%
\pgfpathlineto{\pgfqpoint{2.951448in}{2.659117in}}%
\pgfpathlineto{\pgfqpoint{3.019424in}{2.837353in}}%
\pgfpathlineto{\pgfqpoint{3.087401in}{3.014864in}}%
\pgfpathlineto{\pgfqpoint{3.110059in}{3.070154in}}%
\pgfpathlineto{\pgfqpoint{3.132718in}{3.121845in}}%
\pgfpathlineto{\pgfqpoint{3.155377in}{3.168855in}}%
\pgfpathlineto{\pgfqpoint{3.178036in}{3.210099in}}%
\pgfpathlineto{\pgfqpoint{3.200694in}{3.244550in}}%
\pgfpathlineto{\pgfqpoint{3.223353in}{3.271287in}}%
\pgfpathlineto{\pgfqpoint{3.246012in}{3.289560in}}%
\pgfpathlineto{\pgfqpoint{3.268671in}{3.298834in}}%
\pgfpathlineto{\pgfqpoint{3.291329in}{3.298834in}}%
\pgfpathlineto{\pgfqpoint{3.313988in}{3.289560in}}%
\pgfpathlineto{\pgfqpoint{3.336647in}{3.271287in}}%
\pgfpathlineto{\pgfqpoint{3.359306in}{3.244550in}}%
\pgfpathlineto{\pgfqpoint{3.381964in}{3.210099in}}%
\pgfpathlineto{\pgfqpoint{3.404623in}{3.168855in}}%
\pgfpathlineto{\pgfqpoint{3.427282in}{3.121845in}}%
\pgfpathlineto{\pgfqpoint{3.449941in}{3.070154in}}%
\pgfpathlineto{\pgfqpoint{3.495258in}{2.957011in}}%
\pgfpathlineto{\pgfqpoint{3.631211in}{2.602200in}}%
\pgfpathlineto{\pgfqpoint{3.676528in}{2.494102in}}%
\pgfpathlineto{\pgfqpoint{3.721846in}{2.394759in}}%
\pgfpathlineto{\pgfqpoint{3.767163in}{2.304686in}}%
\pgfpathlineto{\pgfqpoint{3.812481in}{2.223757in}}%
\pgfpathlineto{\pgfqpoint{3.857798in}{2.151469in}}%
\pgfpathlineto{\pgfqpoint{3.903116in}{2.087123in}}%
\pgfpathlineto{\pgfqpoint{3.948433in}{2.029943in}}%
\pgfpathlineto{\pgfqpoint{3.993751in}{1.979154in}}%
\pgfpathlineto{\pgfqpoint{4.039068in}{1.934017in}}%
\pgfpathlineto{\pgfqpoint{4.084386in}{1.893855in}}%
\pgfpathlineto{\pgfqpoint{4.129703in}{1.858060in}}%
\pgfpathlineto{\pgfqpoint{4.175021in}{1.826095in}}%
\pgfpathlineto{\pgfqpoint{4.220338in}{1.797485in}}%
\pgfpathlineto{\pgfqpoint{4.265656in}{1.771817in}}%
\pgfpathlineto{\pgfqpoint{4.310973in}{1.748738in}}%
\pgfpathlineto{\pgfqpoint{4.356291in}{1.727925in}}%
\pgfpathlineto{\pgfqpoint{4.401608in}{1.709110in}}%
\pgfpathlineto{\pgfqpoint{4.446926in}{1.692096in}}%
\pgfpathlineto{\pgfqpoint{4.514902in}{1.669358in}}%
\pgfpathlineto{\pgfqpoint{4.582878in}{1.649796in}}%
\pgfpathlineto{\pgfqpoint{4.673513in}{1.626337in}}%
\pgfpathlineto{\pgfqpoint{4.696172in}{1.621755in}}%
\pgfpathlineto{\pgfqpoint{4.741489in}{1.615797in}}%
\pgfpathlineto{\pgfqpoint{4.764148in}{1.609859in}}%
\pgfpathlineto{\pgfqpoint{4.809466in}{1.586111in}}%
\pgfpathlineto{\pgfqpoint{4.832124in}{1.587012in}}%
\pgfpathlineto{\pgfqpoint{4.854783in}{1.614755in}}%
\pgfpathlineto{\pgfqpoint{4.877442in}{1.653598in}}%
\pgfpathlineto{\pgfqpoint{4.900101in}{1.634709in}}%
\pgfpathlineto{\pgfqpoint{4.922759in}{1.475509in}}%
\pgfpathlineto{\pgfqpoint{4.945418in}{1.229909in}}%
\pgfpathlineto{\pgfqpoint{4.968077in}{1.263874in}}%
\pgfpathlineto{\pgfqpoint{4.990735in}{2.123000in}}%
\pgfpathlineto{\pgfqpoint{5.013394in}{3.671400in}}%
\pgfpathlineto{\pgfqpoint{5.036053in}{3.743851in}}%
\pgfpathlineto{\pgfqpoint{5.049585in}{0.518000in}}%
\pgfpathmoveto{\pgfqpoint{5.113574in}{0.518000in}}%
\pgfpathlineto{\pgfqpoint{5.115746in}{4.234000in}}%
\pgfpathmoveto{\pgfqpoint{5.183067in}{4.234000in}}%
\pgfpathlineto{\pgfqpoint{5.183326in}{0.518000in}}%
\pgfpathmoveto{\pgfqpoint{5.252449in}{0.518000in}}%
\pgfpathlineto{\pgfqpoint{5.252475in}{4.234000in}}%
\pgfpathmoveto{\pgfqpoint{5.321860in}{4.234000in}}%
\pgfpathlineto{\pgfqpoint{5.321862in}{0.518000in}}%
\pgfpathmoveto{\pgfqpoint{5.391614in}{0.518000in}}%
\pgfpathlineto{\pgfqpoint{5.391614in}{4.234000in}}%
\pgfpathmoveto{\pgfqpoint{5.462057in}{4.234000in}}%
\pgfpathlineto{\pgfqpoint{5.462057in}{0.518000in}}%
\pgfpathlineto{\pgfqpoint{5.462057in}{0.518000in}}%
\pgfusepath{stroke}%
\end{pgfscope}%
\begin{pgfscope}%
\pgfpathrectangle{\pgfqpoint{0.800000in}{0.528000in}}{\pgfqpoint{4.960000in}{3.696000in}}%
\pgfusepath{clip}%
\pgfsetrectcap%
\pgfsetroundjoin%
\pgfsetlinewidth{1.505625pt}%
\definecolor{currentstroke}{rgb}{1.000000,0.498039,0.054902}%
\pgfsetstrokecolor{currentstroke}%
\pgfsetdash{}{0pt}%
\pgfpathmoveto{\pgfqpoint{1.025455in}{1.523077in}}%
\pgfpathlineto{\pgfqpoint{1.184066in}{1.533748in}}%
\pgfpathlineto{\pgfqpoint{1.320018in}{1.544892in}}%
\pgfpathlineto{\pgfqpoint{1.455971in}{1.558428in}}%
\pgfpathlineto{\pgfqpoint{1.569265in}{1.572045in}}%
\pgfpathlineto{\pgfqpoint{1.659899in}{1.584860in}}%
\pgfpathlineto{\pgfqpoint{1.750534in}{1.599776in}}%
\pgfpathlineto{\pgfqpoint{1.841169in}{1.617263in}}%
\pgfpathlineto{\pgfqpoint{1.909146in}{1.632419in}}%
\pgfpathlineto{\pgfqpoint{1.977122in}{1.649670in}}%
\pgfpathlineto{\pgfqpoint{2.045098in}{1.669400in}}%
\pgfpathlineto{\pgfqpoint{2.113074in}{1.692080in}}%
\pgfpathlineto{\pgfqpoint{2.158392in}{1.709119in}}%
\pgfpathlineto{\pgfqpoint{2.203709in}{1.727926in}}%
\pgfpathlineto{\pgfqpoint{2.249027in}{1.748735in}}%
\pgfpathlineto{\pgfqpoint{2.294344in}{1.771818in}}%
\pgfpathlineto{\pgfqpoint{2.339662in}{1.797485in}}%
\pgfpathlineto{\pgfqpoint{2.384979in}{1.826095in}}%
\pgfpathlineto{\pgfqpoint{2.430297in}{1.858060in}}%
\pgfpathlineto{\pgfqpoint{2.475614in}{1.893855in}}%
\pgfpathlineto{\pgfqpoint{2.520932in}{1.934017in}}%
\pgfpathlineto{\pgfqpoint{2.566249in}{1.979154in}}%
\pgfpathlineto{\pgfqpoint{2.611567in}{2.029943in}}%
\pgfpathlineto{\pgfqpoint{2.656884in}{2.087123in}}%
\pgfpathlineto{\pgfqpoint{2.702202in}{2.151469in}}%
\pgfpathlineto{\pgfqpoint{2.724861in}{2.186574in}}%
\pgfpathlineto{\pgfqpoint{2.747519in}{2.223757in}}%
\pgfpathlineto{\pgfqpoint{2.770178in}{2.263104in}}%
\pgfpathlineto{\pgfqpoint{2.815496in}{2.348561in}}%
\pgfpathlineto{\pgfqpoint{2.860813in}{2.443286in}}%
\pgfpathlineto{\pgfqpoint{2.906131in}{2.547122in}}%
\pgfpathlineto{\pgfqpoint{2.951448in}{2.659117in}}%
\pgfpathlineto{\pgfqpoint{3.019424in}{2.837353in}}%
\pgfpathlineto{\pgfqpoint{3.087401in}{3.014864in}}%
\pgfpathlineto{\pgfqpoint{3.110059in}{3.070154in}}%
\pgfpathlineto{\pgfqpoint{3.132718in}{3.121845in}}%
\pgfpathlineto{\pgfqpoint{3.155377in}{3.168855in}}%
\pgfpathlineto{\pgfqpoint{3.178036in}{3.210099in}}%
\pgfpathlineto{\pgfqpoint{3.200694in}{3.244550in}}%
\pgfpathlineto{\pgfqpoint{3.223353in}{3.271287in}}%
\pgfpathlineto{\pgfqpoint{3.246012in}{3.289560in}}%
\pgfpathlineto{\pgfqpoint{3.268671in}{3.298834in}}%
\pgfpathlineto{\pgfqpoint{3.291329in}{3.298834in}}%
\pgfpathlineto{\pgfqpoint{3.313988in}{3.289560in}}%
\pgfpathlineto{\pgfqpoint{3.336647in}{3.271287in}}%
\pgfpathlineto{\pgfqpoint{3.359306in}{3.244550in}}%
\pgfpathlineto{\pgfqpoint{3.381964in}{3.210099in}}%
\pgfpathlineto{\pgfqpoint{3.404623in}{3.168855in}}%
\pgfpathlineto{\pgfqpoint{3.427282in}{3.121845in}}%
\pgfpathlineto{\pgfqpoint{3.449941in}{3.070154in}}%
\pgfpathlineto{\pgfqpoint{3.495258in}{2.957011in}}%
\pgfpathlineto{\pgfqpoint{3.631211in}{2.602200in}}%
\pgfpathlineto{\pgfqpoint{3.676528in}{2.494102in}}%
\pgfpathlineto{\pgfqpoint{3.721846in}{2.394759in}}%
\pgfpathlineto{\pgfqpoint{3.767163in}{2.304686in}}%
\pgfpathlineto{\pgfqpoint{3.812481in}{2.223757in}}%
\pgfpathlineto{\pgfqpoint{3.857798in}{2.151469in}}%
\pgfpathlineto{\pgfqpoint{3.903116in}{2.087123in}}%
\pgfpathlineto{\pgfqpoint{3.948433in}{2.029943in}}%
\pgfpathlineto{\pgfqpoint{3.993751in}{1.979154in}}%
\pgfpathlineto{\pgfqpoint{4.039068in}{1.934017in}}%
\pgfpathlineto{\pgfqpoint{4.084386in}{1.893855in}}%
\pgfpathlineto{\pgfqpoint{4.129703in}{1.858060in}}%
\pgfpathlineto{\pgfqpoint{4.175021in}{1.826095in}}%
\pgfpathlineto{\pgfqpoint{4.220338in}{1.797485in}}%
\pgfpathlineto{\pgfqpoint{4.265656in}{1.771818in}}%
\pgfpathlineto{\pgfqpoint{4.310973in}{1.748735in}}%
\pgfpathlineto{\pgfqpoint{4.356291in}{1.727926in}}%
\pgfpathlineto{\pgfqpoint{4.401608in}{1.709119in}}%
\pgfpathlineto{\pgfqpoint{4.446926in}{1.692080in}}%
\pgfpathlineto{\pgfqpoint{4.514902in}{1.669400in}}%
\pgfpathlineto{\pgfqpoint{4.582878in}{1.649670in}}%
\pgfpathlineto{\pgfqpoint{4.650854in}{1.632419in}}%
\pgfpathlineto{\pgfqpoint{4.718831in}{1.617263in}}%
\pgfpathlineto{\pgfqpoint{4.809466in}{1.599776in}}%
\pgfpathlineto{\pgfqpoint{4.900101in}{1.584860in}}%
\pgfpathlineto{\pgfqpoint{4.990735in}{1.572045in}}%
\pgfpathlineto{\pgfqpoint{5.104029in}{1.558428in}}%
\pgfpathlineto{\pgfqpoint{5.217323in}{1.546965in}}%
\pgfpathlineto{\pgfqpoint{5.353275in}{1.535463in}}%
\pgfpathlineto{\pgfqpoint{5.511887in}{1.524471in}}%
\pgfpathlineto{\pgfqpoint{5.534545in}{1.523077in}}%
\pgfpathlineto{\pgfqpoint{5.534545in}{1.523077in}}%
\pgfusepath{stroke}%
\end{pgfscope}%
\begin{pgfscope}%
\pgfpathrectangle{\pgfqpoint{0.800000in}{0.528000in}}{\pgfqpoint{4.960000in}{3.696000in}}%
\pgfusepath{clip}%
\pgfsetbuttcap%
\pgfsetroundjoin%
\definecolor{currentfill}{rgb}{0.172549,0.627451,0.172549}%
\pgfsetfillcolor{currentfill}%
\pgfsetlinewidth{1.003750pt}%
\definecolor{currentstroke}{rgb}{0.172549,0.627451,0.172549}%
\pgfsetstrokecolor{currentstroke}%
\pgfsetdash{}{0pt}%
\pgfsys@defobject{currentmarker}{\pgfqpoint{-0.020833in}{-0.020833in}}{\pgfqpoint{0.020833in}{0.020833in}}{%
\pgfpathmoveto{\pgfqpoint{0.000000in}{-0.020833in}}%
\pgfpathcurveto{\pgfqpoint{0.005525in}{-0.020833in}}{\pgfqpoint{0.010825in}{-0.018638in}}{\pgfqpoint{0.014731in}{-0.014731in}}%
\pgfpathcurveto{\pgfqpoint{0.018638in}{-0.010825in}}{\pgfqpoint{0.020833in}{-0.005525in}}{\pgfqpoint{0.020833in}{0.000000in}}%
\pgfpathcurveto{\pgfqpoint{0.020833in}{0.005525in}}{\pgfqpoint{0.018638in}{0.010825in}}{\pgfqpoint{0.014731in}{0.014731in}}%
\pgfpathcurveto{\pgfqpoint{0.010825in}{0.018638in}}{\pgfqpoint{0.005525in}{0.020833in}}{\pgfqpoint{0.000000in}{0.020833in}}%
\pgfpathcurveto{\pgfqpoint{-0.005525in}{0.020833in}}{\pgfqpoint{-0.010825in}{0.018638in}}{\pgfqpoint{-0.014731in}{0.014731in}}%
\pgfpathcurveto{\pgfqpoint{-0.018638in}{0.010825in}}{\pgfqpoint{-0.020833in}{0.005525in}}{\pgfqpoint{-0.020833in}{0.000000in}}%
\pgfpathcurveto{\pgfqpoint{-0.020833in}{-0.005525in}}{\pgfqpoint{-0.018638in}{-0.010825in}}{\pgfqpoint{-0.014731in}{-0.014731in}}%
\pgfpathcurveto{\pgfqpoint{-0.010825in}{-0.018638in}}{\pgfqpoint{-0.005525in}{-0.020833in}}{\pgfqpoint{0.000000in}{-0.020833in}}%
\pgfpathclose%
\pgfusepath{stroke,fill}%
}%
\begin{pgfscope}%
\pgfsys@transformshift{5.478019in}{1.526630in}%
\pgfsys@useobject{currentmarker}{}%
\end{pgfscope}%
\begin{pgfscope}%
\pgfsys@transformshift{5.042675in}{1.565503in}%
\pgfsys@useobject{currentmarker}{}%
\end{pgfscope}%
\begin{pgfscope}%
\pgfsys@transformshift{4.258211in}{1.775848in}%
\pgfsys@useobject{currentmarker}{}%
\end{pgfscope}%
\begin{pgfscope}%
\pgfsys@transformshift{3.280000in}{3.300000in}%
\pgfsys@useobject{currentmarker}{}%
\end{pgfscope}%
\begin{pgfscope}%
\pgfsys@transformshift{2.301789in}{1.775848in}%
\pgfsys@useobject{currentmarker}{}%
\end{pgfscope}%
\begin{pgfscope}%
\pgfsys@transformshift{1.517325in}{1.565503in}%
\pgfsys@useobject{currentmarker}{}%
\end{pgfscope}%
\begin{pgfscope}%
\pgfsys@transformshift{1.081981in}{1.526630in}%
\pgfsys@useobject{currentmarker}{}%
\end{pgfscope}%
\end{pgfscope}%
\begin{pgfscope}%
\pgfpathrectangle{\pgfqpoint{0.800000in}{0.528000in}}{\pgfqpoint{4.960000in}{3.696000in}}%
\pgfusepath{clip}%
\pgfsetrectcap%
\pgfsetroundjoin%
\pgfsetlinewidth{1.505625pt}%
\definecolor{currentstroke}{rgb}{0.839216,0.152941,0.156863}%
\pgfsetstrokecolor{currentstroke}%
\pgfsetdash{}{0pt}%
\pgfpathmoveto{\pgfqpoint{1.025455in}{1.334510in}}%
\pgfpathlineto{\pgfqpoint{1.048113in}{1.420558in}}%
\pgfpathlineto{\pgfqpoint{1.070772in}{1.494362in}}%
\pgfpathlineto{\pgfqpoint{1.093431in}{1.556846in}}%
\pgfpathlineto{\pgfqpoint{1.116090in}{1.608890in}}%
\pgfpathlineto{\pgfqpoint{1.138748in}{1.651342in}}%
\pgfpathlineto{\pgfqpoint{1.161407in}{1.685009in}}%
\pgfpathlineto{\pgfqpoint{1.184066in}{1.710664in}}%
\pgfpathlineto{\pgfqpoint{1.206725in}{1.729043in}}%
\pgfpathlineto{\pgfqpoint{1.229383in}{1.740848in}}%
\pgfpathlineto{\pgfqpoint{1.252042in}{1.746750in}}%
\pgfpathlineto{\pgfqpoint{1.274701in}{1.747384in}}%
\pgfpathlineto{\pgfqpoint{1.297360in}{1.743355in}}%
\pgfpathlineto{\pgfqpoint{1.320018in}{1.735238in}}%
\pgfpathlineto{\pgfqpoint{1.342677in}{1.723574in}}%
\pgfpathlineto{\pgfqpoint{1.365336in}{1.708880in}}%
\pgfpathlineto{\pgfqpoint{1.387995in}{1.691640in}}%
\pgfpathlineto{\pgfqpoint{1.410653in}{1.672312in}}%
\pgfpathlineto{\pgfqpoint{1.455971in}{1.629092in}}%
\pgfpathlineto{\pgfqpoint{1.523947in}{1.558543in}}%
\pgfpathlineto{\pgfqpoint{1.591923in}{1.488944in}}%
\pgfpathlineto{\pgfqpoint{1.637241in}{1.446662in}}%
\pgfpathlineto{\pgfqpoint{1.659899in}{1.427424in}}%
\pgfpathlineto{\pgfqpoint{1.682558in}{1.409704in}}%
\pgfpathlineto{\pgfqpoint{1.705217in}{1.393668in}}%
\pgfpathlineto{\pgfqpoint{1.727876in}{1.379463in}}%
\pgfpathlineto{\pgfqpoint{1.750534in}{1.367219in}}%
\pgfpathlineto{\pgfqpoint{1.773193in}{1.357052in}}%
\pgfpathlineto{\pgfqpoint{1.795852in}{1.349058in}}%
\pgfpathlineto{\pgfqpoint{1.818511in}{1.343321in}}%
\pgfpathlineto{\pgfqpoint{1.841169in}{1.339911in}}%
\pgfpathlineto{\pgfqpoint{1.863828in}{1.338881in}}%
\pgfpathlineto{\pgfqpoint{1.886487in}{1.340273in}}%
\pgfpathlineto{\pgfqpoint{1.909146in}{1.344117in}}%
\pgfpathlineto{\pgfqpoint{1.931804in}{1.350429in}}%
\pgfpathlineto{\pgfqpoint{1.954463in}{1.359214in}}%
\pgfpathlineto{\pgfqpoint{1.977122in}{1.370467in}}%
\pgfpathlineto{\pgfqpoint{1.999781in}{1.384171in}}%
\pgfpathlineto{\pgfqpoint{2.022439in}{1.400301in}}%
\pgfpathlineto{\pgfqpoint{2.045098in}{1.418821in}}%
\pgfpathlineto{\pgfqpoint{2.067757in}{1.439687in}}%
\pgfpathlineto{\pgfqpoint{2.090416in}{1.462845in}}%
\pgfpathlineto{\pgfqpoint{2.113074in}{1.488237in}}%
\pgfpathlineto{\pgfqpoint{2.135733in}{1.515793in}}%
\pgfpathlineto{\pgfqpoint{2.158392in}{1.545440in}}%
\pgfpathlineto{\pgfqpoint{2.181051in}{1.577096in}}%
\pgfpathlineto{\pgfqpoint{2.226368in}{1.646083in}}%
\pgfpathlineto{\pgfqpoint{2.271686in}{1.721997in}}%
\pgfpathlineto{\pgfqpoint{2.317003in}{1.804006in}}%
\pgfpathlineto{\pgfqpoint{2.362321in}{1.891219in}}%
\pgfpathlineto{\pgfqpoint{2.407638in}{1.982700in}}%
\pgfpathlineto{\pgfqpoint{2.475614in}{2.125791in}}%
\pgfpathlineto{\pgfqpoint{2.702202in}{2.613028in}}%
\pgfpathlineto{\pgfqpoint{2.747519in}{2.704999in}}%
\pgfpathlineto{\pgfqpoint{2.792837in}{2.792976in}}%
\pgfpathlineto{\pgfqpoint{2.838154in}{2.876112in}}%
\pgfpathlineto{\pgfqpoint{2.883472in}{2.953611in}}%
\pgfpathlineto{\pgfqpoint{2.928789in}{3.024739in}}%
\pgfpathlineto{\pgfqpoint{2.974107in}{3.088826in}}%
\pgfpathlineto{\pgfqpoint{2.996766in}{3.118039in}}%
\pgfpathlineto{\pgfqpoint{3.019424in}{3.145272in}}%
\pgfpathlineto{\pgfqpoint{3.042083in}{3.170463in}}%
\pgfpathlineto{\pgfqpoint{3.064742in}{3.193553in}}%
\pgfpathlineto{\pgfqpoint{3.087401in}{3.214488in}}%
\pgfpathlineto{\pgfqpoint{3.110059in}{3.233220in}}%
\pgfpathlineto{\pgfqpoint{3.132718in}{3.249706in}}%
\pgfpathlineto{\pgfqpoint{3.155377in}{3.263908in}}%
\pgfpathlineto{\pgfqpoint{3.178036in}{3.275793in}}%
\pgfpathlineto{\pgfqpoint{3.200694in}{3.285334in}}%
\pgfpathlineto{\pgfqpoint{3.223353in}{3.292509in}}%
\pgfpathlineto{\pgfqpoint{3.246012in}{3.297301in}}%
\pgfpathlineto{\pgfqpoint{3.268671in}{3.299700in}}%
\pgfpathlineto{\pgfqpoint{3.291329in}{3.299700in}}%
\pgfpathlineto{\pgfqpoint{3.313988in}{3.297301in}}%
\pgfpathlineto{\pgfqpoint{3.336647in}{3.292509in}}%
\pgfpathlineto{\pgfqpoint{3.359306in}{3.285334in}}%
\pgfpathlineto{\pgfqpoint{3.381964in}{3.275793in}}%
\pgfpathlineto{\pgfqpoint{3.404623in}{3.263908in}}%
\pgfpathlineto{\pgfqpoint{3.427282in}{3.249706in}}%
\pgfpathlineto{\pgfqpoint{3.449941in}{3.233220in}}%
\pgfpathlineto{\pgfqpoint{3.472599in}{3.214488in}}%
\pgfpathlineto{\pgfqpoint{3.495258in}{3.193553in}}%
\pgfpathlineto{\pgfqpoint{3.517917in}{3.170463in}}%
\pgfpathlineto{\pgfqpoint{3.540576in}{3.145272in}}%
\pgfpathlineto{\pgfqpoint{3.563234in}{3.118039in}}%
\pgfpathlineto{\pgfqpoint{3.585893in}{3.088826in}}%
\pgfpathlineto{\pgfqpoint{3.631211in}{3.024739in}}%
\pgfpathlineto{\pgfqpoint{3.676528in}{2.953611in}}%
\pgfpathlineto{\pgfqpoint{3.721846in}{2.876112in}}%
\pgfpathlineto{\pgfqpoint{3.767163in}{2.792976in}}%
\pgfpathlineto{\pgfqpoint{3.812481in}{2.704999in}}%
\pgfpathlineto{\pgfqpoint{3.857798in}{2.613028in}}%
\pgfpathlineto{\pgfqpoint{3.925774in}{2.469551in}}%
\pgfpathlineto{\pgfqpoint{4.152362in}{1.982700in}}%
\pgfpathlineto{\pgfqpoint{4.197679in}{1.891219in}}%
\pgfpathlineto{\pgfqpoint{4.242997in}{1.804006in}}%
\pgfpathlineto{\pgfqpoint{4.288314in}{1.721997in}}%
\pgfpathlineto{\pgfqpoint{4.333632in}{1.646083in}}%
\pgfpathlineto{\pgfqpoint{4.378949in}{1.577096in}}%
\pgfpathlineto{\pgfqpoint{4.401608in}{1.545440in}}%
\pgfpathlineto{\pgfqpoint{4.424267in}{1.515793in}}%
\pgfpathlineto{\pgfqpoint{4.446926in}{1.488237in}}%
\pgfpathlineto{\pgfqpoint{4.469584in}{1.462845in}}%
\pgfpathlineto{\pgfqpoint{4.492243in}{1.439687in}}%
\pgfpathlineto{\pgfqpoint{4.514902in}{1.418821in}}%
\pgfpathlineto{\pgfqpoint{4.537561in}{1.400301in}}%
\pgfpathlineto{\pgfqpoint{4.560219in}{1.384171in}}%
\pgfpathlineto{\pgfqpoint{4.582878in}{1.370467in}}%
\pgfpathlineto{\pgfqpoint{4.605537in}{1.359214in}}%
\pgfpathlineto{\pgfqpoint{4.628196in}{1.350429in}}%
\pgfpathlineto{\pgfqpoint{4.650854in}{1.344117in}}%
\pgfpathlineto{\pgfqpoint{4.673513in}{1.340273in}}%
\pgfpathlineto{\pgfqpoint{4.696172in}{1.338881in}}%
\pgfpathlineto{\pgfqpoint{4.718831in}{1.339911in}}%
\pgfpathlineto{\pgfqpoint{4.741489in}{1.343321in}}%
\pgfpathlineto{\pgfqpoint{4.764148in}{1.349058in}}%
\pgfpathlineto{\pgfqpoint{4.786807in}{1.357052in}}%
\pgfpathlineto{\pgfqpoint{4.809466in}{1.367219in}}%
\pgfpathlineto{\pgfqpoint{4.832124in}{1.379463in}}%
\pgfpathlineto{\pgfqpoint{4.854783in}{1.393668in}}%
\pgfpathlineto{\pgfqpoint{4.877442in}{1.409704in}}%
\pgfpathlineto{\pgfqpoint{4.900101in}{1.427424in}}%
\pgfpathlineto{\pgfqpoint{4.945418in}{1.467237in}}%
\pgfpathlineto{\pgfqpoint{4.990735in}{1.511562in}}%
\pgfpathlineto{\pgfqpoint{5.126688in}{1.651328in}}%
\pgfpathlineto{\pgfqpoint{5.149347in}{1.672312in}}%
\pgfpathlineto{\pgfqpoint{5.172005in}{1.691640in}}%
\pgfpathlineto{\pgfqpoint{5.194664in}{1.708880in}}%
\pgfpathlineto{\pgfqpoint{5.217323in}{1.723574in}}%
\pgfpathlineto{\pgfqpoint{5.239982in}{1.735238in}}%
\pgfpathlineto{\pgfqpoint{5.262640in}{1.743355in}}%
\pgfpathlineto{\pgfqpoint{5.285299in}{1.747384in}}%
\pgfpathlineto{\pgfqpoint{5.307958in}{1.746750in}}%
\pgfpathlineto{\pgfqpoint{5.330617in}{1.740848in}}%
\pgfpathlineto{\pgfqpoint{5.353275in}{1.729043in}}%
\pgfpathlineto{\pgfqpoint{5.375934in}{1.710664in}}%
\pgfpathlineto{\pgfqpoint{5.398593in}{1.685009in}}%
\pgfpathlineto{\pgfqpoint{5.421252in}{1.651342in}}%
\pgfpathlineto{\pgfqpoint{5.443910in}{1.608890in}}%
\pgfpathlineto{\pgfqpoint{5.466569in}{1.556846in}}%
\pgfpathlineto{\pgfqpoint{5.489228in}{1.494362in}}%
\pgfpathlineto{\pgfqpoint{5.511887in}{1.420558in}}%
\pgfpathlineto{\pgfqpoint{5.534545in}{1.334510in}}%
\pgfpathlineto{\pgfqpoint{5.534545in}{1.334510in}}%
\pgfusepath{stroke}%
\end{pgfscope}%
\begin{pgfscope}%
\pgfpathrectangle{\pgfqpoint{0.800000in}{0.528000in}}{\pgfqpoint{4.960000in}{3.696000in}}%
\pgfusepath{clip}%
\pgfsetbuttcap%
\pgfsetroundjoin%
\definecolor{currentfill}{rgb}{0.580392,0.403922,0.741176}%
\pgfsetfillcolor{currentfill}%
\pgfsetlinewidth{1.003750pt}%
\definecolor{currentstroke}{rgb}{0.580392,0.403922,0.741176}%
\pgfsetstrokecolor{currentstroke}%
\pgfsetdash{}{0pt}%
\pgfsys@defobject{currentmarker}{\pgfqpoint{-0.020833in}{-0.020833in}}{\pgfqpoint{0.020833in}{0.020833in}}{%
\pgfpathmoveto{\pgfqpoint{0.000000in}{-0.020833in}}%
\pgfpathcurveto{\pgfqpoint{0.005525in}{-0.020833in}}{\pgfqpoint{0.010825in}{-0.018638in}}{\pgfqpoint{0.014731in}{-0.014731in}}%
\pgfpathcurveto{\pgfqpoint{0.018638in}{-0.010825in}}{\pgfqpoint{0.020833in}{-0.005525in}}{\pgfqpoint{0.020833in}{0.000000in}}%
\pgfpathcurveto{\pgfqpoint{0.020833in}{0.005525in}}{\pgfqpoint{0.018638in}{0.010825in}}{\pgfqpoint{0.014731in}{0.014731in}}%
\pgfpathcurveto{\pgfqpoint{0.010825in}{0.018638in}}{\pgfqpoint{0.005525in}{0.020833in}}{\pgfqpoint{0.000000in}{0.020833in}}%
\pgfpathcurveto{\pgfqpoint{-0.005525in}{0.020833in}}{\pgfqpoint{-0.010825in}{0.018638in}}{\pgfqpoint{-0.014731in}{0.014731in}}%
\pgfpathcurveto{\pgfqpoint{-0.018638in}{0.010825in}}{\pgfqpoint{-0.020833in}{0.005525in}}{\pgfqpoint{-0.020833in}{0.000000in}}%
\pgfpathcurveto{\pgfqpoint{-0.020833in}{-0.005525in}}{\pgfqpoint{-0.018638in}{-0.010825in}}{\pgfqpoint{-0.014731in}{-0.014731in}}%
\pgfpathcurveto{\pgfqpoint{-0.010825in}{-0.018638in}}{\pgfqpoint{-0.005525in}{-0.020833in}}{\pgfqpoint{0.000000in}{-0.020833in}}%
\pgfpathclose%
\pgfusepath{stroke,fill}%
}%
\begin{pgfscope}%
\pgfsys@transformshift{5.533887in}{1.523117in}%
\pgfsys@useobject{currentmarker}{}%
\end{pgfscope}%
\begin{pgfscope}%
\pgfsys@transformshift{5.528623in}{1.523437in}%
\pgfsys@useobject{currentmarker}{}%
\end{pgfscope}%
\begin{pgfscope}%
\pgfsys@transformshift{5.518107in}{1.524084in}%
\pgfsys@useobject{currentmarker}{}%
\end{pgfscope}%
\begin{pgfscope}%
\pgfsys@transformshift{5.502364in}{1.525068in}%
\pgfsys@useobject{currentmarker}{}%
\end{pgfscope}%
\begin{pgfscope}%
\pgfsys@transformshift{5.481431in}{1.526408in}%
\pgfsys@useobject{currentmarker}{}%
\end{pgfscope}%
\begin{pgfscope}%
\pgfsys@transformshift{5.455356in}{1.528129in}%
\pgfsys@useobject{currentmarker}{}%
\end{pgfscope}%
\begin{pgfscope}%
\pgfsys@transformshift{5.424200in}{1.530263in}%
\pgfsys@useobject{currentmarker}{}%
\end{pgfscope}%
\begin{pgfscope}%
\pgfsys@transformshift{5.388037in}{1.532853in}%
\pgfsys@useobject{currentmarker}{}%
\end{pgfscope}%
\begin{pgfscope}%
\pgfsys@transformshift{5.346950in}{1.535952in}%
\pgfsys@useobject{currentmarker}{}%
\end{pgfscope}%
\begin{pgfscope}%
\pgfsys@transformshift{5.301035in}{1.539626in}%
\pgfsys@useobject{currentmarker}{}%
\end{pgfscope}%
\begin{pgfscope}%
\pgfsys@transformshift{5.250401in}{1.543961in}%
\pgfsys@useobject{currentmarker}{}%
\end{pgfscope}%
\begin{pgfscope}%
\pgfsys@transformshift{5.195164in}{1.549059in}%
\pgfsys@useobject{currentmarker}{}%
\end{pgfscope}%
\begin{pgfscope}%
\pgfsys@transformshift{5.135455in}{1.555053in}%
\pgfsys@useobject{currentmarker}{}%
\end{pgfscope}%
\begin{pgfscope}%
\pgfsys@transformshift{5.071411in}{1.562106in}%
\pgfsys@useobject{currentmarker}{}%
\end{pgfscope}%
\begin{pgfscope}%
\pgfsys@transformshift{5.003185in}{1.570428in}%
\pgfsys@useobject{currentmarker}{}%
\end{pgfscope}%
\begin{pgfscope}%
\pgfsys@transformshift{4.930933in}{1.580285in}%
\pgfsys@useobject{currentmarker}{}%
\end{pgfscope}%
\begin{pgfscope}%
\pgfsys@transformshift{4.854826in}{1.592022in}%
\pgfsys@useobject{currentmarker}{}%
\end{pgfscope}%
\begin{pgfscope}%
\pgfsys@transformshift{4.775040in}{1.606086in}%
\pgfsys@useobject{currentmarker}{}%
\end{pgfscope}%
\begin{pgfscope}%
\pgfsys@transformshift{4.691763in}{1.623068in}%
\pgfsys@useobject{currentmarker}{}%
\end{pgfscope}%
\begin{pgfscope}%
\pgfsys@transformshift{4.605189in}{1.643755in}%
\pgfsys@useobject{currentmarker}{}%
\end{pgfscope}%
\begin{pgfscope}%
\pgfsys@transformshift{4.515519in}{1.669209in}%
\pgfsys@useobject{currentmarker}{}%
\end{pgfscope}%
\begin{pgfscope}%
\pgfsys@transformshift{4.422964in}{1.700882in}%
\pgfsys@useobject{currentmarker}{}%
\end{pgfscope}%
\begin{pgfscope}%
\pgfsys@transformshift{4.327740in}{1.740787in}%
\pgfsys@useobject{currentmarker}{}%
\end{pgfscope}%
\begin{pgfscope}%
\pgfsys@transformshift{4.230068in}{1.791739in}%
\pgfsys@useobject{currentmarker}{}%
\end{pgfscope}%
\begin{pgfscope}%
\pgfsys@transformshift{4.130177in}{1.857707in}%
\pgfsys@useobject{currentmarker}{}%
\end{pgfscope}%
\begin{pgfscope}%
\pgfsys@transformshift{4.028301in}{1.944266in}%
\pgfsys@useobject{currentmarker}{}%
\end{pgfscope}%
\begin{pgfscope}%
\pgfsys@transformshift{3.924678in}{2.059071in}%
\pgfsys@useobject{currentmarker}{}%
\end{pgfscope}%
\begin{pgfscope}%
\pgfsys@transformshift{3.819548in}{2.211931in}%
\pgfsys@useobject{currentmarker}{}%
\end{pgfscope}%
\begin{pgfscope}%
\pgfsys@transformshift{3.713158in}{2.413090in}%
\pgfsys@useobject{currentmarker}{}%
\end{pgfscope}%
\begin{pgfscope}%
\pgfsys@transformshift{3.605757in}{2.666250in}%
\pgfsys@useobject{currentmarker}{}%
\end{pgfscope}%
\begin{pgfscope}%
\pgfsys@transformshift{3.497595in}{2.950937in}%
\pgfsys@useobject{currentmarker}{}%
\end{pgfscope}%
\begin{pgfscope}%
\pgfsys@transformshift{3.388925in}{3.198107in}%
\pgfsys@useobject{currentmarker}{}%
\end{pgfscope}%
\begin{pgfscope}%
\pgfsys@transformshift{3.280000in}{3.300000in}%
\pgfsys@useobject{currentmarker}{}%
\end{pgfscope}%
\begin{pgfscope}%
\pgfsys@transformshift{3.171075in}{3.198107in}%
\pgfsys@useobject{currentmarker}{}%
\end{pgfscope}%
\begin{pgfscope}%
\pgfsys@transformshift{3.062405in}{2.950937in}%
\pgfsys@useobject{currentmarker}{}%
\end{pgfscope}%
\begin{pgfscope}%
\pgfsys@transformshift{2.954243in}{2.666250in}%
\pgfsys@useobject{currentmarker}{}%
\end{pgfscope}%
\begin{pgfscope}%
\pgfsys@transformshift{2.846842in}{2.413090in}%
\pgfsys@useobject{currentmarker}{}%
\end{pgfscope}%
\begin{pgfscope}%
\pgfsys@transformshift{2.740452in}{2.211931in}%
\pgfsys@useobject{currentmarker}{}%
\end{pgfscope}%
\begin{pgfscope}%
\pgfsys@transformshift{2.635322in}{2.059071in}%
\pgfsys@useobject{currentmarker}{}%
\end{pgfscope}%
\begin{pgfscope}%
\pgfsys@transformshift{2.531699in}{1.944266in}%
\pgfsys@useobject{currentmarker}{}%
\end{pgfscope}%
\begin{pgfscope}%
\pgfsys@transformshift{2.429823in}{1.857707in}%
\pgfsys@useobject{currentmarker}{}%
\end{pgfscope}%
\begin{pgfscope}%
\pgfsys@transformshift{2.329932in}{1.791739in}%
\pgfsys@useobject{currentmarker}{}%
\end{pgfscope}%
\begin{pgfscope}%
\pgfsys@transformshift{2.232260in}{1.740787in}%
\pgfsys@useobject{currentmarker}{}%
\end{pgfscope}%
\begin{pgfscope}%
\pgfsys@transformshift{2.137036in}{1.700882in}%
\pgfsys@useobject{currentmarker}{}%
\end{pgfscope}%
\begin{pgfscope}%
\pgfsys@transformshift{2.044481in}{1.669209in}%
\pgfsys@useobject{currentmarker}{}%
\end{pgfscope}%
\begin{pgfscope}%
\pgfsys@transformshift{1.954811in}{1.643755in}%
\pgfsys@useobject{currentmarker}{}%
\end{pgfscope}%
\begin{pgfscope}%
\pgfsys@transformshift{1.868237in}{1.623068in}%
\pgfsys@useobject{currentmarker}{}%
\end{pgfscope}%
\begin{pgfscope}%
\pgfsys@transformshift{1.784960in}{1.606086in}%
\pgfsys@useobject{currentmarker}{}%
\end{pgfscope}%
\begin{pgfscope}%
\pgfsys@transformshift{1.705174in}{1.592022in}%
\pgfsys@useobject{currentmarker}{}%
\end{pgfscope}%
\begin{pgfscope}%
\pgfsys@transformshift{1.629067in}{1.580285in}%
\pgfsys@useobject{currentmarker}{}%
\end{pgfscope}%
\begin{pgfscope}%
\pgfsys@transformshift{1.556815in}{1.570428in}%
\pgfsys@useobject{currentmarker}{}%
\end{pgfscope}%
\begin{pgfscope}%
\pgfsys@transformshift{1.488589in}{1.562106in}%
\pgfsys@useobject{currentmarker}{}%
\end{pgfscope}%
\begin{pgfscope}%
\pgfsys@transformshift{1.424545in}{1.555053in}%
\pgfsys@useobject{currentmarker}{}%
\end{pgfscope}%
\begin{pgfscope}%
\pgfsys@transformshift{1.364836in}{1.549059in}%
\pgfsys@useobject{currentmarker}{}%
\end{pgfscope}%
\begin{pgfscope}%
\pgfsys@transformshift{1.309599in}{1.543961in}%
\pgfsys@useobject{currentmarker}{}%
\end{pgfscope}%
\begin{pgfscope}%
\pgfsys@transformshift{1.258965in}{1.539626in}%
\pgfsys@useobject{currentmarker}{}%
\end{pgfscope}%
\begin{pgfscope}%
\pgfsys@transformshift{1.213050in}{1.535952in}%
\pgfsys@useobject{currentmarker}{}%
\end{pgfscope}%
\begin{pgfscope}%
\pgfsys@transformshift{1.171963in}{1.532853in}%
\pgfsys@useobject{currentmarker}{}%
\end{pgfscope}%
\begin{pgfscope}%
\pgfsys@transformshift{1.135800in}{1.530263in}%
\pgfsys@useobject{currentmarker}{}%
\end{pgfscope}%
\begin{pgfscope}%
\pgfsys@transformshift{1.104644in}{1.528129in}%
\pgfsys@useobject{currentmarker}{}%
\end{pgfscope}%
\begin{pgfscope}%
\pgfsys@transformshift{1.078569in}{1.526408in}%
\pgfsys@useobject{currentmarker}{}%
\end{pgfscope}%
\begin{pgfscope}%
\pgfsys@transformshift{1.057636in}{1.525068in}%
\pgfsys@useobject{currentmarker}{}%
\end{pgfscope}%
\begin{pgfscope}%
\pgfsys@transformshift{1.041893in}{1.524084in}%
\pgfsys@useobject{currentmarker}{}%
\end{pgfscope}%
\begin{pgfscope}%
\pgfsys@transformshift{1.031377in}{1.523437in}%
\pgfsys@useobject{currentmarker}{}%
\end{pgfscope}%
\begin{pgfscope}%
\pgfsys@transformshift{1.026113in}{1.523117in}%
\pgfsys@useobject{currentmarker}{}%
\end{pgfscope}%
\end{pgfscope}%
\begin{pgfscope}%
\pgfpathrectangle{\pgfqpoint{0.800000in}{0.528000in}}{\pgfqpoint{4.960000in}{3.696000in}}%
\pgfusepath{clip}%
\pgfsetrectcap%
\pgfsetroundjoin%
\pgfsetlinewidth{1.505625pt}%
\definecolor{currentstroke}{rgb}{0.549020,0.337255,0.294118}%
\pgfsetstrokecolor{currentstroke}%
\pgfsetdash{}{0pt}%
\pgfpathmoveto{\pgfqpoint{1.025455in}{1.523079in}}%
\pgfpathlineto{\pgfqpoint{1.184066in}{1.533749in}}%
\pgfpathlineto{\pgfqpoint{1.320018in}{1.544891in}}%
\pgfpathlineto{\pgfqpoint{1.455971in}{1.558426in}}%
\pgfpathlineto{\pgfqpoint{1.569265in}{1.572044in}}%
\pgfpathlineto{\pgfqpoint{1.659899in}{1.584862in}}%
\pgfpathlineto{\pgfqpoint{1.750534in}{1.599774in}}%
\pgfpathlineto{\pgfqpoint{1.841169in}{1.617265in}}%
\pgfpathlineto{\pgfqpoint{1.909146in}{1.632416in}}%
\pgfpathlineto{\pgfqpoint{1.977122in}{1.649672in}}%
\pgfpathlineto{\pgfqpoint{2.045098in}{1.669400in}}%
\pgfpathlineto{\pgfqpoint{2.113074in}{1.692078in}}%
\pgfpathlineto{\pgfqpoint{2.158392in}{1.709121in}}%
\pgfpathlineto{\pgfqpoint{2.203709in}{1.727928in}}%
\pgfpathlineto{\pgfqpoint{2.249027in}{1.748734in}}%
\pgfpathlineto{\pgfqpoint{2.294344in}{1.771815in}}%
\pgfpathlineto{\pgfqpoint{2.339662in}{1.797486in}}%
\pgfpathlineto{\pgfqpoint{2.384979in}{1.826098in}}%
\pgfpathlineto{\pgfqpoint{2.430297in}{1.858060in}}%
\pgfpathlineto{\pgfqpoint{2.475614in}{1.893851in}}%
\pgfpathlineto{\pgfqpoint{2.520932in}{1.934015in}}%
\pgfpathlineto{\pgfqpoint{2.566249in}{1.979157in}}%
\pgfpathlineto{\pgfqpoint{2.611567in}{2.029946in}}%
\pgfpathlineto{\pgfqpoint{2.656884in}{2.087120in}}%
\pgfpathlineto{\pgfqpoint{2.679543in}{2.118347in}}%
\pgfpathlineto{\pgfqpoint{2.702202in}{2.151465in}}%
\pgfpathlineto{\pgfqpoint{2.724861in}{2.186572in}}%
\pgfpathlineto{\pgfqpoint{2.747519in}{2.223758in}}%
\pgfpathlineto{\pgfqpoint{2.770178in}{2.263107in}}%
\pgfpathlineto{\pgfqpoint{2.815496in}{2.348564in}}%
\pgfpathlineto{\pgfqpoint{2.860813in}{2.443284in}}%
\pgfpathlineto{\pgfqpoint{2.906131in}{2.547118in}}%
\pgfpathlineto{\pgfqpoint{2.951448in}{2.659116in}}%
\pgfpathlineto{\pgfqpoint{3.019424in}{2.837357in}}%
\pgfpathlineto{\pgfqpoint{3.087401in}{3.014861in}}%
\pgfpathlineto{\pgfqpoint{3.110059in}{3.070151in}}%
\pgfpathlineto{\pgfqpoint{3.132718in}{3.121843in}}%
\pgfpathlineto{\pgfqpoint{3.155377in}{3.168854in}}%
\pgfpathlineto{\pgfqpoint{3.178036in}{3.210100in}}%
\pgfpathlineto{\pgfqpoint{3.200694in}{3.244551in}}%
\pgfpathlineto{\pgfqpoint{3.223353in}{3.271288in}}%
\pgfpathlineto{\pgfqpoint{3.246012in}{3.289560in}}%
\pgfpathlineto{\pgfqpoint{3.268671in}{3.298834in}}%
\pgfpathlineto{\pgfqpoint{3.291329in}{3.298834in}}%
\pgfpathlineto{\pgfqpoint{3.313988in}{3.289560in}}%
\pgfpathlineto{\pgfqpoint{3.336647in}{3.271288in}}%
\pgfpathlineto{\pgfqpoint{3.359306in}{3.244551in}}%
\pgfpathlineto{\pgfqpoint{3.381964in}{3.210100in}}%
\pgfpathlineto{\pgfqpoint{3.404623in}{3.168854in}}%
\pgfpathlineto{\pgfqpoint{3.427282in}{3.121843in}}%
\pgfpathlineto{\pgfqpoint{3.449941in}{3.070151in}}%
\pgfpathlineto{\pgfqpoint{3.495258in}{2.957010in}}%
\pgfpathlineto{\pgfqpoint{3.631211in}{2.602197in}}%
\pgfpathlineto{\pgfqpoint{3.676528in}{2.494098in}}%
\pgfpathlineto{\pgfqpoint{3.721846in}{2.394761in}}%
\pgfpathlineto{\pgfqpoint{3.767163in}{2.304691in}}%
\pgfpathlineto{\pgfqpoint{3.812481in}{2.223758in}}%
\pgfpathlineto{\pgfqpoint{3.857798in}{2.151465in}}%
\pgfpathlineto{\pgfqpoint{3.903116in}{2.087120in}}%
\pgfpathlineto{\pgfqpoint{3.948433in}{2.029946in}}%
\pgfpathlineto{\pgfqpoint{3.993751in}{1.979157in}}%
\pgfpathlineto{\pgfqpoint{4.039068in}{1.934015in}}%
\pgfpathlineto{\pgfqpoint{4.084386in}{1.893851in}}%
\pgfpathlineto{\pgfqpoint{4.129703in}{1.858060in}}%
\pgfpathlineto{\pgfqpoint{4.175021in}{1.826098in}}%
\pgfpathlineto{\pgfqpoint{4.220338in}{1.797486in}}%
\pgfpathlineto{\pgfqpoint{4.265656in}{1.771815in}}%
\pgfpathlineto{\pgfqpoint{4.310973in}{1.748734in}}%
\pgfpathlineto{\pgfqpoint{4.356291in}{1.727928in}}%
\pgfpathlineto{\pgfqpoint{4.401608in}{1.709121in}}%
\pgfpathlineto{\pgfqpoint{4.446926in}{1.692078in}}%
\pgfpathlineto{\pgfqpoint{4.514902in}{1.669400in}}%
\pgfpathlineto{\pgfqpoint{4.582878in}{1.649672in}}%
\pgfpathlineto{\pgfqpoint{4.650854in}{1.632416in}}%
\pgfpathlineto{\pgfqpoint{4.718831in}{1.617265in}}%
\pgfpathlineto{\pgfqpoint{4.809466in}{1.599774in}}%
\pgfpathlineto{\pgfqpoint{4.900101in}{1.584862in}}%
\pgfpathlineto{\pgfqpoint{4.990735in}{1.572044in}}%
\pgfpathlineto{\pgfqpoint{5.104029in}{1.558426in}}%
\pgfpathlineto{\pgfqpoint{5.217323in}{1.546963in}}%
\pgfpathlineto{\pgfqpoint{5.353275in}{1.535464in}}%
\pgfpathlineto{\pgfqpoint{5.511887in}{1.524469in}}%
\pgfpathlineto{\pgfqpoint{5.534545in}{1.523079in}}%
\pgfpathlineto{\pgfqpoint{5.534545in}{1.523079in}}%
\pgfusepath{stroke}%
\end{pgfscope}%
\begin{pgfscope}%
\pgfsetrectcap%
\pgfsetmiterjoin%
\pgfsetlinewidth{0.803000pt}%
\definecolor{currentstroke}{rgb}{0.000000,0.000000,0.000000}%
\pgfsetstrokecolor{currentstroke}%
\pgfsetdash{}{0pt}%
\pgfpathmoveto{\pgfqpoint{0.800000in}{0.528000in}}%
\pgfpathlineto{\pgfqpoint{0.800000in}{4.224000in}}%
\pgfusepath{stroke}%
\end{pgfscope}%
\begin{pgfscope}%
\pgfsetrectcap%
\pgfsetmiterjoin%
\pgfsetlinewidth{0.803000pt}%
\definecolor{currentstroke}{rgb}{0.000000,0.000000,0.000000}%
\pgfsetstrokecolor{currentstroke}%
\pgfsetdash{}{0pt}%
\pgfpathmoveto{\pgfqpoint{5.760000in}{0.528000in}}%
\pgfpathlineto{\pgfqpoint{5.760000in}{4.224000in}}%
\pgfusepath{stroke}%
\end{pgfscope}%
\begin{pgfscope}%
\pgfsetrectcap%
\pgfsetmiterjoin%
\pgfsetlinewidth{0.803000pt}%
\definecolor{currentstroke}{rgb}{0.000000,0.000000,0.000000}%
\pgfsetstrokecolor{currentstroke}%
\pgfsetdash{}{0pt}%
\pgfpathmoveto{\pgfqpoint{0.800000in}{0.528000in}}%
\pgfpathlineto{\pgfqpoint{5.760000in}{0.528000in}}%
\pgfusepath{stroke}%
\end{pgfscope}%
\begin{pgfscope}%
\pgfsetrectcap%
\pgfsetmiterjoin%
\pgfsetlinewidth{0.803000pt}%
\definecolor{currentstroke}{rgb}{0.000000,0.000000,0.000000}%
\pgfsetstrokecolor{currentstroke}%
\pgfsetdash{}{0pt}%
\pgfpathmoveto{\pgfqpoint{0.800000in}{4.224000in}}%
\pgfpathlineto{\pgfqpoint{5.760000in}{4.224000in}}%
\pgfusepath{stroke}%
\end{pgfscope}%
\begin{pgfscope}%
\definecolor{textcolor}{rgb}{0.000000,0.000000,0.000000}%
\pgfsetstrokecolor{textcolor}%
\pgfsetfillcolor{textcolor}%
\pgftext[x=3.280000in,y=4.307333in,,base]{\color{textcolor}\rmfamily\fontsize{12.000000}{14.400000}\selectfont Wielomiany Lagrange'a dla N=6, 64}%
\end{pgfscope}%
\begin{pgfscope}%
\pgfsetbuttcap%
\pgfsetmiterjoin%
\definecolor{currentfill}{rgb}{1.000000,1.000000,1.000000}%
\pgfsetfillcolor{currentfill}%
\pgfsetfillopacity{0.800000}%
\pgfsetlinewidth{1.003750pt}%
\definecolor{currentstroke}{rgb}{0.800000,0.800000,0.800000}%
\pgfsetstrokecolor{currentstroke}%
\pgfsetstrokeopacity{0.800000}%
\pgfsetdash{}{0pt}%
\pgfpathmoveto{\pgfqpoint{4.258142in}{0.597444in}}%
\pgfpathlineto{\pgfqpoint{5.662778in}{0.597444in}}%
\pgfpathquadraticcurveto{\pgfqpoint{5.690556in}{0.597444in}}{\pgfqpoint{5.690556in}{0.625222in}}%
\pgfpathlineto{\pgfqpoint{5.690556in}{10.042217in}}%
\pgfpathquadraticcurveto{\pgfqpoint{5.690556in}{10.069994in}}{\pgfqpoint{5.662778in}{10.069994in}}%
\pgfpathlineto{\pgfqpoint{4.258142in}{10.069994in}}%
\pgfpathquadraticcurveto{\pgfqpoint{4.230364in}{10.069994in}}{\pgfqpoint{4.230364in}{10.042217in}}%
\pgfpathlineto{\pgfqpoint{4.230364in}{0.625222in}}%
\pgfpathquadraticcurveto{\pgfqpoint{4.230364in}{0.597444in}}{\pgfqpoint{4.258142in}{0.597444in}}%
\pgfpathclose%
\pgfusepath{stroke,fill}%
\end{pgfscope}%
\begin{pgfscope}%
\pgfsetrectcap%
\pgfsetroundjoin%
\pgfsetlinewidth{1.505625pt}%
\definecolor{currentstroke}{rgb}{0.121569,0.466667,0.705882}%
\pgfsetstrokecolor{currentstroke}%
\pgfsetdash{}{0pt}%
\pgfpathmoveto{\pgfqpoint{4.285920in}{9.879584in}}%
\pgfpathlineto{\pgfqpoint{4.563698in}{9.879584in}}%
\pgfusepath{stroke}%
\end{pgfscope}%
\begin{pgfscope}%
\definecolor{textcolor}{rgb}{0.000000,0.000000,0.000000}%
\pgfsetstrokecolor{textcolor}%
\pgfsetfillcolor{textcolor}%
\pgftext[x=4.674809in,y=9.830973in,left,base]{\color{textcolor}\rmfamily\fontsize{10.000000}{12.000000}\selectfont \(\displaystyle y(x)=\)\(\displaystyle \frac{1}{1+25x^{2}}\)}%
\end{pgfscope}%
\begin{pgfscope}%
\pgfsetrectcap%
\pgfsetroundjoin%
\pgfsetlinewidth{1.505625pt}%
\definecolor{currentstroke}{rgb}{0.172549,0.627451,0.172549}%
\pgfsetstrokecolor{currentstroke}%
\pgfsetdash{}{0pt}%
\pgfpathmoveto{\pgfqpoint{4.285920in}{9.599128in}}%
\pgfpathlineto{\pgfqpoint{4.563698in}{9.599128in}}%
\pgfusepath{stroke}%
\end{pgfscope}%
\begin{pgfscope}%
\definecolor{textcolor}{rgb}{0.000000,0.000000,0.000000}%
\pgfsetstrokecolor{textcolor}%
\pgfsetfillcolor{textcolor}%
\pgftext[x=4.674809in,y=9.550517in,left,base]{\color{textcolor}\rmfamily\fontsize{10.000000}{12.000000}\selectfont W2(x)}%
\end{pgfscope}%
\begin{pgfscope}%
\pgfsetrectcap%
\pgfsetroundjoin%
\pgfsetlinewidth{1.505625pt}%
\definecolor{currentstroke}{rgb}{0.580392,0.403922,0.741176}%
\pgfsetstrokecolor{currentstroke}%
\pgfsetdash{}{0pt}%
\pgfpathmoveto{\pgfqpoint{4.285920in}{9.390795in}}%
\pgfpathlineto{\pgfqpoint{4.563698in}{9.390795in}}%
\pgfusepath{stroke}%
\end{pgfscope}%
\begin{pgfscope}%
\definecolor{textcolor}{rgb}{0.000000,0.000000,0.000000}%
\pgfsetstrokecolor{textcolor}%
\pgfsetfillcolor{textcolor}%
\pgftext[x=4.674809in,y=9.342184in,left,base]{\color{textcolor}\rmfamily\fontsize{10.000000}{12.000000}\selectfont W4(x)}%
\end{pgfscope}%
\begin{pgfscope}%
\pgfsetrectcap%
\pgfsetroundjoin%
\pgfsetlinewidth{1.505625pt}%
\definecolor{currentstroke}{rgb}{0.549020,0.337255,0.294118}%
\pgfsetstrokecolor{currentstroke}%
\pgfsetdash{}{0pt}%
\pgfpathmoveto{\pgfqpoint{4.285920in}{9.103163in}}%
\pgfpathlineto{\pgfqpoint{4.563698in}{9.103163in}}%
\pgfusepath{stroke}%
\end{pgfscope}%
\begin{pgfscope}%
\definecolor{textcolor}{rgb}{0.000000,0.000000,0.000000}%
\pgfsetstrokecolor{textcolor}%
\pgfsetfillcolor{textcolor}%
\pgftext[x=4.674809in,y=9.054552in,left,base]{\color{textcolor}\rmfamily\fontsize{10.000000}{12.000000}\selectfont \(\displaystyle y(x)=\)\(\displaystyle \frac{1}{1+25x^{2}}\)}%
\end{pgfscope}%
\begin{pgfscope}%
\pgfsetrectcap%
\pgfsetroundjoin%
\pgfsetlinewidth{1.505625pt}%
\definecolor{currentstroke}{rgb}{0.498039,0.498039,0.498039}%
\pgfsetstrokecolor{currentstroke}%
\pgfsetdash{}{0pt}%
\pgfpathmoveto{\pgfqpoint{4.285920in}{8.822707in}}%
\pgfpathlineto{\pgfqpoint{4.563698in}{8.822707in}}%
\pgfusepath{stroke}%
\end{pgfscope}%
\begin{pgfscope}%
\definecolor{textcolor}{rgb}{0.000000,0.000000,0.000000}%
\pgfsetstrokecolor{textcolor}%
\pgfsetfillcolor{textcolor}%
\pgftext[x=4.674809in,y=8.774096in,left,base]{\color{textcolor}\rmfamily\fontsize{10.000000}{12.000000}\selectfont W2(x)}%
\end{pgfscope}%
\begin{pgfscope}%
\pgfsetrectcap%
\pgfsetroundjoin%
\pgfsetlinewidth{1.505625pt}%
\definecolor{currentstroke}{rgb}{0.090196,0.745098,0.811765}%
\pgfsetstrokecolor{currentstroke}%
\pgfsetdash{}{0pt}%
\pgfpathmoveto{\pgfqpoint{4.285920in}{8.614373in}}%
\pgfpathlineto{\pgfqpoint{4.563698in}{8.614373in}}%
\pgfusepath{stroke}%
\end{pgfscope}%
\begin{pgfscope}%
\definecolor{textcolor}{rgb}{0.000000,0.000000,0.000000}%
\pgfsetstrokecolor{textcolor}%
\pgfsetfillcolor{textcolor}%
\pgftext[x=4.674809in,y=8.565762in,left,base]{\color{textcolor}\rmfamily\fontsize{10.000000}{12.000000}\selectfont W4(x)}%
\end{pgfscope}%
\begin{pgfscope}%
\pgfsetrectcap%
\pgfsetroundjoin%
\pgfsetlinewidth{1.505625pt}%
\definecolor{currentstroke}{rgb}{0.121569,0.466667,0.705882}%
\pgfsetstrokecolor{currentstroke}%
\pgfsetdash{}{0pt}%
\pgfpathmoveto{\pgfqpoint{4.285920in}{8.326741in}}%
\pgfpathlineto{\pgfqpoint{4.563698in}{8.326741in}}%
\pgfusepath{stroke}%
\end{pgfscope}%
\begin{pgfscope}%
\definecolor{textcolor}{rgb}{0.000000,0.000000,0.000000}%
\pgfsetstrokecolor{textcolor}%
\pgfsetfillcolor{textcolor}%
\pgftext[x=4.674809in,y=8.278130in,left,base]{\color{textcolor}\rmfamily\fontsize{10.000000}{12.000000}\selectfont \(\displaystyle y(x)=\)\(\displaystyle \frac{1}{1+x^{2}}\)}%
\end{pgfscope}%
\begin{pgfscope}%
\pgfsetrectcap%
\pgfsetroundjoin%
\pgfsetlinewidth{1.505625pt}%
\definecolor{currentstroke}{rgb}{0.172549,0.627451,0.172549}%
\pgfsetstrokecolor{currentstroke}%
\pgfsetdash{}{0pt}%
\pgfpathmoveto{\pgfqpoint{4.285920in}{8.046285in}}%
\pgfpathlineto{\pgfqpoint{4.563698in}{8.046285in}}%
\pgfusepath{stroke}%
\end{pgfscope}%
\begin{pgfscope}%
\definecolor{textcolor}{rgb}{0.000000,0.000000,0.000000}%
\pgfsetstrokecolor{textcolor}%
\pgfsetfillcolor{textcolor}%
\pgftext[x=4.674809in,y=7.997674in,left,base]{\color{textcolor}\rmfamily\fontsize{10.000000}{12.000000}\selectfont W2(x)}%
\end{pgfscope}%
\begin{pgfscope}%
\pgfsetrectcap%
\pgfsetroundjoin%
\pgfsetlinewidth{1.505625pt}%
\definecolor{currentstroke}{rgb}{0.580392,0.403922,0.741176}%
\pgfsetstrokecolor{currentstroke}%
\pgfsetdash{}{0pt}%
\pgfpathmoveto{\pgfqpoint{4.285920in}{7.837952in}}%
\pgfpathlineto{\pgfqpoint{4.563698in}{7.837952in}}%
\pgfusepath{stroke}%
\end{pgfscope}%
\begin{pgfscope}%
\definecolor{textcolor}{rgb}{0.000000,0.000000,0.000000}%
\pgfsetstrokecolor{textcolor}%
\pgfsetfillcolor{textcolor}%
\pgftext[x=4.674809in,y=7.789341in,left,base]{\color{textcolor}\rmfamily\fontsize{10.000000}{12.000000}\selectfont W4(x)}%
\end{pgfscope}%
\begin{pgfscope}%
\pgfsetrectcap%
\pgfsetroundjoin%
\pgfsetlinewidth{1.505625pt}%
\definecolor{currentstroke}{rgb}{0.549020,0.337255,0.294118}%
\pgfsetstrokecolor{currentstroke}%
\pgfsetdash{}{0pt}%
\pgfpathmoveto{\pgfqpoint{4.285920in}{7.550319in}}%
\pgfpathlineto{\pgfqpoint{4.563698in}{7.550319in}}%
\pgfusepath{stroke}%
\end{pgfscope}%
\begin{pgfscope}%
\definecolor{textcolor}{rgb}{0.000000,0.000000,0.000000}%
\pgfsetstrokecolor{textcolor}%
\pgfsetfillcolor{textcolor}%
\pgftext[x=4.674809in,y=7.501708in,left,base]{\color{textcolor}\rmfamily\fontsize{10.000000}{12.000000}\selectfont \(\displaystyle y(x)=\)\(\displaystyle \frac{1}{1+x^{2}}\)}%
\end{pgfscope}%
\begin{pgfscope}%
\pgfsetrectcap%
\pgfsetroundjoin%
\pgfsetlinewidth{1.505625pt}%
\definecolor{currentstroke}{rgb}{0.498039,0.498039,0.498039}%
\pgfsetstrokecolor{currentstroke}%
\pgfsetdash{}{0pt}%
\pgfpathmoveto{\pgfqpoint{4.285920in}{7.269863in}}%
\pgfpathlineto{\pgfqpoint{4.563698in}{7.269863in}}%
\pgfusepath{stroke}%
\end{pgfscope}%
\begin{pgfscope}%
\definecolor{textcolor}{rgb}{0.000000,0.000000,0.000000}%
\pgfsetstrokecolor{textcolor}%
\pgfsetfillcolor{textcolor}%
\pgftext[x=4.674809in,y=7.221252in,left,base]{\color{textcolor}\rmfamily\fontsize{10.000000}{12.000000}\selectfont W2(x)}%
\end{pgfscope}%
\begin{pgfscope}%
\pgfsetrectcap%
\pgfsetroundjoin%
\pgfsetlinewidth{1.505625pt}%
\definecolor{currentstroke}{rgb}{0.090196,0.745098,0.811765}%
\pgfsetstrokecolor{currentstroke}%
\pgfsetdash{}{0pt}%
\pgfpathmoveto{\pgfqpoint{4.285920in}{7.061530in}}%
\pgfpathlineto{\pgfqpoint{4.563698in}{7.061530in}}%
\pgfusepath{stroke}%
\end{pgfscope}%
\begin{pgfscope}%
\definecolor{textcolor}{rgb}{0.000000,0.000000,0.000000}%
\pgfsetstrokecolor{textcolor}%
\pgfsetfillcolor{textcolor}%
\pgftext[x=4.674809in,y=7.012919in,left,base]{\color{textcolor}\rmfamily\fontsize{10.000000}{12.000000}\selectfont W4(x)}%
\end{pgfscope}%
\begin{pgfscope}%
\pgfsetrectcap%
\pgfsetroundjoin%
\pgfsetlinewidth{1.505625pt}%
\definecolor{currentstroke}{rgb}{0.121569,0.466667,0.705882}%
\pgfsetstrokecolor{currentstroke}%
\pgfsetdash{}{0pt}%
\pgfpathmoveto{\pgfqpoint{4.285920in}{6.773898in}}%
\pgfpathlineto{\pgfqpoint{4.563698in}{6.773898in}}%
\pgfusepath{stroke}%
\end{pgfscope}%
\begin{pgfscope}%
\definecolor{textcolor}{rgb}{0.000000,0.000000,0.000000}%
\pgfsetstrokecolor{textcolor}%
\pgfsetfillcolor{textcolor}%
\pgftext[x=4.674809in,y=6.725287in,left,base]{\color{textcolor}\rmfamily\fontsize{10.000000}{12.000000}\selectfont \(\displaystyle y(x)=\)\(\displaystyle \frac{1}{1+25x^{2}}\)}%
\end{pgfscope}%
\begin{pgfscope}%
\pgfsetrectcap%
\pgfsetroundjoin%
\pgfsetlinewidth{1.505625pt}%
\definecolor{currentstroke}{rgb}{0.172549,0.627451,0.172549}%
\pgfsetstrokecolor{currentstroke}%
\pgfsetdash{}{0pt}%
\pgfpathmoveto{\pgfqpoint{4.285920in}{6.493442in}}%
\pgfpathlineto{\pgfqpoint{4.563698in}{6.493442in}}%
\pgfusepath{stroke}%
\end{pgfscope}%
\begin{pgfscope}%
\definecolor{textcolor}{rgb}{0.000000,0.000000,0.000000}%
\pgfsetstrokecolor{textcolor}%
\pgfsetfillcolor{textcolor}%
\pgftext[x=4.674809in,y=6.444831in,left,base]{\color{textcolor}\rmfamily\fontsize{10.000000}{12.000000}\selectfont W6(x)}%
\end{pgfscope}%
\begin{pgfscope}%
\pgfsetrectcap%
\pgfsetroundjoin%
\pgfsetlinewidth{1.505625pt}%
\definecolor{currentstroke}{rgb}{0.580392,0.403922,0.741176}%
\pgfsetstrokecolor{currentstroke}%
\pgfsetdash{}{0pt}%
\pgfpathmoveto{\pgfqpoint{4.285920in}{6.285108in}}%
\pgfpathlineto{\pgfqpoint{4.563698in}{6.285108in}}%
\pgfusepath{stroke}%
\end{pgfscope}%
\begin{pgfscope}%
\definecolor{textcolor}{rgb}{0.000000,0.000000,0.000000}%
\pgfsetstrokecolor{textcolor}%
\pgfsetfillcolor{textcolor}%
\pgftext[x=4.674809in,y=6.236497in,left,base]{\color{textcolor}\rmfamily\fontsize{10.000000}{12.000000}\selectfont W8(x)}%
\end{pgfscope}%
\begin{pgfscope}%
\pgfsetrectcap%
\pgfsetroundjoin%
\pgfsetlinewidth{1.505625pt}%
\definecolor{currentstroke}{rgb}{0.890196,0.466667,0.760784}%
\pgfsetstrokecolor{currentstroke}%
\pgfsetdash{}{0pt}%
\pgfpathmoveto{\pgfqpoint{4.285920in}{6.076775in}}%
\pgfpathlineto{\pgfqpoint{4.563698in}{6.076775in}}%
\pgfusepath{stroke}%
\end{pgfscope}%
\begin{pgfscope}%
\definecolor{textcolor}{rgb}{0.000000,0.000000,0.000000}%
\pgfsetstrokecolor{textcolor}%
\pgfsetfillcolor{textcolor}%
\pgftext[x=4.674809in,y=6.028164in,left,base]{\color{textcolor}\rmfamily\fontsize{10.000000}{12.000000}\selectfont W16(x)}%
\end{pgfscope}%
\begin{pgfscope}%
\pgfsetrectcap%
\pgfsetroundjoin%
\pgfsetlinewidth{1.505625pt}%
\definecolor{currentstroke}{rgb}{0.737255,0.741176,0.133333}%
\pgfsetstrokecolor{currentstroke}%
\pgfsetdash{}{0pt}%
\pgfpathmoveto{\pgfqpoint{4.285920in}{5.868442in}}%
\pgfpathlineto{\pgfqpoint{4.563698in}{5.868442in}}%
\pgfusepath{stroke}%
\end{pgfscope}%
\begin{pgfscope}%
\definecolor{textcolor}{rgb}{0.000000,0.000000,0.000000}%
\pgfsetstrokecolor{textcolor}%
\pgfsetfillcolor{textcolor}%
\pgftext[x=4.674809in,y=5.819831in,left,base]{\color{textcolor}\rmfamily\fontsize{10.000000}{12.000000}\selectfont W32(x)}%
\end{pgfscope}%
\begin{pgfscope}%
\pgfsetrectcap%
\pgfsetroundjoin%
\pgfsetlinewidth{1.505625pt}%
\definecolor{currentstroke}{rgb}{0.090196,0.745098,0.811765}%
\pgfsetstrokecolor{currentstroke}%
\pgfsetdash{}{0pt}%
\pgfpathmoveto{\pgfqpoint{4.285920in}{5.580809in}}%
\pgfpathlineto{\pgfqpoint{4.563698in}{5.580809in}}%
\pgfusepath{stroke}%
\end{pgfscope}%
\begin{pgfscope}%
\definecolor{textcolor}{rgb}{0.000000,0.000000,0.000000}%
\pgfsetstrokecolor{textcolor}%
\pgfsetfillcolor{textcolor}%
\pgftext[x=4.674809in,y=5.532198in,left,base]{\color{textcolor}\rmfamily\fontsize{10.000000}{12.000000}\selectfont \(\displaystyle y(x)=\)\(\displaystyle \frac{1}{1+25x^{2}}\)}%
\end{pgfscope}%
\begin{pgfscope}%
\pgfsetrectcap%
\pgfsetroundjoin%
\pgfsetlinewidth{1.505625pt}%
\definecolor{currentstroke}{rgb}{1.000000,0.498039,0.054902}%
\pgfsetstrokecolor{currentstroke}%
\pgfsetdash{}{0pt}%
\pgfpathmoveto{\pgfqpoint{4.285920in}{5.300353in}}%
\pgfpathlineto{\pgfqpoint{4.563698in}{5.300353in}}%
\pgfusepath{stroke}%
\end{pgfscope}%
\begin{pgfscope}%
\definecolor{textcolor}{rgb}{0.000000,0.000000,0.000000}%
\pgfsetstrokecolor{textcolor}%
\pgfsetfillcolor{textcolor}%
\pgftext[x=4.674809in,y=5.251742in,left,base]{\color{textcolor}\rmfamily\fontsize{10.000000}{12.000000}\selectfont W6(x)}%
\end{pgfscope}%
\begin{pgfscope}%
\pgfsetrectcap%
\pgfsetroundjoin%
\pgfsetlinewidth{1.505625pt}%
\definecolor{currentstroke}{rgb}{0.839216,0.152941,0.156863}%
\pgfsetstrokecolor{currentstroke}%
\pgfsetdash{}{0pt}%
\pgfpathmoveto{\pgfqpoint{4.285920in}{5.092020in}}%
\pgfpathlineto{\pgfqpoint{4.563698in}{5.092020in}}%
\pgfusepath{stroke}%
\end{pgfscope}%
\begin{pgfscope}%
\definecolor{textcolor}{rgb}{0.000000,0.000000,0.000000}%
\pgfsetstrokecolor{textcolor}%
\pgfsetfillcolor{textcolor}%
\pgftext[x=4.674809in,y=5.043409in,left,base]{\color{textcolor}\rmfamily\fontsize{10.000000}{12.000000}\selectfont W8(x)}%
\end{pgfscope}%
\begin{pgfscope}%
\pgfsetrectcap%
\pgfsetroundjoin%
\pgfsetlinewidth{1.505625pt}%
\definecolor{currentstroke}{rgb}{0.549020,0.337255,0.294118}%
\pgfsetstrokecolor{currentstroke}%
\pgfsetdash{}{0pt}%
\pgfpathmoveto{\pgfqpoint{4.285920in}{4.883687in}}%
\pgfpathlineto{\pgfqpoint{4.563698in}{4.883687in}}%
\pgfusepath{stroke}%
\end{pgfscope}%
\begin{pgfscope}%
\definecolor{textcolor}{rgb}{0.000000,0.000000,0.000000}%
\pgfsetstrokecolor{textcolor}%
\pgfsetfillcolor{textcolor}%
\pgftext[x=4.674809in,y=4.835076in,left,base]{\color{textcolor}\rmfamily\fontsize{10.000000}{12.000000}\selectfont W16(x)}%
\end{pgfscope}%
\begin{pgfscope}%
\pgfsetrectcap%
\pgfsetroundjoin%
\pgfsetlinewidth{1.505625pt}%
\definecolor{currentstroke}{rgb}{0.498039,0.498039,0.498039}%
\pgfsetstrokecolor{currentstroke}%
\pgfsetdash{}{0pt}%
\pgfpathmoveto{\pgfqpoint{4.285920in}{4.675353in}}%
\pgfpathlineto{\pgfqpoint{4.563698in}{4.675353in}}%
\pgfusepath{stroke}%
\end{pgfscope}%
\begin{pgfscope}%
\definecolor{textcolor}{rgb}{0.000000,0.000000,0.000000}%
\pgfsetstrokecolor{textcolor}%
\pgfsetfillcolor{textcolor}%
\pgftext[x=4.674809in,y=4.626742in,left,base]{\color{textcolor}\rmfamily\fontsize{10.000000}{12.000000}\selectfont W32(x)}%
\end{pgfscope}%
\begin{pgfscope}%
\pgfsetrectcap%
\pgfsetroundjoin%
\pgfsetlinewidth{1.505625pt}%
\definecolor{currentstroke}{rgb}{0.737255,0.741176,0.133333}%
\pgfsetstrokecolor{currentstroke}%
\pgfsetdash{}{0pt}%
\pgfpathmoveto{\pgfqpoint{4.285920in}{4.387721in}}%
\pgfpathlineto{\pgfqpoint{4.563698in}{4.387721in}}%
\pgfusepath{stroke}%
\end{pgfscope}%
\begin{pgfscope}%
\definecolor{textcolor}{rgb}{0.000000,0.000000,0.000000}%
\pgfsetstrokecolor{textcolor}%
\pgfsetfillcolor{textcolor}%
\pgftext[x=4.674809in,y=4.339110in,left,base]{\color{textcolor}\rmfamily\fontsize{10.000000}{12.000000}\selectfont \(\displaystyle y(x)=\)\(\displaystyle \frac{1}{1+x^{2}}\)}%
\end{pgfscope}%
\begin{pgfscope}%
\pgfsetrectcap%
\pgfsetroundjoin%
\pgfsetlinewidth{1.505625pt}%
\definecolor{currentstroke}{rgb}{0.121569,0.466667,0.705882}%
\pgfsetstrokecolor{currentstroke}%
\pgfsetdash{}{0pt}%
\pgfpathmoveto{\pgfqpoint{4.285920in}{4.107265in}}%
\pgfpathlineto{\pgfqpoint{4.563698in}{4.107265in}}%
\pgfusepath{stroke}%
\end{pgfscope}%
\begin{pgfscope}%
\definecolor{textcolor}{rgb}{0.000000,0.000000,0.000000}%
\pgfsetstrokecolor{textcolor}%
\pgfsetfillcolor{textcolor}%
\pgftext[x=4.674809in,y=4.058654in,left,base]{\color{textcolor}\rmfamily\fontsize{10.000000}{12.000000}\selectfont W6(x)}%
\end{pgfscope}%
\begin{pgfscope}%
\pgfsetrectcap%
\pgfsetroundjoin%
\pgfsetlinewidth{1.505625pt}%
\definecolor{currentstroke}{rgb}{0.172549,0.627451,0.172549}%
\pgfsetstrokecolor{currentstroke}%
\pgfsetdash{}{0pt}%
\pgfpathmoveto{\pgfqpoint{4.285920in}{3.898932in}}%
\pgfpathlineto{\pgfqpoint{4.563698in}{3.898932in}}%
\pgfusepath{stroke}%
\end{pgfscope}%
\begin{pgfscope}%
\definecolor{textcolor}{rgb}{0.000000,0.000000,0.000000}%
\pgfsetstrokecolor{textcolor}%
\pgfsetfillcolor{textcolor}%
\pgftext[x=4.674809in,y=3.850321in,left,base]{\color{textcolor}\rmfamily\fontsize{10.000000}{12.000000}\selectfont W8(x)}%
\end{pgfscope}%
\begin{pgfscope}%
\pgfsetrectcap%
\pgfsetroundjoin%
\pgfsetlinewidth{1.505625pt}%
\definecolor{currentstroke}{rgb}{0.580392,0.403922,0.741176}%
\pgfsetstrokecolor{currentstroke}%
\pgfsetdash{}{0pt}%
\pgfpathmoveto{\pgfqpoint{4.285920in}{3.690598in}}%
\pgfpathlineto{\pgfqpoint{4.563698in}{3.690598in}}%
\pgfusepath{stroke}%
\end{pgfscope}%
\begin{pgfscope}%
\definecolor{textcolor}{rgb}{0.000000,0.000000,0.000000}%
\pgfsetstrokecolor{textcolor}%
\pgfsetfillcolor{textcolor}%
\pgftext[x=4.674809in,y=3.641987in,left,base]{\color{textcolor}\rmfamily\fontsize{10.000000}{12.000000}\selectfont W16(x)}%
\end{pgfscope}%
\begin{pgfscope}%
\pgfsetrectcap%
\pgfsetroundjoin%
\pgfsetlinewidth{1.505625pt}%
\definecolor{currentstroke}{rgb}{0.890196,0.466667,0.760784}%
\pgfsetstrokecolor{currentstroke}%
\pgfsetdash{}{0pt}%
\pgfpathmoveto{\pgfqpoint{4.285920in}{3.482265in}}%
\pgfpathlineto{\pgfqpoint{4.563698in}{3.482265in}}%
\pgfusepath{stroke}%
\end{pgfscope}%
\begin{pgfscope}%
\definecolor{textcolor}{rgb}{0.000000,0.000000,0.000000}%
\pgfsetstrokecolor{textcolor}%
\pgfsetfillcolor{textcolor}%
\pgftext[x=4.674809in,y=3.433654in,left,base]{\color{textcolor}\rmfamily\fontsize{10.000000}{12.000000}\selectfont W32(x)}%
\end{pgfscope}%
\begin{pgfscope}%
\pgfsetrectcap%
\pgfsetroundjoin%
\pgfsetlinewidth{1.505625pt}%
\definecolor{currentstroke}{rgb}{0.498039,0.498039,0.498039}%
\pgfsetstrokecolor{currentstroke}%
\pgfsetdash{}{0pt}%
\pgfpathmoveto{\pgfqpoint{4.285920in}{3.194633in}}%
\pgfpathlineto{\pgfqpoint{4.563698in}{3.194633in}}%
\pgfusepath{stroke}%
\end{pgfscope}%
\begin{pgfscope}%
\definecolor{textcolor}{rgb}{0.000000,0.000000,0.000000}%
\pgfsetstrokecolor{textcolor}%
\pgfsetfillcolor{textcolor}%
\pgftext[x=4.674809in,y=3.146022in,left,base]{\color{textcolor}\rmfamily\fontsize{10.000000}{12.000000}\selectfont \(\displaystyle y(x)=\)\(\displaystyle \frac{1}{1+x^{2}}\)}%
\end{pgfscope}%
\begin{pgfscope}%
\pgfsetrectcap%
\pgfsetroundjoin%
\pgfsetlinewidth{1.505625pt}%
\definecolor{currentstroke}{rgb}{0.090196,0.745098,0.811765}%
\pgfsetstrokecolor{currentstroke}%
\pgfsetdash{}{0pt}%
\pgfpathmoveto{\pgfqpoint{4.285920in}{2.914177in}}%
\pgfpathlineto{\pgfqpoint{4.563698in}{2.914177in}}%
\pgfusepath{stroke}%
\end{pgfscope}%
\begin{pgfscope}%
\definecolor{textcolor}{rgb}{0.000000,0.000000,0.000000}%
\pgfsetstrokecolor{textcolor}%
\pgfsetfillcolor{textcolor}%
\pgftext[x=4.674809in,y=2.865566in,left,base]{\color{textcolor}\rmfamily\fontsize{10.000000}{12.000000}\selectfont W6(x)}%
\end{pgfscope}%
\begin{pgfscope}%
\pgfsetrectcap%
\pgfsetroundjoin%
\pgfsetlinewidth{1.505625pt}%
\definecolor{currentstroke}{rgb}{1.000000,0.498039,0.054902}%
\pgfsetstrokecolor{currentstroke}%
\pgfsetdash{}{0pt}%
\pgfpathmoveto{\pgfqpoint{4.285920in}{2.705843in}}%
\pgfpathlineto{\pgfqpoint{4.563698in}{2.705843in}}%
\pgfusepath{stroke}%
\end{pgfscope}%
\begin{pgfscope}%
\definecolor{textcolor}{rgb}{0.000000,0.000000,0.000000}%
\pgfsetstrokecolor{textcolor}%
\pgfsetfillcolor{textcolor}%
\pgftext[x=4.674809in,y=2.657232in,left,base]{\color{textcolor}\rmfamily\fontsize{10.000000}{12.000000}\selectfont W8(x)}%
\end{pgfscope}%
\begin{pgfscope}%
\pgfsetrectcap%
\pgfsetroundjoin%
\pgfsetlinewidth{1.505625pt}%
\definecolor{currentstroke}{rgb}{0.839216,0.152941,0.156863}%
\pgfsetstrokecolor{currentstroke}%
\pgfsetdash{}{0pt}%
\pgfpathmoveto{\pgfqpoint{4.285920in}{2.497510in}}%
\pgfpathlineto{\pgfqpoint{4.563698in}{2.497510in}}%
\pgfusepath{stroke}%
\end{pgfscope}%
\begin{pgfscope}%
\definecolor{textcolor}{rgb}{0.000000,0.000000,0.000000}%
\pgfsetstrokecolor{textcolor}%
\pgfsetfillcolor{textcolor}%
\pgftext[x=4.674809in,y=2.448899in,left,base]{\color{textcolor}\rmfamily\fontsize{10.000000}{12.000000}\selectfont W16(x)}%
\end{pgfscope}%
\begin{pgfscope}%
\pgfsetrectcap%
\pgfsetroundjoin%
\pgfsetlinewidth{1.505625pt}%
\definecolor{currentstroke}{rgb}{0.549020,0.337255,0.294118}%
\pgfsetstrokecolor{currentstroke}%
\pgfsetdash{}{0pt}%
\pgfpathmoveto{\pgfqpoint{4.285920in}{2.289177in}}%
\pgfpathlineto{\pgfqpoint{4.563698in}{2.289177in}}%
\pgfusepath{stroke}%
\end{pgfscope}%
\begin{pgfscope}%
\definecolor{textcolor}{rgb}{0.000000,0.000000,0.000000}%
\pgfsetstrokecolor{textcolor}%
\pgfsetfillcolor{textcolor}%
\pgftext[x=4.674809in,y=2.240566in,left,base]{\color{textcolor}\rmfamily\fontsize{10.000000}{12.000000}\selectfont W32(x)}%
\end{pgfscope}%
\begin{pgfscope}%
\pgfsetrectcap%
\pgfsetroundjoin%
\pgfsetlinewidth{1.505625pt}%
\definecolor{currentstroke}{rgb}{0.890196,0.466667,0.760784}%
\pgfsetstrokecolor{currentstroke}%
\pgfsetdash{}{0pt}%
\pgfpathmoveto{\pgfqpoint{4.285920in}{2.001544in}}%
\pgfpathlineto{\pgfqpoint{4.563698in}{2.001544in}}%
\pgfusepath{stroke}%
\end{pgfscope}%
\begin{pgfscope}%
\definecolor{textcolor}{rgb}{0.000000,0.000000,0.000000}%
\pgfsetstrokecolor{textcolor}%
\pgfsetfillcolor{textcolor}%
\pgftext[x=4.674809in,y=1.952933in,left,base]{\color{textcolor}\rmfamily\fontsize{10.000000}{12.000000}\selectfont \(\displaystyle y(x)=\)\(\displaystyle \frac{1}{1+25x^{2}}\)}%
\end{pgfscope}%
\begin{pgfscope}%
\pgfsetrectcap%
\pgfsetroundjoin%
\pgfsetlinewidth{1.505625pt}%
\definecolor{currentstroke}{rgb}{0.737255,0.741176,0.133333}%
\pgfsetstrokecolor{currentstroke}%
\pgfsetdash{}{0pt}%
\pgfpathmoveto{\pgfqpoint{4.285920in}{1.721088in}}%
\pgfpathlineto{\pgfqpoint{4.563698in}{1.721088in}}%
\pgfusepath{stroke}%
\end{pgfscope}%
\begin{pgfscope}%
\definecolor{textcolor}{rgb}{0.000000,0.000000,0.000000}%
\pgfsetstrokecolor{textcolor}%
\pgfsetfillcolor{textcolor}%
\pgftext[x=4.674809in,y=1.672477in,left,base]{\color{textcolor}\rmfamily\fontsize{10.000000}{12.000000}\selectfont W6(x)}%
\end{pgfscope}%
\begin{pgfscope}%
\pgfsetrectcap%
\pgfsetroundjoin%
\pgfsetlinewidth{1.505625pt}%
\definecolor{currentstroke}{rgb}{0.121569,0.466667,0.705882}%
\pgfsetstrokecolor{currentstroke}%
\pgfsetdash{}{0pt}%
\pgfpathmoveto{\pgfqpoint{4.285920in}{1.512755in}}%
\pgfpathlineto{\pgfqpoint{4.563698in}{1.512755in}}%
\pgfusepath{stroke}%
\end{pgfscope}%
\begin{pgfscope}%
\definecolor{textcolor}{rgb}{0.000000,0.000000,0.000000}%
\pgfsetstrokecolor{textcolor}%
\pgfsetfillcolor{textcolor}%
\pgftext[x=4.674809in,y=1.464144in,left,base]{\color{textcolor}\rmfamily\fontsize{10.000000}{12.000000}\selectfont W64(x)}%
\end{pgfscope}%
\begin{pgfscope}%
\pgfsetrectcap%
\pgfsetroundjoin%
\pgfsetlinewidth{1.505625pt}%
\definecolor{currentstroke}{rgb}{1.000000,0.498039,0.054902}%
\pgfsetstrokecolor{currentstroke}%
\pgfsetdash{}{0pt}%
\pgfpathmoveto{\pgfqpoint{4.285920in}{1.225123in}}%
\pgfpathlineto{\pgfqpoint{4.563698in}{1.225123in}}%
\pgfusepath{stroke}%
\end{pgfscope}%
\begin{pgfscope}%
\definecolor{textcolor}{rgb}{0.000000,0.000000,0.000000}%
\pgfsetstrokecolor{textcolor}%
\pgfsetfillcolor{textcolor}%
\pgftext[x=4.674809in,y=1.176512in,left,base]{\color{textcolor}\rmfamily\fontsize{10.000000}{12.000000}\selectfont \(\displaystyle y(x)=\)\(\displaystyle \frac{1}{1+25x^{2}}\)}%
\end{pgfscope}%
\begin{pgfscope}%
\pgfsetrectcap%
\pgfsetroundjoin%
\pgfsetlinewidth{1.505625pt}%
\definecolor{currentstroke}{rgb}{0.839216,0.152941,0.156863}%
\pgfsetstrokecolor{currentstroke}%
\pgfsetdash{}{0pt}%
\pgfpathmoveto{\pgfqpoint{4.285920in}{0.944667in}}%
\pgfpathlineto{\pgfqpoint{4.563698in}{0.944667in}}%
\pgfusepath{stroke}%
\end{pgfscope}%
\begin{pgfscope}%
\definecolor{textcolor}{rgb}{0.000000,0.000000,0.000000}%
\pgfsetstrokecolor{textcolor}%
\pgfsetfillcolor{textcolor}%
\pgftext[x=4.674809in,y=0.896056in,left,base]{\color{textcolor}\rmfamily\fontsize{10.000000}{12.000000}\selectfont W6(x)}%
\end{pgfscope}%
\begin{pgfscope}%
\pgfsetrectcap%
\pgfsetroundjoin%
\pgfsetlinewidth{1.505625pt}%
\definecolor{currentstroke}{rgb}{0.549020,0.337255,0.294118}%
\pgfsetstrokecolor{currentstroke}%
\pgfsetdash{}{0pt}%
\pgfpathmoveto{\pgfqpoint{4.285920in}{0.736333in}}%
\pgfpathlineto{\pgfqpoint{4.563698in}{0.736333in}}%
\pgfusepath{stroke}%
\end{pgfscope}%
\begin{pgfscope}%
\definecolor{textcolor}{rgb}{0.000000,0.000000,0.000000}%
\pgfsetstrokecolor{textcolor}%
\pgfsetfillcolor{textcolor}%
\pgftext[x=4.674809in,y=0.687722in,left,base]{\color{textcolor}\rmfamily\fontsize{10.000000}{12.000000}\selectfont W64(x)}%
\end{pgfscope}%
\end{pgfpicture}%
\makeatother%
\endgroup%
        
    \end{center}
    \caption{Węzły cosinus, funkcja \(y\), \(N=6,64\)}
\end{figure}

\begin{figure}[h]
    \begin{center}
        %% Creator: Matplotlib, PGF backend
%%
%% To include the figure in your LaTeX document, write
%%   \input{<filename>.pgf}
%%
%% Make sure the required packages are loaded in your preamble
%%   \usepackage{pgf}
%%
%% Figures using additional raster images can only be included by \input if
%% they are in the same directory as the main LaTeX file. For loading figures
%% from other directories you can use the `import` package
%%   \usepackage{import}
%% and then include the figures with
%%   \import{<path to file>}{<filename>.pgf}
%%
%% Matplotlib used the following preamble
%%
\begingroup%
\makeatletter%
\begin{pgfpicture}%
\pgfpathrectangle{\pgfpointorigin}{\pgfqpoint{5.500000in}{3.500000in}}%
\pgfusepath{use as bounding box, clip}%
\begin{pgfscope}%
\pgfsetbuttcap%
\pgfsetmiterjoin%
\definecolor{currentfill}{rgb}{1.000000,1.000000,1.000000}%
\pgfsetfillcolor{currentfill}%
\pgfsetlinewidth{0.000000pt}%
\definecolor{currentstroke}{rgb}{1.000000,1.000000,1.000000}%
\pgfsetstrokecolor{currentstroke}%
\pgfsetdash{}{0pt}%
\pgfpathmoveto{\pgfqpoint{0.000000in}{0.000000in}}%
\pgfpathlineto{\pgfqpoint{5.500000in}{0.000000in}}%
\pgfpathlineto{\pgfqpoint{5.500000in}{3.500000in}}%
\pgfpathlineto{\pgfqpoint{0.000000in}{3.500000in}}%
\pgfpathclose%
\pgfusepath{fill}%
\end{pgfscope}%
\begin{pgfscope}%
\pgfsetbuttcap%
\pgfsetmiterjoin%
\definecolor{currentfill}{rgb}{1.000000,1.000000,1.000000}%
\pgfsetfillcolor{currentfill}%
\pgfsetlinewidth{0.000000pt}%
\definecolor{currentstroke}{rgb}{0.000000,0.000000,0.000000}%
\pgfsetstrokecolor{currentstroke}%
\pgfsetstrokeopacity{0.000000}%
\pgfsetdash{}{0pt}%
\pgfpathmoveto{\pgfqpoint{0.687500in}{0.385000in}}%
\pgfpathlineto{\pgfqpoint{4.950000in}{0.385000in}}%
\pgfpathlineto{\pgfqpoint{4.950000in}{3.080000in}}%
\pgfpathlineto{\pgfqpoint{0.687500in}{3.080000in}}%
\pgfpathclose%
\pgfusepath{fill}%
\end{pgfscope}%
\begin{pgfscope}%
\pgfsetbuttcap%
\pgfsetroundjoin%
\definecolor{currentfill}{rgb}{0.000000,0.000000,0.000000}%
\pgfsetfillcolor{currentfill}%
\pgfsetlinewidth{0.803000pt}%
\definecolor{currentstroke}{rgb}{0.000000,0.000000,0.000000}%
\pgfsetstrokecolor{currentstroke}%
\pgfsetdash{}{0pt}%
\pgfsys@defobject{currentmarker}{\pgfqpoint{0.000000in}{-0.048611in}}{\pgfqpoint{0.000000in}{0.000000in}}{%
\pgfpathmoveto{\pgfqpoint{0.000000in}{0.000000in}}%
\pgfpathlineto{\pgfqpoint{0.000000in}{-0.048611in}}%
\pgfusepath{stroke,fill}%
}%
\begin{pgfscope}%
\pgfsys@transformshift{0.881250in}{0.385000in}%
\pgfsys@useobject{currentmarker}{}%
\end{pgfscope}%
\end{pgfscope}%
\begin{pgfscope}%
\definecolor{textcolor}{rgb}{0.000000,0.000000,0.000000}%
\pgfsetstrokecolor{textcolor}%
\pgfsetfillcolor{textcolor}%
\pgftext[x=0.881250in,y=0.287778in,,top]{\color{textcolor}\rmfamily\fontsize{10.000000}{12.000000}\selectfont \(\displaystyle -1.00\)}%
\end{pgfscope}%
\begin{pgfscope}%
\pgfsetbuttcap%
\pgfsetroundjoin%
\definecolor{currentfill}{rgb}{0.000000,0.000000,0.000000}%
\pgfsetfillcolor{currentfill}%
\pgfsetlinewidth{0.803000pt}%
\definecolor{currentstroke}{rgb}{0.000000,0.000000,0.000000}%
\pgfsetstrokecolor{currentstroke}%
\pgfsetdash{}{0pt}%
\pgfsys@defobject{currentmarker}{\pgfqpoint{0.000000in}{-0.048611in}}{\pgfqpoint{0.000000in}{0.000000in}}{%
\pgfpathmoveto{\pgfqpoint{0.000000in}{0.000000in}}%
\pgfpathlineto{\pgfqpoint{0.000000in}{-0.048611in}}%
\pgfusepath{stroke,fill}%
}%
\begin{pgfscope}%
\pgfsys@transformshift{1.365625in}{0.385000in}%
\pgfsys@useobject{currentmarker}{}%
\end{pgfscope}%
\end{pgfscope}%
\begin{pgfscope}%
\definecolor{textcolor}{rgb}{0.000000,0.000000,0.000000}%
\pgfsetstrokecolor{textcolor}%
\pgfsetfillcolor{textcolor}%
\pgftext[x=1.365625in,y=0.287778in,,top]{\color{textcolor}\rmfamily\fontsize{10.000000}{12.000000}\selectfont \(\displaystyle -0.75\)}%
\end{pgfscope}%
\begin{pgfscope}%
\pgfsetbuttcap%
\pgfsetroundjoin%
\definecolor{currentfill}{rgb}{0.000000,0.000000,0.000000}%
\pgfsetfillcolor{currentfill}%
\pgfsetlinewidth{0.803000pt}%
\definecolor{currentstroke}{rgb}{0.000000,0.000000,0.000000}%
\pgfsetstrokecolor{currentstroke}%
\pgfsetdash{}{0pt}%
\pgfsys@defobject{currentmarker}{\pgfqpoint{0.000000in}{-0.048611in}}{\pgfqpoint{0.000000in}{0.000000in}}{%
\pgfpathmoveto{\pgfqpoint{0.000000in}{0.000000in}}%
\pgfpathlineto{\pgfqpoint{0.000000in}{-0.048611in}}%
\pgfusepath{stroke,fill}%
}%
\begin{pgfscope}%
\pgfsys@transformshift{1.850000in}{0.385000in}%
\pgfsys@useobject{currentmarker}{}%
\end{pgfscope}%
\end{pgfscope}%
\begin{pgfscope}%
\definecolor{textcolor}{rgb}{0.000000,0.000000,0.000000}%
\pgfsetstrokecolor{textcolor}%
\pgfsetfillcolor{textcolor}%
\pgftext[x=1.850000in,y=0.287778in,,top]{\color{textcolor}\rmfamily\fontsize{10.000000}{12.000000}\selectfont \(\displaystyle -0.50\)}%
\end{pgfscope}%
\begin{pgfscope}%
\pgfsetbuttcap%
\pgfsetroundjoin%
\definecolor{currentfill}{rgb}{0.000000,0.000000,0.000000}%
\pgfsetfillcolor{currentfill}%
\pgfsetlinewidth{0.803000pt}%
\definecolor{currentstroke}{rgb}{0.000000,0.000000,0.000000}%
\pgfsetstrokecolor{currentstroke}%
\pgfsetdash{}{0pt}%
\pgfsys@defobject{currentmarker}{\pgfqpoint{0.000000in}{-0.048611in}}{\pgfqpoint{0.000000in}{0.000000in}}{%
\pgfpathmoveto{\pgfqpoint{0.000000in}{0.000000in}}%
\pgfpathlineto{\pgfqpoint{0.000000in}{-0.048611in}}%
\pgfusepath{stroke,fill}%
}%
\begin{pgfscope}%
\pgfsys@transformshift{2.334375in}{0.385000in}%
\pgfsys@useobject{currentmarker}{}%
\end{pgfscope}%
\end{pgfscope}%
\begin{pgfscope}%
\definecolor{textcolor}{rgb}{0.000000,0.000000,0.000000}%
\pgfsetstrokecolor{textcolor}%
\pgfsetfillcolor{textcolor}%
\pgftext[x=2.334375in,y=0.287778in,,top]{\color{textcolor}\rmfamily\fontsize{10.000000}{12.000000}\selectfont \(\displaystyle -0.25\)}%
\end{pgfscope}%
\begin{pgfscope}%
\pgfsetbuttcap%
\pgfsetroundjoin%
\definecolor{currentfill}{rgb}{0.000000,0.000000,0.000000}%
\pgfsetfillcolor{currentfill}%
\pgfsetlinewidth{0.803000pt}%
\definecolor{currentstroke}{rgb}{0.000000,0.000000,0.000000}%
\pgfsetstrokecolor{currentstroke}%
\pgfsetdash{}{0pt}%
\pgfsys@defobject{currentmarker}{\pgfqpoint{0.000000in}{-0.048611in}}{\pgfqpoint{0.000000in}{0.000000in}}{%
\pgfpathmoveto{\pgfqpoint{0.000000in}{0.000000in}}%
\pgfpathlineto{\pgfqpoint{0.000000in}{-0.048611in}}%
\pgfusepath{stroke,fill}%
}%
\begin{pgfscope}%
\pgfsys@transformshift{2.818750in}{0.385000in}%
\pgfsys@useobject{currentmarker}{}%
\end{pgfscope}%
\end{pgfscope}%
\begin{pgfscope}%
\definecolor{textcolor}{rgb}{0.000000,0.000000,0.000000}%
\pgfsetstrokecolor{textcolor}%
\pgfsetfillcolor{textcolor}%
\pgftext[x=2.818750in,y=0.287778in,,top]{\color{textcolor}\rmfamily\fontsize{10.000000}{12.000000}\selectfont \(\displaystyle 0.00\)}%
\end{pgfscope}%
\begin{pgfscope}%
\pgfsetbuttcap%
\pgfsetroundjoin%
\definecolor{currentfill}{rgb}{0.000000,0.000000,0.000000}%
\pgfsetfillcolor{currentfill}%
\pgfsetlinewidth{0.803000pt}%
\definecolor{currentstroke}{rgb}{0.000000,0.000000,0.000000}%
\pgfsetstrokecolor{currentstroke}%
\pgfsetdash{}{0pt}%
\pgfsys@defobject{currentmarker}{\pgfqpoint{0.000000in}{-0.048611in}}{\pgfqpoint{0.000000in}{0.000000in}}{%
\pgfpathmoveto{\pgfqpoint{0.000000in}{0.000000in}}%
\pgfpathlineto{\pgfqpoint{0.000000in}{-0.048611in}}%
\pgfusepath{stroke,fill}%
}%
\begin{pgfscope}%
\pgfsys@transformshift{3.303125in}{0.385000in}%
\pgfsys@useobject{currentmarker}{}%
\end{pgfscope}%
\end{pgfscope}%
\begin{pgfscope}%
\definecolor{textcolor}{rgb}{0.000000,0.000000,0.000000}%
\pgfsetstrokecolor{textcolor}%
\pgfsetfillcolor{textcolor}%
\pgftext[x=3.303125in,y=0.287778in,,top]{\color{textcolor}\rmfamily\fontsize{10.000000}{12.000000}\selectfont \(\displaystyle 0.25\)}%
\end{pgfscope}%
\begin{pgfscope}%
\pgfsetbuttcap%
\pgfsetroundjoin%
\definecolor{currentfill}{rgb}{0.000000,0.000000,0.000000}%
\pgfsetfillcolor{currentfill}%
\pgfsetlinewidth{0.803000pt}%
\definecolor{currentstroke}{rgb}{0.000000,0.000000,0.000000}%
\pgfsetstrokecolor{currentstroke}%
\pgfsetdash{}{0pt}%
\pgfsys@defobject{currentmarker}{\pgfqpoint{0.000000in}{-0.048611in}}{\pgfqpoint{0.000000in}{0.000000in}}{%
\pgfpathmoveto{\pgfqpoint{0.000000in}{0.000000in}}%
\pgfpathlineto{\pgfqpoint{0.000000in}{-0.048611in}}%
\pgfusepath{stroke,fill}%
}%
\begin{pgfscope}%
\pgfsys@transformshift{3.787500in}{0.385000in}%
\pgfsys@useobject{currentmarker}{}%
\end{pgfscope}%
\end{pgfscope}%
\begin{pgfscope}%
\definecolor{textcolor}{rgb}{0.000000,0.000000,0.000000}%
\pgfsetstrokecolor{textcolor}%
\pgfsetfillcolor{textcolor}%
\pgftext[x=3.787500in,y=0.287778in,,top]{\color{textcolor}\rmfamily\fontsize{10.000000}{12.000000}\selectfont \(\displaystyle 0.50\)}%
\end{pgfscope}%
\begin{pgfscope}%
\pgfsetbuttcap%
\pgfsetroundjoin%
\definecolor{currentfill}{rgb}{0.000000,0.000000,0.000000}%
\pgfsetfillcolor{currentfill}%
\pgfsetlinewidth{0.803000pt}%
\definecolor{currentstroke}{rgb}{0.000000,0.000000,0.000000}%
\pgfsetstrokecolor{currentstroke}%
\pgfsetdash{}{0pt}%
\pgfsys@defobject{currentmarker}{\pgfqpoint{0.000000in}{-0.048611in}}{\pgfqpoint{0.000000in}{0.000000in}}{%
\pgfpathmoveto{\pgfqpoint{0.000000in}{0.000000in}}%
\pgfpathlineto{\pgfqpoint{0.000000in}{-0.048611in}}%
\pgfusepath{stroke,fill}%
}%
\begin{pgfscope}%
\pgfsys@transformshift{4.271875in}{0.385000in}%
\pgfsys@useobject{currentmarker}{}%
\end{pgfscope}%
\end{pgfscope}%
\begin{pgfscope}%
\definecolor{textcolor}{rgb}{0.000000,0.000000,0.000000}%
\pgfsetstrokecolor{textcolor}%
\pgfsetfillcolor{textcolor}%
\pgftext[x=4.271875in,y=0.287778in,,top]{\color{textcolor}\rmfamily\fontsize{10.000000}{12.000000}\selectfont \(\displaystyle 0.75\)}%
\end{pgfscope}%
\begin{pgfscope}%
\pgfsetbuttcap%
\pgfsetroundjoin%
\definecolor{currentfill}{rgb}{0.000000,0.000000,0.000000}%
\pgfsetfillcolor{currentfill}%
\pgfsetlinewidth{0.803000pt}%
\definecolor{currentstroke}{rgb}{0.000000,0.000000,0.000000}%
\pgfsetstrokecolor{currentstroke}%
\pgfsetdash{}{0pt}%
\pgfsys@defobject{currentmarker}{\pgfqpoint{0.000000in}{-0.048611in}}{\pgfqpoint{0.000000in}{0.000000in}}{%
\pgfpathmoveto{\pgfqpoint{0.000000in}{0.000000in}}%
\pgfpathlineto{\pgfqpoint{0.000000in}{-0.048611in}}%
\pgfusepath{stroke,fill}%
}%
\begin{pgfscope}%
\pgfsys@transformshift{4.756250in}{0.385000in}%
\pgfsys@useobject{currentmarker}{}%
\end{pgfscope}%
\end{pgfscope}%
\begin{pgfscope}%
\definecolor{textcolor}{rgb}{0.000000,0.000000,0.000000}%
\pgfsetstrokecolor{textcolor}%
\pgfsetfillcolor{textcolor}%
\pgftext[x=4.756250in,y=0.287778in,,top]{\color{textcolor}\rmfamily\fontsize{10.000000}{12.000000}\selectfont \(\displaystyle 1.00\)}%
\end{pgfscope}%
\begin{pgfscope}%
\definecolor{textcolor}{rgb}{0.000000,0.000000,0.000000}%
\pgfsetstrokecolor{textcolor}%
\pgfsetfillcolor{textcolor}%
\pgftext[x=2.818750in,y=0.108766in,,top]{\color{textcolor}\rmfamily\fontsize{10.000000}{12.000000}\selectfont x}%
\end{pgfscope}%
\begin{pgfscope}%
\pgfsetbuttcap%
\pgfsetroundjoin%
\definecolor{currentfill}{rgb}{0.000000,0.000000,0.000000}%
\pgfsetfillcolor{currentfill}%
\pgfsetlinewidth{0.803000pt}%
\definecolor{currentstroke}{rgb}{0.000000,0.000000,0.000000}%
\pgfsetstrokecolor{currentstroke}%
\pgfsetdash{}{0pt}%
\pgfsys@defobject{currentmarker}{\pgfqpoint{-0.048611in}{0.000000in}}{\pgfqpoint{0.000000in}{0.000000in}}{%
\pgfpathmoveto{\pgfqpoint{0.000000in}{0.000000in}}%
\pgfpathlineto{\pgfqpoint{-0.048611in}{0.000000in}}%
\pgfusepath{stroke,fill}%
}%
\begin{pgfscope}%
\pgfsys@transformshift{0.687500in}{0.474833in}%
\pgfsys@useobject{currentmarker}{}%
\end{pgfscope}%
\end{pgfscope}%
\begin{pgfscope}%
\definecolor{textcolor}{rgb}{0.000000,0.000000,0.000000}%
\pgfsetstrokecolor{textcolor}%
\pgfsetfillcolor{textcolor}%
\pgftext[x=0.304783in,y=0.426608in,left,base]{\color{textcolor}\rmfamily\fontsize{10.000000}{12.000000}\selectfont \(\displaystyle -0.2\)}%
\end{pgfscope}%
\begin{pgfscope}%
\pgfsetbuttcap%
\pgfsetroundjoin%
\definecolor{currentfill}{rgb}{0.000000,0.000000,0.000000}%
\pgfsetfillcolor{currentfill}%
\pgfsetlinewidth{0.803000pt}%
\definecolor{currentstroke}{rgb}{0.000000,0.000000,0.000000}%
\pgfsetstrokecolor{currentstroke}%
\pgfsetdash{}{0pt}%
\pgfsys@defobject{currentmarker}{\pgfqpoint{-0.048611in}{0.000000in}}{\pgfqpoint{0.000000in}{0.000000in}}{%
\pgfpathmoveto{\pgfqpoint{0.000000in}{0.000000in}}%
\pgfpathlineto{\pgfqpoint{-0.048611in}{0.000000in}}%
\pgfusepath{stroke,fill}%
}%
\begin{pgfscope}%
\pgfsys@transformshift{0.687500in}{0.834167in}%
\pgfsys@useobject{currentmarker}{}%
\end{pgfscope}%
\end{pgfscope}%
\begin{pgfscope}%
\definecolor{textcolor}{rgb}{0.000000,0.000000,0.000000}%
\pgfsetstrokecolor{textcolor}%
\pgfsetfillcolor{textcolor}%
\pgftext[x=0.412808in,y=0.785941in,left,base]{\color{textcolor}\rmfamily\fontsize{10.000000}{12.000000}\selectfont \(\displaystyle 0.0\)}%
\end{pgfscope}%
\begin{pgfscope}%
\pgfsetbuttcap%
\pgfsetroundjoin%
\definecolor{currentfill}{rgb}{0.000000,0.000000,0.000000}%
\pgfsetfillcolor{currentfill}%
\pgfsetlinewidth{0.803000pt}%
\definecolor{currentstroke}{rgb}{0.000000,0.000000,0.000000}%
\pgfsetstrokecolor{currentstroke}%
\pgfsetdash{}{0pt}%
\pgfsys@defobject{currentmarker}{\pgfqpoint{-0.048611in}{0.000000in}}{\pgfqpoint{0.000000in}{0.000000in}}{%
\pgfpathmoveto{\pgfqpoint{0.000000in}{0.000000in}}%
\pgfpathlineto{\pgfqpoint{-0.048611in}{0.000000in}}%
\pgfusepath{stroke,fill}%
}%
\begin{pgfscope}%
\pgfsys@transformshift{0.687500in}{1.193500in}%
\pgfsys@useobject{currentmarker}{}%
\end{pgfscope}%
\end{pgfscope}%
\begin{pgfscope}%
\definecolor{textcolor}{rgb}{0.000000,0.000000,0.000000}%
\pgfsetstrokecolor{textcolor}%
\pgfsetfillcolor{textcolor}%
\pgftext[x=0.412808in,y=1.145275in,left,base]{\color{textcolor}\rmfamily\fontsize{10.000000}{12.000000}\selectfont \(\displaystyle 0.2\)}%
\end{pgfscope}%
\begin{pgfscope}%
\pgfsetbuttcap%
\pgfsetroundjoin%
\definecolor{currentfill}{rgb}{0.000000,0.000000,0.000000}%
\pgfsetfillcolor{currentfill}%
\pgfsetlinewidth{0.803000pt}%
\definecolor{currentstroke}{rgb}{0.000000,0.000000,0.000000}%
\pgfsetstrokecolor{currentstroke}%
\pgfsetdash{}{0pt}%
\pgfsys@defobject{currentmarker}{\pgfqpoint{-0.048611in}{0.000000in}}{\pgfqpoint{0.000000in}{0.000000in}}{%
\pgfpathmoveto{\pgfqpoint{0.000000in}{0.000000in}}%
\pgfpathlineto{\pgfqpoint{-0.048611in}{0.000000in}}%
\pgfusepath{stroke,fill}%
}%
\begin{pgfscope}%
\pgfsys@transformshift{0.687500in}{1.552833in}%
\pgfsys@useobject{currentmarker}{}%
\end{pgfscope}%
\end{pgfscope}%
\begin{pgfscope}%
\definecolor{textcolor}{rgb}{0.000000,0.000000,0.000000}%
\pgfsetstrokecolor{textcolor}%
\pgfsetfillcolor{textcolor}%
\pgftext[x=0.412808in,y=1.504608in,left,base]{\color{textcolor}\rmfamily\fontsize{10.000000}{12.000000}\selectfont \(\displaystyle 0.4\)}%
\end{pgfscope}%
\begin{pgfscope}%
\pgfsetbuttcap%
\pgfsetroundjoin%
\definecolor{currentfill}{rgb}{0.000000,0.000000,0.000000}%
\pgfsetfillcolor{currentfill}%
\pgfsetlinewidth{0.803000pt}%
\definecolor{currentstroke}{rgb}{0.000000,0.000000,0.000000}%
\pgfsetstrokecolor{currentstroke}%
\pgfsetdash{}{0pt}%
\pgfsys@defobject{currentmarker}{\pgfqpoint{-0.048611in}{0.000000in}}{\pgfqpoint{0.000000in}{0.000000in}}{%
\pgfpathmoveto{\pgfqpoint{0.000000in}{0.000000in}}%
\pgfpathlineto{\pgfqpoint{-0.048611in}{0.000000in}}%
\pgfusepath{stroke,fill}%
}%
\begin{pgfscope}%
\pgfsys@transformshift{0.687500in}{1.912167in}%
\pgfsys@useobject{currentmarker}{}%
\end{pgfscope}%
\end{pgfscope}%
\begin{pgfscope}%
\definecolor{textcolor}{rgb}{0.000000,0.000000,0.000000}%
\pgfsetstrokecolor{textcolor}%
\pgfsetfillcolor{textcolor}%
\pgftext[x=0.412808in,y=1.863941in,left,base]{\color{textcolor}\rmfamily\fontsize{10.000000}{12.000000}\selectfont \(\displaystyle 0.6\)}%
\end{pgfscope}%
\begin{pgfscope}%
\pgfsetbuttcap%
\pgfsetroundjoin%
\definecolor{currentfill}{rgb}{0.000000,0.000000,0.000000}%
\pgfsetfillcolor{currentfill}%
\pgfsetlinewidth{0.803000pt}%
\definecolor{currentstroke}{rgb}{0.000000,0.000000,0.000000}%
\pgfsetstrokecolor{currentstroke}%
\pgfsetdash{}{0pt}%
\pgfsys@defobject{currentmarker}{\pgfqpoint{-0.048611in}{0.000000in}}{\pgfqpoint{0.000000in}{0.000000in}}{%
\pgfpathmoveto{\pgfqpoint{0.000000in}{0.000000in}}%
\pgfpathlineto{\pgfqpoint{-0.048611in}{0.000000in}}%
\pgfusepath{stroke,fill}%
}%
\begin{pgfscope}%
\pgfsys@transformshift{0.687500in}{2.271500in}%
\pgfsys@useobject{currentmarker}{}%
\end{pgfscope}%
\end{pgfscope}%
\begin{pgfscope}%
\definecolor{textcolor}{rgb}{0.000000,0.000000,0.000000}%
\pgfsetstrokecolor{textcolor}%
\pgfsetfillcolor{textcolor}%
\pgftext[x=0.412808in,y=2.223275in,left,base]{\color{textcolor}\rmfamily\fontsize{10.000000}{12.000000}\selectfont \(\displaystyle 0.8\)}%
\end{pgfscope}%
\begin{pgfscope}%
\pgfsetbuttcap%
\pgfsetroundjoin%
\definecolor{currentfill}{rgb}{0.000000,0.000000,0.000000}%
\pgfsetfillcolor{currentfill}%
\pgfsetlinewidth{0.803000pt}%
\definecolor{currentstroke}{rgb}{0.000000,0.000000,0.000000}%
\pgfsetstrokecolor{currentstroke}%
\pgfsetdash{}{0pt}%
\pgfsys@defobject{currentmarker}{\pgfqpoint{-0.048611in}{0.000000in}}{\pgfqpoint{0.000000in}{0.000000in}}{%
\pgfpathmoveto{\pgfqpoint{0.000000in}{0.000000in}}%
\pgfpathlineto{\pgfqpoint{-0.048611in}{0.000000in}}%
\pgfusepath{stroke,fill}%
}%
\begin{pgfscope}%
\pgfsys@transformshift{0.687500in}{2.630833in}%
\pgfsys@useobject{currentmarker}{}%
\end{pgfscope}%
\end{pgfscope}%
\begin{pgfscope}%
\definecolor{textcolor}{rgb}{0.000000,0.000000,0.000000}%
\pgfsetstrokecolor{textcolor}%
\pgfsetfillcolor{textcolor}%
\pgftext[x=0.412808in,y=2.582608in,left,base]{\color{textcolor}\rmfamily\fontsize{10.000000}{12.000000}\selectfont \(\displaystyle 1.0\)}%
\end{pgfscope}%
\begin{pgfscope}%
\pgfsetbuttcap%
\pgfsetroundjoin%
\definecolor{currentfill}{rgb}{0.000000,0.000000,0.000000}%
\pgfsetfillcolor{currentfill}%
\pgfsetlinewidth{0.803000pt}%
\definecolor{currentstroke}{rgb}{0.000000,0.000000,0.000000}%
\pgfsetstrokecolor{currentstroke}%
\pgfsetdash{}{0pt}%
\pgfsys@defobject{currentmarker}{\pgfqpoint{-0.048611in}{0.000000in}}{\pgfqpoint{0.000000in}{0.000000in}}{%
\pgfpathmoveto{\pgfqpoint{0.000000in}{0.000000in}}%
\pgfpathlineto{\pgfqpoint{-0.048611in}{0.000000in}}%
\pgfusepath{stroke,fill}%
}%
\begin{pgfscope}%
\pgfsys@transformshift{0.687500in}{2.990167in}%
\pgfsys@useobject{currentmarker}{}%
\end{pgfscope}%
\end{pgfscope}%
\begin{pgfscope}%
\definecolor{textcolor}{rgb}{0.000000,0.000000,0.000000}%
\pgfsetstrokecolor{textcolor}%
\pgfsetfillcolor{textcolor}%
\pgftext[x=0.412808in,y=2.941941in,left,base]{\color{textcolor}\rmfamily\fontsize{10.000000}{12.000000}\selectfont \(\displaystyle 1.2\)}%
\end{pgfscope}%
\begin{pgfscope}%
\definecolor{textcolor}{rgb}{0.000000,0.000000,0.000000}%
\pgfsetstrokecolor{textcolor}%
\pgfsetfillcolor{textcolor}%
\pgftext[x=0.249228in,y=1.732500in,,bottom,rotate=90.000000]{\color{textcolor}\rmfamily\fontsize{10.000000}{12.000000}\selectfont y}%
\end{pgfscope}%
\begin{pgfscope}%
\pgfpathrectangle{\pgfqpoint{0.687500in}{0.385000in}}{\pgfqpoint{4.262500in}{2.695000in}}%
\pgfusepath{clip}%
\pgfsetrectcap%
\pgfsetroundjoin%
\pgfsetlinewidth{1.505625pt}%
\definecolor{currentstroke}{rgb}{0.121569,0.466667,0.705882}%
\pgfsetstrokecolor{currentstroke}%
\pgfsetdash{}{0pt}%
\pgfpathmoveto{\pgfqpoint{0.881250in}{0.903269in}}%
\pgfpathlineto{\pgfqpoint{1.017557in}{0.913644in}}%
\pgfpathlineto{\pgfqpoint{1.134391in}{0.924478in}}%
\pgfpathlineto{\pgfqpoint{1.251225in}{0.937638in}}%
\pgfpathlineto{\pgfqpoint{1.348587in}{0.950877in}}%
\pgfpathlineto{\pgfqpoint{1.426476in}{0.963336in}}%
\pgfpathlineto{\pgfqpoint{1.504366in}{0.977838in}}%
\pgfpathlineto{\pgfqpoint{1.582255in}{0.994839in}}%
\pgfpathlineto{\pgfqpoint{1.640672in}{1.009574in}}%
\pgfpathlineto{\pgfqpoint{1.699089in}{1.026346in}}%
\pgfpathlineto{\pgfqpoint{1.757506in}{1.045528in}}%
\pgfpathlineto{\pgfqpoint{1.815923in}{1.067578in}}%
\pgfpathlineto{\pgfqpoint{1.854868in}{1.084143in}}%
\pgfpathlineto{\pgfqpoint{1.893813in}{1.102428in}}%
\pgfpathlineto{\pgfqpoint{1.932758in}{1.122659in}}%
\pgfpathlineto{\pgfqpoint{1.971702in}{1.145101in}}%
\pgfpathlineto{\pgfqpoint{2.010647in}{1.170055in}}%
\pgfpathlineto{\pgfqpoint{2.049592in}{1.197870in}}%
\pgfpathlineto{\pgfqpoint{2.088536in}{1.228948in}}%
\pgfpathlineto{\pgfqpoint{2.127481in}{1.263748in}}%
\pgfpathlineto{\pgfqpoint{2.166426in}{1.302794in}}%
\pgfpathlineto{\pgfqpoint{2.205371in}{1.346677in}}%
\pgfpathlineto{\pgfqpoint{2.244315in}{1.396056in}}%
\pgfpathlineto{\pgfqpoint{2.283260in}{1.451647in}}%
\pgfpathlineto{\pgfqpoint{2.322205in}{1.514206in}}%
\pgfpathlineto{\pgfqpoint{2.361149in}{1.584486in}}%
\pgfpathlineto{\pgfqpoint{2.400094in}{1.663167in}}%
\pgfpathlineto{\pgfqpoint{2.439039in}{1.750738in}}%
\pgfpathlineto{\pgfqpoint{2.477984in}{1.847321in}}%
\pgfpathlineto{\pgfqpoint{2.516928in}{1.952417in}}%
\pgfpathlineto{\pgfqpoint{2.555873in}{2.064579in}}%
\pgfpathlineto{\pgfqpoint{2.653235in}{2.353617in}}%
\pgfpathlineto{\pgfqpoint{2.672707in}{2.407372in}}%
\pgfpathlineto{\pgfqpoint{2.692180in}{2.457628in}}%
\pgfpathlineto{\pgfqpoint{2.711652in}{2.503331in}}%
\pgfpathlineto{\pgfqpoint{2.731124in}{2.543430in}}%
\pgfpathlineto{\pgfqpoint{2.750597in}{2.576924in}}%
\pgfpathlineto{\pgfqpoint{2.770069in}{2.602918in}}%
\pgfpathlineto{\pgfqpoint{2.789541in}{2.620683in}}%
\pgfpathlineto{\pgfqpoint{2.809014in}{2.629700in}}%
\pgfpathlineto{\pgfqpoint{2.828486in}{2.629700in}}%
\pgfpathlineto{\pgfqpoint{2.847959in}{2.620683in}}%
\pgfpathlineto{\pgfqpoint{2.867431in}{2.602918in}}%
\pgfpathlineto{\pgfqpoint{2.886903in}{2.576924in}}%
\pgfpathlineto{\pgfqpoint{2.906376in}{2.543430in}}%
\pgfpathlineto{\pgfqpoint{2.925848in}{2.503331in}}%
\pgfpathlineto{\pgfqpoint{2.945320in}{2.457628in}}%
\pgfpathlineto{\pgfqpoint{2.964793in}{2.407372in}}%
\pgfpathlineto{\pgfqpoint{3.003737in}{2.297371in}}%
\pgfpathlineto{\pgfqpoint{3.120572in}{1.952417in}}%
\pgfpathlineto{\pgfqpoint{3.159516in}{1.847321in}}%
\pgfpathlineto{\pgfqpoint{3.198461in}{1.750738in}}%
\pgfpathlineto{\pgfqpoint{3.237406in}{1.663167in}}%
\pgfpathlineto{\pgfqpoint{3.276351in}{1.584486in}}%
\pgfpathlineto{\pgfqpoint{3.315295in}{1.514206in}}%
\pgfpathlineto{\pgfqpoint{3.354240in}{1.451647in}}%
\pgfpathlineto{\pgfqpoint{3.393185in}{1.396056in}}%
\pgfpathlineto{\pgfqpoint{3.432129in}{1.346677in}}%
\pgfpathlineto{\pgfqpoint{3.471074in}{1.302794in}}%
\pgfpathlineto{\pgfqpoint{3.510019in}{1.263748in}}%
\pgfpathlineto{\pgfqpoint{3.548964in}{1.228948in}}%
\pgfpathlineto{\pgfqpoint{3.587908in}{1.197870in}}%
\pgfpathlineto{\pgfqpoint{3.626853in}{1.170055in}}%
\pgfpathlineto{\pgfqpoint{3.665798in}{1.145101in}}%
\pgfpathlineto{\pgfqpoint{3.704742in}{1.122659in}}%
\pgfpathlineto{\pgfqpoint{3.743687in}{1.102428in}}%
\pgfpathlineto{\pgfqpoint{3.782632in}{1.084143in}}%
\pgfpathlineto{\pgfqpoint{3.841049in}{1.059877in}}%
\pgfpathlineto{\pgfqpoint{3.899466in}{1.038840in}}%
\pgfpathlineto{\pgfqpoint{3.957883in}{1.020508in}}%
\pgfpathlineto{\pgfqpoint{4.016300in}{1.004452in}}%
\pgfpathlineto{\pgfqpoint{4.074717in}{0.990325in}}%
\pgfpathlineto{\pgfqpoint{4.152607in}{0.973998in}}%
\pgfpathlineto{\pgfqpoint{4.230496in}{0.960045in}}%
\pgfpathlineto{\pgfqpoint{4.327858in}{0.945299in}}%
\pgfpathlineto{\pgfqpoint{4.425220in}{0.932955in}}%
\pgfpathlineto{\pgfqpoint{4.542054in}{0.920637in}}%
\pgfpathlineto{\pgfqpoint{4.678361in}{0.908933in}}%
\pgfpathlineto{\pgfqpoint{4.756250in}{0.903269in}}%
\pgfpathlineto{\pgfqpoint{4.756250in}{0.903269in}}%
\pgfusepath{stroke}%
\end{pgfscope}%
\begin{pgfscope}%
\pgfpathrectangle{\pgfqpoint{0.687500in}{0.385000in}}{\pgfqpoint{4.262500in}{2.695000in}}%
\pgfusepath{clip}%
\pgfsetbuttcap%
\pgfsetroundjoin%
\definecolor{currentfill}{rgb}{1.000000,0.498039,0.054902}%
\pgfsetfillcolor{currentfill}%
\pgfsetlinewidth{1.003750pt}%
\definecolor{currentstroke}{rgb}{1.000000,0.498039,0.054902}%
\pgfsetstrokecolor{currentstroke}%
\pgfsetdash{}{0pt}%
\pgfsys@defobject{currentmarker}{\pgfqpoint{-0.020833in}{-0.020833in}}{\pgfqpoint{0.020833in}{0.020833in}}{%
\pgfpathmoveto{\pgfqpoint{0.000000in}{-0.020833in}}%
\pgfpathcurveto{\pgfqpoint{0.005525in}{-0.020833in}}{\pgfqpoint{0.010825in}{-0.018638in}}{\pgfqpoint{0.014731in}{-0.014731in}}%
\pgfpathcurveto{\pgfqpoint{0.018638in}{-0.010825in}}{\pgfqpoint{0.020833in}{-0.005525in}}{\pgfqpoint{0.020833in}{0.000000in}}%
\pgfpathcurveto{\pgfqpoint{0.020833in}{0.005525in}}{\pgfqpoint{0.018638in}{0.010825in}}{\pgfqpoint{0.014731in}{0.014731in}}%
\pgfpathcurveto{\pgfqpoint{0.010825in}{0.018638in}}{\pgfqpoint{0.005525in}{0.020833in}}{\pgfqpoint{0.000000in}{0.020833in}}%
\pgfpathcurveto{\pgfqpoint{-0.005525in}{0.020833in}}{\pgfqpoint{-0.010825in}{0.018638in}}{\pgfqpoint{-0.014731in}{0.014731in}}%
\pgfpathcurveto{\pgfqpoint{-0.018638in}{0.010825in}}{\pgfqpoint{-0.020833in}{0.005525in}}{\pgfqpoint{-0.020833in}{0.000000in}}%
\pgfpathcurveto{\pgfqpoint{-0.020833in}{-0.005525in}}{\pgfqpoint{-0.018638in}{-0.010825in}}{\pgfqpoint{-0.014731in}{-0.014731in}}%
\pgfpathcurveto{\pgfqpoint{-0.010825in}{-0.018638in}}{\pgfqpoint{-0.005525in}{-0.020833in}}{\pgfqpoint{0.000000in}{-0.020833in}}%
\pgfpathclose%
\pgfusepath{stroke,fill}%
}%
\begin{pgfscope}%
\pgfsys@transformshift{4.690231in}{0.908027in}%
\pgfsys@useobject{currentmarker}{}%
\end{pgfscope}%
\begin{pgfscope}%
\pgfsys@transformshift{4.188769in}{0.967253in}%
\pgfsys@useobject{currentmarker}{}%
\end{pgfscope}%
\begin{pgfscope}%
\pgfsys@transformshift{3.320212in}{1.505897in}%
\pgfsys@useobject{currentmarker}{}%
\end{pgfscope}%
\begin{pgfscope}%
\pgfsys@transformshift{2.317288in}{1.505897in}%
\pgfsys@useobject{currentmarker}{}%
\end{pgfscope}%
\begin{pgfscope}%
\pgfsys@transformshift{1.448731in}{0.967253in}%
\pgfsys@useobject{currentmarker}{}%
\end{pgfscope}%
\begin{pgfscope}%
\pgfsys@transformshift{0.947269in}{0.908027in}%
\pgfsys@useobject{currentmarker}{}%
\end{pgfscope}%
\end{pgfscope}%
\begin{pgfscope}%
\pgfpathrectangle{\pgfqpoint{0.687500in}{0.385000in}}{\pgfqpoint{4.262500in}{2.695000in}}%
\pgfusepath{clip}%
\pgfsetrectcap%
\pgfsetroundjoin%
\pgfsetlinewidth{1.505625pt}%
\definecolor{currentstroke}{rgb}{0.172549,0.627451,0.172549}%
\pgfsetstrokecolor{currentstroke}%
\pgfsetdash{}{0pt}%
\pgfpathmoveto{\pgfqpoint{0.881250in}{0.941684in}}%
\pgfpathlineto{\pgfqpoint{0.920195in}{0.920305in}}%
\pgfpathlineto{\pgfqpoint{0.959139in}{0.903287in}}%
\pgfpathlineto{\pgfqpoint{0.998084in}{0.890388in}}%
\pgfpathlineto{\pgfqpoint{1.037029in}{0.881373in}}%
\pgfpathlineto{\pgfqpoint{1.075974in}{0.876008in}}%
\pgfpathlineto{\pgfqpoint{1.114918in}{0.874068in}}%
\pgfpathlineto{\pgfqpoint{1.153863in}{0.875332in}}%
\pgfpathlineto{\pgfqpoint{1.192808in}{0.879581in}}%
\pgfpathlineto{\pgfqpoint{1.231753in}{0.886605in}}%
\pgfpathlineto{\pgfqpoint{1.270697in}{0.896197in}}%
\pgfpathlineto{\pgfqpoint{1.309642in}{0.908156in}}%
\pgfpathlineto{\pgfqpoint{1.348587in}{0.922285in}}%
\pgfpathlineto{\pgfqpoint{1.387531in}{0.938392in}}%
\pgfpathlineto{\pgfqpoint{1.426476in}{0.956291in}}%
\pgfpathlineto{\pgfqpoint{1.484893in}{0.986103in}}%
\pgfpathlineto{\pgfqpoint{1.543310in}{1.018947in}}%
\pgfpathlineto{\pgfqpoint{1.601727in}{1.054257in}}%
\pgfpathlineto{\pgfqpoint{1.679617in}{1.104243in}}%
\pgfpathlineto{\pgfqpoint{1.776979in}{1.169703in}}%
\pgfpathlineto{\pgfqpoint{2.010647in}{1.328240in}}%
\pgfpathlineto{\pgfqpoint{2.088536in}{1.378186in}}%
\pgfpathlineto{\pgfqpoint{2.166426in}{1.425298in}}%
\pgfpathlineto{\pgfqpoint{2.224843in}{1.458339in}}%
\pgfpathlineto{\pgfqpoint{2.283260in}{1.489114in}}%
\pgfpathlineto{\pgfqpoint{2.341677in}{1.517377in}}%
\pgfpathlineto{\pgfqpoint{2.400094in}{1.542908in}}%
\pgfpathlineto{\pgfqpoint{2.458511in}{1.565512in}}%
\pgfpathlineto{\pgfqpoint{2.516928in}{1.585022in}}%
\pgfpathlineto{\pgfqpoint{2.575345in}{1.601292in}}%
\pgfpathlineto{\pgfqpoint{2.633763in}{1.614205in}}%
\pgfpathlineto{\pgfqpoint{2.672707in}{1.620901in}}%
\pgfpathlineto{\pgfqpoint{2.711652in}{1.626042in}}%
\pgfpathlineto{\pgfqpoint{2.750597in}{1.629612in}}%
\pgfpathlineto{\pgfqpoint{2.789541in}{1.631599in}}%
\pgfpathlineto{\pgfqpoint{2.828486in}{1.631996in}}%
\pgfpathlineto{\pgfqpoint{2.867431in}{1.630804in}}%
\pgfpathlineto{\pgfqpoint{2.906376in}{1.628024in}}%
\pgfpathlineto{\pgfqpoint{2.945320in}{1.623667in}}%
\pgfpathlineto{\pgfqpoint{2.984265in}{1.617746in}}%
\pgfpathlineto{\pgfqpoint{3.023210in}{1.610280in}}%
\pgfpathlineto{\pgfqpoint{3.081627in}{1.596236in}}%
\pgfpathlineto{\pgfqpoint{3.140044in}{1.578872in}}%
\pgfpathlineto{\pgfqpoint{3.198461in}{1.558314in}}%
\pgfpathlineto{\pgfqpoint{3.256878in}{1.534714in}}%
\pgfpathlineto{\pgfqpoint{3.315295in}{1.508249in}}%
\pgfpathlineto{\pgfqpoint{3.373712in}{1.479124in}}%
\pgfpathlineto{\pgfqpoint{3.432129in}{1.447565in}}%
\pgfpathlineto{\pgfqpoint{3.490546in}{1.413826in}}%
\pgfpathlineto{\pgfqpoint{3.568436in}{1.365936in}}%
\pgfpathlineto{\pgfqpoint{3.646325in}{1.315401in}}%
\pgfpathlineto{\pgfqpoint{3.763160in}{1.236437in}}%
\pgfpathlineto{\pgfqpoint{3.957883in}{1.104243in}}%
\pgfpathlineto{\pgfqpoint{4.035773in}{1.054257in}}%
\pgfpathlineto{\pgfqpoint{4.094190in}{1.018947in}}%
\pgfpathlineto{\pgfqpoint{4.152607in}{0.986103in}}%
\pgfpathlineto{\pgfqpoint{4.211024in}{0.956291in}}%
\pgfpathlineto{\pgfqpoint{4.249969in}{0.938392in}}%
\pgfpathlineto{\pgfqpoint{4.288913in}{0.922285in}}%
\pgfpathlineto{\pgfqpoint{4.327858in}{0.908156in}}%
\pgfpathlineto{\pgfqpoint{4.366803in}{0.896197in}}%
\pgfpathlineto{\pgfqpoint{4.405747in}{0.886605in}}%
\pgfpathlineto{\pgfqpoint{4.444692in}{0.879581in}}%
\pgfpathlineto{\pgfqpoint{4.483637in}{0.875332in}}%
\pgfpathlineto{\pgfqpoint{4.522582in}{0.874068in}}%
\pgfpathlineto{\pgfqpoint{4.561526in}{0.876008in}}%
\pgfpathlineto{\pgfqpoint{4.600471in}{0.881373in}}%
\pgfpathlineto{\pgfqpoint{4.639416in}{0.890388in}}%
\pgfpathlineto{\pgfqpoint{4.678361in}{0.903287in}}%
\pgfpathlineto{\pgfqpoint{4.717305in}{0.920305in}}%
\pgfpathlineto{\pgfqpoint{4.756250in}{0.941684in}}%
\pgfpathlineto{\pgfqpoint{4.756250in}{0.941684in}}%
\pgfusepath{stroke}%
\end{pgfscope}%
\begin{pgfscope}%
\pgfpathrectangle{\pgfqpoint{0.687500in}{0.385000in}}{\pgfqpoint{4.262500in}{2.695000in}}%
\pgfusepath{clip}%
\pgfsetbuttcap%
\pgfsetroundjoin%
\definecolor{currentfill}{rgb}{0.839216,0.152941,0.156863}%
\pgfsetfillcolor{currentfill}%
\pgfsetlinewidth{1.003750pt}%
\definecolor{currentstroke}{rgb}{0.839216,0.152941,0.156863}%
\pgfsetstrokecolor{currentstroke}%
\pgfsetdash{}{0pt}%
\pgfsys@defobject{currentmarker}{\pgfqpoint{-0.020833in}{-0.020833in}}{\pgfqpoint{0.020833in}{0.020833in}}{%
\pgfpathmoveto{\pgfqpoint{0.000000in}{-0.020833in}}%
\pgfpathcurveto{\pgfqpoint{0.005525in}{-0.020833in}}{\pgfqpoint{0.010825in}{-0.018638in}}{\pgfqpoint{0.014731in}{-0.014731in}}%
\pgfpathcurveto{\pgfqpoint{0.018638in}{-0.010825in}}{\pgfqpoint{0.020833in}{-0.005525in}}{\pgfqpoint{0.020833in}{0.000000in}}%
\pgfpathcurveto{\pgfqpoint{0.020833in}{0.005525in}}{\pgfqpoint{0.018638in}{0.010825in}}{\pgfqpoint{0.014731in}{0.014731in}}%
\pgfpathcurveto{\pgfqpoint{0.010825in}{0.018638in}}{\pgfqpoint{0.005525in}{0.020833in}}{\pgfqpoint{0.000000in}{0.020833in}}%
\pgfpathcurveto{\pgfqpoint{-0.005525in}{0.020833in}}{\pgfqpoint{-0.010825in}{0.018638in}}{\pgfqpoint{-0.014731in}{0.014731in}}%
\pgfpathcurveto{\pgfqpoint{-0.018638in}{0.010825in}}{\pgfqpoint{-0.020833in}{0.005525in}}{\pgfqpoint{-0.020833in}{0.000000in}}%
\pgfpathcurveto{\pgfqpoint{-0.020833in}{-0.005525in}}{\pgfqpoint{-0.018638in}{-0.010825in}}{\pgfqpoint{-0.014731in}{-0.014731in}}%
\pgfpathcurveto{\pgfqpoint{-0.010825in}{-0.018638in}}{\pgfqpoint{-0.005525in}{-0.020833in}}{\pgfqpoint{0.000000in}{-0.020833in}}%
\pgfpathclose%
\pgfusepath{stroke,fill}%
}%
\begin{pgfscope}%
\pgfsys@transformshift{4.719021in}{0.905894in}%
\pgfsys@useobject{currentmarker}{}%
\end{pgfscope}%
\begin{pgfscope}%
\pgfsys@transformshift{4.429722in}{0.932434in}%
\pgfsys@useobject{currentmarker}{}%
\end{pgfscope}%
\begin{pgfscope}%
\pgfsys@transformshift{3.895167in}{1.040290in}%
\pgfsys@useobject{currentmarker}{}%
\end{pgfscope}%
\begin{pgfscope}%
\pgfsys@transformshift{3.196737in}{1.754823in}%
\pgfsys@useobject{currentmarker}{}%
\end{pgfscope}%
\begin{pgfscope}%
\pgfsys@transformshift{2.440763in}{1.754823in}%
\pgfsys@useobject{currentmarker}{}%
\end{pgfscope}%
\begin{pgfscope}%
\pgfsys@transformshift{1.742333in}{1.040290in}%
\pgfsys@useobject{currentmarker}{}%
\end{pgfscope}%
\begin{pgfscope}%
\pgfsys@transformshift{1.207778in}{0.932434in}%
\pgfsys@useobject{currentmarker}{}%
\end{pgfscope}%
\begin{pgfscope}%
\pgfsys@transformshift{0.918479in}{0.905894in}%
\pgfsys@useobject{currentmarker}{}%
\end{pgfscope}%
\end{pgfscope}%
\begin{pgfscope}%
\pgfpathrectangle{\pgfqpoint{0.687500in}{0.385000in}}{\pgfqpoint{4.262500in}{2.695000in}}%
\pgfusepath{clip}%
\pgfsetrectcap%
\pgfsetroundjoin%
\pgfsetlinewidth{1.505625pt}%
\definecolor{currentstroke}{rgb}{0.580392,0.403922,0.741176}%
\pgfsetstrokecolor{currentstroke}%
\pgfsetdash{}{0pt}%
\pgfpathmoveto{\pgfqpoint{0.881250in}{0.876199in}}%
\pgfpathlineto{\pgfqpoint{0.900722in}{0.892977in}}%
\pgfpathlineto{\pgfqpoint{0.920195in}{0.907030in}}%
\pgfpathlineto{\pgfqpoint{0.939667in}{0.918600in}}%
\pgfpathlineto{\pgfqpoint{0.959139in}{0.927920in}}%
\pgfpathlineto{\pgfqpoint{0.978612in}{0.935212in}}%
\pgfpathlineto{\pgfqpoint{0.998084in}{0.940686in}}%
\pgfpathlineto{\pgfqpoint{1.017557in}{0.944543in}}%
\pgfpathlineto{\pgfqpoint{1.037029in}{0.946975in}}%
\pgfpathlineto{\pgfqpoint{1.075974in}{0.948279in}}%
\pgfpathlineto{\pgfqpoint{1.114918in}{0.945937in}}%
\pgfpathlineto{\pgfqpoint{1.153863in}{0.941148in}}%
\pgfpathlineto{\pgfqpoint{1.231753in}{0.928349in}}%
\pgfpathlineto{\pgfqpoint{1.290170in}{0.919320in}}%
\pgfpathlineto{\pgfqpoint{1.329114in}{0.914889in}}%
\pgfpathlineto{\pgfqpoint{1.368059in}{0.912378in}}%
\pgfpathlineto{\pgfqpoint{1.407004in}{0.912236in}}%
\pgfpathlineto{\pgfqpoint{1.445948in}{0.914821in}}%
\pgfpathlineto{\pgfqpoint{1.484893in}{0.920406in}}%
\pgfpathlineto{\pgfqpoint{1.523838in}{0.929189in}}%
\pgfpathlineto{\pgfqpoint{1.562783in}{0.941295in}}%
\pgfpathlineto{\pgfqpoint{1.601727in}{0.956782in}}%
\pgfpathlineto{\pgfqpoint{1.640672in}{0.975651in}}%
\pgfpathlineto{\pgfqpoint{1.679617in}{0.997844in}}%
\pgfpathlineto{\pgfqpoint{1.718562in}{1.023258in}}%
\pgfpathlineto{\pgfqpoint{1.757506in}{1.051742in}}%
\pgfpathlineto{\pgfqpoint{1.796451in}{1.083107in}}%
\pgfpathlineto{\pgfqpoint{1.835396in}{1.117130in}}%
\pgfpathlineto{\pgfqpoint{1.874340in}{1.153556in}}%
\pgfpathlineto{\pgfqpoint{1.932758in}{1.212082in}}%
\pgfpathlineto{\pgfqpoint{1.991175in}{1.274350in}}%
\pgfpathlineto{\pgfqpoint{2.069064in}{1.361267in}}%
\pgfpathlineto{\pgfqpoint{2.244315in}{1.559560in}}%
\pgfpathlineto{\pgfqpoint{2.302732in}{1.622335in}}%
\pgfpathlineto{\pgfqpoint{2.361149in}{1.681586in}}%
\pgfpathlineto{\pgfqpoint{2.400094in}{1.718618in}}%
\pgfpathlineto{\pgfqpoint{2.439039in}{1.753344in}}%
\pgfpathlineto{\pgfqpoint{2.477984in}{1.785498in}}%
\pgfpathlineto{\pgfqpoint{2.516928in}{1.814838in}}%
\pgfpathlineto{\pgfqpoint{2.555873in}{1.841141in}}%
\pgfpathlineto{\pgfqpoint{2.594818in}{1.864210in}}%
\pgfpathlineto{\pgfqpoint{2.633763in}{1.883872in}}%
\pgfpathlineto{\pgfqpoint{2.672707in}{1.899980in}}%
\pgfpathlineto{\pgfqpoint{2.711652in}{1.912413in}}%
\pgfpathlineto{\pgfqpoint{2.750597in}{1.921080in}}%
\pgfpathlineto{\pgfqpoint{2.789541in}{1.925916in}}%
\pgfpathlineto{\pgfqpoint{2.828486in}{1.926885in}}%
\pgfpathlineto{\pgfqpoint{2.867431in}{1.923980in}}%
\pgfpathlineto{\pgfqpoint{2.906376in}{1.917222in}}%
\pgfpathlineto{\pgfqpoint{2.945320in}{1.906662in}}%
\pgfpathlineto{\pgfqpoint{2.984265in}{1.892378in}}%
\pgfpathlineto{\pgfqpoint{3.023210in}{1.874476in}}%
\pgfpathlineto{\pgfqpoint{3.062155in}{1.853091in}}%
\pgfpathlineto{\pgfqpoint{3.101099in}{1.828382in}}%
\pgfpathlineto{\pgfqpoint{3.140044in}{1.800534in}}%
\pgfpathlineto{\pgfqpoint{3.178989in}{1.769758in}}%
\pgfpathlineto{\pgfqpoint{3.217933in}{1.736286in}}%
\pgfpathlineto{\pgfqpoint{3.256878in}{1.700373in}}%
\pgfpathlineto{\pgfqpoint{3.315295in}{1.642529in}}%
\pgfpathlineto{\pgfqpoint{3.373712in}{1.580810in}}%
\pgfpathlineto{\pgfqpoint{3.451602in}{1.494355in}}%
\pgfpathlineto{\pgfqpoint{3.646325in}{1.274350in}}%
\pgfpathlineto{\pgfqpoint{3.704742in}{1.212082in}}%
\pgfpathlineto{\pgfqpoint{3.763160in}{1.153556in}}%
\pgfpathlineto{\pgfqpoint{3.802104in}{1.117130in}}%
\pgfpathlineto{\pgfqpoint{3.841049in}{1.083107in}}%
\pgfpathlineto{\pgfqpoint{3.879994in}{1.051742in}}%
\pgfpathlineto{\pgfqpoint{3.918938in}{1.023258in}}%
\pgfpathlineto{\pgfqpoint{3.957883in}{0.997844in}}%
\pgfpathlineto{\pgfqpoint{3.996828in}{0.975651in}}%
\pgfpathlineto{\pgfqpoint{4.035773in}{0.956782in}}%
\pgfpathlineto{\pgfqpoint{4.074717in}{0.941295in}}%
\pgfpathlineto{\pgfqpoint{4.113662in}{0.929189in}}%
\pgfpathlineto{\pgfqpoint{4.152607in}{0.920406in}}%
\pgfpathlineto{\pgfqpoint{4.191552in}{0.914821in}}%
\pgfpathlineto{\pgfqpoint{4.230496in}{0.912236in}}%
\pgfpathlineto{\pgfqpoint{4.269441in}{0.912378in}}%
\pgfpathlineto{\pgfqpoint{4.308386in}{0.914889in}}%
\pgfpathlineto{\pgfqpoint{4.347330in}{0.919320in}}%
\pgfpathlineto{\pgfqpoint{4.405747in}{0.928349in}}%
\pgfpathlineto{\pgfqpoint{4.483637in}{0.941148in}}%
\pgfpathlineto{\pgfqpoint{4.522582in}{0.945937in}}%
\pgfpathlineto{\pgfqpoint{4.561526in}{0.948279in}}%
\pgfpathlineto{\pgfqpoint{4.600471in}{0.946975in}}%
\pgfpathlineto{\pgfqpoint{4.619943in}{0.944543in}}%
\pgfpathlineto{\pgfqpoint{4.639416in}{0.940686in}}%
\pgfpathlineto{\pgfqpoint{4.658888in}{0.935212in}}%
\pgfpathlineto{\pgfqpoint{4.678361in}{0.927920in}}%
\pgfpathlineto{\pgfqpoint{4.697833in}{0.918600in}}%
\pgfpathlineto{\pgfqpoint{4.717305in}{0.907030in}}%
\pgfpathlineto{\pgfqpoint{4.736778in}{0.892977in}}%
\pgfpathlineto{\pgfqpoint{4.756250in}{0.876199in}}%
\pgfpathlineto{\pgfqpoint{4.756250in}{0.876199in}}%
\pgfusepath{stroke}%
\end{pgfscope}%
\begin{pgfscope}%
\pgfpathrectangle{\pgfqpoint{0.687500in}{0.385000in}}{\pgfqpoint{4.262500in}{2.695000in}}%
\pgfusepath{clip}%
\pgfsetbuttcap%
\pgfsetroundjoin%
\definecolor{currentfill}{rgb}{0.549020,0.337255,0.294118}%
\pgfsetfillcolor{currentfill}%
\pgfsetlinewidth{1.003750pt}%
\definecolor{currentstroke}{rgb}{0.549020,0.337255,0.294118}%
\pgfsetstrokecolor{currentstroke}%
\pgfsetdash{}{0pt}%
\pgfsys@defobject{currentmarker}{\pgfqpoint{-0.020833in}{-0.020833in}}{\pgfqpoint{0.020833in}{0.020833in}}{%
\pgfpathmoveto{\pgfqpoint{0.000000in}{-0.020833in}}%
\pgfpathcurveto{\pgfqpoint{0.005525in}{-0.020833in}}{\pgfqpoint{0.010825in}{-0.018638in}}{\pgfqpoint{0.014731in}{-0.014731in}}%
\pgfpathcurveto{\pgfqpoint{0.018638in}{-0.010825in}}{\pgfqpoint{0.020833in}{-0.005525in}}{\pgfqpoint{0.020833in}{0.000000in}}%
\pgfpathcurveto{\pgfqpoint{0.020833in}{0.005525in}}{\pgfqpoint{0.018638in}{0.010825in}}{\pgfqpoint{0.014731in}{0.014731in}}%
\pgfpathcurveto{\pgfqpoint{0.010825in}{0.018638in}}{\pgfqpoint{0.005525in}{0.020833in}}{\pgfqpoint{0.000000in}{0.020833in}}%
\pgfpathcurveto{\pgfqpoint{-0.005525in}{0.020833in}}{\pgfqpoint{-0.010825in}{0.018638in}}{\pgfqpoint{-0.014731in}{0.014731in}}%
\pgfpathcurveto{\pgfqpoint{-0.018638in}{0.010825in}}{\pgfqpoint{-0.020833in}{0.005525in}}{\pgfqpoint{-0.020833in}{0.000000in}}%
\pgfpathcurveto{\pgfqpoint{-0.020833in}{-0.005525in}}{\pgfqpoint{-0.018638in}{-0.010825in}}{\pgfqpoint{-0.014731in}{-0.014731in}}%
\pgfpathcurveto{\pgfqpoint{-0.010825in}{-0.018638in}}{\pgfqpoint{-0.005525in}{-0.020833in}}{\pgfqpoint{0.000000in}{-0.020833in}}%
\pgfpathclose%
\pgfusepath{stroke,fill}%
}%
\begin{pgfscope}%
\pgfsys@transformshift{4.746920in}{0.903914in}%
\pgfsys@useobject{currentmarker}{}%
\end{pgfscope}%
\begin{pgfscope}%
\pgfsys@transformshift{4.672822in}{0.909362in}%
\pgfsys@useobject{currentmarker}{}%
\end{pgfscope}%
\begin{pgfscope}%
\pgfsys@transformshift{4.527472in}{0.922046in}%
\pgfsys@useobject{currentmarker}{}%
\end{pgfscope}%
\begin{pgfscope}%
\pgfsys@transformshift{4.316458in}{0.946891in}%
\pgfsys@useobject{currentmarker}{}%
\end{pgfscope}%
\begin{pgfscope}%
\pgfsys@transformshift{4.047887in}{0.996594in}%
\pgfsys@useobject{currentmarker}{}%
\end{pgfscope}%
\begin{pgfscope}%
\pgfsys@transformshift{3.732081in}{1.108242in}%
\pgfsys@useobject{currentmarker}{}%
\end{pgfscope}%
\begin{pgfscope}%
\pgfsys@transformshift{3.381177in}{1.412500in}%
\pgfsys@useobject{currentmarker}{}%
\end{pgfscope}%
\begin{pgfscope}%
\pgfsys@transformshift{3.008658in}{2.282876in}%
\pgfsys@useobject{currentmarker}{}%
\end{pgfscope}%
\begin{pgfscope}%
\pgfsys@transformshift{2.628842in}{2.282876in}%
\pgfsys@useobject{currentmarker}{}%
\end{pgfscope}%
\begin{pgfscope}%
\pgfsys@transformshift{2.256323in}{1.412500in}%
\pgfsys@useobject{currentmarker}{}%
\end{pgfscope}%
\begin{pgfscope}%
\pgfsys@transformshift{1.905419in}{1.108242in}%
\pgfsys@useobject{currentmarker}{}%
\end{pgfscope}%
\begin{pgfscope}%
\pgfsys@transformshift{1.589613in}{0.996594in}%
\pgfsys@useobject{currentmarker}{}%
\end{pgfscope}%
\begin{pgfscope}%
\pgfsys@transformshift{1.321042in}{0.946891in}%
\pgfsys@useobject{currentmarker}{}%
\end{pgfscope}%
\begin{pgfscope}%
\pgfsys@transformshift{1.110028in}{0.922046in}%
\pgfsys@useobject{currentmarker}{}%
\end{pgfscope}%
\begin{pgfscope}%
\pgfsys@transformshift{0.964678in}{0.909362in}%
\pgfsys@useobject{currentmarker}{}%
\end{pgfscope}%
\begin{pgfscope}%
\pgfsys@transformshift{0.890580in}{0.903914in}%
\pgfsys@useobject{currentmarker}{}%
\end{pgfscope}%
\end{pgfscope}%
\begin{pgfscope}%
\pgfpathrectangle{\pgfqpoint{0.687500in}{0.385000in}}{\pgfqpoint{4.262500in}{2.695000in}}%
\pgfusepath{clip}%
\pgfsetrectcap%
\pgfsetroundjoin%
\pgfsetlinewidth{1.505625pt}%
\definecolor{currentstroke}{rgb}{0.890196,0.466667,0.760784}%
\pgfsetstrokecolor{currentstroke}%
\pgfsetdash{}{0pt}%
\pgfpathmoveto{\pgfqpoint{0.881250in}{0.897526in}}%
\pgfpathlineto{\pgfqpoint{0.900722in}{0.908394in}}%
\pgfpathlineto{\pgfqpoint{0.920195in}{0.911975in}}%
\pgfpathlineto{\pgfqpoint{0.939667in}{0.911710in}}%
\pgfpathlineto{\pgfqpoint{0.998084in}{0.907070in}}%
\pgfpathlineto{\pgfqpoint{1.017557in}{0.907237in}}%
\pgfpathlineto{\pgfqpoint{1.037029in}{0.908673in}}%
\pgfpathlineto{\pgfqpoint{1.075974in}{0.914724in}}%
\pgfpathlineto{\pgfqpoint{1.192808in}{0.938552in}}%
\pgfpathlineto{\pgfqpoint{1.231753in}{0.943147in}}%
\pgfpathlineto{\pgfqpoint{1.270697in}{0.945590in}}%
\pgfpathlineto{\pgfqpoint{1.407004in}{0.950840in}}%
\pgfpathlineto{\pgfqpoint{1.445948in}{0.955741in}}%
\pgfpathlineto{\pgfqpoint{1.484893in}{0.963373in}}%
\pgfpathlineto{\pgfqpoint{1.523838in}{0.973778in}}%
\pgfpathlineto{\pgfqpoint{1.562783in}{0.986629in}}%
\pgfpathlineto{\pgfqpoint{1.621200in}{1.009100in}}%
\pgfpathlineto{\pgfqpoint{1.738034in}{1.055834in}}%
\pgfpathlineto{\pgfqpoint{1.796451in}{1.076046in}}%
\pgfpathlineto{\pgfqpoint{1.854868in}{1.093685in}}%
\pgfpathlineto{\pgfqpoint{1.932758in}{1.116613in}}%
\pgfpathlineto{\pgfqpoint{1.971702in}{1.130202in}}%
\pgfpathlineto{\pgfqpoint{2.010647in}{1.147025in}}%
\pgfpathlineto{\pgfqpoint{2.030119in}{1.157115in}}%
\pgfpathlineto{\pgfqpoint{2.049592in}{1.168572in}}%
\pgfpathlineto{\pgfqpoint{2.069064in}{1.181578in}}%
\pgfpathlineto{\pgfqpoint{2.088536in}{1.196303in}}%
\pgfpathlineto{\pgfqpoint{2.108009in}{1.212909in}}%
\pgfpathlineto{\pgfqpoint{2.127481in}{1.231539in}}%
\pgfpathlineto{\pgfqpoint{2.146954in}{1.252318in}}%
\pgfpathlineto{\pgfqpoint{2.166426in}{1.275352in}}%
\pgfpathlineto{\pgfqpoint{2.185898in}{1.300718in}}%
\pgfpathlineto{\pgfqpoint{2.205371in}{1.328473in}}%
\pgfpathlineto{\pgfqpoint{2.224843in}{1.358642in}}%
\pgfpathlineto{\pgfqpoint{2.244315in}{1.391221in}}%
\pgfpathlineto{\pgfqpoint{2.263788in}{1.426177in}}%
\pgfpathlineto{\pgfqpoint{2.302732in}{1.502926in}}%
\pgfpathlineto{\pgfqpoint{2.341677in}{1.587993in}}%
\pgfpathlineto{\pgfqpoint{2.380622in}{1.679998in}}%
\pgfpathlineto{\pgfqpoint{2.439039in}{1.826904in}}%
\pgfpathlineto{\pgfqpoint{2.536401in}{2.075592in}}%
\pgfpathlineto{\pgfqpoint{2.575345in}{2.168554in}}%
\pgfpathlineto{\pgfqpoint{2.614290in}{2.253641in}}%
\pgfpathlineto{\pgfqpoint{2.633763in}{2.292408in}}%
\pgfpathlineto{\pgfqpoint{2.653235in}{2.328246in}}%
\pgfpathlineto{\pgfqpoint{2.672707in}{2.360872in}}%
\pgfpathlineto{\pgfqpoint{2.692180in}{2.390024in}}%
\pgfpathlineto{\pgfqpoint{2.711652in}{2.415468in}}%
\pgfpathlineto{\pgfqpoint{2.731124in}{2.436997in}}%
\pgfpathlineto{\pgfqpoint{2.750597in}{2.454437in}}%
\pgfpathlineto{\pgfqpoint{2.770069in}{2.467644in}}%
\pgfpathlineto{\pgfqpoint{2.789541in}{2.476509in}}%
\pgfpathlineto{\pgfqpoint{2.809014in}{2.480960in}}%
\pgfpathlineto{\pgfqpoint{2.828486in}{2.480960in}}%
\pgfpathlineto{\pgfqpoint{2.847959in}{2.476509in}}%
\pgfpathlineto{\pgfqpoint{2.867431in}{2.467644in}}%
\pgfpathlineto{\pgfqpoint{2.886903in}{2.454437in}}%
\pgfpathlineto{\pgfqpoint{2.906376in}{2.436997in}}%
\pgfpathlineto{\pgfqpoint{2.925848in}{2.415468in}}%
\pgfpathlineto{\pgfqpoint{2.945320in}{2.390024in}}%
\pgfpathlineto{\pgfqpoint{2.964793in}{2.360872in}}%
\pgfpathlineto{\pgfqpoint{2.984265in}{2.328246in}}%
\pgfpathlineto{\pgfqpoint{3.003737in}{2.292408in}}%
\pgfpathlineto{\pgfqpoint{3.023210in}{2.253641in}}%
\pgfpathlineto{\pgfqpoint{3.062155in}{2.168554in}}%
\pgfpathlineto{\pgfqpoint{3.101099in}{2.075592in}}%
\pgfpathlineto{\pgfqpoint{3.159516in}{1.927436in}}%
\pgfpathlineto{\pgfqpoint{3.237406in}{1.728046in}}%
\pgfpathlineto{\pgfqpoint{3.276351in}{1.633233in}}%
\pgfpathlineto{\pgfqpoint{3.315295in}{1.544495in}}%
\pgfpathlineto{\pgfqpoint{3.354240in}{1.463444in}}%
\pgfpathlineto{\pgfqpoint{3.393185in}{1.391221in}}%
\pgfpathlineto{\pgfqpoint{3.412657in}{1.358642in}}%
\pgfpathlineto{\pgfqpoint{3.432129in}{1.328473in}}%
\pgfpathlineto{\pgfqpoint{3.451602in}{1.300718in}}%
\pgfpathlineto{\pgfqpoint{3.471074in}{1.275352in}}%
\pgfpathlineto{\pgfqpoint{3.490546in}{1.252318in}}%
\pgfpathlineto{\pgfqpoint{3.510019in}{1.231539in}}%
\pgfpathlineto{\pgfqpoint{3.529491in}{1.212909in}}%
\pgfpathlineto{\pgfqpoint{3.548964in}{1.196303in}}%
\pgfpathlineto{\pgfqpoint{3.568436in}{1.181578in}}%
\pgfpathlineto{\pgfqpoint{3.587908in}{1.168572in}}%
\pgfpathlineto{\pgfqpoint{3.607381in}{1.157115in}}%
\pgfpathlineto{\pgfqpoint{3.646325in}{1.138117in}}%
\pgfpathlineto{\pgfqpoint{3.685270in}{1.123095in}}%
\pgfpathlineto{\pgfqpoint{3.724215in}{1.110586in}}%
\pgfpathlineto{\pgfqpoint{3.860521in}{1.069633in}}%
\pgfpathlineto{\pgfqpoint{3.918938in}{1.048469in}}%
\pgfpathlineto{\pgfqpoint{3.996828in}{1.017046in}}%
\pgfpathlineto{\pgfqpoint{4.055245in}{0.993788in}}%
\pgfpathlineto{\pgfqpoint{4.094190in}{0.979931in}}%
\pgfpathlineto{\pgfqpoint{4.133134in}{0.968241in}}%
\pgfpathlineto{\pgfqpoint{4.172079in}{0.959203in}}%
\pgfpathlineto{\pgfqpoint{4.211024in}{0.952968in}}%
\pgfpathlineto{\pgfqpoint{4.249969in}{0.949288in}}%
\pgfpathlineto{\pgfqpoint{4.308386in}{0.947049in}}%
\pgfpathlineto{\pgfqpoint{4.386275in}{0.944599in}}%
\pgfpathlineto{\pgfqpoint{4.425220in}{0.941147in}}%
\pgfpathlineto{\pgfqpoint{4.464165in}{0.935366in}}%
\pgfpathlineto{\pgfqpoint{4.522582in}{0.923152in}}%
\pgfpathlineto{\pgfqpoint{4.561526in}{0.914724in}}%
\pgfpathlineto{\pgfqpoint{4.600471in}{0.908673in}}%
\pgfpathlineto{\pgfqpoint{4.619943in}{0.907237in}}%
\pgfpathlineto{\pgfqpoint{4.639416in}{0.907070in}}%
\pgfpathlineto{\pgfqpoint{4.678361in}{0.909924in}}%
\pgfpathlineto{\pgfqpoint{4.697833in}{0.911710in}}%
\pgfpathlineto{\pgfqpoint{4.717305in}{0.911975in}}%
\pgfpathlineto{\pgfqpoint{4.736778in}{0.908394in}}%
\pgfpathlineto{\pgfqpoint{4.756250in}{0.897526in}}%
\pgfpathlineto{\pgfqpoint{4.756250in}{0.897526in}}%
\pgfusepath{stroke}%
\end{pgfscope}%
\begin{pgfscope}%
\pgfpathrectangle{\pgfqpoint{0.687500in}{0.385000in}}{\pgfqpoint{4.262500in}{2.695000in}}%
\pgfusepath{clip}%
\pgfsetbuttcap%
\pgfsetroundjoin%
\definecolor{currentfill}{rgb}{0.498039,0.498039,0.498039}%
\pgfsetfillcolor{currentfill}%
\pgfsetlinewidth{1.003750pt}%
\definecolor{currentstroke}{rgb}{0.498039,0.498039,0.498039}%
\pgfsetstrokecolor{currentstroke}%
\pgfsetdash{}{0pt}%
\pgfsys@defobject{currentmarker}{\pgfqpoint{-0.020833in}{-0.020833in}}{\pgfqpoint{0.020833in}{0.020833in}}{%
\pgfpathmoveto{\pgfqpoint{0.000000in}{-0.020833in}}%
\pgfpathcurveto{\pgfqpoint{0.005525in}{-0.020833in}}{\pgfqpoint{0.010825in}{-0.018638in}}{\pgfqpoint{0.014731in}{-0.014731in}}%
\pgfpathcurveto{\pgfqpoint{0.018638in}{-0.010825in}}{\pgfqpoint{0.020833in}{-0.005525in}}{\pgfqpoint{0.020833in}{0.000000in}}%
\pgfpathcurveto{\pgfqpoint{0.020833in}{0.005525in}}{\pgfqpoint{0.018638in}{0.010825in}}{\pgfqpoint{0.014731in}{0.014731in}}%
\pgfpathcurveto{\pgfqpoint{0.010825in}{0.018638in}}{\pgfqpoint{0.005525in}{0.020833in}}{\pgfqpoint{0.000000in}{0.020833in}}%
\pgfpathcurveto{\pgfqpoint{-0.005525in}{0.020833in}}{\pgfqpoint{-0.010825in}{0.018638in}}{\pgfqpoint{-0.014731in}{0.014731in}}%
\pgfpathcurveto{\pgfqpoint{-0.018638in}{0.010825in}}{\pgfqpoint{-0.020833in}{0.005525in}}{\pgfqpoint{-0.020833in}{0.000000in}}%
\pgfpathcurveto{\pgfqpoint{-0.020833in}{-0.005525in}}{\pgfqpoint{-0.018638in}{-0.010825in}}{\pgfqpoint{-0.014731in}{-0.014731in}}%
\pgfpathcurveto{\pgfqpoint{-0.010825in}{-0.018638in}}{\pgfqpoint{-0.005525in}{-0.020833in}}{\pgfqpoint{0.000000in}{-0.020833in}}%
\pgfpathclose%
\pgfusepath{stroke,fill}%
}%
\begin{pgfscope}%
\pgfsys@transformshift{4.753916in}{0.903430in}%
\pgfsys@useobject{currentmarker}{}%
\end{pgfscope}%
\begin{pgfscope}%
\pgfsys@transformshift{4.735279in}{0.904730in}%
\pgfsys@useobject{currentmarker}{}%
\end{pgfscope}%
\begin{pgfscope}%
\pgfsys@transformshift{4.698186in}{0.907428in}%
\pgfsys@useobject{currentmarker}{}%
\end{pgfscope}%
\begin{pgfscope}%
\pgfsys@transformshift{4.642992in}{0.911734in}%
\pgfsys@useobject{currentmarker}{}%
\end{pgfscope}%
\begin{pgfscope}%
\pgfsys@transformshift{4.570229in}{0.918006in}%
\pgfsys@useobject{currentmarker}{}%
\end{pgfscope}%
\begin{pgfscope}%
\pgfsys@transformshift{4.480599in}{0.926814in}%
\pgfsys@useobject{currentmarker}{}%
\end{pgfscope}%
\begin{pgfscope}%
\pgfsys@transformshift{4.374965in}{0.939060in}%
\pgfsys@useobject{currentmarker}{}%
\end{pgfscope}%
\begin{pgfscope}%
\pgfsys@transformshift{4.254343in}{0.956180in}%
\pgfsys@useobject{currentmarker}{}%
\end{pgfscope}%
\begin{pgfscope}%
\pgfsys@transformshift{4.119895in}{0.980537in}%
\pgfsys@useobject{currentmarker}{}%
\end{pgfscope}%
\begin{pgfscope}%
\pgfsys@transformshift{3.972917in}{1.016173in}%
\pgfsys@useobject{currentmarker}{}%
\end{pgfscope}%
\begin{pgfscope}%
\pgfsys@transformshift{3.814824in}{1.070336in}%
\pgfsys@useobject{currentmarker}{}%
\end{pgfscope}%
\begin{pgfscope}%
\pgfsys@transformshift{3.647138in}{1.156723in}%
\pgfsys@useobject{currentmarker}{}%
\end{pgfscope}%
\begin{pgfscope}%
\pgfsys@transformshift{3.471474in}{1.302369in}%
\pgfsys@useobject{currentmarker}{}%
\end{pgfscope}%
\begin{pgfscope}%
\pgfsys@transformshift{3.289524in}{1.559804in}%
\pgfsys@useobject{currentmarker}{}%
\end{pgfscope}%
\begin{pgfscope}%
\pgfsys@transformshift{3.103040in}{2.002164in}%
\pgfsys@useobject{currentmarker}{}%
\end{pgfscope}%
\begin{pgfscope}%
\pgfsys@transformshift{2.913819in}{2.528830in}%
\pgfsys@useobject{currentmarker}{}%
\end{pgfscope}%
\begin{pgfscope}%
\pgfsys@transformshift{2.723681in}{2.528830in}%
\pgfsys@useobject{currentmarker}{}%
\end{pgfscope}%
\begin{pgfscope}%
\pgfsys@transformshift{2.534460in}{2.002164in}%
\pgfsys@useobject{currentmarker}{}%
\end{pgfscope}%
\begin{pgfscope}%
\pgfsys@transformshift{2.347976in}{1.559804in}%
\pgfsys@useobject{currentmarker}{}%
\end{pgfscope}%
\begin{pgfscope}%
\pgfsys@transformshift{2.166026in}{1.302369in}%
\pgfsys@useobject{currentmarker}{}%
\end{pgfscope}%
\begin{pgfscope}%
\pgfsys@transformshift{1.990362in}{1.156723in}%
\pgfsys@useobject{currentmarker}{}%
\end{pgfscope}%
\begin{pgfscope}%
\pgfsys@transformshift{1.822676in}{1.070336in}%
\pgfsys@useobject{currentmarker}{}%
\end{pgfscope}%
\begin{pgfscope}%
\pgfsys@transformshift{1.664583in}{1.016173in}%
\pgfsys@useobject{currentmarker}{}%
\end{pgfscope}%
\begin{pgfscope}%
\pgfsys@transformshift{1.517605in}{0.980537in}%
\pgfsys@useobject{currentmarker}{}%
\end{pgfscope}%
\begin{pgfscope}%
\pgfsys@transformshift{1.383157in}{0.956180in}%
\pgfsys@useobject{currentmarker}{}%
\end{pgfscope}%
\begin{pgfscope}%
\pgfsys@transformshift{1.262535in}{0.939060in}%
\pgfsys@useobject{currentmarker}{}%
\end{pgfscope}%
\begin{pgfscope}%
\pgfsys@transformshift{1.156901in}{0.926814in}%
\pgfsys@useobject{currentmarker}{}%
\end{pgfscope}%
\begin{pgfscope}%
\pgfsys@transformshift{1.067271in}{0.918006in}%
\pgfsys@useobject{currentmarker}{}%
\end{pgfscope}%
\begin{pgfscope}%
\pgfsys@transformshift{0.994508in}{0.911734in}%
\pgfsys@useobject{currentmarker}{}%
\end{pgfscope}%
\begin{pgfscope}%
\pgfsys@transformshift{0.939314in}{0.907428in}%
\pgfsys@useobject{currentmarker}{}%
\end{pgfscope}%
\begin{pgfscope}%
\pgfsys@transformshift{0.902221in}{0.904730in}%
\pgfsys@useobject{currentmarker}{}%
\end{pgfscope}%
\begin{pgfscope}%
\pgfsys@transformshift{0.883584in}{0.903430in}%
\pgfsys@useobject{currentmarker}{}%
\end{pgfscope}%
\end{pgfscope}%
\begin{pgfscope}%
\pgfpathrectangle{\pgfqpoint{0.687500in}{0.385000in}}{\pgfqpoint{4.262500in}{2.695000in}}%
\pgfusepath{clip}%
\pgfsetrectcap%
\pgfsetroundjoin%
\pgfsetlinewidth{1.505625pt}%
\definecolor{currentstroke}{rgb}{0.737255,0.741176,0.133333}%
\pgfsetstrokecolor{currentstroke}%
\pgfsetdash{}{0pt}%
\pgfpathmoveto{\pgfqpoint{0.881250in}{0.903030in}}%
\pgfpathlineto{\pgfqpoint{1.173335in}{0.928414in}}%
\pgfpathlineto{\pgfqpoint{1.251225in}{0.937527in}}%
\pgfpathlineto{\pgfqpoint{1.407004in}{0.959806in}}%
\pgfpathlineto{\pgfqpoint{1.465421in}{0.969869in}}%
\pgfpathlineto{\pgfqpoint{1.523838in}{0.981904in}}%
\pgfpathlineto{\pgfqpoint{1.601727in}{1.000099in}}%
\pgfpathlineto{\pgfqpoint{1.679617in}{1.020312in}}%
\pgfpathlineto{\pgfqpoint{1.738034in}{1.038133in}}%
\pgfpathlineto{\pgfqpoint{1.776979in}{1.051947in}}%
\pgfpathlineto{\pgfqpoint{1.815923in}{1.067473in}}%
\pgfpathlineto{\pgfqpoint{1.874340in}{1.093805in}}%
\pgfpathlineto{\pgfqpoint{1.932758in}{1.123532in}}%
\pgfpathlineto{\pgfqpoint{1.971702in}{1.145462in}}%
\pgfpathlineto{\pgfqpoint{2.010647in}{1.169634in}}%
\pgfpathlineto{\pgfqpoint{2.049592in}{1.196762in}}%
\pgfpathlineto{\pgfqpoint{2.088536in}{1.227607in}}%
\pgfpathlineto{\pgfqpoint{2.127481in}{1.262813in}}%
\pgfpathlineto{\pgfqpoint{2.166426in}{1.302805in}}%
\pgfpathlineto{\pgfqpoint{2.205371in}{1.347804in}}%
\pgfpathlineto{\pgfqpoint{2.244315in}{1.397961in}}%
\pgfpathlineto{\pgfqpoint{2.283260in}{1.453561in}}%
\pgfpathlineto{\pgfqpoint{2.322205in}{1.515209in}}%
\pgfpathlineto{\pgfqpoint{2.361149in}{1.583908in}}%
\pgfpathlineto{\pgfqpoint{2.400094in}{1.660945in}}%
\pgfpathlineto{\pgfqpoint{2.439039in}{1.747563in}}%
\pgfpathlineto{\pgfqpoint{2.477984in}{1.844475in}}%
\pgfpathlineto{\pgfqpoint{2.516928in}{1.951298in}}%
\pgfpathlineto{\pgfqpoint{2.555873in}{2.066070in}}%
\pgfpathlineto{\pgfqpoint{2.633763in}{2.302425in}}%
\pgfpathlineto{\pgfqpoint{2.672707in}{2.411445in}}%
\pgfpathlineto{\pgfqpoint{2.692180in}{2.460429in}}%
\pgfpathlineto{\pgfqpoint{2.711652in}{2.504474in}}%
\pgfpathlineto{\pgfqpoint{2.731124in}{2.542703in}}%
\pgfpathlineto{\pgfqpoint{2.750597in}{2.574324in}}%
\pgfpathlineto{\pgfqpoint{2.770069in}{2.598666in}}%
\pgfpathlineto{\pgfqpoint{2.789541in}{2.615199in}}%
\pgfpathlineto{\pgfqpoint{2.809014in}{2.623558in}}%
\pgfpathlineto{\pgfqpoint{2.828486in}{2.623558in}}%
\pgfpathlineto{\pgfqpoint{2.847959in}{2.615199in}}%
\pgfpathlineto{\pgfqpoint{2.867431in}{2.598666in}}%
\pgfpathlineto{\pgfqpoint{2.886903in}{2.574324in}}%
\pgfpathlineto{\pgfqpoint{2.906376in}{2.542703in}}%
\pgfpathlineto{\pgfqpoint{2.925848in}{2.504474in}}%
\pgfpathlineto{\pgfqpoint{2.945320in}{2.460429in}}%
\pgfpathlineto{\pgfqpoint{2.964793in}{2.411445in}}%
\pgfpathlineto{\pgfqpoint{3.003737in}{2.302425in}}%
\pgfpathlineto{\pgfqpoint{3.062155in}{2.125318in}}%
\pgfpathlineto{\pgfqpoint{3.101099in}{2.007884in}}%
\pgfpathlineto{\pgfqpoint{3.140044in}{1.896729in}}%
\pgfpathlineto{\pgfqpoint{3.178989in}{1.794725in}}%
\pgfpathlineto{\pgfqpoint{3.217933in}{1.702993in}}%
\pgfpathlineto{\pgfqpoint{3.256878in}{1.621301in}}%
\pgfpathlineto{\pgfqpoint{3.295823in}{1.548601in}}%
\pgfpathlineto{\pgfqpoint{3.334768in}{1.483575in}}%
\pgfpathlineto{\pgfqpoint{3.373712in}{1.425052in}}%
\pgfpathlineto{\pgfqpoint{3.412657in}{1.372226in}}%
\pgfpathlineto{\pgfqpoint{3.451602in}{1.324670in}}%
\pgfpathlineto{\pgfqpoint{3.490546in}{1.282193in}}%
\pgfpathlineto{\pgfqpoint{3.529491in}{1.244633in}}%
\pgfpathlineto{\pgfqpoint{3.568436in}{1.211675in}}%
\pgfpathlineto{\pgfqpoint{3.607381in}{1.182780in}}%
\pgfpathlineto{\pgfqpoint{3.646325in}{1.157226in}}%
\pgfpathlineto{\pgfqpoint{3.685270in}{1.134255in}}%
\pgfpathlineto{\pgfqpoint{3.743687in}{1.103335in}}%
\pgfpathlineto{\pgfqpoint{3.802104in}{1.075861in}}%
\pgfpathlineto{\pgfqpoint{3.841049in}{1.059497in}}%
\pgfpathlineto{\pgfqpoint{3.879994in}{1.044829in}}%
\pgfpathlineto{\pgfqpoint{3.938411in}{1.025915in}}%
\pgfpathlineto{\pgfqpoint{3.996828in}{1.009863in}}%
\pgfpathlineto{\pgfqpoint{4.074717in}{0.990778in}}%
\pgfpathlineto{\pgfqpoint{4.152607in}{0.973672in}}%
\pgfpathlineto{\pgfqpoint{4.211024in}{0.962948in}}%
\pgfpathlineto{\pgfqpoint{4.288913in}{0.951185in}}%
\pgfpathlineto{\pgfqpoint{4.425220in}{0.932616in}}%
\pgfpathlineto{\pgfqpoint{4.503109in}{0.924689in}}%
\pgfpathlineto{\pgfqpoint{4.658888in}{0.910648in}}%
\pgfpathlineto{\pgfqpoint{4.756250in}{0.903030in}}%
\pgfpathlineto{\pgfqpoint{4.756250in}{0.903030in}}%
\pgfusepath{stroke}%
\end{pgfscope}%
\begin{pgfscope}%
\pgfsetrectcap%
\pgfsetmiterjoin%
\pgfsetlinewidth{0.803000pt}%
\definecolor{currentstroke}{rgb}{0.000000,0.000000,0.000000}%
\pgfsetstrokecolor{currentstroke}%
\pgfsetdash{}{0pt}%
\pgfpathmoveto{\pgfqpoint{0.687500in}{0.385000in}}%
\pgfpathlineto{\pgfqpoint{0.687500in}{3.080000in}}%
\pgfusepath{stroke}%
\end{pgfscope}%
\begin{pgfscope}%
\pgfsetrectcap%
\pgfsetmiterjoin%
\pgfsetlinewidth{0.803000pt}%
\definecolor{currentstroke}{rgb}{0.000000,0.000000,0.000000}%
\pgfsetstrokecolor{currentstroke}%
\pgfsetdash{}{0pt}%
\pgfpathmoveto{\pgfqpoint{4.950000in}{0.385000in}}%
\pgfpathlineto{\pgfqpoint{4.950000in}{3.080000in}}%
\pgfusepath{stroke}%
\end{pgfscope}%
\begin{pgfscope}%
\pgfsetrectcap%
\pgfsetmiterjoin%
\pgfsetlinewidth{0.803000pt}%
\definecolor{currentstroke}{rgb}{0.000000,0.000000,0.000000}%
\pgfsetstrokecolor{currentstroke}%
\pgfsetdash{}{0pt}%
\pgfpathmoveto{\pgfqpoint{0.687500in}{0.385000in}}%
\pgfpathlineto{\pgfqpoint{4.950000in}{0.385000in}}%
\pgfusepath{stroke}%
\end{pgfscope}%
\begin{pgfscope}%
\pgfsetrectcap%
\pgfsetmiterjoin%
\pgfsetlinewidth{0.803000pt}%
\definecolor{currentstroke}{rgb}{0.000000,0.000000,0.000000}%
\pgfsetstrokecolor{currentstroke}%
\pgfsetdash{}{0pt}%
\pgfpathmoveto{\pgfqpoint{0.687500in}{3.080000in}}%
\pgfpathlineto{\pgfqpoint{4.950000in}{3.080000in}}%
\pgfusepath{stroke}%
\end{pgfscope}%
\begin{pgfscope}%
\definecolor{textcolor}{rgb}{0.000000,0.000000,0.000000}%
\pgfsetstrokecolor{textcolor}%
\pgfsetfillcolor{textcolor}%
\pgftext[x=2.818750in,y=3.163333in,,base]{\color{textcolor}\rmfamily\fontsize{12.000000}{14.400000}\selectfont N=5, 7, 15, 31}%
\end{pgfscope}%
\begin{pgfscope}%
\pgfsetbuttcap%
\pgfsetmiterjoin%
\definecolor{currentfill}{rgb}{1.000000,1.000000,1.000000}%
\pgfsetfillcolor{currentfill}%
\pgfsetfillopacity{0.800000}%
\pgfsetlinewidth{1.003750pt}%
\definecolor{currentstroke}{rgb}{0.800000,0.800000,0.800000}%
\pgfsetstrokecolor{currentstroke}%
\pgfsetstrokeopacity{0.800000}%
\pgfsetdash{}{0pt}%
\pgfpathmoveto{\pgfqpoint{3.448142in}{1.775801in}}%
\pgfpathlineto{\pgfqpoint{4.852778in}{1.775801in}}%
\pgfpathquadraticcurveto{\pgfqpoint{4.880556in}{1.775801in}}{\pgfqpoint{4.880556in}{1.803578in}}%
\pgfpathlineto{\pgfqpoint{4.880556in}{2.982778in}}%
\pgfpathquadraticcurveto{\pgfqpoint{4.880556in}{3.010556in}}{\pgfqpoint{4.852778in}{3.010556in}}%
\pgfpathlineto{\pgfqpoint{3.448142in}{3.010556in}}%
\pgfpathquadraticcurveto{\pgfqpoint{3.420364in}{3.010556in}}{\pgfqpoint{3.420364in}{2.982778in}}%
\pgfpathlineto{\pgfqpoint{3.420364in}{1.803578in}}%
\pgfpathquadraticcurveto{\pgfqpoint{3.420364in}{1.775801in}}{\pgfqpoint{3.448142in}{1.775801in}}%
\pgfpathclose%
\pgfusepath{stroke,fill}%
\end{pgfscope}%
\begin{pgfscope}%
\pgfsetrectcap%
\pgfsetroundjoin%
\pgfsetlinewidth{1.505625pt}%
\definecolor{currentstroke}{rgb}{0.121569,0.466667,0.705882}%
\pgfsetstrokecolor{currentstroke}%
\pgfsetdash{}{0pt}%
\pgfpathmoveto{\pgfqpoint{3.475920in}{2.820146in}}%
\pgfpathlineto{\pgfqpoint{3.753698in}{2.820146in}}%
\pgfusepath{stroke}%
\end{pgfscope}%
\begin{pgfscope}%
\definecolor{textcolor}{rgb}{0.000000,0.000000,0.000000}%
\pgfsetstrokecolor{textcolor}%
\pgfsetfillcolor{textcolor}%
\pgftext[x=3.864809in,y=2.771534in,left,base]{\color{textcolor}\rmfamily\fontsize{10.000000}{12.000000}\selectfont \(\displaystyle y(x)=\)\(\displaystyle \frac{1}{1+25x^{2}}\)}%
\end{pgfscope}%
\begin{pgfscope}%
\pgfsetrectcap%
\pgfsetroundjoin%
\pgfsetlinewidth{1.505625pt}%
\definecolor{currentstroke}{rgb}{0.172549,0.627451,0.172549}%
\pgfsetstrokecolor{currentstroke}%
\pgfsetdash{}{0pt}%
\pgfpathmoveto{\pgfqpoint{3.475920in}{2.539689in}}%
\pgfpathlineto{\pgfqpoint{3.753698in}{2.539689in}}%
\pgfusepath{stroke}%
\end{pgfscope}%
\begin{pgfscope}%
\definecolor{textcolor}{rgb}{0.000000,0.000000,0.000000}%
\pgfsetstrokecolor{textcolor}%
\pgfsetfillcolor{textcolor}%
\pgftext[x=3.864809in,y=2.491078in,left,base]{\color{textcolor}\rmfamily\fontsize{10.000000}{12.000000}\selectfont W5(x)}%
\end{pgfscope}%
\begin{pgfscope}%
\pgfsetrectcap%
\pgfsetroundjoin%
\pgfsetlinewidth{1.505625pt}%
\definecolor{currentstroke}{rgb}{0.580392,0.403922,0.741176}%
\pgfsetstrokecolor{currentstroke}%
\pgfsetdash{}{0pt}%
\pgfpathmoveto{\pgfqpoint{3.475920in}{2.331356in}}%
\pgfpathlineto{\pgfqpoint{3.753698in}{2.331356in}}%
\pgfusepath{stroke}%
\end{pgfscope}%
\begin{pgfscope}%
\definecolor{textcolor}{rgb}{0.000000,0.000000,0.000000}%
\pgfsetstrokecolor{textcolor}%
\pgfsetfillcolor{textcolor}%
\pgftext[x=3.864809in,y=2.282745in,left,base]{\color{textcolor}\rmfamily\fontsize{10.000000}{12.000000}\selectfont W7(x)}%
\end{pgfscope}%
\begin{pgfscope}%
\pgfsetrectcap%
\pgfsetroundjoin%
\pgfsetlinewidth{1.505625pt}%
\definecolor{currentstroke}{rgb}{0.890196,0.466667,0.760784}%
\pgfsetstrokecolor{currentstroke}%
\pgfsetdash{}{0pt}%
\pgfpathmoveto{\pgfqpoint{3.475920in}{2.123023in}}%
\pgfpathlineto{\pgfqpoint{3.753698in}{2.123023in}}%
\pgfusepath{stroke}%
\end{pgfscope}%
\begin{pgfscope}%
\definecolor{textcolor}{rgb}{0.000000,0.000000,0.000000}%
\pgfsetstrokecolor{textcolor}%
\pgfsetfillcolor{textcolor}%
\pgftext[x=3.864809in,y=2.074412in,left,base]{\color{textcolor}\rmfamily\fontsize{10.000000}{12.000000}\selectfont W15(x)}%
\end{pgfscope}%
\begin{pgfscope}%
\pgfsetrectcap%
\pgfsetroundjoin%
\pgfsetlinewidth{1.505625pt}%
\definecolor{currentstroke}{rgb}{0.737255,0.741176,0.133333}%
\pgfsetstrokecolor{currentstroke}%
\pgfsetdash{}{0pt}%
\pgfpathmoveto{\pgfqpoint{3.475920in}{1.914689in}}%
\pgfpathlineto{\pgfqpoint{3.753698in}{1.914689in}}%
\pgfusepath{stroke}%
\end{pgfscope}%
\begin{pgfscope}%
\definecolor{textcolor}{rgb}{0.000000,0.000000,0.000000}%
\pgfsetstrokecolor{textcolor}%
\pgfsetfillcolor{textcolor}%
\pgftext[x=3.864809in,y=1.866078in,left,base]{\color{textcolor}\rmfamily\fontsize{10.000000}{12.000000}\selectfont W31(x)}%
\end{pgfscope}%
\end{pgfpicture}%
\makeatother%
\endgroup%
        
    \end{center}
    \caption{Węzły cosinus, funkcja \(y\), \(N=5,7,15,31\)}
\end{figure}

% ------------------------------------------------------------------------------

\begin{figure}[h]
    \begin{center}
        %% Creator: Matplotlib, PGF backend
%%
%% To include the figure in your LaTeX document, write
%%   \input{<filename>.pgf}
%%
%% Make sure the required packages are loaded in your preamble
%%   \usepackage{pgf}
%%
%% Figures using additional raster images can only be included by \input if
%% they are in the same directory as the main LaTeX file. For loading figures
%% from other directories you can use the `import` package
%%   \usepackage{import}
%% and then include the figures with
%%   \import{<path to file>}{<filename>.pgf}
%%
%% Matplotlib used the following preamble
%%
\begingroup%
\makeatletter%
\begin{pgfpicture}%
\pgfpathrectangle{\pgfpointorigin}{\pgfqpoint{5.500000in}{3.500000in}}%
\pgfusepath{use as bounding box, clip}%
\begin{pgfscope}%
\pgfsetbuttcap%
\pgfsetmiterjoin%
\definecolor{currentfill}{rgb}{1.000000,1.000000,1.000000}%
\pgfsetfillcolor{currentfill}%
\pgfsetlinewidth{0.000000pt}%
\definecolor{currentstroke}{rgb}{1.000000,1.000000,1.000000}%
\pgfsetstrokecolor{currentstroke}%
\pgfsetdash{}{0pt}%
\pgfpathmoveto{\pgfqpoint{0.000000in}{0.000000in}}%
\pgfpathlineto{\pgfqpoint{5.500000in}{0.000000in}}%
\pgfpathlineto{\pgfqpoint{5.500000in}{3.500000in}}%
\pgfpathlineto{\pgfqpoint{0.000000in}{3.500000in}}%
\pgfpathclose%
\pgfusepath{fill}%
\end{pgfscope}%
\begin{pgfscope}%
\pgfsetbuttcap%
\pgfsetmiterjoin%
\definecolor{currentfill}{rgb}{1.000000,1.000000,1.000000}%
\pgfsetfillcolor{currentfill}%
\pgfsetlinewidth{0.000000pt}%
\definecolor{currentstroke}{rgb}{0.000000,0.000000,0.000000}%
\pgfsetstrokecolor{currentstroke}%
\pgfsetstrokeopacity{0.000000}%
\pgfsetdash{}{0pt}%
\pgfpathmoveto{\pgfqpoint{0.687500in}{0.385000in}}%
\pgfpathlineto{\pgfqpoint{4.950000in}{0.385000in}}%
\pgfpathlineto{\pgfqpoint{4.950000in}{3.080000in}}%
\pgfpathlineto{\pgfqpoint{0.687500in}{3.080000in}}%
\pgfpathclose%
\pgfusepath{fill}%
\end{pgfscope}%
\begin{pgfscope}%
\pgfsetbuttcap%
\pgfsetroundjoin%
\definecolor{currentfill}{rgb}{0.000000,0.000000,0.000000}%
\pgfsetfillcolor{currentfill}%
\pgfsetlinewidth{0.803000pt}%
\definecolor{currentstroke}{rgb}{0.000000,0.000000,0.000000}%
\pgfsetstrokecolor{currentstroke}%
\pgfsetdash{}{0pt}%
\pgfsys@defobject{currentmarker}{\pgfqpoint{0.000000in}{-0.048611in}}{\pgfqpoint{0.000000in}{0.000000in}}{%
\pgfpathmoveto{\pgfqpoint{0.000000in}{0.000000in}}%
\pgfpathlineto{\pgfqpoint{0.000000in}{-0.048611in}}%
\pgfusepath{stroke,fill}%
}%
\begin{pgfscope}%
\pgfsys@transformshift{0.881250in}{0.385000in}%
\pgfsys@useobject{currentmarker}{}%
\end{pgfscope}%
\end{pgfscope}%
\begin{pgfscope}%
\definecolor{textcolor}{rgb}{0.000000,0.000000,0.000000}%
\pgfsetstrokecolor{textcolor}%
\pgfsetfillcolor{textcolor}%
\pgftext[x=0.881250in,y=0.287778in,,top]{\color{textcolor}\rmfamily\fontsize{10.000000}{12.000000}\selectfont \(\displaystyle -1.00\)}%
\end{pgfscope}%
\begin{pgfscope}%
\pgfsetbuttcap%
\pgfsetroundjoin%
\definecolor{currentfill}{rgb}{0.000000,0.000000,0.000000}%
\pgfsetfillcolor{currentfill}%
\pgfsetlinewidth{0.803000pt}%
\definecolor{currentstroke}{rgb}{0.000000,0.000000,0.000000}%
\pgfsetstrokecolor{currentstroke}%
\pgfsetdash{}{0pt}%
\pgfsys@defobject{currentmarker}{\pgfqpoint{0.000000in}{-0.048611in}}{\pgfqpoint{0.000000in}{0.000000in}}{%
\pgfpathmoveto{\pgfqpoint{0.000000in}{0.000000in}}%
\pgfpathlineto{\pgfqpoint{0.000000in}{-0.048611in}}%
\pgfusepath{stroke,fill}%
}%
\begin{pgfscope}%
\pgfsys@transformshift{1.365625in}{0.385000in}%
\pgfsys@useobject{currentmarker}{}%
\end{pgfscope}%
\end{pgfscope}%
\begin{pgfscope}%
\definecolor{textcolor}{rgb}{0.000000,0.000000,0.000000}%
\pgfsetstrokecolor{textcolor}%
\pgfsetfillcolor{textcolor}%
\pgftext[x=1.365625in,y=0.287778in,,top]{\color{textcolor}\rmfamily\fontsize{10.000000}{12.000000}\selectfont \(\displaystyle -0.75\)}%
\end{pgfscope}%
\begin{pgfscope}%
\pgfsetbuttcap%
\pgfsetroundjoin%
\definecolor{currentfill}{rgb}{0.000000,0.000000,0.000000}%
\pgfsetfillcolor{currentfill}%
\pgfsetlinewidth{0.803000pt}%
\definecolor{currentstroke}{rgb}{0.000000,0.000000,0.000000}%
\pgfsetstrokecolor{currentstroke}%
\pgfsetdash{}{0pt}%
\pgfsys@defobject{currentmarker}{\pgfqpoint{0.000000in}{-0.048611in}}{\pgfqpoint{0.000000in}{0.000000in}}{%
\pgfpathmoveto{\pgfqpoint{0.000000in}{0.000000in}}%
\pgfpathlineto{\pgfqpoint{0.000000in}{-0.048611in}}%
\pgfusepath{stroke,fill}%
}%
\begin{pgfscope}%
\pgfsys@transformshift{1.850000in}{0.385000in}%
\pgfsys@useobject{currentmarker}{}%
\end{pgfscope}%
\end{pgfscope}%
\begin{pgfscope}%
\definecolor{textcolor}{rgb}{0.000000,0.000000,0.000000}%
\pgfsetstrokecolor{textcolor}%
\pgfsetfillcolor{textcolor}%
\pgftext[x=1.850000in,y=0.287778in,,top]{\color{textcolor}\rmfamily\fontsize{10.000000}{12.000000}\selectfont \(\displaystyle -0.50\)}%
\end{pgfscope}%
\begin{pgfscope}%
\pgfsetbuttcap%
\pgfsetroundjoin%
\definecolor{currentfill}{rgb}{0.000000,0.000000,0.000000}%
\pgfsetfillcolor{currentfill}%
\pgfsetlinewidth{0.803000pt}%
\definecolor{currentstroke}{rgb}{0.000000,0.000000,0.000000}%
\pgfsetstrokecolor{currentstroke}%
\pgfsetdash{}{0pt}%
\pgfsys@defobject{currentmarker}{\pgfqpoint{0.000000in}{-0.048611in}}{\pgfqpoint{0.000000in}{0.000000in}}{%
\pgfpathmoveto{\pgfqpoint{0.000000in}{0.000000in}}%
\pgfpathlineto{\pgfqpoint{0.000000in}{-0.048611in}}%
\pgfusepath{stroke,fill}%
}%
\begin{pgfscope}%
\pgfsys@transformshift{2.334375in}{0.385000in}%
\pgfsys@useobject{currentmarker}{}%
\end{pgfscope}%
\end{pgfscope}%
\begin{pgfscope}%
\definecolor{textcolor}{rgb}{0.000000,0.000000,0.000000}%
\pgfsetstrokecolor{textcolor}%
\pgfsetfillcolor{textcolor}%
\pgftext[x=2.334375in,y=0.287778in,,top]{\color{textcolor}\rmfamily\fontsize{10.000000}{12.000000}\selectfont \(\displaystyle -0.25\)}%
\end{pgfscope}%
\begin{pgfscope}%
\pgfsetbuttcap%
\pgfsetroundjoin%
\definecolor{currentfill}{rgb}{0.000000,0.000000,0.000000}%
\pgfsetfillcolor{currentfill}%
\pgfsetlinewidth{0.803000pt}%
\definecolor{currentstroke}{rgb}{0.000000,0.000000,0.000000}%
\pgfsetstrokecolor{currentstroke}%
\pgfsetdash{}{0pt}%
\pgfsys@defobject{currentmarker}{\pgfqpoint{0.000000in}{-0.048611in}}{\pgfqpoint{0.000000in}{0.000000in}}{%
\pgfpathmoveto{\pgfqpoint{0.000000in}{0.000000in}}%
\pgfpathlineto{\pgfqpoint{0.000000in}{-0.048611in}}%
\pgfusepath{stroke,fill}%
}%
\begin{pgfscope}%
\pgfsys@transformshift{2.818750in}{0.385000in}%
\pgfsys@useobject{currentmarker}{}%
\end{pgfscope}%
\end{pgfscope}%
\begin{pgfscope}%
\definecolor{textcolor}{rgb}{0.000000,0.000000,0.000000}%
\pgfsetstrokecolor{textcolor}%
\pgfsetfillcolor{textcolor}%
\pgftext[x=2.818750in,y=0.287778in,,top]{\color{textcolor}\rmfamily\fontsize{10.000000}{12.000000}\selectfont \(\displaystyle 0.00\)}%
\end{pgfscope}%
\begin{pgfscope}%
\pgfsetbuttcap%
\pgfsetroundjoin%
\definecolor{currentfill}{rgb}{0.000000,0.000000,0.000000}%
\pgfsetfillcolor{currentfill}%
\pgfsetlinewidth{0.803000pt}%
\definecolor{currentstroke}{rgb}{0.000000,0.000000,0.000000}%
\pgfsetstrokecolor{currentstroke}%
\pgfsetdash{}{0pt}%
\pgfsys@defobject{currentmarker}{\pgfqpoint{0.000000in}{-0.048611in}}{\pgfqpoint{0.000000in}{0.000000in}}{%
\pgfpathmoveto{\pgfqpoint{0.000000in}{0.000000in}}%
\pgfpathlineto{\pgfqpoint{0.000000in}{-0.048611in}}%
\pgfusepath{stroke,fill}%
}%
\begin{pgfscope}%
\pgfsys@transformshift{3.303125in}{0.385000in}%
\pgfsys@useobject{currentmarker}{}%
\end{pgfscope}%
\end{pgfscope}%
\begin{pgfscope}%
\definecolor{textcolor}{rgb}{0.000000,0.000000,0.000000}%
\pgfsetstrokecolor{textcolor}%
\pgfsetfillcolor{textcolor}%
\pgftext[x=3.303125in,y=0.287778in,,top]{\color{textcolor}\rmfamily\fontsize{10.000000}{12.000000}\selectfont \(\displaystyle 0.25\)}%
\end{pgfscope}%
\begin{pgfscope}%
\pgfsetbuttcap%
\pgfsetroundjoin%
\definecolor{currentfill}{rgb}{0.000000,0.000000,0.000000}%
\pgfsetfillcolor{currentfill}%
\pgfsetlinewidth{0.803000pt}%
\definecolor{currentstroke}{rgb}{0.000000,0.000000,0.000000}%
\pgfsetstrokecolor{currentstroke}%
\pgfsetdash{}{0pt}%
\pgfsys@defobject{currentmarker}{\pgfqpoint{0.000000in}{-0.048611in}}{\pgfqpoint{0.000000in}{0.000000in}}{%
\pgfpathmoveto{\pgfqpoint{0.000000in}{0.000000in}}%
\pgfpathlineto{\pgfqpoint{0.000000in}{-0.048611in}}%
\pgfusepath{stroke,fill}%
}%
\begin{pgfscope}%
\pgfsys@transformshift{3.787500in}{0.385000in}%
\pgfsys@useobject{currentmarker}{}%
\end{pgfscope}%
\end{pgfscope}%
\begin{pgfscope}%
\definecolor{textcolor}{rgb}{0.000000,0.000000,0.000000}%
\pgfsetstrokecolor{textcolor}%
\pgfsetfillcolor{textcolor}%
\pgftext[x=3.787500in,y=0.287778in,,top]{\color{textcolor}\rmfamily\fontsize{10.000000}{12.000000}\selectfont \(\displaystyle 0.50\)}%
\end{pgfscope}%
\begin{pgfscope}%
\pgfsetbuttcap%
\pgfsetroundjoin%
\definecolor{currentfill}{rgb}{0.000000,0.000000,0.000000}%
\pgfsetfillcolor{currentfill}%
\pgfsetlinewidth{0.803000pt}%
\definecolor{currentstroke}{rgb}{0.000000,0.000000,0.000000}%
\pgfsetstrokecolor{currentstroke}%
\pgfsetdash{}{0pt}%
\pgfsys@defobject{currentmarker}{\pgfqpoint{0.000000in}{-0.048611in}}{\pgfqpoint{0.000000in}{0.000000in}}{%
\pgfpathmoveto{\pgfqpoint{0.000000in}{0.000000in}}%
\pgfpathlineto{\pgfqpoint{0.000000in}{-0.048611in}}%
\pgfusepath{stroke,fill}%
}%
\begin{pgfscope}%
\pgfsys@transformshift{4.271875in}{0.385000in}%
\pgfsys@useobject{currentmarker}{}%
\end{pgfscope}%
\end{pgfscope}%
\begin{pgfscope}%
\definecolor{textcolor}{rgb}{0.000000,0.000000,0.000000}%
\pgfsetstrokecolor{textcolor}%
\pgfsetfillcolor{textcolor}%
\pgftext[x=4.271875in,y=0.287778in,,top]{\color{textcolor}\rmfamily\fontsize{10.000000}{12.000000}\selectfont \(\displaystyle 0.75\)}%
\end{pgfscope}%
\begin{pgfscope}%
\pgfsetbuttcap%
\pgfsetroundjoin%
\definecolor{currentfill}{rgb}{0.000000,0.000000,0.000000}%
\pgfsetfillcolor{currentfill}%
\pgfsetlinewidth{0.803000pt}%
\definecolor{currentstroke}{rgb}{0.000000,0.000000,0.000000}%
\pgfsetstrokecolor{currentstroke}%
\pgfsetdash{}{0pt}%
\pgfsys@defobject{currentmarker}{\pgfqpoint{0.000000in}{-0.048611in}}{\pgfqpoint{0.000000in}{0.000000in}}{%
\pgfpathmoveto{\pgfqpoint{0.000000in}{0.000000in}}%
\pgfpathlineto{\pgfqpoint{0.000000in}{-0.048611in}}%
\pgfusepath{stroke,fill}%
}%
\begin{pgfscope}%
\pgfsys@transformshift{4.756250in}{0.385000in}%
\pgfsys@useobject{currentmarker}{}%
\end{pgfscope}%
\end{pgfscope}%
\begin{pgfscope}%
\definecolor{textcolor}{rgb}{0.000000,0.000000,0.000000}%
\pgfsetstrokecolor{textcolor}%
\pgfsetfillcolor{textcolor}%
\pgftext[x=4.756250in,y=0.287778in,,top]{\color{textcolor}\rmfamily\fontsize{10.000000}{12.000000}\selectfont \(\displaystyle 1.00\)}%
\end{pgfscope}%
\begin{pgfscope}%
\definecolor{textcolor}{rgb}{0.000000,0.000000,0.000000}%
\pgfsetstrokecolor{textcolor}%
\pgfsetfillcolor{textcolor}%
\pgftext[x=2.818750in,y=0.108766in,,top]{\color{textcolor}\rmfamily\fontsize{10.000000}{12.000000}\selectfont x}%
\end{pgfscope}%
\begin{pgfscope}%
\pgfsetbuttcap%
\pgfsetroundjoin%
\definecolor{currentfill}{rgb}{0.000000,0.000000,0.000000}%
\pgfsetfillcolor{currentfill}%
\pgfsetlinewidth{0.803000pt}%
\definecolor{currentstroke}{rgb}{0.000000,0.000000,0.000000}%
\pgfsetstrokecolor{currentstroke}%
\pgfsetdash{}{0pt}%
\pgfsys@defobject{currentmarker}{\pgfqpoint{-0.048611in}{0.000000in}}{\pgfqpoint{0.000000in}{0.000000in}}{%
\pgfpathmoveto{\pgfqpoint{0.000000in}{0.000000in}}%
\pgfpathlineto{\pgfqpoint{-0.048611in}{0.000000in}}%
\pgfusepath{stroke,fill}%
}%
\begin{pgfscope}%
\pgfsys@transformshift{0.687500in}{0.474833in}%
\pgfsys@useobject{currentmarker}{}%
\end{pgfscope}%
\end{pgfscope}%
\begin{pgfscope}%
\definecolor{textcolor}{rgb}{0.000000,0.000000,0.000000}%
\pgfsetstrokecolor{textcolor}%
\pgfsetfillcolor{textcolor}%
\pgftext[x=0.304783in,y=0.426608in,left,base]{\color{textcolor}\rmfamily\fontsize{10.000000}{12.000000}\selectfont \(\displaystyle -0.2\)}%
\end{pgfscope}%
\begin{pgfscope}%
\pgfsetbuttcap%
\pgfsetroundjoin%
\definecolor{currentfill}{rgb}{0.000000,0.000000,0.000000}%
\pgfsetfillcolor{currentfill}%
\pgfsetlinewidth{0.803000pt}%
\definecolor{currentstroke}{rgb}{0.000000,0.000000,0.000000}%
\pgfsetstrokecolor{currentstroke}%
\pgfsetdash{}{0pt}%
\pgfsys@defobject{currentmarker}{\pgfqpoint{-0.048611in}{0.000000in}}{\pgfqpoint{0.000000in}{0.000000in}}{%
\pgfpathmoveto{\pgfqpoint{0.000000in}{0.000000in}}%
\pgfpathlineto{\pgfqpoint{-0.048611in}{0.000000in}}%
\pgfusepath{stroke,fill}%
}%
\begin{pgfscope}%
\pgfsys@transformshift{0.687500in}{0.834167in}%
\pgfsys@useobject{currentmarker}{}%
\end{pgfscope}%
\end{pgfscope}%
\begin{pgfscope}%
\definecolor{textcolor}{rgb}{0.000000,0.000000,0.000000}%
\pgfsetstrokecolor{textcolor}%
\pgfsetfillcolor{textcolor}%
\pgftext[x=0.412808in,y=0.785941in,left,base]{\color{textcolor}\rmfamily\fontsize{10.000000}{12.000000}\selectfont \(\displaystyle 0.0\)}%
\end{pgfscope}%
\begin{pgfscope}%
\pgfsetbuttcap%
\pgfsetroundjoin%
\definecolor{currentfill}{rgb}{0.000000,0.000000,0.000000}%
\pgfsetfillcolor{currentfill}%
\pgfsetlinewidth{0.803000pt}%
\definecolor{currentstroke}{rgb}{0.000000,0.000000,0.000000}%
\pgfsetstrokecolor{currentstroke}%
\pgfsetdash{}{0pt}%
\pgfsys@defobject{currentmarker}{\pgfqpoint{-0.048611in}{0.000000in}}{\pgfqpoint{0.000000in}{0.000000in}}{%
\pgfpathmoveto{\pgfqpoint{0.000000in}{0.000000in}}%
\pgfpathlineto{\pgfqpoint{-0.048611in}{0.000000in}}%
\pgfusepath{stroke,fill}%
}%
\begin{pgfscope}%
\pgfsys@transformshift{0.687500in}{1.193500in}%
\pgfsys@useobject{currentmarker}{}%
\end{pgfscope}%
\end{pgfscope}%
\begin{pgfscope}%
\definecolor{textcolor}{rgb}{0.000000,0.000000,0.000000}%
\pgfsetstrokecolor{textcolor}%
\pgfsetfillcolor{textcolor}%
\pgftext[x=0.412808in,y=1.145275in,left,base]{\color{textcolor}\rmfamily\fontsize{10.000000}{12.000000}\selectfont \(\displaystyle 0.2\)}%
\end{pgfscope}%
\begin{pgfscope}%
\pgfsetbuttcap%
\pgfsetroundjoin%
\definecolor{currentfill}{rgb}{0.000000,0.000000,0.000000}%
\pgfsetfillcolor{currentfill}%
\pgfsetlinewidth{0.803000pt}%
\definecolor{currentstroke}{rgb}{0.000000,0.000000,0.000000}%
\pgfsetstrokecolor{currentstroke}%
\pgfsetdash{}{0pt}%
\pgfsys@defobject{currentmarker}{\pgfqpoint{-0.048611in}{0.000000in}}{\pgfqpoint{0.000000in}{0.000000in}}{%
\pgfpathmoveto{\pgfqpoint{0.000000in}{0.000000in}}%
\pgfpathlineto{\pgfqpoint{-0.048611in}{0.000000in}}%
\pgfusepath{stroke,fill}%
}%
\begin{pgfscope}%
\pgfsys@transformshift{0.687500in}{1.552833in}%
\pgfsys@useobject{currentmarker}{}%
\end{pgfscope}%
\end{pgfscope}%
\begin{pgfscope}%
\definecolor{textcolor}{rgb}{0.000000,0.000000,0.000000}%
\pgfsetstrokecolor{textcolor}%
\pgfsetfillcolor{textcolor}%
\pgftext[x=0.412808in,y=1.504608in,left,base]{\color{textcolor}\rmfamily\fontsize{10.000000}{12.000000}\selectfont \(\displaystyle 0.4\)}%
\end{pgfscope}%
\begin{pgfscope}%
\pgfsetbuttcap%
\pgfsetroundjoin%
\definecolor{currentfill}{rgb}{0.000000,0.000000,0.000000}%
\pgfsetfillcolor{currentfill}%
\pgfsetlinewidth{0.803000pt}%
\definecolor{currentstroke}{rgb}{0.000000,0.000000,0.000000}%
\pgfsetstrokecolor{currentstroke}%
\pgfsetdash{}{0pt}%
\pgfsys@defobject{currentmarker}{\pgfqpoint{-0.048611in}{0.000000in}}{\pgfqpoint{0.000000in}{0.000000in}}{%
\pgfpathmoveto{\pgfqpoint{0.000000in}{0.000000in}}%
\pgfpathlineto{\pgfqpoint{-0.048611in}{0.000000in}}%
\pgfusepath{stroke,fill}%
}%
\begin{pgfscope}%
\pgfsys@transformshift{0.687500in}{1.912167in}%
\pgfsys@useobject{currentmarker}{}%
\end{pgfscope}%
\end{pgfscope}%
\begin{pgfscope}%
\definecolor{textcolor}{rgb}{0.000000,0.000000,0.000000}%
\pgfsetstrokecolor{textcolor}%
\pgfsetfillcolor{textcolor}%
\pgftext[x=0.412808in,y=1.863941in,left,base]{\color{textcolor}\rmfamily\fontsize{10.000000}{12.000000}\selectfont \(\displaystyle 0.6\)}%
\end{pgfscope}%
\begin{pgfscope}%
\pgfsetbuttcap%
\pgfsetroundjoin%
\definecolor{currentfill}{rgb}{0.000000,0.000000,0.000000}%
\pgfsetfillcolor{currentfill}%
\pgfsetlinewidth{0.803000pt}%
\definecolor{currentstroke}{rgb}{0.000000,0.000000,0.000000}%
\pgfsetstrokecolor{currentstroke}%
\pgfsetdash{}{0pt}%
\pgfsys@defobject{currentmarker}{\pgfqpoint{-0.048611in}{0.000000in}}{\pgfqpoint{0.000000in}{0.000000in}}{%
\pgfpathmoveto{\pgfqpoint{0.000000in}{0.000000in}}%
\pgfpathlineto{\pgfqpoint{-0.048611in}{0.000000in}}%
\pgfusepath{stroke,fill}%
}%
\begin{pgfscope}%
\pgfsys@transformshift{0.687500in}{2.271500in}%
\pgfsys@useobject{currentmarker}{}%
\end{pgfscope}%
\end{pgfscope}%
\begin{pgfscope}%
\definecolor{textcolor}{rgb}{0.000000,0.000000,0.000000}%
\pgfsetstrokecolor{textcolor}%
\pgfsetfillcolor{textcolor}%
\pgftext[x=0.412808in,y=2.223275in,left,base]{\color{textcolor}\rmfamily\fontsize{10.000000}{12.000000}\selectfont \(\displaystyle 0.8\)}%
\end{pgfscope}%
\begin{pgfscope}%
\pgfsetbuttcap%
\pgfsetroundjoin%
\definecolor{currentfill}{rgb}{0.000000,0.000000,0.000000}%
\pgfsetfillcolor{currentfill}%
\pgfsetlinewidth{0.803000pt}%
\definecolor{currentstroke}{rgb}{0.000000,0.000000,0.000000}%
\pgfsetstrokecolor{currentstroke}%
\pgfsetdash{}{0pt}%
\pgfsys@defobject{currentmarker}{\pgfqpoint{-0.048611in}{0.000000in}}{\pgfqpoint{0.000000in}{0.000000in}}{%
\pgfpathmoveto{\pgfqpoint{0.000000in}{0.000000in}}%
\pgfpathlineto{\pgfqpoint{-0.048611in}{0.000000in}}%
\pgfusepath{stroke,fill}%
}%
\begin{pgfscope}%
\pgfsys@transformshift{0.687500in}{2.630833in}%
\pgfsys@useobject{currentmarker}{}%
\end{pgfscope}%
\end{pgfscope}%
\begin{pgfscope}%
\definecolor{textcolor}{rgb}{0.000000,0.000000,0.000000}%
\pgfsetstrokecolor{textcolor}%
\pgfsetfillcolor{textcolor}%
\pgftext[x=0.412808in,y=2.582608in,left,base]{\color{textcolor}\rmfamily\fontsize{10.000000}{12.000000}\selectfont \(\displaystyle 1.0\)}%
\end{pgfscope}%
\begin{pgfscope}%
\pgfsetbuttcap%
\pgfsetroundjoin%
\definecolor{currentfill}{rgb}{0.000000,0.000000,0.000000}%
\pgfsetfillcolor{currentfill}%
\pgfsetlinewidth{0.803000pt}%
\definecolor{currentstroke}{rgb}{0.000000,0.000000,0.000000}%
\pgfsetstrokecolor{currentstroke}%
\pgfsetdash{}{0pt}%
\pgfsys@defobject{currentmarker}{\pgfqpoint{-0.048611in}{0.000000in}}{\pgfqpoint{0.000000in}{0.000000in}}{%
\pgfpathmoveto{\pgfqpoint{0.000000in}{0.000000in}}%
\pgfpathlineto{\pgfqpoint{-0.048611in}{0.000000in}}%
\pgfusepath{stroke,fill}%
}%
\begin{pgfscope}%
\pgfsys@transformshift{0.687500in}{2.990167in}%
\pgfsys@useobject{currentmarker}{}%
\end{pgfscope}%
\end{pgfscope}%
\begin{pgfscope}%
\definecolor{textcolor}{rgb}{0.000000,0.000000,0.000000}%
\pgfsetstrokecolor{textcolor}%
\pgfsetfillcolor{textcolor}%
\pgftext[x=0.412808in,y=2.941941in,left,base]{\color{textcolor}\rmfamily\fontsize{10.000000}{12.000000}\selectfont \(\displaystyle 1.2\)}%
\end{pgfscope}%
\begin{pgfscope}%
\definecolor{textcolor}{rgb}{0.000000,0.000000,0.000000}%
\pgfsetstrokecolor{textcolor}%
\pgfsetfillcolor{textcolor}%
\pgftext[x=0.249228in,y=1.732500in,,bottom,rotate=90.000000]{\color{textcolor}\rmfamily\fontsize{10.000000}{12.000000}\selectfont y}%
\end{pgfscope}%
\begin{pgfscope}%
\pgfpathrectangle{\pgfqpoint{0.687500in}{0.385000in}}{\pgfqpoint{4.262500in}{2.695000in}}%
\pgfusepath{clip}%
\pgfsetrectcap%
\pgfsetroundjoin%
\pgfsetlinewidth{1.505625pt}%
\definecolor{currentstroke}{rgb}{0.121569,0.466667,0.705882}%
\pgfsetstrokecolor{currentstroke}%
\pgfsetdash{}{0pt}%
\pgfpathmoveto{\pgfqpoint{0.881250in}{1.732500in}}%
\pgfpathlineto{\pgfqpoint{0.978612in}{1.778775in}}%
\pgfpathlineto{\pgfqpoint{1.075974in}{1.827296in}}%
\pgfpathlineto{\pgfqpoint{1.173335in}{1.878000in}}%
\pgfpathlineto{\pgfqpoint{1.290170in}{1.941556in}}%
\pgfpathlineto{\pgfqpoint{1.407004in}{2.007752in}}%
\pgfpathlineto{\pgfqpoint{1.543310in}{2.087643in}}%
\pgfpathlineto{\pgfqpoint{1.796451in}{2.239567in}}%
\pgfpathlineto{\pgfqpoint{1.932758in}{2.320107in}}%
\pgfpathlineto{\pgfqpoint{2.030119in}{2.375473in}}%
\pgfpathlineto{\pgfqpoint{2.108009in}{2.417737in}}%
\pgfpathlineto{\pgfqpoint{2.185898in}{2.457628in}}%
\pgfpathlineto{\pgfqpoint{2.263788in}{2.494605in}}%
\pgfpathlineto{\pgfqpoint{2.322205in}{2.520101in}}%
\pgfpathlineto{\pgfqpoint{2.380622in}{2.543430in}}%
\pgfpathlineto{\pgfqpoint{2.439039in}{2.564379in}}%
\pgfpathlineto{\pgfqpoint{2.497456in}{2.582749in}}%
\pgfpathlineto{\pgfqpoint{2.555873in}{2.598357in}}%
\pgfpathlineto{\pgfqpoint{2.614290in}{2.611046in}}%
\pgfpathlineto{\pgfqpoint{2.672707in}{2.620683in}}%
\pgfpathlineto{\pgfqpoint{2.731124in}{2.627166in}}%
\pgfpathlineto{\pgfqpoint{2.789541in}{2.630425in}}%
\pgfpathlineto{\pgfqpoint{2.847959in}{2.630425in}}%
\pgfpathlineto{\pgfqpoint{2.906376in}{2.627166in}}%
\pgfpathlineto{\pgfqpoint{2.964793in}{2.620683in}}%
\pgfpathlineto{\pgfqpoint{3.023210in}{2.611046in}}%
\pgfpathlineto{\pgfqpoint{3.081627in}{2.598357in}}%
\pgfpathlineto{\pgfqpoint{3.140044in}{2.582749in}}%
\pgfpathlineto{\pgfqpoint{3.198461in}{2.564379in}}%
\pgfpathlineto{\pgfqpoint{3.256878in}{2.543430in}}%
\pgfpathlineto{\pgfqpoint{3.315295in}{2.520101in}}%
\pgfpathlineto{\pgfqpoint{3.373712in}{2.494605in}}%
\pgfpathlineto{\pgfqpoint{3.451602in}{2.457628in}}%
\pgfpathlineto{\pgfqpoint{3.529491in}{2.417737in}}%
\pgfpathlineto{\pgfqpoint{3.607381in}{2.375473in}}%
\pgfpathlineto{\pgfqpoint{3.704742in}{2.320107in}}%
\pgfpathlineto{\pgfqpoint{3.841049in}{2.239567in}}%
\pgfpathlineto{\pgfqpoint{4.211024in}{2.019009in}}%
\pgfpathlineto{\pgfqpoint{4.327858in}{1.952417in}}%
\pgfpathlineto{\pgfqpoint{4.444692in}{1.888394in}}%
\pgfpathlineto{\pgfqpoint{4.542054in}{1.837265in}}%
\pgfpathlineto{\pgfqpoint{4.639416in}{1.788301in}}%
\pgfpathlineto{\pgfqpoint{4.736778in}{1.741574in}}%
\pgfpathlineto{\pgfqpoint{4.756250in}{1.732500in}}%
\pgfpathlineto{\pgfqpoint{4.756250in}{1.732500in}}%
\pgfusepath{stroke}%
\end{pgfscope}%
\begin{pgfscope}%
\pgfpathrectangle{\pgfqpoint{0.687500in}{0.385000in}}{\pgfqpoint{4.262500in}{2.695000in}}%
\pgfusepath{clip}%
\pgfsetbuttcap%
\pgfsetroundjoin%
\definecolor{currentfill}{rgb}{1.000000,0.498039,0.054902}%
\pgfsetfillcolor{currentfill}%
\pgfsetlinewidth{1.003750pt}%
\definecolor{currentstroke}{rgb}{1.000000,0.498039,0.054902}%
\pgfsetstrokecolor{currentstroke}%
\pgfsetdash{}{0pt}%
\pgfsys@defobject{currentmarker}{\pgfqpoint{-0.020833in}{-0.020833in}}{\pgfqpoint{0.020833in}{0.020833in}}{%
\pgfpathmoveto{\pgfqpoint{0.000000in}{-0.020833in}}%
\pgfpathcurveto{\pgfqpoint{0.005525in}{-0.020833in}}{\pgfqpoint{0.010825in}{-0.018638in}}{\pgfqpoint{0.014731in}{-0.014731in}}%
\pgfpathcurveto{\pgfqpoint{0.018638in}{-0.010825in}}{\pgfqpoint{0.020833in}{-0.005525in}}{\pgfqpoint{0.020833in}{0.000000in}}%
\pgfpathcurveto{\pgfqpoint{0.020833in}{0.005525in}}{\pgfqpoint{0.018638in}{0.010825in}}{\pgfqpoint{0.014731in}{0.014731in}}%
\pgfpathcurveto{\pgfqpoint{0.010825in}{0.018638in}}{\pgfqpoint{0.005525in}{0.020833in}}{\pgfqpoint{0.000000in}{0.020833in}}%
\pgfpathcurveto{\pgfqpoint{-0.005525in}{0.020833in}}{\pgfqpoint{-0.010825in}{0.018638in}}{\pgfqpoint{-0.014731in}{0.014731in}}%
\pgfpathcurveto{\pgfqpoint{-0.018638in}{0.010825in}}{\pgfqpoint{-0.020833in}{0.005525in}}{\pgfqpoint{-0.020833in}{0.000000in}}%
\pgfpathcurveto{\pgfqpoint{-0.020833in}{-0.005525in}}{\pgfqpoint{-0.018638in}{-0.010825in}}{\pgfqpoint{-0.014731in}{-0.014731in}}%
\pgfpathcurveto{\pgfqpoint{-0.010825in}{-0.018638in}}{\pgfqpoint{-0.005525in}{-0.020833in}}{\pgfqpoint{0.000000in}{-0.020833in}}%
\pgfpathclose%
\pgfusepath{stroke,fill}%
}%
\begin{pgfscope}%
\pgfsys@transformshift{4.496674in}{1.860833in}%
\pgfsys@useobject{currentmarker}{}%
\end{pgfscope}%
\begin{pgfscope}%
\pgfsys@transformshift{2.818750in}{2.630833in}%
\pgfsys@useobject{currentmarker}{}%
\end{pgfscope}%
\begin{pgfscope}%
\pgfsys@transformshift{1.140826in}{1.860833in}%
\pgfsys@useobject{currentmarker}{}%
\end{pgfscope}%
\end{pgfscope}%
\begin{pgfscope}%
\pgfpathrectangle{\pgfqpoint{0.687500in}{0.385000in}}{\pgfqpoint{4.262500in}{2.695000in}}%
\pgfusepath{clip}%
\pgfsetrectcap%
\pgfsetroundjoin%
\pgfsetlinewidth{1.505625pt}%
\definecolor{currentstroke}{rgb}{0.172549,0.627451,0.172549}%
\pgfsetstrokecolor{currentstroke}%
\pgfsetdash{}{0pt}%
\pgfpathmoveto{\pgfqpoint{0.881250in}{1.604167in}}%
\pgfpathlineto{\pgfqpoint{0.959139in}{1.685054in}}%
\pgfpathlineto{\pgfqpoint{1.037029in}{1.762622in}}%
\pgfpathlineto{\pgfqpoint{1.114918in}{1.836872in}}%
\pgfpathlineto{\pgfqpoint{1.192808in}{1.907803in}}%
\pgfpathlineto{\pgfqpoint{1.270697in}{1.975417in}}%
\pgfpathlineto{\pgfqpoint{1.348587in}{2.039711in}}%
\pgfpathlineto{\pgfqpoint{1.426476in}{2.100687in}}%
\pgfpathlineto{\pgfqpoint{1.504366in}{2.158345in}}%
\pgfpathlineto{\pgfqpoint{1.582255in}{2.212685in}}%
\pgfpathlineto{\pgfqpoint{1.660144in}{2.263706in}}%
\pgfpathlineto{\pgfqpoint{1.738034in}{2.311408in}}%
\pgfpathlineto{\pgfqpoint{1.815923in}{2.355792in}}%
\pgfpathlineto{\pgfqpoint{1.893813in}{2.396858in}}%
\pgfpathlineto{\pgfqpoint{1.952230in}{2.425479in}}%
\pgfpathlineto{\pgfqpoint{2.010647in}{2.452234in}}%
\pgfpathlineto{\pgfqpoint{2.069064in}{2.477122in}}%
\pgfpathlineto{\pgfqpoint{2.127481in}{2.500144in}}%
\pgfpathlineto{\pgfqpoint{2.185898in}{2.521299in}}%
\pgfpathlineto{\pgfqpoint{2.244315in}{2.540587in}}%
\pgfpathlineto{\pgfqpoint{2.302732in}{2.558009in}}%
\pgfpathlineto{\pgfqpoint{2.361149in}{2.573564in}}%
\pgfpathlineto{\pgfqpoint{2.419567in}{2.587253in}}%
\pgfpathlineto{\pgfqpoint{2.477984in}{2.599075in}}%
\pgfpathlineto{\pgfqpoint{2.536401in}{2.609030in}}%
\pgfpathlineto{\pgfqpoint{2.594818in}{2.617119in}}%
\pgfpathlineto{\pgfqpoint{2.653235in}{2.623341in}}%
\pgfpathlineto{\pgfqpoint{2.711652in}{2.627696in}}%
\pgfpathlineto{\pgfqpoint{2.770069in}{2.630185in}}%
\pgfpathlineto{\pgfqpoint{2.828486in}{2.630807in}}%
\pgfpathlineto{\pgfqpoint{2.886903in}{2.629563in}}%
\pgfpathlineto{\pgfqpoint{2.945320in}{2.626452in}}%
\pgfpathlineto{\pgfqpoint{3.003737in}{2.621474in}}%
\pgfpathlineto{\pgfqpoint{3.062155in}{2.614630in}}%
\pgfpathlineto{\pgfqpoint{3.120572in}{2.605919in}}%
\pgfpathlineto{\pgfqpoint{3.178989in}{2.595342in}}%
\pgfpathlineto{\pgfqpoint{3.237406in}{2.582898in}}%
\pgfpathlineto{\pgfqpoint{3.295823in}{2.568587in}}%
\pgfpathlineto{\pgfqpoint{3.354240in}{2.552409in}}%
\pgfpathlineto{\pgfqpoint{3.412657in}{2.534365in}}%
\pgfpathlineto{\pgfqpoint{3.471074in}{2.514455in}}%
\pgfpathlineto{\pgfqpoint{3.529491in}{2.492678in}}%
\pgfpathlineto{\pgfqpoint{3.587908in}{2.469034in}}%
\pgfpathlineto{\pgfqpoint{3.646325in}{2.443523in}}%
\pgfpathlineto{\pgfqpoint{3.704742in}{2.416146in}}%
\pgfpathlineto{\pgfqpoint{3.763160in}{2.386902in}}%
\pgfpathlineto{\pgfqpoint{3.841049in}{2.345007in}}%
\pgfpathlineto{\pgfqpoint{3.918938in}{2.299794in}}%
\pgfpathlineto{\pgfqpoint{3.996828in}{2.251261in}}%
\pgfpathlineto{\pgfqpoint{4.074717in}{2.199411in}}%
\pgfpathlineto{\pgfqpoint{4.152607in}{2.144242in}}%
\pgfpathlineto{\pgfqpoint{4.230496in}{2.085755in}}%
\pgfpathlineto{\pgfqpoint{4.308386in}{2.023949in}}%
\pgfpathlineto{\pgfqpoint{4.386275in}{1.958824in}}%
\pgfpathlineto{\pgfqpoint{4.464165in}{1.890382in}}%
\pgfpathlineto{\pgfqpoint{4.542054in}{1.818621in}}%
\pgfpathlineto{\pgfqpoint{4.619943in}{1.743541in}}%
\pgfpathlineto{\pgfqpoint{4.697833in}{1.665143in}}%
\pgfpathlineto{\pgfqpoint{4.756250in}{1.604167in}}%
\pgfpathlineto{\pgfqpoint{4.756250in}{1.604167in}}%
\pgfusepath{stroke}%
\end{pgfscope}%
\begin{pgfscope}%
\pgfpathrectangle{\pgfqpoint{0.687500in}{0.385000in}}{\pgfqpoint{4.262500in}{2.695000in}}%
\pgfusepath{clip}%
\pgfsetbuttcap%
\pgfsetroundjoin%
\definecolor{currentfill}{rgb}{0.839216,0.152941,0.156863}%
\pgfsetfillcolor{currentfill}%
\pgfsetlinewidth{1.003750pt}%
\definecolor{currentstroke}{rgb}{0.839216,0.152941,0.156863}%
\pgfsetstrokecolor{currentstroke}%
\pgfsetdash{}{0pt}%
\pgfsys@defobject{currentmarker}{\pgfqpoint{-0.020833in}{-0.020833in}}{\pgfqpoint{0.020833in}{0.020833in}}{%
\pgfpathmoveto{\pgfqpoint{0.000000in}{-0.020833in}}%
\pgfpathcurveto{\pgfqpoint{0.005525in}{-0.020833in}}{\pgfqpoint{0.010825in}{-0.018638in}}{\pgfqpoint{0.014731in}{-0.014731in}}%
\pgfpathcurveto{\pgfqpoint{0.018638in}{-0.010825in}}{\pgfqpoint{0.020833in}{-0.005525in}}{\pgfqpoint{0.020833in}{0.000000in}}%
\pgfpathcurveto{\pgfqpoint{0.020833in}{0.005525in}}{\pgfqpoint{0.018638in}{0.010825in}}{\pgfqpoint{0.014731in}{0.014731in}}%
\pgfpathcurveto{\pgfqpoint{0.010825in}{0.018638in}}{\pgfqpoint{0.005525in}{0.020833in}}{\pgfqpoint{0.000000in}{0.020833in}}%
\pgfpathcurveto{\pgfqpoint{-0.005525in}{0.020833in}}{\pgfqpoint{-0.010825in}{0.018638in}}{\pgfqpoint{-0.014731in}{0.014731in}}%
\pgfpathcurveto{\pgfqpoint{-0.018638in}{0.010825in}}{\pgfqpoint{-0.020833in}{0.005525in}}{\pgfqpoint{-0.020833in}{0.000000in}}%
\pgfpathcurveto{\pgfqpoint{-0.020833in}{-0.005525in}}{\pgfqpoint{-0.018638in}{-0.010825in}}{\pgfqpoint{-0.014731in}{-0.014731in}}%
\pgfpathcurveto{\pgfqpoint{-0.010825in}{-0.018638in}}{\pgfqpoint{-0.005525in}{-0.020833in}}{\pgfqpoint{0.000000in}{-0.020833in}}%
\pgfpathclose%
\pgfusepath{stroke,fill}%
}%
\begin{pgfscope}%
\pgfsys@transformshift{4.661422in}{1.777542in}%
\pgfsys@useobject{currentmarker}{}%
\end{pgfscope}%
\begin{pgfscope}%
\pgfsys@transformshift{3.957584in}{2.169490in}%
\pgfsys@useobject{currentmarker}{}%
\end{pgfscope}%
\begin{pgfscope}%
\pgfsys@transformshift{2.818750in}{2.630833in}%
\pgfsys@useobject{currentmarker}{}%
\end{pgfscope}%
\begin{pgfscope}%
\pgfsys@transformshift{1.679916in}{2.169490in}%
\pgfsys@useobject{currentmarker}{}%
\end{pgfscope}%
\begin{pgfscope}%
\pgfsys@transformshift{0.976078in}{1.777542in}%
\pgfsys@useobject{currentmarker}{}%
\end{pgfscope}%
\end{pgfscope}%
\begin{pgfscope}%
\pgfpathrectangle{\pgfqpoint{0.687500in}{0.385000in}}{\pgfqpoint{4.262500in}{2.695000in}}%
\pgfusepath{clip}%
\pgfsetrectcap%
\pgfsetroundjoin%
\pgfsetlinewidth{1.505625pt}%
\definecolor{currentstroke}{rgb}{0.580392,0.403922,0.741176}%
\pgfsetstrokecolor{currentstroke}%
\pgfsetdash{}{0pt}%
\pgfpathmoveto{\pgfqpoint{0.881250in}{1.754411in}}%
\pgfpathlineto{\pgfqpoint{0.920195in}{1.762497in}}%
\pgfpathlineto{\pgfqpoint{0.959139in}{1.772573in}}%
\pgfpathlineto{\pgfqpoint{0.998084in}{1.784505in}}%
\pgfpathlineto{\pgfqpoint{1.037029in}{1.798166in}}%
\pgfpathlineto{\pgfqpoint{1.095446in}{1.821617in}}%
\pgfpathlineto{\pgfqpoint{1.153863in}{1.848252in}}%
\pgfpathlineto{\pgfqpoint{1.212280in}{1.877668in}}%
\pgfpathlineto{\pgfqpoint{1.270697in}{1.909475in}}%
\pgfpathlineto{\pgfqpoint{1.348587in}{1.954957in}}%
\pgfpathlineto{\pgfqpoint{1.426476in}{2.003173in}}%
\pgfpathlineto{\pgfqpoint{1.523838in}{2.066061in}}%
\pgfpathlineto{\pgfqpoint{1.874340in}{2.295593in}}%
\pgfpathlineto{\pgfqpoint{1.952230in}{2.343340in}}%
\pgfpathlineto{\pgfqpoint{2.030119in}{2.388712in}}%
\pgfpathlineto{\pgfqpoint{2.108009in}{2.431242in}}%
\pgfpathlineto{\pgfqpoint{2.166426in}{2.461017in}}%
\pgfpathlineto{\pgfqpoint{2.224843in}{2.488792in}}%
\pgfpathlineto{\pgfqpoint{2.283260in}{2.514419in}}%
\pgfpathlineto{\pgfqpoint{2.341677in}{2.537764in}}%
\pgfpathlineto{\pgfqpoint{2.400094in}{2.558704in}}%
\pgfpathlineto{\pgfqpoint{2.458511in}{2.577135in}}%
\pgfpathlineto{\pgfqpoint{2.516928in}{2.592963in}}%
\pgfpathlineto{\pgfqpoint{2.575345in}{2.606110in}}%
\pgfpathlineto{\pgfqpoint{2.633763in}{2.616511in}}%
\pgfpathlineto{\pgfqpoint{2.692180in}{2.624114in}}%
\pgfpathlineto{\pgfqpoint{2.750597in}{2.628882in}}%
\pgfpathlineto{\pgfqpoint{2.809014in}{2.630793in}}%
\pgfpathlineto{\pgfqpoint{2.867431in}{2.629838in}}%
\pgfpathlineto{\pgfqpoint{2.925848in}{2.626020in}}%
\pgfpathlineto{\pgfqpoint{2.984265in}{2.619358in}}%
\pgfpathlineto{\pgfqpoint{3.042682in}{2.609885in}}%
\pgfpathlineto{\pgfqpoint{3.101099in}{2.597647in}}%
\pgfpathlineto{\pgfqpoint{3.159516in}{2.582705in}}%
\pgfpathlineto{\pgfqpoint{3.217933in}{2.565132in}}%
\pgfpathlineto{\pgfqpoint{3.276351in}{2.545016in}}%
\pgfpathlineto{\pgfqpoint{3.334768in}{2.522461in}}%
\pgfpathlineto{\pgfqpoint{3.393185in}{2.497580in}}%
\pgfpathlineto{\pgfqpoint{3.451602in}{2.470505in}}%
\pgfpathlineto{\pgfqpoint{3.510019in}{2.441379in}}%
\pgfpathlineto{\pgfqpoint{3.587908in}{2.399628in}}%
\pgfpathlineto{\pgfqpoint{3.665798in}{2.354925in}}%
\pgfpathlineto{\pgfqpoint{3.743687in}{2.307725in}}%
\pgfpathlineto{\pgfqpoint{3.841049in}{2.245981in}}%
\pgfpathlineto{\pgfqpoint{3.977356in}{2.156366in}}%
\pgfpathlineto{\pgfqpoint{4.172079in}{2.028052in}}%
\pgfpathlineto{\pgfqpoint{4.269441in}{1.966787in}}%
\pgfpathlineto{\pgfqpoint{4.347330in}{1.920544in}}%
\pgfpathlineto{\pgfqpoint{4.405747in}{1.888024in}}%
\pgfpathlineto{\pgfqpoint{4.464165in}{1.857768in}}%
\pgfpathlineto{\pgfqpoint{4.522582in}{1.830162in}}%
\pgfpathlineto{\pgfqpoint{4.580999in}{1.805603in}}%
\pgfpathlineto{\pgfqpoint{4.619943in}{1.791128in}}%
\pgfpathlineto{\pgfqpoint{4.658888in}{1.778315in}}%
\pgfpathlineto{\pgfqpoint{4.697833in}{1.767294in}}%
\pgfpathlineto{\pgfqpoint{4.736778in}{1.758197in}}%
\pgfpathlineto{\pgfqpoint{4.756250in}{1.754411in}}%
\pgfpathlineto{\pgfqpoint{4.756250in}{1.754411in}}%
\pgfusepath{stroke}%
\end{pgfscope}%
\begin{pgfscope}%
\pgfsetrectcap%
\pgfsetmiterjoin%
\pgfsetlinewidth{0.803000pt}%
\definecolor{currentstroke}{rgb}{0.000000,0.000000,0.000000}%
\pgfsetstrokecolor{currentstroke}%
\pgfsetdash{}{0pt}%
\pgfpathmoveto{\pgfqpoint{0.687500in}{0.385000in}}%
\pgfpathlineto{\pgfqpoint{0.687500in}{3.080000in}}%
\pgfusepath{stroke}%
\end{pgfscope}%
\begin{pgfscope}%
\pgfsetrectcap%
\pgfsetmiterjoin%
\pgfsetlinewidth{0.803000pt}%
\definecolor{currentstroke}{rgb}{0.000000,0.000000,0.000000}%
\pgfsetstrokecolor{currentstroke}%
\pgfsetdash{}{0pt}%
\pgfpathmoveto{\pgfqpoint{4.950000in}{0.385000in}}%
\pgfpathlineto{\pgfqpoint{4.950000in}{3.080000in}}%
\pgfusepath{stroke}%
\end{pgfscope}%
\begin{pgfscope}%
\pgfsetrectcap%
\pgfsetmiterjoin%
\pgfsetlinewidth{0.803000pt}%
\definecolor{currentstroke}{rgb}{0.000000,0.000000,0.000000}%
\pgfsetstrokecolor{currentstroke}%
\pgfsetdash{}{0pt}%
\pgfpathmoveto{\pgfqpoint{0.687500in}{0.385000in}}%
\pgfpathlineto{\pgfqpoint{4.950000in}{0.385000in}}%
\pgfusepath{stroke}%
\end{pgfscope}%
\begin{pgfscope}%
\pgfsetrectcap%
\pgfsetmiterjoin%
\pgfsetlinewidth{0.803000pt}%
\definecolor{currentstroke}{rgb}{0.000000,0.000000,0.000000}%
\pgfsetstrokecolor{currentstroke}%
\pgfsetdash{}{0pt}%
\pgfpathmoveto{\pgfqpoint{0.687500in}{3.080000in}}%
\pgfpathlineto{\pgfqpoint{4.950000in}{3.080000in}}%
\pgfusepath{stroke}%
\end{pgfscope}%
\begin{pgfscope}%
\definecolor{textcolor}{rgb}{0.000000,0.000000,0.000000}%
\pgfsetstrokecolor{textcolor}%
\pgfsetfillcolor{textcolor}%
\pgftext[x=2.818750in,y=3.163333in,,base]{\color{textcolor}\rmfamily\fontsize{12.000000}{14.400000}\selectfont N=2, 4}%
\end{pgfscope}%
\begin{pgfscope}%
\pgfsetbuttcap%
\pgfsetmiterjoin%
\definecolor{currentfill}{rgb}{1.000000,1.000000,1.000000}%
\pgfsetfillcolor{currentfill}%
\pgfsetfillopacity{0.800000}%
\pgfsetlinewidth{1.003750pt}%
\definecolor{currentstroke}{rgb}{0.800000,0.800000,0.800000}%
\pgfsetstrokecolor{currentstroke}%
\pgfsetstrokeopacity{0.800000}%
\pgfsetdash{}{0pt}%
\pgfpathmoveto{\pgfqpoint{0.784722in}{0.454444in}}%
\pgfpathlineto{\pgfqpoint{2.050468in}{0.454444in}}%
\pgfpathquadraticcurveto{\pgfqpoint{2.078246in}{0.454444in}}{\pgfqpoint{2.078246in}{0.482222in}}%
\pgfpathlineto{\pgfqpoint{2.078246in}{1.244755in}}%
\pgfpathquadraticcurveto{\pgfqpoint{2.078246in}{1.272533in}}{\pgfqpoint{2.050468in}{1.272533in}}%
\pgfpathlineto{\pgfqpoint{0.784722in}{1.272533in}}%
\pgfpathquadraticcurveto{\pgfqpoint{0.756944in}{1.272533in}}{\pgfqpoint{0.756944in}{1.244755in}}%
\pgfpathlineto{\pgfqpoint{0.756944in}{0.482222in}}%
\pgfpathquadraticcurveto{\pgfqpoint{0.756944in}{0.454444in}}{\pgfqpoint{0.784722in}{0.454444in}}%
\pgfpathclose%
\pgfusepath{stroke,fill}%
\end{pgfscope}%
\begin{pgfscope}%
\pgfsetrectcap%
\pgfsetroundjoin%
\pgfsetlinewidth{1.505625pt}%
\definecolor{currentstroke}{rgb}{0.121569,0.466667,0.705882}%
\pgfsetstrokecolor{currentstroke}%
\pgfsetdash{}{0pt}%
\pgfpathmoveto{\pgfqpoint{0.812500in}{1.082123in}}%
\pgfpathlineto{\pgfqpoint{1.090278in}{1.082123in}}%
\pgfusepath{stroke}%
\end{pgfscope}%
\begin{pgfscope}%
\definecolor{textcolor}{rgb}{0.000000,0.000000,0.000000}%
\pgfsetstrokecolor{textcolor}%
\pgfsetfillcolor{textcolor}%
\pgftext[x=1.201389in,y=1.033512in,left,base]{\color{textcolor}\rmfamily\fontsize{10.000000}{12.000000}\selectfont \(\displaystyle y(x)=\)\(\displaystyle \frac{1}{1+x^{2}}\)}%
\end{pgfscope}%
\begin{pgfscope}%
\pgfsetrectcap%
\pgfsetroundjoin%
\pgfsetlinewidth{1.505625pt}%
\definecolor{currentstroke}{rgb}{0.172549,0.627451,0.172549}%
\pgfsetstrokecolor{currentstroke}%
\pgfsetdash{}{0pt}%
\pgfpathmoveto{\pgfqpoint{0.812500in}{0.801667in}}%
\pgfpathlineto{\pgfqpoint{1.090278in}{0.801667in}}%
\pgfusepath{stroke}%
\end{pgfscope}%
\begin{pgfscope}%
\definecolor{textcolor}{rgb}{0.000000,0.000000,0.000000}%
\pgfsetstrokecolor{textcolor}%
\pgfsetfillcolor{textcolor}%
\pgftext[x=1.201389in,y=0.753056in,left,base]{\color{textcolor}\rmfamily\fontsize{10.000000}{12.000000}\selectfont W2(x)}%
\end{pgfscope}%
\begin{pgfscope}%
\pgfsetrectcap%
\pgfsetroundjoin%
\pgfsetlinewidth{1.505625pt}%
\definecolor{currentstroke}{rgb}{0.580392,0.403922,0.741176}%
\pgfsetstrokecolor{currentstroke}%
\pgfsetdash{}{0pt}%
\pgfpathmoveto{\pgfqpoint{0.812500in}{0.593333in}}%
\pgfpathlineto{\pgfqpoint{1.090278in}{0.593333in}}%
\pgfusepath{stroke}%
\end{pgfscope}%
\begin{pgfscope}%
\definecolor{textcolor}{rgb}{0.000000,0.000000,0.000000}%
\pgfsetstrokecolor{textcolor}%
\pgfsetfillcolor{textcolor}%
\pgftext[x=1.201389in,y=0.544722in,left,base]{\color{textcolor}\rmfamily\fontsize{10.000000}{12.000000}\selectfont W4(x)}%
\end{pgfscope}%
\end{pgfpicture}%
\makeatother%
\endgroup%
        
    \end{center}
    \caption{Węzły cosinus, funkcja \(\tilde{y}\), \(N=2,4\)}
\end{figure}

\begin{figure}[h]
    \begin{center}
        %% Creator: Matplotlib, PGF backend
%%
%% To include the figure in your LaTeX document, write
%%   \input{<filename>.pgf}
%%
%% Make sure the required packages are loaded in your preamble
%%   \usepackage{pgf}
%%
%% Figures using additional raster images can only be included by \input if
%% they are in the same directory as the main LaTeX file. For loading figures
%% from other directories you can use the `import` package
%%   \usepackage{import}
%% and then include the figures with
%%   \import{<path to file>}{<filename>.pgf}
%%
%% Matplotlib used the following preamble
%%
\begingroup%
\makeatletter%
\begin{pgfpicture}%
\pgfpathrectangle{\pgfpointorigin}{\pgfqpoint{5.500000in}{3.500000in}}%
\pgfusepath{use as bounding box, clip}%
\begin{pgfscope}%
\pgfsetbuttcap%
\pgfsetmiterjoin%
\definecolor{currentfill}{rgb}{1.000000,1.000000,1.000000}%
\pgfsetfillcolor{currentfill}%
\pgfsetlinewidth{0.000000pt}%
\definecolor{currentstroke}{rgb}{1.000000,1.000000,1.000000}%
\pgfsetstrokecolor{currentstroke}%
\pgfsetdash{}{0pt}%
\pgfpathmoveto{\pgfqpoint{0.000000in}{0.000000in}}%
\pgfpathlineto{\pgfqpoint{5.500000in}{0.000000in}}%
\pgfpathlineto{\pgfqpoint{5.500000in}{3.500000in}}%
\pgfpathlineto{\pgfqpoint{0.000000in}{3.500000in}}%
\pgfpathclose%
\pgfusepath{fill}%
\end{pgfscope}%
\begin{pgfscope}%
\pgfsetbuttcap%
\pgfsetmiterjoin%
\definecolor{currentfill}{rgb}{1.000000,1.000000,1.000000}%
\pgfsetfillcolor{currentfill}%
\pgfsetlinewidth{0.000000pt}%
\definecolor{currentstroke}{rgb}{0.000000,0.000000,0.000000}%
\pgfsetstrokecolor{currentstroke}%
\pgfsetstrokeopacity{0.000000}%
\pgfsetdash{}{0pt}%
\pgfpathmoveto{\pgfqpoint{0.687500in}{0.385000in}}%
\pgfpathlineto{\pgfqpoint{4.950000in}{0.385000in}}%
\pgfpathlineto{\pgfqpoint{4.950000in}{3.080000in}}%
\pgfpathlineto{\pgfqpoint{0.687500in}{3.080000in}}%
\pgfpathclose%
\pgfusepath{fill}%
\end{pgfscope}%
\begin{pgfscope}%
\pgfsetbuttcap%
\pgfsetroundjoin%
\definecolor{currentfill}{rgb}{0.000000,0.000000,0.000000}%
\pgfsetfillcolor{currentfill}%
\pgfsetlinewidth{0.803000pt}%
\definecolor{currentstroke}{rgb}{0.000000,0.000000,0.000000}%
\pgfsetstrokecolor{currentstroke}%
\pgfsetdash{}{0pt}%
\pgfsys@defobject{currentmarker}{\pgfqpoint{0.000000in}{-0.048611in}}{\pgfqpoint{0.000000in}{0.000000in}}{%
\pgfpathmoveto{\pgfqpoint{0.000000in}{0.000000in}}%
\pgfpathlineto{\pgfqpoint{0.000000in}{-0.048611in}}%
\pgfusepath{stroke,fill}%
}%
\begin{pgfscope}%
\pgfsys@transformshift{0.881250in}{0.385000in}%
\pgfsys@useobject{currentmarker}{}%
\end{pgfscope}%
\end{pgfscope}%
\begin{pgfscope}%
\definecolor{textcolor}{rgb}{0.000000,0.000000,0.000000}%
\pgfsetstrokecolor{textcolor}%
\pgfsetfillcolor{textcolor}%
\pgftext[x=0.881250in,y=0.287778in,,top]{\color{textcolor}\rmfamily\fontsize{10.000000}{12.000000}\selectfont \(\displaystyle -1.00\)}%
\end{pgfscope}%
\begin{pgfscope}%
\pgfsetbuttcap%
\pgfsetroundjoin%
\definecolor{currentfill}{rgb}{0.000000,0.000000,0.000000}%
\pgfsetfillcolor{currentfill}%
\pgfsetlinewidth{0.803000pt}%
\definecolor{currentstroke}{rgb}{0.000000,0.000000,0.000000}%
\pgfsetstrokecolor{currentstroke}%
\pgfsetdash{}{0pt}%
\pgfsys@defobject{currentmarker}{\pgfqpoint{0.000000in}{-0.048611in}}{\pgfqpoint{0.000000in}{0.000000in}}{%
\pgfpathmoveto{\pgfqpoint{0.000000in}{0.000000in}}%
\pgfpathlineto{\pgfqpoint{0.000000in}{-0.048611in}}%
\pgfusepath{stroke,fill}%
}%
\begin{pgfscope}%
\pgfsys@transformshift{1.365625in}{0.385000in}%
\pgfsys@useobject{currentmarker}{}%
\end{pgfscope}%
\end{pgfscope}%
\begin{pgfscope}%
\definecolor{textcolor}{rgb}{0.000000,0.000000,0.000000}%
\pgfsetstrokecolor{textcolor}%
\pgfsetfillcolor{textcolor}%
\pgftext[x=1.365625in,y=0.287778in,,top]{\color{textcolor}\rmfamily\fontsize{10.000000}{12.000000}\selectfont \(\displaystyle -0.75\)}%
\end{pgfscope}%
\begin{pgfscope}%
\pgfsetbuttcap%
\pgfsetroundjoin%
\definecolor{currentfill}{rgb}{0.000000,0.000000,0.000000}%
\pgfsetfillcolor{currentfill}%
\pgfsetlinewidth{0.803000pt}%
\definecolor{currentstroke}{rgb}{0.000000,0.000000,0.000000}%
\pgfsetstrokecolor{currentstroke}%
\pgfsetdash{}{0pt}%
\pgfsys@defobject{currentmarker}{\pgfqpoint{0.000000in}{-0.048611in}}{\pgfqpoint{0.000000in}{0.000000in}}{%
\pgfpathmoveto{\pgfqpoint{0.000000in}{0.000000in}}%
\pgfpathlineto{\pgfqpoint{0.000000in}{-0.048611in}}%
\pgfusepath{stroke,fill}%
}%
\begin{pgfscope}%
\pgfsys@transformshift{1.850000in}{0.385000in}%
\pgfsys@useobject{currentmarker}{}%
\end{pgfscope}%
\end{pgfscope}%
\begin{pgfscope}%
\definecolor{textcolor}{rgb}{0.000000,0.000000,0.000000}%
\pgfsetstrokecolor{textcolor}%
\pgfsetfillcolor{textcolor}%
\pgftext[x=1.850000in,y=0.287778in,,top]{\color{textcolor}\rmfamily\fontsize{10.000000}{12.000000}\selectfont \(\displaystyle -0.50\)}%
\end{pgfscope}%
\begin{pgfscope}%
\pgfsetbuttcap%
\pgfsetroundjoin%
\definecolor{currentfill}{rgb}{0.000000,0.000000,0.000000}%
\pgfsetfillcolor{currentfill}%
\pgfsetlinewidth{0.803000pt}%
\definecolor{currentstroke}{rgb}{0.000000,0.000000,0.000000}%
\pgfsetstrokecolor{currentstroke}%
\pgfsetdash{}{0pt}%
\pgfsys@defobject{currentmarker}{\pgfqpoint{0.000000in}{-0.048611in}}{\pgfqpoint{0.000000in}{0.000000in}}{%
\pgfpathmoveto{\pgfqpoint{0.000000in}{0.000000in}}%
\pgfpathlineto{\pgfqpoint{0.000000in}{-0.048611in}}%
\pgfusepath{stroke,fill}%
}%
\begin{pgfscope}%
\pgfsys@transformshift{2.334375in}{0.385000in}%
\pgfsys@useobject{currentmarker}{}%
\end{pgfscope}%
\end{pgfscope}%
\begin{pgfscope}%
\definecolor{textcolor}{rgb}{0.000000,0.000000,0.000000}%
\pgfsetstrokecolor{textcolor}%
\pgfsetfillcolor{textcolor}%
\pgftext[x=2.334375in,y=0.287778in,,top]{\color{textcolor}\rmfamily\fontsize{10.000000}{12.000000}\selectfont \(\displaystyle -0.25\)}%
\end{pgfscope}%
\begin{pgfscope}%
\pgfsetbuttcap%
\pgfsetroundjoin%
\definecolor{currentfill}{rgb}{0.000000,0.000000,0.000000}%
\pgfsetfillcolor{currentfill}%
\pgfsetlinewidth{0.803000pt}%
\definecolor{currentstroke}{rgb}{0.000000,0.000000,0.000000}%
\pgfsetstrokecolor{currentstroke}%
\pgfsetdash{}{0pt}%
\pgfsys@defobject{currentmarker}{\pgfqpoint{0.000000in}{-0.048611in}}{\pgfqpoint{0.000000in}{0.000000in}}{%
\pgfpathmoveto{\pgfqpoint{0.000000in}{0.000000in}}%
\pgfpathlineto{\pgfqpoint{0.000000in}{-0.048611in}}%
\pgfusepath{stroke,fill}%
}%
\begin{pgfscope}%
\pgfsys@transformshift{2.818750in}{0.385000in}%
\pgfsys@useobject{currentmarker}{}%
\end{pgfscope}%
\end{pgfscope}%
\begin{pgfscope}%
\definecolor{textcolor}{rgb}{0.000000,0.000000,0.000000}%
\pgfsetstrokecolor{textcolor}%
\pgfsetfillcolor{textcolor}%
\pgftext[x=2.818750in,y=0.287778in,,top]{\color{textcolor}\rmfamily\fontsize{10.000000}{12.000000}\selectfont \(\displaystyle 0.00\)}%
\end{pgfscope}%
\begin{pgfscope}%
\pgfsetbuttcap%
\pgfsetroundjoin%
\definecolor{currentfill}{rgb}{0.000000,0.000000,0.000000}%
\pgfsetfillcolor{currentfill}%
\pgfsetlinewidth{0.803000pt}%
\definecolor{currentstroke}{rgb}{0.000000,0.000000,0.000000}%
\pgfsetstrokecolor{currentstroke}%
\pgfsetdash{}{0pt}%
\pgfsys@defobject{currentmarker}{\pgfqpoint{0.000000in}{-0.048611in}}{\pgfqpoint{0.000000in}{0.000000in}}{%
\pgfpathmoveto{\pgfqpoint{0.000000in}{0.000000in}}%
\pgfpathlineto{\pgfqpoint{0.000000in}{-0.048611in}}%
\pgfusepath{stroke,fill}%
}%
\begin{pgfscope}%
\pgfsys@transformshift{3.303125in}{0.385000in}%
\pgfsys@useobject{currentmarker}{}%
\end{pgfscope}%
\end{pgfscope}%
\begin{pgfscope}%
\definecolor{textcolor}{rgb}{0.000000,0.000000,0.000000}%
\pgfsetstrokecolor{textcolor}%
\pgfsetfillcolor{textcolor}%
\pgftext[x=3.303125in,y=0.287778in,,top]{\color{textcolor}\rmfamily\fontsize{10.000000}{12.000000}\selectfont \(\displaystyle 0.25\)}%
\end{pgfscope}%
\begin{pgfscope}%
\pgfsetbuttcap%
\pgfsetroundjoin%
\definecolor{currentfill}{rgb}{0.000000,0.000000,0.000000}%
\pgfsetfillcolor{currentfill}%
\pgfsetlinewidth{0.803000pt}%
\definecolor{currentstroke}{rgb}{0.000000,0.000000,0.000000}%
\pgfsetstrokecolor{currentstroke}%
\pgfsetdash{}{0pt}%
\pgfsys@defobject{currentmarker}{\pgfqpoint{0.000000in}{-0.048611in}}{\pgfqpoint{0.000000in}{0.000000in}}{%
\pgfpathmoveto{\pgfqpoint{0.000000in}{0.000000in}}%
\pgfpathlineto{\pgfqpoint{0.000000in}{-0.048611in}}%
\pgfusepath{stroke,fill}%
}%
\begin{pgfscope}%
\pgfsys@transformshift{3.787500in}{0.385000in}%
\pgfsys@useobject{currentmarker}{}%
\end{pgfscope}%
\end{pgfscope}%
\begin{pgfscope}%
\definecolor{textcolor}{rgb}{0.000000,0.000000,0.000000}%
\pgfsetstrokecolor{textcolor}%
\pgfsetfillcolor{textcolor}%
\pgftext[x=3.787500in,y=0.287778in,,top]{\color{textcolor}\rmfamily\fontsize{10.000000}{12.000000}\selectfont \(\displaystyle 0.50\)}%
\end{pgfscope}%
\begin{pgfscope}%
\pgfsetbuttcap%
\pgfsetroundjoin%
\definecolor{currentfill}{rgb}{0.000000,0.000000,0.000000}%
\pgfsetfillcolor{currentfill}%
\pgfsetlinewidth{0.803000pt}%
\definecolor{currentstroke}{rgb}{0.000000,0.000000,0.000000}%
\pgfsetstrokecolor{currentstroke}%
\pgfsetdash{}{0pt}%
\pgfsys@defobject{currentmarker}{\pgfqpoint{0.000000in}{-0.048611in}}{\pgfqpoint{0.000000in}{0.000000in}}{%
\pgfpathmoveto{\pgfqpoint{0.000000in}{0.000000in}}%
\pgfpathlineto{\pgfqpoint{0.000000in}{-0.048611in}}%
\pgfusepath{stroke,fill}%
}%
\begin{pgfscope}%
\pgfsys@transformshift{4.271875in}{0.385000in}%
\pgfsys@useobject{currentmarker}{}%
\end{pgfscope}%
\end{pgfscope}%
\begin{pgfscope}%
\definecolor{textcolor}{rgb}{0.000000,0.000000,0.000000}%
\pgfsetstrokecolor{textcolor}%
\pgfsetfillcolor{textcolor}%
\pgftext[x=4.271875in,y=0.287778in,,top]{\color{textcolor}\rmfamily\fontsize{10.000000}{12.000000}\selectfont \(\displaystyle 0.75\)}%
\end{pgfscope}%
\begin{pgfscope}%
\pgfsetbuttcap%
\pgfsetroundjoin%
\definecolor{currentfill}{rgb}{0.000000,0.000000,0.000000}%
\pgfsetfillcolor{currentfill}%
\pgfsetlinewidth{0.803000pt}%
\definecolor{currentstroke}{rgb}{0.000000,0.000000,0.000000}%
\pgfsetstrokecolor{currentstroke}%
\pgfsetdash{}{0pt}%
\pgfsys@defobject{currentmarker}{\pgfqpoint{0.000000in}{-0.048611in}}{\pgfqpoint{0.000000in}{0.000000in}}{%
\pgfpathmoveto{\pgfqpoint{0.000000in}{0.000000in}}%
\pgfpathlineto{\pgfqpoint{0.000000in}{-0.048611in}}%
\pgfusepath{stroke,fill}%
}%
\begin{pgfscope}%
\pgfsys@transformshift{4.756250in}{0.385000in}%
\pgfsys@useobject{currentmarker}{}%
\end{pgfscope}%
\end{pgfscope}%
\begin{pgfscope}%
\definecolor{textcolor}{rgb}{0.000000,0.000000,0.000000}%
\pgfsetstrokecolor{textcolor}%
\pgfsetfillcolor{textcolor}%
\pgftext[x=4.756250in,y=0.287778in,,top]{\color{textcolor}\rmfamily\fontsize{10.000000}{12.000000}\selectfont \(\displaystyle 1.00\)}%
\end{pgfscope}%
\begin{pgfscope}%
\definecolor{textcolor}{rgb}{0.000000,0.000000,0.000000}%
\pgfsetstrokecolor{textcolor}%
\pgfsetfillcolor{textcolor}%
\pgftext[x=2.818750in,y=0.108766in,,top]{\color{textcolor}\rmfamily\fontsize{10.000000}{12.000000}\selectfont x}%
\end{pgfscope}%
\begin{pgfscope}%
\pgfsetbuttcap%
\pgfsetroundjoin%
\definecolor{currentfill}{rgb}{0.000000,0.000000,0.000000}%
\pgfsetfillcolor{currentfill}%
\pgfsetlinewidth{0.803000pt}%
\definecolor{currentstroke}{rgb}{0.000000,0.000000,0.000000}%
\pgfsetstrokecolor{currentstroke}%
\pgfsetdash{}{0pt}%
\pgfsys@defobject{currentmarker}{\pgfqpoint{-0.048611in}{0.000000in}}{\pgfqpoint{0.000000in}{0.000000in}}{%
\pgfpathmoveto{\pgfqpoint{0.000000in}{0.000000in}}%
\pgfpathlineto{\pgfqpoint{-0.048611in}{0.000000in}}%
\pgfusepath{stroke,fill}%
}%
\begin{pgfscope}%
\pgfsys@transformshift{0.687500in}{0.474833in}%
\pgfsys@useobject{currentmarker}{}%
\end{pgfscope}%
\end{pgfscope}%
\begin{pgfscope}%
\definecolor{textcolor}{rgb}{0.000000,0.000000,0.000000}%
\pgfsetstrokecolor{textcolor}%
\pgfsetfillcolor{textcolor}%
\pgftext[x=0.304783in,y=0.426608in,left,base]{\color{textcolor}\rmfamily\fontsize{10.000000}{12.000000}\selectfont \(\displaystyle -0.2\)}%
\end{pgfscope}%
\begin{pgfscope}%
\pgfsetbuttcap%
\pgfsetroundjoin%
\definecolor{currentfill}{rgb}{0.000000,0.000000,0.000000}%
\pgfsetfillcolor{currentfill}%
\pgfsetlinewidth{0.803000pt}%
\definecolor{currentstroke}{rgb}{0.000000,0.000000,0.000000}%
\pgfsetstrokecolor{currentstroke}%
\pgfsetdash{}{0pt}%
\pgfsys@defobject{currentmarker}{\pgfqpoint{-0.048611in}{0.000000in}}{\pgfqpoint{0.000000in}{0.000000in}}{%
\pgfpathmoveto{\pgfqpoint{0.000000in}{0.000000in}}%
\pgfpathlineto{\pgfqpoint{-0.048611in}{0.000000in}}%
\pgfusepath{stroke,fill}%
}%
\begin{pgfscope}%
\pgfsys@transformshift{0.687500in}{0.834167in}%
\pgfsys@useobject{currentmarker}{}%
\end{pgfscope}%
\end{pgfscope}%
\begin{pgfscope}%
\definecolor{textcolor}{rgb}{0.000000,0.000000,0.000000}%
\pgfsetstrokecolor{textcolor}%
\pgfsetfillcolor{textcolor}%
\pgftext[x=0.412808in,y=0.785941in,left,base]{\color{textcolor}\rmfamily\fontsize{10.000000}{12.000000}\selectfont \(\displaystyle 0.0\)}%
\end{pgfscope}%
\begin{pgfscope}%
\pgfsetbuttcap%
\pgfsetroundjoin%
\definecolor{currentfill}{rgb}{0.000000,0.000000,0.000000}%
\pgfsetfillcolor{currentfill}%
\pgfsetlinewidth{0.803000pt}%
\definecolor{currentstroke}{rgb}{0.000000,0.000000,0.000000}%
\pgfsetstrokecolor{currentstroke}%
\pgfsetdash{}{0pt}%
\pgfsys@defobject{currentmarker}{\pgfqpoint{-0.048611in}{0.000000in}}{\pgfqpoint{0.000000in}{0.000000in}}{%
\pgfpathmoveto{\pgfqpoint{0.000000in}{0.000000in}}%
\pgfpathlineto{\pgfqpoint{-0.048611in}{0.000000in}}%
\pgfusepath{stroke,fill}%
}%
\begin{pgfscope}%
\pgfsys@transformshift{0.687500in}{1.193500in}%
\pgfsys@useobject{currentmarker}{}%
\end{pgfscope}%
\end{pgfscope}%
\begin{pgfscope}%
\definecolor{textcolor}{rgb}{0.000000,0.000000,0.000000}%
\pgfsetstrokecolor{textcolor}%
\pgfsetfillcolor{textcolor}%
\pgftext[x=0.412808in,y=1.145275in,left,base]{\color{textcolor}\rmfamily\fontsize{10.000000}{12.000000}\selectfont \(\displaystyle 0.2\)}%
\end{pgfscope}%
\begin{pgfscope}%
\pgfsetbuttcap%
\pgfsetroundjoin%
\definecolor{currentfill}{rgb}{0.000000,0.000000,0.000000}%
\pgfsetfillcolor{currentfill}%
\pgfsetlinewidth{0.803000pt}%
\definecolor{currentstroke}{rgb}{0.000000,0.000000,0.000000}%
\pgfsetstrokecolor{currentstroke}%
\pgfsetdash{}{0pt}%
\pgfsys@defobject{currentmarker}{\pgfqpoint{-0.048611in}{0.000000in}}{\pgfqpoint{0.000000in}{0.000000in}}{%
\pgfpathmoveto{\pgfqpoint{0.000000in}{0.000000in}}%
\pgfpathlineto{\pgfqpoint{-0.048611in}{0.000000in}}%
\pgfusepath{stroke,fill}%
}%
\begin{pgfscope}%
\pgfsys@transformshift{0.687500in}{1.552833in}%
\pgfsys@useobject{currentmarker}{}%
\end{pgfscope}%
\end{pgfscope}%
\begin{pgfscope}%
\definecolor{textcolor}{rgb}{0.000000,0.000000,0.000000}%
\pgfsetstrokecolor{textcolor}%
\pgfsetfillcolor{textcolor}%
\pgftext[x=0.412808in,y=1.504608in,left,base]{\color{textcolor}\rmfamily\fontsize{10.000000}{12.000000}\selectfont \(\displaystyle 0.4\)}%
\end{pgfscope}%
\begin{pgfscope}%
\pgfsetbuttcap%
\pgfsetroundjoin%
\definecolor{currentfill}{rgb}{0.000000,0.000000,0.000000}%
\pgfsetfillcolor{currentfill}%
\pgfsetlinewidth{0.803000pt}%
\definecolor{currentstroke}{rgb}{0.000000,0.000000,0.000000}%
\pgfsetstrokecolor{currentstroke}%
\pgfsetdash{}{0pt}%
\pgfsys@defobject{currentmarker}{\pgfqpoint{-0.048611in}{0.000000in}}{\pgfqpoint{0.000000in}{0.000000in}}{%
\pgfpathmoveto{\pgfqpoint{0.000000in}{0.000000in}}%
\pgfpathlineto{\pgfqpoint{-0.048611in}{0.000000in}}%
\pgfusepath{stroke,fill}%
}%
\begin{pgfscope}%
\pgfsys@transformshift{0.687500in}{1.912167in}%
\pgfsys@useobject{currentmarker}{}%
\end{pgfscope}%
\end{pgfscope}%
\begin{pgfscope}%
\definecolor{textcolor}{rgb}{0.000000,0.000000,0.000000}%
\pgfsetstrokecolor{textcolor}%
\pgfsetfillcolor{textcolor}%
\pgftext[x=0.412808in,y=1.863941in,left,base]{\color{textcolor}\rmfamily\fontsize{10.000000}{12.000000}\selectfont \(\displaystyle 0.6\)}%
\end{pgfscope}%
\begin{pgfscope}%
\pgfsetbuttcap%
\pgfsetroundjoin%
\definecolor{currentfill}{rgb}{0.000000,0.000000,0.000000}%
\pgfsetfillcolor{currentfill}%
\pgfsetlinewidth{0.803000pt}%
\definecolor{currentstroke}{rgb}{0.000000,0.000000,0.000000}%
\pgfsetstrokecolor{currentstroke}%
\pgfsetdash{}{0pt}%
\pgfsys@defobject{currentmarker}{\pgfqpoint{-0.048611in}{0.000000in}}{\pgfqpoint{0.000000in}{0.000000in}}{%
\pgfpathmoveto{\pgfqpoint{0.000000in}{0.000000in}}%
\pgfpathlineto{\pgfqpoint{-0.048611in}{0.000000in}}%
\pgfusepath{stroke,fill}%
}%
\begin{pgfscope}%
\pgfsys@transformshift{0.687500in}{2.271500in}%
\pgfsys@useobject{currentmarker}{}%
\end{pgfscope}%
\end{pgfscope}%
\begin{pgfscope}%
\definecolor{textcolor}{rgb}{0.000000,0.000000,0.000000}%
\pgfsetstrokecolor{textcolor}%
\pgfsetfillcolor{textcolor}%
\pgftext[x=0.412808in,y=2.223275in,left,base]{\color{textcolor}\rmfamily\fontsize{10.000000}{12.000000}\selectfont \(\displaystyle 0.8\)}%
\end{pgfscope}%
\begin{pgfscope}%
\pgfsetbuttcap%
\pgfsetroundjoin%
\definecolor{currentfill}{rgb}{0.000000,0.000000,0.000000}%
\pgfsetfillcolor{currentfill}%
\pgfsetlinewidth{0.803000pt}%
\definecolor{currentstroke}{rgb}{0.000000,0.000000,0.000000}%
\pgfsetstrokecolor{currentstroke}%
\pgfsetdash{}{0pt}%
\pgfsys@defobject{currentmarker}{\pgfqpoint{-0.048611in}{0.000000in}}{\pgfqpoint{0.000000in}{0.000000in}}{%
\pgfpathmoveto{\pgfqpoint{0.000000in}{0.000000in}}%
\pgfpathlineto{\pgfqpoint{-0.048611in}{0.000000in}}%
\pgfusepath{stroke,fill}%
}%
\begin{pgfscope}%
\pgfsys@transformshift{0.687500in}{2.630833in}%
\pgfsys@useobject{currentmarker}{}%
\end{pgfscope}%
\end{pgfscope}%
\begin{pgfscope}%
\definecolor{textcolor}{rgb}{0.000000,0.000000,0.000000}%
\pgfsetstrokecolor{textcolor}%
\pgfsetfillcolor{textcolor}%
\pgftext[x=0.412808in,y=2.582608in,left,base]{\color{textcolor}\rmfamily\fontsize{10.000000}{12.000000}\selectfont \(\displaystyle 1.0\)}%
\end{pgfscope}%
\begin{pgfscope}%
\pgfsetbuttcap%
\pgfsetroundjoin%
\definecolor{currentfill}{rgb}{0.000000,0.000000,0.000000}%
\pgfsetfillcolor{currentfill}%
\pgfsetlinewidth{0.803000pt}%
\definecolor{currentstroke}{rgb}{0.000000,0.000000,0.000000}%
\pgfsetstrokecolor{currentstroke}%
\pgfsetdash{}{0pt}%
\pgfsys@defobject{currentmarker}{\pgfqpoint{-0.048611in}{0.000000in}}{\pgfqpoint{0.000000in}{0.000000in}}{%
\pgfpathmoveto{\pgfqpoint{0.000000in}{0.000000in}}%
\pgfpathlineto{\pgfqpoint{-0.048611in}{0.000000in}}%
\pgfusepath{stroke,fill}%
}%
\begin{pgfscope}%
\pgfsys@transformshift{0.687500in}{2.990167in}%
\pgfsys@useobject{currentmarker}{}%
\end{pgfscope}%
\end{pgfscope}%
\begin{pgfscope}%
\definecolor{textcolor}{rgb}{0.000000,0.000000,0.000000}%
\pgfsetstrokecolor{textcolor}%
\pgfsetfillcolor{textcolor}%
\pgftext[x=0.412808in,y=2.941941in,left,base]{\color{textcolor}\rmfamily\fontsize{10.000000}{12.000000}\selectfont \(\displaystyle 1.2\)}%
\end{pgfscope}%
\begin{pgfscope}%
\definecolor{textcolor}{rgb}{0.000000,0.000000,0.000000}%
\pgfsetstrokecolor{textcolor}%
\pgfsetfillcolor{textcolor}%
\pgftext[x=0.249228in,y=1.732500in,,bottom,rotate=90.000000]{\color{textcolor}\rmfamily\fontsize{10.000000}{12.000000}\selectfont y}%
\end{pgfscope}%
\begin{pgfscope}%
\pgfpathrectangle{\pgfqpoint{0.687500in}{0.385000in}}{\pgfqpoint{4.262500in}{2.695000in}}%
\pgfusepath{clip}%
\pgfsetrectcap%
\pgfsetroundjoin%
\pgfsetlinewidth{1.505625pt}%
\definecolor{currentstroke}{rgb}{0.121569,0.466667,0.705882}%
\pgfsetstrokecolor{currentstroke}%
\pgfsetdash{}{0pt}%
\pgfpathmoveto{\pgfqpoint{0.881250in}{1.732500in}}%
\pgfpathlineto{\pgfqpoint{0.978612in}{1.778775in}}%
\pgfpathlineto{\pgfqpoint{1.075974in}{1.827296in}}%
\pgfpathlineto{\pgfqpoint{1.173335in}{1.878000in}}%
\pgfpathlineto{\pgfqpoint{1.290170in}{1.941556in}}%
\pgfpathlineto{\pgfqpoint{1.407004in}{2.007752in}}%
\pgfpathlineto{\pgfqpoint{1.543310in}{2.087643in}}%
\pgfpathlineto{\pgfqpoint{1.796451in}{2.239567in}}%
\pgfpathlineto{\pgfqpoint{1.932758in}{2.320107in}}%
\pgfpathlineto{\pgfqpoint{2.030119in}{2.375473in}}%
\pgfpathlineto{\pgfqpoint{2.108009in}{2.417737in}}%
\pgfpathlineto{\pgfqpoint{2.185898in}{2.457628in}}%
\pgfpathlineto{\pgfqpoint{2.263788in}{2.494605in}}%
\pgfpathlineto{\pgfqpoint{2.322205in}{2.520101in}}%
\pgfpathlineto{\pgfqpoint{2.380622in}{2.543430in}}%
\pgfpathlineto{\pgfqpoint{2.439039in}{2.564379in}}%
\pgfpathlineto{\pgfqpoint{2.497456in}{2.582749in}}%
\pgfpathlineto{\pgfqpoint{2.555873in}{2.598357in}}%
\pgfpathlineto{\pgfqpoint{2.614290in}{2.611046in}}%
\pgfpathlineto{\pgfqpoint{2.672707in}{2.620683in}}%
\pgfpathlineto{\pgfqpoint{2.731124in}{2.627166in}}%
\pgfpathlineto{\pgfqpoint{2.789541in}{2.630425in}}%
\pgfpathlineto{\pgfqpoint{2.847959in}{2.630425in}}%
\pgfpathlineto{\pgfqpoint{2.906376in}{2.627166in}}%
\pgfpathlineto{\pgfqpoint{2.964793in}{2.620683in}}%
\pgfpathlineto{\pgfqpoint{3.023210in}{2.611046in}}%
\pgfpathlineto{\pgfqpoint{3.081627in}{2.598357in}}%
\pgfpathlineto{\pgfqpoint{3.140044in}{2.582749in}}%
\pgfpathlineto{\pgfqpoint{3.198461in}{2.564379in}}%
\pgfpathlineto{\pgfqpoint{3.256878in}{2.543430in}}%
\pgfpathlineto{\pgfqpoint{3.315295in}{2.520101in}}%
\pgfpathlineto{\pgfqpoint{3.373712in}{2.494605in}}%
\pgfpathlineto{\pgfqpoint{3.451602in}{2.457628in}}%
\pgfpathlineto{\pgfqpoint{3.529491in}{2.417737in}}%
\pgfpathlineto{\pgfqpoint{3.607381in}{2.375473in}}%
\pgfpathlineto{\pgfqpoint{3.704742in}{2.320107in}}%
\pgfpathlineto{\pgfqpoint{3.841049in}{2.239567in}}%
\pgfpathlineto{\pgfqpoint{4.211024in}{2.019009in}}%
\pgfpathlineto{\pgfqpoint{4.327858in}{1.952417in}}%
\pgfpathlineto{\pgfqpoint{4.444692in}{1.888394in}}%
\pgfpathlineto{\pgfqpoint{4.542054in}{1.837265in}}%
\pgfpathlineto{\pgfqpoint{4.639416in}{1.788301in}}%
\pgfpathlineto{\pgfqpoint{4.736778in}{1.741574in}}%
\pgfpathlineto{\pgfqpoint{4.756250in}{1.732500in}}%
\pgfpathlineto{\pgfqpoint{4.756250in}{1.732500in}}%
\pgfusepath{stroke}%
\end{pgfscope}%
\begin{pgfscope}%
\pgfpathrectangle{\pgfqpoint{0.687500in}{0.385000in}}{\pgfqpoint{4.262500in}{2.695000in}}%
\pgfusepath{clip}%
\pgfsetbuttcap%
\pgfsetroundjoin%
\definecolor{currentfill}{rgb}{1.000000,0.498039,0.054902}%
\pgfsetfillcolor{currentfill}%
\pgfsetlinewidth{1.003750pt}%
\definecolor{currentstroke}{rgb}{1.000000,0.498039,0.054902}%
\pgfsetstrokecolor{currentstroke}%
\pgfsetdash{}{0pt}%
\pgfsys@defobject{currentmarker}{\pgfqpoint{-0.020833in}{-0.020833in}}{\pgfqpoint{0.020833in}{0.020833in}}{%
\pgfpathmoveto{\pgfqpoint{0.000000in}{-0.020833in}}%
\pgfpathcurveto{\pgfqpoint{0.005525in}{-0.020833in}}{\pgfqpoint{0.010825in}{-0.018638in}}{\pgfqpoint{0.014731in}{-0.014731in}}%
\pgfpathcurveto{\pgfqpoint{0.018638in}{-0.010825in}}{\pgfqpoint{0.020833in}{-0.005525in}}{\pgfqpoint{0.020833in}{0.000000in}}%
\pgfpathcurveto{\pgfqpoint{0.020833in}{0.005525in}}{\pgfqpoint{0.018638in}{0.010825in}}{\pgfqpoint{0.014731in}{0.014731in}}%
\pgfpathcurveto{\pgfqpoint{0.010825in}{0.018638in}}{\pgfqpoint{0.005525in}{0.020833in}}{\pgfqpoint{0.000000in}{0.020833in}}%
\pgfpathcurveto{\pgfqpoint{-0.005525in}{0.020833in}}{\pgfqpoint{-0.010825in}{0.018638in}}{\pgfqpoint{-0.014731in}{0.014731in}}%
\pgfpathcurveto{\pgfqpoint{-0.018638in}{0.010825in}}{\pgfqpoint{-0.020833in}{0.005525in}}{\pgfqpoint{-0.020833in}{0.000000in}}%
\pgfpathcurveto{\pgfqpoint{-0.020833in}{-0.005525in}}{\pgfqpoint{-0.018638in}{-0.010825in}}{\pgfqpoint{-0.014731in}{-0.014731in}}%
\pgfpathcurveto{\pgfqpoint{-0.010825in}{-0.018638in}}{\pgfqpoint{-0.005525in}{-0.020833in}}{\pgfqpoint{0.000000in}{-0.020833in}}%
\pgfpathclose%
\pgfusepath{stroke,fill}%
}%
\begin{pgfscope}%
\pgfsys@transformshift{4.707673in}{1.755305in}%
\pgfsys@useobject{currentmarker}{}%
\end{pgfscope}%
\begin{pgfscope}%
\pgfsys@transformshift{4.333548in}{1.949236in}%
\pgfsys@useobject{currentmarker}{}%
\end{pgfscope}%
\begin{pgfscope}%
\pgfsys@transformshift{3.659400in}{2.346188in}%
\pgfsys@useobject{currentmarker}{}%
\end{pgfscope}%
\begin{pgfscope}%
\pgfsys@transformshift{2.818750in}{2.630833in}%
\pgfsys@useobject{currentmarker}{}%
\end{pgfscope}%
\begin{pgfscope}%
\pgfsys@transformshift{1.978100in}{2.346188in}%
\pgfsys@useobject{currentmarker}{}%
\end{pgfscope}%
\begin{pgfscope}%
\pgfsys@transformshift{1.303952in}{1.949236in}%
\pgfsys@useobject{currentmarker}{}%
\end{pgfscope}%
\begin{pgfscope}%
\pgfsys@transformshift{0.929827in}{1.755305in}%
\pgfsys@useobject{currentmarker}{}%
\end{pgfscope}%
\end{pgfscope}%
\begin{pgfscope}%
\pgfpathrectangle{\pgfqpoint{0.687500in}{0.385000in}}{\pgfqpoint{4.262500in}{2.695000in}}%
\pgfusepath{clip}%
\pgfsetrectcap%
\pgfsetroundjoin%
\pgfsetlinewidth{1.505625pt}%
\definecolor{currentstroke}{rgb}{0.172549,0.627451,0.172549}%
\pgfsetstrokecolor{currentstroke}%
\pgfsetdash{}{0pt}%
\pgfpathmoveto{\pgfqpoint{0.881250in}{1.728741in}}%
\pgfpathlineto{\pgfqpoint{0.959139in}{1.770873in}}%
\pgfpathlineto{\pgfqpoint{1.075974in}{1.831033in}}%
\pgfpathlineto{\pgfqpoint{1.251225in}{1.921195in}}%
\pgfpathlineto{\pgfqpoint{1.368059in}{1.984150in}}%
\pgfpathlineto{\pgfqpoint{1.465421in}{2.038994in}}%
\pgfpathlineto{\pgfqpoint{1.582255in}{2.107475in}}%
\pgfpathlineto{\pgfqpoint{1.738034in}{2.201796in}}%
\pgfpathlineto{\pgfqpoint{1.932758in}{2.319590in}}%
\pgfpathlineto{\pgfqpoint{2.030119in}{2.376013in}}%
\pgfpathlineto{\pgfqpoint{2.108009in}{2.418928in}}%
\pgfpathlineto{\pgfqpoint{2.185898in}{2.459238in}}%
\pgfpathlineto{\pgfqpoint{2.244315in}{2.487425in}}%
\pgfpathlineto{\pgfqpoint{2.302732in}{2.513609in}}%
\pgfpathlineto{\pgfqpoint{2.361149in}{2.537581in}}%
\pgfpathlineto{\pgfqpoint{2.419567in}{2.559149in}}%
\pgfpathlineto{\pgfqpoint{2.477984in}{2.578136in}}%
\pgfpathlineto{\pgfqpoint{2.536401in}{2.594387in}}%
\pgfpathlineto{\pgfqpoint{2.594818in}{2.607771in}}%
\pgfpathlineto{\pgfqpoint{2.653235in}{2.618176in}}%
\pgfpathlineto{\pgfqpoint{2.711652in}{2.625517in}}%
\pgfpathlineto{\pgfqpoint{2.770069in}{2.629733in}}%
\pgfpathlineto{\pgfqpoint{2.828486in}{2.630789in}}%
\pgfpathlineto{\pgfqpoint{2.886903in}{2.628677in}}%
\pgfpathlineto{\pgfqpoint{2.945320in}{2.623415in}}%
\pgfpathlineto{\pgfqpoint{3.003737in}{2.615044in}}%
\pgfpathlineto{\pgfqpoint{3.062155in}{2.603636in}}%
\pgfpathlineto{\pgfqpoint{3.120572in}{2.589283in}}%
\pgfpathlineto{\pgfqpoint{3.178989in}{2.572103in}}%
\pgfpathlineto{\pgfqpoint{3.237406in}{2.552238in}}%
\pgfpathlineto{\pgfqpoint{3.295823in}{2.529849in}}%
\pgfpathlineto{\pgfqpoint{3.354240in}{2.505117in}}%
\pgfpathlineto{\pgfqpoint{3.412657in}{2.478241in}}%
\pgfpathlineto{\pgfqpoint{3.490546in}{2.439444in}}%
\pgfpathlineto{\pgfqpoint{3.568436in}{2.397760in}}%
\pgfpathlineto{\pgfqpoint{3.665798in}{2.342465in}}%
\pgfpathlineto{\pgfqpoint{3.782632in}{2.272942in}}%
\pgfpathlineto{\pgfqpoint{4.133134in}{2.061525in}}%
\pgfpathlineto{\pgfqpoint{4.249969in}{1.994943in}}%
\pgfpathlineto{\pgfqpoint{4.347330in}{1.941850in}}%
\pgfpathlineto{\pgfqpoint{4.464165in}{1.880709in}}%
\pgfpathlineto{\pgfqpoint{4.717305in}{1.750121in}}%
\pgfpathlineto{\pgfqpoint{4.756250in}{1.728741in}}%
\pgfpathlineto{\pgfqpoint{4.756250in}{1.728741in}}%
\pgfusepath{stroke}%
\end{pgfscope}%
\begin{pgfscope}%
\pgfpathrectangle{\pgfqpoint{0.687500in}{0.385000in}}{\pgfqpoint{4.262500in}{2.695000in}}%
\pgfusepath{clip}%
\pgfsetbuttcap%
\pgfsetroundjoin%
\definecolor{currentfill}{rgb}{0.839216,0.152941,0.156863}%
\pgfsetfillcolor{currentfill}%
\pgfsetlinewidth{1.003750pt}%
\definecolor{currentstroke}{rgb}{0.839216,0.152941,0.156863}%
\pgfsetstrokecolor{currentstroke}%
\pgfsetdash{}{0pt}%
\pgfsys@defobject{currentmarker}{\pgfqpoint{-0.020833in}{-0.020833in}}{\pgfqpoint{0.020833in}{0.020833in}}{%
\pgfpathmoveto{\pgfqpoint{0.000000in}{-0.020833in}}%
\pgfpathcurveto{\pgfqpoint{0.005525in}{-0.020833in}}{\pgfqpoint{0.010825in}{-0.018638in}}{\pgfqpoint{0.014731in}{-0.014731in}}%
\pgfpathcurveto{\pgfqpoint{0.018638in}{-0.010825in}}{\pgfqpoint{0.020833in}{-0.005525in}}{\pgfqpoint{0.020833in}{0.000000in}}%
\pgfpathcurveto{\pgfqpoint{0.020833in}{0.005525in}}{\pgfqpoint{0.018638in}{0.010825in}}{\pgfqpoint{0.014731in}{0.014731in}}%
\pgfpathcurveto{\pgfqpoint{0.010825in}{0.018638in}}{\pgfqpoint{0.005525in}{0.020833in}}{\pgfqpoint{0.000000in}{0.020833in}}%
\pgfpathcurveto{\pgfqpoint{-0.005525in}{0.020833in}}{\pgfqpoint{-0.010825in}{0.018638in}}{\pgfqpoint{-0.014731in}{0.014731in}}%
\pgfpathcurveto{\pgfqpoint{-0.018638in}{0.010825in}}{\pgfqpoint{-0.020833in}{0.005525in}}{\pgfqpoint{-0.020833in}{0.000000in}}%
\pgfpathcurveto{\pgfqpoint{-0.020833in}{-0.005525in}}{\pgfqpoint{-0.018638in}{-0.010825in}}{\pgfqpoint{-0.014731in}{-0.014731in}}%
\pgfpathcurveto{\pgfqpoint{-0.010825in}{-0.018638in}}{\pgfqpoint{-0.005525in}{-0.020833in}}{\pgfqpoint{0.000000in}{-0.020833in}}%
\pgfpathclose%
\pgfusepath{stroke,fill}%
}%
\begin{pgfscope}%
\pgfsys@transformshift{4.726815in}{1.746251in}%
\pgfsys@useobject{currentmarker}{}%
\end{pgfscope}%
\begin{pgfscope}%
\pgfsys@transformshift{4.496674in}{1.860833in}%
\pgfsys@useobject{currentmarker}{}%
\end{pgfscope}%
\begin{pgfscope}%
\pgfsys@transformshift{4.064151in}{2.105535in}%
\pgfsys@useobject{currentmarker}{}%
\end{pgfscope}%
\begin{pgfscope}%
\pgfsys@transformshift{3.481414in}{2.442674in}%
\pgfsys@useobject{currentmarker}{}%
\end{pgfscope}%
\begin{pgfscope}%
\pgfsys@transformshift{2.818750in}{2.630833in}%
\pgfsys@useobject{currentmarker}{}%
\end{pgfscope}%
\begin{pgfscope}%
\pgfsys@transformshift{2.156086in}{2.442674in}%
\pgfsys@useobject{currentmarker}{}%
\end{pgfscope}%
\begin{pgfscope}%
\pgfsys@transformshift{1.573349in}{2.105535in}%
\pgfsys@useobject{currentmarker}{}%
\end{pgfscope}%
\begin{pgfscope}%
\pgfsys@transformshift{1.140826in}{1.860833in}%
\pgfsys@useobject{currentmarker}{}%
\end{pgfscope}%
\begin{pgfscope}%
\pgfsys@transformshift{0.910685in}{1.746251in}%
\pgfsys@useobject{currentmarker}{}%
\end{pgfscope}%
\end{pgfscope}%
\begin{pgfscope}%
\pgfpathrectangle{\pgfqpoint{0.687500in}{0.385000in}}{\pgfqpoint{4.262500in}{2.695000in}}%
\pgfusepath{clip}%
\pgfsetrectcap%
\pgfsetroundjoin%
\pgfsetlinewidth{1.505625pt}%
\definecolor{currentstroke}{rgb}{0.580392,0.403922,0.741176}%
\pgfsetstrokecolor{currentstroke}%
\pgfsetdash{}{0pt}%
\pgfpathmoveto{\pgfqpoint{0.881250in}{1.733145in}}%
\pgfpathlineto{\pgfqpoint{0.959139in}{1.768801in}}%
\pgfpathlineto{\pgfqpoint{1.037029in}{1.807055in}}%
\pgfpathlineto{\pgfqpoint{1.134391in}{1.857425in}}%
\pgfpathlineto{\pgfqpoint{1.251225in}{1.920568in}}%
\pgfpathlineto{\pgfqpoint{1.387531in}{1.997129in}}%
\pgfpathlineto{\pgfqpoint{1.523838in}{2.076269in}}%
\pgfpathlineto{\pgfqpoint{1.718562in}{2.192350in}}%
\pgfpathlineto{\pgfqpoint{1.932758in}{2.319665in}}%
\pgfpathlineto{\pgfqpoint{2.030119in}{2.375208in}}%
\pgfpathlineto{\pgfqpoint{2.108009in}{2.417638in}}%
\pgfpathlineto{\pgfqpoint{2.185898in}{2.457683in}}%
\pgfpathlineto{\pgfqpoint{2.263788in}{2.494777in}}%
\pgfpathlineto{\pgfqpoint{2.322205in}{2.520325in}}%
\pgfpathlineto{\pgfqpoint{2.380622in}{2.543676in}}%
\pgfpathlineto{\pgfqpoint{2.439039in}{2.564617in}}%
\pgfpathlineto{\pgfqpoint{2.497456in}{2.582956in}}%
\pgfpathlineto{\pgfqpoint{2.555873in}{2.598519in}}%
\pgfpathlineto{\pgfqpoint{2.614290in}{2.611155in}}%
\pgfpathlineto{\pgfqpoint{2.672707in}{2.620744in}}%
\pgfpathlineto{\pgfqpoint{2.731124in}{2.627189in}}%
\pgfpathlineto{\pgfqpoint{2.789541in}{2.630428in}}%
\pgfpathlineto{\pgfqpoint{2.847959in}{2.630428in}}%
\pgfpathlineto{\pgfqpoint{2.906376in}{2.627189in}}%
\pgfpathlineto{\pgfqpoint{2.964793in}{2.620744in}}%
\pgfpathlineto{\pgfqpoint{3.023210in}{2.611155in}}%
\pgfpathlineto{\pgfqpoint{3.081627in}{2.598519in}}%
\pgfpathlineto{\pgfqpoint{3.140044in}{2.582956in}}%
\pgfpathlineto{\pgfqpoint{3.198461in}{2.564617in}}%
\pgfpathlineto{\pgfqpoint{3.256878in}{2.543676in}}%
\pgfpathlineto{\pgfqpoint{3.315295in}{2.520325in}}%
\pgfpathlineto{\pgfqpoint{3.373712in}{2.494777in}}%
\pgfpathlineto{\pgfqpoint{3.432129in}{2.467255in}}%
\pgfpathlineto{\pgfqpoint{3.510019in}{2.427895in}}%
\pgfpathlineto{\pgfqpoint{3.587908in}{2.386008in}}%
\pgfpathlineto{\pgfqpoint{3.685270in}{2.330951in}}%
\pgfpathlineto{\pgfqpoint{3.821577in}{2.250693in}}%
\pgfpathlineto{\pgfqpoint{4.191552in}{2.030761in}}%
\pgfpathlineto{\pgfqpoint{4.327858in}{1.953032in}}%
\pgfpathlineto{\pgfqpoint{4.444692in}{1.888677in}}%
\pgfpathlineto{\pgfqpoint{4.561526in}{1.826912in}}%
\pgfpathlineto{\pgfqpoint{4.658888in}{1.778155in}}%
\pgfpathlineto{\pgfqpoint{4.736778in}{1.741761in}}%
\pgfpathlineto{\pgfqpoint{4.756250in}{1.733145in}}%
\pgfpathlineto{\pgfqpoint{4.756250in}{1.733145in}}%
\pgfusepath{stroke}%
\end{pgfscope}%
\begin{pgfscope}%
\pgfpathrectangle{\pgfqpoint{0.687500in}{0.385000in}}{\pgfqpoint{4.262500in}{2.695000in}}%
\pgfusepath{clip}%
\pgfsetbuttcap%
\pgfsetroundjoin%
\definecolor{currentfill}{rgb}{0.549020,0.337255,0.294118}%
\pgfsetfillcolor{currentfill}%
\pgfsetlinewidth{1.003750pt}%
\definecolor{currentstroke}{rgb}{0.549020,0.337255,0.294118}%
\pgfsetstrokecolor{currentstroke}%
\pgfsetdash{}{0pt}%
\pgfsys@defobject{currentmarker}{\pgfqpoint{-0.020833in}{-0.020833in}}{\pgfqpoint{0.020833in}{0.020833in}}{%
\pgfpathmoveto{\pgfqpoint{0.000000in}{-0.020833in}}%
\pgfpathcurveto{\pgfqpoint{0.005525in}{-0.020833in}}{\pgfqpoint{0.010825in}{-0.018638in}}{\pgfqpoint{0.014731in}{-0.014731in}}%
\pgfpathcurveto{\pgfqpoint{0.018638in}{-0.010825in}}{\pgfqpoint{0.020833in}{-0.005525in}}{\pgfqpoint{0.020833in}{0.000000in}}%
\pgfpathcurveto{\pgfqpoint{0.020833in}{0.005525in}}{\pgfqpoint{0.018638in}{0.010825in}}{\pgfqpoint{0.014731in}{0.014731in}}%
\pgfpathcurveto{\pgfqpoint{0.010825in}{0.018638in}}{\pgfqpoint{0.005525in}{0.020833in}}{\pgfqpoint{0.000000in}{0.020833in}}%
\pgfpathcurveto{\pgfqpoint{-0.005525in}{0.020833in}}{\pgfqpoint{-0.010825in}{0.018638in}}{\pgfqpoint{-0.014731in}{0.014731in}}%
\pgfpathcurveto{\pgfqpoint{-0.018638in}{0.010825in}}{\pgfqpoint{-0.020833in}{0.005525in}}{\pgfqpoint{-0.020833in}{0.000000in}}%
\pgfpathcurveto{\pgfqpoint{-0.020833in}{-0.005525in}}{\pgfqpoint{-0.018638in}{-0.010825in}}{\pgfqpoint{-0.014731in}{-0.014731in}}%
\pgfpathcurveto{\pgfqpoint{-0.010825in}{-0.018638in}}{\pgfqpoint{-0.005525in}{-0.020833in}}{\pgfqpoint{0.000000in}{-0.020833in}}%
\pgfpathclose%
\pgfusepath{stroke,fill}%
}%
\begin{pgfscope}%
\pgfsys@transformshift{4.747985in}{1.736340in}%
\pgfsys@useobject{currentmarker}{}%
\end{pgfscope}%
\begin{pgfscope}%
\pgfsys@transformshift{4.682287in}{1.767447in}%
\pgfsys@useobject{currentmarker}{}%
\end{pgfscope}%
\begin{pgfscope}%
\pgfsys@transformshift{4.553129in}{1.831585in}%
\pgfsys@useobject{currentmarker}{}%
\end{pgfscope}%
\begin{pgfscope}%
\pgfsys@transformshift{4.364908in}{1.931816in}%
\pgfsys@useobject{currentmarker}{}%
\end{pgfscope}%
\begin{pgfscope}%
\pgfsys@transformshift{4.124035in}{2.069952in}%
\pgfsys@useobject{currentmarker}{}%
\end{pgfscope}%
\begin{pgfscope}%
\pgfsys@transformshift{3.838712in}{2.240966in}%
\pgfsys@useobject{currentmarker}{}%
\end{pgfscope}%
\begin{pgfscope}%
\pgfsys@transformshift{3.518656in}{2.423440in}%
\pgfsys@useobject{currentmarker}{}%
\end{pgfscope}%
\begin{pgfscope}%
\pgfsys@transformshift{3.174765in}{2.572152in}%
\pgfsys@useobject{currentmarker}{}%
\end{pgfscope}%
\begin{pgfscope}%
\pgfsys@transformshift{2.818750in}{2.630833in}%
\pgfsys@useobject{currentmarker}{}%
\end{pgfscope}%
\begin{pgfscope}%
\pgfsys@transformshift{2.462735in}{2.572152in}%
\pgfsys@useobject{currentmarker}{}%
\end{pgfscope}%
\begin{pgfscope}%
\pgfsys@transformshift{2.118844in}{2.423440in}%
\pgfsys@useobject{currentmarker}{}%
\end{pgfscope}%
\begin{pgfscope}%
\pgfsys@transformshift{1.798788in}{2.240966in}%
\pgfsys@useobject{currentmarker}{}%
\end{pgfscope}%
\begin{pgfscope}%
\pgfsys@transformshift{1.513465in}{2.069952in}%
\pgfsys@useobject{currentmarker}{}%
\end{pgfscope}%
\begin{pgfscope}%
\pgfsys@transformshift{1.272592in}{1.931816in}%
\pgfsys@useobject{currentmarker}{}%
\end{pgfscope}%
\begin{pgfscope}%
\pgfsys@transformshift{1.084371in}{1.831585in}%
\pgfsys@useobject{currentmarker}{}%
\end{pgfscope}%
\begin{pgfscope}%
\pgfsys@transformshift{0.955213in}{1.767447in}%
\pgfsys@useobject{currentmarker}{}%
\end{pgfscope}%
\begin{pgfscope}%
\pgfsys@transformshift{0.889515in}{1.736340in}%
\pgfsys@useobject{currentmarker}{}%
\end{pgfscope}%
\end{pgfscope}%
\begin{pgfscope}%
\pgfpathrectangle{\pgfqpoint{0.687500in}{0.385000in}}{\pgfqpoint{4.262500in}{2.695000in}}%
\pgfusepath{clip}%
\pgfsetrectcap%
\pgfsetroundjoin%
\pgfsetlinewidth{1.505625pt}%
\definecolor{currentstroke}{rgb}{0.890196,0.466667,0.760784}%
\pgfsetstrokecolor{currentstroke}%
\pgfsetdash{}{0pt}%
\pgfpathmoveto{\pgfqpoint{0.881250in}{1.732501in}}%
\pgfpathlineto{\pgfqpoint{0.978612in}{1.778775in}}%
\pgfpathlineto{\pgfqpoint{1.075974in}{1.827296in}}%
\pgfpathlineto{\pgfqpoint{1.173335in}{1.878000in}}%
\pgfpathlineto{\pgfqpoint{1.290170in}{1.941557in}}%
\pgfpathlineto{\pgfqpoint{1.407004in}{2.007753in}}%
\pgfpathlineto{\pgfqpoint{1.543310in}{2.087642in}}%
\pgfpathlineto{\pgfqpoint{1.796451in}{2.239567in}}%
\pgfpathlineto{\pgfqpoint{1.932758in}{2.320107in}}%
\pgfpathlineto{\pgfqpoint{2.030119in}{2.375474in}}%
\pgfpathlineto{\pgfqpoint{2.108009in}{2.417737in}}%
\pgfpathlineto{\pgfqpoint{2.185898in}{2.457627in}}%
\pgfpathlineto{\pgfqpoint{2.263788in}{2.494605in}}%
\pgfpathlineto{\pgfqpoint{2.322205in}{2.520101in}}%
\pgfpathlineto{\pgfqpoint{2.380622in}{2.543430in}}%
\pgfpathlineto{\pgfqpoint{2.439039in}{2.564379in}}%
\pgfpathlineto{\pgfqpoint{2.497456in}{2.582749in}}%
\pgfpathlineto{\pgfqpoint{2.555873in}{2.598357in}}%
\pgfpathlineto{\pgfqpoint{2.614290in}{2.611046in}}%
\pgfpathlineto{\pgfqpoint{2.672707in}{2.620683in}}%
\pgfpathlineto{\pgfqpoint{2.731124in}{2.627166in}}%
\pgfpathlineto{\pgfqpoint{2.789541in}{2.630425in}}%
\pgfpathlineto{\pgfqpoint{2.847959in}{2.630425in}}%
\pgfpathlineto{\pgfqpoint{2.906376in}{2.627166in}}%
\pgfpathlineto{\pgfqpoint{2.964793in}{2.620683in}}%
\pgfpathlineto{\pgfqpoint{3.023210in}{2.611046in}}%
\pgfpathlineto{\pgfqpoint{3.081627in}{2.598357in}}%
\pgfpathlineto{\pgfqpoint{3.140044in}{2.582749in}}%
\pgfpathlineto{\pgfqpoint{3.198461in}{2.564379in}}%
\pgfpathlineto{\pgfqpoint{3.256878in}{2.543430in}}%
\pgfpathlineto{\pgfqpoint{3.315295in}{2.520101in}}%
\pgfpathlineto{\pgfqpoint{3.373712in}{2.494605in}}%
\pgfpathlineto{\pgfqpoint{3.451602in}{2.457627in}}%
\pgfpathlineto{\pgfqpoint{3.529491in}{2.417737in}}%
\pgfpathlineto{\pgfqpoint{3.607381in}{2.375474in}}%
\pgfpathlineto{\pgfqpoint{3.704742in}{2.320107in}}%
\pgfpathlineto{\pgfqpoint{3.841049in}{2.239567in}}%
\pgfpathlineto{\pgfqpoint{4.211024in}{2.019010in}}%
\pgfpathlineto{\pgfqpoint{4.327858in}{1.952417in}}%
\pgfpathlineto{\pgfqpoint{4.444692in}{1.888393in}}%
\pgfpathlineto{\pgfqpoint{4.542054in}{1.837265in}}%
\pgfpathlineto{\pgfqpoint{4.639416in}{1.788302in}}%
\pgfpathlineto{\pgfqpoint{4.736778in}{1.741573in}}%
\pgfpathlineto{\pgfqpoint{4.756250in}{1.732501in}}%
\pgfpathlineto{\pgfqpoint{4.756250in}{1.732501in}}%
\pgfusepath{stroke}%
\end{pgfscope}%
\begin{pgfscope}%
\pgfpathrectangle{\pgfqpoint{0.687500in}{0.385000in}}{\pgfqpoint{4.262500in}{2.695000in}}%
\pgfusepath{clip}%
\pgfsetbuttcap%
\pgfsetroundjoin%
\definecolor{currentfill}{rgb}{0.498039,0.498039,0.498039}%
\pgfsetfillcolor{currentfill}%
\pgfsetlinewidth{1.003750pt}%
\definecolor{currentstroke}{rgb}{0.498039,0.498039,0.498039}%
\pgfsetstrokecolor{currentstroke}%
\pgfsetdash{}{0pt}%
\pgfsys@defobject{currentmarker}{\pgfqpoint{-0.020833in}{-0.020833in}}{\pgfqpoint{0.020833in}{0.020833in}}{%
\pgfpathmoveto{\pgfqpoint{0.000000in}{-0.020833in}}%
\pgfpathcurveto{\pgfqpoint{0.005525in}{-0.020833in}}{\pgfqpoint{0.010825in}{-0.018638in}}{\pgfqpoint{0.014731in}{-0.014731in}}%
\pgfpathcurveto{\pgfqpoint{0.018638in}{-0.010825in}}{\pgfqpoint{0.020833in}{-0.005525in}}{\pgfqpoint{0.020833in}{0.000000in}}%
\pgfpathcurveto{\pgfqpoint{0.020833in}{0.005525in}}{\pgfqpoint{0.018638in}{0.010825in}}{\pgfqpoint{0.014731in}{0.014731in}}%
\pgfpathcurveto{\pgfqpoint{0.010825in}{0.018638in}}{\pgfqpoint{0.005525in}{0.020833in}}{\pgfqpoint{0.000000in}{0.020833in}}%
\pgfpathcurveto{\pgfqpoint{-0.005525in}{0.020833in}}{\pgfqpoint{-0.010825in}{0.018638in}}{\pgfqpoint{-0.014731in}{0.014731in}}%
\pgfpathcurveto{\pgfqpoint{-0.018638in}{0.010825in}}{\pgfqpoint{-0.020833in}{0.005525in}}{\pgfqpoint{-0.020833in}{0.000000in}}%
\pgfpathcurveto{\pgfqpoint{-0.020833in}{-0.005525in}}{\pgfqpoint{-0.018638in}{-0.010825in}}{\pgfqpoint{-0.014731in}{-0.014731in}}%
\pgfpathcurveto{\pgfqpoint{-0.010825in}{-0.018638in}}{\pgfqpoint{-0.005525in}{-0.020833in}}{\pgfqpoint{0.000000in}{-0.020833in}}%
\pgfpathclose%
\pgfusepath{stroke,fill}%
}%
\begin{pgfscope}%
\pgfsys@transformshift{4.754055in}{1.733518in}%
\pgfsys@useobject{currentmarker}{}%
\end{pgfscope}%
\begin{pgfscope}%
\pgfsys@transformshift{4.736529in}{1.741690in}%
\pgfsys@useobject{currentmarker}{}%
\end{pgfscope}%
\begin{pgfscope}%
\pgfsys@transformshift{4.701635in}{1.758179in}%
\pgfsys@useobject{currentmarker}{}%
\end{pgfscope}%
\begin{pgfscope}%
\pgfsys@transformshift{4.649689in}{1.783264in}%
\pgfsys@useobject{currentmarker}{}%
\end{pgfscope}%
\begin{pgfscope}%
\pgfsys@transformshift{4.581162in}{1.817332in}%
\pgfsys@useobject{currentmarker}{}%
\end{pgfscope}%
\begin{pgfscope}%
\pgfsys@transformshift{4.496674in}{1.860833in}%
\pgfsys@useobject{currentmarker}{}%
\end{pgfscope}%
\begin{pgfscope}%
\pgfsys@transformshift{4.396991in}{1.914197in}%
\pgfsys@useobject{currentmarker}{}%
\end{pgfscope}%
\begin{pgfscope}%
\pgfsys@transformshift{4.283015in}{1.977697in}%
\pgfsys@useobject{currentmarker}{}%
\end{pgfscope}%
\begin{pgfscope}%
\pgfsys@transformshift{4.155778in}{2.051248in}%
\pgfsys@useobject{currentmarker}{}%
\end{pgfscope}%
\begin{pgfscope}%
\pgfsys@transformshift{4.016433in}{2.134102in}%
\pgfsys@useobject{currentmarker}{}%
\end{pgfscope}%
\begin{pgfscope}%
\pgfsys@transformshift{3.866242in}{2.224461in}%
\pgfsys@useobject{currentmarker}{}%
\end{pgfscope}%
\begin{pgfscope}%
\pgfsys@transformshift{3.706564in}{2.319050in}%
\pgfsys@useobject{currentmarker}{}%
\end{pgfscope}%
\begin{pgfscope}%
\pgfsys@transformshift{3.538846in}{2.412775in}%
\pgfsys@useobject{currentmarker}{}%
\end{pgfscope}%
\begin{pgfscope}%
\pgfsys@transformshift{3.364607in}{2.498713in}%
\pgfsys@useobject{currentmarker}{}%
\end{pgfscope}%
\begin{pgfscope}%
\pgfsys@transformshift{3.185424in}{2.568709in}%
\pgfsys@useobject{currentmarker}{}%
\end{pgfscope}%
\begin{pgfscope}%
\pgfsys@transformshift{3.002921in}{2.614745in}%
\pgfsys@useobject{currentmarker}{}%
\end{pgfscope}%
\begin{pgfscope}%
\pgfsys@transformshift{2.818750in}{2.630833in}%
\pgfsys@useobject{currentmarker}{}%
\end{pgfscope}%
\begin{pgfscope}%
\pgfsys@transformshift{2.634579in}{2.614745in}%
\pgfsys@useobject{currentmarker}{}%
\end{pgfscope}%
\begin{pgfscope}%
\pgfsys@transformshift{2.452076in}{2.568709in}%
\pgfsys@useobject{currentmarker}{}%
\end{pgfscope}%
\begin{pgfscope}%
\pgfsys@transformshift{2.272893in}{2.498713in}%
\pgfsys@useobject{currentmarker}{}%
\end{pgfscope}%
\begin{pgfscope}%
\pgfsys@transformshift{2.098654in}{2.412775in}%
\pgfsys@useobject{currentmarker}{}%
\end{pgfscope}%
\begin{pgfscope}%
\pgfsys@transformshift{1.930936in}{2.319050in}%
\pgfsys@useobject{currentmarker}{}%
\end{pgfscope}%
\begin{pgfscope}%
\pgfsys@transformshift{1.771258in}{2.224461in}%
\pgfsys@useobject{currentmarker}{}%
\end{pgfscope}%
\begin{pgfscope}%
\pgfsys@transformshift{1.621067in}{2.134102in}%
\pgfsys@useobject{currentmarker}{}%
\end{pgfscope}%
\begin{pgfscope}%
\pgfsys@transformshift{1.481722in}{2.051248in}%
\pgfsys@useobject{currentmarker}{}%
\end{pgfscope}%
\begin{pgfscope}%
\pgfsys@transformshift{1.354485in}{1.977697in}%
\pgfsys@useobject{currentmarker}{}%
\end{pgfscope}%
\begin{pgfscope}%
\pgfsys@transformshift{1.240509in}{1.914197in}%
\pgfsys@useobject{currentmarker}{}%
\end{pgfscope}%
\begin{pgfscope}%
\pgfsys@transformshift{1.140826in}{1.860833in}%
\pgfsys@useobject{currentmarker}{}%
\end{pgfscope}%
\begin{pgfscope}%
\pgfsys@transformshift{1.056338in}{1.817332in}%
\pgfsys@useobject{currentmarker}{}%
\end{pgfscope}%
\begin{pgfscope}%
\pgfsys@transformshift{0.987811in}{1.783264in}%
\pgfsys@useobject{currentmarker}{}%
\end{pgfscope}%
\begin{pgfscope}%
\pgfsys@transformshift{0.935865in}{1.758179in}%
\pgfsys@useobject{currentmarker}{}%
\end{pgfscope}%
\begin{pgfscope}%
\pgfsys@transformshift{0.900971in}{1.741690in}%
\pgfsys@useobject{currentmarker}{}%
\end{pgfscope}%
\begin{pgfscope}%
\pgfsys@transformshift{0.883445in}{1.733518in}%
\pgfsys@useobject{currentmarker}{}%
\end{pgfscope}%
\end{pgfscope}%
\begin{pgfscope}%
\pgfpathrectangle{\pgfqpoint{0.687500in}{0.385000in}}{\pgfqpoint{4.262500in}{2.695000in}}%
\pgfusepath{clip}%
\pgfsetrectcap%
\pgfsetroundjoin%
\pgfsetlinewidth{1.505625pt}%
\definecolor{currentstroke}{rgb}{0.737255,0.741176,0.133333}%
\pgfsetstrokecolor{currentstroke}%
\pgfsetdash{}{0pt}%
\pgfpathmoveto{\pgfqpoint{0.881250in}{1.732500in}}%
\pgfpathlineto{\pgfqpoint{0.978612in}{1.778775in}}%
\pgfpathlineto{\pgfqpoint{1.075974in}{1.827296in}}%
\pgfpathlineto{\pgfqpoint{1.173335in}{1.878000in}}%
\pgfpathlineto{\pgfqpoint{1.290170in}{1.941556in}}%
\pgfpathlineto{\pgfqpoint{1.407004in}{2.007752in}}%
\pgfpathlineto{\pgfqpoint{1.543310in}{2.087643in}}%
\pgfpathlineto{\pgfqpoint{1.796451in}{2.239567in}}%
\pgfpathlineto{\pgfqpoint{1.932758in}{2.320107in}}%
\pgfpathlineto{\pgfqpoint{2.030119in}{2.375473in}}%
\pgfpathlineto{\pgfqpoint{2.108009in}{2.417737in}}%
\pgfpathlineto{\pgfqpoint{2.185898in}{2.457628in}}%
\pgfpathlineto{\pgfqpoint{2.263788in}{2.494605in}}%
\pgfpathlineto{\pgfqpoint{2.322205in}{2.520101in}}%
\pgfpathlineto{\pgfqpoint{2.380622in}{2.543430in}}%
\pgfpathlineto{\pgfqpoint{2.439039in}{2.564379in}}%
\pgfpathlineto{\pgfqpoint{2.497456in}{2.582749in}}%
\pgfpathlineto{\pgfqpoint{2.555873in}{2.598357in}}%
\pgfpathlineto{\pgfqpoint{2.614290in}{2.611046in}}%
\pgfpathlineto{\pgfqpoint{2.672707in}{2.620683in}}%
\pgfpathlineto{\pgfqpoint{2.731124in}{2.627166in}}%
\pgfpathlineto{\pgfqpoint{2.789541in}{2.630425in}}%
\pgfpathlineto{\pgfqpoint{2.847959in}{2.630425in}}%
\pgfpathlineto{\pgfqpoint{2.906376in}{2.627166in}}%
\pgfpathlineto{\pgfqpoint{2.964793in}{2.620683in}}%
\pgfpathlineto{\pgfqpoint{3.023210in}{2.611046in}}%
\pgfpathlineto{\pgfqpoint{3.081627in}{2.598357in}}%
\pgfpathlineto{\pgfqpoint{3.140044in}{2.582749in}}%
\pgfpathlineto{\pgfqpoint{3.198461in}{2.564379in}}%
\pgfpathlineto{\pgfqpoint{3.256878in}{2.543430in}}%
\pgfpathlineto{\pgfqpoint{3.315295in}{2.520101in}}%
\pgfpathlineto{\pgfqpoint{3.373712in}{2.494605in}}%
\pgfpathlineto{\pgfqpoint{3.451602in}{2.457628in}}%
\pgfpathlineto{\pgfqpoint{3.529491in}{2.417737in}}%
\pgfpathlineto{\pgfqpoint{3.607381in}{2.375473in}}%
\pgfpathlineto{\pgfqpoint{3.704742in}{2.320107in}}%
\pgfpathlineto{\pgfqpoint{3.841049in}{2.239567in}}%
\pgfpathlineto{\pgfqpoint{4.211024in}{2.019009in}}%
\pgfpathlineto{\pgfqpoint{4.327858in}{1.952417in}}%
\pgfpathlineto{\pgfqpoint{4.444692in}{1.888394in}}%
\pgfpathlineto{\pgfqpoint{4.542054in}{1.837265in}}%
\pgfpathlineto{\pgfqpoint{4.639416in}{1.788301in}}%
\pgfpathlineto{\pgfqpoint{4.736778in}{1.741574in}}%
\pgfpathlineto{\pgfqpoint{4.756250in}{1.732500in}}%
\pgfpathlineto{\pgfqpoint{4.756250in}{1.732500in}}%
\pgfusepath{stroke}%
\end{pgfscope}%
\begin{pgfscope}%
\pgfsetrectcap%
\pgfsetmiterjoin%
\pgfsetlinewidth{0.803000pt}%
\definecolor{currentstroke}{rgb}{0.000000,0.000000,0.000000}%
\pgfsetstrokecolor{currentstroke}%
\pgfsetdash{}{0pt}%
\pgfpathmoveto{\pgfqpoint{0.687500in}{0.385000in}}%
\pgfpathlineto{\pgfqpoint{0.687500in}{3.080000in}}%
\pgfusepath{stroke}%
\end{pgfscope}%
\begin{pgfscope}%
\pgfsetrectcap%
\pgfsetmiterjoin%
\pgfsetlinewidth{0.803000pt}%
\definecolor{currentstroke}{rgb}{0.000000,0.000000,0.000000}%
\pgfsetstrokecolor{currentstroke}%
\pgfsetdash{}{0pt}%
\pgfpathmoveto{\pgfqpoint{4.950000in}{0.385000in}}%
\pgfpathlineto{\pgfqpoint{4.950000in}{3.080000in}}%
\pgfusepath{stroke}%
\end{pgfscope}%
\begin{pgfscope}%
\pgfsetrectcap%
\pgfsetmiterjoin%
\pgfsetlinewidth{0.803000pt}%
\definecolor{currentstroke}{rgb}{0.000000,0.000000,0.000000}%
\pgfsetstrokecolor{currentstroke}%
\pgfsetdash{}{0pt}%
\pgfpathmoveto{\pgfqpoint{0.687500in}{0.385000in}}%
\pgfpathlineto{\pgfqpoint{4.950000in}{0.385000in}}%
\pgfusepath{stroke}%
\end{pgfscope}%
\begin{pgfscope}%
\pgfsetrectcap%
\pgfsetmiterjoin%
\pgfsetlinewidth{0.803000pt}%
\definecolor{currentstroke}{rgb}{0.000000,0.000000,0.000000}%
\pgfsetstrokecolor{currentstroke}%
\pgfsetdash{}{0pt}%
\pgfpathmoveto{\pgfqpoint{0.687500in}{3.080000in}}%
\pgfpathlineto{\pgfqpoint{4.950000in}{3.080000in}}%
\pgfusepath{stroke}%
\end{pgfscope}%
\begin{pgfscope}%
\definecolor{textcolor}{rgb}{0.000000,0.000000,0.000000}%
\pgfsetstrokecolor{textcolor}%
\pgfsetfillcolor{textcolor}%
\pgftext[x=2.818750in,y=3.163333in,,base]{\color{textcolor}\rmfamily\fontsize{12.000000}{14.400000}\selectfont N=6, 8, 16, 32}%
\end{pgfscope}%
\begin{pgfscope}%
\pgfsetbuttcap%
\pgfsetmiterjoin%
\definecolor{currentfill}{rgb}{1.000000,1.000000,1.000000}%
\pgfsetfillcolor{currentfill}%
\pgfsetfillopacity{0.800000}%
\pgfsetlinewidth{1.003750pt}%
\definecolor{currentstroke}{rgb}{0.800000,0.800000,0.800000}%
\pgfsetstrokecolor{currentstroke}%
\pgfsetstrokeopacity{0.800000}%
\pgfsetdash{}{0pt}%
\pgfpathmoveto{\pgfqpoint{0.784722in}{0.454444in}}%
\pgfpathlineto{\pgfqpoint{2.050468in}{0.454444in}}%
\pgfpathquadraticcurveto{\pgfqpoint{2.078246in}{0.454444in}}{\pgfqpoint{2.078246in}{0.482222in}}%
\pgfpathlineto{\pgfqpoint{2.078246in}{1.661422in}}%
\pgfpathquadraticcurveto{\pgfqpoint{2.078246in}{1.689199in}}{\pgfqpoint{2.050468in}{1.689199in}}%
\pgfpathlineto{\pgfqpoint{0.784722in}{1.689199in}}%
\pgfpathquadraticcurveto{\pgfqpoint{0.756944in}{1.689199in}}{\pgfqpoint{0.756944in}{1.661422in}}%
\pgfpathlineto{\pgfqpoint{0.756944in}{0.482222in}}%
\pgfpathquadraticcurveto{\pgfqpoint{0.756944in}{0.454444in}}{\pgfqpoint{0.784722in}{0.454444in}}%
\pgfpathclose%
\pgfusepath{stroke,fill}%
\end{pgfscope}%
\begin{pgfscope}%
\pgfsetrectcap%
\pgfsetroundjoin%
\pgfsetlinewidth{1.505625pt}%
\definecolor{currentstroke}{rgb}{0.121569,0.466667,0.705882}%
\pgfsetstrokecolor{currentstroke}%
\pgfsetdash{}{0pt}%
\pgfpathmoveto{\pgfqpoint{0.812500in}{1.498789in}}%
\pgfpathlineto{\pgfqpoint{1.090278in}{1.498789in}}%
\pgfusepath{stroke}%
\end{pgfscope}%
\begin{pgfscope}%
\definecolor{textcolor}{rgb}{0.000000,0.000000,0.000000}%
\pgfsetstrokecolor{textcolor}%
\pgfsetfillcolor{textcolor}%
\pgftext[x=1.201389in,y=1.450178in,left,base]{\color{textcolor}\rmfamily\fontsize{10.000000}{12.000000}\selectfont \(\displaystyle y(x)=\)\(\displaystyle \frac{1}{1+x^{2}}\)}%
\end{pgfscope}%
\begin{pgfscope}%
\pgfsetrectcap%
\pgfsetroundjoin%
\pgfsetlinewidth{1.505625pt}%
\definecolor{currentstroke}{rgb}{0.172549,0.627451,0.172549}%
\pgfsetstrokecolor{currentstroke}%
\pgfsetdash{}{0pt}%
\pgfpathmoveto{\pgfqpoint{0.812500in}{1.218333in}}%
\pgfpathlineto{\pgfqpoint{1.090278in}{1.218333in}}%
\pgfusepath{stroke}%
\end{pgfscope}%
\begin{pgfscope}%
\definecolor{textcolor}{rgb}{0.000000,0.000000,0.000000}%
\pgfsetstrokecolor{textcolor}%
\pgfsetfillcolor{textcolor}%
\pgftext[x=1.201389in,y=1.169722in,left,base]{\color{textcolor}\rmfamily\fontsize{10.000000}{12.000000}\selectfont W6(x)}%
\end{pgfscope}%
\begin{pgfscope}%
\pgfsetrectcap%
\pgfsetroundjoin%
\pgfsetlinewidth{1.505625pt}%
\definecolor{currentstroke}{rgb}{0.580392,0.403922,0.741176}%
\pgfsetstrokecolor{currentstroke}%
\pgfsetdash{}{0pt}%
\pgfpathmoveto{\pgfqpoint{0.812500in}{1.010000in}}%
\pgfpathlineto{\pgfqpoint{1.090278in}{1.010000in}}%
\pgfusepath{stroke}%
\end{pgfscope}%
\begin{pgfscope}%
\definecolor{textcolor}{rgb}{0.000000,0.000000,0.000000}%
\pgfsetstrokecolor{textcolor}%
\pgfsetfillcolor{textcolor}%
\pgftext[x=1.201389in,y=0.961389in,left,base]{\color{textcolor}\rmfamily\fontsize{10.000000}{12.000000}\selectfont W8(x)}%
\end{pgfscope}%
\begin{pgfscope}%
\pgfsetrectcap%
\pgfsetroundjoin%
\pgfsetlinewidth{1.505625pt}%
\definecolor{currentstroke}{rgb}{0.890196,0.466667,0.760784}%
\pgfsetstrokecolor{currentstroke}%
\pgfsetdash{}{0pt}%
\pgfpathmoveto{\pgfqpoint{0.812500in}{0.801667in}}%
\pgfpathlineto{\pgfqpoint{1.090278in}{0.801667in}}%
\pgfusepath{stroke}%
\end{pgfscope}%
\begin{pgfscope}%
\definecolor{textcolor}{rgb}{0.000000,0.000000,0.000000}%
\pgfsetstrokecolor{textcolor}%
\pgfsetfillcolor{textcolor}%
\pgftext[x=1.201389in,y=0.753056in,left,base]{\color{textcolor}\rmfamily\fontsize{10.000000}{12.000000}\selectfont W16(x)}%
\end{pgfscope}%
\begin{pgfscope}%
\pgfsetrectcap%
\pgfsetroundjoin%
\pgfsetlinewidth{1.505625pt}%
\definecolor{currentstroke}{rgb}{0.737255,0.741176,0.133333}%
\pgfsetstrokecolor{currentstroke}%
\pgfsetdash{}{0pt}%
\pgfpathmoveto{\pgfqpoint{0.812500in}{0.593333in}}%
\pgfpathlineto{\pgfqpoint{1.090278in}{0.593333in}}%
\pgfusepath{stroke}%
\end{pgfscope}%
\begin{pgfscope}%
\definecolor{textcolor}{rgb}{0.000000,0.000000,0.000000}%
\pgfsetstrokecolor{textcolor}%
\pgfsetfillcolor{textcolor}%
\pgftext[x=1.201389in,y=0.544722in,left,base]{\color{textcolor}\rmfamily\fontsize{10.000000}{12.000000}\selectfont W32(x)}%
\end{pgfscope}%
\end{pgfpicture}%
\makeatother%
\endgroup%
        
    \end{center}
    \caption{Węzły cosinus, funkcja \(\tilde{y}\), \(N=6,8,16,32\)}
\end{figure}

\begin{figure}[h]
    \begin{center}
        %% Creator: Matplotlib, PGF backend
%%
%% To include the figure in your LaTeX document, write
%%   \input{<filename>.pgf}
%%
%% Make sure the required packages are loaded in your preamble
%%   \usepackage{pgf}
%%
%% Figures using additional raster images can only be included by \input if
%% they are in the same directory as the main LaTeX file. For loading figures
%% from other directories you can use the `import` package
%%   \usepackage{import}
%% and then include the figures with
%%   \import{<path to file>}{<filename>.pgf}
%%
%% Matplotlib used the following preamble
%%
\begingroup%
\makeatletter%
\begin{pgfpicture}%
\pgfpathrectangle{\pgfpointorigin}{\pgfqpoint{5.500000in}{3.500000in}}%
\pgfusepath{use as bounding box, clip}%
\begin{pgfscope}%
\pgfsetbuttcap%
\pgfsetmiterjoin%
\definecolor{currentfill}{rgb}{1.000000,1.000000,1.000000}%
\pgfsetfillcolor{currentfill}%
\pgfsetlinewidth{0.000000pt}%
\definecolor{currentstroke}{rgb}{1.000000,1.000000,1.000000}%
\pgfsetstrokecolor{currentstroke}%
\pgfsetdash{}{0pt}%
\pgfpathmoveto{\pgfqpoint{0.000000in}{0.000000in}}%
\pgfpathlineto{\pgfqpoint{5.500000in}{0.000000in}}%
\pgfpathlineto{\pgfqpoint{5.500000in}{3.500000in}}%
\pgfpathlineto{\pgfqpoint{0.000000in}{3.500000in}}%
\pgfpathclose%
\pgfusepath{fill}%
\end{pgfscope}%
\begin{pgfscope}%
\pgfsetbuttcap%
\pgfsetmiterjoin%
\definecolor{currentfill}{rgb}{1.000000,1.000000,1.000000}%
\pgfsetfillcolor{currentfill}%
\pgfsetlinewidth{0.000000pt}%
\definecolor{currentstroke}{rgb}{0.000000,0.000000,0.000000}%
\pgfsetstrokecolor{currentstroke}%
\pgfsetstrokeopacity{0.000000}%
\pgfsetdash{}{0pt}%
\pgfpathmoveto{\pgfqpoint{0.687500in}{0.385000in}}%
\pgfpathlineto{\pgfqpoint{4.950000in}{0.385000in}}%
\pgfpathlineto{\pgfqpoint{4.950000in}{3.080000in}}%
\pgfpathlineto{\pgfqpoint{0.687500in}{3.080000in}}%
\pgfpathclose%
\pgfusepath{fill}%
\end{pgfscope}%
\begin{pgfscope}%
\pgfsetbuttcap%
\pgfsetroundjoin%
\definecolor{currentfill}{rgb}{0.000000,0.000000,0.000000}%
\pgfsetfillcolor{currentfill}%
\pgfsetlinewidth{0.803000pt}%
\definecolor{currentstroke}{rgb}{0.000000,0.000000,0.000000}%
\pgfsetstrokecolor{currentstroke}%
\pgfsetdash{}{0pt}%
\pgfsys@defobject{currentmarker}{\pgfqpoint{0.000000in}{-0.048611in}}{\pgfqpoint{0.000000in}{0.000000in}}{%
\pgfpathmoveto{\pgfqpoint{0.000000in}{0.000000in}}%
\pgfpathlineto{\pgfqpoint{0.000000in}{-0.048611in}}%
\pgfusepath{stroke,fill}%
}%
\begin{pgfscope}%
\pgfsys@transformshift{0.881250in}{0.385000in}%
\pgfsys@useobject{currentmarker}{}%
\end{pgfscope}%
\end{pgfscope}%
\begin{pgfscope}%
\definecolor{textcolor}{rgb}{0.000000,0.000000,0.000000}%
\pgfsetstrokecolor{textcolor}%
\pgfsetfillcolor{textcolor}%
\pgftext[x=0.881250in,y=0.287778in,,top]{\color{textcolor}\rmfamily\fontsize{10.000000}{12.000000}\selectfont \(\displaystyle -1.00\)}%
\end{pgfscope}%
\begin{pgfscope}%
\pgfsetbuttcap%
\pgfsetroundjoin%
\definecolor{currentfill}{rgb}{0.000000,0.000000,0.000000}%
\pgfsetfillcolor{currentfill}%
\pgfsetlinewidth{0.803000pt}%
\definecolor{currentstroke}{rgb}{0.000000,0.000000,0.000000}%
\pgfsetstrokecolor{currentstroke}%
\pgfsetdash{}{0pt}%
\pgfsys@defobject{currentmarker}{\pgfqpoint{0.000000in}{-0.048611in}}{\pgfqpoint{0.000000in}{0.000000in}}{%
\pgfpathmoveto{\pgfqpoint{0.000000in}{0.000000in}}%
\pgfpathlineto{\pgfqpoint{0.000000in}{-0.048611in}}%
\pgfusepath{stroke,fill}%
}%
\begin{pgfscope}%
\pgfsys@transformshift{1.365625in}{0.385000in}%
\pgfsys@useobject{currentmarker}{}%
\end{pgfscope}%
\end{pgfscope}%
\begin{pgfscope}%
\definecolor{textcolor}{rgb}{0.000000,0.000000,0.000000}%
\pgfsetstrokecolor{textcolor}%
\pgfsetfillcolor{textcolor}%
\pgftext[x=1.365625in,y=0.287778in,,top]{\color{textcolor}\rmfamily\fontsize{10.000000}{12.000000}\selectfont \(\displaystyle -0.75\)}%
\end{pgfscope}%
\begin{pgfscope}%
\pgfsetbuttcap%
\pgfsetroundjoin%
\definecolor{currentfill}{rgb}{0.000000,0.000000,0.000000}%
\pgfsetfillcolor{currentfill}%
\pgfsetlinewidth{0.803000pt}%
\definecolor{currentstroke}{rgb}{0.000000,0.000000,0.000000}%
\pgfsetstrokecolor{currentstroke}%
\pgfsetdash{}{0pt}%
\pgfsys@defobject{currentmarker}{\pgfqpoint{0.000000in}{-0.048611in}}{\pgfqpoint{0.000000in}{0.000000in}}{%
\pgfpathmoveto{\pgfqpoint{0.000000in}{0.000000in}}%
\pgfpathlineto{\pgfqpoint{0.000000in}{-0.048611in}}%
\pgfusepath{stroke,fill}%
}%
\begin{pgfscope}%
\pgfsys@transformshift{1.850000in}{0.385000in}%
\pgfsys@useobject{currentmarker}{}%
\end{pgfscope}%
\end{pgfscope}%
\begin{pgfscope}%
\definecolor{textcolor}{rgb}{0.000000,0.000000,0.000000}%
\pgfsetstrokecolor{textcolor}%
\pgfsetfillcolor{textcolor}%
\pgftext[x=1.850000in,y=0.287778in,,top]{\color{textcolor}\rmfamily\fontsize{10.000000}{12.000000}\selectfont \(\displaystyle -0.50\)}%
\end{pgfscope}%
\begin{pgfscope}%
\pgfsetbuttcap%
\pgfsetroundjoin%
\definecolor{currentfill}{rgb}{0.000000,0.000000,0.000000}%
\pgfsetfillcolor{currentfill}%
\pgfsetlinewidth{0.803000pt}%
\definecolor{currentstroke}{rgb}{0.000000,0.000000,0.000000}%
\pgfsetstrokecolor{currentstroke}%
\pgfsetdash{}{0pt}%
\pgfsys@defobject{currentmarker}{\pgfqpoint{0.000000in}{-0.048611in}}{\pgfqpoint{0.000000in}{0.000000in}}{%
\pgfpathmoveto{\pgfqpoint{0.000000in}{0.000000in}}%
\pgfpathlineto{\pgfqpoint{0.000000in}{-0.048611in}}%
\pgfusepath{stroke,fill}%
}%
\begin{pgfscope}%
\pgfsys@transformshift{2.334375in}{0.385000in}%
\pgfsys@useobject{currentmarker}{}%
\end{pgfscope}%
\end{pgfscope}%
\begin{pgfscope}%
\definecolor{textcolor}{rgb}{0.000000,0.000000,0.000000}%
\pgfsetstrokecolor{textcolor}%
\pgfsetfillcolor{textcolor}%
\pgftext[x=2.334375in,y=0.287778in,,top]{\color{textcolor}\rmfamily\fontsize{10.000000}{12.000000}\selectfont \(\displaystyle -0.25\)}%
\end{pgfscope}%
\begin{pgfscope}%
\pgfsetbuttcap%
\pgfsetroundjoin%
\definecolor{currentfill}{rgb}{0.000000,0.000000,0.000000}%
\pgfsetfillcolor{currentfill}%
\pgfsetlinewidth{0.803000pt}%
\definecolor{currentstroke}{rgb}{0.000000,0.000000,0.000000}%
\pgfsetstrokecolor{currentstroke}%
\pgfsetdash{}{0pt}%
\pgfsys@defobject{currentmarker}{\pgfqpoint{0.000000in}{-0.048611in}}{\pgfqpoint{0.000000in}{0.000000in}}{%
\pgfpathmoveto{\pgfqpoint{0.000000in}{0.000000in}}%
\pgfpathlineto{\pgfqpoint{0.000000in}{-0.048611in}}%
\pgfusepath{stroke,fill}%
}%
\begin{pgfscope}%
\pgfsys@transformshift{2.818750in}{0.385000in}%
\pgfsys@useobject{currentmarker}{}%
\end{pgfscope}%
\end{pgfscope}%
\begin{pgfscope}%
\definecolor{textcolor}{rgb}{0.000000,0.000000,0.000000}%
\pgfsetstrokecolor{textcolor}%
\pgfsetfillcolor{textcolor}%
\pgftext[x=2.818750in,y=0.287778in,,top]{\color{textcolor}\rmfamily\fontsize{10.000000}{12.000000}\selectfont \(\displaystyle 0.00\)}%
\end{pgfscope}%
\begin{pgfscope}%
\pgfsetbuttcap%
\pgfsetroundjoin%
\definecolor{currentfill}{rgb}{0.000000,0.000000,0.000000}%
\pgfsetfillcolor{currentfill}%
\pgfsetlinewidth{0.803000pt}%
\definecolor{currentstroke}{rgb}{0.000000,0.000000,0.000000}%
\pgfsetstrokecolor{currentstroke}%
\pgfsetdash{}{0pt}%
\pgfsys@defobject{currentmarker}{\pgfqpoint{0.000000in}{-0.048611in}}{\pgfqpoint{0.000000in}{0.000000in}}{%
\pgfpathmoveto{\pgfqpoint{0.000000in}{0.000000in}}%
\pgfpathlineto{\pgfqpoint{0.000000in}{-0.048611in}}%
\pgfusepath{stroke,fill}%
}%
\begin{pgfscope}%
\pgfsys@transformshift{3.303125in}{0.385000in}%
\pgfsys@useobject{currentmarker}{}%
\end{pgfscope}%
\end{pgfscope}%
\begin{pgfscope}%
\definecolor{textcolor}{rgb}{0.000000,0.000000,0.000000}%
\pgfsetstrokecolor{textcolor}%
\pgfsetfillcolor{textcolor}%
\pgftext[x=3.303125in,y=0.287778in,,top]{\color{textcolor}\rmfamily\fontsize{10.000000}{12.000000}\selectfont \(\displaystyle 0.25\)}%
\end{pgfscope}%
\begin{pgfscope}%
\pgfsetbuttcap%
\pgfsetroundjoin%
\definecolor{currentfill}{rgb}{0.000000,0.000000,0.000000}%
\pgfsetfillcolor{currentfill}%
\pgfsetlinewidth{0.803000pt}%
\definecolor{currentstroke}{rgb}{0.000000,0.000000,0.000000}%
\pgfsetstrokecolor{currentstroke}%
\pgfsetdash{}{0pt}%
\pgfsys@defobject{currentmarker}{\pgfqpoint{0.000000in}{-0.048611in}}{\pgfqpoint{0.000000in}{0.000000in}}{%
\pgfpathmoveto{\pgfqpoint{0.000000in}{0.000000in}}%
\pgfpathlineto{\pgfqpoint{0.000000in}{-0.048611in}}%
\pgfusepath{stroke,fill}%
}%
\begin{pgfscope}%
\pgfsys@transformshift{3.787500in}{0.385000in}%
\pgfsys@useobject{currentmarker}{}%
\end{pgfscope}%
\end{pgfscope}%
\begin{pgfscope}%
\definecolor{textcolor}{rgb}{0.000000,0.000000,0.000000}%
\pgfsetstrokecolor{textcolor}%
\pgfsetfillcolor{textcolor}%
\pgftext[x=3.787500in,y=0.287778in,,top]{\color{textcolor}\rmfamily\fontsize{10.000000}{12.000000}\selectfont \(\displaystyle 0.50\)}%
\end{pgfscope}%
\begin{pgfscope}%
\pgfsetbuttcap%
\pgfsetroundjoin%
\definecolor{currentfill}{rgb}{0.000000,0.000000,0.000000}%
\pgfsetfillcolor{currentfill}%
\pgfsetlinewidth{0.803000pt}%
\definecolor{currentstroke}{rgb}{0.000000,0.000000,0.000000}%
\pgfsetstrokecolor{currentstroke}%
\pgfsetdash{}{0pt}%
\pgfsys@defobject{currentmarker}{\pgfqpoint{0.000000in}{-0.048611in}}{\pgfqpoint{0.000000in}{0.000000in}}{%
\pgfpathmoveto{\pgfqpoint{0.000000in}{0.000000in}}%
\pgfpathlineto{\pgfqpoint{0.000000in}{-0.048611in}}%
\pgfusepath{stroke,fill}%
}%
\begin{pgfscope}%
\pgfsys@transformshift{4.271875in}{0.385000in}%
\pgfsys@useobject{currentmarker}{}%
\end{pgfscope}%
\end{pgfscope}%
\begin{pgfscope}%
\definecolor{textcolor}{rgb}{0.000000,0.000000,0.000000}%
\pgfsetstrokecolor{textcolor}%
\pgfsetfillcolor{textcolor}%
\pgftext[x=4.271875in,y=0.287778in,,top]{\color{textcolor}\rmfamily\fontsize{10.000000}{12.000000}\selectfont \(\displaystyle 0.75\)}%
\end{pgfscope}%
\begin{pgfscope}%
\pgfsetbuttcap%
\pgfsetroundjoin%
\definecolor{currentfill}{rgb}{0.000000,0.000000,0.000000}%
\pgfsetfillcolor{currentfill}%
\pgfsetlinewidth{0.803000pt}%
\definecolor{currentstroke}{rgb}{0.000000,0.000000,0.000000}%
\pgfsetstrokecolor{currentstroke}%
\pgfsetdash{}{0pt}%
\pgfsys@defobject{currentmarker}{\pgfqpoint{0.000000in}{-0.048611in}}{\pgfqpoint{0.000000in}{0.000000in}}{%
\pgfpathmoveto{\pgfqpoint{0.000000in}{0.000000in}}%
\pgfpathlineto{\pgfqpoint{0.000000in}{-0.048611in}}%
\pgfusepath{stroke,fill}%
}%
\begin{pgfscope}%
\pgfsys@transformshift{4.756250in}{0.385000in}%
\pgfsys@useobject{currentmarker}{}%
\end{pgfscope}%
\end{pgfscope}%
\begin{pgfscope}%
\definecolor{textcolor}{rgb}{0.000000,0.000000,0.000000}%
\pgfsetstrokecolor{textcolor}%
\pgfsetfillcolor{textcolor}%
\pgftext[x=4.756250in,y=0.287778in,,top]{\color{textcolor}\rmfamily\fontsize{10.000000}{12.000000}\selectfont \(\displaystyle 1.00\)}%
\end{pgfscope}%
\begin{pgfscope}%
\definecolor{textcolor}{rgb}{0.000000,0.000000,0.000000}%
\pgfsetstrokecolor{textcolor}%
\pgfsetfillcolor{textcolor}%
\pgftext[x=2.818750in,y=0.108766in,,top]{\color{textcolor}\rmfamily\fontsize{10.000000}{12.000000}\selectfont x}%
\end{pgfscope}%
\begin{pgfscope}%
\pgfsetbuttcap%
\pgfsetroundjoin%
\definecolor{currentfill}{rgb}{0.000000,0.000000,0.000000}%
\pgfsetfillcolor{currentfill}%
\pgfsetlinewidth{0.803000pt}%
\definecolor{currentstroke}{rgb}{0.000000,0.000000,0.000000}%
\pgfsetstrokecolor{currentstroke}%
\pgfsetdash{}{0pt}%
\pgfsys@defobject{currentmarker}{\pgfqpoint{-0.048611in}{0.000000in}}{\pgfqpoint{0.000000in}{0.000000in}}{%
\pgfpathmoveto{\pgfqpoint{0.000000in}{0.000000in}}%
\pgfpathlineto{\pgfqpoint{-0.048611in}{0.000000in}}%
\pgfusepath{stroke,fill}%
}%
\begin{pgfscope}%
\pgfsys@transformshift{0.687500in}{0.474833in}%
\pgfsys@useobject{currentmarker}{}%
\end{pgfscope}%
\end{pgfscope}%
\begin{pgfscope}%
\definecolor{textcolor}{rgb}{0.000000,0.000000,0.000000}%
\pgfsetstrokecolor{textcolor}%
\pgfsetfillcolor{textcolor}%
\pgftext[x=0.304783in,y=0.426608in,left,base]{\color{textcolor}\rmfamily\fontsize{10.000000}{12.000000}\selectfont \(\displaystyle -0.2\)}%
\end{pgfscope}%
\begin{pgfscope}%
\pgfsetbuttcap%
\pgfsetroundjoin%
\definecolor{currentfill}{rgb}{0.000000,0.000000,0.000000}%
\pgfsetfillcolor{currentfill}%
\pgfsetlinewidth{0.803000pt}%
\definecolor{currentstroke}{rgb}{0.000000,0.000000,0.000000}%
\pgfsetstrokecolor{currentstroke}%
\pgfsetdash{}{0pt}%
\pgfsys@defobject{currentmarker}{\pgfqpoint{-0.048611in}{0.000000in}}{\pgfqpoint{0.000000in}{0.000000in}}{%
\pgfpathmoveto{\pgfqpoint{0.000000in}{0.000000in}}%
\pgfpathlineto{\pgfqpoint{-0.048611in}{0.000000in}}%
\pgfusepath{stroke,fill}%
}%
\begin{pgfscope}%
\pgfsys@transformshift{0.687500in}{0.834167in}%
\pgfsys@useobject{currentmarker}{}%
\end{pgfscope}%
\end{pgfscope}%
\begin{pgfscope}%
\definecolor{textcolor}{rgb}{0.000000,0.000000,0.000000}%
\pgfsetstrokecolor{textcolor}%
\pgfsetfillcolor{textcolor}%
\pgftext[x=0.412808in,y=0.785941in,left,base]{\color{textcolor}\rmfamily\fontsize{10.000000}{12.000000}\selectfont \(\displaystyle 0.0\)}%
\end{pgfscope}%
\begin{pgfscope}%
\pgfsetbuttcap%
\pgfsetroundjoin%
\definecolor{currentfill}{rgb}{0.000000,0.000000,0.000000}%
\pgfsetfillcolor{currentfill}%
\pgfsetlinewidth{0.803000pt}%
\definecolor{currentstroke}{rgb}{0.000000,0.000000,0.000000}%
\pgfsetstrokecolor{currentstroke}%
\pgfsetdash{}{0pt}%
\pgfsys@defobject{currentmarker}{\pgfqpoint{-0.048611in}{0.000000in}}{\pgfqpoint{0.000000in}{0.000000in}}{%
\pgfpathmoveto{\pgfqpoint{0.000000in}{0.000000in}}%
\pgfpathlineto{\pgfqpoint{-0.048611in}{0.000000in}}%
\pgfusepath{stroke,fill}%
}%
\begin{pgfscope}%
\pgfsys@transformshift{0.687500in}{1.193500in}%
\pgfsys@useobject{currentmarker}{}%
\end{pgfscope}%
\end{pgfscope}%
\begin{pgfscope}%
\definecolor{textcolor}{rgb}{0.000000,0.000000,0.000000}%
\pgfsetstrokecolor{textcolor}%
\pgfsetfillcolor{textcolor}%
\pgftext[x=0.412808in,y=1.145275in,left,base]{\color{textcolor}\rmfamily\fontsize{10.000000}{12.000000}\selectfont \(\displaystyle 0.2\)}%
\end{pgfscope}%
\begin{pgfscope}%
\pgfsetbuttcap%
\pgfsetroundjoin%
\definecolor{currentfill}{rgb}{0.000000,0.000000,0.000000}%
\pgfsetfillcolor{currentfill}%
\pgfsetlinewidth{0.803000pt}%
\definecolor{currentstroke}{rgb}{0.000000,0.000000,0.000000}%
\pgfsetstrokecolor{currentstroke}%
\pgfsetdash{}{0pt}%
\pgfsys@defobject{currentmarker}{\pgfqpoint{-0.048611in}{0.000000in}}{\pgfqpoint{0.000000in}{0.000000in}}{%
\pgfpathmoveto{\pgfqpoint{0.000000in}{0.000000in}}%
\pgfpathlineto{\pgfqpoint{-0.048611in}{0.000000in}}%
\pgfusepath{stroke,fill}%
}%
\begin{pgfscope}%
\pgfsys@transformshift{0.687500in}{1.552833in}%
\pgfsys@useobject{currentmarker}{}%
\end{pgfscope}%
\end{pgfscope}%
\begin{pgfscope}%
\definecolor{textcolor}{rgb}{0.000000,0.000000,0.000000}%
\pgfsetstrokecolor{textcolor}%
\pgfsetfillcolor{textcolor}%
\pgftext[x=0.412808in,y=1.504608in,left,base]{\color{textcolor}\rmfamily\fontsize{10.000000}{12.000000}\selectfont \(\displaystyle 0.4\)}%
\end{pgfscope}%
\begin{pgfscope}%
\pgfsetbuttcap%
\pgfsetroundjoin%
\definecolor{currentfill}{rgb}{0.000000,0.000000,0.000000}%
\pgfsetfillcolor{currentfill}%
\pgfsetlinewidth{0.803000pt}%
\definecolor{currentstroke}{rgb}{0.000000,0.000000,0.000000}%
\pgfsetstrokecolor{currentstroke}%
\pgfsetdash{}{0pt}%
\pgfsys@defobject{currentmarker}{\pgfqpoint{-0.048611in}{0.000000in}}{\pgfqpoint{0.000000in}{0.000000in}}{%
\pgfpathmoveto{\pgfqpoint{0.000000in}{0.000000in}}%
\pgfpathlineto{\pgfqpoint{-0.048611in}{0.000000in}}%
\pgfusepath{stroke,fill}%
}%
\begin{pgfscope}%
\pgfsys@transformshift{0.687500in}{1.912167in}%
\pgfsys@useobject{currentmarker}{}%
\end{pgfscope}%
\end{pgfscope}%
\begin{pgfscope}%
\definecolor{textcolor}{rgb}{0.000000,0.000000,0.000000}%
\pgfsetstrokecolor{textcolor}%
\pgfsetfillcolor{textcolor}%
\pgftext[x=0.412808in,y=1.863941in,left,base]{\color{textcolor}\rmfamily\fontsize{10.000000}{12.000000}\selectfont \(\displaystyle 0.6\)}%
\end{pgfscope}%
\begin{pgfscope}%
\pgfsetbuttcap%
\pgfsetroundjoin%
\definecolor{currentfill}{rgb}{0.000000,0.000000,0.000000}%
\pgfsetfillcolor{currentfill}%
\pgfsetlinewidth{0.803000pt}%
\definecolor{currentstroke}{rgb}{0.000000,0.000000,0.000000}%
\pgfsetstrokecolor{currentstroke}%
\pgfsetdash{}{0pt}%
\pgfsys@defobject{currentmarker}{\pgfqpoint{-0.048611in}{0.000000in}}{\pgfqpoint{0.000000in}{0.000000in}}{%
\pgfpathmoveto{\pgfqpoint{0.000000in}{0.000000in}}%
\pgfpathlineto{\pgfqpoint{-0.048611in}{0.000000in}}%
\pgfusepath{stroke,fill}%
}%
\begin{pgfscope}%
\pgfsys@transformshift{0.687500in}{2.271500in}%
\pgfsys@useobject{currentmarker}{}%
\end{pgfscope}%
\end{pgfscope}%
\begin{pgfscope}%
\definecolor{textcolor}{rgb}{0.000000,0.000000,0.000000}%
\pgfsetstrokecolor{textcolor}%
\pgfsetfillcolor{textcolor}%
\pgftext[x=0.412808in,y=2.223275in,left,base]{\color{textcolor}\rmfamily\fontsize{10.000000}{12.000000}\selectfont \(\displaystyle 0.8\)}%
\end{pgfscope}%
\begin{pgfscope}%
\pgfsetbuttcap%
\pgfsetroundjoin%
\definecolor{currentfill}{rgb}{0.000000,0.000000,0.000000}%
\pgfsetfillcolor{currentfill}%
\pgfsetlinewidth{0.803000pt}%
\definecolor{currentstroke}{rgb}{0.000000,0.000000,0.000000}%
\pgfsetstrokecolor{currentstroke}%
\pgfsetdash{}{0pt}%
\pgfsys@defobject{currentmarker}{\pgfqpoint{-0.048611in}{0.000000in}}{\pgfqpoint{0.000000in}{0.000000in}}{%
\pgfpathmoveto{\pgfqpoint{0.000000in}{0.000000in}}%
\pgfpathlineto{\pgfqpoint{-0.048611in}{0.000000in}}%
\pgfusepath{stroke,fill}%
}%
\begin{pgfscope}%
\pgfsys@transformshift{0.687500in}{2.630833in}%
\pgfsys@useobject{currentmarker}{}%
\end{pgfscope}%
\end{pgfscope}%
\begin{pgfscope}%
\definecolor{textcolor}{rgb}{0.000000,0.000000,0.000000}%
\pgfsetstrokecolor{textcolor}%
\pgfsetfillcolor{textcolor}%
\pgftext[x=0.412808in,y=2.582608in,left,base]{\color{textcolor}\rmfamily\fontsize{10.000000}{12.000000}\selectfont \(\displaystyle 1.0\)}%
\end{pgfscope}%
\begin{pgfscope}%
\pgfsetbuttcap%
\pgfsetroundjoin%
\definecolor{currentfill}{rgb}{0.000000,0.000000,0.000000}%
\pgfsetfillcolor{currentfill}%
\pgfsetlinewidth{0.803000pt}%
\definecolor{currentstroke}{rgb}{0.000000,0.000000,0.000000}%
\pgfsetstrokecolor{currentstroke}%
\pgfsetdash{}{0pt}%
\pgfsys@defobject{currentmarker}{\pgfqpoint{-0.048611in}{0.000000in}}{\pgfqpoint{0.000000in}{0.000000in}}{%
\pgfpathmoveto{\pgfqpoint{0.000000in}{0.000000in}}%
\pgfpathlineto{\pgfqpoint{-0.048611in}{0.000000in}}%
\pgfusepath{stroke,fill}%
}%
\begin{pgfscope}%
\pgfsys@transformshift{0.687500in}{2.990167in}%
\pgfsys@useobject{currentmarker}{}%
\end{pgfscope}%
\end{pgfscope}%
\begin{pgfscope}%
\definecolor{textcolor}{rgb}{0.000000,0.000000,0.000000}%
\pgfsetstrokecolor{textcolor}%
\pgfsetfillcolor{textcolor}%
\pgftext[x=0.412808in,y=2.941941in,left,base]{\color{textcolor}\rmfamily\fontsize{10.000000}{12.000000}\selectfont \(\displaystyle 1.2\)}%
\end{pgfscope}%
\begin{pgfscope}%
\definecolor{textcolor}{rgb}{0.000000,0.000000,0.000000}%
\pgfsetstrokecolor{textcolor}%
\pgfsetfillcolor{textcolor}%
\pgftext[x=0.249228in,y=1.732500in,,bottom,rotate=90.000000]{\color{textcolor}\rmfamily\fontsize{10.000000}{12.000000}\selectfont y}%
\end{pgfscope}%
\begin{pgfscope}%
\pgfpathrectangle{\pgfqpoint{0.687500in}{0.385000in}}{\pgfqpoint{4.262500in}{2.695000in}}%
\pgfusepath{clip}%
\pgfsetrectcap%
\pgfsetroundjoin%
\pgfsetlinewidth{1.505625pt}%
\definecolor{currentstroke}{rgb}{0.121569,0.466667,0.705882}%
\pgfsetstrokecolor{currentstroke}%
\pgfsetdash{}{0pt}%
\pgfpathmoveto{\pgfqpoint{0.881250in}{1.732500in}}%
\pgfpathlineto{\pgfqpoint{0.978612in}{1.778775in}}%
\pgfpathlineto{\pgfqpoint{1.075974in}{1.827296in}}%
\pgfpathlineto{\pgfqpoint{1.173335in}{1.878000in}}%
\pgfpathlineto{\pgfqpoint{1.290170in}{1.941556in}}%
\pgfpathlineto{\pgfqpoint{1.407004in}{2.007752in}}%
\pgfpathlineto{\pgfqpoint{1.543310in}{2.087643in}}%
\pgfpathlineto{\pgfqpoint{1.796451in}{2.239567in}}%
\pgfpathlineto{\pgfqpoint{1.932758in}{2.320107in}}%
\pgfpathlineto{\pgfqpoint{2.030119in}{2.375473in}}%
\pgfpathlineto{\pgfqpoint{2.108009in}{2.417737in}}%
\pgfpathlineto{\pgfqpoint{2.185898in}{2.457628in}}%
\pgfpathlineto{\pgfqpoint{2.263788in}{2.494605in}}%
\pgfpathlineto{\pgfqpoint{2.322205in}{2.520101in}}%
\pgfpathlineto{\pgfqpoint{2.380622in}{2.543430in}}%
\pgfpathlineto{\pgfqpoint{2.439039in}{2.564379in}}%
\pgfpathlineto{\pgfqpoint{2.497456in}{2.582749in}}%
\pgfpathlineto{\pgfqpoint{2.555873in}{2.598357in}}%
\pgfpathlineto{\pgfqpoint{2.614290in}{2.611046in}}%
\pgfpathlineto{\pgfqpoint{2.672707in}{2.620683in}}%
\pgfpathlineto{\pgfqpoint{2.731124in}{2.627166in}}%
\pgfpathlineto{\pgfqpoint{2.789541in}{2.630425in}}%
\pgfpathlineto{\pgfqpoint{2.847959in}{2.630425in}}%
\pgfpathlineto{\pgfqpoint{2.906376in}{2.627166in}}%
\pgfpathlineto{\pgfqpoint{2.964793in}{2.620683in}}%
\pgfpathlineto{\pgfqpoint{3.023210in}{2.611046in}}%
\pgfpathlineto{\pgfqpoint{3.081627in}{2.598357in}}%
\pgfpathlineto{\pgfqpoint{3.140044in}{2.582749in}}%
\pgfpathlineto{\pgfqpoint{3.198461in}{2.564379in}}%
\pgfpathlineto{\pgfqpoint{3.256878in}{2.543430in}}%
\pgfpathlineto{\pgfqpoint{3.315295in}{2.520101in}}%
\pgfpathlineto{\pgfqpoint{3.373712in}{2.494605in}}%
\pgfpathlineto{\pgfqpoint{3.451602in}{2.457628in}}%
\pgfpathlineto{\pgfqpoint{3.529491in}{2.417737in}}%
\pgfpathlineto{\pgfqpoint{3.607381in}{2.375473in}}%
\pgfpathlineto{\pgfqpoint{3.704742in}{2.320107in}}%
\pgfpathlineto{\pgfqpoint{3.841049in}{2.239567in}}%
\pgfpathlineto{\pgfqpoint{4.211024in}{2.019009in}}%
\pgfpathlineto{\pgfqpoint{4.327858in}{1.952417in}}%
\pgfpathlineto{\pgfqpoint{4.444692in}{1.888394in}}%
\pgfpathlineto{\pgfqpoint{4.542054in}{1.837265in}}%
\pgfpathlineto{\pgfqpoint{4.639416in}{1.788301in}}%
\pgfpathlineto{\pgfqpoint{4.736778in}{1.741574in}}%
\pgfpathlineto{\pgfqpoint{4.756250in}{1.732500in}}%
\pgfpathlineto{\pgfqpoint{4.756250in}{1.732500in}}%
\pgfusepath{stroke}%
\end{pgfscope}%
\begin{pgfscope}%
\pgfpathrectangle{\pgfqpoint{0.687500in}{0.385000in}}{\pgfqpoint{4.262500in}{2.695000in}}%
\pgfusepath{clip}%
\pgfsetbuttcap%
\pgfsetroundjoin%
\definecolor{currentfill}{rgb}{1.000000,0.498039,0.054902}%
\pgfsetfillcolor{currentfill}%
\pgfsetlinewidth{1.003750pt}%
\definecolor{currentstroke}{rgb}{1.000000,0.498039,0.054902}%
\pgfsetstrokecolor{currentstroke}%
\pgfsetdash{}{0pt}%
\pgfsys@defobject{currentmarker}{\pgfqpoint{-0.020833in}{-0.020833in}}{\pgfqpoint{0.020833in}{0.020833in}}{%
\pgfpathmoveto{\pgfqpoint{0.000000in}{-0.020833in}}%
\pgfpathcurveto{\pgfqpoint{0.005525in}{-0.020833in}}{\pgfqpoint{0.010825in}{-0.018638in}}{\pgfqpoint{0.014731in}{-0.014731in}}%
\pgfpathcurveto{\pgfqpoint{0.018638in}{-0.010825in}}{\pgfqpoint{0.020833in}{-0.005525in}}{\pgfqpoint{0.020833in}{0.000000in}}%
\pgfpathcurveto{\pgfqpoint{0.020833in}{0.005525in}}{\pgfqpoint{0.018638in}{0.010825in}}{\pgfqpoint{0.014731in}{0.014731in}}%
\pgfpathcurveto{\pgfqpoint{0.010825in}{0.018638in}}{\pgfqpoint{0.005525in}{0.020833in}}{\pgfqpoint{0.000000in}{0.020833in}}%
\pgfpathcurveto{\pgfqpoint{-0.005525in}{0.020833in}}{\pgfqpoint{-0.010825in}{0.018638in}}{\pgfqpoint{-0.014731in}{0.014731in}}%
\pgfpathcurveto{\pgfqpoint{-0.018638in}{0.010825in}}{\pgfqpoint{-0.020833in}{0.005525in}}{\pgfqpoint{-0.020833in}{0.000000in}}%
\pgfpathcurveto{\pgfqpoint{-0.020833in}{-0.005525in}}{\pgfqpoint{-0.018638in}{-0.010825in}}{\pgfqpoint{-0.014731in}{-0.014731in}}%
\pgfpathcurveto{\pgfqpoint{-0.010825in}{-0.018638in}}{\pgfqpoint{-0.005525in}{-0.020833in}}{\pgfqpoint{0.000000in}{-0.020833in}}%
\pgfpathclose%
\pgfusepath{stroke,fill}%
}%
\begin{pgfscope}%
\pgfsys@transformshift{4.707673in}{1.755305in}%
\pgfsys@useobject{currentmarker}{}%
\end{pgfscope}%
\begin{pgfscope}%
\pgfsys@transformshift{4.333548in}{1.949236in}%
\pgfsys@useobject{currentmarker}{}%
\end{pgfscope}%
\begin{pgfscope}%
\pgfsys@transformshift{3.659400in}{2.346188in}%
\pgfsys@useobject{currentmarker}{}%
\end{pgfscope}%
\begin{pgfscope}%
\pgfsys@transformshift{2.818750in}{2.630833in}%
\pgfsys@useobject{currentmarker}{}%
\end{pgfscope}%
\begin{pgfscope}%
\pgfsys@transformshift{1.978100in}{2.346188in}%
\pgfsys@useobject{currentmarker}{}%
\end{pgfscope}%
\begin{pgfscope}%
\pgfsys@transformshift{1.303952in}{1.949236in}%
\pgfsys@useobject{currentmarker}{}%
\end{pgfscope}%
\begin{pgfscope}%
\pgfsys@transformshift{0.929827in}{1.755305in}%
\pgfsys@useobject{currentmarker}{}%
\end{pgfscope}%
\end{pgfscope}%
\begin{pgfscope}%
\pgfpathrectangle{\pgfqpoint{0.687500in}{0.385000in}}{\pgfqpoint{4.262500in}{2.695000in}}%
\pgfusepath{clip}%
\pgfsetrectcap%
\pgfsetroundjoin%
\pgfsetlinewidth{1.505625pt}%
\definecolor{currentstroke}{rgb}{0.172549,0.627451,0.172549}%
\pgfsetstrokecolor{currentstroke}%
\pgfsetdash{}{0pt}%
\pgfpathmoveto{\pgfqpoint{0.881250in}{1.728741in}}%
\pgfpathlineto{\pgfqpoint{0.959139in}{1.770873in}}%
\pgfpathlineto{\pgfqpoint{1.075974in}{1.831033in}}%
\pgfpathlineto{\pgfqpoint{1.251225in}{1.921195in}}%
\pgfpathlineto{\pgfqpoint{1.368059in}{1.984150in}}%
\pgfpathlineto{\pgfqpoint{1.465421in}{2.038994in}}%
\pgfpathlineto{\pgfqpoint{1.582255in}{2.107475in}}%
\pgfpathlineto{\pgfqpoint{1.738034in}{2.201796in}}%
\pgfpathlineto{\pgfqpoint{1.932758in}{2.319590in}}%
\pgfpathlineto{\pgfqpoint{2.030119in}{2.376013in}}%
\pgfpathlineto{\pgfqpoint{2.108009in}{2.418928in}}%
\pgfpathlineto{\pgfqpoint{2.185898in}{2.459238in}}%
\pgfpathlineto{\pgfqpoint{2.244315in}{2.487425in}}%
\pgfpathlineto{\pgfqpoint{2.302732in}{2.513609in}}%
\pgfpathlineto{\pgfqpoint{2.361149in}{2.537581in}}%
\pgfpathlineto{\pgfqpoint{2.419567in}{2.559149in}}%
\pgfpathlineto{\pgfqpoint{2.477984in}{2.578136in}}%
\pgfpathlineto{\pgfqpoint{2.536401in}{2.594387in}}%
\pgfpathlineto{\pgfqpoint{2.594818in}{2.607771in}}%
\pgfpathlineto{\pgfqpoint{2.653235in}{2.618176in}}%
\pgfpathlineto{\pgfqpoint{2.711652in}{2.625517in}}%
\pgfpathlineto{\pgfqpoint{2.770069in}{2.629733in}}%
\pgfpathlineto{\pgfqpoint{2.828486in}{2.630789in}}%
\pgfpathlineto{\pgfqpoint{2.886903in}{2.628677in}}%
\pgfpathlineto{\pgfqpoint{2.945320in}{2.623415in}}%
\pgfpathlineto{\pgfqpoint{3.003737in}{2.615044in}}%
\pgfpathlineto{\pgfqpoint{3.062155in}{2.603636in}}%
\pgfpathlineto{\pgfqpoint{3.120572in}{2.589283in}}%
\pgfpathlineto{\pgfqpoint{3.178989in}{2.572103in}}%
\pgfpathlineto{\pgfqpoint{3.237406in}{2.552238in}}%
\pgfpathlineto{\pgfqpoint{3.295823in}{2.529849in}}%
\pgfpathlineto{\pgfqpoint{3.354240in}{2.505117in}}%
\pgfpathlineto{\pgfqpoint{3.412657in}{2.478241in}}%
\pgfpathlineto{\pgfqpoint{3.490546in}{2.439444in}}%
\pgfpathlineto{\pgfqpoint{3.568436in}{2.397760in}}%
\pgfpathlineto{\pgfqpoint{3.665798in}{2.342465in}}%
\pgfpathlineto{\pgfqpoint{3.782632in}{2.272942in}}%
\pgfpathlineto{\pgfqpoint{4.133134in}{2.061525in}}%
\pgfpathlineto{\pgfqpoint{4.249969in}{1.994943in}}%
\pgfpathlineto{\pgfqpoint{4.347330in}{1.941850in}}%
\pgfpathlineto{\pgfqpoint{4.464165in}{1.880709in}}%
\pgfpathlineto{\pgfqpoint{4.717305in}{1.750121in}}%
\pgfpathlineto{\pgfqpoint{4.756250in}{1.728741in}}%
\pgfpathlineto{\pgfqpoint{4.756250in}{1.728741in}}%
\pgfusepath{stroke}%
\end{pgfscope}%
\begin{pgfscope}%
\pgfpathrectangle{\pgfqpoint{0.687500in}{0.385000in}}{\pgfqpoint{4.262500in}{2.695000in}}%
\pgfusepath{clip}%
\pgfsetbuttcap%
\pgfsetroundjoin%
\definecolor{currentfill}{rgb}{0.839216,0.152941,0.156863}%
\pgfsetfillcolor{currentfill}%
\pgfsetlinewidth{1.003750pt}%
\definecolor{currentstroke}{rgb}{0.839216,0.152941,0.156863}%
\pgfsetstrokecolor{currentstroke}%
\pgfsetdash{}{0pt}%
\pgfsys@defobject{currentmarker}{\pgfqpoint{-0.020833in}{-0.020833in}}{\pgfqpoint{0.020833in}{0.020833in}}{%
\pgfpathmoveto{\pgfqpoint{0.000000in}{-0.020833in}}%
\pgfpathcurveto{\pgfqpoint{0.005525in}{-0.020833in}}{\pgfqpoint{0.010825in}{-0.018638in}}{\pgfqpoint{0.014731in}{-0.014731in}}%
\pgfpathcurveto{\pgfqpoint{0.018638in}{-0.010825in}}{\pgfqpoint{0.020833in}{-0.005525in}}{\pgfqpoint{0.020833in}{0.000000in}}%
\pgfpathcurveto{\pgfqpoint{0.020833in}{0.005525in}}{\pgfqpoint{0.018638in}{0.010825in}}{\pgfqpoint{0.014731in}{0.014731in}}%
\pgfpathcurveto{\pgfqpoint{0.010825in}{0.018638in}}{\pgfqpoint{0.005525in}{0.020833in}}{\pgfqpoint{0.000000in}{0.020833in}}%
\pgfpathcurveto{\pgfqpoint{-0.005525in}{0.020833in}}{\pgfqpoint{-0.010825in}{0.018638in}}{\pgfqpoint{-0.014731in}{0.014731in}}%
\pgfpathcurveto{\pgfqpoint{-0.018638in}{0.010825in}}{\pgfqpoint{-0.020833in}{0.005525in}}{\pgfqpoint{-0.020833in}{0.000000in}}%
\pgfpathcurveto{\pgfqpoint{-0.020833in}{-0.005525in}}{\pgfqpoint{-0.018638in}{-0.010825in}}{\pgfqpoint{-0.014731in}{-0.014731in}}%
\pgfpathcurveto{\pgfqpoint{-0.010825in}{-0.018638in}}{\pgfqpoint{-0.005525in}{-0.020833in}}{\pgfqpoint{0.000000in}{-0.020833in}}%
\pgfpathclose%
\pgfusepath{stroke,fill}%
}%
\begin{pgfscope}%
\pgfsys@transformshift{4.755684in}{1.732762in}%
\pgfsys@useobject{currentmarker}{}%
\end{pgfscope}%
\begin{pgfscope}%
\pgfsys@transformshift{4.751160in}{1.734863in}%
\pgfsys@useobject{currentmarker}{}%
\end{pgfscope}%
\begin{pgfscope}%
\pgfsys@transformshift{4.742123in}{1.739074in}%
\pgfsys@useobject{currentmarker}{}%
\end{pgfscope}%
\begin{pgfscope}%
\pgfsys@transformshift{4.728594in}{1.745414in}%
\pgfsys@useobject{currentmarker}{}%
\end{pgfscope}%
\begin{pgfscope}%
\pgfsys@transformshift{4.710605in}{1.753913in}%
\pgfsys@useobject{currentmarker}{}%
\end{pgfscope}%
\begin{pgfscope}%
\pgfsys@transformshift{4.688196in}{1.764607in}%
\pgfsys@useobject{currentmarker}{}%
\end{pgfscope}%
\begin{pgfscope}%
\pgfsys@transformshift{4.661422in}{1.777542in}%
\pgfsys@useobject{currentmarker}{}%
\end{pgfscope}%
\begin{pgfscope}%
\pgfsys@transformshift{4.630344in}{1.792770in}%
\pgfsys@useobject{currentmarker}{}%
\end{pgfscope}%
\begin{pgfscope}%
\pgfsys@transformshift{4.595035in}{1.810346in}%
\pgfsys@useobject{currentmarker}{}%
\end{pgfscope}%
\begin{pgfscope}%
\pgfsys@transformshift{4.555577in}{1.830333in}%
\pgfsys@useobject{currentmarker}{}%
\end{pgfscope}%
\begin{pgfscope}%
\pgfsys@transformshift{4.512063in}{1.852789in}%
\pgfsys@useobject{currentmarker}{}%
\end{pgfscope}%
\begin{pgfscope}%
\pgfsys@transformshift{4.464594in}{1.877772in}%
\pgfsys@useobject{currentmarker}{}%
\end{pgfscope}%
\begin{pgfscope}%
\pgfsys@transformshift{4.413281in}{1.905331in}%
\pgfsys@useobject{currentmarker}{}%
\end{pgfscope}%
\begin{pgfscope}%
\pgfsys@transformshift{4.358244in}{1.935502in}%
\pgfsys@useobject{currentmarker}{}%
\end{pgfscope}%
\begin{pgfscope}%
\pgfsys@transformshift{4.299612in}{1.968298in}%
\pgfsys@useobject{currentmarker}{}%
\end{pgfscope}%
\begin{pgfscope}%
\pgfsys@transformshift{4.237521in}{2.003706in}%
\pgfsys@useobject{currentmarker}{}%
\end{pgfscope}%
\begin{pgfscope}%
\pgfsys@transformshift{4.172116in}{2.041670in}%
\pgfsys@useobject{currentmarker}{}%
\end{pgfscope}%
\begin{pgfscope}%
\pgfsys@transformshift{4.103550in}{2.082084in}%
\pgfsys@useobject{currentmarker}{}%
\end{pgfscope}%
\begin{pgfscope}%
\pgfsys@transformshift{4.031984in}{2.124775in}%
\pgfsys@useobject{currentmarker}{}%
\end{pgfscope}%
\begin{pgfscope}%
\pgfsys@transformshift{3.957584in}{2.169490in}%
\pgfsys@useobject{currentmarker}{}%
\end{pgfscope}%
\begin{pgfscope}%
\pgfsys@transformshift{3.880524in}{2.215880in}%
\pgfsys@useobject{currentmarker}{}%
\end{pgfscope}%
\begin{pgfscope}%
\pgfsys@transformshift{3.800985in}{2.263486in}%
\pgfsys@useobject{currentmarker}{}%
\end{pgfscope}%
\begin{pgfscope}%
\pgfsys@transformshift{3.719151in}{2.311728in}%
\pgfsys@useobject{currentmarker}{}%
\end{pgfscope}%
\begin{pgfscope}%
\pgfsys@transformshift{3.635215in}{2.359896in}%
\pgfsys@useobject{currentmarker}{}%
\end{pgfscope}%
\begin{pgfscope}%
\pgfsys@transformshift{3.549371in}{2.407154in}%
\pgfsys@useobject{currentmarker}{}%
\end{pgfscope}%
\begin{pgfscope}%
\pgfsys@transformshift{3.461821in}{2.452548in}%
\pgfsys@useobject{currentmarker}{}%
\end{pgfscope}%
\begin{pgfscope}%
\pgfsys@transformshift{3.372770in}{2.495033in}%
\pgfsys@useobject{currentmarker}{}%
\end{pgfscope}%
\begin{pgfscope}%
\pgfsys@transformshift{3.282424in}{2.533509in}%
\pgfsys@useobject{currentmarker}{}%
\end{pgfscope}%
\begin{pgfscope}%
\pgfsys@transformshift{3.190996in}{2.566874in}%
\pgfsys@useobject{currentmarker}{}%
\end{pgfscope}%
\begin{pgfscope}%
\pgfsys@transformshift{3.098698in}{2.594091in}%
\pgfsys@useobject{currentmarker}{}%
\end{pgfscope}%
\begin{pgfscope}%
\pgfsys@transformshift{3.005746in}{2.614252in}%
\pgfsys@useobject{currentmarker}{}%
\end{pgfscope}%
\begin{pgfscope}%
\pgfsys@transformshift{2.912357in}{2.626649in}%
\pgfsys@useobject{currentmarker}{}%
\end{pgfscope}%
\begin{pgfscope}%
\pgfsys@transformshift{2.818750in}{2.630833in}%
\pgfsys@useobject{currentmarker}{}%
\end{pgfscope}%
\begin{pgfscope}%
\pgfsys@transformshift{2.725143in}{2.626649in}%
\pgfsys@useobject{currentmarker}{}%
\end{pgfscope}%
\begin{pgfscope}%
\pgfsys@transformshift{2.631754in}{2.614252in}%
\pgfsys@useobject{currentmarker}{}%
\end{pgfscope}%
\begin{pgfscope}%
\pgfsys@transformshift{2.538802in}{2.594091in}%
\pgfsys@useobject{currentmarker}{}%
\end{pgfscope}%
\begin{pgfscope}%
\pgfsys@transformshift{2.446504in}{2.566874in}%
\pgfsys@useobject{currentmarker}{}%
\end{pgfscope}%
\begin{pgfscope}%
\pgfsys@transformshift{2.355076in}{2.533509in}%
\pgfsys@useobject{currentmarker}{}%
\end{pgfscope}%
\begin{pgfscope}%
\pgfsys@transformshift{2.264730in}{2.495033in}%
\pgfsys@useobject{currentmarker}{}%
\end{pgfscope}%
\begin{pgfscope}%
\pgfsys@transformshift{2.175679in}{2.452548in}%
\pgfsys@useobject{currentmarker}{}%
\end{pgfscope}%
\begin{pgfscope}%
\pgfsys@transformshift{2.088129in}{2.407154in}%
\pgfsys@useobject{currentmarker}{}%
\end{pgfscope}%
\begin{pgfscope}%
\pgfsys@transformshift{2.002285in}{2.359896in}%
\pgfsys@useobject{currentmarker}{}%
\end{pgfscope}%
\begin{pgfscope}%
\pgfsys@transformshift{1.918349in}{2.311728in}%
\pgfsys@useobject{currentmarker}{}%
\end{pgfscope}%
\begin{pgfscope}%
\pgfsys@transformshift{1.836515in}{2.263486in}%
\pgfsys@useobject{currentmarker}{}%
\end{pgfscope}%
\begin{pgfscope}%
\pgfsys@transformshift{1.756976in}{2.215880in}%
\pgfsys@useobject{currentmarker}{}%
\end{pgfscope}%
\begin{pgfscope}%
\pgfsys@transformshift{1.679916in}{2.169490in}%
\pgfsys@useobject{currentmarker}{}%
\end{pgfscope}%
\begin{pgfscope}%
\pgfsys@transformshift{1.605516in}{2.124775in}%
\pgfsys@useobject{currentmarker}{}%
\end{pgfscope}%
\begin{pgfscope}%
\pgfsys@transformshift{1.533950in}{2.082084in}%
\pgfsys@useobject{currentmarker}{}%
\end{pgfscope}%
\begin{pgfscope}%
\pgfsys@transformshift{1.465384in}{2.041670in}%
\pgfsys@useobject{currentmarker}{}%
\end{pgfscope}%
\begin{pgfscope}%
\pgfsys@transformshift{1.399979in}{2.003706in}%
\pgfsys@useobject{currentmarker}{}%
\end{pgfscope}%
\begin{pgfscope}%
\pgfsys@transformshift{1.337888in}{1.968298in}%
\pgfsys@useobject{currentmarker}{}%
\end{pgfscope}%
\begin{pgfscope}%
\pgfsys@transformshift{1.279256in}{1.935502in}%
\pgfsys@useobject{currentmarker}{}%
\end{pgfscope}%
\begin{pgfscope}%
\pgfsys@transformshift{1.224219in}{1.905331in}%
\pgfsys@useobject{currentmarker}{}%
\end{pgfscope}%
\begin{pgfscope}%
\pgfsys@transformshift{1.172906in}{1.877772in}%
\pgfsys@useobject{currentmarker}{}%
\end{pgfscope}%
\begin{pgfscope}%
\pgfsys@transformshift{1.125437in}{1.852789in}%
\pgfsys@useobject{currentmarker}{}%
\end{pgfscope}%
\begin{pgfscope}%
\pgfsys@transformshift{1.081923in}{1.830333in}%
\pgfsys@useobject{currentmarker}{}%
\end{pgfscope}%
\begin{pgfscope}%
\pgfsys@transformshift{1.042465in}{1.810346in}%
\pgfsys@useobject{currentmarker}{}%
\end{pgfscope}%
\begin{pgfscope}%
\pgfsys@transformshift{1.007156in}{1.792770in}%
\pgfsys@useobject{currentmarker}{}%
\end{pgfscope}%
\begin{pgfscope}%
\pgfsys@transformshift{0.976078in}{1.777542in}%
\pgfsys@useobject{currentmarker}{}%
\end{pgfscope}%
\begin{pgfscope}%
\pgfsys@transformshift{0.949304in}{1.764607in}%
\pgfsys@useobject{currentmarker}{}%
\end{pgfscope}%
\begin{pgfscope}%
\pgfsys@transformshift{0.926895in}{1.753913in}%
\pgfsys@useobject{currentmarker}{}%
\end{pgfscope}%
\begin{pgfscope}%
\pgfsys@transformshift{0.908906in}{1.745414in}%
\pgfsys@useobject{currentmarker}{}%
\end{pgfscope}%
\begin{pgfscope}%
\pgfsys@transformshift{0.895377in}{1.739074in}%
\pgfsys@useobject{currentmarker}{}%
\end{pgfscope}%
\begin{pgfscope}%
\pgfsys@transformshift{0.886340in}{1.734863in}%
\pgfsys@useobject{currentmarker}{}%
\end{pgfscope}%
\begin{pgfscope}%
\pgfsys@transformshift{0.881816in}{1.732762in}%
\pgfsys@useobject{currentmarker}{}%
\end{pgfscope}%
\end{pgfscope}%
\begin{pgfscope}%
\pgfpathrectangle{\pgfqpoint{0.687500in}{0.385000in}}{\pgfqpoint{4.262500in}{2.695000in}}%
\pgfusepath{clip}%
\pgfsetrectcap%
\pgfsetroundjoin%
\pgfsetlinewidth{1.505625pt}%
\definecolor{currentstroke}{rgb}{0.580392,0.403922,0.741176}%
\pgfsetstrokecolor{currentstroke}%
\pgfsetdash{}{0pt}%
\pgfpathmoveto{\pgfqpoint{0.881250in}{1.732500in}}%
\pgfpathlineto{\pgfqpoint{0.978612in}{1.778775in}}%
\pgfpathlineto{\pgfqpoint{1.075974in}{1.827296in}}%
\pgfpathlineto{\pgfqpoint{1.173335in}{1.878000in}}%
\pgfpathlineto{\pgfqpoint{1.290170in}{1.941556in}}%
\pgfpathlineto{\pgfqpoint{1.407004in}{2.007752in}}%
\pgfpathlineto{\pgfqpoint{1.543310in}{2.087643in}}%
\pgfpathlineto{\pgfqpoint{1.796451in}{2.239567in}}%
\pgfpathlineto{\pgfqpoint{1.932758in}{2.320107in}}%
\pgfpathlineto{\pgfqpoint{2.030119in}{2.375473in}}%
\pgfpathlineto{\pgfqpoint{2.108009in}{2.417737in}}%
\pgfpathlineto{\pgfqpoint{2.185898in}{2.457628in}}%
\pgfpathlineto{\pgfqpoint{2.263788in}{2.494605in}}%
\pgfpathlineto{\pgfqpoint{2.322205in}{2.520101in}}%
\pgfpathlineto{\pgfqpoint{2.380622in}{2.543430in}}%
\pgfpathlineto{\pgfqpoint{2.439039in}{2.564379in}}%
\pgfpathlineto{\pgfqpoint{2.497456in}{2.582749in}}%
\pgfpathlineto{\pgfqpoint{2.555873in}{2.598357in}}%
\pgfpathlineto{\pgfqpoint{2.614290in}{2.611046in}}%
\pgfpathlineto{\pgfqpoint{2.672707in}{2.620683in}}%
\pgfpathlineto{\pgfqpoint{2.731124in}{2.627166in}}%
\pgfpathlineto{\pgfqpoint{2.789541in}{2.630425in}}%
\pgfpathlineto{\pgfqpoint{2.847959in}{2.630425in}}%
\pgfpathlineto{\pgfqpoint{2.906376in}{2.627166in}}%
\pgfpathlineto{\pgfqpoint{2.964793in}{2.620683in}}%
\pgfpathlineto{\pgfqpoint{3.023210in}{2.611046in}}%
\pgfpathlineto{\pgfqpoint{3.081627in}{2.598357in}}%
\pgfpathlineto{\pgfqpoint{3.140044in}{2.582749in}}%
\pgfpathlineto{\pgfqpoint{3.198461in}{2.564379in}}%
\pgfpathlineto{\pgfqpoint{3.256878in}{2.543430in}}%
\pgfpathlineto{\pgfqpoint{3.315295in}{2.520101in}}%
\pgfpathlineto{\pgfqpoint{3.373712in}{2.494605in}}%
\pgfpathlineto{\pgfqpoint{3.451602in}{2.457628in}}%
\pgfpathlineto{\pgfqpoint{3.529491in}{2.417737in}}%
\pgfpathlineto{\pgfqpoint{3.607381in}{2.375473in}}%
\pgfpathlineto{\pgfqpoint{3.704742in}{2.320107in}}%
\pgfpathlineto{\pgfqpoint{3.841049in}{2.239567in}}%
\pgfpathlineto{\pgfqpoint{4.211024in}{2.019009in}}%
\pgfpathlineto{\pgfqpoint{4.327858in}{1.952417in}}%
\pgfpathlineto{\pgfqpoint{4.444692in}{1.888394in}}%
\pgfpathlineto{\pgfqpoint{4.542054in}{1.837265in}}%
\pgfpathlineto{\pgfqpoint{4.639416in}{1.788301in}}%
\pgfpathlineto{\pgfqpoint{4.736778in}{1.741574in}}%
\pgfpathlineto{\pgfqpoint{4.756250in}{1.732500in}}%
\pgfpathlineto{\pgfqpoint{4.756250in}{1.732500in}}%
\pgfusepath{stroke}%
\end{pgfscope}%
\begin{pgfscope}%
\pgfsetrectcap%
\pgfsetmiterjoin%
\pgfsetlinewidth{0.803000pt}%
\definecolor{currentstroke}{rgb}{0.000000,0.000000,0.000000}%
\pgfsetstrokecolor{currentstroke}%
\pgfsetdash{}{0pt}%
\pgfpathmoveto{\pgfqpoint{0.687500in}{0.385000in}}%
\pgfpathlineto{\pgfqpoint{0.687500in}{3.080000in}}%
\pgfusepath{stroke}%
\end{pgfscope}%
\begin{pgfscope}%
\pgfsetrectcap%
\pgfsetmiterjoin%
\pgfsetlinewidth{0.803000pt}%
\definecolor{currentstroke}{rgb}{0.000000,0.000000,0.000000}%
\pgfsetstrokecolor{currentstroke}%
\pgfsetdash{}{0pt}%
\pgfpathmoveto{\pgfqpoint{4.950000in}{0.385000in}}%
\pgfpathlineto{\pgfqpoint{4.950000in}{3.080000in}}%
\pgfusepath{stroke}%
\end{pgfscope}%
\begin{pgfscope}%
\pgfsetrectcap%
\pgfsetmiterjoin%
\pgfsetlinewidth{0.803000pt}%
\definecolor{currentstroke}{rgb}{0.000000,0.000000,0.000000}%
\pgfsetstrokecolor{currentstroke}%
\pgfsetdash{}{0pt}%
\pgfpathmoveto{\pgfqpoint{0.687500in}{0.385000in}}%
\pgfpathlineto{\pgfqpoint{4.950000in}{0.385000in}}%
\pgfusepath{stroke}%
\end{pgfscope}%
\begin{pgfscope}%
\pgfsetrectcap%
\pgfsetmiterjoin%
\pgfsetlinewidth{0.803000pt}%
\definecolor{currentstroke}{rgb}{0.000000,0.000000,0.000000}%
\pgfsetstrokecolor{currentstroke}%
\pgfsetdash{}{0pt}%
\pgfpathmoveto{\pgfqpoint{0.687500in}{3.080000in}}%
\pgfpathlineto{\pgfqpoint{4.950000in}{3.080000in}}%
\pgfusepath{stroke}%
\end{pgfscope}%
\begin{pgfscope}%
\definecolor{textcolor}{rgb}{0.000000,0.000000,0.000000}%
\pgfsetstrokecolor{textcolor}%
\pgfsetfillcolor{textcolor}%
\pgftext[x=2.818750in,y=3.163333in,,base]{\color{textcolor}\rmfamily\fontsize{12.000000}{14.400000}\selectfont N=6, 64}%
\end{pgfscope}%
\begin{pgfscope}%
\pgfsetbuttcap%
\pgfsetmiterjoin%
\definecolor{currentfill}{rgb}{1.000000,1.000000,1.000000}%
\pgfsetfillcolor{currentfill}%
\pgfsetfillopacity{0.800000}%
\pgfsetlinewidth{1.003750pt}%
\definecolor{currentstroke}{rgb}{0.800000,0.800000,0.800000}%
\pgfsetstrokecolor{currentstroke}%
\pgfsetstrokeopacity{0.800000}%
\pgfsetdash{}{0pt}%
\pgfpathmoveto{\pgfqpoint{0.784722in}{0.454444in}}%
\pgfpathlineto{\pgfqpoint{2.050468in}{0.454444in}}%
\pgfpathquadraticcurveto{\pgfqpoint{2.078246in}{0.454444in}}{\pgfqpoint{2.078246in}{0.482222in}}%
\pgfpathlineto{\pgfqpoint{2.078246in}{1.244755in}}%
\pgfpathquadraticcurveto{\pgfqpoint{2.078246in}{1.272533in}}{\pgfqpoint{2.050468in}{1.272533in}}%
\pgfpathlineto{\pgfqpoint{0.784722in}{1.272533in}}%
\pgfpathquadraticcurveto{\pgfqpoint{0.756944in}{1.272533in}}{\pgfqpoint{0.756944in}{1.244755in}}%
\pgfpathlineto{\pgfqpoint{0.756944in}{0.482222in}}%
\pgfpathquadraticcurveto{\pgfqpoint{0.756944in}{0.454444in}}{\pgfqpoint{0.784722in}{0.454444in}}%
\pgfpathclose%
\pgfusepath{stroke,fill}%
\end{pgfscope}%
\begin{pgfscope}%
\pgfsetrectcap%
\pgfsetroundjoin%
\pgfsetlinewidth{1.505625pt}%
\definecolor{currentstroke}{rgb}{0.121569,0.466667,0.705882}%
\pgfsetstrokecolor{currentstroke}%
\pgfsetdash{}{0pt}%
\pgfpathmoveto{\pgfqpoint{0.812500in}{1.082123in}}%
\pgfpathlineto{\pgfqpoint{1.090278in}{1.082123in}}%
\pgfusepath{stroke}%
\end{pgfscope}%
\begin{pgfscope}%
\definecolor{textcolor}{rgb}{0.000000,0.000000,0.000000}%
\pgfsetstrokecolor{textcolor}%
\pgfsetfillcolor{textcolor}%
\pgftext[x=1.201389in,y=1.033512in,left,base]{\color{textcolor}\rmfamily\fontsize{10.000000}{12.000000}\selectfont \(\displaystyle y(x)=\)\(\displaystyle \frac{1}{1+x^{2}}\)}%
\end{pgfscope}%
\begin{pgfscope}%
\pgfsetrectcap%
\pgfsetroundjoin%
\pgfsetlinewidth{1.505625pt}%
\definecolor{currentstroke}{rgb}{0.172549,0.627451,0.172549}%
\pgfsetstrokecolor{currentstroke}%
\pgfsetdash{}{0pt}%
\pgfpathmoveto{\pgfqpoint{0.812500in}{0.801667in}}%
\pgfpathlineto{\pgfqpoint{1.090278in}{0.801667in}}%
\pgfusepath{stroke}%
\end{pgfscope}%
\begin{pgfscope}%
\definecolor{textcolor}{rgb}{0.000000,0.000000,0.000000}%
\pgfsetstrokecolor{textcolor}%
\pgfsetfillcolor{textcolor}%
\pgftext[x=1.201389in,y=0.753056in,left,base]{\color{textcolor}\rmfamily\fontsize{10.000000}{12.000000}\selectfont W6(x)}%
\end{pgfscope}%
\begin{pgfscope}%
\pgfsetrectcap%
\pgfsetroundjoin%
\pgfsetlinewidth{1.505625pt}%
\definecolor{currentstroke}{rgb}{0.580392,0.403922,0.741176}%
\pgfsetstrokecolor{currentstroke}%
\pgfsetdash{}{0pt}%
\pgfpathmoveto{\pgfqpoint{0.812500in}{0.593333in}}%
\pgfpathlineto{\pgfqpoint{1.090278in}{0.593333in}}%
\pgfusepath{stroke}%
\end{pgfscope}%
\begin{pgfscope}%
\definecolor{textcolor}{rgb}{0.000000,0.000000,0.000000}%
\pgfsetstrokecolor{textcolor}%
\pgfsetfillcolor{textcolor}%
\pgftext[x=1.201389in,y=0.544722in,left,base]{\color{textcolor}\rmfamily\fontsize{10.000000}{12.000000}\selectfont W64(x)}%
\end{pgfscope}%
\end{pgfpicture}%
\makeatother%
\endgroup%
        
    \end{center}
    \caption{Węzły cosinus, funkcja \(\tilde{y}\), \(N=6,64\)}
\end{figure}

\begin{figure}[h]
    \begin{center}
        %% Creator: Matplotlib, PGF backend
%%
%% To include the figure in your LaTeX document, write
%%   \input{<filename>.pgf}
%%
%% Make sure the required packages are loaded in your preamble
%%   \usepackage{pgf}
%%
%% Figures using additional raster images can only be included by \input if
%% they are in the same directory as the main LaTeX file. For loading figures
%% from other directories you can use the `import` package
%%   \usepackage{import}
%% and then include the figures with
%%   \import{<path to file>}{<filename>.pgf}
%%
%% Matplotlib used the following preamble
%%
\begingroup%
\makeatletter%
\begin{pgfpicture}%
\pgfpathrectangle{\pgfpointorigin}{\pgfqpoint{5.500000in}{3.500000in}}%
\pgfusepath{use as bounding box, clip}%
\begin{pgfscope}%
\pgfsetbuttcap%
\pgfsetmiterjoin%
\definecolor{currentfill}{rgb}{1.000000,1.000000,1.000000}%
\pgfsetfillcolor{currentfill}%
\pgfsetlinewidth{0.000000pt}%
\definecolor{currentstroke}{rgb}{1.000000,1.000000,1.000000}%
\pgfsetstrokecolor{currentstroke}%
\pgfsetdash{}{0pt}%
\pgfpathmoveto{\pgfqpoint{0.000000in}{0.000000in}}%
\pgfpathlineto{\pgfqpoint{5.500000in}{0.000000in}}%
\pgfpathlineto{\pgfqpoint{5.500000in}{3.500000in}}%
\pgfpathlineto{\pgfqpoint{0.000000in}{3.500000in}}%
\pgfpathclose%
\pgfusepath{fill}%
\end{pgfscope}%
\begin{pgfscope}%
\pgfsetbuttcap%
\pgfsetmiterjoin%
\definecolor{currentfill}{rgb}{1.000000,1.000000,1.000000}%
\pgfsetfillcolor{currentfill}%
\pgfsetlinewidth{0.000000pt}%
\definecolor{currentstroke}{rgb}{0.000000,0.000000,0.000000}%
\pgfsetstrokecolor{currentstroke}%
\pgfsetstrokeopacity{0.000000}%
\pgfsetdash{}{0pt}%
\pgfpathmoveto{\pgfqpoint{0.687500in}{0.385000in}}%
\pgfpathlineto{\pgfqpoint{4.950000in}{0.385000in}}%
\pgfpathlineto{\pgfqpoint{4.950000in}{3.080000in}}%
\pgfpathlineto{\pgfqpoint{0.687500in}{3.080000in}}%
\pgfpathclose%
\pgfusepath{fill}%
\end{pgfscope}%
\begin{pgfscope}%
\pgfsetbuttcap%
\pgfsetroundjoin%
\definecolor{currentfill}{rgb}{0.000000,0.000000,0.000000}%
\pgfsetfillcolor{currentfill}%
\pgfsetlinewidth{0.803000pt}%
\definecolor{currentstroke}{rgb}{0.000000,0.000000,0.000000}%
\pgfsetstrokecolor{currentstroke}%
\pgfsetdash{}{0pt}%
\pgfsys@defobject{currentmarker}{\pgfqpoint{0.000000in}{-0.048611in}}{\pgfqpoint{0.000000in}{0.000000in}}{%
\pgfpathmoveto{\pgfqpoint{0.000000in}{0.000000in}}%
\pgfpathlineto{\pgfqpoint{0.000000in}{-0.048611in}}%
\pgfusepath{stroke,fill}%
}%
\begin{pgfscope}%
\pgfsys@transformshift{0.881250in}{0.385000in}%
\pgfsys@useobject{currentmarker}{}%
\end{pgfscope}%
\end{pgfscope}%
\begin{pgfscope}%
\definecolor{textcolor}{rgb}{0.000000,0.000000,0.000000}%
\pgfsetstrokecolor{textcolor}%
\pgfsetfillcolor{textcolor}%
\pgftext[x=0.881250in,y=0.287778in,,top]{\color{textcolor}\rmfamily\fontsize{10.000000}{12.000000}\selectfont \(\displaystyle -1.00\)}%
\end{pgfscope}%
\begin{pgfscope}%
\pgfsetbuttcap%
\pgfsetroundjoin%
\definecolor{currentfill}{rgb}{0.000000,0.000000,0.000000}%
\pgfsetfillcolor{currentfill}%
\pgfsetlinewidth{0.803000pt}%
\definecolor{currentstroke}{rgb}{0.000000,0.000000,0.000000}%
\pgfsetstrokecolor{currentstroke}%
\pgfsetdash{}{0pt}%
\pgfsys@defobject{currentmarker}{\pgfqpoint{0.000000in}{-0.048611in}}{\pgfqpoint{0.000000in}{0.000000in}}{%
\pgfpathmoveto{\pgfqpoint{0.000000in}{0.000000in}}%
\pgfpathlineto{\pgfqpoint{0.000000in}{-0.048611in}}%
\pgfusepath{stroke,fill}%
}%
\begin{pgfscope}%
\pgfsys@transformshift{1.365625in}{0.385000in}%
\pgfsys@useobject{currentmarker}{}%
\end{pgfscope}%
\end{pgfscope}%
\begin{pgfscope}%
\definecolor{textcolor}{rgb}{0.000000,0.000000,0.000000}%
\pgfsetstrokecolor{textcolor}%
\pgfsetfillcolor{textcolor}%
\pgftext[x=1.365625in,y=0.287778in,,top]{\color{textcolor}\rmfamily\fontsize{10.000000}{12.000000}\selectfont \(\displaystyle -0.75\)}%
\end{pgfscope}%
\begin{pgfscope}%
\pgfsetbuttcap%
\pgfsetroundjoin%
\definecolor{currentfill}{rgb}{0.000000,0.000000,0.000000}%
\pgfsetfillcolor{currentfill}%
\pgfsetlinewidth{0.803000pt}%
\definecolor{currentstroke}{rgb}{0.000000,0.000000,0.000000}%
\pgfsetstrokecolor{currentstroke}%
\pgfsetdash{}{0pt}%
\pgfsys@defobject{currentmarker}{\pgfqpoint{0.000000in}{-0.048611in}}{\pgfqpoint{0.000000in}{0.000000in}}{%
\pgfpathmoveto{\pgfqpoint{0.000000in}{0.000000in}}%
\pgfpathlineto{\pgfqpoint{0.000000in}{-0.048611in}}%
\pgfusepath{stroke,fill}%
}%
\begin{pgfscope}%
\pgfsys@transformshift{1.850000in}{0.385000in}%
\pgfsys@useobject{currentmarker}{}%
\end{pgfscope}%
\end{pgfscope}%
\begin{pgfscope}%
\definecolor{textcolor}{rgb}{0.000000,0.000000,0.000000}%
\pgfsetstrokecolor{textcolor}%
\pgfsetfillcolor{textcolor}%
\pgftext[x=1.850000in,y=0.287778in,,top]{\color{textcolor}\rmfamily\fontsize{10.000000}{12.000000}\selectfont \(\displaystyle -0.50\)}%
\end{pgfscope}%
\begin{pgfscope}%
\pgfsetbuttcap%
\pgfsetroundjoin%
\definecolor{currentfill}{rgb}{0.000000,0.000000,0.000000}%
\pgfsetfillcolor{currentfill}%
\pgfsetlinewidth{0.803000pt}%
\definecolor{currentstroke}{rgb}{0.000000,0.000000,0.000000}%
\pgfsetstrokecolor{currentstroke}%
\pgfsetdash{}{0pt}%
\pgfsys@defobject{currentmarker}{\pgfqpoint{0.000000in}{-0.048611in}}{\pgfqpoint{0.000000in}{0.000000in}}{%
\pgfpathmoveto{\pgfqpoint{0.000000in}{0.000000in}}%
\pgfpathlineto{\pgfqpoint{0.000000in}{-0.048611in}}%
\pgfusepath{stroke,fill}%
}%
\begin{pgfscope}%
\pgfsys@transformshift{2.334375in}{0.385000in}%
\pgfsys@useobject{currentmarker}{}%
\end{pgfscope}%
\end{pgfscope}%
\begin{pgfscope}%
\definecolor{textcolor}{rgb}{0.000000,0.000000,0.000000}%
\pgfsetstrokecolor{textcolor}%
\pgfsetfillcolor{textcolor}%
\pgftext[x=2.334375in,y=0.287778in,,top]{\color{textcolor}\rmfamily\fontsize{10.000000}{12.000000}\selectfont \(\displaystyle -0.25\)}%
\end{pgfscope}%
\begin{pgfscope}%
\pgfsetbuttcap%
\pgfsetroundjoin%
\definecolor{currentfill}{rgb}{0.000000,0.000000,0.000000}%
\pgfsetfillcolor{currentfill}%
\pgfsetlinewidth{0.803000pt}%
\definecolor{currentstroke}{rgb}{0.000000,0.000000,0.000000}%
\pgfsetstrokecolor{currentstroke}%
\pgfsetdash{}{0pt}%
\pgfsys@defobject{currentmarker}{\pgfqpoint{0.000000in}{-0.048611in}}{\pgfqpoint{0.000000in}{0.000000in}}{%
\pgfpathmoveto{\pgfqpoint{0.000000in}{0.000000in}}%
\pgfpathlineto{\pgfqpoint{0.000000in}{-0.048611in}}%
\pgfusepath{stroke,fill}%
}%
\begin{pgfscope}%
\pgfsys@transformshift{2.818750in}{0.385000in}%
\pgfsys@useobject{currentmarker}{}%
\end{pgfscope}%
\end{pgfscope}%
\begin{pgfscope}%
\definecolor{textcolor}{rgb}{0.000000,0.000000,0.000000}%
\pgfsetstrokecolor{textcolor}%
\pgfsetfillcolor{textcolor}%
\pgftext[x=2.818750in,y=0.287778in,,top]{\color{textcolor}\rmfamily\fontsize{10.000000}{12.000000}\selectfont \(\displaystyle 0.00\)}%
\end{pgfscope}%
\begin{pgfscope}%
\pgfsetbuttcap%
\pgfsetroundjoin%
\definecolor{currentfill}{rgb}{0.000000,0.000000,0.000000}%
\pgfsetfillcolor{currentfill}%
\pgfsetlinewidth{0.803000pt}%
\definecolor{currentstroke}{rgb}{0.000000,0.000000,0.000000}%
\pgfsetstrokecolor{currentstroke}%
\pgfsetdash{}{0pt}%
\pgfsys@defobject{currentmarker}{\pgfqpoint{0.000000in}{-0.048611in}}{\pgfqpoint{0.000000in}{0.000000in}}{%
\pgfpathmoveto{\pgfqpoint{0.000000in}{0.000000in}}%
\pgfpathlineto{\pgfqpoint{0.000000in}{-0.048611in}}%
\pgfusepath{stroke,fill}%
}%
\begin{pgfscope}%
\pgfsys@transformshift{3.303125in}{0.385000in}%
\pgfsys@useobject{currentmarker}{}%
\end{pgfscope}%
\end{pgfscope}%
\begin{pgfscope}%
\definecolor{textcolor}{rgb}{0.000000,0.000000,0.000000}%
\pgfsetstrokecolor{textcolor}%
\pgfsetfillcolor{textcolor}%
\pgftext[x=3.303125in,y=0.287778in,,top]{\color{textcolor}\rmfamily\fontsize{10.000000}{12.000000}\selectfont \(\displaystyle 0.25\)}%
\end{pgfscope}%
\begin{pgfscope}%
\pgfsetbuttcap%
\pgfsetroundjoin%
\definecolor{currentfill}{rgb}{0.000000,0.000000,0.000000}%
\pgfsetfillcolor{currentfill}%
\pgfsetlinewidth{0.803000pt}%
\definecolor{currentstroke}{rgb}{0.000000,0.000000,0.000000}%
\pgfsetstrokecolor{currentstroke}%
\pgfsetdash{}{0pt}%
\pgfsys@defobject{currentmarker}{\pgfqpoint{0.000000in}{-0.048611in}}{\pgfqpoint{0.000000in}{0.000000in}}{%
\pgfpathmoveto{\pgfqpoint{0.000000in}{0.000000in}}%
\pgfpathlineto{\pgfqpoint{0.000000in}{-0.048611in}}%
\pgfusepath{stroke,fill}%
}%
\begin{pgfscope}%
\pgfsys@transformshift{3.787500in}{0.385000in}%
\pgfsys@useobject{currentmarker}{}%
\end{pgfscope}%
\end{pgfscope}%
\begin{pgfscope}%
\definecolor{textcolor}{rgb}{0.000000,0.000000,0.000000}%
\pgfsetstrokecolor{textcolor}%
\pgfsetfillcolor{textcolor}%
\pgftext[x=3.787500in,y=0.287778in,,top]{\color{textcolor}\rmfamily\fontsize{10.000000}{12.000000}\selectfont \(\displaystyle 0.50\)}%
\end{pgfscope}%
\begin{pgfscope}%
\pgfsetbuttcap%
\pgfsetroundjoin%
\definecolor{currentfill}{rgb}{0.000000,0.000000,0.000000}%
\pgfsetfillcolor{currentfill}%
\pgfsetlinewidth{0.803000pt}%
\definecolor{currentstroke}{rgb}{0.000000,0.000000,0.000000}%
\pgfsetstrokecolor{currentstroke}%
\pgfsetdash{}{0pt}%
\pgfsys@defobject{currentmarker}{\pgfqpoint{0.000000in}{-0.048611in}}{\pgfqpoint{0.000000in}{0.000000in}}{%
\pgfpathmoveto{\pgfqpoint{0.000000in}{0.000000in}}%
\pgfpathlineto{\pgfqpoint{0.000000in}{-0.048611in}}%
\pgfusepath{stroke,fill}%
}%
\begin{pgfscope}%
\pgfsys@transformshift{4.271875in}{0.385000in}%
\pgfsys@useobject{currentmarker}{}%
\end{pgfscope}%
\end{pgfscope}%
\begin{pgfscope}%
\definecolor{textcolor}{rgb}{0.000000,0.000000,0.000000}%
\pgfsetstrokecolor{textcolor}%
\pgfsetfillcolor{textcolor}%
\pgftext[x=4.271875in,y=0.287778in,,top]{\color{textcolor}\rmfamily\fontsize{10.000000}{12.000000}\selectfont \(\displaystyle 0.75\)}%
\end{pgfscope}%
\begin{pgfscope}%
\pgfsetbuttcap%
\pgfsetroundjoin%
\definecolor{currentfill}{rgb}{0.000000,0.000000,0.000000}%
\pgfsetfillcolor{currentfill}%
\pgfsetlinewidth{0.803000pt}%
\definecolor{currentstroke}{rgb}{0.000000,0.000000,0.000000}%
\pgfsetstrokecolor{currentstroke}%
\pgfsetdash{}{0pt}%
\pgfsys@defobject{currentmarker}{\pgfqpoint{0.000000in}{-0.048611in}}{\pgfqpoint{0.000000in}{0.000000in}}{%
\pgfpathmoveto{\pgfqpoint{0.000000in}{0.000000in}}%
\pgfpathlineto{\pgfqpoint{0.000000in}{-0.048611in}}%
\pgfusepath{stroke,fill}%
}%
\begin{pgfscope}%
\pgfsys@transformshift{4.756250in}{0.385000in}%
\pgfsys@useobject{currentmarker}{}%
\end{pgfscope}%
\end{pgfscope}%
\begin{pgfscope}%
\definecolor{textcolor}{rgb}{0.000000,0.000000,0.000000}%
\pgfsetstrokecolor{textcolor}%
\pgfsetfillcolor{textcolor}%
\pgftext[x=4.756250in,y=0.287778in,,top]{\color{textcolor}\rmfamily\fontsize{10.000000}{12.000000}\selectfont \(\displaystyle 1.00\)}%
\end{pgfscope}%
\begin{pgfscope}%
\definecolor{textcolor}{rgb}{0.000000,0.000000,0.000000}%
\pgfsetstrokecolor{textcolor}%
\pgfsetfillcolor{textcolor}%
\pgftext[x=2.818750in,y=0.108766in,,top]{\color{textcolor}\rmfamily\fontsize{10.000000}{12.000000}\selectfont x}%
\end{pgfscope}%
\begin{pgfscope}%
\pgfsetbuttcap%
\pgfsetroundjoin%
\definecolor{currentfill}{rgb}{0.000000,0.000000,0.000000}%
\pgfsetfillcolor{currentfill}%
\pgfsetlinewidth{0.803000pt}%
\definecolor{currentstroke}{rgb}{0.000000,0.000000,0.000000}%
\pgfsetstrokecolor{currentstroke}%
\pgfsetdash{}{0pt}%
\pgfsys@defobject{currentmarker}{\pgfqpoint{-0.048611in}{0.000000in}}{\pgfqpoint{0.000000in}{0.000000in}}{%
\pgfpathmoveto{\pgfqpoint{0.000000in}{0.000000in}}%
\pgfpathlineto{\pgfqpoint{-0.048611in}{0.000000in}}%
\pgfusepath{stroke,fill}%
}%
\begin{pgfscope}%
\pgfsys@transformshift{0.687500in}{0.474833in}%
\pgfsys@useobject{currentmarker}{}%
\end{pgfscope}%
\end{pgfscope}%
\begin{pgfscope}%
\definecolor{textcolor}{rgb}{0.000000,0.000000,0.000000}%
\pgfsetstrokecolor{textcolor}%
\pgfsetfillcolor{textcolor}%
\pgftext[x=0.304783in,y=0.426608in,left,base]{\color{textcolor}\rmfamily\fontsize{10.000000}{12.000000}\selectfont \(\displaystyle -0.2\)}%
\end{pgfscope}%
\begin{pgfscope}%
\pgfsetbuttcap%
\pgfsetroundjoin%
\definecolor{currentfill}{rgb}{0.000000,0.000000,0.000000}%
\pgfsetfillcolor{currentfill}%
\pgfsetlinewidth{0.803000pt}%
\definecolor{currentstroke}{rgb}{0.000000,0.000000,0.000000}%
\pgfsetstrokecolor{currentstroke}%
\pgfsetdash{}{0pt}%
\pgfsys@defobject{currentmarker}{\pgfqpoint{-0.048611in}{0.000000in}}{\pgfqpoint{0.000000in}{0.000000in}}{%
\pgfpathmoveto{\pgfqpoint{0.000000in}{0.000000in}}%
\pgfpathlineto{\pgfqpoint{-0.048611in}{0.000000in}}%
\pgfusepath{stroke,fill}%
}%
\begin{pgfscope}%
\pgfsys@transformshift{0.687500in}{0.834167in}%
\pgfsys@useobject{currentmarker}{}%
\end{pgfscope}%
\end{pgfscope}%
\begin{pgfscope}%
\definecolor{textcolor}{rgb}{0.000000,0.000000,0.000000}%
\pgfsetstrokecolor{textcolor}%
\pgfsetfillcolor{textcolor}%
\pgftext[x=0.412808in,y=0.785941in,left,base]{\color{textcolor}\rmfamily\fontsize{10.000000}{12.000000}\selectfont \(\displaystyle 0.0\)}%
\end{pgfscope}%
\begin{pgfscope}%
\pgfsetbuttcap%
\pgfsetroundjoin%
\definecolor{currentfill}{rgb}{0.000000,0.000000,0.000000}%
\pgfsetfillcolor{currentfill}%
\pgfsetlinewidth{0.803000pt}%
\definecolor{currentstroke}{rgb}{0.000000,0.000000,0.000000}%
\pgfsetstrokecolor{currentstroke}%
\pgfsetdash{}{0pt}%
\pgfsys@defobject{currentmarker}{\pgfqpoint{-0.048611in}{0.000000in}}{\pgfqpoint{0.000000in}{0.000000in}}{%
\pgfpathmoveto{\pgfqpoint{0.000000in}{0.000000in}}%
\pgfpathlineto{\pgfqpoint{-0.048611in}{0.000000in}}%
\pgfusepath{stroke,fill}%
}%
\begin{pgfscope}%
\pgfsys@transformshift{0.687500in}{1.193500in}%
\pgfsys@useobject{currentmarker}{}%
\end{pgfscope}%
\end{pgfscope}%
\begin{pgfscope}%
\definecolor{textcolor}{rgb}{0.000000,0.000000,0.000000}%
\pgfsetstrokecolor{textcolor}%
\pgfsetfillcolor{textcolor}%
\pgftext[x=0.412808in,y=1.145275in,left,base]{\color{textcolor}\rmfamily\fontsize{10.000000}{12.000000}\selectfont \(\displaystyle 0.2\)}%
\end{pgfscope}%
\begin{pgfscope}%
\pgfsetbuttcap%
\pgfsetroundjoin%
\definecolor{currentfill}{rgb}{0.000000,0.000000,0.000000}%
\pgfsetfillcolor{currentfill}%
\pgfsetlinewidth{0.803000pt}%
\definecolor{currentstroke}{rgb}{0.000000,0.000000,0.000000}%
\pgfsetstrokecolor{currentstroke}%
\pgfsetdash{}{0pt}%
\pgfsys@defobject{currentmarker}{\pgfqpoint{-0.048611in}{0.000000in}}{\pgfqpoint{0.000000in}{0.000000in}}{%
\pgfpathmoveto{\pgfqpoint{0.000000in}{0.000000in}}%
\pgfpathlineto{\pgfqpoint{-0.048611in}{0.000000in}}%
\pgfusepath{stroke,fill}%
}%
\begin{pgfscope}%
\pgfsys@transformshift{0.687500in}{1.552833in}%
\pgfsys@useobject{currentmarker}{}%
\end{pgfscope}%
\end{pgfscope}%
\begin{pgfscope}%
\definecolor{textcolor}{rgb}{0.000000,0.000000,0.000000}%
\pgfsetstrokecolor{textcolor}%
\pgfsetfillcolor{textcolor}%
\pgftext[x=0.412808in,y=1.504608in,left,base]{\color{textcolor}\rmfamily\fontsize{10.000000}{12.000000}\selectfont \(\displaystyle 0.4\)}%
\end{pgfscope}%
\begin{pgfscope}%
\pgfsetbuttcap%
\pgfsetroundjoin%
\definecolor{currentfill}{rgb}{0.000000,0.000000,0.000000}%
\pgfsetfillcolor{currentfill}%
\pgfsetlinewidth{0.803000pt}%
\definecolor{currentstroke}{rgb}{0.000000,0.000000,0.000000}%
\pgfsetstrokecolor{currentstroke}%
\pgfsetdash{}{0pt}%
\pgfsys@defobject{currentmarker}{\pgfqpoint{-0.048611in}{0.000000in}}{\pgfqpoint{0.000000in}{0.000000in}}{%
\pgfpathmoveto{\pgfqpoint{0.000000in}{0.000000in}}%
\pgfpathlineto{\pgfqpoint{-0.048611in}{0.000000in}}%
\pgfusepath{stroke,fill}%
}%
\begin{pgfscope}%
\pgfsys@transformshift{0.687500in}{1.912167in}%
\pgfsys@useobject{currentmarker}{}%
\end{pgfscope}%
\end{pgfscope}%
\begin{pgfscope}%
\definecolor{textcolor}{rgb}{0.000000,0.000000,0.000000}%
\pgfsetstrokecolor{textcolor}%
\pgfsetfillcolor{textcolor}%
\pgftext[x=0.412808in,y=1.863941in,left,base]{\color{textcolor}\rmfamily\fontsize{10.000000}{12.000000}\selectfont \(\displaystyle 0.6\)}%
\end{pgfscope}%
\begin{pgfscope}%
\pgfsetbuttcap%
\pgfsetroundjoin%
\definecolor{currentfill}{rgb}{0.000000,0.000000,0.000000}%
\pgfsetfillcolor{currentfill}%
\pgfsetlinewidth{0.803000pt}%
\definecolor{currentstroke}{rgb}{0.000000,0.000000,0.000000}%
\pgfsetstrokecolor{currentstroke}%
\pgfsetdash{}{0pt}%
\pgfsys@defobject{currentmarker}{\pgfqpoint{-0.048611in}{0.000000in}}{\pgfqpoint{0.000000in}{0.000000in}}{%
\pgfpathmoveto{\pgfqpoint{0.000000in}{0.000000in}}%
\pgfpathlineto{\pgfqpoint{-0.048611in}{0.000000in}}%
\pgfusepath{stroke,fill}%
}%
\begin{pgfscope}%
\pgfsys@transformshift{0.687500in}{2.271500in}%
\pgfsys@useobject{currentmarker}{}%
\end{pgfscope}%
\end{pgfscope}%
\begin{pgfscope}%
\definecolor{textcolor}{rgb}{0.000000,0.000000,0.000000}%
\pgfsetstrokecolor{textcolor}%
\pgfsetfillcolor{textcolor}%
\pgftext[x=0.412808in,y=2.223275in,left,base]{\color{textcolor}\rmfamily\fontsize{10.000000}{12.000000}\selectfont \(\displaystyle 0.8\)}%
\end{pgfscope}%
\begin{pgfscope}%
\pgfsetbuttcap%
\pgfsetroundjoin%
\definecolor{currentfill}{rgb}{0.000000,0.000000,0.000000}%
\pgfsetfillcolor{currentfill}%
\pgfsetlinewidth{0.803000pt}%
\definecolor{currentstroke}{rgb}{0.000000,0.000000,0.000000}%
\pgfsetstrokecolor{currentstroke}%
\pgfsetdash{}{0pt}%
\pgfsys@defobject{currentmarker}{\pgfqpoint{-0.048611in}{0.000000in}}{\pgfqpoint{0.000000in}{0.000000in}}{%
\pgfpathmoveto{\pgfqpoint{0.000000in}{0.000000in}}%
\pgfpathlineto{\pgfqpoint{-0.048611in}{0.000000in}}%
\pgfusepath{stroke,fill}%
}%
\begin{pgfscope}%
\pgfsys@transformshift{0.687500in}{2.630833in}%
\pgfsys@useobject{currentmarker}{}%
\end{pgfscope}%
\end{pgfscope}%
\begin{pgfscope}%
\definecolor{textcolor}{rgb}{0.000000,0.000000,0.000000}%
\pgfsetstrokecolor{textcolor}%
\pgfsetfillcolor{textcolor}%
\pgftext[x=0.412808in,y=2.582608in,left,base]{\color{textcolor}\rmfamily\fontsize{10.000000}{12.000000}\selectfont \(\displaystyle 1.0\)}%
\end{pgfscope}%
\begin{pgfscope}%
\pgfsetbuttcap%
\pgfsetroundjoin%
\definecolor{currentfill}{rgb}{0.000000,0.000000,0.000000}%
\pgfsetfillcolor{currentfill}%
\pgfsetlinewidth{0.803000pt}%
\definecolor{currentstroke}{rgb}{0.000000,0.000000,0.000000}%
\pgfsetstrokecolor{currentstroke}%
\pgfsetdash{}{0pt}%
\pgfsys@defobject{currentmarker}{\pgfqpoint{-0.048611in}{0.000000in}}{\pgfqpoint{0.000000in}{0.000000in}}{%
\pgfpathmoveto{\pgfqpoint{0.000000in}{0.000000in}}%
\pgfpathlineto{\pgfqpoint{-0.048611in}{0.000000in}}%
\pgfusepath{stroke,fill}%
}%
\begin{pgfscope}%
\pgfsys@transformshift{0.687500in}{2.990167in}%
\pgfsys@useobject{currentmarker}{}%
\end{pgfscope}%
\end{pgfscope}%
\begin{pgfscope}%
\definecolor{textcolor}{rgb}{0.000000,0.000000,0.000000}%
\pgfsetstrokecolor{textcolor}%
\pgfsetfillcolor{textcolor}%
\pgftext[x=0.412808in,y=2.941941in,left,base]{\color{textcolor}\rmfamily\fontsize{10.000000}{12.000000}\selectfont \(\displaystyle 1.2\)}%
\end{pgfscope}%
\begin{pgfscope}%
\definecolor{textcolor}{rgb}{0.000000,0.000000,0.000000}%
\pgfsetstrokecolor{textcolor}%
\pgfsetfillcolor{textcolor}%
\pgftext[x=0.249228in,y=1.732500in,,bottom,rotate=90.000000]{\color{textcolor}\rmfamily\fontsize{10.000000}{12.000000}\selectfont y}%
\end{pgfscope}%
\begin{pgfscope}%
\pgfpathrectangle{\pgfqpoint{0.687500in}{0.385000in}}{\pgfqpoint{4.262500in}{2.695000in}}%
\pgfusepath{clip}%
\pgfsetrectcap%
\pgfsetroundjoin%
\pgfsetlinewidth{1.505625pt}%
\definecolor{currentstroke}{rgb}{0.121569,0.466667,0.705882}%
\pgfsetstrokecolor{currentstroke}%
\pgfsetdash{}{0pt}%
\pgfpathmoveto{\pgfqpoint{0.881250in}{1.732500in}}%
\pgfpathlineto{\pgfqpoint{0.978612in}{1.778775in}}%
\pgfpathlineto{\pgfqpoint{1.075974in}{1.827296in}}%
\pgfpathlineto{\pgfqpoint{1.173335in}{1.878000in}}%
\pgfpathlineto{\pgfqpoint{1.290170in}{1.941556in}}%
\pgfpathlineto{\pgfqpoint{1.407004in}{2.007752in}}%
\pgfpathlineto{\pgfqpoint{1.543310in}{2.087643in}}%
\pgfpathlineto{\pgfqpoint{1.796451in}{2.239567in}}%
\pgfpathlineto{\pgfqpoint{1.932758in}{2.320107in}}%
\pgfpathlineto{\pgfqpoint{2.030119in}{2.375473in}}%
\pgfpathlineto{\pgfqpoint{2.108009in}{2.417737in}}%
\pgfpathlineto{\pgfqpoint{2.185898in}{2.457628in}}%
\pgfpathlineto{\pgfqpoint{2.263788in}{2.494605in}}%
\pgfpathlineto{\pgfqpoint{2.322205in}{2.520101in}}%
\pgfpathlineto{\pgfqpoint{2.380622in}{2.543430in}}%
\pgfpathlineto{\pgfqpoint{2.439039in}{2.564379in}}%
\pgfpathlineto{\pgfqpoint{2.497456in}{2.582749in}}%
\pgfpathlineto{\pgfqpoint{2.555873in}{2.598357in}}%
\pgfpathlineto{\pgfqpoint{2.614290in}{2.611046in}}%
\pgfpathlineto{\pgfqpoint{2.672707in}{2.620683in}}%
\pgfpathlineto{\pgfqpoint{2.731124in}{2.627166in}}%
\pgfpathlineto{\pgfqpoint{2.789541in}{2.630425in}}%
\pgfpathlineto{\pgfqpoint{2.847959in}{2.630425in}}%
\pgfpathlineto{\pgfqpoint{2.906376in}{2.627166in}}%
\pgfpathlineto{\pgfqpoint{2.964793in}{2.620683in}}%
\pgfpathlineto{\pgfqpoint{3.023210in}{2.611046in}}%
\pgfpathlineto{\pgfqpoint{3.081627in}{2.598357in}}%
\pgfpathlineto{\pgfqpoint{3.140044in}{2.582749in}}%
\pgfpathlineto{\pgfqpoint{3.198461in}{2.564379in}}%
\pgfpathlineto{\pgfqpoint{3.256878in}{2.543430in}}%
\pgfpathlineto{\pgfqpoint{3.315295in}{2.520101in}}%
\pgfpathlineto{\pgfqpoint{3.373712in}{2.494605in}}%
\pgfpathlineto{\pgfqpoint{3.451602in}{2.457628in}}%
\pgfpathlineto{\pgfqpoint{3.529491in}{2.417737in}}%
\pgfpathlineto{\pgfqpoint{3.607381in}{2.375473in}}%
\pgfpathlineto{\pgfqpoint{3.704742in}{2.320107in}}%
\pgfpathlineto{\pgfqpoint{3.841049in}{2.239567in}}%
\pgfpathlineto{\pgfqpoint{4.211024in}{2.019009in}}%
\pgfpathlineto{\pgfqpoint{4.327858in}{1.952417in}}%
\pgfpathlineto{\pgfqpoint{4.444692in}{1.888394in}}%
\pgfpathlineto{\pgfqpoint{4.542054in}{1.837265in}}%
\pgfpathlineto{\pgfqpoint{4.639416in}{1.788301in}}%
\pgfpathlineto{\pgfqpoint{4.736778in}{1.741574in}}%
\pgfpathlineto{\pgfqpoint{4.756250in}{1.732500in}}%
\pgfpathlineto{\pgfqpoint{4.756250in}{1.732500in}}%
\pgfusepath{stroke}%
\end{pgfscope}%
\begin{pgfscope}%
\pgfpathrectangle{\pgfqpoint{0.687500in}{0.385000in}}{\pgfqpoint{4.262500in}{2.695000in}}%
\pgfusepath{clip}%
\pgfsetbuttcap%
\pgfsetroundjoin%
\definecolor{currentfill}{rgb}{1.000000,0.498039,0.054902}%
\pgfsetfillcolor{currentfill}%
\pgfsetlinewidth{1.003750pt}%
\definecolor{currentstroke}{rgb}{1.000000,0.498039,0.054902}%
\pgfsetstrokecolor{currentstroke}%
\pgfsetdash{}{0pt}%
\pgfsys@defobject{currentmarker}{\pgfqpoint{-0.020833in}{-0.020833in}}{\pgfqpoint{0.020833in}{0.020833in}}{%
\pgfpathmoveto{\pgfqpoint{0.000000in}{-0.020833in}}%
\pgfpathcurveto{\pgfqpoint{0.005525in}{-0.020833in}}{\pgfqpoint{0.010825in}{-0.018638in}}{\pgfqpoint{0.014731in}{-0.014731in}}%
\pgfpathcurveto{\pgfqpoint{0.018638in}{-0.010825in}}{\pgfqpoint{0.020833in}{-0.005525in}}{\pgfqpoint{0.020833in}{0.000000in}}%
\pgfpathcurveto{\pgfqpoint{0.020833in}{0.005525in}}{\pgfqpoint{0.018638in}{0.010825in}}{\pgfqpoint{0.014731in}{0.014731in}}%
\pgfpathcurveto{\pgfqpoint{0.010825in}{0.018638in}}{\pgfqpoint{0.005525in}{0.020833in}}{\pgfqpoint{0.000000in}{0.020833in}}%
\pgfpathcurveto{\pgfqpoint{-0.005525in}{0.020833in}}{\pgfqpoint{-0.010825in}{0.018638in}}{\pgfqpoint{-0.014731in}{0.014731in}}%
\pgfpathcurveto{\pgfqpoint{-0.018638in}{0.010825in}}{\pgfqpoint{-0.020833in}{0.005525in}}{\pgfqpoint{-0.020833in}{0.000000in}}%
\pgfpathcurveto{\pgfqpoint{-0.020833in}{-0.005525in}}{\pgfqpoint{-0.018638in}{-0.010825in}}{\pgfqpoint{-0.014731in}{-0.014731in}}%
\pgfpathcurveto{\pgfqpoint{-0.010825in}{-0.018638in}}{\pgfqpoint{-0.005525in}{-0.020833in}}{\pgfqpoint{0.000000in}{-0.020833in}}%
\pgfpathclose%
\pgfusepath{stroke,fill}%
}%
\begin{pgfscope}%
\pgfsys@transformshift{4.690231in}{1.763631in}%
\pgfsys@useobject{currentmarker}{}%
\end{pgfscope}%
\begin{pgfscope}%
\pgfsys@transformshift{4.188769in}{2.031944in}%
\pgfsys@useobject{currentmarker}{}%
\end{pgfscope}%
\begin{pgfscope}%
\pgfsys@transformshift{3.320212in}{2.518036in}%
\pgfsys@useobject{currentmarker}{}%
\end{pgfscope}%
\begin{pgfscope}%
\pgfsys@transformshift{2.317288in}{2.518036in}%
\pgfsys@useobject{currentmarker}{}%
\end{pgfscope}%
\begin{pgfscope}%
\pgfsys@transformshift{1.448731in}{2.031944in}%
\pgfsys@useobject{currentmarker}{}%
\end{pgfscope}%
\begin{pgfscope}%
\pgfsys@transformshift{0.947269in}{1.763631in}%
\pgfsys@useobject{currentmarker}{}%
\end{pgfscope}%
\end{pgfscope}%
\begin{pgfscope}%
\pgfpathrectangle{\pgfqpoint{0.687500in}{0.385000in}}{\pgfqpoint{4.262500in}{2.695000in}}%
\pgfusepath{clip}%
\pgfsetrectcap%
\pgfsetroundjoin%
\pgfsetlinewidth{1.505625pt}%
\definecolor{currentstroke}{rgb}{0.172549,0.627451,0.172549}%
\pgfsetstrokecolor{currentstroke}%
\pgfsetdash{}{0pt}%
\pgfpathmoveto{\pgfqpoint{0.881250in}{1.741574in}}%
\pgfpathlineto{\pgfqpoint{0.920195in}{1.754050in}}%
\pgfpathlineto{\pgfqpoint{0.978612in}{1.775599in}}%
\pgfpathlineto{\pgfqpoint{1.037029in}{1.800223in}}%
\pgfpathlineto{\pgfqpoint{1.095446in}{1.827567in}}%
\pgfpathlineto{\pgfqpoint{1.153863in}{1.857283in}}%
\pgfpathlineto{\pgfqpoint{1.231753in}{1.900021in}}%
\pgfpathlineto{\pgfqpoint{1.309642in}{1.945626in}}%
\pgfpathlineto{\pgfqpoint{1.407004in}{2.005564in}}%
\pgfpathlineto{\pgfqpoint{1.543310in}{2.092586in}}%
\pgfpathlineto{\pgfqpoint{1.757506in}{2.229377in}}%
\pgfpathlineto{\pgfqpoint{1.854868in}{2.288933in}}%
\pgfpathlineto{\pgfqpoint{1.932758in}{2.334481in}}%
\pgfpathlineto{\pgfqpoint{2.010647in}{2.377695in}}%
\pgfpathlineto{\pgfqpoint{2.088536in}{2.418178in}}%
\pgfpathlineto{\pgfqpoint{2.166426in}{2.455572in}}%
\pgfpathlineto{\pgfqpoint{2.224843in}{2.481393in}}%
\pgfpathlineto{\pgfqpoint{2.283260in}{2.505171in}}%
\pgfpathlineto{\pgfqpoint{2.341677in}{2.526795in}}%
\pgfpathlineto{\pgfqpoint{2.400094in}{2.546163in}}%
\pgfpathlineto{\pgfqpoint{2.458511in}{2.563189in}}%
\pgfpathlineto{\pgfqpoint{2.516928in}{2.577795in}}%
\pgfpathlineto{\pgfqpoint{2.575345in}{2.589916in}}%
\pgfpathlineto{\pgfqpoint{2.633763in}{2.599498in}}%
\pgfpathlineto{\pgfqpoint{2.692180in}{2.606500in}}%
\pgfpathlineto{\pgfqpoint{2.750597in}{2.610890in}}%
\pgfpathlineto{\pgfqpoint{2.809014in}{2.612649in}}%
\pgfpathlineto{\pgfqpoint{2.867431in}{2.611769in}}%
\pgfpathlineto{\pgfqpoint{2.925848in}{2.608255in}}%
\pgfpathlineto{\pgfqpoint{2.984265in}{2.602121in}}%
\pgfpathlineto{\pgfqpoint{3.042682in}{2.593395in}}%
\pgfpathlineto{\pgfqpoint{3.101099in}{2.582114in}}%
\pgfpathlineto{\pgfqpoint{3.159516in}{2.568330in}}%
\pgfpathlineto{\pgfqpoint{3.217933in}{2.552103in}}%
\pgfpathlineto{\pgfqpoint{3.276351in}{2.533506in}}%
\pgfpathlineto{\pgfqpoint{3.334768in}{2.512624in}}%
\pgfpathlineto{\pgfqpoint{3.393185in}{2.489552in}}%
\pgfpathlineto{\pgfqpoint{3.451602in}{2.464399in}}%
\pgfpathlineto{\pgfqpoint{3.510019in}{2.437282in}}%
\pgfpathlineto{\pgfqpoint{3.587908in}{2.398301in}}%
\pgfpathlineto{\pgfqpoint{3.665798in}{2.356406in}}%
\pgfpathlineto{\pgfqpoint{3.743687in}{2.311973in}}%
\pgfpathlineto{\pgfqpoint{3.841049in}{2.253499in}}%
\pgfpathlineto{\pgfqpoint{3.957883in}{2.180212in}}%
\pgfpathlineto{\pgfqpoint{4.308386in}{1.957390in}}%
\pgfpathlineto{\pgfqpoint{4.386275in}{1.911183in}}%
\pgfpathlineto{\pgfqpoint{4.464165in}{1.867658in}}%
\pgfpathlineto{\pgfqpoint{4.522582in}{1.837225in}}%
\pgfpathlineto{\pgfqpoint{4.580999in}{1.809053in}}%
\pgfpathlineto{\pgfqpoint{4.639416in}{1.783483in}}%
\pgfpathlineto{\pgfqpoint{4.697833in}{1.760868in}}%
\pgfpathlineto{\pgfqpoint{4.756250in}{1.741574in}}%
\pgfpathlineto{\pgfqpoint{4.756250in}{1.741574in}}%
\pgfusepath{stroke}%
\end{pgfscope}%
\begin{pgfscope}%
\pgfpathrectangle{\pgfqpoint{0.687500in}{0.385000in}}{\pgfqpoint{4.262500in}{2.695000in}}%
\pgfusepath{clip}%
\pgfsetbuttcap%
\pgfsetroundjoin%
\definecolor{currentfill}{rgb}{0.839216,0.152941,0.156863}%
\pgfsetfillcolor{currentfill}%
\pgfsetlinewidth{1.003750pt}%
\definecolor{currentstroke}{rgb}{0.839216,0.152941,0.156863}%
\pgfsetstrokecolor{currentstroke}%
\pgfsetdash{}{0pt}%
\pgfsys@defobject{currentmarker}{\pgfqpoint{-0.020833in}{-0.020833in}}{\pgfqpoint{0.020833in}{0.020833in}}{%
\pgfpathmoveto{\pgfqpoint{0.000000in}{-0.020833in}}%
\pgfpathcurveto{\pgfqpoint{0.005525in}{-0.020833in}}{\pgfqpoint{0.010825in}{-0.018638in}}{\pgfqpoint{0.014731in}{-0.014731in}}%
\pgfpathcurveto{\pgfqpoint{0.018638in}{-0.010825in}}{\pgfqpoint{0.020833in}{-0.005525in}}{\pgfqpoint{0.020833in}{0.000000in}}%
\pgfpathcurveto{\pgfqpoint{0.020833in}{0.005525in}}{\pgfqpoint{0.018638in}{0.010825in}}{\pgfqpoint{0.014731in}{0.014731in}}%
\pgfpathcurveto{\pgfqpoint{0.010825in}{0.018638in}}{\pgfqpoint{0.005525in}{0.020833in}}{\pgfqpoint{0.000000in}{0.020833in}}%
\pgfpathcurveto{\pgfqpoint{-0.005525in}{0.020833in}}{\pgfqpoint{-0.010825in}{0.018638in}}{\pgfqpoint{-0.014731in}{0.014731in}}%
\pgfpathcurveto{\pgfqpoint{-0.018638in}{0.010825in}}{\pgfqpoint{-0.020833in}{0.005525in}}{\pgfqpoint{-0.020833in}{0.000000in}}%
\pgfpathcurveto{\pgfqpoint{-0.020833in}{-0.005525in}}{\pgfqpoint{-0.018638in}{-0.010825in}}{\pgfqpoint{-0.014731in}{-0.014731in}}%
\pgfpathcurveto{\pgfqpoint{-0.010825in}{-0.018638in}}{\pgfqpoint{-0.005525in}{-0.020833in}}{\pgfqpoint{0.000000in}{-0.020833in}}%
\pgfpathclose%
\pgfusepath{stroke,fill}%
}%
\begin{pgfscope}%
\pgfsys@transformshift{4.719021in}{1.749927in}%
\pgfsys@useobject{currentmarker}{}%
\end{pgfscope}%
\begin{pgfscope}%
\pgfsys@transformshift{4.429722in}{1.896440in}%
\pgfsys@useobject{currentmarker}{}%
\end{pgfscope}%
\begin{pgfscope}%
\pgfsys@transformshift{3.895167in}{2.207074in}%
\pgfsys@useobject{currentmarker}{}%
\end{pgfscope}%
\begin{pgfscope}%
\pgfsys@transformshift{3.196737in}{2.564959in}%
\pgfsys@useobject{currentmarker}{}%
\end{pgfscope}%
\begin{pgfscope}%
\pgfsys@transformshift{2.440763in}{2.564959in}%
\pgfsys@useobject{currentmarker}{}%
\end{pgfscope}%
\begin{pgfscope}%
\pgfsys@transformshift{1.742333in}{2.207074in}%
\pgfsys@useobject{currentmarker}{}%
\end{pgfscope}%
\begin{pgfscope}%
\pgfsys@transformshift{1.207778in}{1.896440in}%
\pgfsys@useobject{currentmarker}{}%
\end{pgfscope}%
\begin{pgfscope}%
\pgfsys@transformshift{0.918479in}{1.749927in}%
\pgfsys@useobject{currentmarker}{}%
\end{pgfscope}%
\end{pgfscope}%
\begin{pgfscope}%
\pgfpathrectangle{\pgfqpoint{0.687500in}{0.385000in}}{\pgfqpoint{4.262500in}{2.695000in}}%
\pgfusepath{clip}%
\pgfsetrectcap%
\pgfsetroundjoin%
\pgfsetlinewidth{1.505625pt}%
\definecolor{currentstroke}{rgb}{0.580392,0.403922,0.741176}%
\pgfsetstrokecolor{currentstroke}%
\pgfsetdash{}{0pt}%
\pgfpathmoveto{\pgfqpoint{0.881250in}{1.730943in}}%
\pgfpathlineto{\pgfqpoint{1.017557in}{1.799574in}}%
\pgfpathlineto{\pgfqpoint{1.173335in}{1.878470in}}%
\pgfpathlineto{\pgfqpoint{1.270697in}{1.929945in}}%
\pgfpathlineto{\pgfqpoint{1.368059in}{1.983657in}}%
\pgfpathlineto{\pgfqpoint{1.465421in}{2.039609in}}%
\pgfpathlineto{\pgfqpoint{1.582255in}{2.109229in}}%
\pgfpathlineto{\pgfqpoint{1.991175in}{2.356054in}}%
\pgfpathlineto{\pgfqpoint{2.069064in}{2.399575in}}%
\pgfpathlineto{\pgfqpoint{2.146954in}{2.440660in}}%
\pgfpathlineto{\pgfqpoint{2.224843in}{2.478782in}}%
\pgfpathlineto{\pgfqpoint{2.283260in}{2.505126in}}%
\pgfpathlineto{\pgfqpoint{2.341677in}{2.529325in}}%
\pgfpathlineto{\pgfqpoint{2.400094in}{2.551196in}}%
\pgfpathlineto{\pgfqpoint{2.458511in}{2.570574in}}%
\pgfpathlineto{\pgfqpoint{2.516928in}{2.587312in}}%
\pgfpathlineto{\pgfqpoint{2.575345in}{2.601282in}}%
\pgfpathlineto{\pgfqpoint{2.633763in}{2.612377in}}%
\pgfpathlineto{\pgfqpoint{2.692180in}{2.620512in}}%
\pgfpathlineto{\pgfqpoint{2.750597in}{2.625625in}}%
\pgfpathlineto{\pgfqpoint{2.809014in}{2.627677in}}%
\pgfpathlineto{\pgfqpoint{2.867431in}{2.626651in}}%
\pgfpathlineto{\pgfqpoint{2.925848in}{2.622555in}}%
\pgfpathlineto{\pgfqpoint{2.984265in}{2.615421in}}%
\pgfpathlineto{\pgfqpoint{3.042682in}{2.605304in}}%
\pgfpathlineto{\pgfqpoint{3.101099in}{2.592282in}}%
\pgfpathlineto{\pgfqpoint{3.159516in}{2.576453in}}%
\pgfpathlineto{\pgfqpoint{3.217933in}{2.557940in}}%
\pgfpathlineto{\pgfqpoint{3.276351in}{2.536882in}}%
\pgfpathlineto{\pgfqpoint{3.334768in}{2.513440in}}%
\pgfpathlineto{\pgfqpoint{3.393185in}{2.487790in}}%
\pgfpathlineto{\pgfqpoint{3.451602in}{2.460124in}}%
\pgfpathlineto{\pgfqpoint{3.529491in}{2.420455in}}%
\pgfpathlineto{\pgfqpoint{3.607381in}{2.378085in}}%
\pgfpathlineto{\pgfqpoint{3.704742in}{2.322145in}}%
\pgfpathlineto{\pgfqpoint{3.821577in}{2.252077in}}%
\pgfpathlineto{\pgfqpoint{4.172079in}{2.039609in}}%
\pgfpathlineto{\pgfqpoint{4.288913in}{1.972730in}}%
\pgfpathlineto{\pgfqpoint{4.386275in}{1.919478in}}%
\pgfpathlineto{\pgfqpoint{4.503109in}{1.858431in}}%
\pgfpathlineto{\pgfqpoint{4.658888in}{1.780131in}}%
\pgfpathlineto{\pgfqpoint{4.756250in}{1.730943in}}%
\pgfpathlineto{\pgfqpoint{4.756250in}{1.730943in}}%
\pgfusepath{stroke}%
\end{pgfscope}%
\begin{pgfscope}%
\pgfpathrectangle{\pgfqpoint{0.687500in}{0.385000in}}{\pgfqpoint{4.262500in}{2.695000in}}%
\pgfusepath{clip}%
\pgfsetbuttcap%
\pgfsetroundjoin%
\definecolor{currentfill}{rgb}{0.549020,0.337255,0.294118}%
\pgfsetfillcolor{currentfill}%
\pgfsetlinewidth{1.003750pt}%
\definecolor{currentstroke}{rgb}{0.549020,0.337255,0.294118}%
\pgfsetstrokecolor{currentstroke}%
\pgfsetdash{}{0pt}%
\pgfsys@defobject{currentmarker}{\pgfqpoint{-0.020833in}{-0.020833in}}{\pgfqpoint{0.020833in}{0.020833in}}{%
\pgfpathmoveto{\pgfqpoint{0.000000in}{-0.020833in}}%
\pgfpathcurveto{\pgfqpoint{0.005525in}{-0.020833in}}{\pgfqpoint{0.010825in}{-0.018638in}}{\pgfqpoint{0.014731in}{-0.014731in}}%
\pgfpathcurveto{\pgfqpoint{0.018638in}{-0.010825in}}{\pgfqpoint{0.020833in}{-0.005525in}}{\pgfqpoint{0.020833in}{0.000000in}}%
\pgfpathcurveto{\pgfqpoint{0.020833in}{0.005525in}}{\pgfqpoint{0.018638in}{0.010825in}}{\pgfqpoint{0.014731in}{0.014731in}}%
\pgfpathcurveto{\pgfqpoint{0.010825in}{0.018638in}}{\pgfqpoint{0.005525in}{0.020833in}}{\pgfqpoint{0.000000in}{0.020833in}}%
\pgfpathcurveto{\pgfqpoint{-0.005525in}{0.020833in}}{\pgfqpoint{-0.010825in}{0.018638in}}{\pgfqpoint{-0.014731in}{0.014731in}}%
\pgfpathcurveto{\pgfqpoint{-0.018638in}{0.010825in}}{\pgfqpoint{-0.020833in}{0.005525in}}{\pgfqpoint{-0.020833in}{0.000000in}}%
\pgfpathcurveto{\pgfqpoint{-0.020833in}{-0.005525in}}{\pgfqpoint{-0.018638in}{-0.010825in}}{\pgfqpoint{-0.014731in}{-0.014731in}}%
\pgfpathcurveto{\pgfqpoint{-0.010825in}{-0.018638in}}{\pgfqpoint{-0.005525in}{-0.020833in}}{\pgfqpoint{0.000000in}{-0.020833in}}%
\pgfpathclose%
\pgfusepath{stroke,fill}%
}%
\begin{pgfscope}%
\pgfsys@transformshift{4.746920in}{1.736836in}%
\pgfsys@useobject{currentmarker}{}%
\end{pgfscope}%
\begin{pgfscope}%
\pgfsys@transformshift{4.672822in}{1.772014in}%
\pgfsys@useobject{currentmarker}{}%
\end{pgfscope}%
\begin{pgfscope}%
\pgfsys@transformshift{4.527472in}{1.844787in}%
\pgfsys@useobject{currentmarker}{}%
\end{pgfscope}%
\begin{pgfscope}%
\pgfsys@transformshift{4.316458in}{1.958809in}%
\pgfsys@useobject{currentmarker}{}%
\end{pgfscope}%
\begin{pgfscope}%
\pgfsys@transformshift{4.047887in}{2.115254in}%
\pgfsys@useobject{currentmarker}{}%
\end{pgfscope}%
\begin{pgfscope}%
\pgfsys@transformshift{3.732081in}{2.304175in}%
\pgfsys@useobject{currentmarker}{}%
\end{pgfscope}%
\begin{pgfscope}%
\pgfsys@transformshift{3.381177in}{2.491203in}%
\pgfsys@useobject{currentmarker}{}%
\end{pgfscope}%
\begin{pgfscope}%
\pgfsys@transformshift{3.008658in}{2.613736in}%
\pgfsys@useobject{currentmarker}{}%
\end{pgfscope}%
\begin{pgfscope}%
\pgfsys@transformshift{2.628842in}{2.613736in}%
\pgfsys@useobject{currentmarker}{}%
\end{pgfscope}%
\begin{pgfscope}%
\pgfsys@transformshift{2.256323in}{2.491203in}%
\pgfsys@useobject{currentmarker}{}%
\end{pgfscope}%
\begin{pgfscope}%
\pgfsys@transformshift{1.905419in}{2.304175in}%
\pgfsys@useobject{currentmarker}{}%
\end{pgfscope}%
\begin{pgfscope}%
\pgfsys@transformshift{1.589613in}{2.115254in}%
\pgfsys@useobject{currentmarker}{}%
\end{pgfscope}%
\begin{pgfscope}%
\pgfsys@transformshift{1.321042in}{1.958809in}%
\pgfsys@useobject{currentmarker}{}%
\end{pgfscope}%
\begin{pgfscope}%
\pgfsys@transformshift{1.110028in}{1.844787in}%
\pgfsys@useobject{currentmarker}{}%
\end{pgfscope}%
\begin{pgfscope}%
\pgfsys@transformshift{0.964678in}{1.772014in}%
\pgfsys@useobject{currentmarker}{}%
\end{pgfscope}%
\begin{pgfscope}%
\pgfsys@transformshift{0.890580in}{1.736836in}%
\pgfsys@useobject{currentmarker}{}%
\end{pgfscope}%
\end{pgfscope}%
\begin{pgfscope}%
\pgfpathrectangle{\pgfqpoint{0.687500in}{0.385000in}}{\pgfqpoint{4.262500in}{2.695000in}}%
\pgfusepath{clip}%
\pgfsetrectcap%
\pgfsetroundjoin%
\pgfsetlinewidth{1.505625pt}%
\definecolor{currentstroke}{rgb}{0.890196,0.466667,0.760784}%
\pgfsetstrokecolor{currentstroke}%
\pgfsetdash{}{0pt}%
\pgfpathmoveto{\pgfqpoint{0.881250in}{1.732499in}}%
\pgfpathlineto{\pgfqpoint{0.978612in}{1.778775in}}%
\pgfpathlineto{\pgfqpoint{1.075974in}{1.827295in}}%
\pgfpathlineto{\pgfqpoint{1.173335in}{1.878002in}}%
\pgfpathlineto{\pgfqpoint{1.290170in}{1.941557in}}%
\pgfpathlineto{\pgfqpoint{1.407004in}{2.007751in}}%
\pgfpathlineto{\pgfqpoint{1.543310in}{2.087642in}}%
\pgfpathlineto{\pgfqpoint{1.796451in}{2.239569in}}%
\pgfpathlineto{\pgfqpoint{1.932758in}{2.320106in}}%
\pgfpathlineto{\pgfqpoint{2.030119in}{2.375471in}}%
\pgfpathlineto{\pgfqpoint{2.108009in}{2.417734in}}%
\pgfpathlineto{\pgfqpoint{2.185898in}{2.457626in}}%
\pgfpathlineto{\pgfqpoint{2.263788in}{2.494605in}}%
\pgfpathlineto{\pgfqpoint{2.322205in}{2.520102in}}%
\pgfpathlineto{\pgfqpoint{2.380622in}{2.543432in}}%
\pgfpathlineto{\pgfqpoint{2.439039in}{2.564382in}}%
\pgfpathlineto{\pgfqpoint{2.497456in}{2.582751in}}%
\pgfpathlineto{\pgfqpoint{2.555873in}{2.598359in}}%
\pgfpathlineto{\pgfqpoint{2.614290in}{2.611046in}}%
\pgfpathlineto{\pgfqpoint{2.672707in}{2.620682in}}%
\pgfpathlineto{\pgfqpoint{2.731124in}{2.627164in}}%
\pgfpathlineto{\pgfqpoint{2.789541in}{2.630422in}}%
\pgfpathlineto{\pgfqpoint{2.847959in}{2.630422in}}%
\pgfpathlineto{\pgfqpoint{2.906376in}{2.627164in}}%
\pgfpathlineto{\pgfqpoint{2.964793in}{2.620682in}}%
\pgfpathlineto{\pgfqpoint{3.023210in}{2.611046in}}%
\pgfpathlineto{\pgfqpoint{3.081627in}{2.598359in}}%
\pgfpathlineto{\pgfqpoint{3.140044in}{2.582751in}}%
\pgfpathlineto{\pgfqpoint{3.198461in}{2.564382in}}%
\pgfpathlineto{\pgfqpoint{3.256878in}{2.543432in}}%
\pgfpathlineto{\pgfqpoint{3.315295in}{2.520102in}}%
\pgfpathlineto{\pgfqpoint{3.373712in}{2.494605in}}%
\pgfpathlineto{\pgfqpoint{3.451602in}{2.457626in}}%
\pgfpathlineto{\pgfqpoint{3.529491in}{2.417734in}}%
\pgfpathlineto{\pgfqpoint{3.607381in}{2.375471in}}%
\pgfpathlineto{\pgfqpoint{3.704742in}{2.320106in}}%
\pgfpathlineto{\pgfqpoint{3.841049in}{2.239569in}}%
\pgfpathlineto{\pgfqpoint{4.211024in}{2.019008in}}%
\pgfpathlineto{\pgfqpoint{4.327858in}{1.952417in}}%
\pgfpathlineto{\pgfqpoint{4.444692in}{1.888395in}}%
\pgfpathlineto{\pgfqpoint{4.542054in}{1.837265in}}%
\pgfpathlineto{\pgfqpoint{4.639416in}{1.788300in}}%
\pgfpathlineto{\pgfqpoint{4.736778in}{1.741575in}}%
\pgfpathlineto{\pgfqpoint{4.756250in}{1.732499in}}%
\pgfpathlineto{\pgfqpoint{4.756250in}{1.732499in}}%
\pgfusepath{stroke}%
\end{pgfscope}%
\begin{pgfscope}%
\pgfpathrectangle{\pgfqpoint{0.687500in}{0.385000in}}{\pgfqpoint{4.262500in}{2.695000in}}%
\pgfusepath{clip}%
\pgfsetbuttcap%
\pgfsetroundjoin%
\definecolor{currentfill}{rgb}{0.498039,0.498039,0.498039}%
\pgfsetfillcolor{currentfill}%
\pgfsetlinewidth{1.003750pt}%
\definecolor{currentstroke}{rgb}{0.498039,0.498039,0.498039}%
\pgfsetstrokecolor{currentstroke}%
\pgfsetdash{}{0pt}%
\pgfsys@defobject{currentmarker}{\pgfqpoint{-0.020833in}{-0.020833in}}{\pgfqpoint{0.020833in}{0.020833in}}{%
\pgfpathmoveto{\pgfqpoint{0.000000in}{-0.020833in}}%
\pgfpathcurveto{\pgfqpoint{0.005525in}{-0.020833in}}{\pgfqpoint{0.010825in}{-0.018638in}}{\pgfqpoint{0.014731in}{-0.014731in}}%
\pgfpathcurveto{\pgfqpoint{0.018638in}{-0.010825in}}{\pgfqpoint{0.020833in}{-0.005525in}}{\pgfqpoint{0.020833in}{0.000000in}}%
\pgfpathcurveto{\pgfqpoint{0.020833in}{0.005525in}}{\pgfqpoint{0.018638in}{0.010825in}}{\pgfqpoint{0.014731in}{0.014731in}}%
\pgfpathcurveto{\pgfqpoint{0.010825in}{0.018638in}}{\pgfqpoint{0.005525in}{0.020833in}}{\pgfqpoint{0.000000in}{0.020833in}}%
\pgfpathcurveto{\pgfqpoint{-0.005525in}{0.020833in}}{\pgfqpoint{-0.010825in}{0.018638in}}{\pgfqpoint{-0.014731in}{0.014731in}}%
\pgfpathcurveto{\pgfqpoint{-0.018638in}{0.010825in}}{\pgfqpoint{-0.020833in}{0.005525in}}{\pgfqpoint{-0.020833in}{0.000000in}}%
\pgfpathcurveto{\pgfqpoint{-0.020833in}{-0.005525in}}{\pgfqpoint{-0.018638in}{-0.010825in}}{\pgfqpoint{-0.014731in}{-0.014731in}}%
\pgfpathcurveto{\pgfqpoint{-0.010825in}{-0.018638in}}{\pgfqpoint{-0.005525in}{-0.020833in}}{\pgfqpoint{0.000000in}{-0.020833in}}%
\pgfpathclose%
\pgfusepath{stroke,fill}%
}%
\begin{pgfscope}%
\pgfsys@transformshift{4.753916in}{1.733583in}%
\pgfsys@useobject{currentmarker}{}%
\end{pgfscope}%
\begin{pgfscope}%
\pgfsys@transformshift{4.735279in}{1.742276in}%
\pgfsys@useobject{currentmarker}{}%
\end{pgfscope}%
\begin{pgfscope}%
\pgfsys@transformshift{4.698186in}{1.759825in}%
\pgfsys@useobject{currentmarker}{}%
\end{pgfscope}%
\begin{pgfscope}%
\pgfsys@transformshift{4.642992in}{1.786545in}%
\pgfsys@useobject{currentmarker}{}%
\end{pgfscope}%
\begin{pgfscope}%
\pgfsys@transformshift{4.570229in}{1.822869in}%
\pgfsys@useobject{currentmarker}{}%
\end{pgfscope}%
\begin{pgfscope}%
\pgfsys@transformshift{4.480599in}{1.869293in}%
\pgfsys@useobject{currentmarker}{}%
\end{pgfscope}%
\begin{pgfscope}%
\pgfsys@transformshift{4.374965in}{1.926271in}%
\pgfsys@useobject{currentmarker}{}%
\end{pgfscope}%
\begin{pgfscope}%
\pgfsys@transformshift{4.254343in}{1.994048in}%
\pgfsys@useobject{currentmarker}{}%
\end{pgfscope}%
\begin{pgfscope}%
\pgfsys@transformshift{4.119895in}{2.072400in}%
\pgfsys@useobject{currentmarker}{}%
\end{pgfscope}%
\begin{pgfscope}%
\pgfsys@transformshift{3.972917in}{2.160259in}%
\pgfsys@useobject{currentmarker}{}%
\end{pgfscope}%
\begin{pgfscope}%
\pgfsys@transformshift{3.814824in}{2.255241in}%
\pgfsys@useobject{currentmarker}{}%
\end{pgfscope}%
\begin{pgfscope}%
\pgfsys@transformshift{3.647138in}{2.353157in}%
\pgfsys@useobject{currentmarker}{}%
\end{pgfscope}%
\begin{pgfscope}%
\pgfsys@transformshift{3.471474in}{2.447705in}%
\pgfsys@useobject{currentmarker}{}%
\end{pgfscope}%
\begin{pgfscope}%
\pgfsys@transformshift{3.289524in}{2.530673in}%
\pgfsys@useobject{currentmarker}{}%
\end{pgfscope}%
\begin{pgfscope}%
\pgfsys@transformshift{3.103040in}{2.592967in}%
\pgfsys@useobject{currentmarker}{}%
\end{pgfscope}%
\begin{pgfscope}%
\pgfsys@transformshift{2.913819in}{2.626518in}%
\pgfsys@useobject{currentmarker}{}%
\end{pgfscope}%
\begin{pgfscope}%
\pgfsys@transformshift{2.723681in}{2.626518in}%
\pgfsys@useobject{currentmarker}{}%
\end{pgfscope}%
\begin{pgfscope}%
\pgfsys@transformshift{2.534460in}{2.592967in}%
\pgfsys@useobject{currentmarker}{}%
\end{pgfscope}%
\begin{pgfscope}%
\pgfsys@transformshift{2.347976in}{2.530673in}%
\pgfsys@useobject{currentmarker}{}%
\end{pgfscope}%
\begin{pgfscope}%
\pgfsys@transformshift{2.166026in}{2.447705in}%
\pgfsys@useobject{currentmarker}{}%
\end{pgfscope}%
\begin{pgfscope}%
\pgfsys@transformshift{1.990362in}{2.353157in}%
\pgfsys@useobject{currentmarker}{}%
\end{pgfscope}%
\begin{pgfscope}%
\pgfsys@transformshift{1.822676in}{2.255241in}%
\pgfsys@useobject{currentmarker}{}%
\end{pgfscope}%
\begin{pgfscope}%
\pgfsys@transformshift{1.664583in}{2.160259in}%
\pgfsys@useobject{currentmarker}{}%
\end{pgfscope}%
\begin{pgfscope}%
\pgfsys@transformshift{1.517605in}{2.072400in}%
\pgfsys@useobject{currentmarker}{}%
\end{pgfscope}%
\begin{pgfscope}%
\pgfsys@transformshift{1.383157in}{1.994048in}%
\pgfsys@useobject{currentmarker}{}%
\end{pgfscope}%
\begin{pgfscope}%
\pgfsys@transformshift{1.262535in}{1.926271in}%
\pgfsys@useobject{currentmarker}{}%
\end{pgfscope}%
\begin{pgfscope}%
\pgfsys@transformshift{1.156901in}{1.869293in}%
\pgfsys@useobject{currentmarker}{}%
\end{pgfscope}%
\begin{pgfscope}%
\pgfsys@transformshift{1.067271in}{1.822869in}%
\pgfsys@useobject{currentmarker}{}%
\end{pgfscope}%
\begin{pgfscope}%
\pgfsys@transformshift{0.994508in}{1.786545in}%
\pgfsys@useobject{currentmarker}{}%
\end{pgfscope}%
\begin{pgfscope}%
\pgfsys@transformshift{0.939314in}{1.759825in}%
\pgfsys@useobject{currentmarker}{}%
\end{pgfscope}%
\begin{pgfscope}%
\pgfsys@transformshift{0.902221in}{1.742276in}%
\pgfsys@useobject{currentmarker}{}%
\end{pgfscope}%
\begin{pgfscope}%
\pgfsys@transformshift{0.883584in}{1.733583in}%
\pgfsys@useobject{currentmarker}{}%
\end{pgfscope}%
\end{pgfscope}%
\begin{pgfscope}%
\pgfpathrectangle{\pgfqpoint{0.687500in}{0.385000in}}{\pgfqpoint{4.262500in}{2.695000in}}%
\pgfusepath{clip}%
\pgfsetrectcap%
\pgfsetroundjoin%
\pgfsetlinewidth{1.505625pt}%
\definecolor{currentstroke}{rgb}{0.737255,0.741176,0.133333}%
\pgfsetstrokecolor{currentstroke}%
\pgfsetdash{}{0pt}%
\pgfpathmoveto{\pgfqpoint{0.881250in}{1.732500in}}%
\pgfpathlineto{\pgfqpoint{0.978612in}{1.778775in}}%
\pgfpathlineto{\pgfqpoint{1.075974in}{1.827296in}}%
\pgfpathlineto{\pgfqpoint{1.173335in}{1.878000in}}%
\pgfpathlineto{\pgfqpoint{1.290170in}{1.941556in}}%
\pgfpathlineto{\pgfqpoint{1.407004in}{2.007752in}}%
\pgfpathlineto{\pgfqpoint{1.543310in}{2.087643in}}%
\pgfpathlineto{\pgfqpoint{1.796451in}{2.239567in}}%
\pgfpathlineto{\pgfqpoint{1.932758in}{2.320107in}}%
\pgfpathlineto{\pgfqpoint{2.030119in}{2.375473in}}%
\pgfpathlineto{\pgfqpoint{2.108009in}{2.417737in}}%
\pgfpathlineto{\pgfqpoint{2.185898in}{2.457628in}}%
\pgfpathlineto{\pgfqpoint{2.263788in}{2.494605in}}%
\pgfpathlineto{\pgfqpoint{2.322205in}{2.520101in}}%
\pgfpathlineto{\pgfqpoint{2.380622in}{2.543430in}}%
\pgfpathlineto{\pgfqpoint{2.439039in}{2.564379in}}%
\pgfpathlineto{\pgfqpoint{2.497456in}{2.582749in}}%
\pgfpathlineto{\pgfqpoint{2.555873in}{2.598357in}}%
\pgfpathlineto{\pgfqpoint{2.614290in}{2.611046in}}%
\pgfpathlineto{\pgfqpoint{2.672707in}{2.620683in}}%
\pgfpathlineto{\pgfqpoint{2.731124in}{2.627166in}}%
\pgfpathlineto{\pgfqpoint{2.789541in}{2.630425in}}%
\pgfpathlineto{\pgfqpoint{2.847959in}{2.630425in}}%
\pgfpathlineto{\pgfqpoint{2.906376in}{2.627166in}}%
\pgfpathlineto{\pgfqpoint{2.964793in}{2.620683in}}%
\pgfpathlineto{\pgfqpoint{3.023210in}{2.611046in}}%
\pgfpathlineto{\pgfqpoint{3.081627in}{2.598357in}}%
\pgfpathlineto{\pgfqpoint{3.140044in}{2.582749in}}%
\pgfpathlineto{\pgfqpoint{3.198461in}{2.564379in}}%
\pgfpathlineto{\pgfqpoint{3.256878in}{2.543430in}}%
\pgfpathlineto{\pgfqpoint{3.315295in}{2.520101in}}%
\pgfpathlineto{\pgfqpoint{3.373712in}{2.494605in}}%
\pgfpathlineto{\pgfqpoint{3.451602in}{2.457628in}}%
\pgfpathlineto{\pgfqpoint{3.529491in}{2.417737in}}%
\pgfpathlineto{\pgfqpoint{3.607381in}{2.375473in}}%
\pgfpathlineto{\pgfqpoint{3.704742in}{2.320107in}}%
\pgfpathlineto{\pgfqpoint{3.841049in}{2.239567in}}%
\pgfpathlineto{\pgfqpoint{4.211024in}{2.019009in}}%
\pgfpathlineto{\pgfqpoint{4.327858in}{1.952417in}}%
\pgfpathlineto{\pgfqpoint{4.444692in}{1.888394in}}%
\pgfpathlineto{\pgfqpoint{4.542054in}{1.837265in}}%
\pgfpathlineto{\pgfqpoint{4.639416in}{1.788301in}}%
\pgfpathlineto{\pgfqpoint{4.736778in}{1.741574in}}%
\pgfpathlineto{\pgfqpoint{4.756250in}{1.732500in}}%
\pgfpathlineto{\pgfqpoint{4.756250in}{1.732500in}}%
\pgfusepath{stroke}%
\end{pgfscope}%
\begin{pgfscope}%
\pgfsetrectcap%
\pgfsetmiterjoin%
\pgfsetlinewidth{0.803000pt}%
\definecolor{currentstroke}{rgb}{0.000000,0.000000,0.000000}%
\pgfsetstrokecolor{currentstroke}%
\pgfsetdash{}{0pt}%
\pgfpathmoveto{\pgfqpoint{0.687500in}{0.385000in}}%
\pgfpathlineto{\pgfqpoint{0.687500in}{3.080000in}}%
\pgfusepath{stroke}%
\end{pgfscope}%
\begin{pgfscope}%
\pgfsetrectcap%
\pgfsetmiterjoin%
\pgfsetlinewidth{0.803000pt}%
\definecolor{currentstroke}{rgb}{0.000000,0.000000,0.000000}%
\pgfsetstrokecolor{currentstroke}%
\pgfsetdash{}{0pt}%
\pgfpathmoveto{\pgfqpoint{4.950000in}{0.385000in}}%
\pgfpathlineto{\pgfqpoint{4.950000in}{3.080000in}}%
\pgfusepath{stroke}%
\end{pgfscope}%
\begin{pgfscope}%
\pgfsetrectcap%
\pgfsetmiterjoin%
\pgfsetlinewidth{0.803000pt}%
\definecolor{currentstroke}{rgb}{0.000000,0.000000,0.000000}%
\pgfsetstrokecolor{currentstroke}%
\pgfsetdash{}{0pt}%
\pgfpathmoveto{\pgfqpoint{0.687500in}{0.385000in}}%
\pgfpathlineto{\pgfqpoint{4.950000in}{0.385000in}}%
\pgfusepath{stroke}%
\end{pgfscope}%
\begin{pgfscope}%
\pgfsetrectcap%
\pgfsetmiterjoin%
\pgfsetlinewidth{0.803000pt}%
\definecolor{currentstroke}{rgb}{0.000000,0.000000,0.000000}%
\pgfsetstrokecolor{currentstroke}%
\pgfsetdash{}{0pt}%
\pgfpathmoveto{\pgfqpoint{0.687500in}{3.080000in}}%
\pgfpathlineto{\pgfqpoint{4.950000in}{3.080000in}}%
\pgfusepath{stroke}%
\end{pgfscope}%
\begin{pgfscope}%
\definecolor{textcolor}{rgb}{0.000000,0.000000,0.000000}%
\pgfsetstrokecolor{textcolor}%
\pgfsetfillcolor{textcolor}%
\pgftext[x=2.818750in,y=3.163333in,,base]{\color{textcolor}\rmfamily\fontsize{12.000000}{14.400000}\selectfont N=5, 7, 15, 31}%
\end{pgfscope}%
\begin{pgfscope}%
\pgfsetbuttcap%
\pgfsetmiterjoin%
\definecolor{currentfill}{rgb}{1.000000,1.000000,1.000000}%
\pgfsetfillcolor{currentfill}%
\pgfsetfillopacity{0.800000}%
\pgfsetlinewidth{1.003750pt}%
\definecolor{currentstroke}{rgb}{0.800000,0.800000,0.800000}%
\pgfsetstrokecolor{currentstroke}%
\pgfsetstrokeopacity{0.800000}%
\pgfsetdash{}{0pt}%
\pgfpathmoveto{\pgfqpoint{0.784722in}{0.454444in}}%
\pgfpathlineto{\pgfqpoint{2.050468in}{0.454444in}}%
\pgfpathquadraticcurveto{\pgfqpoint{2.078246in}{0.454444in}}{\pgfqpoint{2.078246in}{0.482222in}}%
\pgfpathlineto{\pgfqpoint{2.078246in}{1.661422in}}%
\pgfpathquadraticcurveto{\pgfqpoint{2.078246in}{1.689199in}}{\pgfqpoint{2.050468in}{1.689199in}}%
\pgfpathlineto{\pgfqpoint{0.784722in}{1.689199in}}%
\pgfpathquadraticcurveto{\pgfqpoint{0.756944in}{1.689199in}}{\pgfqpoint{0.756944in}{1.661422in}}%
\pgfpathlineto{\pgfqpoint{0.756944in}{0.482222in}}%
\pgfpathquadraticcurveto{\pgfqpoint{0.756944in}{0.454444in}}{\pgfqpoint{0.784722in}{0.454444in}}%
\pgfpathclose%
\pgfusepath{stroke,fill}%
\end{pgfscope}%
\begin{pgfscope}%
\pgfsetrectcap%
\pgfsetroundjoin%
\pgfsetlinewidth{1.505625pt}%
\definecolor{currentstroke}{rgb}{0.121569,0.466667,0.705882}%
\pgfsetstrokecolor{currentstroke}%
\pgfsetdash{}{0pt}%
\pgfpathmoveto{\pgfqpoint{0.812500in}{1.498789in}}%
\pgfpathlineto{\pgfqpoint{1.090278in}{1.498789in}}%
\pgfusepath{stroke}%
\end{pgfscope}%
\begin{pgfscope}%
\definecolor{textcolor}{rgb}{0.000000,0.000000,0.000000}%
\pgfsetstrokecolor{textcolor}%
\pgfsetfillcolor{textcolor}%
\pgftext[x=1.201389in,y=1.450178in,left,base]{\color{textcolor}\rmfamily\fontsize{10.000000}{12.000000}\selectfont \(\displaystyle y(x)=\)\(\displaystyle \frac{1}{1+x^{2}}\)}%
\end{pgfscope}%
\begin{pgfscope}%
\pgfsetrectcap%
\pgfsetroundjoin%
\pgfsetlinewidth{1.505625pt}%
\definecolor{currentstroke}{rgb}{0.172549,0.627451,0.172549}%
\pgfsetstrokecolor{currentstroke}%
\pgfsetdash{}{0pt}%
\pgfpathmoveto{\pgfqpoint{0.812500in}{1.218333in}}%
\pgfpathlineto{\pgfqpoint{1.090278in}{1.218333in}}%
\pgfusepath{stroke}%
\end{pgfscope}%
\begin{pgfscope}%
\definecolor{textcolor}{rgb}{0.000000,0.000000,0.000000}%
\pgfsetstrokecolor{textcolor}%
\pgfsetfillcolor{textcolor}%
\pgftext[x=1.201389in,y=1.169722in,left,base]{\color{textcolor}\rmfamily\fontsize{10.000000}{12.000000}\selectfont W5(x)}%
\end{pgfscope}%
\begin{pgfscope}%
\pgfsetrectcap%
\pgfsetroundjoin%
\pgfsetlinewidth{1.505625pt}%
\definecolor{currentstroke}{rgb}{0.580392,0.403922,0.741176}%
\pgfsetstrokecolor{currentstroke}%
\pgfsetdash{}{0pt}%
\pgfpathmoveto{\pgfqpoint{0.812500in}{1.010000in}}%
\pgfpathlineto{\pgfqpoint{1.090278in}{1.010000in}}%
\pgfusepath{stroke}%
\end{pgfscope}%
\begin{pgfscope}%
\definecolor{textcolor}{rgb}{0.000000,0.000000,0.000000}%
\pgfsetstrokecolor{textcolor}%
\pgfsetfillcolor{textcolor}%
\pgftext[x=1.201389in,y=0.961389in,left,base]{\color{textcolor}\rmfamily\fontsize{10.000000}{12.000000}\selectfont W7(x)}%
\end{pgfscope}%
\begin{pgfscope}%
\pgfsetrectcap%
\pgfsetroundjoin%
\pgfsetlinewidth{1.505625pt}%
\definecolor{currentstroke}{rgb}{0.890196,0.466667,0.760784}%
\pgfsetstrokecolor{currentstroke}%
\pgfsetdash{}{0pt}%
\pgfpathmoveto{\pgfqpoint{0.812500in}{0.801667in}}%
\pgfpathlineto{\pgfqpoint{1.090278in}{0.801667in}}%
\pgfusepath{stroke}%
\end{pgfscope}%
\begin{pgfscope}%
\definecolor{textcolor}{rgb}{0.000000,0.000000,0.000000}%
\pgfsetstrokecolor{textcolor}%
\pgfsetfillcolor{textcolor}%
\pgftext[x=1.201389in,y=0.753056in,left,base]{\color{textcolor}\rmfamily\fontsize{10.000000}{12.000000}\selectfont W15(x)}%
\end{pgfscope}%
\begin{pgfscope}%
\pgfsetrectcap%
\pgfsetroundjoin%
\pgfsetlinewidth{1.505625pt}%
\definecolor{currentstroke}{rgb}{0.737255,0.741176,0.133333}%
\pgfsetstrokecolor{currentstroke}%
\pgfsetdash{}{0pt}%
\pgfpathmoveto{\pgfqpoint{0.812500in}{0.593333in}}%
\pgfpathlineto{\pgfqpoint{1.090278in}{0.593333in}}%
\pgfusepath{stroke}%
\end{pgfscope}%
\begin{pgfscope}%
\definecolor{textcolor}{rgb}{0.000000,0.000000,0.000000}%
\pgfsetstrokecolor{textcolor}%
\pgfsetfillcolor{textcolor}%
\pgftext[x=1.201389in,y=0.544722in,left,base]{\color{textcolor}\rmfamily\fontsize{10.000000}{12.000000}\selectfont W31(x)}%
\end{pgfscope}%
\end{pgfpicture}%
\makeatother%
\endgroup%
        
    \end{center}
    \caption{Węzły cosinus, funkcja \(\tilde{y}\), \(N=5,7,15,31\)}
\end{figure}


\end{document}