\documentclass[a4paper,11pt]{article}
\usepackage{graphicx}
\usepackage[]{latexsym}
\usepackage[]{amsmath}
\usepackage[]{amsfonts}
\usepackage[]{mathtools}
\usepackage[]{bm}
\usepackage[]{listings}
\usepackage[]{color}
\usepackage[]{pgf}
\usepackage[]{float}
\usepackage[MeX]{polski}
\usepackage[utf8]{inputenc}

\definecolor{dkgreen}{rgb}{0,0.6,0}
\definecolor{gray}{rgb}{0.5,0.5,0.5}
\definecolor{mauve}{rgb}{0.58,0,0.82}
\lstset{
    language=Python,
    frame=tb,
    aboveskip=3mm,
    belowskip=3mm,
    numbers=none,
    basicstyle=\ttfamily\small, 
    keywordstyle=\color{blue},
    commentstyle=\color{dkgreen},
    stringstyle=\color{mauve},
    showstringspaces=false,
    breaklines=true,
}

\author{Wojciech Lepich \and Nomen Nescio}
\title{NUM14. Zagadnienie własne}

\begin{document}
\maketitle

\section{Teoria}

\subsection{Zagadnienie własne}

Niech \(\mathbf{A} \in \mathbb{C}^{N\times N} \). Wartość własna macierzy A
to liczba \(\lambda \in \mathbb{C}\), taka że dla niezerowego wektora
\(\mathbf{x} \in \mathbb{C}^{N}\):
\begin{equation}
    \mathbf{A}\mathbf{x} = \lambda\mathbf{x}.
\end{equation}
Wektor \(\mathbf{x}\) jest wektorem własnym macierzy \(\mathbf{A}\) do wartości
własnej \(\lambda \). Problem szukania wartości własnch macierzy to
zagadnienie własne.

\subsection{Algorytm QR}

Niech macierz \(\mathbf{A}\) posiada faktoryzację QR, \(\mathbf{A = QR}\).
Wtedy:

\begin{equation}
    \mathbf{A}' = RQ = Q^T AQ    
\end{equation}

Wymnożenie czynników QR w odwrotnej kolejności daje ortogonalną transformację
podobienśtwa macierzy \(\mathbf{A}\). Można tę procedurę iterować. Zaczynamy
od \(\mathbf{A_1 = A}\).
\begin{equation}
    \begin{aligned}
        \mathbf{A}_1 = \mathbf{Q}_1\mathbf{R}_1 \\
        \mathbf{A}_2 = \mathbf{R}_1\mathbf{Q}_1 = \mathbf{Q}_2\mathbf{R}_2 \\
        \mathbf{A}_3 = \mathbf{R}_2\mathbf{Q}_2 = \mathbf{Q}_3\mathbf{R}_3 \\
        \vdots \\
        \mathbf{A}_n = \mathbf{R}_{n-1}\mathbf{Q}_{n-1} = \mathbf{Q}_n\mathbf{R}_n
    \end{aligned}
\end{equation}

W każdym kroku iteracji dokonywana jest faktoryzacja QR, która ma złożoność
obliczeniową \(O(N^3)\). Dlatego na samym początku należy przekształcić macierz
do postaci Hessenberga przez podobieństwa ortogonalne (czyli nie zmieniając
wartości własnych). Dla takiej macierzy złożoność obliczeniową faktoryzacji QR
wynosi \(O(N^2)\).

\subsection{Metoda potęgowa}

Niech \(\mathbf{A} \in \mathbb{R}^{N\times N} \) będzie macierzą symetryczną,
\(\mathbf{A=A}^T\). Taka macierz jest diagonalizowalna, jej wartości własne
są rzeczywiste, jej unormowane wektory własne tworzą bazę ortonormalną
w \(\mathbb{R}^N\). Tą bazą niech będzie \( \{\mathbf{e}\} ^N_{i=1} \),
gdzie \(\mathbf{Ae}_i = \lambda_i\mathbf{e}_i\). Przyjmijmy, że wartości własne
macierzy \(\mathbf{A}\) są dodatnie i uporządkowane malejąco 
\(\lambda_1 > \lambda_2 > \cdots > \lambda_0 > 0\).


\[
\begin{bmatrix}
    3 & 4 & 2 & 4 \\
    4 & 7 & 1 & -3 \\
    2 & 1 & 3 & 2 \\
    4 & -3 & 2 & 2
\end{bmatrix}
\]

\[
\begin{bmatrix}
    3 & 6 & 6 & 9 \\
    1 & 4 & 0 & 9 \\
    0 & 0.2 & 6 & 12 \\
    0 & 0 & 0.1 & 6
\end{bmatrix}
\]

\section{Omówienie}

\subsection{Kod}


\subsection{Wyniki}


\end{document}