\documentclass[a4paper,11pt]{article}
\usepackage{graphicx}
\usepackage[]{latexsym}
\usepackage[]{amsmath}
\usepackage[]{amsfonts}
\usepackage[]{mathtools}
\usepackage[]{bm}
\usepackage[]{listings}
\usepackage[]{color}
\usepackage[]{pgf}
\usepackage[]{float}
\usepackage[]{enumitem}
\usepackage[MeX]{polski}
\usepackage[utf8]{inputenc}

\definecolor{dkgreen}{rgb}{0,0.6,0}
\definecolor{gray}{rgb}{0.5,0.5,0.5}
\definecolor{mauve}{rgb}{0.58,0,0.82}
\lstset{
    language=Python,
    frame=tb,
    aboveskip=3mm,
    belowskip=3mm,
    numbers=left,
    basicstyle=\ttfamily\small, 
    keywordstyle=\color{blue},
    commentstyle=\color{dkgreen},
    stringstyle=\color{mauve},
    showstringspaces=false,
    breaklines=true,
}

\author{Wojciech Lepich \and Nomen Nescio}
\title{NUM14. Zagadnienie własne}

\begin{document}
\maketitle

\section{Teoria}

\subsection{Zagadnienie własne}

Niech \(\mathbf{A} \in \mathbb{C}^{N\times N} \). Wartość własna macierzy
\(\mathbf{A}\) to liczba \(\lambda \in \mathbb{C}\), taka że:
\begin{equation}
    \mathbf{A}\mathbf{x} = \lambda\mathbf{x},
\end{equation}
gdzie \(\mathbf{x}\in \mathbb{C}^N\) to wektor niezerowy. Wektor \(\mathbf{x}\) 
to wektor własny macierzy \(\mathbf{A}\) do wartości własnej \(\lambda \).
Zagadnienie własne to problem poszukiwania wartości własnych macierzy.

\subsection{Algorytm QR}

Jeśli macierz \(\mathbf{A}\) posiada faktoryzację QR, to można wyznaczyć jej
wartości własne algorytmem QR.\@
Jest to iteracyjna metoda wyznaczania wartości własnych, polegająca na
wymnażaniu czynników faktoryzacji QR w odwróconej kolejności:
\begin{equation}
    \mathbf{A}' = \mathbf{R}\mathbf{Q} = \mathbf{Q}^T \mathbf{A}\mathbf{Q}
\end{equation}


W pierwszym kroku zaczynamy od od \(\mathbf{A_1 = A}\).
\begin{equation}
    \begin{aligned}
        \mathbf{A}_1 = \mathbf{Q}_1\mathbf{R}_1 \\
        \mathbf{A}_2 = \mathbf{R}_1\mathbf{Q}_1 = \mathbf{Q}_2\mathbf{R}_2 \\
        \mathbf{A}_3 = \mathbf{R}_2\mathbf{Q}_2 = \mathbf{Q}_3\mathbf{R}_3 \\
        \vdots \\
        \mathbf{A}_n = \mathbf{R}_{n-1}\mathbf{Q}_{n-1} = \mathbf{Q}_n\mathbf{R}_n
    \end{aligned}
\end{equation}

Powyższa iteracja zbieżna jest do macierzy trójkątnej górnej, w której na
głównej diagonali znajdują się kolejne wartości własne.

Faktoryzacja QR ma złożoność obliczeniową \(\mathcal{O}(N^3)\) na każdą
iterację, w związku z czym przed iterowaniem należy sprowadzić macierz
(niesymetryczną) do postaci Hessenberga --- macierzy trójkątnej  górnej
z dodatkową diagonalą pod diagonalą główną. Dla takiej macierzy złożoność
faktoryzacji QR wynosi \(\mathcal{O}(N^2)\).

\subsection{Metoda potęgowa}

Wykorzystując metodę potęgową łatwmo można znaleźć największą i najmniejszą
wartość własną macierzy symetrycznej. Taka macierz jest diagonalizowalna,
ma rzeczywiste wartości własne, a jej unormowane wektory tworzą bazę ortogonalną
w \(\mathbb{R}^N\).

Zakładamy, że wartości własne macierzy są dodatnie i uporządkowane malejąco:
\(\lambda_1 > \lambda_2 > \cdots > \lambda_n > 0\).
Biorąc wektor \(\mathbf{y}\in\mathbb{R}^N\) (\( \|\mathbf{y}\|=1\)) mamy
jego rozkład:
\begin{equation}
    \mathbf{y} = \sum_{i=1}^{N}\beta_i \mathbf{e}_i
\end{equation}
Obliczając kolejno:
\begin{equation}
    \begin{aligned}
        & \mathbf{A}\mathbf{y} = 
            \mathbf{A}\sum_{i=1}^{N}\beta_i \mathbf{e}_i = 
            \sum_{i=1}^{N}\beta_i \mathbf{A}\mathbf{e}_i = 
            \sum_{i=1}^{N}\beta_i\lambda_i \mathbf{e}_i \\
        & \mathbf{A}^2\mathbf{y} = 
            \mathbf{A}\sum_{i=1}^{N}\lambda_i\beta_i \mathbf{e}_i = 
            \sum_{i=1}^{N}\beta_i\lambda_i^2 \mathbf{e}_i \\
        & \cdots \\
        & \mathbf{A}^k\mathbf{y} =
            \sum_{i=1}^{N}\beta_i\lambda_i^k\mathbf{e}_i
    \end{aligned}
\end{equation}

Dla dużych \(k\) wyraz zawierający \(\lambda_1^k\) będzie dominujący w sumie,
a pozostałe współczynniki numerycznie pomijalnie małe --- prawa strona dla
\(k\to\infty \) dąży do wektora proporcjonalnego do \(\mathbf{e}_1\), czyli do
wektora własnego do największej wartości własnej.
Iteracja:
\begin{equation}
    \begin{gathered}
        \mathbf{A}\mathbf{y}_k = \mathbf{z}_k \\
        \mathbf{y}_{k+1} = \frac{\mathbf{z}_k}{\|\mathbf{z}_k\|}
    \end{gathered}
\end{equation}
zbiega do unormowanego wektora własnego macierzy \(\mathbf{A}\) do największej
wartości własnej. Gdy iteracja zbiegnie do punktu stałego, wartość własna będzie
dana \(\lambda_1=\|\mathbf{z}_k\| \).

\pagebreak

\section{Omówienie}

Zadane są macierze \(\mathbf{M}\) i \(\mathbf{B}\):

\[
\mathbf{B} = 
\begin{bmatrix}
    3 & 4 & 2 & 4 \\
    4 & 7 & 1 & -3 \\
    2 & 1 & 3 & 2 \\
    4 & -3 & 2 & 2
\end{bmatrix},
\mathbf{M} = 
\begin{bmatrix}
    3 & 6 & 6 & 9 \\
    1 & 4 & 0 & 9 \\
    0 & 0.2 & 6 & 12 \\
    0 & 0 & 0.1 & 6
\end{bmatrix}
\]

W zadaniu należy:
\begin{enumerate}[label=\alph*)]
    \item algorytmem QR znaleźć wszystkie wartości własne macierzy M
    \item metodą potęgową znaleźć największą co do modułu wartość własną 
    macierzy B
\end{enumerate}

\subsection{Kod}

W programie skorzystałem z bibliotek NumPy i SciPy.

\subsubsection{Algorytm QR}

\begin{lstlisting}[caption={Algorytm QR}]
def algorytmQR(A):
    A_old = A.copy()
    Q, R = linalg.qr(A)
    A = np.dot(R, Q)
    i = 0
    while not wektoryZbiezne(np.diag(A), np.diag(A_old)):
        i += 1
        A_old = A
        Q, R = linalg.qr(A)
        A = np.dot(R, Q)
    return np.diag(A), i
\end{lstlisting}

Funkcja opiera się o iterację pętlą \texttt{do..while}. Tworzona jest kopia
macierzy A, następnie dokonywana faktoryzacja QR z wykorzystaniem pakietu
algebry liniowej. Czynniki faktoryzacji mnożone są w odwróconej kolejności.
Następnie powtarzane jest to tak długo, aż diagonala nowo wyznaczjonej macierzy
nie będzie wystarczająco zbliżona do diagonali macierzy z poprzedniego kroku.
Funkcja zwraca diagonalę \texttt{np.diag(A)} (listę wartości własnych) oraz ilość
iteracji \texttt{i}

\pagebreak

\subsubsection{Metoda potęgowa}

\begin{lstlisting}[caption={Metoda potęgowa}]
def metodaPotegowa(A):
    y = np.ones(A.shape[1])
    y_old = y.copy()

    z = np.dot(A, y)
    y = z / np.linalg.norm(z)
    i = 0
    while not wektoryZbiezne(y, y_old):
        i += 1
        y_old = y
        z = np.dot(A, y)
        y = z / np.linalg.norm(z)
    return y, np.linalg.norm(z), i
\end{lstlisting}

W tej funkcji tworzony jest wektor \(\mathbf{y}\) złożony z samych jedynek
oraz jego kopia. Obliczany jest wektor \(\mathbf{z} = \mathbf{A}\mathbf{y}\),
następnie \(\mathbf{y}=\frac{\mathbf{z}}{\|\mathbf{z}\|}\). Procedura powtarzana
jest tak długo, aż wektor \(\mathbf{y}\) nie będzie zbliżony do wektora
z poprzedniej iteracji. Zwracane są wektor własny do największej wartości 
własnej \texttt{y}, największa wartość własna \texttt{np.linalg.norm(z)} oraz
ilość iteracji~\texttt{i}.

\subsection{Wyniki}

Otrzymane wyniki zgadzają się z tymi otrzymanymi przy użyciu pakietu algebry
liniowej.

Wartości własne otrzymane algorytmem QR po 108 iteracjach 
(równe tym obliczonymi przez SciPy):
\[
\begin{bmatrix}
    7.23099231 \\ 5.90015727 \\ 4.81580659 \\ 1.05304383
\end{bmatrix}
\]

Największa wartość własna macierzy B, otrzymana po 76 iteracjach, niżej wynik
otrzymany z SciPy:
\[10.015982848255216\]
\[10.015982848255222\]

Oraz wektor własny do powyższej wartości własnej:

\[
\begin{bmatrix}
    0.5582969 \\ 0.77620837 \\ 0.28678781 \\ 0.0596481
\end{bmatrix}
\]

\end{document}